\documentclass[11pt]{article}

\usepackage[T1]{fontenc}
\usepackage{geometry}
\usepackage{amsmath, amssymb, amsthm}
\usepackage[scr]{rsfso}
\usepackage{bm}
\usepackage{listings}
\usepackage[%
    hidealllines=true,%
    innerbottommargin=15,%
    nobreak=true,%
]{mdframed}
\usepackage{xcolor}
\usepackage{graphicx}
\usepackage{fancyhdr}
\usepackage{hyperref}

\geometry{a4paper, margin=1in, headheight=14pt}

\pagestyle{fancy}
\fancyhf{}
\renewcommand\headrulewidth{0.4pt}
\fancyhead[L]{\scshape MA3105: Numerical Analysis}
\fancyhead[R]{\scshape \leftmark}
\rfoot{\footnotesize\it Updated on \today}
\cfoot{\thepage}

\newcommand{\C}{\mathbb{C}}
\newcommand{\R}{\mathbb{R}}
\newcommand{\Q}{\mathbb{Q}}
\newcommand{\Z}{\mathbb{Z}}
\newcommand{\N}{\mathbb{N}}

\newcommand{\ip}[2]{\langle #1, #2 \rangle}
\newcommand{\norm}[1]{\Vert #1 \Vert}
\renewcommand{\vec}[1]{\boldsymbol{#1}}

\newcommand{\vx}{\vec{x}}
\newcommand{\vy}{\vec{y}}
\newcommand{\vv}{\vec{v}}
\newcommand{\vw}{\vec{w}}
\newcommand{\ve}{\vec{e}}

\definecolor{codegreen}{rgb}{0,0.6,0}
\definecolor{codegray}{rgb}{0.5,0.5,0.5}
\definecolor{codepurple}{rgb}{0.58,0,0.82}
\definecolor{backcolour}{rgb}{1,1,1}

\newmdtheoremenv[%
    backgroundcolor=blue!10!white,%
]{theorem}{Theorem}[section]
\newmdtheoremenv[%
    backgroundcolor=violet!10!white,%
]{corollary}{Corollary}[theorem]
\newmdtheoremenv[%
    backgroundcolor=teal!10!white,%
]{lemma}[theorem]{Lemma}

\theoremstyle{definition}
\newmdtheoremenv[%
    backgroundcolor=green!10!white,%
]{definition}{Definition}[section]
\newmdtheoremenv[%
    backgroundcolor=red!10!white,%
]{exercise}{Exercise}[section]

\theoremstyle{remark}
\newtheorem*{remark}{Remark}
\newtheorem*{example}{Example}
\newtheorem*{solution}{Solution}

\surroundwithmdframed[%
    linecolor=black!20!white,%
    hidealllines=false,%
    innertopmargin=5,%
    innerbottommargin=10,%
    skipabove=0,%
    skipbelow=0,%
]{example}

\numberwithin{equation}{section}

\lstdefinestyle{mystyle}{
    backgroundcolor=\color{backcolour},
    commentstyle=\color{codegreen},
    keywordstyle=\color{blue},
    numbers=none,
    stringstyle=\color{codepurple},
    basicstyle=\ttfamily\footnotesize,
    breakatwhitespace=false,
    breaklines=true,
    keepspaces=true,
    numbers=left,
    numbersep=5pt,
    showspaces=false,
    showstringspaces=false,
    showtabs=false,
    tabsize=4
}

\lstset{style=mystyle}

\title{
    \Large\textsc{MA3105} \\
    \Huge \textbf{Numerical Analysis} \\
    \vspace{5pt}
    \Large{Autumn 2021}
}
\author{
    \large Satvik Saha
    \\\textsc{\small 19MS154}
}
\date{\normalsize
    \textit{Indian Institute of Science Education and Research, Kolkata, \\
    Mohanpur, West Bengal, 741246, India.} \\
}

\begin{document}
    \maketitle

    \tableofcontents

    \section{Time complexity}

    \subsection{Runtime cost}
    When designing or implementing an algorithm, we care about its efficiency -- both
    in terms of execution time, and the use of resources. This gives us a rough way
    of comparing two algorithms. However, such metrics are architecture and language
    dependent; different machines, or the same program implemented in different
    programming languages, may consume different amounts of time or resources while
    executing the same algorithm. Thus, we seek a way of measuring the `cost' in time
    for a given algorithm.

    For example, we may look at each statement in a program, and associate a cost
    $c_i$ with each of them. Consider the following statements.

    \begin{minipage}{\linewidth}  
    \begin{lstlisting}[language=C, numbers=none]

                one = 1;                            // c_1
                two = 2;                            // c_2
                three = 3;                          // c_3
    \end{lstlisting}
    \end{minipage}

    The total cost of running these statements can be calculated as $T = c_1 + c_2 +
    c_3$, simply by adding up the cost of each statement. Similarly, consider the
    following loop construct.

    \begin{minipage}{\linewidth}  
    \begin{lstlisting}[language=C, numbers=none]

                sum = 0;                            // c_1
                for (i = 0; i < n; i++)             // c_2
                    sum += a[i];                    // c_3
    \end{lstlisting}
    \end{minipage}

    The total cost can be shown to be $T(n) = c_1 + c_2(n + 1) + c_3n$; this time, we
    must take into account the number of times a given statement is executed. Note
    that this is linear. Another example is as follows.

    \begin{minipage}{\linewidth}  
    \begin{lstlisting}[language=C, numbers=none]

                sum = 0;                            // c_1
                for (i = 0; i < n; i++)             // c_2
                    for (j = 0; j < n; j++)         // c_2
                        sum += a[i][j];             // c_4
    \end{lstlisting}
    \end{minipage}

    The total cost can be shown to be $T(n) = c_1 + c_2(n + 1) + c_3n(n + 1) +
    c_4n^2$. Note that this is quadratic. Finally, consider the following recursive call.

    \begin{minipage}{\linewidth}  
    \begin{lstlisting}[language=C, numbers=none]

                int factorial (int n) {             // c_1
                    if (n == 0)                     // c_2
                        return 1;                   // c_3
                    return n * factorial(n - 1);    // c_4
                }

                f = factorial(n);                   // c_5
    \end{lstlisting}
    \end{minipage}
    The cost can be shown to be $T(n) = c_5 + (c_1 + c_2)(n + 1) + c_3 + c_4 n$. This
    turns out to be linear.

    In all these cases, we care about our total cost as a function of the input size
    $n$. Moreover, we are interested mostly in the \emph{growth} of our total cost;
    as our input size grows, the total cost can often be compared with some simple
    function of $n$. Thus, we can classify our cost functions in terms of their
    asymptotic growths.

    \subsection{Asymptotic growth}

    \begin{definition}
        The set $O(g(n))$ denotes the class of functions $f$ which are asymptotically
        bounded above by $g$. In other words, $f(n) \in O(g(n))$ if there exists $M >
        0$ and $n_0 \in \N$ such that for all $n \geq n_0$, \[
            |f(n)| \leq M g(n).
        \] This amounts to writing \[
            \limsup_{n \to \infty} \frac{|f(n)|}{g(n)} < \infty.
        \] 
    \end{definition}
    \begin{example}
        Consider a function defined by $f(n) = an + b$, where $a > 0$. Then, we can
        write $f(n) \in O(n)$. To see why, note that for all $n \geq 1$, we have \[
            |f(n)| = |an + b| \leq an + |b| \leq (a + |b|)n.
        \] Thus, setting $M = a + |b| > 0$ completes the proof.
    \end{example}
    \begin{example}
        Consider a polynomial function defined by \[
            f(n) = a_kn^k + a_{k - 1}n^{k - 1} + \dots + a_1n + a_0,
        \] with some non-zero coefficient. Then, we can write $f(n) \in O(n^k)$. Like
        before, note that for all $n \geq 1$, we have \[
            |f(n)| \leq \sum_{i = 0}^k |a_{i}|n^i \leq \sum_{i = 0}^k |a_i| n^k =
            (|a_k| + |a_{k - 1}| + \dots + |a_0|)n^k.
        \]  Thus, setting $M = |a_k| + \dots + |a_0| > 0$ completes the proof.
    \end{example}

    \begin{theorem}
        If $f_1(n) \in O(g_1(n))$ and $f_2(n) \in O(g_2(n))$, then \[
            f_1(n) + f_2(n) \in O(\max\{g_1(n), g_2(n)\}).
        \]
    \end{theorem}

    \begin{definition}
        The set $\Omega(g(n))$ denotes the class of functions $f$ are asymptotically
        bounded below by $g$. In other words, $f(n) \in \Omega(g(n))$ if there exists
        $M > 0$ and $n_0 \in \N$ such that for all $n \geq n_0$, \[
            |f(n)| \geq M g(n).
        \] This amounts to writing \[
            \liminf_{n \to \infty} \frac{f(n)}{g(n)} > 0.
        \] 
    \end{definition}

    \begin{definition}
        The set $\Theta(g(n))$ denotes the class of functions $f$ which are
        asymptotically bounded both above and below by $g$.  In other words, $f(n)
        \in \Theta(g(n))$ if there exist $M_1, M_2 > 0$ and $n_0 \in \N$ such that
        for all $n \geq n_0$, \[
            M_1 g(n) \leq |f(n)| \leq M_2 g(n).
        \] This amounts to writing $f(n) \in O(g(n))$ and $f(n) \in \Omega(g(n))$.
    \end{definition}
    
    Another class of notation uses the idea of dominated growth.

    \begin{definition}
        The set $o(g(n))$ denotes the class of functions $f$ which are asymptotically
        dominated by $g$. In other words, $f(n) \in o(g(n))$ if for all $M >
        0$, there exists $n_0 \in \N$ such that for all $n \geq n_0$, \[
            |f(n)| < M g(n).
        \] This amounts to writing \[
            \lim_{n \to \infty} \frac{|f(n)|}{g(n)} = 0.
        \] 
    \end{definition}
    
    \begin{definition}
        The set $\omega(g(n))$ denotes the class of functions $f$ which
        asymptotically dominate $g$. In other words, $f(n) \in \omega(g(n))$ if for
        all $M > 0$, there exists $n_0 \in \N$ such that for all $n \geq n_0$, \[
            |f(n)| > M g(n).
        \] This amounts to writing \[
            \lim_{n \to \infty} \frac{|f(n)|}{g(n)} = \infty.
        \] 
    \end{definition}
    
    \begin{definition}
        We say that $f(n) \sim g(n)$ if $f$ is asymptotically equal to $g$.
        In other words, $f(n) \sim g(n)$ if for all $\epsilon > 0$, there exists $n_0
        \in \N$ such that for all $n \geq n_0$, \[
            \left| \frac{f(n)}{g(n)} - 1 \right| < \epsilon.
        \] This amounts to writing \[
            \lim_{n \to \infty} \frac{f(n)}{g(n)} = 1.
        \] 
    \end{definition}

    We often abuse notation and treat the following as equivalent. \[
        T(n) \in O(g(n)), \qquad T(n) = O(g(n)).
    \] 


    \section{Root finding methods}

    Consider an equation of the form $f(x) = 0$, where $f\colon [a, b] \to \R$ is
    given. We wish to solve this equation, i.e.\ find the roots of $f$.

    Note that for \emph{arbitrary} functions, this task is impossible. To see this,
    consider a function $f$ which assumes the value $1$ on $[0, 1] \setminus
    \{\alpha\}$ and $f(\alpha) = 0$, for some $\alpha \in [0, 1]$. There is no way of
    pinpointing $\alpha$ without checking $f$ at every point in $[0, 1]$. Besides, a
    computer cannot reasonably store real numbers with arbitrary precision.

    Thus, we direct our attention towards \emph{continuous} functions $f$. We only
    seek sufficiently accurate approximations of its root $\alpha \in (a, b)$.

    \begin{theorem}[Intermediate Value Theorem]
        Let $f\colon [a, b] \to \R$ be continuous. If $f(a) f(b) < 0$, then there
        exists $\alpha \in (a, b)$ such that $f(\alpha) = 0$.
    \end{theorem}

    \subsection{Tabulation method}
    To identify the location of a root of $f$ on an interval $I = [a, b]$, we
    subdivide $I$ into $n$ subintervals $[x_i, x_{i + 1}]$ where $x_i = a + (b - a) i
    / n$. Now, we simply apply the Intermediate Value Theorem to $f$ on each of these
    intervals. If $f(x_i) f(x_{i + 1}) < 0$, then $f$ has a root somewhere in $(x_i,
    x_{i + 1})$. Note that the error in our approximation is on the order of $|b - a|
    / n$. The precision of this method can be improved by increasing $n$.
    
    To reach a degree of approximation $\epsilon$, we must iterate $n$ times, where \[
        n > \frac{b - a}{\epsilon}.
    \] 


    \subsection{Bisection method}
    Here, we first verify that $f(a) f(b) < 0$, thus ensuring that $f$ has a root
    within $(a, b)$. Now, set $x_1 = a + (b - a) / 2$ and apply the Intermediate
    Value Theorem on the subintervals $[a, x_1]$ and $[x_1, b]$. One of these
    \emph{must} contain a root of $f$. Note that if $f(x_1) = 0$, we are done;
    otherwise, let $I_1 = [a_1, b_1]$ be the subinterval containing the root.  Repeat
    the above process, obtaining successive subintervals $I_n$ with lengths $|b - a|
    / 2^n$. The error in our approximation is of this order, and can be controlled by
    stopping at appropriately large $n$.

    The quantity $x_{n + 1} = (a_n + b_n) / 2$ is a good approximation for the actual
    root $\alpha$ since we know that $x_{n + 1}, \alpha \in [a_n, b_n]$, so \[
        |x_{n + 1} - \alpha| \leq |b_n - a_n| = 2^{-n} |b - a| \to 0.
    \] 

    To reach a degree of approximation $\epsilon$, we must iterate $n$ times, where \[
        n > \log_2{\frac{b - a}{\epsilon}}.
    \] 

    \subsection{Newton's method}

    \subsection{Secant method}

    \subsection{Fixed point method}
    
    
    
\end{document}
