\documentclass[11pt]{article}

\usepackage[T1]{fontenc}
\usepackage{geometry}
\usepackage{amsmath, amssymb, amsthm}
\usepackage[scr]{rsfso}
\usepackage{bm}
\usepackage{listings}
\usepackage[%
    hidealllines=true,%
    innerbottommargin=15,%
    nobreak=true,%
]{mdframed}
\usepackage{xcolor}
\usepackage{graphicx}
\usepackage{fancyhdr}
\usepackage{hyperref}

\geometry{a4paper, margin=1in, headheight=14pt}

\pagestyle{fancy}
\fancyhf{}
\renewcommand\headrulewidth{0.4pt}
\fancyhead[L]{\scshape MA3105: Numerical Analysis}
\fancyhead[R]{\scshape \leftmark}
\rfoot{\footnotesize\it Updated on \today}
\cfoot{\thepage}

\newcommand{\C}{\mathbb{C}}
\newcommand{\R}{\mathbb{R}}
\newcommand{\Q}{\mathbb{Q}}
\newcommand{\Z}{\mathbb{Z}}
\newcommand{\N}{\mathbb{N}}

\newcommand{\ip}[2]{\langle #1, #2 \rangle}
\newcommand{\norm}[1]{\Vert #1 \Vert}
\renewcommand{\vec}[1]{\boldsymbol{#1}}

\newcommand{\vx}{\vec{x}}
\newcommand{\vy}{\vec{y}}
\newcommand{\vv}{\vec{v}}
\newcommand{\vw}{\vec{w}}
\newcommand{\ve}{\vec{e}}

\definecolor{codegreen}{rgb}{0,0.6,0}
\definecolor{codegray}{rgb}{0.5,0.5,0.5}
\definecolor{codepurple}{rgb}{0.58,0,0.82}
\definecolor{backcolour}{rgb}{1,1,1}

\newmdtheoremenv[%
    backgroundcolor=blue!10!white,%
]{theorem}{Theorem}[section]
\newmdtheoremenv[%
    backgroundcolor=violet!10!white,%
]{corollary}{Corollary}[theorem]
\newmdtheoremenv[%
    backgroundcolor=teal!10!white,%
]{lemma}[theorem]{Lemma}

\theoremstyle{definition}
\newmdtheoremenv[%
    backgroundcolor=green!10!white,%
]{definition}{Definition}[section]
\newmdtheoremenv[%
    backgroundcolor=red!10!white,%
]{exercise}{Exercise}[section]

\theoremstyle{remark}
\newtheorem*{remark}{Remark}
\newtheorem*{example}{Example}
\newtheorem*{solution}{Solution}

\surroundwithmdframed[%
    linecolor=black!20!white,%
    hidealllines=false,%
    innertopmargin=5,%
    innerbottommargin=10,%
    skipabove=0,%
    skipbelow=0,%
]{example}

\numberwithin{equation}{section}

\lstdefinestyle{mystyle}{
    backgroundcolor=\color{backcolour},
    commentstyle=\color{codegreen},
    keywordstyle=\color{blue},
    numbers=none,
    stringstyle=\color{codepurple},
    basicstyle=\ttfamily\footnotesize,
    breakatwhitespace=false,
    breaklines=true,
    keepspaces=true,
    numbers=left,
    numbersep=5pt,
    showspaces=false,
    showstringspaces=false,
    showtabs=false,
    tabsize=4
}

\lstset{style=mystyle}

\title{
    \Large\textsc{MA3105} \\
    \Huge \textbf{Numerical Analysis} \\
    \vspace{5pt}
    \Large{Autumn 2021}
}
\author{
    \large Satvik Saha
    \\\textsc{\small 19MS154}
}
\date{\normalsize
    \textit{Indian Institute of Science Education and Research, Kolkata, \\
    Mohanpur, West Bengal, 741246, India.} \\
}

\begin{document}
    \maketitle

    \tableofcontents

    \section{Time complexity}

    \subsection{Runtime cost}
    When designing or implementing an algorithm, we care about its efficiency -- both
    in terms of execution time, and the use of resources. This gives us a rough way
    of comparing two algorithms. However, such metrics are architecture and language
    dependent; different machines, or the same program implemented in different
    programming languages, may consume different amounts of time or resources while
    executing the same algorithm. Thus, we seek a way of measuring the `cost' in time
    for a given algorithm.

    For example, we may look at each statement in a program, and associate a cost
    $c_i$ with each of them. Consider the following statements.

    \begin{minipage}{\linewidth}  
    \begin{lstlisting}[language=C, numbers=none]

                one = 1;                            // c_1
                two = 2;                            // c_2
                three = 3;                          // c_3
    \end{lstlisting}
    \end{minipage}

    The total cost of running these statements can be calculated as $T = c_1 + c_2 +
    c_3$, simply by adding up the cost of each statement. Similarly, consider the
    following loop construct.

    \begin{minipage}{\linewidth}  
    \begin{lstlisting}[language=C, numbers=none]

                sum = 0;                            // c_1
                for (i = 0; i < n; i++)             // c_2
                    sum += a[i];                    // c_3
    \end{lstlisting}
    \end{minipage}

    The total cost can be shown to be $T(n) = c_1 + c_2(n + 1) + c_3n$; this time, we
    must take into account the number of times a given statement is executed. Note
    that this is linear. Another example is as follows.

    \begin{minipage}{\linewidth}  
    \begin{lstlisting}[language=C, numbers=none]

                sum = 0;                            // c_1
                for (i = 0; i < n; i++)             // c_2
                    for (j = 0; j < n; j++)         // c_2
                        sum += a[i][j];             // c_4
    \end{lstlisting}
    \end{minipage}

    The total cost can be shown to be $T(n) = c_1 + c_2(n + 1) + c_3n(n + 1) +
    c_4n^2$. Note that this is quadratic. Finally, consider the following recursive call.

    \begin{minipage}{\linewidth}  
    \begin{lstlisting}[language=C, numbers=none]

                int factorial (int n) {             // c_1
                    if (n == 0)                     // c_2
                        return 1;                   // c_3
                    return n * factorial(n - 1);    // c_4
                }

                f = factorial(n);                   // c_5
    \end{lstlisting}
    \end{minipage}
    The cost can be shown to be $T(n) = c_5 + (c_1 + c_2)(n + 1) + c_3 + c_4 n$. This
    turns out to be linear.

    In all these cases, we care about our total cost as a function of the input size
    $n$. Moreover, we are interested mostly in the \emph{growth} of our total cost;
    as our input size grows, the total cost can often be compared with some simple
    function of $n$. Thus, we can classify our cost functions in terms of their
    asymptotic growths.

    \subsection{Asymptotic growth}

    \begin{definition}
        The set $O(g(n))$ denotes the class of functions $f$ which are asymptotically
        bounded above by $g$. In other words, $f(n) \in O(g(n))$ if there exists $M >
        0$ and $n_0 \in \N$ such that for all $n \geq n_0$, \[
            |f(n)| \leq M g(n).
        \] This amounts to writing \[
            \limsup_{n \to \infty} \frac{|f(n)|}{g(n)} < \infty.
        \] 
    \end{definition}
    \begin{example}
        Consider a function defined by $f(n) = an + b$, where $a > 0$. Then, we can
        write $f(n) \in O(n)$. To see why, note that for all $n \geq 1$, we have \[
            |f(n)| = |an + b| \leq an + |b| \leq (a + |b|)n.
        \] Thus, setting $M = a + |b| > 0$ completes the proof.
    \end{example}
    \begin{example}
        Consider a polynomial function defined by \[
            f(n) = a_kn^k + a_{k - 1}n^{k - 1} + \dots + a_1n + a_0,
        \] with some non-zero coefficient. Then, we can write $f(n) \in O(n^k)$. Like
        before, note that for all $n \geq 1$, we have \[
            |f(n)| \leq \sum_{i = 0}^k |a_{i}|n^i \leq \sum_{i = 0}^k |a_i| n^k =
            (|a_k| + |a_{k - 1}| + \dots + |a_0|)n^k.
        \]  Thus, setting $M = |a_k| + \dots + |a_0| > 0$ completes the proof.
    \end{example}

    \begin{theorem}
        If $f_1(n) \in O(g_1(n))$ and $f_2(n) \in O(g_2(n))$, then \[
            f_1(n) + f_2(n) \in O(\max\{g_1(n), g_2(n)\}).
        \]
    \end{theorem}

    \begin{definition}
        The set $\Omega(g(n))$ denotes the class of functions $f$ are asymptotically
        bounded below by $g$. In other words, $f(n) \in \Omega(g(n))$ if there exists
        $M > 0$ and $n_0 \in \N$ such that for all $n \geq n_0$, \[
            |f(n)| \geq M g(n).
        \] This amounts to writing \[
            \liminf_{n \to \infty} \frac{f(n)}{g(n)} > 0.
        \] 
    \end{definition}

    \begin{definition}
        The set $\Theta(g(n))$ denotes the class of functions $f$ which are
        asymptotically bounded both above and below by $g$.  In other words, $f(n)
        \in \Theta(g(n))$ if there exist $M_1, M_2 > 0$ and $n_0 \in \N$ such that
        for all $n \geq n_0$, \[
            M_1 g(n) \leq |f(n)| \leq M_2 g(n).
        \] This amounts to writing $f(n) \in O(g(n))$ and $f(n) \in \Omega(g(n))$.
    \end{definition}
    
    Another class of notation uses the idea of dominated growth.

    \begin{definition}
        The set $o(g(n))$ denotes the class of functions $f$ which are asymptotically
        dominated by $g$. In other words, $f(n) \in o(g(n))$ if for all $M >
        0$, there exists $n_0 \in \N$ such that for all $n \geq n_0$, \[
            |f(n)| < M g(n).
        \] This amounts to writing \[
            \lim_{n \to \infty} \frac{|f(n)|}{g(n)} = 0.
        \] 
    \end{definition}
    
    \begin{definition}
        The set $\omega(g(n))$ denotes the class of functions $f$ which
        asymptotically dominate $g$. In other words, $f(n) \in \omega(g(n))$ if for
        all $M > 0$, there exists $n_0 \in \N$ such that for all $n \geq n_0$, \[
            |f(n)| > M g(n).
        \] This amounts to writing \[
            \lim_{n \to \infty} \frac{|f(n)|}{g(n)} = \infty.
        \] 
    \end{definition}
    
    \begin{definition}
        We say that $f(n) \sim g(n)$ if $f$ is asymptotically equal to $g$.
        In other words, $f(n) \sim g(n)$ if for all $\epsilon > 0$, there exists $n_0
        \in \N$ such that for all $n \geq n_0$, \[
            \left| \frac{f(n)}{g(n)} - 1 \right| < \epsilon.
        \] This amounts to writing \[
            \lim_{n \to \infty} \frac{f(n)}{g(n)} = 1.
        \] 
    \end{definition}

    We often abuse notation and treat the following as equivalent. \[
        T(n) \in O(g(n)), \qquad T(n) = O(g(n)).
    \] 


    \section{Root finding methods}

    Consider an equation of the form $f(x) = 0$, where $f\colon [a, b] \to \R$ is
    given. We wish to solve this equation, i.e.\ find the roots of $f$.

    Note that for \emph{arbitrary} functions, this task is impossible. To see this,
    consider a function $f$ which assumes the value $1$ on $[0, 1] \setminus
    \{\alpha\}$ and $f(\alpha) = 0$, for some $\alpha \in [0, 1]$. There is no way of
    pinpointing $\alpha$ without checking $f$ at every point in $[0, 1]$. Besides, a
    computer cannot reasonably store real numbers with arbitrary precision.

    Thus, we direct our attention towards \emph{continuous} functions $f$. We only
    seek sufficiently accurate approximations of its root $\alpha \in (a, b)$.

    \begin{theorem}[Intermediate Value Theorem]
        Let $f\colon [a, b] \to \R$ be continuous. If $f(a) f(b) < 0$, then there
        exists $\alpha \in (a, b)$ such that $f(\alpha) = 0$.
    \end{theorem}

    \subsection{Tabulation method}
    To identify the location of a root of $f$ on an interval $I = [a, b]$, we
    subdivide $I$ into $n$ subintervals $[x_i, x_{i + 1}]$ where $x_i = a + (b - a) i
    / n$. Now, we simply apply the Intermediate Value Theorem to $f$ on each of these
    intervals. If $f(x_i) f(x_{i + 1}) < 0$, then $f$ has a root somewhere in $(x_i,
    x_{i + 1})$. Note that the error in our approximation is on the order of $|b - a|
    / n$. The precision of this method can be improved by increasing $n$.
    
    To reach a degree of approximation $\epsilon$, we must iterate $n$ times, where \[
        n > \frac{b - a}{\epsilon}.
    \] 


    \subsection{Bisection method}
    Here, we first verify that $f(a) f(b) < 0$, thus ensuring that $f$ has a root
    within $(a, b)$. Now, set $x_1 = a + (b - a) / 2$ and apply the Intermediate
    Value Theorem on the subintervals $[a, x_1]$ and $[x_1, b]$. One of these
    \emph{must} contain a root of $f$. Note that if $f(x_1) = 0$, we are done;
    otherwise, let $I_1 = [a_1, b_1]$ be the subinterval containing the root.  Repeat
    the above process, obtaining successive subintervals $I_n$ with lengths $|b - a|
    / 2^n$. The error in our approximation is of this order, and can be controlled by
    stopping at appropriately large $n$.

    The quantity $x_{n + 1} = (a_n + b_n) / 2$ is a good approximation for the actual
    root $\alpha$ since we know that $x_{n + 1}, \alpha \in [a_n, b_n]$, so \[
        |x_{n + 1} - \alpha| \leq |b_n - a_n| = 2^{-n} |b - a| \to 0.
    \] 

    To reach a degree of approximation $\epsilon$, we must iterate $n$ times, where \[
        n > \log_2{\frac{b - a}{\epsilon}}.
    \] 

    \subsection{Newton-Raphson method}
    Assuming that $f$ is twice differentiable, use Taylor's theorem to write \[
        f(x) = f(x_0) + f'(x)(x - x_0) + \frac{1}{2}f''(c)(x - x_0)^2
    \] for all $x \in [a, b]$, where $c$ is between $x$ and $x_0$. The first two
    terms represent the tangent line to $f$, drawn at $(x_0, f(x_0))$. Now, define \[
        x_1 = x_0 - \frac{f(x_0)}{f'(x_0)}.
    \] Note that this is the point at which the tangent line to $f$ at $x_0$ cuts the
    $x$-axis. We have implicitly assumed that $f'(x_0) \neq 0$. In this manner,
    create the sequence of points \[
        x_{n + 1} = x_n - \frac{f(x_n)}{f'(x_n)}.
    \] We wish to show that $x_n \to \alpha$, under certain circumstances.

    \begin{definition}[Order of convergence]
        Let $x_n \to \alpha$. We say that this convergence is of order $p \geq 1$ if
        \[
            \lim_{n \to \infty} \frac{|\alpha - x_{n + 1}|}{|\alpha - x_n|^p} > 0.
        \] 
    \end{definition}

    \begin{theorem}
        Let $f$ be a real function on $[\alpha - \delta, \alpha + \delta]$ such that
        \begin{enumerate}
            \itemsep0em
            \item $f(\alpha) = 0$.
            \item $f$ is twice differentiable, with non-zero derivatives.
            \item $f''$ is continuous.
            \item $|f''(x) / f'(y)| \leq M$ for all $x, y$.
        \end{enumerate}
        If $x_0 \in [\alpha - h, \alpha + h]$ where $h = \min\{\delta, 1 / M\}$, then
        the Newton-Raphson sequence generated by $x_0$ converges to the root
        $\alpha$ quadratically.
    \end{theorem}
    \begin{proof}
        Pick $x_n \in [\alpha - h, \alpha + h]$. Using Taylor's theorem, \[
            f(\alpha) = f(x_n) + f'(x_n)(\alpha - x_n) + \frac{1}{2}f''(c)(\alpha -
            x_n)^2.
        \] Also note that $f(\alpha) = 0$, and $x_n - x_{n + 1} = f(x_n) / f'(x_n)$.
        Thus, dividing by $f'(x_n)$ and substituting gives \[
            \alpha - x_{n + 1} = -\frac{1}{2}\frac{f''(c)}{f'(x_n)}(\alpha - x_n)^2.
        \] Using our estimates on $f''(c) / f'(x_n)$ and $x_n$ along with $h \leq 1
        / M$, we see that \[
            |\alpha - x_{n + 1}| \leq \frac{1}{2}Mh |\alpha - x_n| \leq \frac{1}{2}
            |\alpha - x_n|.
        \] Indeed, we have shown that \[
            |\alpha - x_n| \leq \frac{1}{2^n}|\alpha - x_0|,
        \] which directly gives the convergence $x_n \to \alpha$. Furthermore, we
        have \[
            \frac{|\alpha - x_{n + 1}|}{|\alpha - x_n|^2} = \frac{1}{2}
            \left|\frac{f''(c)}{f'(x_n)}\right| \leq \frac{1}{2}M,
        \] hence taking the limit $n \to \infty$ proves that the convergence is quadratic.
    \end{proof}
    \begin{corollary}
        Suppose that $f$ satisfies the conditions of the previous theorem, along with
        $f' > 0$ and $f'' > 0$ on some interval $[\alpha, x]$. Then, the
        Newton-Raphson sequence generated by $x_0 \in [\alpha, x]$ converges to the
        root $\alpha$ quadratically.
        \begin{remark}
            The convexity of $f$ means that the tangent drawn at $x_n$ lies below the
            curve, and hence cuts the $x$-axis between $\alpha$ and $x_n$.
        \end{remark}
    \end{corollary}

    \begin{theorem}
        If $\alpha$ is a multiple root of $f$ such that $f(\alpha) = 0$, $f'(\alpha)
        = 0$, $f''(\alpha) \neq 0$, then the Newton-Raphson sequence converges to
        $\alpha$ linearly under suitable conditions.
    \end{theorem}
    \begin{proof}
        Use Rolle's Theorem to replace $f'(x_n) = f'(x_n) - f'(\alpha) = f''(a)(x_n
        - \alpha)$.
    \end{proof}

    \subsection{Secant method}
    The chief difference between this method as Newton's method is that we
    approximate the tangent with a secant, i.e.\ perform an approximation of the
    derivative, \[
        f'(x)h \approx f(x + h) - f(x)
    \] for small $h$. Thus, our iterations proceed as \[
        x_{n + 1} = x_n - f(x_n) \frac{x_n - x_{n - 1}}{f(x_n) - f(x_{n - 1})}.
    \] 

    \begin{theorem}
        Let $f$ be a real function on $[a, b]$ such that \begin{enumerate}
            \itemsep0em
            \item $f(\alpha) = 0$ where $\alpha \in (a, b)$.
            \item $f$ is continuously differentiable, with non-zero derivatives.
        \end{enumerate}
        Then, there exists $\delta > 0$ such that the sequence generated by the
        secant method converges to $\alpha$ when $x_0, x_1 \in (\alpha - \delta,
        \alpha + \delta)$.
    \end{theorem}
    \begin{proof}
        Consider \[
            \alpha - x_{n + 1} = \alpha - x_n + f(x_n)\frac{x_{n} - x_{n - 1}}{f(x_n)
            - x_{n - 1}}.
        \] Now, use the Mean Value Theorem to write $f(x_n) = f(x_n) - f(\alpha) =
        f'(\xi)(x_n - \alpha)$ for some $\xi$ between $\alpha$ and $x_n$. Similarly,
        write $f(x_n) - f(x_{n - 1}) = f'(\zeta)(x_n - x_{n - 1})$ for some $\zeta$
        between $x_n$ and $x_{n - 1}$. Thus, \[
            \alpha - x_{n - 1} = \alpha - x_n + \frac{f'(\xi)(x_n -
            \alpha)}{f'(\zeta)} = (\alpha - x_n)\left(1 -
            \frac{f'(\xi)}{f'(\zeta)}\right).
        \] We want $|1 - f'(\xi) / f'(\zeta)| < 1$. Since $f'(\alpha) \neq 0$, there
        is a $\delta$-neighbourhood of $\alpha$ where $3f'(\alpha) / 4 < f'(x) <
        5f'(\alpha) / 4$ (without loss of generality) using the continuity of $f'$.
        Thus, whenever $x_0, x_1 \in (\alpha - \delta, \alpha + \delta)$, we have
        $\xi, \zeta$ belonging to the same neighbourhood. This gives $3 / 5 <
        f'(\zeta) / f'(\xi) < 5 / 3$. This gives \[
            -\frac{2}{3} < 1 - \frac{f'(\xi)}{f'(\zeta)} < \frac{2}{5}.
        \] In other words, $|1 - f'(\xi) / f'(\zeta)| < 2 / 3$, so \[
            |\alpha - x_{n + 1}| < \frac{2}{3}|\alpha - x_n|,
        \] which directly gives $x_{n} \to \alpha$. \\

        The order of convergence turns out to be $\varphi = (1 + \sqrt{5}) / 2$. To
        show this, we want \[
            \lim_{n \to \infty} \frac{|\alpha - x_{n + 1}|}{|\alpha - x_n|^\varphi} >
            0.
        \] Assume that $f'(\alpha) > 0$, $f''(\alpha) > 0$. First, we will show that
        \[
            \lim_{n \to \infty} \frac{|\alpha - x_{n + 1}|}{|\alpha - x_n| |\alpha -
            x_{n - 1}|} = \frac{f''(\alpha)}{2f'(\alpha)}.
        \] Denote the quantity in the limit as $\psi(x_n, x_{n - 1})$. We examine the
        equivalent limit \[
            \lim_{x_{n - 1} \to \alpha} \lim_{x_n \to \alpha} \psi(x_n, x_{n - 1}).
        \] Like before, write \[
            \alpha - x_{n + 1} = (\alpha - x_n)\left(1 - \frac{f'(\xi)(x_n - x_{n -
            1})}{f(x_n) - f(x_{n - 1})}\right), 
        \] hence \[
            \frac{\alpha - x_{n + 1}}{(\alpha - x_n)(\alpha - x_{n - 1})} =
            \frac{1}{\alpha - x_{n - 1}} \left[1 - \frac{f'(\xi)(x_n - x_{n -
            1})}{f(x_n) - f(x_{n - 1})}\right].
        \] Thus, \begin{align*}
            \lim_{x_n \to \alpha} \psi(x_n, x_{n - 1}) &= \frac{1}{\alpha - x_{n -
            1}}\left[1 + \frac{f'(\alpha)(\alpha - x_{n - 1})}{f(x_{n - 1})}\right]
            \\
            &= \frac{f(x_{n - 1}) + f'(\alpha)(\alpha - x_{n - 1})}{f(x_{n -
            1})(\alpha - x_{n - 1})}.
        \end{align*}
        Use Taylor's Theorem to approximate \[
            f(x_{n - 1}) = f(\alpha) + f'(\alpha)(x_{n - 1} - \alpha) +
            \frac{1}{2}f''(\eta)(x_{n - 1} - \alpha)^2,
        \] giving \[
            \lim_{x_n \to \alpha} \psi(x_n, x_{n - 1}) =
            \frac{f''(\eta)(\alpha - x_{n - 1})^2}{2f(x_{n - 1})(\alpha - x_{n -
            1})},
        \] and use the Mean Value Theorem to write $f(x_{n - 1}) =
        f'(\kappa)(x_{n - 1} - \alpha)$ giving \[
            \lim_{x_n \to \alpha} \psi(x_n, x_{n - 1}) =
            -\frac{f''(\eta)}{2f'(\kappa)},
        \] This gives \[
            \lim_{x_{n - 1} \to \alpha}\lim_{x_n \to \alpha} |\psi(x_n, x_{n - 1})| =
            \frac{f''(\alpha)}{2f'(\alpha)} = C.
        \] Now, suppose that \[
            \lim_{n \to \infty} \frac{|\alpha - x_{n + 1}|}{|\alpha - x_n|^q} = A >
            0.
        \] Dividing, we have \[
            \lim_{n \to \infty} \frac{|\alpha - x_n|^{q - 1}}{|\alpha - x_{n - 1}|} =
            \frac{C}{A}, \qquad
            \lim_{n \to \infty} \frac{|\alpha - x_n|}{|\alpha - x_{n - 1}|^{1 / (q -
            1)}} = \left(\frac{C}{A}\right)^{1 / (q - 1)} > 0.
        \] For $q$ to be minimal, we must have $1 / (q - 1) = q$, or $q$ is the
        golden ratio
        $\varphi$.
    \end{proof}

    \subsection{Fixed point method}
    Note that a root of $f$ is simply a fixed point of $f + x$.

    \begin{theorem}
        Let $f\colon [a, b] \to [a, b]$ be continuous. Then, $f$ has a fixed point
        $\beta \in [a, b]$, $f(\beta) = \beta$.
    \end{theorem}
    
    Thus, let $f\colon [a, b] \to [a, b]$ be continuous. Define the fixed point
    sequence $x_{n + 1} = f(x_n)$, seeded by some $x_0 \in [a, b]$. Note that if this
    sequence converges with $x_n \to \beta$, then $\beta$ is a fixed point of $f$.

    \begin{definition}
        A function $f\colon [a, b] \to \R$ is said to be a contraction if there
        exists $L \in (0, 1)$ such that $|f(x) - f(y)| \leq L|x - y|$ for all $x,
        y\in [a, b]$.
        \begin{remark}
            Note that $f$ is Lipschitz continuous. If $f$ is also differentiable,
            then $|f'| < 1$.
        \end{remark}
    \end{definition}
    
    \begin{theorem}
        Let $f\colon [a, b] \to [a, b]$ be a contraction map. Then, any fixed point
        sequence converges to the unique fixed point of $f$.
    \end{theorem}
    \begin{proof}
        First, we show that $f$ has at most one fixed point. Let $\beta_1, \beta_2$
        be fixed points of $f$. Then, $|f(\beta_1) - f(\beta_2)| \leq L|\beta_1 -
        \beta_2|$ where $L \in (0, 1)$. This forces $\beta_1 = \beta_2$. Thus, $f$
        has a unique fixed point in $[a, b]$.

        Let $\{x_n\}$ be a fixed point iteration. Then, \[
            |x_{n + 1} - \beta| = |f(x_n) - f(\beta)| \leq L |x_n - \beta|,
        \] which directly gives $x_n \to \beta$.
    \end{proof}


    \section{Interpolation}
    \subsection{Lagrange interpolation}
    \begin{theorem}
        Let $x_1, \dots, x_n \in \R$ be distinct, and let $y_1, \dots, y_n \in \R$.
        Then, the following polynomial of degree $n - 1$ satisfies $p(x_i) = y_i$. \[
            p(x) = \sum_{i = 1}^{n} \prod_{j \neq i} \frac{x - x_j}{x_i - x_j} y_i.
        \] Furthermore, this choice of $p$ is unique.
    \end{theorem}
    \begin{proof}
        The polynomials \[
            p_i(x) = \prod_{j \neq i} \frac{x - x_j}{x_i - x_j}
        \] satisfy $p_i(x_j) = \delta_{ij}$. These $p_i$ form a basis of
        $\mathscr{P}^{n - 1}$, the space of polynomials of degree at most $n - 1$.
    \end{proof}
    
\end{document}
