\documentclass[10pt]{article}

\usepackage[T1]{fontenc}
\usepackage{geometry}
\usepackage{amsmath, amssymb, amsthm}
\usepackage{xcolor}
\usepackage{listings}

\geometry{a4paper, margin=1in}

\renewcommand{\labelenumi}{(\roman{enumi})}

\newcounter{prob}
\newcommand{\problem}{\stepcounter{prob}\paragraph{Exercise \arabic{prob}}}
\newcommand{\solution}{\paragraph{Solution}}

\newcommand{\C}{\mathbb{C}}
\newcommand{\R}{\mathbb{R}}
\newcommand{\Q}{\mathbb{Q}}
\newcommand{\Z}{\mathbb{Z}}
\newcommand{\N}{\mathbb{N}}

\newtheorem*{lemma}{Lemma}

\definecolor{codegreen}{rgb}{0,0.6,0}
\definecolor{codegray}{rgb}{0.5,0.5,0.5}
\definecolor{codepurple}{rgb}{0.58,0,0.82}
\definecolor{backcolour}{rgb}{1,1,1}

\lstdefinestyle{mystyle}{
    backgroundcolor=\color{backcolour},
    commentstyle=\color{codegreen},
    keywordstyle=\color{blue},
    numbers=none,
    stringstyle=\color{codepurple},
    basicstyle=\ttfamily\footnotesize,
    breakatwhitespace=false,
    breaklines=true,
    keepspaces=true,
    numbers=left,
    numbersep=5pt,
    showspaces=false,
    showstringspaces=false,
    showtabs=false,
    tabsize=4
}

\lstset{style=mystyle}

\title{Time complexity}
\author{Satvik Saha}
\date{}

\begin{document}
    \noindent\textbf{IISER Kolkata} \hfill \textbf{Time complexity}
    \vspace{3pt}
    \hrule
    \vspace{3pt}
    \begin{center}
    \LARGE{\textbf{MA3105: Numerical Analysis}}
    \end{center}
    \vspace{3pt}
    \hrule
    \vspace{3pt}
    Satvik Saha, \texttt{19MS154} \hfill \today
    \vspace{20pt}

    Consider the following code block, and suppose that the commented $c_i$'s
    represent the time cost of executing the corresponding statement.
    
    \begin{minipage}{\linewidth}  
    \begin{lstlisting}[language=C, numbers=none]

                bool unique (int a[]) {
                    for (i = 0; i <= n - 2; i++)            // c_1
                        for (j = i + 1; i <= n - 1; j++)    // c_2
                            if (a[i] == a[j])               // c_3
                                return false;
                    return true;
                }
    \end{lstlisting}
    \end{minipage}

    It is clear that the first \texttt{for} statement is executed $n$ times;
    the interior is executed $n - 1$ times. Similarly, the second \texttt{for}
    statement is executed $n - i$ times, with the interior executed $n - i - 1$
    times. The \texttt{if} block is executed the full $n - i - 1$ times every loop at
    worst, $1$ time at best. Thus, the total cost can be estimated as 
    \begin{align*}
        T(n) &= \sum_{i = 0}^{n - 1} c_1 + \sum_{i = 0}^{n - 2}\sum_{j = i + 1}^{n}
        c_2 + \sum_{i = 0}^{n - 2}\sum_{j = i + 1}^{n - 1} c_3 \\
        &= c_1n + \sum_{i = 0}^{n - 2} c_2(n - i) + \sum_{i = 0}^{n - 2} c_3(n - i -
        1) \\
        &= c_1n + (c_2 + c_3) \sum_{i = 0}^{n - 2} (n - i) - \sum_{i = 0}^{n - 2} c_3
        \\
        &= c_1n + (c_2 + c_3)n(n - 1) - \frac{1}{2}(c_2 + c_3)(n - 1)(n - 2) - c_3(n
        - 1).
    \end{align*}
    By expanding further and collecting coefficients, we have
    \begin{align*}
        T(n) &= c_1n + (c_2 + c_3)n^2 - (c_2 + c_3)n - \frac{1}{2}(c_2 + c_3)n^2 +
        \frac{3}{2}(c_2 + c_3)n - (c_2 + c_3) - c_3n + c_3 \\
        &= \frac{1}{2}(c_2 + c_3)n^2 + \frac{1}{2}(2c_1 + c_2 - c_3)n - c_2.
    \end{align*}
    As $n \to \infty$, we see that $T(n)$ grows quadratically. We write $T(n) \in
    O(n^2)$.

    \begin{lemma}
        Given any polynomial \[
            f(n) = a_kn^k + a_{k - 1}n^{k - 1} + \dots + a_1n + a_0
        \] with at least one non-zero coefficient $a_i$, we can write $f(n) \in
        O(n^k)$.
    \end{lemma}
    \begin{proof}
        Set $M = |a_k| + |a_{k - 1}| + \dots + |a_0|$. Then for all $n \geq 1$, the
        triangle inequality gives
        \begin{align*}
            |f(n)| &= \left|a_kn^k + a_{k - 1}n^{k - 1} + \dots + a_1n + a_0\right|
            \\
            &\leq |a_k|n^k + |a_{k - 1}|n^{k - 1} + \dots + |a_1|n + |a_0| \\
            &\leq |a_k|n^k + |a_{k - 1}|n^k + \dots + |a_1|n^k + |a_0|n^k \\
            &= M n^k. \qedhere
        \end{align*}
    \end{proof}

\end{document}
