\documentclass[10pt]{article}

\usepackage[T1]{fontenc}
\usepackage{geometry}
\usepackage{amsmath, amssymb, amsthm}
\usepackage{array} 
\usepackage{enumitem}
\usepackage{siunitx}

\geometry{a4paper, margin=1in}
\setlength\parindent{0pt}
% \renewcommand\qedsymbol{$\blacksquare$}
\newcolumntype{L}{l@{\quad\quad}}
\newcounter{prob}
\def\problem{\stepcounter{prob}\paragraph{Problem \arabic{prob}}}
\def\solution{\\\\\textbf{Solution }}

\begin{document}
        \par\textbf{IISER Kolkata} \hfill \textbf{Assignment IV}
        \vspace{3pt}
        \hrule
        \vspace{3pt}
        \begin{center}
                \LARGE{\textbf{CH1201 : General Physical Chemistry}}
        \end{center}
        \vspace{3pt}
        \hrule
        \vspace{3pt}
        Satvik Saha, \texttt{19MS154}\hfill\today
        \vspace{20pt}

        \problem What happens when the vapour pressure of a liquid is equal to the external pressure?
        \solution We know that the vapour pressure of a liquid varies non-linearly with temperature according to the Clausius-Clapyron equation,
        \[
        \frac{\mathrm{d} p}{\mathrm{d}T} \;=\; \frac{\Delta H_v}{T\Delta v} > 0.
        \]
        Here, $\Delta H_v$ is the specific latent heat of vapourisation and $\Delta v$ is the change in specific volume during the phase transition from
        liquid to vapour. Thus, the vapour pressure increases with increase in temperature. \\

        When the vapour pressure becomes equal to the external pressure, the liquid is said to boil. Bubbles of vapour are able to form within the
        bulk of the liquid and this marks the phase transition from a liquid to vapour. The temperature at which this happens is called the boiling
        point of the liquid.

        \problem What is angle of contact of a liquid? On what factors does the angle of contact depend?
        \solution The angle of contact between a liquid and a solid is the angle between the tangen tot the liquid drawn at the point of contact
        and the surface of the solid inside the liquid. This is specific to that particular liquid-solid pair. It depends on the following factors.
        \begin{enumerate}[label=(\roman*), itemsep=0pt, topsep=\parsep]
                \item The nature of the solid surface.
                \item The nature of the liquid.
                \item The value of the interfacial tension.
                \item The temperature of the system.
        \end{enumerate}
        For example, consider a droplet of liquid on top of a solid surface, with the top exposed to air. We denote $\gamma_{la}$ to be the 
        surface tension along the liquid-air boundary, and similarly $\gamma_{ls}$ and $\gamma_{as}$ to be the surface tensions along the 
        liquid-solid and air-solid boundaries respectively. The former is directed upwards, at an angle $\theta$ with the solid surface, 
        while the latter two act opposite to one another along the solid surface. At equilibrium, the forces along the solid surface balance,
        and we obtain
        \[
                \gamma_{as} \;=\; \gamma_{ls} \,+\, \gamma_{la}\cos{\theta}.
        \]
        Here, $\theta$ is the angle of contact.
        
        \problem What is the fluidity of a liquid? On what factors does it depend?
        \solution The fluidity of a liquid, $\phi$, is the reciprocal of its dynamic viscosity $\eta$. While viscosity is a measure of a liquid's
        resistance to deformation/flow, fluidity is a measure of its flowing capacity. Fluidity depends on the following factors.
        \begin{enumerate}[label=(\roman*), itemsep=0pt, topsep=\parsep]
                \item \textit{Size of molecules:} Liquids with larger/heavier molecules have less fluidity than those with smaller/lighter molecules. 
                \item \textit{Shape of molecules:} Liquids with spherical molecules have greater fluidity than those with planar/irregular molecules. 
                \item \textit{Impurities:} The presence of impurities decreases the fluidity of a liquid. 
        \end{enumerate}

        \problem What is the difference between nematic, smectic and cholesteric liquid crystals? Name two molecules in each category.
        \solution These molecules are calamitic (rod-like) liquid crystals. Nematic liquid crystals show a thread-like structure, which is visible under
        polarized light (not X-rays). They are orientationally ordered, i.e.\ when stress is applied, their planar structure is lost but the molecules
        remain oriented together.\\

        Cholesteric liquid crystals are a class of nematic crystals which also show a colour effect under polarized light.
        This is because of the presence of chiral groups, which gives a helical twist to the orientation of the long axis of the molecules. \\

        Smectic liquid crystals are both orientationally and translationally ordered. When stress is applied, the different layers glide over each
        other, and the orientation of molecules within a layer is preserved. Smectic liquid crystals can be identified by both polarized light
        and by X-rays. \\

        Examples of each are as follows.
        \begin{enumerate}[label=(\roman*), itemsep=0pt, topsep=\parsep]
                \item \textit{Nematic (ordinary):} p-azoxy anisole, p-methoxy cinnamic acid.
                \item \textit{Cholesteric:} hydroxypropyl cellulose, cholesteryl benzoate.
                \item \textit{Smectic:} Ethyl p-azoxy benzoate, Ethyl p-azoxy cinnamate.
        \end{enumerate}
        

        \problem The surface tension of ethyl acetate ($T_C = \SI{523}{\kelvin}$) is \SI{25}{dyne\per\cm} at \SI{0}{\celsius}.
        Estimate its value at \SI{50}{\celsius}.
        \solution We use the E\"otv\"os-Ramsay-Shields equation
        \[
        \gamma(MV)^{2 /3} \;\propto\; T_C - T - \SI{6}{\kelvin}.
        \]
        Note that \SI{523}{\kelvin} = \SI{250}{\celsius}. Thus,
        \[
        \gamma_{\SI{50}{\celsius}} \;=\; \gamma_{\SI{0}{\celsius}}\cdot\frac{250 - 50 - 6}{250 - 0 - 6} \;=\; 25\times\frac{194}{244} \;=\; 
                \SI{20}{dyne\per\cm}.
        \]

        \problem If the levels of water and benzene that rose in the same capillary are \SI{9.9}{\cm} and \SI{4.5}{\cm} respectively,
        calculate the surface tension of benzene. Also calculate the radius of the tube.
        Given that $\gamma_\text{water} = \SI{72.75}{dyne\per\cm}$ at \SI{20}{\celsius}, and the densities of water and benzene are
        \SI{0.9982e3}{\kg\per\m^3} and \SI{0.8785e3}{\kg\per\m^3} respectively.
        \solution We use the formula 
        \[
                \gamma \;=\; \frac{1}{2}r\rho gh,
        \]
        where $r$ is the radius of the tube, $\rho$ is the density of the liquid and $h$ is the height of the liquid column.
        Thus, we have
        \[\gamma_\text{benzene} \;=\; \gamma_\text{water}\cdot\frac{(\rho h)_\text{benzene}}{(\rho h)_\text{water}}
                \;=\; 72.75\times\frac{0.8785 \times 4.5}{0.9982 \times 9.9} \;=\; \SI{29.10}{dyne\per\cm}.\]
        
        Note that \SI{1}{dyne\per\cm} = \SI{1e-3}{\newton\per\m}.
        Thus, the radius of the capillary is simply
        \[
        r \;=\; \left.\frac{2\gamma}{\rho gh}\;\right|_\text{water} \;=\; \frac{2\times 72.75\times 10^{-3}}{\SI{0.9982e3}{}\times 9.8\times \SI{9.9e-2}{}}
                \;=\; \SI{0.15}{\mm}.
        \]

        \problem A liquid $\mathcal{A}$ has half the surface tension and twice the density of liquid $\mathcal{B}$ at \SI{25}{\celsius}.
        If in a capillary the rise is \SI{10}{\cm} for liquid $\mathcal{A}$ , what will the rise of liquid $\mathcal{B}$ be at \SI{25}{\celsius}? 
        \solution For a given capillary tube, we have $h \propto {\gamma}/{\rho}$. Thus,
        \[
        h_\mathcal{B} \;=\; h_\mathcal{A}\cdot\frac{\gamma_\mathcal{B}}{\gamma_\mathcal{A}}\cdot\frac{\rho_\mathcal{A}}{\rho_\mathcal{B}}
                \;=\; h_\mathcal{A}\cdot \frac{2}{1}\cdot \frac{2}{1} \;=\; 4 h_\mathcal{A} \;=\; \SI{40}{\cm}.
        \]

        \problem The viscosities of water are \SI{0.018}{poise} and \SI{0.009}{poise} at \SI{0}{\celsius} and \SI{25}{\celsius} respectively.
        Calculate the average value of viscosity activation energy, assuming it to be constant over this temperature.
        \solution We use the relation
        \[
        \eta \;\propto\; \exp\left(\frac{E_\text{viscosity}}{RT}\right).
        \]
        Thus,
        \[
        \frac{\eta_1}{\eta_2} \;=\; \exp\left(\frac{E_\text{viscosity}}{R}\left[\frac{1}{T_1} - \frac{1}{T_2}\right]\right) .
        \]
        Substituting,
        \[
        \frac{0.018}{0.009} \;=\; \exp\left(\frac{E_\text{viscosity}}{8.314}\left[\frac{1}{273} - \frac{1}{298}\right]\right),
        \]
        \[
        E_\text{viscosity} \;=\; 8.314\times\frac{273\times 298}{298 - 273}\times \log{2} \;=\; \SI{18.75}{\kilo\joule\per\mole}.
        \]

        \problem What will be the pressure inside a soap bubble of radius \SI{0.1}{\mm} kept in air?
        Given that $\gamma_\text{soap water} = \SI{150}{dyne\per\cm}$ and atmospheric pressure is \SI{760}{\mm} of Hg.
        \solution
        Note that 760 mm of Hg = \SI{1.013e5}{\pascal}.
        We use the relation
        \[
                p \;=\; p_0 \,+\, \frac{4\gamma}{r} \;=\; p_0 + \frac{4 \times 150}{0.1} \;=\; p_0 +\SI{6000}{\pascal} \;=\; \SI{1.073e5}{\pascal}.
        \]

        \problem In the absolute method of determination of the viscosity coefficient $\eta$ using the Poiseuille equation, what should the error in the
        radius be if the error in $\eta$ is to be kept within \SI{4}{\percent}?
        \solution From Poiseuille's equation,
        \[
        \eta \;=\; \frac{\pi r^4 \Delta{p}}{8Ql},
        \]
        where $Q$ is the volumetric rate of flow. Thus, we can write
        \[ \frac{\delta\eta}{\eta} \;=\; 4 \frac{\delta r}{r} \,+\, \frac{\delta(\Delta p)}{\Delta p} \,+\, \frac{\delta Q}{Q} \,+\, \frac{\delta l}{l}
                \;\geq\; 4 \frac{\delta r}{r}.      
        \]
        We do this by taking logarithms, differentiating, and considering absolute values of deviations $\delta x$.
        Hence, the relative error must be bound by
        \[
        \frac{\delta r}{r} \;\leq\; \frac{1}{4}\frac{\delta\eta}{\eta} \;=\; \SI{1}{\percent}.
        \]

\end{document}
