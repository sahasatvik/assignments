\documentclass[10pt]{article}

\usepackage[T1]{fontenc}
\usepackage{geometry}
\usepackage{amsmath, amssymb, amsthm}
\usepackage{array} 
\usepackage{enumitem}
\usepackage{siunitx}
\usepackage{chemformula}

\geometry{a4paper, margin=1in}
\setlength\parindent{0pt}
% \renewcommand\qedsymbol{$\blacksquare$}
\newcolumntype{L}{l@{\quad\quad}}
\newcounter{prob}
\def\problem{\stepcounter{prob}\paragraph{Problem \arabic{prob}}}
\def\solution{\\\\\textbf{Solution }}

\begin{document}
        \par\textbf{IISER Kolkata} \hfill \textbf{Assignment V}
        \vspace{3pt}
        \hrule
        \vspace{3pt}
        \begin{center}
                \LARGE{\textbf{CH1201 : General Physical Chemistry}}
        \end{center}
        \vspace{3pt}
        \hrule
        \vspace{3pt}
        Satvik Saha, \texttt{19MS154}\hfill\today
        \vspace{20pt}

        \textit{Note:} All logarithms are natural unless specified otherwise.

        \problem Distinguish between the {\it molecularity} and {\it order} of a chemical reaction.
        For which type of reactions are the order and molecularity the same?
        \solution The \textit{molecularity} of a reaction is the number of particles (atoms/molecules/ions) which must collide simultaneously
        together for the formation of the products. Consider a reaction with reactants $A_1, A_1, \dots, A_n$. At a given temperature,
        the rate of this reaction can be written in the form $R = k[A_1]^{\alpha_1}[A_2]^{\alpha_2}\dots[A_n]^{\alpha_n}$, where the 
        exponents $\alpha_i$ are all determined experimentally. Each exponent $\alpha_i$ is called the \textit{order} of the reaction with respect to 
        the corresponding reactant $A_i$. The sum of these exponents $\alpha_1 + \alpha_2 + \dots \alpha_n$ is called the \textit{order} of
        the chemical reaction. \\

        The order and molecularity of a chemical reaction are equal when the reaction is \textit{elementary}, i.e.\ it proceeds in a single
        step exactly as written in the chemical equation. For example, consider the reaction \ch{a A + b B -> c C + dD}.
        If this proceeds in a single step, with exactly $a$ molecules of $A$ colliding with $b$ molecules of $B$, to produce
        the products as written, we expect the rate law to have the form $R = k[A]^a[B]^b$. In this case, both order and molecularity
        of the reaction are $a + b$.

        \problem While it is expected that a large amount of substance would take a longer time to decompose, the dependence of the half-life
        ($t_{1/2}$) on the initial concentration does not indicate this in general. Explain.
        \solution We first derive the general expression for the half life of an $n^\text{th}$ order reaction, governed by
        $-\mathrm{d}{x} /\mathrm{d}{t} \;=\; kx^n $. When $n = 1$, it is easily seen that $x(t) = x_0 e^{-kt}$ is the unique solution.
        Otherwise, rearranging and integrating, we have $1 /x^{n-1} = 1 /x_0^{n-1} + (n-1)kt$. Substituting $x = x_0 /2$, we have
        \begin{align*}
        t_{1 /2} \;&=\; \frac{\log{2}}{k}, & n = 1,\\
        t_{1 /2} \;&=\; \frac{2^{n-1} - 1}{(n-1)k}\cdot \frac{1}{x_0^{n-1}} \;\propto\; \frac{1}{x_0^{n-1}}, & n > 1.
        \end{align*}
        Thus, the half life of a reaction generally decreases with the increase in initial concentration of the reactant, and is a contstant
        for first order reactions. Hence, a larger initial concentration generally means that the reaction proceeds faster, i.e.\ 
        the substance decomposes quickly. \\

        We may explain this qualitatively by noting that with a higher initial concentration, reactant particles are more abundant
        and hence have a greater probability of colliding with each other and proceeding with the reaction. This makes the process faster.
        In the special case of the first order reaction, each particle decomposes independently of each other,
        and hence has a mean lifespan of $1 /k$. Here, the initial concentration has no effect on the half life.

        \problem For a chemical reaction, the rate constant is given by $k = \SI{1.5e-3}{\per\s}$ at \SI{25}{\celsius}.
        If the initial concentration of the reactant is \SI{0.5}{\mole\per\litre}, determine the rate of the reaction after \SI{30}{\minute}.
        \solution Note that the given reaction must be of the first order, since the rate constant has dimension $T^{-1}$. Let the
        concentration of the only reactant be expressed by $x(t)$. Thus, we have
        \[
        -\frac{\mathrm{d} x}{\mathrm{d}t} \;=\; kx.
        \]
        Integrating, and setting $x_0 = x(0)$, we have
        \[
                x(t) \;=\; x_0 e^{-kt}.
        \]
        Plugging in $t = \SI{30}{min} = \SI{1800}{\s}$, we have 
        \[
                x(\SI{30}{\min}) \;=\; 0.5 \times e^{- 1.5\times 10^{-3} \times 1800} \;=\; \SI{0.034}{\mole\per\litre}.
        \]

        Thus, the rate of the reaction is given by
        \[
                R \;=\; k x \;=\; 1.5\times 10^{-3} \times 0.034 \;=\; \SI{5.04e-5}{\mole\per\litre\per\s}.
        \]

        \problem The rate expression for the reaction \ch{A_{(g)} + B_{(g)} -> C_{(g)}} is given by $R = k [A]^{\frac{1}{2}}\, [B]^{2}$.
        What changes in rate will occur if the initial concentrations of A and B increase by factors of 4 and 2 respectively?
        \solution Using proportionality, we have
        \[R' \;=\; R \times 4^{\frac{1}{2}} \times 2^{2} \;=\; 8R.\]
        Hence, the rate of the reaction increases by a factor of 8.

        \problem Show that the concentration of the product P for a first order irreversible reaction during the initial period (time $t$ is very small)
        is given by the equation $[P] \;=\; [A]_0\, (kt - \frac{1}{2}k^2t^2)$.
        \solution For the irreversible reaction \ch{A ->[ k ] P}, we have
        \[
        \frac{\mathrm{d} [P]}{\mathrm{d}t} \;=\; -\frac{\mathrm{d} [A]}{\mathrm{d}t} \;=\; k[A].
        \]
        Rearranging and integrating over time,
        \[
        \int_{[A]_0}^{[A]} \frac{\mathrm{d}{[A]}}{[A]} \;=\; - \int_0^t k \mathrm{d}{t},   
        \]
        \[
        \log\frac{[A]}{[A]_0} \;=\; -kt.
        \]

        Note that if $n$ moles of A react, we must end up with the same number of moles of P. Thus, $[P] = [A]_0 - [A]$.
        Plugging this in,
        \[
        [P] \;=\; [A]_0 (1 - e^{-kt}).
        \]
        We now use the Taylor expansion $e^x = 1 + x + \frac{1}{2}x^2 + O(x^3)$, which serves as an good approximation for small $x$.
        Thus, for small $t$, we have
        \[
        [P] \;=\; [A]_0 \left(kt - \frac{1}{2}k^2t^2 \right).
        \]

        \problem Prove that the half-life ($t_{1/2}$) of a first order chemical reaction varies with temperature according to $\log{t_{1/2}} \propto 1 /T$.
        \solution First, we derive the expression for $t_{1 /2}$ for a first order reaction. We demand that after one half-life,
        the concentration of the reactant halves. Thus, using our previously derived expression $x = x_0 e^{-kt}$, we see that
        when $x = x_0 / 2$,
        \[t_{1 /2} \;=\; \frac{\log{2}}{k}.\]
        Now, we invoke the Arrhenius equation which says that the rate constant $k$ varies with temperature $T$ as
        \[k \;=\; A e^{-E_a /RT},\]
        where $E_a$ is the activation energy, assumed to be contstant with temperature.
        This means that $\log{k} = -\left(\frac{1}{T}\right)\frac{E_a}{R} + \log{A}$, which is approximately proportional to $-1 /T$.
        Combining this with $t_{1 /2} \propto 1 / k$, we have $\log{t_{1 / 2}} = \log\log{2} - \log{k}$, which is approximately
        \[
        \log{t_{1 /2}} \;\stackrel{?}{\propto}\; -\log{k} \;\stackrel{?}{\propto}\; \frac{1}{T}.
        \]
        More accurately, a graph of $\log{t_{1 /2}}$ versus $1 /T$ will be linear, with a positive slope,
        since
        \[
        \log{t_{1 /2}} \;=\; \log\log{2} - \log{A} + \left( \frac{1}{T}  \right)\frac{E_a}{R}.
        \]

        \problem For a chemical reaction at \SI{60}{\celsius}, a plot of the inverse of the reactant concentration $(1 / [A])$ versus time 
        is a straight line with a slope of \SI{4e-2}{\litre\per\mole\per\s}. Calculate the time required (in seconds) for 1.0 M of reactant
        to decrease to 0.25 M.
        \solution We denote $[A] = x(t)$. Since the plot of $1 /x$ versus $t$ is a straight line, we must have $1 /x = bt + c$, for constants
        $b$ and $c$. Differentiating, $-\mathrm{d}{x}/ x^2 = b\,\mathrm{d}{t}$, i.e.\ $-{\mathrm{d} x}/{\mathrm{d}t} = bx^2$.
        Thus, this reaction is of the second order, with rate constant $b$ equal to the slope of the plot. \\
        To calulate the time taken, we simply use the equation of the straight line, $\Delta{(1 /x)}= b\Delta{t}  $. Thus,
        \[
        \Delta{t} \;=\; \frac{1}{b} \left[ \frac{1}{x_f} - \frac{1}{x_0} \right]
                \;=\; \frac{1}{4\times10^{-2}} \left[ \frac{1}{0.25} - \frac{1}{1.0} \right] 
                \;=\; 25 \times 3 \;=\; \SI{75}{\s}.
        \]

        \problem Ethyl acetate undergoes hydrolysis reaction in presence of \ch{NaOH} in an ethanol-water mixture at \SI{30}{\celsius}.
        In an experiment in which \SI{0.05}{\mole\, \dm^{-3}} of each reactant was present at time $t = 0$, the time for half change was
        \SI{1800}{\s} and the time for three-quarters change was \SI{5400}{\s}. Deduce the order of the reaction and calculate the rate constant.
        How much time is required to complete 10\% of the reaction?
        \solution
        For a general $n^\text{th}$ order reaction, we use the integrated rate law, $1 /x^{n-1} = 1 /x_0^{n-1} + (n-1)kt$.
        Substituting $x = x_0 /2$ and $x = x_0 /4$, which represent the half and three-quarter reaction points respectively, we obtain
        \[
        \frac{t_{1 /2}}{t_{3 /4}} \;=\; \frac{1}{2^{n-1} + 1}.
        \]
        Thus, we have $2^{n-1} + 1 = 5400 /1800 = 3$. Hence, $n = 2$, and the reaction is of the second order. Using $x_0 = 0.05$, we have
        \[
        k \;=\; \frac{1}{t_{1 /2} x_0} \;=\; \frac{1}{1800 \times 0.05} \;=\; \SI{1.1e-2}{\litre\per\mole\per\s}.
        \]

        For 10\% of the reaction, we substitute $x = 9x_0 / 10 = 0.045$ to obtain
        \[
        t \;=\; \frac{1}{k} \left[ \frac{1}{x} - \frac{1}{x_0} \right] \;=\; 90\times \left[ \frac{1}{0.045} - \frac{1}{0.05} \right]
                \;=\; \SI{200}{\s}.
        \]

        \problem In a gaseous reaction, when the inverse of the pressure of the reactant is plotted against time, a straight line is obtained with
        intercept \SI{150}{\per\bar} and slope \SI{2e-3}{\per\bar\per\s}. Calculate the half-life for the reaction.
        \solution We denote the pressure of the gas as $p$. We have $1 /p = kt + 1 /p_0$, where $p_0$ is the initial pressure.
        It is given that $1 /p_0 = \SI{150}{\per\bar}$, the intercept of the plot. Like before, differentiating and rearranging
        yields $-\mathrm{d}{p}/\mathrm{d}{t} = kp^2$, a second order rate law with rate constant $k$ equal to the slope of the plot.
        When $p = p_0 /2$, we have
        \[
        t_{1 /2} \;=\; \frac{1}{k p_0} \;=\; \frac{150}{2\times 10^{-3}} \;=\; \SI{75e3}{\s}.
        \]

        \problem In the first order reaction \ch{A <=>[ k1 ][ k_{-1} ] B}, the initial concentration of A and B are [A]$_0$ and 0 respectively.
        At equilibrium the concentration of A and B becomes [A]$_e$ and [B]$_e$ respectively.
        Derive an expression for the time taken by B to attain a concentration equal to 0.5[B]$_e$.
        For this reaction at \SI{75}{\celsius}, let $k_1 = \SI{1.2e-3}{\per\s}$ and $k_{-1} = \SI{3.3e-2}{\min^{-1}}$.
        Find the time required to produce B to half of its equilibrium concentration.
        \solution We denote $[A] = x(t)$ and $[B] = y(t)$. Note that at any point, the amount of B produced must equal the amount of
        A reacted, i.e.\ $y = x_0 - x$. We thus write
        \[
        \frac{\mathrm{d} y}{\mathrm{d}t} \;=\; k_1 x \,-\, k_{-1}y \;=\; -(k_1 + k_{-1})y + k_1x_0.
        \]
        Rearranging and integrating,
        \[
        \int_0^y \frac{\mathrm{d}{y}}{(k_1 + k_{-1})y - k_1x_0} \;=\; -\int_0^t \mathrm{d}{t},
        \]
        \[
        \log \frac{k_1x_0 - (k_1 + k_{-1})y}{k_1x_0}  \;=\; -(k_1 + k_{-1})t.
        \]
        At equilibrium, we must have $\mathrm{d}{y}/ \mathrm{d}{t} = 0$, thus $k_1x_0 = (k_1 + k_{-1})y_e$. Plugging this in,
        \[\log\frac{y_e}{y_e - y} \;=\; (k_1 + k_{-1})t.\]
        Hence, when $y = y_e / 2$, we have
        \[
        (k_1 + k_{-1})t_{1 / 2} \;=\; \log\frac{y_e}{y_e - \frac{1}{2}y_e} \;=\; \log{2}.
        \]
        Rearranging,
        \[
        t_{1 /2} \;=\; \frac{\log 2}{k_1 + k_{-1}}.
        \]
        Plugging in the given values for $k_1$ and $k_{-1}$, and using \SI{60}{\min^{-1}} = \SI{1}{\per\s}, we have
        \[
        t_{1 /2} \;=\; \frac{\log{2}}{1.2\times 10^{-3} + 0.55\times 10^{-3}} \;=\; \SI{396}{\s} \;=\; \SI{6}{\min}\:\SI{36}{\s}.
        \]

        \problem For a parallel reaction with $k_1$ and $k_2$ equal to \SI{3.42e-2}{\per\min} and \SI{1.14e-2}{{\per\min}} respectively,
        calculate the percentage of A converted into B and C, and also find the ratio of [B] and [C] after \SI{20}{\min}.
        \ch{A ->[ k_1 ] B} and \ch{A ->[ k_2 ] C} are in parallel.
        (\textit{We relabel $k_1'$ to $k_2$ for convenience.})
        \solution 
        We denote $[A] = x(t)$, $[B] = y(t)$ and $[C] = z(t)$.
        For a parallel reaction, we write
        \[
        -\frac{\mathrm{d}x}{\mathrm{d}t} \;=\; \frac{\mathrm{d} y}{\mathrm{d}t} \,+\, \frac{\mathrm{d} z}{\mathrm{d}t} \;=\; k_1x \,+\, k_2x
                \;=\; (k_1 + k_2)x.
        \]
        This is equivalent to the standard result $x(t) = x_0e^{-(k_1 + k_2)t}$. Thus,
        \[
        \frac{\mathrm{d} y}{\mathrm{d}t} \;=\; k_1x_0e^{-(k_1 + k_2)t},\quad\quad\text{and}\quad\quad
                \frac{\mathrm{d} z}{\mathrm{d}t} \;=\; k_2x_0e^{-(k_1 + k_2)t}.
        \]
        Rearranging and integrating,
        \[
        y(t) \;=\; \frac{k_1}{k_1 + k_2} x_0(1 - e^{-(k_1 + k_2)t}), \quad\quad\text{and}\quad\quad
                z(t) \;=\; \frac{k_2}{k_1 + k_2} x_0(1 - e^{-(k_1 + k_2)t}).
        \]
        As $t \to \infty$, we have
        \[
        y_{\infty} \;=\; \frac{k_1}{k_1 + k_2} x_0,  \quad\quad\text{and}\quad\quad z_{\infty} \;=\; \frac{k_2}{k_1 + k_2}x_0.
        \]

        Plugging in the given values for $k_1$ and $k_2$, we have
        \[
        y_\infty \;=\; \frac{3.42}{3.42 + 1.14} x_0 \;=\; 0.75 x_0,\quad\quad\text{and}\quad\quad
                z_\infty \;=\; \frac{1.14}{3.42 + 1.14} x_0 \;=\; 0.25x_0.
        \]
        Thus, \SI{75}{\percent} of A gets converted into B, and the remaining \SI{25}{\percent} gets converted into C.
        Also, note that at any time, $y/ z = k_1 / k_2 = 3$. Hence, after \SI{20}{\min}, the ratio of concentrations of B and C is 3.

        \problem The decomposition of ozone, \ch{2 O_3 <> 3 O_2}, is observed to obey the rate law $R = k[\ch{O_3}]^2\, [\ch{O_2}]^{-1}$.
        Suggest a mechanism that agrees with the rate law.
        \solution We propose the mechanism
        \[ \ch{O_3 <=>[ k_1 ][ k_{-1} ] O_2 + O},   \]
        \[ \ch{O_3 + O ->[ k_2 ] 2 O_2}. \]
        At equilibrium, the rates of the forward and backward reactions in the first step are equal, and thus $k_1[O_3] = k_{-1}[O_2][O]$.
        The rate of formation of \ch{O_2} in the second step is given by $k_2[O_3][O]$.
        Substituting for $[O]$, we have $R = (k_1 k_2 /k_{-1})\cdot [O_3]^2 /[O_2]$.
        Setting $k = k_1 k_2 /k_{-1}$, we obtain the desired rate equation.
\end{document}
