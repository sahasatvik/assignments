\documentclass[10pt]{article}

\usepackage[T1]{fontenc}
\usepackage{geometry}
\usepackage{amsmath, amssymb, amsthm}
\usepackage{array} 
\usepackage{enumitem}
\usepackage{siunitx}

\geometry{a4paper, margin=1in}
\setlength\parindent{0pt}
% \renewcommand\qedsymbol{$\blacksquare$}
\newcolumntype{L}{l@{\quad\quad}}
\newcounter{prob}
\def\problem{\stepcounter{prob}\paragraph{Problem \arabic{prob}}}
\def\solution{\\\\\textbf{Solution }}

\begin{document}
        \par\textbf{IISER Kolkata} \hfill \textbf{Assignment I}
        \vspace{3pt}
        \hrule
        \vspace{3pt}
        \begin{center}
                \LARGE{\textbf{ES1201 : Earth System Processes}}
        \end{center}
        \vspace{3pt}
        \hrule
        \vspace{3pt}
        Satvik Saha, \texttt{19MS154}\hfill\today
        \vspace{20pt}

        \problem List the four layers of Earth's atmosphere. How are they defined? Explain the thermal variation for each layer.
        \solution The four layers of the Earth's atmosphere are demarcated by the variation of temperature within them.
        \begin{enumerate}[label=(\roman*), itemsep=0pt, topsep=\parsep]
                \item \textit{Troposphere.} This layer extends from the surface to \SI{\sim 15}{\kilo\metre} near the equator and 
                \SI{\sim 10}{\kilo\metre} near the poled. Temperature decreases with increase in altitude.
                \item \textit{Stratosphere.} This layer extends from the top of the troposphere to \SI{50}{\kilo\metre}.
                Temperature increases with increase in altitude.
                \item \textit{Mesosphere.} This layer extends from the top of the stratosphere to \SI{90}{\kilo\metre}.
                Temperatur decreases with altitude.
                \item \textit{Thermosphere.} This layer extends beyond the top of the mesosphere.
                Temperature increases with increase in altitude.
        \end{enumerate}

        \paragraph{}
        These different trends in temperature can be explained by the different dominant modes of heating and heat transfer, and different gaseous
        composition in different layers. We start with the Earth's surface, which is heated by around half of all incoming solar radiation.
        This energy tries to escape into space as infrared radiation, but cannot penetrate the lower atmosphere easily because of absorption
        by greenhouse gases and clouds. Thus, the mode of convection dominates heat transfer in the troposhere.
        Hence, temperature decreases as the distance from the heated Earth surface increases.
        \paragraph{}
        Temperatures start increasing again in the stratosphere because of the presence of atmospheric ozone, which peaks at around
        \SI{30}{\kilo\metre}. This absorbs ultraviolet radiation, which is available higher up. Maximum heating is thus seen at
        \SI{50}{\kilo\metre}. Atmospheric ozone concentration drops off above the stratosphere, so this heating effect decreases, causing
        a temperature decrease in the mesosphere. In the thermosphere, the absorption of short wave ultraviolet radiation, this time
        by molecular oxygen, leads to an increase in temperature.
        
        \problem Identify two physical processes by which gases can absorb infrared radiation. Give examples of each process.
        \solution Gas molecules can absorb radiation by changing their {\it rate of rotation} or their {\it amplitude of vibration}.
        \paragraph{}
        A molecule is permitted to have certain discrete rotational frequencies, as governed by quantum mechanics. Each of these is
        associated with a certain amount of energy. When incident infrared radiation has exactly the right amount of energy,
        corresponding to the difference between two allowed rotational frequency energies, the molecule abosrbs the radiation
        i.e. the molecule absorbs a single photon of light.
        Analogously, molecules vibrate in different ways, and have discrete permitted amplitudes of vibration, each associated with 
        an energy level. The molecule can thus `jump' between energy levels when it absorbs or emits photons with energy exactly equal
        to the difference between the two energy levels.
        \paragraph{}
        For example, water molecules are capable of absorbing radiation with wavelength \SI{12}{\micro\metre} and above, due
        to their mode of rotation. This effect, when oberved in the Earth's atmosphere, is called the H\textsubscript{2}O rotation band.
        Carbon dioxide molecules have three modes of vibration, one of which is a bending mode. This allows the absorption of
        infrared radiation of wavelength \SI{15}{\micro\metre}, which when observed in the Earth's atmosphere is also given its own band.

        \problem Explain the physical causes of the greenhouse effect. Why are O\textsubscript{2} and N\textsubscript{2} not greenhouse gases?
        \solution The greenhouse effect is caused by the absorption of infrared radiation by certain gases in the atmosphere, called 
        greenhouse gases. These gas molecules absorb IR radiation by processes explained in the preceding answer. Hence, each 
        gas molecule is associated with a certain wavelength `window' where it can efficiently absorb radiation. \\
        Dioxygen and dinitrogen are perfectly symmetric diatomic molecules, and hence show no separation of charge within their molecules.
        As a result, they do not interact well with electric and magnetic fields. Hence, electromagnetic radiation like infrared
        radiation pass through unabsorbed, and these molecules do not contribute much to the greenhouse effect.\\
        Note that although carbon dioxide is a symmetric molecule, it is capable of bending while vibrating, which leads to 
        asymmetry.

        \problem Given that a \SI{300}{\kelvin} blackbody radiates its peak energy at the wavelength of about \SI{10}{\mm}, at what wavelength
        would a \SI{900}{\kelvin} blackbody radiate its peak energy?
        \solution We use Wien's Law, which states that the wavelength at which a blackbody emits the maximum amount of radiation flux
        is inversely proportional to its absolute temperature.
        \[
        \lambda_{max} \;\propto\; \frac{1}{T}.
        \]

        Applying this formula directly to the given problem,
        \[
        \lambda_{max, 900K} \;=\; \frac{T_{300K}}{T_{900K}}\cdot\lambda_{max, 300K} \;=\; \frac{10}{3}\; \SI{}{\mm}.
        \]
        
        Hence, a \SI{900}{\kelvin} blackbody emits its peak energy at a wavelength of approximately \SI{3.3}{\mm}.

        \problem Venus and Mars orbit the Sun at average distances of \SI{0.72}{AU} and \SI{1.52}{AU}, respectively.
        What is the solar flux at each planet? Venus has a planetary albedo of 0.8, and Mars has an albedo of 0.22.
        Determine the effective radiating temperatures of these planets.
        \solution We use the inverse square law, which states that the solar flux $S$ received by a planet is inversely proportional
        to the square of its radial distance $r$ from the Sun.
        \[
        S \;\propto\; \frac{1}{r^2}.
        \]

        Using the fact that the solar flux received by the Earth is \SI{1366}{\watt\per\metre^2}, we have
        \begin{align*}
                S_{\text{Venus}} \;&=\; \frac{1366}{0.72^2} \;=\; \SI{2635}{\watt\per\metre^2},\\
                S_{\text{Mars}} \;&=\; \frac{1366}{1.52^2} \;=\; \SI{591}{\watt\per\metre^2}.
        \end{align*}

        The effective radiating tempertaure $T$ of a planet can be obtained by assuming them to be blackbodies, then using
        the Stefan-Boltzmann Law to balance the incoming and outgoing radiation energies. A blackbody with radius $R$ emits 
        radiation at a rate $4\pi R^2\cdot \sigma T^4$, since it emits radiation proportional to the fourth power of its effective radiating
        temperature equally in all directions. However, a planet absorbs incoming solar radiation only from one direction,
        from which it has a projected area of only $\pi R^2$. Hence, the planet absorbs radiation at a rate $\pi R^2\cdot S$.
        The albedo $A$ simply means that a portion of this energy, $\pi R^2\cdot SA$, is reflected back into space. Putting this together,
        \[
        4\pi R^2\cdot \sigma T^4 \;=\; \pi R^2\,S \,-\, \pi R^2\,SA,
        \]
        \[
        T \;=\; \sqrt[4]{\frac{S}{4\sigma}(1 - A)}.
        \]

        Plugging in known values, with the Stefan-Boltzmann constant $\sigma = $ \SI{5.67e-8}{\watt\per\metre^2\per\kelvin^4},
        \begin{align*}
                T_{\text{Venus}} \;&=\; \SI{220}{\kelvin}, \\
                T_{\text{Mars}} \;&=\; \SI{212}{\kelvin}.
        \end{align*}
        We note that these are lower than Earth's effective radiating temperature \SI{255}{\kelvin} --- Venus has a far higher albedo than
        Earth, and Mars has a far lower incoming solar flux. Also, these values do not reflect the surface temperatures of these planets.
        For example, Venus has a significantly higher surface tempertaure due to the greenhouse gas effect.
\end{document}
