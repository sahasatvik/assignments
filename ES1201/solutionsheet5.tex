\documentclass[10pt]{article}

\usepackage[T1]{fontenc}
\usepackage{geometry}
\usepackage{amsmath, amssymb, amsthm}
\usepackage{array} 
\usepackage{enumitem}
\usepackage{siunitx}
\usepackage{chemformula}
% \usepackage{graphicx}
% \usepackage{caption}

\geometry{a4paper, margin=1in}
\setlength\parindent{0pt}
% \renewcommand\qedsymbol{$\blacksquare$}
\newcolumntype{L}{l@{\quad\quad}}
\newcounter{prob}
\def\problem{\stepcounter{prob}\paragraph{Problem \arabic{prob}}}
\def\solution{\\\\\textbf{Solution }}

\begin{document}
        \par\textbf{IISER Kolkata} \hfill \textbf{Assignment V}
        \vspace{3pt}
        \hrule
        \vspace{3pt}
        \begin{center}
                \LARGE{\textbf{ES1201 : Earth System Processes}}
        \end{center}
        \vspace{3pt}
        \hrule
        \vspace{3pt}
        Satvik Saha, \texttt{19MS154}\hfill\today
        \vspace{20pt}

        \problem What are the conditions that determine whether a planet retains or loses a gas from its atmosphere over geological time?
        What system of gases is used to obtain information about the early atmosphere, during the formation of the earth?
        \solution The retention of atmospheric gases is based on the strength of the planet's gravitational field, as well as the temperature
        of the given gas in the upper atmosphere. While gas molecules are in constant motion, the gravitational field ensures that it doesn't
        escape. However, a sufficiently fast/energetic particle can escape a planet's gravitational influence completely. The minimum
        velocity which a particle must possess in order to do so is called the escape velocity for a planet, and is given by
        \[
                v_{esc} \;=\; \sqrt{\frac{2GM}{r}.}
        \]
        Here, $M$ is the mass of the planet and $r$ is its radius. Now, molecules of gas move randomly, with different particles
        having different velocities, but their \textit{average} speed is related to the temperature of the gas by
        \[
                v_{avg} \;=\; \sqrt{\frac{8k_BT}{\pi m}}.
        \]
        Here, $m$ is the mass of one gas molecule and $T$ is the temperature of the gas. Note that this is expected of an ideal gas,
        but real gases follow similar behaviour. Thus, we are also only concerned with temperatures in the upper atmosphere, where
        molecules are unlikely to collide with each other and are thus free to escape purely by virtue of their speed.
        Now, the actual speeds of gas molecules lie on a Maxwell distribution, so some are faster
        and some are slower than $v_{avg}$. If a sufficient fraction of them exceed the escape velocity $v_{esc}$ of a planet,
        then that particular gas will be lost over time. We know that for retention over geologic time, of the order of the age of the Solar System,
        we must have $v_{avg} < v_{esc} /6$. Using this criterion for different gases (which have different molecular weights, hence different
        $v_{avg}$ for the same temperature), we can determine which gases are retained by a planet from its mass, radius and
        upper atmosphere temperature. \\

        The \ch{^{40}K}-\ch{^{40}Ar} and \ch{^{129}I}-\ch{^{129}Xe} systems are of particular interest in studying the evolution of the early atmosphere.
        These radioactive systems proceed with half lives of \SI{1.28}{} billion years and \SI{15.7}{} million years respectively.
        Thus, these systems allow us to determine both long term and early phase evolution of the Earth's atmosphere, and pinpoint
        degassing events. Essentially, the Earth's mantle had low ratios of \ch{I/Xe} and \ch{K/Ar} in the early stage.
        Thus, the ratios of the daughter products, the gases \ch{Xe} and \ch{Ar} would increase in the mantle over time at a steady rate.
        If a degassing event occurs, then these gases would be lost to the atmosphere, causing the parent/daughter ratio in the mantle
        to abruptly increase. We can measure this by considering the amount of mantle-bound daughter isotope, relative to a non-radiogenic isotope.
        With decay of the parent, this ratio increases in the mantle over time steadily. As soon as there is a degassing event, both
        radiogenic and non-radiogenic gases enter the atmosphere and the ratios in the mantle abruptly change, subsequently increasing at a much
        faster rate. The ratio in the atmosphere thus follows a different path from the ratio in the mantle.
        Note that the parent materials are preferentially retained in the mantle during this process. This is analogous to 
        the \ch{Hf}-\ch{W} radiogenic system. \\

        The different timescales of both systems allow us to consider two scenarios: degassing before the complete decay of \ch{^{129}I},
        or after the complete decay of \ch{^{129}I}. In the first case, the \ch{^{129}Xe}/\ch{^{130}Xe} ratios in the mantle and atmosphere
        deviate from each other from the time of degassing. In the second case, they remain the same in the mantle and atmosphere as
        the parent has had time to completely decay. Observations show high \ch{^{129}Xe}/\ch{^{130}Xe} and \ch{^{40}Ar}/\ch{^{36}Ar}
        ratios in the mantle relative to the atmosphere, which shows that about 80\% of the Earth's atmosphere must have outgassed
        early in its history, within tens of millions of years after accretion. This method has also been used to show that
        over 99\% of the Earth's early atmosphere was lost within the first \SI{100}{} million years.

\end{document}
