\documentclass[10pt]{article}

\usepackage[T1]{fontenc}
\usepackage{geometry}
\usepackage{amsmath, amssymb, amsthm}
\usepackage{array} 
\usepackage{enumitem}
\usepackage{siunitx}
\usepackage{graphicx}
\usepackage{caption}

\geometry{a4paper, margin=1in}
\setlength\parindent{0pt}
% \renewcommand\qedsymbol{$\blacksquare$}
\newcolumntype{L}{l@{\quad\quad}}
\newcounter{prob}
\def\problem{\stepcounter{prob}\paragraph{Problem \arabic{prob}}}
\def\solution{\\\\\textbf{Solution }}

\begin{document}
        \par\textbf{IISER Kolkata} \hfill \textbf{Assignment III}
        \vspace{3pt}
        \hrule
        \vspace{3pt}
        \begin{center}
                \LARGE{\textbf{ES1201 : Earth System Processes}}
        \end{center}
        \vspace{3pt}
        \hrule
        \vspace{3pt}
        Satvik Saha, \texttt{19MS154}\hfill\today
        \vspace{20pt}

        \problem Explain the effects of surface wind pattern on the oceanic circulation.
        \solution Surface winds drag surface ocean water along their path due to friction, giving rise to drift currents. This is complicated by the 
        Coriolis effect, which causes drift currents to deviate by approximately \SI{20}{\degree} from the direction of the wind (rightwards in the
        Northern hemisphere, leftwards in the Southern hemisphere). The propagation of this deflection downwards due to friction between layers of
        water leads to a phenomenon called the Ekman spiral, discussed in the next answer. \\

        Surface winds also lead to the formation of circular surface circulation patterns called gyres. For example, in the tropics in the Northern
        hemisphere, trade winds over the oceans blowing from the north-east produce westward flowing waters. When these encounter a continent boundary,
        they are forced north and south. The current flowing north eventually leaves the tropics and comes under the westerlies, which drive the
        waters eastwards. These similarly strike another continent boundary, and the southward portion of the deflected current completes the cycle.
        This gyre was a clockwise one -- an analogous process forms an anticlockwise gyre in the Southern hemisphere.


        \problem What is the Ekman spiral?
        Explain why Ekman transport occurs and what role it plays in producing oceanic gyres in the surface waters of the subtropical oceans.
        \solution The Ekman spiral is a phenomenon involving varying velocities of water layers with depth, caused by the combined effect of 
        surface winds and the Coriolis force. As explained previously, the surface layer of water generally moves at an angle of around
        \SI{20}{\degree} from the wind direction. Now, the layer below this one also has a tendency to move in this direction, due
        to friction with the upper layer. Because some kinetic energy is lost from friction (as heat), this layer moves more slowly.
        The Coriolis force further acts on this layer, so it deviates even more from the wind direction. This effect propagates
        downwards, with lower layers dragged along by upper layers, so the velocities of water decrease in magnitude and their deviation
        increases with depth, forming a spiral. \\

        Ekman transport is the net movement of water caused by the Ekman spiral effect. When all the movements of the layers in the spiral are added,
        we see that the net movement of water is perpendicular to the direction of surface wind (rightwards in the North, leftwards in the South).
        Thus, we see that in a gyre (as described in the previous answer), the surface water moves in a circular path (clockwise in the North,
        anticlockwise in the South) while the net movement of water is actually inwards, into the center of the gyre.  Thus, water masses
        converge here. Near the equator, we have ocean currents flowing westwards due to the northeast and southeast trade winds. This means
        that due to Ekman transport, the net movement of water is perpendicular, i.e.\ polewards. Thus, the water masses diverge here.


        \problem What is upwelling and where does it occur? Does Ekman transport play a role in upwelling and downwelling? Please explain.
        \solution The rise of ocean water from lower layers to the surface is called upwelling. Typically, this water is cooler than surface
        water. This happens when surface water diverges, forcing water from below to replace it.\\

        Similarly, downwelling is the sinking of surface water to lower levels. This happens when surface water converges and accumulates, forcing
        it to move downwards. \\

        We have already seen than Ekman transport leads to convergence of water towards the center of gyres. Thus, there are sites of downwelling.
        Also, we know that Ekman transport leads to divergence of west-flowing water near the equator. Thus, upwelling occurs here.


        \problem Explain the different characteristics of western and eastern boundary currents.
        \solution Boundary currents are the currents in the outermost parts of a gyre. Typically, the western boundary current is narrower
        and faster than the esatern boundary current, which is spread out over a broader region and has slower moving water. \\

        This phenomenon can be explained using the concept of vorticity -- the tendency of a fluid to rotate. This must be conserved throughout the
        flow of water. In the Northern hemisphere, the tendency of water to rotate anticlockwise increases with latitude; in the Southern hemisphere,
        the clockwise rotation increases with latitude. This is caused by earth's rotation, and is hence called planetary vorticity.
        The anticlockwise rotation tendency is chosen to be positive vorticity. In addition, varying velocities of water across the breadth
        of a current can cause it's flow to curl in a particular direction, i.e.\ into the side with the slower moving water. This is called
        current shear. This, together with additional sources of vorticity such as surface winds, comprises the relative vorticity. When
        the planetary vorticity is added, we obtain the absolute vorticity.\\
        
        Consider a gyre in the Northern hemisphere.
        The northeast trade winds and the westerlies produce a clockwise rotation of water, i.e.\ a negative vorticity.
        Now, when water flows along a coastline, friction causes the water closer to shore to slow down relative to the water further away.
        This current shear thus causes an anticlockwise tendency, i.e.\ increases the positive relative vorticity of water on both the 
        eastern and western boundaries. However, water flowing along the western boundary flows northwards, where the planetary vorticity is
        more positive. To conserve vorticity, the relative vorticity of the water has a negative contribution. On the other hand,
        water flows southwards along the eastern boundary, so the decrease in planetary vorticity is compensated by the increase in positive
        relative vorticity. To balance vorticities along both boundaries, we must have a greater positive vorticity contribution from current 
        shear on the western boundary compared to the eastern boundary. Thus, the western boundary current must be narrower and deeper,
        maximizing friction, while the eastern boundary current is broader and slower, with minimum coastline interaction. \\

        This asymmetry is analogously reflected in the Southern hemisphere.

        \problem Explain the physics of geostrophic currents.
        \solution Geostrophic currents circle sites of convergence and divergence. They are formed by the balance of the Coriolis force
        and the pressure gradient force. The pressure gradient force is simply the tendency of water to flow down slopes due to gravity. In regions of
        convergence, the accumulated water causes slight changes in elevation of the ocean level (few metres per 100 or 100,000 km).
        Thus, this water tries to flow downwards, away from the zone of convergence. For example, the center of a gyre is a region of convergence,
        so the water-level in the center of the ocean is slightly higher. This elevated water is pulled away from the center by gravity and is deflected
        by the Coriolis force. When these two balance, the net movement of water is perpendicular to the water-level gradient (or the pressure
        gradient force). Hence, this flow is geostrophic. As a result, the net movement of water is approximately parallel to the slope,
        clockwise in the North and anticlockwise in the South.

        \problem Explain the Southern Oscillation.
        What happens to atmospheric and oceanic circulation in the tropical pacific during an ENSO event?
        How is the biological productivity affected during this (ENSO) event?
        \solution Southern Oscillations refers to the periodic oscillation in sea level pressures in the tropical Pacific.
        Sea Surface Temperature (SST) data shows that the western part of this ocean has the highest temperatures on the globe, hence
        experience intense atmospheric convection. We know that this rising air forms Hadley cells by moving polewards, but there is also a
        flow of air in the east-west direction, along the equator. The eastward component crosses the Pacific and subsides when it
        hits South America. There is an accompanying westwards flow along the surface across the Pacific, completing the cycle. This
        atmospheric cycle is called Walker circulation. Now, this surface westward wind causes a westward ocean current, and thus the
        accumulation of warm surface water in the western part of the Pacific. Thus, the western Pacific has a thicker layer
        of surface water than the eastern Pacific. The thinner surface water in the east means that it becomes a site of upwelling of
        colder water from underneath, which is rich in nutrients. This promotes biological productivity and a large fish population. \\

        The breakdown of this normal pattern is called an El Ni\~{n}o Southern Oscillation (ENSO). This can happen when the easterly winds are not
        as strong as usual. In this case, the accumulated warm surface water in the west flows back to the east in the from of a Kelvin wave.
        This takes around 60 days. This means that the regions of warmest ocean temperatures shift to the central Pacific. This in turn
        dirsupts the atmospheric circulation, since the major sites of convection now lie over the central Pacific, from where the rising air travels
        eastwards as well as westwards and meet over Africa. Usually, the region in the west Pacific (Australia, Indonesia) experiences
        low atmospheric pressure, and the regions in the central and east Pacific experience high pressure. This means convection and rainfall
        over Australia and Indonesia in the summer, along with Africa and the Amazon Basin. The west slope of the Andes remains dry.
        During an ENSO event, this pattern reverses, so rainfall over Australia and Indonesia decreases. Thus, there is drought in these
        areas along with America, Brazil, and Africa while the western slopes of the Andes (Ecuador, Peru) and the central Pacific experience
        an increase in rainfall. \\

        An ENSO event also shuts off upwelling in the east, causing great loss in biological productivity.
        Lots of marine organisms die due to the lack of nutrients, and so do associated predators such as birds which feed on them. \\


        \problem Explain the differences among the pycnocline, the halocline, and the thermocline.
        \solution The pycnocline, halocline and thermocline describe the boundary between the surface zone and the deep ocean, in terms of
        density, salinity and temperature gradients respectively. \\

        Ocean water is layered such that denser water is deeper. The density of water depends broadly on salinity (directly proportional) and
        temperature (maximum density at \SI{4}{\celsius}). The top 60-100 m of water is generally well mixed due to wind, and is hence
        called the surface zone. The density of water here is low. The transition zone between this layer and the deep ocean is about \SI{1}{\kilo\metre}
        deep, and exhibits a rapid increase in density with depth. This sharp increase in density is called the pycnocline. If this change
        is dominated by a rapid salinity increase, it is called the halocline; if instead it is dominated by a temperature decrease, it
        is called the thermocline.


        \problem Explain the processes that drive the circulation of the deep ocean.
        \solution Deep ocean circulation is driven by density gradients in water, caused by differences in salinity and temperature. Thus, it
        is called thermohaline circulation. Sea salt, which mainly consists of chloride, sodium, sulphate, magnesium, calcium and potassium,
        is introduced by weathering of crustal rocks. The salinity of an ocean can increase by the evaporation of ocean water, which leaves the
        salts behind. It can be removed by marine organisms, chemical reactions on the seafloor, or by sea spray. Other variations are 
        caused by precipitation, the formation and melting of sea ice, and river runoffs. These changes in salinity means that ocean water
        of different densities are layered by depth. The densest water lies at the bottom, making the vertical structure of ocean water quite stable.
        There is very little horizontal variation in density, but this is enough to drive currents.\\

        The process of thermohaline circulation begins with the formation of bottom water, which is dense (cold and salty). This is typically
        formed at high latitudes in the margins of sea ice, where the water is cold. The formation of sea ice floating on the surface means
        that salt is left behind in the ocean, as salt crystals do not fit in the ice. This cold, saline water sinks,
        slowly flowing downwards and towards the equator. This mixes with other bottom water and eventually reaches sites of upwelling.


        \problem Explain the bottom-water formation and its importance for driving deep ocean circulation.
        \solution Bottom water is generally formed in high latitudes, near the poles. Surface water is cooled below its freezing point
        (\SI{-1.9}{\celsius}), which is less than usual because of the presence of dissolved salts. This freezes and forms a layer of
        ice which floats on the surface. Now, salts do not fit well in the structure of ice crystals, so they are left behind and accumulate
        in the water beneath. Thus, a layer of saline, cold (hence very dense) water is formed just below the ice. This sinks and flows down
        the slope of the ocean basin and spreads towards the equator. This water is called bottom water. As it moves, its dissolved oxygen content
        decreases, while its dissolved carbon-dioxide content and nutrient levels increase. It mixes with other bottom water and
        rises to the surface at sites of upwelling. The sinking water at the poles is replaced by poleward flowing warm water from the equatorial
        regions. Thus, this completes a thermohaline cycle or conveyor belt. \\
        
        For example, Antarctic Bottom Water (AABW) form in the Weddell Sea and flows north into the ocean basins of the Pacific, Atlantic and
        Indian oceans. North Atlantic Deep Water (NADW) is formed in the Arctic Ocean off the coast of Greenland and flows into the western
        part of the North Atlantic Ocean, where it joins the AABW.


        \problem Explain the linking of thermohaline and wind-driven surface ocean circulation.
        What is meant by thermohaline conveyor belt?
        \solution The thermohaline conveyor belt is the circulation of water between the major ocean basins across the globe. This consists
        of both surface and deep water circulation, which are linked by upwelling and downwelling. This conveyor belt plays a major role in 
        the recycling of ocean nutrients and the distribution of heat energy, thus affecting our climate. \\

        On a broad scale, surface winds drive the circulation of surface ocean water, transport warm water from the equator towards the poles
        and create sites of upwelling and downwelling via convergence and divergence of water. Thermohaline circulation transports deep water
        from the poles towards the equator, which rises to the surface and completes the cycle. Broadly speaking, deep water
        from the Arctic flows into the North Atlantic, meets deep water from the Antarctic, then flows south and around Antarctica in the 
        Antarctic Circumpolar Current. This branches off into the Indian and Pacific oceans, where upwelling brings it to the surface.
        Surface currents bring this water back into the North Atlantic, completing the circulation.

\end{document}
