\documentclass[10pt]{article}

\usepackage[T1]{fontenc}
\usepackage{geometry}
\usepackage{amsmath, amssymb, amsthm}
\usepackage{array} 
\usepackage{enumitem}
\usepackage{siunitx}
\usepackage{chemformula}
% \usepackage{graphicx}
% \usepackage{caption}

\geometry{a4paper, margin=1in}
\setlength\parindent{0pt}
% \renewcommand\qedsymbol{$\blacksquare$}
\newcolumntype{L}{l@{\quad\quad}}
\newcounter{prob}
\def\problem{\stepcounter{prob}\paragraph{Problem \arabic{prob}}}
\def\solution{\\\\\textbf{Solution }}

\begin{document}
        \par\textbf{IISER Kolkata} \hfill \textbf{Assignment IV}
        \vspace{3pt}
        \hrule
        \vspace{3pt}
        \begin{center}
                \LARGE{\textbf{ES1201 : Earth System Processes}}
        \end{center}
        \vspace{3pt}
        \hrule
        \vspace{3pt}
        Satvik Saha, \texttt{19MS154}\hfill\today
        \vspace{20pt}
        
        \problem Comment on the uniform values of measured abundances of highly siderophilic elements in the mantle. 
        Explain the set of conditions/reasons for that.
        \solution The uniformity of highly siderophilic elements in the mantle, such as \ch{Os}, \ch{Ir}, \ch{Pt}, \ch{Au}, \ch{Pd}, etc.\ 
        can be explained by the \textit{late veneer model} -- essentially, all siderophilic elements were removed from the mantle
        during core formation, and the current levels have been reached by other processes, such as bombardment by meteorites. \\

        It was earlier thought that the equilibration of the iron-rich core and the silicate-rich mantle took place under low temperature and pressure
        (surface-like) conditions. Thus, the segregation of the core would strip out most of the siderophilic elements from the mantle, bringing
        their concentrations to between $10^{-5}$ and $10^{-6}$ of the chondritic abundance, with different elements having different abundances.
        However, this predicted value is too low; the actual abundance is around \SI{2e-3} relative to chondrites, and remains uniform across the highly
        siderophilic elements. Thus, we must abandon the low temperature and pressure assumption. Experimentally obtained partition coefficients
        can be used to test other starting conditions, but only data for \ch{Re} is available. Besides, siderophilic behaviour (related to the 
        partition coefficient) differs greatly among the different elements under low temperature and pressure conditions, so the uniformity of their
        abundance remains unexplained.\\

        Thus, alternate models such as an incomplete/inefficient partitioning of the core, or the partitioning of a sulphur rich liquid metal core
        and a silicate mantle were proposed. The late veneer model seems to be the most promising. It assumes that the highly siderophilic elements
        were more or less completely partitioned into the core, and then later added to the mantle via meteorites.
        These meteorites would have to largely have the same, chondritic proportions of siderophilic elements, thus explaining the
        uniformity of abundances in the mantle. \\

        Now, the partitioning of moderately siderophilic elements during core formation was likely not as efficient as that of highly siderophilic
        elements, so their concentrations would remain higher. Thus, the later addition of meteoritic material would not affect these values
        significantly, and so we may consider their present abundances to reflect those after core formation. This allows us to pinpoint
        the temperature and pressure conditions during partitioning. \ch{Ni} and \ch{Co} are two such moderately siderophilic elements whose
        relative abundances are known today, around $10^{-1}$ relative to chondrites. To reach this value, the ratio of their partition
        coefficients must be around $1.1$. Both these elements become less siderophilic with increase in temperature and pressure at different rates,
        so there is one common pressure (\SI{28}{\giga\pascal}) where we have $D_{\ch{Ni}} / D_{\ch{Co}} = 1.1$. This confirms that
        the metal-silicate equilibration and core segregation took place under high temperature and pressure conditions. \\

        The pressure of \SI{28}{\giga\pascal} indicates a depth of around \SIrange{900}{1000}{\km}. Thus, it was suggested that during the time
        of equilibration, the upper mantle formed a molten magma ocean and the lower mantle was solid. Droplets of metal would
        rain down the magma ocean, stripping out siderophiles under differing temperatures and pressures. These would then reach the boundary
        between the upper and lower mantle, where the molten metal would collect in pools. The equilibration between these pools and the upper
        mantle would result in the currently observed siderophilic compositions. With time, large globules (diapirs) would split off from the pool
        and sink down the lower mantle and into the (still forming) core.

        \clearpage
        \problem Explain the radioactive system of \ch{Hf}-\ch{W} used to understand the partition of the core and mantle.
        Predict the age of core formation. 
        \solution The \ch{Hf}-\ch{W} system is particularly useful in determining the time of partitioning of the core and mantle beacuse
        of two reasons -- they partition differently in metal-silicate differentiation, and the $\beta$-decay of \ch{^{182}Hf} to \ch{^{182}W}
        has a relatively short half-life of \SI{8.9} million years. Both these elements are refractory, and hence accreted to Earth in the same
        proportion as in chondritic meteorites. Thus, their proportions in the Bulk Silicate Earth (BSE) are well known. \\

        To quantify the deviation of \ch{^{182}W} present in a given sample, relative to the BSE, we use a metric called $\varepsilon\ch{^{182}W}$.
        The amount of \ch{^{182}W}, which is a daughter product of the parent \ch{^{182}Hf}, is compared to the \ch{^{184}W} isotope of tungsten.
        This is representative of the \ch{^{182}Hf}/\ch{^{182}W} originally present in the sample before decay. Since these ratios are of
        the order of magnitude of $10^{-4}$, we define
        \[
        \varepsilon\ch{^{182}W} \;=\; \left[\frac{\left(\displaystyle{\ch{^{182}W}}/{\ch{^{184}W}}\right)_\text{sample}}
                                                {\left(\displaystyle{\ch{^{182}W}}/{\ch{^{184}W}}\right)_\text{BSE}}
                                                \,-\, 1\right] \times 10^4.
        \]

        Now, hafnium is lithophilic and would thus partition into the silicate mantle, while tungsten is siderophilic and would thus partition
        into the metallic core during core formation. Thus, depending on the time of partitioning, we have two scenarios.
        \begin{enumerate}[label=(\roman*), itemsep=0pt, topsep=\parsep]
        \item \textit{Early core formation:} Here, the core partitions from the mantle before most of the hafnium has had time to decay.
                During fractionation, there is very little \ch{^{182}W} formed as a product of $\beta$-decay, and both isotopes of tungsten
                segregate in the same ratio.
                Therefore, the \ch{^{182}W}/\ch{^{184}W} ratio in the mantle and core both remain similar to the original (chondritic) value initially.
                However, most of the \ch{^{182}Hf} would remain in the mantle due to its lithophilic nature. Over time, this would
                decay into \ch{^{182}W}, raising the \ch{^{182}W}/\ch{^{184}W} ratio of the mantle (BSE) above both the core and the chondritic values.
                {\it (This ratio in the core would also increase because of the presence of some amount of \ch{^{182}Hf}, but not as quickly
                as in the mantle.)}

        \item \textit{Late core formation:} Here, the core partitions from the mantle after most of the hafnium decays into \ch{^{182}W}.
                Thus, the \ch{^{182}W}/\ch{^{184}W} ratio in the undifferentiated earth and the chondritic meteorites evolve in parallel
                and remain similar. During differentiation, both isotopes of tungsten partition into the core equally well, further
                preserving their ratio. Thus, the \ch{^{182}W}/\ch{^{184}W} ratio in the BSE would remain the same as that of chondritic meteorites.
        \end{enumerate}

        Measurements show that $\varepsilon\ch{^{182}W}$ of chondritic meteorites falls around $-2$. Thus, the \ch{^{182}W}/\ch{^{184}W} ratio
        in the BSE is indeed higher than in chondritic meteorites. This indicates an early core formation. As a very rough estimate,
        using only the fact that $\varepsilon\ch{^{182}W}$ is negative, we can say that the core formed within 10 half-lives, i.e.\ 
        \SI{90} million years of the start of the Solar system. More careful calculations yield a range of \SIrange{30}{50}{million} years
        after the start of the Solar system.

\end{document}
