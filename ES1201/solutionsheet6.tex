\documentclass[10pt]{article}

\usepackage[T1]{fontenc}
\usepackage{geometry}
\usepackage{amsmath, amssymb, amsthm}
\usepackage{array} 
\usepackage{enumitem}
\usepackage{siunitx}
\usepackage{chemformula}

\geometry{a4paper, margin=1in}
\setlength\parindent{0pt}
% \renewcommand\qedsymbol{$\blacksquare$}
\newcolumntype{L}{l@{\quad\quad}}
\newcounter{prob}
\def\problem{\stepcounter{prob}\paragraph{Problem \arabic{prob}}}
\def\solution{\\\\\textbf{Solution }}

\begin{document}
        \par\textbf{IISER Kolkata} \hfill \textbf{Assignment VI}
        \vspace{3pt}
        \hrule
        \vspace{3pt}
        \begin{center}
                \LARGE{\textbf{ES1201 : Earth System Processes}}
        \end{center}
        \vspace{3pt}
        \hrule
        \vspace{3pt}
        Satvik Saha, \texttt{19MS154}\hfill\today
        \vspace{20pt}

        \problem Explain briefly why the source of major fraction of Earth's water seems to be from meteorites but not from comets.
        \solution To make this conclusion, we make use of the measured deuterium to hydrogen (D/H) ratios in water.
        This ratio can be used as a `signature', and we have found that the bulk earth and early carbonaceous meteorites have
        similar values (around $150\times 10^{-6}$ in the bulk earth). This value in comets is much higher (around $310\times 10^{-6}$),
        thus making them an unlikely source of the majority of Earth's water.

        \problem It is believed that the early Earth atmosphere was not as reducing as the gas mixture of Miller-Urey Experiment. Explain.
        \solution The early atmosphere is thought to have oxidized over time with the escape of atomic hydrogen. Essentially,
        volcanos do not generally emit reducing gases like ammonia and methane in quantities required by the Miller-Urey experiment,
        and these gases would anyways be photolyzed by the solar UV radiation, leaving behind nitrogen and carbon-dioxide gases.
        The carbon from carbonaceous meteorites would indeed be released during impacts as carbon-monoxide and methane, but these too
        would be irreversably oxidized by hydroxyl radicals (released by photolysis of water by UV radiation) to form carbon-dioxide.
        Note that this is possible because of the absence of atmospheric oxygen and ozone. All of the atomic hydrogen released in
        these processes would escape into space. Thus, the early atmosphere could not have been as strongly reducing as suggested by
        Miller-Urey.

        \problem Explain nitrogen-fixation and briefly discuss the role of cyanobacteria in N-fixation and photosynthesis.
        \solution Nitrogen fixation is the process by which natural processes or lifeforms convert molecular nitrogen
        into fixed nitrogen -- nitrogeous compounds, such as ammonia and various nitrates. This is essential as most organisms can only use 
        fixed nitrogen to make organic molecules like proteins and nucleic acids. There are two major ways this happens today: lightning
        and marine organisms (mostly cyanobacteria). The immense heat supplied by ightning triggers the reaction of nearby nitrogen and oxygen, producing
        nitric oxide, which subsequently oxidizes to nitric acid in the atmosphere. This rains down and dissociates in water into hydrogen
        and nitrate ions.
        \[\ch{N_2 + O_2 -> 2 NO}\]
        \[\ch{HNO_3 <-> H^+  + NO_3^-}\]
        The early atmosphere did not have enough oxygen for the same mechanism. However, there is an alternate reaction with carbon-dioxide which
        achieves a similar effect.
        \[\ch{N_2 + 2 CO_2 -> 2 NO + 2 CO}\]
        Also, prokaryotes, not cyanobacteria, were likely the first organisms to supply fixed nitrogen.
        There is an opposing process of bacterial denitrification which maintains balance by releasing nitrogen back into the atmosphere.\\

        In addition to their contribution in nitrogen fixation, cyanobacteria were likely the first photosynthetic organisms. These
        organisms released molecular oxygen into the atmosphere, thus leading to the Great Oxidation Event.
        \[\ch{CO_2 + H_2O -> CH_2O + O_2}\]
        The fact that cyanobacteria are able to do both is remarkable since the enzyme \textit{nitrogenase}, which is used to reduce molecular nitrogen,
        is poisoned by oxygen. Thus, they developed solutions to safely deal with oxygen produced during photosynthesis. One way was to create
        separate compartments (heterocysts) devoted solely to nitrogen fixation. Another was to photosynthesize only during the day and
        fix nitrogen at night. Cyanobacteria are also closely related to chloroplasts in plants today, which are thought to have come from
        an endosymbiotic event in the past. A prokaryotic cell must have been engulfed by another cell and entered a symbiotic relationship,
        supplying oxygen to its host via photosynthesis in exchange for nutrients.

        \problem Briefly explain the role of greenhouse gases in the early Earth atmosphere when the luminosity of Sun was 30\% less than that
        of today (Faint Young Sun).
        \solution The Faint Young Sun paradox raises the following question: how did the early earth remain warm enough to support liquid water,
        which we know must have been present as it's essential for life? Solutions include a lower planetary albedo during that time (which is
        unlikely as this requires a near zero albedo) and geothermal heat (which is not sufficient, and would require orders of magnitude more
        heat flux than the current value). The most likely solution involves a greater greenhouse effect in those times. Increased
        levels of greenhouse gases would be trap enough solar energy to keep temperatures above the freezing point of water. We know that ammonia was 
        abundant, but would be rapidly destroyed by UV radiation. Thus, it has been suggested that \ch{CO_2} and \ch{CH_4} were the major
        contributers. The carbonate-silicate cycle relies on silicate weathering on land to remove atmospheric \ch{CO_2} --
        since continents were smaller, this means that \ch{CO_2} levels would have been higher. In addition, the impact of planetesimals
        as well as volcanic eruptions would have released plenty of \ch{CO_2}. Similarly, \ch{CH_4} could have been released by impacts
        or by serpentinization of the seafloor. This is helped by methanogenic bacteria which converted atmospheric hydrogen and
        carbon-dioxide into methane. \\

        Note that the carbonate-silicate cycle has a strong negatove feedback loop which stabilizes global temperatures: high \ch{CO_2} levels
        and high temperatures increase the rate of silicate weathering, thus balancing volcanic outgassing. Similarly,
        low \ch{CO_2} levels and low temperatures would decrease the rate of silicate weathering, allowing \ch{CO_2} from volcanos to 
        accumulate until it's balanced by the loss due to weathering.
        Methane also contributes to an anti-greenhouse effect by forming an organic haze which absorbs visible light and reflects
        IR radiation away from the Earth.

        \problem Discuss briefly the carbonate-silicate cycle for balancing \ch{CO_2} in the atmosphere over long timescales.
        \solution The carbonate-silicate cycle describes the inorganic processes by which carbon-dioxide exits and enters the atmosphere over
        geologic time. Carbon-dioxide is removed from the atmosphere via rainwater, in the form of carbonic acid. This acid is very weak,
        but it's enough to dissolve carbonate and silicate rocks into their constituent ions, which are carried off into the oceans
        by rivers. For example, consider calcium silicate.
        \[
        \ch{CaSiO_3 + H_2O + 2 CO_2 -> Ca^{2+} + SiO2 + 2 HCO_3^-}
        \]
        Marine organisms use these materials to build their shells and skeletons. This is called carbonate precipitation.
        \[
        \ch{Ca^{2+} + 2 HCO_3^- -> CaCO_3 + CO_2 + H_2O}
        \]
        Thus, there is a net removal of \ch{CO_2} (two molecules used, one released). When these marine organisms die, their remains
        fall to the ocean floor, which thus acts as a storehouse for carbon-dioxide. \\

        With time, the action of plate tectonics carries old seafloor to subduction zones, where all the carbonate is forced into the mantle.
        Under high pressure and temperature conditions deep within the mantle, this reacts with silica and releases carbon-dioxide.
        \[
        \ch{CaCO_3 + SiO2 -> CaSiO_3 + CO_2}
        \]
        This carbon-dioxide eventually finds its way out into the atmosphere via volcanism, thermal vents, or springs, thus completing the cycle. \\

        The carbonate-silicate cycle is connected with a strong negative feedback loop, as explained in the previous answer. The rate at which
        carbon-dioxide is bound is proportional to the available land surface. Higher temperatures also increase the rate of silicate weathering.
        For example, if \ch{CO_2} levels rise, the greenhouse effect causes global temperatures to rise, increasing precipitation and silicate weathering.
        This increases the rate of removal of atmospheric \ch{CO_2}, thus restoring balance.
        Similarly, lower \ch{CO_2} levels decrease global temperatures, decrease precipitation and silicate weathering and thus
        decrease the rate of atmospheric \ch{CO_2} removal, allowing volcanism to catch up and restore \ch{CO_2} levels.
        Thus this stabilizing effect has earned the carbonate-silicate cycle the name `Earth's thermostat'.


\end{document}
