\documentclass[11pt]{article}

\usepackage[T1]{fontenc}
\usepackage{geometry}
\usepackage{bm}
\usepackage{amsmath, amssymb, amsthm}
\usepackage{hyperref}

\geometry{a4paper, margin=1in, headheight=14pt}

\def\C{\mathbb{C}}
\def\R{\mathbb{R}}
\def\Q{\mathbb{Q}}
\def\Z{\mathbb{Z}}
\def\N{\mathbb{N}}

\renewcommand\vec\boldsymbol
\def\vx{\vec{x}}
\def\vy{\vec{y}}
\def\vz{\vec{z}}
\def\ve{\vec{e}}

\theoremstyle{remark}
\newtheorem*{remark}{Remark}
\newtheorem*{example}{Example}
\newtheorem*{solution}{Solution}

\title{
    \Large\textsc{Summer Programme 2021} \\
    \vspace{10pt}
    \huge Equivalence of metric spaces \\
}
\author{
    \large Satvik Saha%
    % \thanks{Email: \tt ss19ms154@iiserkol.ac.in}
    \\\textsc{\small 19MS154}
}
\date{\normalsize
    \textit{Indian Institute of Science Education and Research, Kolkata, \\
    Mohanpur, West Bengal, 741246, India.} \\
    % \vspace{10pt}
    % \today
}


\begin{document}
    \maketitle
    
    We claim that the following metrics on $\R^n$ are equivalent, in that they
    induce the same topology on $\R^n$.
    \begin{enumerate}
        \item $d_1(\vx, \vy) = |\vx - \vy|$, the Euclidean metric.
        \item $d_2(\vx, \vy) = \max_i\{|x_i - y_i|\}$, the Chebyshev metric.
        \item $d_3(\vx, \vy) = |\vx - \vy| / (1 + |\vx - \vy|)$.
    \end{enumerate}
    Label the metric spaces $M_i = (\R^n, d_i)$, and their respective collections of
    open sets $\tau_i$. Denote $B^i_r(\vx)$ to be the open ball in $M_i$, centred at
    $\vx$ with radius $r$, i.e.\ the collection of points $\vy \in M_i$ such that
    $d_i(\vx, \vy) < r$.

    \begin{itemize}
    \item[($\tau_1 \subseteq \tau_2$)] 
    Consider an open ball $B^1_r(\vx) \subseteq M_1$. Any point $\vy$ in this ball
    satisfies $d_1(\vy, \vx) = |\vy - \vx| < r$.
    Let $d_1(\vy, \vx) = r - \epsilon$ for some $\epsilon > 0$, and set $\epsilon' =
    \epsilon / \sqrt{n}$. For all $\vz$ in the open ball $B^2_{\epsilon'}(\vy)$,
    i.e.\ such that $d_2(\vz, \vy) < \epsilon'$, we have \[
        d_1(\vz, \vy)^2 = \sum_i (z_i - y_i)^2 \leq n\max\{|z_i - y_i|\}^2 =
        nd_2(\vz, \vy)^2 < \epsilon^2.
    \] The triangle inequality gives \[
        d_1(\vz, \vx) \leq d_1(\vz, \vy) + d_1(\vy, \vx) < \epsilon + r - \epsilon = r,
    \] hence $\vz \in B^1_r(\vx)$.

    Thus, the open ball $B_y := B^2_{\epsilon'}(\vy) \subseteq M_2$ is contained
    within the open ball $B := B^1_r(\vx) \subseteq M_1$.  Take the union of $B_y$
    for all $\vy \in B$, and note that this is precisely equal to $B$. This is
    because any element of $B$ is the center of some $B_y$, and every element in the
    union is contained within some $B_y$, which in turn is contained within $B$.
    Hence, any open ball in $M_1$ is open in $M_2$. Since every open set in a metric
    space can be written as the union of open balls, we see that every open set in
    $M_1$ is an open set in $M_2$.

    \item[($\tau_2 \subseteq \tau_1$)]
    Consider an open ball $B^2_r(\vx) \subseteq M_2$. Any point $\vy$ in this ball
    satisfies $d_2(\vy, \vx) = \max_i\{|y_i - x_i|\} < r$. Let $d_2(\vy, \vx) = r -
    \epsilon$ for some $\epsilon > 0$. For all $\vz$ in the open ball
    $B^1_{\epsilon}(\vy)$, we have \[
        d_1(\vz, \vy)^2 = \sum_i (z_i - y_i)^2 < \epsilon ^2,
    \] hence $|z_i - y_i| < \epsilon$ for all $i$.  Specifically, \[
        d_2(\vz, \vy) = \max_i\{|z_i - y_i|\} < \epsilon.
    \] The triangle inequality further gives \[
        d_2(\vz, \vx) \leq d_2(\vz, \vy) + d_2(\vy, \vx) < \epsilon + r - \epsilon =
        r.
    \]

    Thus, the open ball $B_y := B^1_\epsilon(\vy) \subseteq M_1$ is contained within
    the open ball $B := B^2_r(\vx)$. Like before, the union of all such $B_y$ for
    $\vy \in B$ yields precisely $B$, so any open ball in $M_2$ is an open set in
    $M_1$. It follows that any open set in $M_2$ is an open set in $M_1$.
    
    \item[$(\tau_1 = \tau_3)$] Note that for any open ball $B^1_r(\vx) \subseteq
    M_1$, any point $\vy$ in this ball satisfies $d_1(\vy, \vx) < r$, which is
    equivalent to $d_3(\vy, \vx) = d_1(\vy, \vx) / (1 + d_1(\vy, \vx)) < r / (1 + r)
    := r'$. Thus, $B^1_r(\vx) = B^3_{r'}(\vx)$.

    Similarly, for any open ball $B^3_r(\vx) \subseteq M_3$, if $r \geq 1$ then
    $B^3_r(\vx) = \R^n$ (every point $\vy \in \R^n$ satisfies $d_3(\vy,
    \vx) < 1$ because $s / (1 + s) < 1$ for all non-negative reals $s$). Otherwise,
    we can find $r' \geq 0$ such that $r' / (1 + r') = r$, specifically choose $r' =
    r / (1 - r)$. Again, this gives $B^3_r(\vx) = B^1_{r'}(\vx)$.

    Thus, every open ball in $M_1$ is an open ball in $M_3$, and vice versa. It
    follows that a set is open in $M_1$ if and only if it is open in $M_3$.

    Note that we have not used any specific property of $d_1$ here, merely its
    relation with $d_3$. This means that more generally, the open sets in the metric
    space $(\R^n, d)$ are identical to those in the metric space $(\R^n, d')$, where
    $d'(x, y) = d(x, y) / (1 + d(x, y))$.  \\~\\

    We have used that fact that for all real numbers, \[
        0 \leq x < y \;\Longleftrightarrow\; 0 \leq \frac{x}{1 + x} \leq \frac{y}{1
        + y}.
    \] This is equivalent to stating that the function $f\colon [0, \infty) \to [0,
    1)$ defined by $x \mapsto x / (1 + x)$ is strictly increasing,
    which is evident by \[
        f(x) = \frac{x}{1 + x} = \frac{x + 1 - 1}{1 + x} = 1 - \frac{1}{1 + x}.
    \] The map $x \mapsto 1 / (1 + x)$ is strictly decreasing, hence $x \mapsto 1 -
    1 / (1 + x)$ is strictly increasing.
    \end{itemize}
    
    Together, we have $\tau_1 = \tau_2 = \tau_3$, which means that all three metrics
    induce the same topology on $\R^n$.
\end{document}
