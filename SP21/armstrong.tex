\documentclass[11pt]{report}

\usepackage[T1]{fontenc}
\usepackage{geometry}
\usepackage{amsmath, amssymb, amsthm}
\usepackage[scr]{rsfso}
\usepackage{xcolor}
\usepackage{fancyhdr}
\usepackage{hyperref}

\geometry{a4paper, margin=1in, headheight=14pt}

\pagestyle{fancy}
\renewcommand\headrulewidth{0.4pt}
\fancyhead[L]{\it Groups and Symmetry}
\rfoot{\footnotesize\it Updated on \today}
\cfoot{\thepage}
\renewcommand{\chaptermark}[1]{\markboth{#1}{}}

\renewcommand{\labelenumi}{(\alph{enumi})}
\renewcommand{\labelenumii}{(\roman{enumii})}

\def\C{\mathbb{C}}
\def\R{\mathbb{R}}
\def\Q{\mathbb{Q}}
\def\Z{\mathbb{Z}}
\def\N{\mathbb{N}}

\theoremstyle{remark}
\newtheorem*{remark}{Remark}
\newtheorem*{example}{Example}
\newtheorem*{solution}{Solution}

\title{
    \Large\textsc{Summer Programme 2021} \\
    \vspace{10pt}
    \huge Solutions to exercises from M.A.~Armstrong's \\
    \textit{Groups and Symmetry} 
}
\author{
    \large Satvik Saha%
    % \thanks{Email: \tt ss19ms154@iiserkol.ac.in}
    \\\textsc{\small 19MS154}
}
\date{\normalsize
    \textit{Indian Institute of Science Education and Research, Kolkata, \\
    Mohanpur, West Bengal, 741246, India.} \\
    % \vspace{10pt}
    % \today
}

\begin{document}
    \maketitle

    \chapter{Symmetries of the Tetrahedron}

    \paragraph{Exercise 1.1} Glue two copies of a regular tetrahedron together so
    that they have a triangular face in common, and work out all the rotational
    symmetries of this new solid.
    \begin{solution}
        The resultant bi-pyramid has five vertices; label the ones furthest apart as
        1 and 2, and then label the remaining ones on the equator as 3, 4, 5. The
        rotational symmetries are those which permute these five vertices such that
        1 and 2 never leave the long axis, and the cyclicity of the vertices 1, 2,
        3, 4 is preserved. Thus, we have 3 rotations about the long axis (by $0$,
        $2\pi / 3$, and $4\pi / 3$) which cycle the vertices 3, 4, 5, then three
        rotations, each about an axis through one of the vertices 3, 4, 5 and the
        midpoint of the opposite edge (by $0$, $\pi$) which swap the positions of 1
        and 2 and reverse the cyclicity of 3, 4, 5. This gives a total of $2\times 3
        = 6$ symmetries.
    \end{solution}

    \paragraph{Exercise 1.2} Find all rotational symmetries of a cube.
    \begin{solution}
        First consider the four rotations (by $0$, $\pi / 2$, $\pi$, $3\pi / 2$)
        about an axis which passes through the centres of opposite faces; there are
        three such axes giving $3\times 3 = 9$ such rotational symmetries (excluding
        the identity symmetry, which we will add on at the end).
        Next, consider the two rotations (by $0$, $\pi$) about an axis passing
        through the centres of opposite edges; there are six such axes, giving
        $1\times 6 = 6$ such rotational symmetries.
        Next, consider the three rotations (by $0$, $2\pi / 3$, $4\pi / 3$) about an
        axis passing through opposite vertices; there are four such axes, giving
        $2\times 4 = 8$ such rotational symmetries. Adding these up, we have $9 + 6
        + 8 = 23$ rotational symmetries. The identity symmetry brings the total to
        $24$ rotational symmetries of the cube.
    \end{solution}

    \paragraph{Exercise 1.3} Adopt the notation of Figure~1.4. Show that the axis of
    the composite rotation $srs$ passes through the vertex $4$, and that the axis of
    $rsrr$ is determined by the midpoints of edges $12$ and $34$.
    \begin{solution}
        Let the original state of the tetrahedron be represented by the tuple $(1,
        2, 3, 4)$, indicating the labels on the four vertices. Applying $s$ permutes
        this to $(4, 3, 2, 1)$, and applying $r$ permutes (the original) to $(1, 4,
        2, 3)$. Thus, the action $srs$ is the permutation $(1, 2, 3, 4) \to (4, 3,
        2, 1) \to (4, 1, 3, 2) \to (2, 3, 1, 4)$. Note that the vertex labelled $4$
        is a fixed point, hence the axis of this composite rotation must have passed
        through this vertex.

        Similarly, the action $rsrr$ maps $(1, 2, 3, 4) \to (1, 4, 2, 3) \to (1, 3,
        4, 2) \to (2, 4, 3, 1) \to (2, 1, 4, 3)$. There are no fixed points, hence
        this is not a rotation through a vertex. Instead, the first pair and last
        pair of vertices have swapped, which indicates a rotation about an axis
        through the centres of the edges $12$ and $34$.
    \end{solution}

    \paragraph{Exercise 1.4} Having completed the previous exercise, express each of
    the twelve rotational symmetries of the tetrahedron in terms of $r$ and $s$.
    \begin{solution}
        The twelve rotational symmetries of the tetrahedron are $e$ (the identity),
        $r$, $r^2$, $s$, $rs$, $r^2s$, $srs$, $rsrs$, $r^2srs$, $sr^2s$, $rsr^2s$,
        $r^2sr^2s$. See Exercise~1.7 for their actions on the $(1, 2, 3, 4)$ state.
    \end{solution}

    \paragraph{Exercise 1.5} Again with the notation of Figure~1.4, check that
    $r^{-1} = rr$, $s^{-1} = s$, $(rs)^{-1} = srr$ and $(sr)^{-1} = rrs$.
    \begin{solution}
        Note that $r^3$ maps $(1, 2, 3, 4) \to (1, 4, 2, 3) \to (1, 3, 4, 2) \to (1,
        2, 3, 4)$, so $r^3 = e$. Thus, $(rr)r = e = r(rr)$, so $r^{-1} = rr$.

        Next, note that $ss$ maps $(1, 2, 3, 4) \to (4, 3, 2, 1) \to (1, 2, 3, 4)$,
        so $ss = e$. Thus, $(s)s = e = e(s)$, so $s^{-1} = s$.

        Next, note that $(rs)(srr) = r(ss)(rr) = r(e)(rr) = rrr = e$, and $(srr)(rs)
        = s(rrr)s = s(e)s = ss = e$, so $(rs)^{-1} = srr$.

        Finally, note that $(sr)(rrs) = (srr)(rs) = e$ and $(rrs)(sr) = (rr)(ss)r =
        rrr = e$, so $(sr)^{-1} = rrs$.
    \end{solution}

    \paragraph{Exercise 1.6} Show that a regular tetrahedron has a total of
    twenty-four symmetries if reflections and products of reflections are allowed.
    Identify a symmetry which is not a rotation and not a reflection. Check that
    this symmetry is a product of three reflections.
    \begin{solution}
        Note that a reflection about the plane passing through an edge and the
        centroid swaps the remaining two vertices. Thus, by representing the vertex
        configuration of the tetrahedron as a tuple $(1, 2, 3, 4)$, this can be
        mapped to any of the $4! = 24$ permutations by employing suitable
        reflections (for example, see bubble sort).

        Consider the action which takes the tetrahedron $(1, 2, 3, 4)$ to the state
        $(4, 1, 2, 3)$.  This is not a rotation about a vertex, nor a reflection
        about a plane through an edge because there are no fixed points.
        This is not a rotation about an axis through the centres of opposite sides
        either, since those must swap the labels on two pairs of adjacent vertices.
        However, this can be reached via the reflections $(1, 2, 3, 4) \to (4, 2, 3,
        1) \to (4, 1, 3, 2) \to (4, 1, 2, 3)$; these were reflections about planes
        passing through the edges $23$, $13$, $12$.
    \end{solution}

    \paragraph{Exercise 1.7} Let $q$ denote reflection of a regular tetrahedron in
    the plane determined by its centroid and one of its edges. Show that the
    rotational symmetries, together with those of the form $uq$, where $u$ is a
    rotation, give all twenty-four symmetries of the tetrahedron.
    \begin{solution}
        We let $q$ mean the reflection in the plane through the centroid and the
        side $12$; note that this maps $(1, 2, 3, 4) \to (1, 2, 4, 3)$.
        Below, we list all $24$ permutations of the tuple $(1, 2, 3, 4)$, which
        represent all $24$ symmetries of the tetrahedron.
        \begin{center}
        \begin{tabular}{r|c||r|c}
            Symmetry    & State         & Symmetry      & State         \\\hline
            $e$         & 1, 2, 3, 4    & $q$           & 1, 2, 4, 3 \\
            $r$         & 1, 4, 2, 3    & $rq$          & 1, 3, 2, 4 \\
            $r^2$       & 1, 3, 4, 2    & $r^2q$        & 1, 4, 3, 2 \\
            $s$         & 4, 3, 2, 1    & $sq$          & 3, 4, 2, 1 \\
            $rs$        & 4, 1, 3, 2    & $rsq$         & 3, 1, 4, 2 \\
            $r^2s$      & 4, 2, 1, 3    & $r^2sq$       & 3, 2, 1, 4 \\
            $srs$       & 2, 3, 1, 4    & $srsq$        & 2, 4, 1, 3 \\
            $rsrs$      & 2, 4, 3, 1    & $rsrsq$       & 2, 3, 4, 1 \\
            $r^2srs$    & 2, 1, 4, 3    & $r^2srsq$     & 2, 1, 3, 4 \\
            $sr^2s$     & 3, 1, 2, 4    & $sr^2sq$      & 4, 1, 2, 3 \\
            $rsr^2s$    & 3, 4, 1, 2    & $rsr^2sq$     & 4, 3, 1, 2 \\
            $r^2sr^2s$  & 3, 2, 4, 1    & $r^2sr^2sq$   & 4, 2, 3, 1
        \end{tabular}
        \end{center}
    \end{solution}

    \paragraph{Exercise 1.8} Find all plane symmetries (rotations and reflections)
    of a regular pentagon and of a regular hexagon.
    \begin{solution}
        Any symmetry of a regular $n$-gon is one which preserves adjacent vertices,
        i.e.\ two labelled vertices must remain adjacent before and after the
        symmetry action. Thus, we have $n$ rotations (by $2k\pi / n$), along with
        $n$ rotations followed by a reflection (these are mirror images of the
        previous $n$ symmetries). Thus, a regular $n$-gon has $2n$ plane symmetries;
        ten for a pentagon and twelve for a hexagon.
    \end{solution}

    \paragraph{Exercise 1.9} Show that the hexagonal plate of Figure~1.2 has
    twenty-four symmetries in all. Identify those symmetries which commute with all
    the others.
    \begin{solution}
        By representing the vertices of the hexagon as the tuple $(1, 2, 3, 4, 5,
        6)$, we see symmetries of the plate are precisely those actions which
        permute these elements, preserving the adjacency but allowing a reversal in
        order. There are six tuples of the form $(1 + n, 2 + n, \dots, 6 + n)$ and
        six more of the form $(6 + n, 5 + n, \dots, 1 + n)$.
        Because we are dealing with a hexagonal plate, there is another symmetry
        which is a reflection passing through a plane parallel to the hexagonal
        face. This does not change the labels on the vertices on the hexagonal face
        in any way, but exchanges the top and bottom faces. Thus, this symmetry
        commutes with all other symmetries, and the combination of any one of the
        twelve previously shown symmetries with this reflection symmetry produces a
        new symmetry, bringing our total to twenty-four.
    \end{solution}

    \paragraph{Exercise 1.10} Make models of the octahedron, dodecahedron, and
    icosahedron. Try to spot as many symmetries of these solids as you can.
    
    
\end{document}
% vim: set tabstop=4 shiftwidth=4 softtabstop=4:
