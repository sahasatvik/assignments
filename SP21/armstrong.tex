\documentclass[11pt]{report}

\usepackage[T1]{fontenc}
\usepackage{geometry}
\usepackage{amsmath, amssymb, amsthm}
\usepackage{bm}
\usepackage[scr]{rsfso}
\usepackage{xcolor}
\usepackage{fancyhdr}
\usepackage{hyperref}

\geometry{a4paper, margin=1in, headheight=14pt}

\pagestyle{fancy}
\renewcommand\headrulewidth{0.4pt}
\fancyhead[L]{\it Groups and Symmetry}
\rfoot{\footnotesize\it Updated on \today}
\cfoot{\thepage}
\renewcommand{\chaptermark}[1]{\markboth{#1}{}}

% \renewcommand{\labelenumi}{(\alph{enumi})}
\renewcommand{\labelenumi}{(\roman{enumi})}

\def\C{\mathbb{C}}
\def\R{\mathbb{R}}
\def\Q{\mathbb{Q}}
\def\Z{\mathbb{Z}}
\def\N{\mathbb{N}}

\renewcommand\vec\boldsymbol
\def\vx{\vec{x}}
\def\vy{\vec{y}}
\def\vz{\vec{z}}
\def\ve{\vec{e}}

\newcommand\norm[1]{\left\lVert#1\right\rVert}

\theoremstyle{remark}
\newtheorem*{remark}{Remark}
\newtheorem*{example}{Example}
\newtheorem*{solution}{Solution}

\title{
    \Large\textsc{Summer Programme 2021} \\
    \vspace{10pt}
    \huge Solutions to exercises from M.A.~Armstrong's \\
    \textit{Groups and Symmetry} 
}
\author{
    \large Satvik Saha%
    % \thanks{Email: \tt ss19ms154@iiserkol.ac.in}
    \\\textsc{\small 19MS154}
}
\date{\normalsize
    \textit{Indian Institute of Science Education and Research, Kolkata, \\
    Mohanpur, West Bengal, 741246, India.} \\
    % \vspace{10pt}
    % \today
}

\begin{document}
    \maketitle

    \chapter{Symmetries of the Tetrahedron}

    \paragraph{Exercise 1.1} Glue two copies of a regular tetrahedron together so
    that they have a triangular face in common, and work out all the rotational
    symmetries of this new solid.
    \begin{solution}
        The resultant bi-pyramid has five vertices; label the ones furthest apart as
        1 and 2, and then label the remaining ones on the equator as 3, 4, 5. The
        rotational symmetries are those which permute these five vertices such that
        1 and 2 never leave the long axis, and the cyclicity of the vertices 1, 2,
        3, 4 is preserved. Thus, we have 3 rotations about the long axis (by $0$,
        $2\pi / 3$, and $4\pi / 3$) which cycle the vertices 3, 4, 5, then three
        rotations, each about an axis through one of the vertices 3, 4, 5 and the
        midpoint of the opposite edge (by $0$, $\pi$) which swap the positions of 1
        and 2 and reverse the cyclicity of 3, 4, 5. This gives a total of $2\times 3
        = 6$ symmetries.
    \end{solution}

    \paragraph{Exercise 1.2} Find all rotational symmetries of a cube.
    \begin{solution}
        First consider the four rotations (by $0$, $\pi / 2$, $\pi$, $3\pi / 2$)
        about an axis which passes through the centres of opposite faces; there are
        three such axes giving $3\times 3 = 9$ such rotational symmetries (excluding
        the identity symmetry, which we will add on at the end).
        Next, consider the two rotations (by $0$, $\pi$) about an axis passing
        through the centres of opposite edges; there are six such axes, giving
        $1\times 6 = 6$ such rotational symmetries.
        Next, consider the three rotations (by $0$, $2\pi / 3$, $4\pi / 3$) about an
        axis passing through opposite vertices; there are four such axes, giving
        $2\times 4 = 8$ such rotational symmetries. Adding these up, we have $9 + 6
        + 8 = 23$ rotational symmetries. The identity symmetry brings the total to
        $24$ rotational symmetries of the cube.
    \end{solution}

    \paragraph{Exercise 1.3} Adopt the notation of Figure~1.4. Show that the axis of
    the composite rotation $srs$ passes through the vertex $4$, and that the axis of
    $rsrr$ is determined by the midpoints of edges $12$ and $34$.
    \begin{solution}
        Let the original state of the tetrahedron be represented by the tuple $(1,
        2, 3, 4)$, indicating the labels on the four vertices. Applying $s$ permutes
        this to $(4, 3, 2, 1)$, and applying $r$ permutes (the original) to $(1, 4,
        2, 3)$. Thus, the action $srs$ is the permutation $(1, 2, 3, 4) \to (4, 3,
        2, 1) \to (4, 1, 3, 2) \to (2, 3, 1, 4)$. Note that the vertex labelled $4$
        is a fixed point, hence the axis of this composite rotation must have passed
        through this vertex.

        Similarly, the action $rsrr$ maps $(1, 2, 3, 4) \to (1, 4, 2, 3) \to (1, 3,
        4, 2) \to (2, 4, 3, 1) \to (2, 1, 4, 3)$. There are no fixed points, hence
        this is not a rotation through a vertex. Instead, the first pair and last
        pair of vertices have swapped, which indicates a rotation about an axis
        through the centres of the edges $12$ and $34$.
    \end{solution}

    \paragraph{Exercise 1.4} Having completed the previous exercise, express each of
    the twelve rotational symmetries of the tetrahedron in terms of $r$ and $s$.
    \begin{solution}
        The twelve rotational symmetries of the tetrahedron are $e$ (the identity),
        $r$, $r^2$, $s$, $rs$, $r^2s$, $srs$, $rsrs$, $r^2srs$, $sr^2s$, $rsr^2s$,
        $r^2sr^2s$. See Exercise~1.7 for their actions on the $(1, 2, 3, 4)$ state.
    \end{solution}

    \paragraph{Exercise 1.5} Again with the notation of Figure~1.4, check that
    $r^{-1} = rr$, $s^{-1} = s$, $(rs)^{-1} = srr$ and $(sr)^{-1} = rrs$.
    \begin{solution}
        Note that $r^3$ maps $(1, 2, 3, 4) \to (1, 4, 2, 3) \to (1, 3, 4, 2) \to (1,
        2, 3, 4)$, so $r^3 = e$. Thus, $(rr)r = e = r(rr)$, so $r^{-1} = rr$.

        Next, note that $ss$ maps $(1, 2, 3, 4) \to (4, 3, 2, 1) \to (1, 2, 3, 4)$,
        so $ss = e$. Thus, $(s)s = e = e(s)$, so $s^{-1} = s$.

        Next, note that $(rs)(srr) = r(ss)(rr) = r(e)(rr) = rrr = e$, and $(srr)(rs)
        = s(rrr)s = s(e)s = ss = e$, so $(rs)^{-1} = srr$.

        Finally, note that $(sr)(rrs) = (srr)(rs) = e$ and $(rrs)(sr) = (rr)(ss)r =
        rrr = e$, so $(sr)^{-1} = rrs$.
    \end{solution}

    \paragraph{Exercise 1.6} Show that a regular tetrahedron has a total of
    twenty-four symmetries if reflections and products of reflections are allowed.
    Identify a symmetry which is not a rotation and not a reflection. Check that
    this symmetry is a product of three reflections.
    \begin{solution}
        Note that a reflection about the plane passing through an edge and the
        centroid swaps the remaining two vertices. Thus, by representing the vertex
        configuration of the tetrahedron as a tuple $(1, 2, 3, 4)$, this can be
        mapped to any of the $4! = 24$ permutations by employing suitable
        reflections (for example, see bubble sort).

        Consider the action which takes the tetrahedron $(1, 2, 3, 4)$ to the state
        $(4, 1, 2, 3)$.  This is not a rotation about a vertex, nor a reflection
        about a plane through an edge because there are no fixed points.
        This is not a rotation about an axis through the centres of opposite sides
        either, since those must swap the labels on two pairs of adjacent vertices.
        However, this can be reached via the reflections $(1, 2, 3, 4) \to (4, 2, 3,
        1) \to (4, 1, 3, 2) \to (4, 1, 2, 3)$; these were reflections about planes
        passing through the edges $23$, $13$, $12$.
    \end{solution}

    \paragraph{Exercise 1.7} Let $q$ denote reflection of a regular tetrahedron in
    the plane determined by its centroid and one of its edges. Show that the
    rotational symmetries, together with those of the form $uq$, where $u$ is a
    rotation, give all twenty-four symmetries of the tetrahedron.
    \begin{solution}
        We let $q$ mean the reflection in the plane through the centroid and the
        side $12$; note that this maps $(1, 2, 3, 4) \to (1, 2, 4, 3)$.
        Below, we list all $24$ permutations of the tuple $(1, 2, 3, 4)$, which
        represent all $24$ symmetries of the tetrahedron.
        \begin{center}
        \begin{tabular}{r|c||r|c}
            Symmetry    & State         & Symmetry      & State         \\\hline
            $e$         & 1, 2, 3, 4    & $q$           & 1, 2, 4, 3 \\
            $r$         & 1, 4, 2, 3    & $rq$          & 1, 3, 2, 4 \\
            $r^2$       & 1, 3, 4, 2    & $r^2q$        & 1, 4, 3, 2 \\
            $s$         & 4, 3, 2, 1    & $sq$          & 3, 4, 2, 1 \\
            $rs$        & 4, 1, 3, 2    & $rsq$         & 3, 1, 4, 2 \\
            $r^2s$      & 4, 2, 1, 3    & $r^2sq$       & 3, 2, 1, 4 \\
            $srs$       & 2, 3, 1, 4    & $srsq$        & 2, 4, 1, 3 \\
            $rsrs$      & 2, 4, 3, 1    & $rsrsq$       & 2, 3, 4, 1 \\
            $r^2srs$    & 2, 1, 4, 3    & $r^2srsq$     & 2, 1, 3, 4 \\
            $sr^2s$     & 3, 1, 2, 4    & $sr^2sq$      & 4, 1, 2, 3 \\
            $rsr^2s$    & 3, 4, 1, 2    & $rsr^2sq$     & 4, 3, 1, 2 \\
            $r^2sr^2s$  & 3, 2, 4, 1    & $r^2sr^2sq$   & 4, 2, 3, 1
        \end{tabular}
        \end{center}
    \end{solution}

    \paragraph{Exercise 1.8} Find all plane symmetries (rotations and reflections)
    of a regular pentagon and of a regular hexagon.
    \begin{solution}
        Any symmetry of a regular $n$-gon is one which preserves adjacent vertices,
        i.e.\ two labelled vertices must remain adjacent before and after the
        symmetry action. Thus, we have $n$ rotations (by $2k\pi / n$), along with
        $n$ rotations followed by a reflection (these are mirror images of the
        previous $n$ symmetries). Thus, a regular $n$-gon has $2n$ plane symmetries;
        ten for a pentagon and twelve for a hexagon.
    \end{solution}

    \paragraph{Exercise 1.9} Show that the hexagonal plate of Figure~1.2 has
    twenty-four symmetries in all. Identify those symmetries which commute with all
    the others.
    \begin{solution}
        By representing the vertices of the hexagon as the tuple $(1, 2, 3, 4, 5,
        6)$, we see symmetries of the plate are precisely those actions which
        permute these elements, preserving the adjacency but allowing a reversal in
        order. There are six tuples of the form $(1 + n, 2 + n, \dots, 6 + n)$ and
        six more of the form $(6 + n, 5 + n, \dots, 1 + n)$.
        Because we are dealing with a hexagonal plate, there is another symmetry
        which is a reflection passing through a plane parallel to the hexagonal
        face. This does not change the labels on the vertices on the hexagonal face
        in any way, but exchanges the top and bottom faces. Thus, this symmetry
        commutes with all other symmetries, and the combination of any one of the
        twelve previously shown symmetries with this reflection symmetry produces a
        new symmetry, bringing our total to twenty-four.
    \end{solution}

    \paragraph{Exercise 1.10} Make models of the octahedron, dodecahedron, and
    icosahedron. Try to spot as many symmetries of these solids as you can.
    
    

    \chapter{Axioms}

    \paragraph{Exercise 2.1} Compare the symmetry of a snow crystal with that of the
    hexagonal plate in Figure~1.2.

    \paragraph{Exercise 2.2} Show that the set of positive real numbers form a group
    under multiplication.
    \begin{solution}
        First note that multiplication of real numbers is associative, which means
        that $(xy)z = x(yz)$ for all $x, y, z \in \R^+$. Next, $1x = x = x1$ for all
        $x \in \R^+$, so $1 \in \R^+$ serves as our identity element. Finally, every
        positive real number $x$ has a multiplicative inverse $1 / x \in \R^+$,
        which satisfies $x(1 / x) = 1 = (1 / x) x$.
    \end{solution}

    \paragraph{Exercise 2.3} Which of the following collections of $2\times 2$
    matrices with real entries form groups under matrix multiplication?
    \begin{enumerate}
        \itemsep0em
        \item Those of the form $\begin{bmatrix}
            a & b \\ b & c
        \end{bmatrix}$ for which $ac \neq b^2$.
        \item Those of the form $\begin{bmatrix}
            a & b \\ c & a
        \end{bmatrix}$ for which $a^2 \neq bc$.
        \item Those of the form $\begin{bmatrix}
            a & b \\ 0 & c
        \end{bmatrix}$ for which $ac \neq 0$.
        \item Those which have non-zero determinant and whose entries are integers.
    \end{enumerate}
    \begin{solution} \mbox{}
        \begin{enumerate}
            \item Matrices of this form are not closed under multiplication.
            Consider \[
                \begin{bmatrix}
                    1 & 1 \\ 1 & 0
                \end{bmatrix}\begin{bmatrix}
                    1 & 1 \\ 1 & 2
                \end{bmatrix} = \begin{bmatrix}
                    2 & 3 \\ 1 & 1
                \end{bmatrix}.
            \] 
            \item Matrices of this form are not closed under multiplication.
            Consider \[
                \begin{bmatrix}
                    1 & 1 \\ 2 & 1
                \end{bmatrix}\begin{bmatrix}
                    1 & 1 \\ 3 & 1
                \end{bmatrix} = \begin{bmatrix}
                    4 & 2 \\ 5 & 3
                \end{bmatrix}.
            \] 
            \item First, note that multiplication of such matrices is closed, \[
                \begin{bmatrix}
                    a_1 & b_1 \\ 0 & c_1
                \end{bmatrix}\begin{bmatrix}
                    a_2 & b_2 \\ 0 & c_2
                \end{bmatrix} = \begin{bmatrix}
                    a_1a_2  & a_1b_2 + b_1c_2 \\ 0 & c_1c_2
                \end{bmatrix}.
            \] Next, the identity matrix $\mathbb{I}_2$ satisfies the role of the
            multiplicative identity, with $\mathbb{I}_2M = \mathbb{I}_2 =
            M\mathbb{I}_2$ for all such matrices $M$. Finally, we have \[
                \begin{bmatrix}
                    1 / a & -b / ac \\ 0 & 1 / c
                \end{bmatrix} \begin{bmatrix}
                    a & b \\ 0 & c
                \end{bmatrix} = \begin{bmatrix}
                    1 & 0 \\ 0 & 1    
                \end{bmatrix} = \begin{bmatrix}
                    a & b \\ 0 & c
                \end{bmatrix} \begin{bmatrix}
                    1 / a & -b / ac \\ 0 & 1 / c
                \end{bmatrix},
            \] so all such matrices have a multiplicative inverse. This means that
            this collection of matrices forms a group under matrix multiplication.

            \item Note that all the collection of all matrices must have
            $\mathbb{I}_2$ as the identity element. However, not all matrices have a
            multiplicative inverse in this collection. For example, \[
                \begin{bmatrix}
                    2 & 0 \\ 0 & 2
                \end{bmatrix} \begin{bmatrix}
                    1 / 2 & 0 \\ 0 & 1 / 2
                \end{bmatrix} = \begin{bmatrix}
                    1 & 0 \\ 0 & 1
                \end{bmatrix}.
            \]
        \end{enumerate}
    \end{solution}

    \paragraph{Exercise 2.4} Let $f$ be a similarity of the plane. Show that $f$ is a
    bijection and that the inverse function $f^{-1}$ is also a similarity. Verify
    that the collection of all similarities of the plane forms a group under
    composition of functions.
    \begin{solution}
        Any similarity $f\colon \R^2 \to \R^2$ of the plane can be written in the
        form \[
            f(\vx) = \lambda A\vx + \vx_0,
        \] where $A \in O_2(\R)$ is an orthogonal matrix, $\vx_0 \in \R^2$ is a
        translation vector and $\lambda \in \R\setminus\{0\}$ is a scaling factor.
        The fact that all such $\lambda A$ are invertible guarantees that $f$ is a
        bijection. This immediately gives \[
            f^{-1}(\vx) = \frac{1}{\lambda}A^{-1}\vx - \frac{1}{\lambda}A^{-1}\vx_0,
        \] which is also a similarity of the plane. \\

        A similarity of the plane must send lines to lines, in such a way that
        all lengths scale in the same proportion. This means that \[
            f(\vx + \mu(\vy - \vx)) = f(\vx) + \mu (f(\vy) - f(\vx))
        \] for all scalars $\mu$. Set $f(\vec{0}) = \vx_0$, and set $g = f - \vx_0$.
        Then we have $g(\vec{0}) = \vec{0}$ and \[
            g(\vx + \mu(\vy - \vx)) = g(\vx) + \mu(g(\vy) - g(\vx)).
        \] Setting $\vx = \vec{0}$, next $\mu = 1 / 2$ gives \[
            g(\mu\vy) = \mu g(\vy), \qquad g(\vx + \vy) = g(\vx + \vy).
        \] Thus, $g$ is a linear transformation, and is thus of the form \[
            g(\vx) = B\vx
        \] for some $2\times 2$ matrix $B$. Note that if $B$ were not of full rank,
        then its null space must contain some non-zero vector $\vec{v}$ such that
        $B\vec{v} = \vec{0}$. This cannot be allowed for a similarity of the plane,
        since such a transformation would map the entire line along $\vec{v}$ onto
        the single point at the origin. Thus, $B$ must be invertible making $g$ a
        linear bijection, and this is sufficient to show that $f(\vx) = B\vx +
        \vx_0$ is a bijection.

        To show that $B = \lambda A$ for some $A \in O_2(\R)$, note that we require
        $f$ and $g$ to scale all line segments equally. Thus, $\norm{g(\vx)} =
        \norm{g(\vy)}$ whenever $\norm{\vx} = \norm{\vy}$. Set $\lambda =
        \norm{g(\hat{\vec{v}})}$ for all unit vectors $\hat{\vec{v}}$, and noting
        that $\lambda \neq 0$, set $A = B / \lambda$. Thus, $\norm{g(\vec{v})} =
        \norm{g(v\hat{\vec{v}})} = v \norm{g(\hat{\vec{v}})} =
        \norm{\vec{v}}\lambda$ for all vectors $\vec{v}$. Hence, $\norm{A\vec{v}} =
        \norm{\vec{v}}$ for all vectors $\vec{v}$, so the transformation represented
        by $A$ is an isometry. This can be shown to force $A \in O_2(\R)$.
    \end{solution}

    \paragraph{Exercise 2.5} A function from the plane to itself which preserves the
    distance between any two points is called an \textit{isometry}. Prove that an
    isometry must be a bijection and check that the collection of all isometries of
    the plane forms a group under composition of functions.
    \begin{solution}
        Again, an isometry $f\colon \R^2 \to \R^2$ of the plane must be of the form
        \[
            f(\vx) = A\vx + \vx_0,
        \] where $A \in O_2(\R)$ is an orthogonal matrix and $\vx_0 \in \R^2$ is a
        translation vector. The invertibility of $A$ guarantees that $f$ is a
        bijection.
    \end{solution}
    
    \paragraph{Exercise 2.6} Show that the collection of all rotations of the plane
    about a fixed point $P$ forms a group under composition of functions. Is the
    same true of the set of all reflections in lines which pass through $P$? What
    happens if we take all the rotations and all the reflections?

    \paragraph{Exercise 2.7} Let $x$ and $y$ be elements of a group $G$. Prove that
    $G$ contains $w$, $z$ which satisfy $wx = y$ and $xz = y$, and show that these
    elements are unique.
    \begin{solution}
        Note that $x$ has an inverse $x^{-1} \in G$. Set $w = yx^{-1}$ and $z =
        x^{-1}y$, whence \[
            wx = (yx^{-1})x = y(x^{-1}x) = ye = y, \qquad
            xz = x(x^{-1}y) = (xx^{-1})y = ey = y.
        \] Next, suppose that $w_1x = w_2x = y$. Right multiplying by $x^{-1}$ gives
        $w_1 = w_2 = yx^{-1}$. Similarly, if $xz_1 = xz_2 = y$, left multiplying my
        $x^{-1}$ gives $z_1 = z_2 = x^{-1}y$.
    \end{solution}

    \paragraph{Exercise 2.8} If $x$ and $y$ are elements of a group, prove that
    $(xy)^{-1} = y^{-1}x^{-1}$.
    \begin{solution}
        Using the associativity of multiplication, \[
            (xy)(y^{-1}x^{-1}) = x(yy^{-1})x = xex^{-1} = x x^{-1} = e,
        \]\[
            (y^{-1}x^{-1})(xy) = y^{-1}(x^{-1}x)y = y^{-1}ey = y^{-1}y = e.
        \] 
    \end{solution}


    \chapter{Numbers}

    \paragraph{Exercise 3.1} Show that each of the following collections of numbers
    forms a group under addition.
    \begin{enumerate}
        \itemsep0em
        \item The even integers.
        \item All real numbers of the form $a + b\sqrt{2}$ where $a, b \in \Z$.
        \item All real numbers of the form $a + b\sqrt{2}$ where $a, b \in \Q$.
        \item All complex numbers of the form $a + bi$ where $a, b \in \Z$.
    \end{enumerate}

    \paragraph{Exercise 3.2} Write $\Q(\sqrt{2})$ for the set described in
    Exercise~3.1~(iii). Given a non-zero element $a + b\sqrt{2}$, express $1 / (a +
    b\sqrt{2})$ in the form $c + d\sqrt{2}$ where $c, d \in \Q$. Prove that
    multiplication makes $\Q(\sqrt{2}) - \{0\}$ into a group.
    \begin{solution}
        For all non-zero elements $a + b\sqrt{2}$, write \[
            \frac{1}{a + b\sqrt{2}} = \frac{a - b\sqrt{2}}{(a + b\sqrt{2})(a -
            b\sqrt{2})} \,=\, \frac{a}{a^2 - 2b^2} + \frac{-b}{a^2 - 2b^2}\sqrt{2}.
        \] Note that $a^2 \neq 2b^2$ for any $a, b \in \Q$ other than $a = b = 0$,
        because of the irrationality of $\sqrt{2}$.

        Note that multiplication of real numbers is associative, and that $1 = 1 +
        0\sqrt{2}$ serves as an identity element since $1(a + b\sqrt{2}) = a +
        b\sqrt{2} = (a + b\sqrt{2})1$. Furthermore, every non-zero element $a + b\sqrt{2}$
        has a multiplicative inverse as indicated above. This proves that
        $\Q(\sqrt{2}) - \{0\}$ is a group under multiplication.
    \end{solution}
    
    \paragraph{Exercise 3.3} Let $n$ be a positive integer and let $G$ consist of
    all those complex numbers $z$ which satisfy $z^n = 1$. Show that $G$ forms a
    group under multiplication of complex numbers.
    \begin{solution}
        Note that multiplication of complex numbers is associative, and that $1$
        serves as an identity element. Multiplication is closed since if $x^n = y^n
        = 1$, $(xy)^n = 1$. Given a complex number $z \in G$, we have its
        multiplicative inverse $z^{n - 1}$ since $zz^{n - 1} = 1 = z^{n - 1}z$.
        Thus, $G$ is a group under multiplication of complex numbers.
    \end{solution}

    \paragraph{Exercise 3.4} Vary $n$ in the previous exercise and check that the
    union of all these groups \[
        \bigcup_{n = 1}^\infty \{z \in \C : z^n = 1\}
    \] is also a group under multiplication of complex numbers.
    \begin{solution}
        Let $G$ be the set described above. Note that multiplication of complex
        numbers is associative, and that $1$ serves as the identity element.
        Furthermore, if $z \in G$, then $z^n = 1$ for some $n \in \N$, which means
        that $z^{n - 1}$ is its multiplicative inverse. Thus, $G$ is a group under
        multiplication of complex numbers.
    \end{solution}

    \paragraph{Exercise 3.5} Let $n$ be a positive integer. Prove that \[
        (x ._{n} y) ._{n} z = x ._{n} (y ._{n} z)
    \] for all $x, y, z \in \Z$.
    \begin{solution}
        Using Euclid's Division Lemma, find integers $0\leq a, b, c \leq n - 1$ such
        that \[
            x = pn + a, \qquad y = qn + b, \qquad z = rn + c
        \] for integers $p, q, r$. This gives \[
            xy = (pn + a)(qn + b) = pqn + pbn + qan + ab = (pq + pb + qa)n + ab,
        \] so $x ._{n} y = xy \pmod{n} = ab$. Similarly, $y._{n}z = xy \pmod{n} =
        bc$. Thus, \[
            (x ._{n} y) ._{n} z = ab ._{n} z = ab (rn + c) \pmod{n} = (abr)n + abc
            \pmod{n} = abc,
        \] and \[
            x ._{n} (y ._{n} z) = x ._{n} bc = (pn + a)bc \pmod{n} = (pbc)n + abc
            \pmod{n} = abc.
        \] 
    \end{solution}
    
    \paragraph{Exercise 3.6} Verify that each of the sets \begin{align*}
        &\{1, 3, 7, 9, 11, 13, 17, 19\} \\
        &\{1, 3, 7, 9\} \\
        &\{1, 9, 13, 17\}
    \end{align*} forms a group under multiplication modulo $20$.
    \begin{solution}
        Verifying that a set forms a group comes down to furnishing a complete
        multiplication table, and ensuring that every row and column contains
        every element exactly once.
        \begin{enumerate}
            \item \mbox{}
            \begin{center}
                \begin{tabular}{r|rrrrrrrr}
                    ._{20}  &  1 &  3 &  7 &  9 & 11 & 13 & 17 & 19 \\\hline
                    1       &  1 &  3 &  7 &  9 & 11 & 13 & 17 & 19 \\
                    3       &  3 &  9 &  1 &  7 & 13 & 19 & 11 & 17 \\
                    7       &  7 &  1 &  9 &  3 & 17 & 11 & 19 & 13 \\
                    9       &  9 &  7 &  3 &  1 & 19 & 17 & 13 & 11 \\
                    11      & 11 & 13 & 17 & 19 &  1 &  3 &  7 &  9 \\
                    13      & 13 & 19 & 11 & 17 &  3 &  9 &  1 &  7 \\
                    17      & 17 & 11 & 19 & 13 &  7 &  1 &  9 &  3 \\
                    19      & 19 & 17 & 13 & 11 &  9 &  7 &  3 &  1
                \end{tabular}
            \end{center}
            \item Note that $3^{-1} = 7$, $7^{-1} = 3$, $9^{-1} = 9$.
            \item Note that $9^{-1} = 9$, $13^{-1} = 17$, $17^{-1} = 13$.
        \end{enumerate}
    \end{solution}
    
    \paragraph{Exercise 3.7} Which of the following sets form groups under
    multiplication modulo $14$? \begin{align*}
        &\{1, 3, 5\} && \{1, 3, 5, 7\} \\
        &\{1, 7, 13\} && \{1, 9, 11, 13\}
    \end{align*}
    \begin{solution}
        Note that associativity of multiplication modulo $14$ holds in all cases,
        and $1$ serves as the identity element. The set $\{1, 3, 5\}$ forms a group
        since $3^{-1} = 5$, $5^{-1} = 3$. Similarly, the set $\{1, 9, 11, 13\}$
        forms a group since $9^{-1} = 11$, $11^{-1} = 9$, $13^{-1} = 13$.
        Neither set containing $7$ forms a group, since $7$ has no inverse modulo
        $14$; if $7x \pmod{14} = 1$, then $7x = 14n + 1$ for some integer $n$, which
        is impossible since $7x$ is a positive multiple of $7$ but $14n + 1$ is not.
    \end{solution}
    
    \paragraph{Exercise 3.8} Show that if a subset of $\{1, 2, \dots, 21\}$ contains
    an even number, or contains the number $11$, then it cannot form a group under
    multiplication modulo $22$.
    \begin{solution}
        Suppose that $n = 2m \in G \subseteq \{1, 2, \dots, 21\}$ is even. For $G$ to
        be a group, it must contain the multiplicative inverse of $n$, i.e.\ some
        element $x \in G$ such that $nx \pmod{22} = 1$. This means that $nx = 22k +
        1$ for some integer $k$, which is impossible since $nx = 2mx$ is even but
        $22k + 1 = 2(11k) + 1$ is odd.

        Similarly, suppose that $11 \in G$. Again, $G$ must contain a multiplicative inverse of
        $11$, i.e.\ some $x \in G$ such that $11x \pmod{22} = 1$. This means that
        $11x = 22k + 1$ for some integer $k$, which is impossible since $11x$ is a
        multiple of $11$ but $22k + 1 = 11(2k) + 1$ is not.
    \end{solution}

    \paragraph{Exercise 3.9} Let $p$ be a prime number and let $x$ be an integer
    which satisfies $1 \leq x \leq p - 1$. Show that none of $x$, $2x$, \dots, $(p -
    1)x$ is a multiple of $p$. Deduce the existence of an integer $z$ such that $1
    \leq z \leq (p - 1)$ and $xz \pmod{p} = 1$.
    \begin{solution}
        Let $1 \leq x \leq p - 1$ and let $1 \leq k \leq p - 1$ such that $kx$ is a
        multiple of $p$, i.e.\ $kx = mp$ for some positive integer $m$. Since $p
        \,|\, mp$, we must have $p \,|\, kx$. However, neither $x$ nor $k$ can
        contain $p$ as a prime factor, since they are strictly less than $p$. This
        means that the prime factorisation of $kx$ contains no factors of $p$; no
        factors of $p$ can be introduced by the multiplication of factors from $k$
        and $x$ since $p$ is prime. This is a contradiction, which means that none
        of $x$, $2x$, \dots, $(p - 1)x$ is a multiple of $p$.

        Consider the elements $1$, $x$, $2x$, \dots, $(p - 1)x$. We have $p$ of
        them, but there are $p - 1$ possible remainders modulo $p$, which means that
        two of these elements share the same remainder modulo $p$ by the pigeon-hole
        principle. If we have $mx = nx \pmod{p}$ for $m > n$ and $1 \leq m, n \leq p
        - 1$, then $(m - n)x = 0 \pmod{p}$. This is not possible since $1 \leq m - n
        \leq p - 1$ so $(m - n)$ cannot be a multiple of $p$. The only remaining
        possibility is that $zx = 1 \pmod{p}$ for some $1 \leq z \leq p - 1$.
    \end{solution}
    
    \paragraph{Exercise 3.10} Use the results of Exercises~3.5 and 3.9 to verify
    that multiplication modulo $n$ makes $\{1, 2, \dots, n - 1\}$ into a group if
    $n$ is prime. What goes wrong when $n$ is not a prime number?
    \begin{solution}
        The result of Exercise~3.5 guarantees associativity of multiplication modulo
        $n$, and it is clear that $1$ serves as an identity element. The result of
        Exercise~3.9 guarantees that every element $x \in \{1, 2, \dots, n - 1\}$
        has a corresponding multiplicative inverse $x^{-1}$, with $x x^{-1} = 1
        \pmod{n}$ when $n$ is prime. Thus, this set forms a group.

        When $n$ is not a prime number, certain elements fail to have multiplicative
        inverses as seen in Exercises~3.7 and 3.8. Specifically, the factors of $n$
        not equal to $1$ or $n$ do not have multiplicative inverses. This is clear
        since if $n = mk$ for $m, k \neq 1$, and $mx = 1 \pmod{n}$ for some $1 \leq
        x \leq n - 1$, then $mx = n\ell + 1 = mk\ell + 1$ for some integer $\ell$,
        which is impossible since $mx$ is a multiple of $m$ but $mk\ell + 1 =
        m(k\ell) + 1$ is not.
    \end{solution}
    

    \chapter{Dihedral Groups}

    \paragraph{Exercise 4.1} Work out the multiplication table of the dihedral group
    $D_4$. How many elements of order $2$ are there in $D_n$?
    \begin{solution} \mbox{}
        Note that in $D_n$, we have \[
            r^n = e, \qquad s^2 = e, \qquad sr = r^{-1}s.
        \] Thus, $(r^ks)^2 = r^ksr^ks = s^kr^{-k}ss = e$ for all $k = 0, \dots, n -
        1$. If $n$ is even, we also have $(r^{n / 2})^2 = e$. This gives $n$
        elements of order $2$ for odd $n$, and $n + 1$ elements of order $2$ for
        even $n$.
    \end{solution}

    \paragraph{Exercise 4.2} Find the order of each element of $\Z_5$, $\Z_9$, and
    $\Z_{12}$.
    \begin{solution} \mbox{}
    \begin{enumerate}
        \item $\Z_5$: Note that \[
            1\cdot 5 \equiv 0, \qquad
            2\cdot 5 \equiv 0, \qquad
            3\cdot 5 \equiv 0, \qquad
            4\cdot 5 \equiv 0.
        \] 
        \item $\Z_9$: Note that \[
            1\cdot 9 \equiv 0, \qquad
            2\cdot 9 \equiv 0, \qquad
            3\cdot 3 \equiv 0, \qquad
            4\cdot 9 \equiv 0, \] \[
            5\cdot 9 \equiv 0, \qquad
            6\cdot 3 \equiv 0, \qquad
            7\cdot 9 \equiv 0, \qquad
            8\cdot 9 \equiv 0.
        \] 
    \end{enumerate}
    In general, in a cyclic group of order $n$ generated by an element $x$, the
    order of some $x^r$ is $n / \gcd(n, r)$.
    \end{solution}

    \paragraph{Exercise 4.3} Check that the integers $1, 2, 4, 7, 8, 11, 13, 14$
    form a group under multiplication modulo $15$. Work out its multiplication table
    and find the order of each element.
    \begin{solution} \mbox{}
        \begin{center}
        \begin{tabular}{c|cccccccc}
            \cdot_{15} & 1 & 2 & 4 & 7 & 8 & 11 & 13 & 14 \\\hline
            1  & 1  & 2  & 4  & 7  & 8  & 11 & 13 & 14 \\
            2  & 2  & 4  & 8  & 14 & 1  & 7  & 11 & 13 \\
            4  & 4  & 8  & 1  & 13 & 2  & 14 & 7  & 11 \\
            7  & 7  & 14 & 13 & 4  & 11 & 2  & 1  & 8  \\
            8  & 8  & 1  & 2  & 11 & 4  & 13 & 14 & 7  \\
            11 & 11 & 7  & 14 & 2  & 13 & 1  & 8  & 4  \\
            13 & 13 & 11 & 7  & 1  & 14 & 8  & 4  & 2  \\
            14 & 14 & 13 & 11 & 8  & 7  & 4  & 2  & 1  \\
        \end{tabular}
        \end{center}
    \end{solution}


    \paragraph{Exercise 4.4} Let $g$ be an element of a group $G$. Keep $g$ fixed
    and let $x$ vary through $G$. Prove that the products $gx$ are all distinct and
    fill out $G$. Do the same for the products $xg$.
    \begin{solution}
        Suppose that $gx_1 = gx_2$. Multiplying by $g^{-1}$ gives $x_1 = x_2$, hence
        the products $gx$ are distinct for different $x$. For any element $y \in G$,
        note that setting $x = g^{-1}y$ ensures that $y = gx$, hence the products
        $gx$ fill out the entirety of $G$.

        Similarly, $x_1g = x_2g$ implies $x_1 = x_2$, and for any $y \in G$, set $x
        = yg^{-1}$ so that $y = xg$.
    \end{solution}
    
    
    \paragraph{Exercise 4.5} An element $x$ of a group satisfies $x^2 = e$ precisely
    when $x = x^{-1}$. Use this observation to show that a group of even order must
    contain an odd number of elements of order $2$.
    \begin{solution}
        Any element such that $x^2 = e$ must be of order at most $2$. The identity
        element $e$ is the only element of order $1$, which means that the rest are
        of order $2$.

        Suppose that the group contains an even number $2k$ of elements of order $2$.
        This means that we have $2k + 1$ elements such that $x^2 = e$, i.e.\ $x =
        x^{-1}$. For each of the remaining elements, $x \neq x^{-1}$, which means
        that they can be paired up exactly. Thus, there were an even number $2\ell$
        of elements remaining, so the group has an odd order $2(k + \ell) + 1$. This
        is a contradiction, which means that there must have been an odd number of
        elements of order $2$.
    \end{solution}


    \paragraph{Exercise 4.7} Let $G$ be the collection of all rational numbers which
    satisfy $0 \leq x < 1$. Show that the operation \[
        x + y = \begin{cases}
            x + y, &\text{ if } 0 \leq x + y < 1, \\
            x + y - 1, &\text{ if } x + y \geq 1.
        \end{cases}
    \] makes $G$ into an infinite abelian group all of whose elements have finite
    order.
    \begin{solution}
        First, the group operation is associative and commutative, due to the
        associativity and commutativity of addition of rational numbers. This
        operation is also closed, since $0 \leq x, y < 1$ ensures that $0 \leq x + y
        < 1$ (here, $+$ denotes the group operation). The identity element is $0$,
        since $0 + x = x$ for all $x$. Lastly, each element $x \in G$ has an inverse
        $1 - x \in G$, because $x + (1 - x) = 1 - 1 = 0$.

        Each $x \in G$ has finite order; let $x = p / q$ for positive integers $p,
        q$. Note that $p < q$. If $p = 0$, we are done since $x = 0$. Otherwise,
        consider $qx$, which denotes $x + \dots + x$ with $q$ terms added. This
        evaluates to $qx - k$, where the integer $k$ is chosen such that $0 \leq qx
        - k < 1$. However, $qx = p$ is an integer, so we must have $qx - k = 0$.
        Thus, every $x \in G$ has finite order.
    \end{solution}
    
    
    \chapter{Subgroups and Generators}

    \paragraph{Exercise 5.1} Find all the subgroups of each of the groups $\Z_4$,
    $\Z_7$, $\Z_{12}$, $D_4$, and $D_5$.
    \begin{solution} \mbox{}
    \begin{enumerate}
        \item $\Z_4$ has the following subgroups. \[
            \{0\}, \qquad \{0, 2\}, \qquad \{0, 1, 2, 3\}.
        \] 
        \item $\Z_7$ is a cyclic group of prime order, which means that its only
        subgroups are $\{0\}$ and $\Z_7$ itself.
        \item $\Z_{12}$ has the following subgroups. \[
            \{0\}, \qquad \{0, 6\}, \qquad \{0, 4, 8\}, \qquad \{0, 3, 6, 9\},
            \qquad \{0, 2, 4, 6, 8, 10\}, \qquad \Z_{12}.
        \] 

        \item Write the elements of $D_4$ as $e, r, r^2, r^3, s, rs, r^2s, r^3s$.
        The following are subgroups. \[
            \{e\}, \qquad \{e, r^2\}, \qquad \{e, r, r^2, r^3\},
        \] \[
            \{e, s\}, \qquad \{e, rs\}, \qquad \{e, r^2s\}, \qquad \{e, r^3s\}, 
        \] \[
            \{e, r^2, s, r^2s\}, \qquad \{e, r^2, rs, r^3s\}, \qquad D_4.
        \] 
        \item Write the elements of $D_5$ as $e, r, r^2, r^3, r^4, s, rs, r^2s,
        r^3s, r^4s$. The following are subgroups. \[
            \{e\}, \qquad \{e, r, r^2, r^3, r^4\},
        \] \[
            \{e, s\}, \qquad \{e, rs\}, \qquad \{e, r^2s\}, \qquad \{e, r^3s\},
            \qquad \{e, r^4s\},
        \] \[
            D_5.
        \] 
    \end{enumerate}
    \end{solution}

    \paragraph{Exercise 5.2} If $m$ and $n$ are positive integers, and if $m$ is a
    factor of $n$, show that $\Z_n$ contains a subgroup of order $m$. Does $\Z_n$
    contain more than one subgroup of order $m$? 

    
    \paragraph{Exercise 5.3} Check that $rs$ and $r^2s$ together generate $D_n$.
    \begin{solution}
        Note that the elements of $D_n$ are all of the form $e$, $r^k$ or $r^ks$,
        where $k = 1, \dots, n - 1$. We also have $r^n = e$, $s^2 = e$, and $rs =
        sr^{-1}$. Thus, $D_n$ is generated by $r$ and $s$.

        Now, $(r^2s)(rs) = r^2ssr^{-1} = r^2r^{-1} = r$, so $r \in \langle rs,
        r^2s\rangle$. This also implies that $r^{-1} \in \langle rs, r^2s\rangle$.
        Multiplying $r^{-1}(rs) = s$, we retrieve $r, s \in \langle rs,
        r^2s\rangle$. These two elements are sufficient to generate $D_n$.
    \end{solution}

    \paragraph{Exercise 5.4} Find the subgroup of $D_n$ generated by $r^2$ and
    $r^2s$, distinguishing carefully between the cases $n$ odd and $n$ even.
    \begin{solution}
        When $n$ is odd, say $n = 2m + 1$, we have $\langle r^2, r^2s\rangle = D_n$.
        This is because $e = r^{n} = r^{2m + 1} = (r^2)^m r$. Now, $r^2$ is in the
        generated subgroup, hence so is $(r^2)^{m + 1} = r^{2m}r^2 = r$. So is
        $r^{-2}$, hence $r^{-2}(r^2s) = s$. The elements $r, s$ now suffice to
        generate $D_n$ entirely.

        When $n$ is even, say $n = 2m$, we see that \[
            \langle r^2, r^2s\rangle = \{e, r^2, \dots, r^{2m - 2}, s, r^2s, \dots,
            r^{2m - 2}s\} < D_n.
        \] Since $r^2$ is in the generated group, so is $r^{-2}$, hence
        $r^{-2}(r^2)s = s$ is also present. Successive powers of $r^2$ give the
        elements $r^2, r^4, \dots, r^{2m - 2}$; the next power is $r^{2m} = e$.
        Multiplying each of these with $s$ give the elements $s, r^2s, \dots, r^{2m
        - 2}s$. We claim that there are no more elements. This is because all
        further products are of the form $(r^{2p})(r^{2q}) = r^{2(p + q)}$,
        $(r^{2p})(r^{2q}s) = r^{2(p + q)}s$, $(r^{2p}s)(r^{2q}) = r^{2(p - q)}s$,
        and $(r^{2p}s)(r^{2q}s) = r^{2(p - q)}$. Specifically, all the powers are
        even, and when simplified to be in the range $0, \dots, n$ by considering
        them modulo $n = 2m$, they remain even. Thus, the set of elements we have
        exhibited is closed under multiplication, and hence is the subgroup
        generated by $\langle r^2, r^2s\rangle$.
    \end{solution}

    \paragraph{Exercise 5.5} Suppose $H$ is a \textit{finite} non-empty subset of
    a group $G$. Prove that $H$ is a subgroup of $G$ if and only if $xy$ belongs to
    $H$ whenever $x$ and $y$ belong to $H$.
    \begin{solution}
        First suppose that $xy \in H$ whenever $x \in H$ and $y \in H$. Since $H$ is
        non-empty, pick $x \in H$. If $x = e$ is the identity element of $G$, and
        there are no more elements, we are done. Otherwise, suppose that $x \neq e$.
        We see that $x x = x^2 \in H$, then $x^3 \in H$, and so on, with $x^n \in H$
        for every $n \in \N$. All of these cannot be distinct, since $H$ is finite,
        which means that $x^p = x^q$ for some distinct natural numbers $p > q$. This
        gives $x^{p - q} = e$, which means that we have found $x^{p - q} = e \in H$.
        Closure of multiplication is a given. Finally, note that using the above
        procedure, we have found $x^{p - q} = e$ for any non-identity element $x$,
        hence $x^{p - q - 1} = x^{-1} \in H$. This means that every element in $H$
        has an inverse in $H$, making $H$ a subgroup of $G$.

        Next, suppose that $H$ is a subgroup of $G$. The fact that $xy \in H$
        whenever $x \in H$ and $y \in H$ is a consequence of the closure of
        multiplication in the group.
    \end{solution}
    
    \paragraph{Exercise 5.6} Draw a diagonal in a regular hexagon. List those plane
    symmetries of the hexagon which leave the diagonal fixed, and those which send
    the diagonal to itself. Show that both collections of symmetries are subgroups
    of the group of all plane symmetries of the hexagon.
    \begin{solution}
        The plane symmetries which leave the diagonal fixed are the identity
        symmetry ($e$), and the reflection of the plane about the diagonal ($s$).

        The plane symmetries which send the diagonal to itself include the
        aforementioned two, plus a rotation by $\pi$ ($r^3$), and a reflection of
        the plane about the perpendicular bisector of the diagonal ($r^3s$).

        The group of all plane symmetries of the hexagon can be represented by
        $D_6$. Clearly, $\{e, s\}$ forms a subgroup of $D_6$, as does $\{e, r^3, s,
        r^3s\}$, the latter of which is isomorphic to $D_2$.
    \end{solution}

    \paragraph{Exercise 5.7} Let $G$ be an abelian group, and let $H$ consist of
    those elements of $G$ which have finite order. Prove that $H$ is a subgroup of
    $G$.

    \paragraph{Exercise 5.8} Which elements of the infinite dihedral group have
    finite order? Do these elements form a subgroup of $D_\infty$?
    \begin{solution}
        Only the reflections, of the form $t^ks$ for $k \in Z$, have finite order,
        namely $2$ at most.

        These elements do not form a subgroup. Consider $ts$ and $s$, both of which
        have order $2$. However, $(ts)(s) = ts^2 = t$, which has infinite order
        because there is no $n > 0$ such that $t^n = e$.
    \end{solution}
    
    \paragraph{Exercise 5.9} Let $f$ be a function from the real line to itself which
    preserves the distance between every pair of points and which sends the integers
    among themselves.
    \begin{enumerate}
        \itemsep0em   
        \item Assuming that $f$ has no fixed points, show that $f$ is a
        \textit{translation} through an integral distance.
        \item If $f$ leaves exactly one point fixed, show that this point is either
        an integer or lies midway between two integers, and that $f$ is a reflection
        in this fixed point.
        \item Finally, check that $f$ must be the identity if it leaves more than
        one point fixed.
    \end{enumerate}
    \begin{solution}
        Define $g$ such that $g(x) = f(x) - f(0)$. We see that $g(0) = 0$, and that
        \[
            |g(x) - g(y)| = |f(x) - f(y)| = |x - y|
        \] for all real numbers $x$ and $y$. Specifically, $|g(x)| = |x|$.

        It can be shown that we must have either $g(x) = x$, or $g(x) = -x$.

        \begin{enumerate}
            \item Since $f$ has no fixed points, $f(0) \neq 0$. Now, suppose that
            $g(x) = -x$. This means that $f(x) = -x + f(0)$, which gives $f(f(0) /
            2) = f(0) / 2$, hence $f(0) / 2$ is a fixed point. This is a
            contradiction, hence we must have $g(x) = x$, hence $f(x) = x + f(0)$.
            This has no fixed points as desired. Note that $f$ is a translation by
            $f(0)$.

            \item We saw that $g(x) = -x$, i.e.\ $f(x) = -x + f(0)$ gives exactly
            one fixed point at $f(0) / 2$. Recall that $f$ sends integers to
            integers, hence $x_0 = f(0) / 2$ is either an integer or halfway between
            two integers. We can write $f$ as a reflection about this point since \[
                (f(x) - x_0) = -(x - x_0).
            \] 

            \item If $f$ leaves more than one fixed point, note that we cannot have
            $g(x) = -x$, since $f(x) = -x + f(0)$ has exactly one fixed point. This
            leaves $f(x) = x + f(0)$. For a fixed point $x_0$, we have $x_0 = x_0 +
            f(0)$, hence $f(0) = 0$. This leaves $f(x) = x$, which is the identity
            map.
        \end{enumerate}
    \end{solution}

    \paragraph{Exercise 5.10} Make a list of those elements of $\Z_{12}$ which
    generate $\Z_{12}$. Answer the same question for $\Z_5$ and for $\Z_9$.
    Do your answers suggest a general result?
    \begin{solution}
        The generators of $\Z_{12}$ are $1, 5, 7, 11$. The generators of $\Z_5$ are
        $1, 2, 3, 4$, and the generators of $\Z_9$ are $1, 2, 4, 5, 7, 8$.

        We have already shown (Artin 2.2.16) that $k \in \Z_n$ generates $\Z_n$
        precisely when $k$ and $n$ are coprime.
    \end{solution}

    \paragraph{Exercise 5.11} Show that $\Q$ is not cyclic. Even better, show that
    $\Q$ cannot be generated by a finite number of elements.
    \begin{solution}
        Suppose that the rational numbers $x_1, \dots, x_n$ generate $\Q$, where
        $x_i = p_i / q_i$ for integers $p_i$ and $q_i$. Express $x_i$ such that all
        $q_i$ are positive. Note that for rational numbers, any sum is of the form
        $a / b + c / d = (ad + bc) / bd$. Thus, any linear combination of $x_i$ can
        be directly written in the form $s / (q_1 \cdots q_n)$. Let $q = q_1 \cdots
        q_n$, and consider the rational $x = 1 / (q + 1)$. Suppose that this is 
        generated by $x_1, \dots, x_n$, hence we must be able to write $x = 1 / (q +
        1) = s / q > 0$ for some integer $s$, which must clearly be positive.
        Rearranging, $q = s(q + 1)$, or $q(s - 1) + s = 0$. This is impossible,
        since $q > 0$ and $s - 1 \geq 0$. Thus, $1 / (q + 1) \in \Q$ cannot be
        generated by $x_1, \dots, x_n$. This proves that $\Q$ cannot be generated by
        a finite set of rational numbers.
    \end{solution}

    \paragraph{Exercise 5.12} If $a, b \in \Z$ are both not zero and if $H =
    \{\lambda a + \mu b : \lambda, \mu \in \Z\}$, show that $H$ is a subgroup of
    $\Z$. Let $d$ be the smallest positive integer in $H$. Check that $d$ is the
    highest common factor of $a$ and $b$.
    \begin{solution}
        Clearly, $0 = 0a + 0b \in H$, and if $\lambda_1 a + \mu_1 b, \lambda_2 a +
        \mu_2 b \in H$, then the sum $(\lambda_1 + \lambda_2)a + (\mu_1 + \mu_2)b
        \in H$. We also have the inverse of $\lambda a + \mu b \in H$, namely
        $(-\lambda)a + (-\mu)b \in H$. This proves that $H$ is a subgroup of $\Z$.

        We claim that $d$ generates $H$. First, we show that any subgroup $B < \Z$ is
        generated by the smallest positive integer $b \in B$. Clearly, the cyclic
        group $\langle b\rangle < B$, since all elements $nb \in B$. Next, pick some
        non-zero $n \in B$, which means that $-n \in B$, so without loss of
        generality suppose that $n$ is positive. The minimality of $b$ forces $b
        \leq n$, so use Euclid's Division Lemma to write $n = bq + r$ for integers
        $q \geq 1$, $0 \leq r < b$. Since $b \in B$ and $n \in B$, so is $n - bq =
        r$. By the minimality of $b$, $r$ cannot be positive, and hence must be $0$.
        This gives $n = bq \in \langle b\rangle$ for all $n \in B$, hence $B <
        \langle b\rangle$. Thus, $B = \langle b \rangle$.

        In our case, $H = \langle d\rangle$, and we know that $a = 1a + 0b \in H$
        and $b = 0a + 1b \in H$. This means that $a = pd$ and $b = qd$ for integers
        $p$ and $q$, hence $d$ is a common factor of $a$ and $b$. Furthermore, it is
        the highest common factor, since if $d'$ were another common factor of $a$ and
        $b$, we would have $a = p'd'$ and $b = q'd'$, hence $a, b \in \langle
        d'\rangle$. Thus, all linear combinations $\lambda a + \mu b \in \langle
        d'\rangle$, of which one such choice gives $d$. This means $d \in \langle
        d'\rangle$, hence $d = kd'$, or $d'$ divides $d$.
    \end{solution}
    

\end{document}
% vim: set tabstop=4 shiftwidth=4 softtabstop=4:
