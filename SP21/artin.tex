\documentclass[11pt]{report}

\usepackage[T1]{fontenc}
\usepackage{geometry}
\usepackage{amsmath, amssymb, amsthm}
\usepackage{bm}
\usepackage[scr]{rsfso}
\usepackage{xcolor}
\usepackage{fancyhdr}
\usepackage{hyperref}

\geometry{a4paper, margin=1in, headheight=14pt}

\pagestyle{fancy}
\renewcommand\headrulewidth{0.4pt}
\fancyhead[L]{\it Algebra}
\rfoot{\footnotesize\it Updated on \today}
\cfoot{\thepage}
\renewcommand{\chaptermark}[1]{\markboth{#1}{}}

\renewcommand{\labelenumi}{(\alph{enumi})}
\renewcommand{\labelenumii}{(\roman{enumii})}

\def\C{\mathbb{C}}
\def\R{\mathbb{R}}
\def\Q{\mathbb{Q}}
\def\Z{\mathbb{Z}}
\def\N{\mathbb{N}}

\DeclareMathOperator\sign{sign}
\DeclareMathOperator\lcm{lcm}
\DeclareMathOperator\aut{Aut}
\DeclareMathOperator\im{im}
\DeclareMathOperator\sgn{sgn}

\renewcommand\vec\boldsymbol
\def\vx{\vec{x}}
\def\vy{\vec{y}}
\def\vz{\vec{z}}
\def\ve{\vec{e}}

\theoremstyle{remark}
\newtheorem*{remark}{Remark}
\newtheorem*{example}{Example}
\newtheorem*{solution}{Solution}

\title{
    \Large\textsc{Summer Programme 2021} \\
    \vspace{10pt}
    \huge Solutions to exercises from Michael Artin's \\
    \textit{Algebra}
}
\author{
    \large Satvik Saha%
    % \thanks{Email: \tt ss19ms154@iiserkol.ac.in}
    \\\textsc{\small 19MS154}
}
\date{\normalsize
    \textit{Indian Institute of Science Education and Research, Kolkata, \\
    Mohanpur, West Bengal, 741246, India.} \\
    % \vspace{10pt}
    % \today
}

\begin{document}
    \maketitle
    \tableofcontents

    \chapter{Matrix Operations}
    \setcounter{section}{3}
    \section{Permutation Matrices}

    \paragraph{Exercise 1.} Consider the permutation $p$ defined by
    $1\rightsquigarrow 3$, $2\rightsquigarrow 1$, $3\rightsquigarrow 4$,
    $4\rightsquigarrow 2$.
    \begin{enumerate}
        \itemsep0em 
        \item Find the associated permutation matrix $P$.
        \item Write $p$ as a product of transpositions and evaluate the
        corresponding matrix product.
        \item Compute the sign of $p$.
    \end{enumerate}
    \begin{solution}
        \mbox{}
        \begin{enumerate}
            \itemsep0em
            \item The column $P_i$ must be the standard basis vector $\ve_{p(i)}$, so
            \[
                P = \begin{bmatrix}
                    0 & 1 & 0 & 0 \\
                    0 & 0 & 0 & 1 \\
                    1 & 0 & 0 & 0 \\
                    0 & 0 & 1 & 0 
                \end{bmatrix}.
            \] 
            \item Check that $p = (1, 3, 4, 2) = (1, 2)(1, 4)(1, 3)$. This product
            is given by \[
                \begin{bmatrix}
                    0 & 1 & 0 & 0 \\
                    1 & 0 & 0 & 0 \\
                    0 & 0 & 1 & 0 \\
                    0 & 0 & 0 & 1
                \end{bmatrix}
                \begin{bmatrix}
                    0 & 0 & 0 & 1 \\
                    0 & 1 & 0 & 0 \\
                    0 & 0 & 1 & 0 \\
                    1 & 0 & 0 & 0
                \end{bmatrix}
                \begin{bmatrix}
                    0 & 0 & 1 & 0 \\
                    0 & 1 & 0 & 0 \\
                    1 & 0 & 0 & 0 \\
                    0 & 0 & 0 & 1
                \end{bmatrix}
                = \begin{bmatrix}
                    0 & 1 & 0 & 0 \\
                    0 & 0 & 0 & 1 \\
                    1 & 0 & 0 & 0 \\
                    0 & 0 & 1 & 0 
                \end{bmatrix}
                = P.
            \] 
            \item Since $p$ is the product of an odd number of transpositions, its
            sign is $-1$. This is verified by calculating the determinant \[
                \det \begin{bmatrix}
                    0 & 1 & 0 & 0 \\
                    0 & 0 & 0 & 1 \\
                    1 & 0 & 0 & 0 \\
                    0 & 0 & 1 & 0 
                \end{bmatrix} = - \det \begin{bmatrix}
                    1 & 0 & 0 & 0 \\
                    0 & 0 & 0 & 1 \\
                    0 & 1 & 0 & 0 \\
                    0 & 0 & 1 & 0 
                \end{bmatrix} = \det \begin{bmatrix}
                    1 & 0 & 0 & 0 \\
                    0 & 1 & 0 & 0 \\
                    0 & 0 & 0 & 1 \\
                    0 & 0 & 1 & 0 
                \end{bmatrix} = -\det \begin{bmatrix}
                    1 & 0 & 0 & 0 \\
                    0 & 1 & 0 & 0 \\
                    0 & 0 & 1 & 0 \\
                    0 & 0 & 0 & 1
                \end{bmatrix} = -1.
            \]
        \end{enumerate}
    \end{solution}

    \paragraph{Exercise 2.} Prove that every permutation matrix is a product of
    transpositions.
    \begin{solution}
        Note that this is equivalent to stating that any ordered list can be sorted
        using transpositions.

        The statement is trivially true for all $2\times 2$ permutation matrices, \[
            \begin{bmatrix}
                1 & 0 \\ 0 & 1
            \end{bmatrix},
            \begin{bmatrix}
                0 & 1 \\ 1 & 0
            \end{bmatrix},
        \] the first being the identity and the second being a transposition itself.
        Suppose that any $n\times n$ permutation matrix is the product of
        transpositions. Use the fact that for square matrices $A$ and $B$, \[
            \begin{bmatrix}
                A & 0 \\
                0 & 1
            \end{bmatrix}
            \begin{bmatrix}
                B & 0 \\
                0 & 1
            \end{bmatrix}
            = \begin{bmatrix}
                AB & 0 \\
                0  & 1
            \end{bmatrix},
        \] which means that if a permutation matrix $P = E_1E_2 \dots E_k$, then \[
            \begin{bmatrix}
                P & 0 \\
                0 & 1
            \end{bmatrix} =
            \begin{bmatrix}
                E_1 & 0 \\
                0   & 1
            \end{bmatrix}
            \dots
            \begin{bmatrix}
                E_k & 0 \\
                0   & 1
            \end{bmatrix}.
        \] Now let $Q$ be an arbitrary $(n + 1)\times(n + 1)$ permutation matrix.
        Let $j$ be the index of the row of $Q$ which is precisely $(0\, \dots 0\,
        1)$, and let $E$ be the transposition matrix which interchanges the rows $j
        \leftrightarrow n + 1$. Then, \[
            EQ = \begin{bmatrix}
                Q' & 0 \\
                0  & 1
            \end{bmatrix},
        \] where $Q'$ is an $n \times n$ permutation matrix. This is because $Q'$
        has exactly one $1$ in each row and column, the remaining elements being
        $0$. Multiply both sides by $E$, and use the fact that $E^2 = \mathbb{I}$.
        Now, $Q'$ is a product of transpositions $E_1\dots E_k$, so we finally have
        \[
            Q = E
            \begin{bmatrix}
                Q' & 0 \\
                0  & 1
            \end{bmatrix} = E
            \begin{bmatrix}
                E_1 & 0 \\
                0   & 1
            \end{bmatrix}
            \dots
            \begin{bmatrix}
                E_k & 0 \\
                0   & 1
            \end{bmatrix}.
        \] 
    \end{solution}

    \paragraph{Exercise 3.} Prove that every matrix with a single $1$ in each row
    and a single $1$ in each column, the other entries being zero, is a permutation
    matrix.
    \begin{solution}
        Note that each column of such a matrix $P$ must be a distinct standard basis
        vector $\ve_k$, and we claim that this matrix represents the permutation $p$
        defined as $p(j) = k$, where $P_j = \ve_k$ is the $j$\textsuperscript{th}
        column of $P$. Now, $p$ is a bijection because every column $j$ has one and exactly one
        $1$ in the $k$\textsuperscript{th} row. This justifies that $p$ is indeed a
        permutation. When $P$ acts on a column vector $\vx$, we
        have \[
            P\vx = P_1x_1 + P_2x_2 + \dots + P_nx_n = \ve_{p(1)}x_1 + \ve_{p(2)}x_2 +
            \dots + \ve_{p(n)}x_n.
        \] This means that \[
            P \begin{bmatrix}
                x_1 \\ x_2 \\ \vdots \\ x_n
            \end{bmatrix}
            = \begin{bmatrix}
                x_{p^{-1}(1)} \\ x_{p^{-1}(2)} \\ \vdots \\ x_{p^{-1}(n)}
            \end{bmatrix}.
        \] 
    \end{solution}
    
    \paragraph{Exercise 4.} Let $p$ be a permutation. Prove that $\sign{p} =
    \sign{p^{-1}}$.
    \begin{solution}
        This follows directly from the fact that $\det{P^{-1}} = 1 / \det P$, and
        that $\det P = \pm 1$ so $\det{P^{-1}} = \det{P}$.
    \end{solution}

    \paragraph{Exercise 5.} Prove that the transpose of a permutation matrix $P$ is
    its inverse.
    \begin{solution}
        Recall that $\det{P} = \pm 1$, so $P$ is invertible.
        Write the permutation matrix $P$ in terms of its columns, \[
            P = \begin{bmatrix}
                \vline & \vline & \dots & \vline \\
                \ve_{p(1)} & \ve_{p(2)} & \dots & \ve_{p(n)} \\
                \vline & \vline & \dots & \vline 
            \end{bmatrix},
        \] where $p$ represents the corresponding permutation.
        Now note that the transpose can be written as \[
            P = \begin{bmatrix}
                \text{---}\, \ve_{p(1)}^t\,\text{---} \\
                \text{---}\, \ve_{p(2)}^t\,\text{---} \\
                \vdots \\ 
                \text{---}\, \ve_{p(n)}^t\,\text{---} 
            \end{bmatrix}.
        \] Therefore, the $ij$\textsuperscript{th} element of the product $P^tP$ is
        given by $\ve_{p(i)}^t \ve_{p(j)} = \delta_{p(i)p(j)} = \delta_{ij}$,
        meaning that $P^tP = \mathbb{I}$. We have used the fact that $p$ is a
        bijection, so $p(i) = p(j)$ if and only if $i = j$. Thus, $P^{-1} = P^t$.
    \end{solution}

    \section{Cramer's Rule}

    \paragraph{Exercise 3.} Let $A$ be an $n\times n$ matrix with integer entries
    $a_{ij}$. Prove that $A^{-1}$ has integer entries if and only if $\det{A} = \pm
    1$.
    \begin{solution}
        First, suppose that $\det{A} = \pm 1$. If the entries of $A^{-1}$ are
        $b_{ij}$, use \[
            A^{-1} = \frac{1}{\det{A}} \operatorname{adj}{A}
        \] to conclude that \[
            b_{ij} = \frac{1}{\det{A}} (-1)^{i + j} \det A_{ji}.
        \] Note that $A_{ji}$ contains integer entries, hence its determinant must
        also be an integer via the complete expansion. Putting $\det{A} = \pm 1$
        means that $b_{ij}$ is always an integer.

        Now suppose that $A^{-1}$ has integer entries. Use $\det{A} = 1 /
        \det{A}^{-1}$. Now both $A$ and $A^{-1}$ have integer entries, hence integer
        determinants, with $|\det{A^{-1}}| \geq 1$. This forces $\det{A} = \pm 1$.
    \end{solution}


    
    \section*{Miscellaneous Problems}
    \addcontentsline{toc}{section}{Miscellaneous Problems}

    \paragraph{Exercise 2.} Find a representation of the complex numbers by real 2 x
    2 matrices which is compatible with addition and multiplication.
    \begin{solution}
        Consider the representation \[
            z = a + ib \,\equiv\, \begin{bmatrix}
                a & -b \\ b & a
            \end{bmatrix}.
        \] Now, if $z = a + ib$, $w = c + id$, we have addition defined as \[
            z + w \,\equiv\, \begin{bmatrix}
                a & -b \\ b & a
            \end{bmatrix} + \begin{bmatrix}
                c & -d \\ d & c
            \end{bmatrix} = \begin{bmatrix}
                a + c & -b - d \\ b + d & a + c
            \end{bmatrix} \,\equiv\, (a + c) + i(b + d),
        \] and multiplication as \[
            zw \,\equiv\, \begin{bmatrix}
                a & -b \\ b & a
            \end{bmatrix} \begin{bmatrix}
                c & -d \\ d & c
            \end{bmatrix} = \begin{bmatrix}
                ac - bd & -ad - bc \\ ad + bc & ac - bd
            \end{bmatrix} \,\equiv\, (ac - bd) + i(ad + bc).
        \] Finally, \[
            |z|^2 = z\overline{z} = a^2 + b^2 = \det \begin{bmatrix}
                a & -b \\ b & a
            \end{bmatrix}.
        \] 
    \end{solution}

    \paragraph{Exercise 3.} Find the Vandermonde determinant \[
        \det{A_n} = \det \begin{bmatrix}
            1 & 1 & \cdots & 1 \\
            a_1 & a_2 & \cdots & a_n \\
            a_1^2 & a_2^2 & \cdots & a_n^2 \\
            \vdots & \vdots & \ddots & \vdots \\
            a_1^{n - 1} & a_2^{n - 1} & \cdots & a_n^{n - 1}
        \end{bmatrix}.
    \] 
    \begin{solution}
        First look at the $2 \times 2$ case, \[
            \det{A_2} = \det \begin{bmatrix}
                1 & 1 \\ a_1 & a_2
            \end{bmatrix}
            = a_2 - a_1.
        \] Now, look at the $n \times n$ case. Perform the row operations $R_k \to
        R_k - a_1R_{k - 1}$ for all rows $k = 2, \dots, n$. This leaves the
        determinant unchanged, so \[
            \det{A_2} = \det \begin{bmatrix}
                1 & 1 & 1 & \cdots & 1 \\
                0 & a_2 - a_1 & a_3 - a_1 & \cdots & a_n - a_1 \\
                0 & a_2(a_2 - a_1) & a_3(a_3 - a_1) & \cdots & a_n(a_n - a_1) \\
                \vdots & \vdots & \vdots & \ddots & \vdots \\
                0 & a_2^{n - 2}(a_2 - a_1) & a_3^{n - 2}(a_3 - a_1) & \cdots &
                a_n^{n - 2}(a_n - a_1)
            \end{bmatrix}.
        \] Using expansion by minors on the first column, we have \[
            \det{A_2} = \det \begin{bmatrix}
                 a_2 - a_1 & a_3 - a_1 & \cdots & a_n - a_1 \\
                 a_2(a_2 - a_1) & a_3(a_3 - a_1) & \cdots & a_n(a_n - a_1) \\
                \vdots & \vdots & \ddots & \vdots \\
                a_2^{n - 2}(a_2 - a_1) & a_3^{n - 2}(a_3 - a_1) & \cdots &
                a_n^{n - 2}(a_n - a_1)
            \end{bmatrix}.
        \] Factoring out $a_j - a_1$ from each $j$\textsuperscript{th} column gives
        \[
            \det{A_n} = \prod_{j = 2}^{n} (a_j - a_1) \times \det \begin{bmatrix}
                 1 & 1 & \cdots & 1 \\
                 a_2 & a_3 & \cdots & a_n \\
                \vdots & \vdots & \ddots & \vdots \\
                a_2^{n - 2} & a_3^{n - 2} & \cdots &
                a_n^{n - 2}
            \end{bmatrix}
        \] Continuing in this fashion, we get \[
            \det{A_n} = \prod_{j = 2}^{n} (a_j - a_1) \times \prod_{j = 3}^{n} (a_j
            - a_2) \times \dots \times (a_{n - 1} - a_n).
        \] This can be written down concisely as \[
            \det{A_n} = \prod_{1 \leq i < j \leq n}^{n} (a_j - a_i)
        \] 
    \end{solution}
    
    \paragraph{Exercise 4.} Consider a general system $AX = B$ of $m$ linear
    equations in $n$ unknowns. If the coefficient matrix $A$ has a left inverse
    $A'$, a matrix such that $A'A = \mathbb{I}_n$, then we may try to solve the
    system as follows. \begin{align*}
        AX &= B, \\
        A'AX &= A'B \\
        X &= A'B.
    \end{align*}
    But when we try to check our work by running the solution backward, we get into
    trouble: \begin{align*}
        X &= A'B \\
        AX &= A A' B \\
        AX &\stackrel{?}{=} B.
    \end{align*}
    We seem to want $A'$ to be a right inverse: $A A' = \mathbb{I}_n$, which isn't
    what was given. Explain.
    \begin{solution}
        In the case that $m > n$, note that the left inverse is not necessarily
        unique. An example is \[
            \begin{bmatrix}
                1 & 0 & a \\ 0 & 1 & b
            \end{bmatrix} \begin{bmatrix}
                1 & 0 \\ 0 & 1 \\ 0 & 0
            \end{bmatrix} = \begin{bmatrix}
                1 & 0 \\ 0 & 1
            \end{bmatrix} = \mathbb{I}_2,
        \] irrespective of $a$ and $b$. Hence, $X = A'B$ is not unique, but rather
        is dependent on our choice of $A'$. If we had started with \[
            \begin{bmatrix}
                1 & 0 \\ 0 & 1 \\ 0 & 0
            \end{bmatrix} \begin{bmatrix}
                x \\ y
            \end{bmatrix} = \begin{bmatrix}
                p \\ q \\ r
            \end{bmatrix}
        \] then we would have written \[
            \begin{bmatrix}
                x \\ y
            \end{bmatrix} = \begin{bmatrix}
                1 & 0 & a \\ 0 & 1 & b
            \end{bmatrix} \begin{bmatrix}
                p \\ q \\ r
            \end{bmatrix} = \begin{bmatrix}
                p + ar \\ q + br
            \end{bmatrix} = \begin{bmatrix}
                p \\ q
            \end{bmatrix} + r \begin{bmatrix}
                a \\ b
            \end{bmatrix}.
        \] 
        This means that the given argument is not
        sufficient to conclude $A A' = \mathbb{I}$. \[
            \begin{bmatrix}
                1 & 0 \\ 0 & 1 \\ 0 & 0
            \end{bmatrix} \begin{bmatrix}
                1 & 0 & a \\ 0 & 1 & b
            \end{bmatrix} = \begin{bmatrix}
                1 & 0 & a\\ 0 & 1 & b \\ 0 & 0 & 0
            \end{bmatrix} \neq \mathbb{I}_3,
        \] Note that this system is nonsense for $r \neq 0$ with no solutions, yet
        the left inverses $A'$ do exist nonetheless. Here, $AX \neq B$ when $r \neq
        0$, since \[
            \begin{bmatrix}
                1 & 0 \\ 0 & 1 \\ 0 & 0
            \end{bmatrix} \begin{bmatrix}
                p + ar \\ q + ar
            \end{bmatrix} = \begin{bmatrix}
                 p + ar \\ q + ar \\ 0
            \end{bmatrix}.
        \] \\

        In the case that $m < n$, $A$ has no left inverse. Label the columns of $A$
        as $A_i$. Demanding $A'A = \mathbb{I}_n$ means that we want $A' A_i =
        \ve_{i}$ for all $i = 1, \dots, n$. Since the $m \times n$ matrix $A$ has
        more columns than rows, its columns must be linearly dependent, so without
        loss of generality, write the first column $A_1$ as a non-trivial linear
        combination of the rest, \[
            A_1 = a_2 A_2 + a_3A_3 + \dots + a_nA_n.
        \] Multiplying by $A'$ gives \[
            A'A_1 = \ve_1 = a_2\ve_2 + a_3\ve_3 + \dots + a_n\ve_n,
        \] which is a contradiction since the basis vectors $\{\ve_i\}$ are linearly
        independent. \\

        In the case $m = n$, it is indeed true that $A'$ is also a right inverse of
        $A$. Note that if $A''$ is a right inverse of $A$ with $A A'' =
        \mathbb{I}_n$, then \[
            A' = A'\mathbb{I}_n = A'(A A'') = (A' A) A'' = \mathbb{I}_n A'' = A''.
        \] To justify that $A''$ exists, note that $A'A = \mathbb{I}_n$ gives
        $\det{A'}\det{A} = 1$, so $\det{A} \neq 0$. Thus, $A$ has full rank and its
        range must be the full $n$ dimensional vector space of column vectors.
        Multiplying by $A$, we have $A A' A = A$ or $(A A' - \mathbb{I}_n) A = 0$.
        Recall that the range of $A$ is the entire vector space, so $(A A' -
        \mathbb{I}_n)\vx = \vec{0}$ for all possible column vectors $\vx$. This
        forces $A A' - \mathbb{I}_n = 0$, or $A A' = \mathbb{I}_n$.
    \end{solution}

    \paragraph{Exercise 5.} \begin{enumerate}
        \itemsep0em    
        \item Let $A$ be a real $2\times 2$ matrix, and let $A_1$, $A_2$ be the rows
        of $A$. Let $P$ be the parallelogram whose vertices are $0$, $A_1$, $A_2$,
        $A_1 + A_2$. Prove that the area of $P$ is the absolute value of the
        determinant $\det{A}$ by comparing the effect of and elementary row
        operation on the area and on $\det A$.
        \item Prove an analogous result for $n \times n$ matrices.
    \end{enumerate}
    \begin{solution} \mbox{}
        \begin{enumerate}
            \itemsep0em
            \item First note that \[
                \det \begin{bmatrix}
                    1 & 0 \\ 0 & 1
                \end{bmatrix} = 1,
            \] which is consistent with the fact that the area of a unit square is
            $1$. Now let $a_{ij}$ be the elements of $A$. Perform the row operation
            which multiplies the top row by $a_{11}$, i.e.\ $R_1 \to a_{11}R_1$. We
            have \[
                \det \begin{bmatrix}
                    a_{11} & 0 \\ 0 & 1
                \end{bmatrix} = a_{11}.
            \] Now perform $R_1 \to R_1 + a_{12}R_2$. This gives \[
                \det \begin{bmatrix}
                    a_{11} & a_{12} \\ 0 & 1
                \end{bmatrix} = a_{11}.
            \] Next, perform $R_2 \to (a_{11}a_{22} - a_{12}a_{21})R_2$. This gives
            \[
                \det \begin{bmatrix}
                    a_{11} & a_{12} \\ 0 & a_{11}a_{22} - a_{12}a_{21}
                \end{bmatrix} = (a_{11}a_{22} - a_{12}a_{21})a_{11}.
            \] Next, perform $R_2 \to R_2 + a_{21}R_1$. This gives \[
                \det \begin{bmatrix}
                    a_{11} & a_{12} \\ a_{11}a_{21} & a_{11}a_{22}
                \end{bmatrix} = (a_{11}a_{22} - a_{12}a_{21})a_{11}.
            \] Finally, perform $R_2 \to R_2 / a_{11}$. This gives \[
                \det {A} = \det \begin{bmatrix}
                    a_{11} & a_{12} \\ a_{21} & a_{22}
                \end{bmatrix} = a_{11}a_{22} - a_{12}a_{21}.
            \] Note that if $a_{11} = 0$, we could have interchanged the roles of
            $a_{11}$ and $a_{21}$ at the beginning by interchanging the rows of $A$.
            This would have given the same result, up to a sign which we are not
            interested in.
            If both $a_{11}$ and $a_{22}$ are zero, note that the two rows are
            linearly dependent, with one being a multiple of the other, so the
            parallelogram they form has zero area.
            Thus, we have shown that any matrix $A$ representing a parallelogram
            with non-zero area can be obtained from the identity matrix
            $\mathbb{I}_2$ by performing elementary row operations.

            Now, we consider the effect of these row operations on the area of a
            parallelogram with legs $A_1$ and $A_2$. Note that the operation $A_1
            \to kA_1$ for some real scaling factor $k$ has the effect of scaling the
            area by the same factor $k$. The operation of interchanging the legs
            $A_1$ and $A_2$ has no effect on the area. The operation $A_1 \to A_1 +
            kA_2$ also has no effect on the area, because this has the effect of
            linearly shearing the parallelogram, in a manner parallel to the other
            leg $A_2$ which remains fixed. Thus, when we performed our row
            operations in the square to reach our parallelogram, our area
            transformed in precisely the same way as the unsigned determinant, which
            means that \[
                \operatorname{area} A_{\parallel} = |\det{A}| = |a_{11}a_{22} -
                a_{12}a_{21}|.
            \]

            \item We use the fact that any square matrix $A$ with non-zero
            determinant can be written as the product of row operations acting on
            the identity matrix $\mathbb{I}_n$, which represents the unit hypercube of
            hypervolume $1$. The Gauss-Jordan elimination algorithm can be used to
            extract these operations. We see that all scaling operations will scale
            the hypervolume in the same way, all transpositions have no effect on
            the hypervolume, and all additions of linear combinations of other rows
            also have no effect, since they correspond to successive shearing of the
            hyperparallelopiped along a direction parallel to another leg. Thus, the
            area of the hypercube transformed in the same way as the unsigned
            determinant of $A$, so \[
                \operatorname{hypervolume} A_{\parallel} = |\det{A}|.
            \] 

            Note that we are not interested in matrices
            with zero determinant, because such a matrix is not of full rank, 
            hence its rows are linearly dependent. Thus, one of the legs of the
            corresponding hyperparallelopiped can be sheared until it is parallel to
            another, which immediately gives a zero hypervolume.
    \end{enumerate}
    \end{solution}

    \paragraph{Exercise 6.} Most invertible matrices can be written as a product $A
    = LU$ of a lower triangular matrix $L$ and an upper triangular matrix $U$, where
    in addition all diagonal entries of $U$ are $1$.
    \begin{enumerate}
        \itemsep0em
        \item Prove uniqueness, that is, prove that there is at most one way to
        write $A$ as a product.
        \item Explain how to compute $L$ and $U$ when the matrix $A$ is given.
        \item Show that every invertible matrix can be written as a product $LPU$,
        where $L$, $U$ are as above and $P$ is a permutation matrix.
    \end{enumerate}
    \begin{solution}
        We first show that the determinant of a triangular matrix is equal to the
        product of its diagonals. To see this, note that this holds for all $2
        \times 2$ lower triangular matrices, \[
            \det \begin{bmatrix}
                a & 0 \\ c & d
            \end{bmatrix} = ad.
        \] Next, suppose that this holds for all $n \times n$ lower triangular
        matrices. Using expansion of minors along the first row and our induction
        hypothesis on the minor $A_{11}$, compute \[
            \det \begin{bmatrix}
                a_{11} & 0 & 0 & \cdots & a_{1n} \\
                a_{21} & a_{22} & 0 & \cdots & a_{2n} \\
                a_{31} & a_{32} & a_{33} & \cdots & a_{3n} \\
                \vdots & \vdots & \vdots & \ddots & \vdots \\
                a_{n1} & a_{n2} & a_{n3} & \cdots & a_{nn}
            \end{bmatrix} = a_{11}\det{A_{11}} + 0 + 0 + \dots + 0 = a_{11}
            a_{22}a_{33} \cdots a_{nn}.
        \] For upper triangular matrices, simply note that $\det U = \det U^t$, and
        $U^t$ is lower triangular.

        Now, we show that the inverse of a triangular matrix is also triangular of
        the same kind. Note that this holds for all invertible $2\times 2$ matrices,
        with \[
            \begin{bmatrix}
                a & 0 \\ c & d
            \end{bmatrix}^{-1} = \frac{1}{ad}\begin{bmatrix}
                d & 0 \\ -c & a
            \end{bmatrix}.
        \] Next, suppose that an invertible lower triangular matrix $L$ has an
        inverse $L^{-1}$, whose columns are labelled $\vx_j$. Since $L L^{-1} =
        \mathbb{I}_n$, we want \[
            L\vx_j = \ve_{j}.
        \] We claim that $(x_{j})_i = 0$ for all $i < j$. To see this, note that the
        first $j - 1$ rows expand to \begin{align*}
            0 &= L_{11}x_{j1} \\
            0 &= L_{12}x_{j1} + L_{22}x_{j2} \\
            \vdots &\qquad \vdots \\
            0 &= L_{j - 1, 1}x_{j, j - 1} + \dots + L_{j-1, j-1}x_{j, j - 1}
        \end{align*}
        All $L_{ij}$ with $i < j$ are zero, and $L_{ii}$ are non-zero since $L$ is
        invertible hence $\det{L} \neq 0$. Thus, the first equation gives $x_{j1} =
        0$, which when plugged into the second gives $x_{j2} = 0$, and so on up to
        $x_{j, j - 1} = 0$. Hence, $L^{-1}_{ij} = 0$ for all $i < j$, making it a
        lower triangular matrix. In addition, the $j$\textsuperscript{th} row reads
        \[
            1 = L_{j1}x_{j1} + \dots + L_{j,j - 1}x_{j, j - 1} + L_{jj} x_{jj}.
        \] All terms but the last one are $0$, so the diagonal elements satisfy
        $L_{jj}L^{-1}_{jj} = 1$.
        Like before, for a lower triangular matrix $U$, use
        $(U^t)^{-1} = (U^{-1})^t$.

        Finally, the product of two triangular matrices of the same kind give
        another triangular matrix of the same kind. Suppose that $A$ and $B$ are two
        lower triangular matrices. The $ij$\textsuperscript{th} element of their
        product $AB$ is \[
            c_{ij} = \sum_{k = 1}^n a_{ik}b_{kj}.
        \] Now, $a_{ik} = 0$ for all $i < k$ and $b_{ki} = 0$ for all $k < j$. Thus,
        when $i < j$, we have $c_{ij} = 0$, hence $AB$ is also lower triangular.
        Furthermore, if all the diagonal entries $a_{ii} = b_{ii} = 1$, then the
        only term in the sum is $c_{ii} = a_{ii}b_{ii} = 1$, so the diagonal entries
        of $C$ are all $1$.  Again for upper triangular matrices $X$, $Y$, use
        $(XY)^t = Y^tX^t$.

        \begin{enumerate}
            \itemsep0em
            \item Suppose that $A = LU = L'U'$ are two $LU$ decompositions of $A$.
            Note that $\det{A} \neq 0$ from its invertibility, hence $L$, $U$, $L'$,
            $U'$ are all invertible. This gives \[
                L^{-1}LU = L^{-1}L'U', \qquad U = L^{-1}L'U', \qquad U(U')^{-1} =
                L^{-1}L'.
            \] Now, the left side is upper triangular while the right side is left
            triangular. Also, the left side has all $1$'s along its diagonal. This
            forces \[
                U(U')^{-1} = \mathbb{I}_n = L^{-1}L', \qquad U = U', \quad L = L'.
            \] 
            
            \item The elements of $L$ and $U$ can be obtained by brute force,
            solving the system $A = LU$ with $n(n + 1) / 2 + (n - 1)n/ 2 = n^2$
            unknowns.

            \item Note that after performing Gaussian elimination on an invertible
            matrix $A$, we are left with an upper triangular matrix $U$ with $1$'s
            along its diagonal. Also, each elementary operation we performed can be
            represented by a lower triangular matrix. This is because all scaling
            matrices are diagonal, and in all cases where we added one row to
            another we always added higher row to ones lower down. Thus, the product
            of all these elementary matrices is a lower triangular matrix $L$, which
            means $LA = U$. This gives the desired decomposition, $A = L^{-1}U$.

            However, we may have to exchange rows while performing the elimination
            process, which happens when one of the diagonal elements becomes zero.
            By performing this permutation of rows at the very end, we have actually
            decomposed $PLA = U$. The inverse of a permutation is another
            permutation, hence we have the desired decomposition $A = L^{-1}P^{-1}U$.
        \end{enumerate}
    \end{solution}

    \paragraph{Exercise 7.} Consider a system of $n$ linear equations in $n$ unknowns:
    $AX = B$, where $A$ and $B$ have \textit{integer} entries. Prove or disprove the
    following.
    \begin{enumerate}
        \itemsep0em
        \item The system has a rational solution if $\det{A} \neq 0$.
        \item If the system has a rational solution, then it also has an integer
        solution.
    \end{enumerate}
    \begin{solution}\mbox{}
    \begin{enumerate}
        \itemsep0em
        \item If $\det{A} \neq 0$, then $A$ is invertible. Since $A$ has integer
        entries, its determinant is an integer and its adjoint has integer entries,
        which means that $A^{-1} = (\operatorname{adj}{A})/\det{A}$ has rational
        entries. Also, $B$ has integer entries so the solution $X = A^{-1}B$ must
        also be rational.
        \item This is false. Consider the system \[
            \begin{bmatrix}
                1 & 1 \\ 1 & -1
            \end{bmatrix} \begin{bmatrix}
                x \\ y
            \end{bmatrix} = \begin{bmatrix}
                1 \\ 0
            \end{bmatrix}.
        \] This has the unique solution $x = y = \frac{1}{2}$.
    \end{enumerate}
    \end{solution}
    
    \paragraph{Exercise 8.} Let $A$, $B$ be $m \times n$ and $n \times m$ matrices.
    Prove that $\mathbb{I}_m - AB$ is invertible if and only if $\mathbb{I}_n - BA$
    is invertible.
    \begin{solution}
        Note that \[
            B(\mathbb{I}_m - AB) = B - BAB = (\mathbb{I}_n - BA)B,
        \]\[
            A(\mathbb{I}_n - BA) = A - ABA = (\mathbb{I}_m - AB)A.
        \] Set $X = \mathbb{I}_m - AB$, $Y = \mathbb{I}_n - BA$, whence \[
            BX = YB, \qquad AY = XA.
        \] First suppose that $X$ is invertible. If $A$ is invertible, then $AY =
        XA$ gives $Y = A^{-1}XA$, so we can check that $Y^{-1} = A^{-1}X^{-1}A$. \[
            (A^{-1}X^{-1}A)Y = A^{-1}X^{-1}A\, A^{-1}X A = \mathbb{I}_n.
        \] If $A$ is not invertible but $B$ is invertible, then use $BX = YB$ to
        write $Y = BXB^{-1}$, so we can check that $Y^{-1} = BX^{-1}B^{-1}$. \[
            (BX^{-1}B^{-1})Y = BX^{-1}B^{-1}\, BXB^{-1} = \mathbb{I}_n.
        \] Now suppose that neither $A$ nor $B$ is invertible. Consider the products
        \[
            (\mathbb{I}_n + BX^{-1}A)Y = Y + BX^{-1}AY = Y + BX^{-1}XA = Y + BA =
            \mathbb{I}_n, \] \[
            Y(\mathbb{I}_n + BX^{-1}A) = Y + YBX^{-1}A = Y + BXX^{-1}A = Y + BA =
            \mathbb{I}_n.
        \] Thus, $Y^{-1} = \mathbb{I}_n + BX^{-1}A$.
    \end{solution}
    


    \chapter{Groups}

    \section{The Definition of a Group}
    
    \paragraph{Exercise 1.} \mbox{}
    \begin{enumerate}
        \itemsep0em    
        \item Verify (1.17) and (1.18) by explicit computation.
        \item Make a multiplication table for $S_3$.
    \end{enumerate}
    \begin{solution} \mbox{}
        \begin{enumerate}
            \item We see that \[
                1 = \begin{bmatrix}
                    1 & 0 & 0 \\ 0 & 1 & 0 \\ 0 & 0 & 1
                \end{bmatrix}, \qquad
                x = \begin{bmatrix}
                    0 & 1 & 0 \\ 0 & 0 & 1 \\ 1 & 0 & 0
                \end{bmatrix}, \qquad
                y = \begin{bmatrix}
                    0 & 1 & 0 \\ 1 & 0 & 0 \\ 0 & 0 & 1
                \end{bmatrix}.
            \] Calculate \begin{align*}
                x^2 &= \begin{bmatrix}
                    0 & 1 & 0 \\ 0 & 0 & 1 \\ 1 & 0 & 0
                \end{bmatrix} \begin{bmatrix}
                    0 & 1 & 0 \\ 0 & 0 & 1 \\ 1 & 0 & 0
                \end{bmatrix} = \begin{bmatrix}
                    0 & 0 & 1 \\ 
                    1 & 0 & 0 \\
                    0 & 1 & 0
                \end{bmatrix}, \\
                xy &= \begin{bmatrix}
                    0 & 1 & 0 \\ 0 & 0 & 1 \\ 1 & 0 & 0
                \end{bmatrix} \begin{bmatrix}
                    0 & 1 & 0 \\ 1 & 0 & 0 \\ 0 & 0 & 1
                \end{bmatrix} = \begin{bmatrix}
                    1 & 0 & 0 \\
                    0 & 0 & 1 \\
                    0 & 1 & 0
                \end{bmatrix}, \\
                x^2y &= \begin{bmatrix}
                    0 & 0 & 1 \\ 
                    1 & 0 & 0 \\
                    0 & 1 & 0
                \end{bmatrix} \begin{bmatrix}
                    0 & 1 & 0 \\ 1 & 0 & 0 \\ 0 & 0 & 1
                \end{bmatrix} = \begin{bmatrix}
                    0 & 0 & 1 \\
                    0 & 1 & 0 \\
                    1 & 0 & 0
                \end{bmatrix}.
            \end{align*}
            These six matrices cover all possible permutations of the three rows.

            \item \mbox{}
            \[
                \begin{array}{c|cccccc}
                    \times  & 1     & x     & x^2   & y     & xy    & x^2y  \\\hline 
                    1       & 1     & x     & x^2   & y     & xy    & x^2y  \\
                    x       & x     & x^2   & 1     & xy    & x^2y  & y     \\
                    x^2     & x^2   & 1     & x     & x^2y  & y     & xy    \\
                    y       & y     & x^2y  & xy    & 1     & x^2   & x     \\
                    xy      & xy    & y     & x^2y  & x     & 1     & x^2   \\
                    x^2y    & x^2y  & xy    & y     & x^2   & x     & 1     
                \end{array}
            \]
        \end{enumerate}
    \end{solution}

    \paragraph{Exercise 2.} \mbox{}
    \begin{enumerate}
        \itemsep0em
        \item Prove that $GL_n(\R)$ is a group.
        \item Prove that $S_n$ is a group.
    \end{enumerate}
    \begin{solution} \mbox{}
        \begin{enumerate}
            \item We show that $GL_n(\R)$ is a group under matrix multiplication.
            Note that if $\det{A} \neq 0$ and $\det{B} \neq 0$, then $\det{AB} =
            \det{A}\det{B} \neq 0$, so $GL_n(\R)$ is closed under multiplication.
            This composition is also associative, by virtue of the associativity of
            matrix multiplication. The identity matrix $\mathbb{I}_n$ serves as the
            group identity, since $\mathbb{I}_nA = \mathbb{I}_n = A\mathbb{I}_n$ for
            all $A \in GL_n(\R)$. Finally, all non-singular matrices are invertible,
            with the inverse also being non-singular, which means that every element
            $A \in GL_n(\R)$ has an inverse $A^{-1} \in GL_n(\R)$. Thus, $GL_n(\R)$
            forms a group.

            \item We show that $S_n$ is a group under function composition. Note
            that each element $S_n$ is a bijection $f\colon \{1, \dots, n\} \to \{1,
            \dots, n\}$. Since the composition of two bijections is also a
            bijection, we see that $S_n$ is closed under composition. Also note that
            function composition is associative, with $(f\circ g) \circ h = g \circ
            (g \circ h)$. The identity map $\mathbb{I} \in S_n$ which maps each
            integer to itself serves as the identity element, since $\mathbb{I} \circ f
            = \mathbb{I} = f \circ \mathbb{I}$ for all $f \in S_n$. Finally, all
            bijections have an inverse which is also a bijection, hence every
            element $f \in S_n$ has n inverse $f^{-1} \in S_n$. Thus, $S_n$ forms a
            group.
        \end{enumerate}
    \end{solution}

    \paragraph{Exercise 3.} Let $S$ be a set with an associative law of composition
    and with an identity element. Prove that the subset of $S$ consisting of
    invertible elements is a group.
    \begin{solution}
        Let $S' \subset S$ be the set of all invertible elements of $S$. Note that
        by construction, composition in $S'$ is associative. The identity element
        $e \in S$ must also belong to $S'$, since $ee = e$, hence $e$ is invertible
        with $e^{-1} = e$. This serves as an identity for all elements in $S'$ as
        well. Also, all elements in $S'$ are invertible, and their inverses must
        also be in $S'$, because if $a^{-1} = b$, then $ba^{-1} = e = a^{-1}b$, so
        $b^{-1} = a$ making $b = a^{-1}$ invertible. Finally, $S'$ is closed under
        composition, because the product of invertible elements is invertible, with
        $(ab)^{-1} = b^{-1}a^{-1}$. This means that $S'$ is a group.
    \end{solution}
    
    \paragraph{Exercise 4.} Solve for $y$, given that $xyz^{-1}w = 1$ in a group.
    \begin{solution}
        Write \begin{align*}
            xyz^{-1}w &= 1, \\
            xyz^{-1}ww^{-1}& = w^{-1}, \\
            xyz^{-1} &= w^{-1}, \\
            xyz^{-1}z &= w^{-1}z, \\
            xy &= w^{-1}z, \\
            x^{-1}xy &= x^{-1}w^{-1}z, \\
            y &= x^{-1}w^{-1}z.
        \end{align*}
    \end{solution}
    
    \paragraph{Exercise 5.} Assume that the equation $xyz = 1$ holds in a group $G$.
    Does it follow that $yzx = 1$? That $yxz = 1$?
    \begin{solution}
        We have $xyz = 1$, so \[
            1 = x^{-1}x = x^{-1}1x = x^{-1}(xyz)x = (x^{-1}x)yzx = yzx.
        \] It is not necessarily true that $yxz = 1$. Consider the group $S_3$ as
        seen in Exercise~1. We have $(xy)(x)(y) = 1$, but $(x)(xy)(y) = x^2y^2 = x^2
        \neq 1$.
    \end{solution}

    \paragraph{Exercise 6.} Write out all ways in which one can form a product of
    four elements $a$, $b$, $c$, $d$ in the given order.
    \begin{solution}
        \[
            (ab)(cd) \quad (a(bc))d \quad ((ab)c)d \quad a(b(cd)) \quad a((bc)d)
        \] 
    \end{solution}
    
    \paragraph{Exercise 7.} Let $S$ be any set. Prove that the law of composition
    defined by $ab = a$ is associative.
    \begin{solution}
        It is sufficient to show that $(ab)c = a(bc)$ for all $a, b, c \in S$. We
        have \[
            (ab)c = ac = a, \qquad a(bc) = a.
        \] 
    \end{solution}
    
    \paragraph{Exercise 8.} Give an example of $2\times 2$ matrices such that
    $A^{-1}B \neq BA^{-1}$.
    \begin{solution}
        Consider \[
            A = \begin{bmatrix}
                1 & -1 \\ 0 & 1
            \end{bmatrix}, \qquad
            A^{-1} = \begin{bmatrix}
                1 & 1 \\ 0 & 1
            \end{bmatrix}, \qquad
            B = \begin{bmatrix}
                1 & 0 \\ 1 & 1 
            \end{bmatrix}.
        \] Now, \[
            A^{-1}B = \begin{bmatrix}
                2 & 1 \\ 1 & 1
            \end{bmatrix}, \qquad
            BA^{-1} = \begin{bmatrix}
                1 & 1 \\ 1 & 2
            \end{bmatrix}.
        \] 
    \end{solution}

    \paragraph{Exercise 9.} Show that if $ab = a$ in a group, then $b = 1$, and if
    $ab = 1$, then $b = a^{-1}$.
    \begin{solution}
        If $ab = a$, then \[
           b = 1b = (a^{-1}a)b = a^{-1}(ab) = a^{-1}a = 1. 
        \] If $ab = 1$, then it suffices to show that $ba = 1$ to conclude $b =
        a^{-1}$. \[
           ba = 1(ba) = (a^{-1}a)(ba) = a^{-1}(ab)a = a^{-1}a = 1.
        \] 
    \end{solution}

    \paragraph{Exercise 10.} Let $a$, $b$ be elements of a group $G$. Show that the
    equation $ax = b$ has a unique solution in G.
    \begin{solution}
        The existence of a solution is guaranteed by the inverse of $a$, namely
        $a^{-1} \in G$ such that $aa^{-1} = 1 = a^{-1}a$. Thus, $(a^{-1})ax =
        a^{-1}b$, hence $x = a^{-1}b$.

        Now suppose that $ax = ay = b$. Again, left multiplying by $a^{-1}$ gives $x
        = y$, guaranteeing that the solution is unique.
    \end{solution}
    
    \paragraph{Exercise 11.} Let $G$ be a group, with multiplicative notation. We
    define an \textit{opposite group} $G^\circ$ with law of composition $a \circ b$
    as follows: The underlying set is the same as $G$, but the law of composition is
    the opposite; that is, we define $a \circ b = ba$. Prove that this defines a group.
    \begin{solution}
        The composition in $G^\circ$ is closed, since $a\circ b = ba \in G$ and
        $G^\circ$ shares all elements with $G$. Note that composition is associative
        in $G$, so for all $a, b, c \in G$, \[
            c(ba) = (cb)a, \qquad (a\circ b) \circ c = a \circ (b\circ c).
        \] This shows that composition is associative in $G^\circ$. Next, the
        identity in $G$ serves as the identity in $G^\circ$, since for any $a \in
        G^\circ$, \[
            1\circ a = a1 = a = 1a = a\circ 1.
        \] Each $a \in G$ has an inverse $a^{-1} \in G$, and this guarantees that
        each $a \in G^\circ$ has the same inverse $a^{-1} \in G^\circ$, since \[
            a\circ a^{-1} = a^{-1}a = 1 = aa^{-1} = a^{-1}\circ a. 
        \] Thus, $G^\circ$ is a group.
    \end{solution}


    \section{Subgroups}
    \paragraph{Exercise 1.} Determine the elements of the cyclic group generated by
    the matrix $\begin{bmatrix}
        1 & 1 \\ -1 & 0
    \end{bmatrix}$ explicitly.
    \begin{solution}
        Set \[
            1 = \begin{bmatrix}
                1 & 0 \\ 0 & 1
            \end{bmatrix}, \qquad
            x = \begin{bmatrix}
                1 & 1 \\ -1 & 0
            \end{bmatrix}, \qquad
            x^2 = \begin{bmatrix}
                0 & 1 \\ -1 & -1
            \end{bmatrix}, \qquad
            x^3 = \begin{bmatrix}
                -1 & 0 \\ 0 & -1
            \end{bmatrix}, \qquad
        \] \[
            x^4 = \begin{bmatrix}
                -1 & -1 \\ 1 & 0
            \end{bmatrix}, \qquad
            x^5 = \begin{bmatrix}
                0 & -1 \\ 1 & 1
            \end{bmatrix}, \qquad
            x^6 = \begin{bmatrix}
                1 & 0 \\ 0 & 1
            \end{bmatrix}. 
        \] Therefore, the elements of the cyclic group generated by $x$ are the
        matrices $\{1, x, x^2, x^3, x^4, x^5\}$. This group has order 6.
    \end{solution}

    \paragraph{Exercise 2.} Let $a$, $b$ be elements of a group $G$. Assume that $a$
    has order 5 and that $a^3b = ba^3$. Prove that $ab = ba$.
    \begin{solution}
        We have $a^5 = 1$, therefore \[
            ab = 1ab = a^5ab = a^6b = a^3(a^3b) = a^3(ba^3),
        \] \[
            ba = ba1 = baa^5 = ba^6 = (ba^3)a^3 = (a^3b)a^3.
        \] These are equal by associativity.
    \end{solution}
    
    \paragraph{Exercise 3.} Which of the following are subgroups?
    \begin{enumerate}
        \itemsep0em
        \item $GL_n(\R) \subset GL_n(\C)$.
        \item $\{1, -1\} \subset \R^\times$.
        \item The set of positive integers in $\Z^+$.
        \item The set of positive reals in $\R^\times$.
        \item The set of all matrices $\begin{bmatrix}
            a & 0 \\ 0 & 0
        \end{bmatrix}$, with $a \neq 0$, in $GL_2(\R)$.
    \end{enumerate}
    \begin{solution}
        It can be shown that (a), (b), (d) meet the axioms required for a subgroup.
        In (c), the additive identity $0$ is missing. In (e), the identity matrix
        $\mathbb{I}_2$ is missing (and none of the described matrices belong to
        $GL_2(\R)$ in any case).
    \end{solution}

    \paragraph{Exercise 4.} Prove that a non-empty subset $H$ of a group $G$ is a
    subgroup if for all $x, y \in H$ the element $xy^{-1}$ is also in $H$.
    \begin{solution}
        Since $H$ is non-empty, we can pick $x \in H$. Then, $x \in H$ and $x \in
        H$, so $x x^{-1} = 1 \in H$. Next, for any $x \in H$, we have $1 \in H$ and 
        $x \in H$, so $1x^{-1} = x^{-1} \in H$. Finally, for any $x, y \in H$, we
        have $x \in H$ and $y^{-1} \in H$, so $x(y^{-1})^{-1} = xy \in H$. This
        means that $H \subseteq G$ is a subgroup.
    \end{solution}
    
    \paragraph{Exercise 5.} An $n$th root of unity is a complex number $z$ such that
    $z^n = 1$. Prove that the $n$th roots of unity form a cyclic subgroup of
    $\C^\times$ of order $n$.
    \begin{solution}
        Let the set of all $n$th roots of unity be $G$.
        First we have $1^n = 1$ so $1 \in G$. Next, if $x, y \in G$, then $x^n = y^n
        = 1$, so $(xy)^n = 1$, thus $xy \in G$. Finally, note that $x^{-1} = x^{n -
        1}$, because $x x^{n - 1} = x^{n - 1}x = x^n = 1$. To see that this is a
        cyclic subgroup, set $x = e^{1\pi i / n}$, then $G = \{1, x, \dots, x^{n -
        1}\}$. There are no other elements of $G$, since the polynomial $x^n - 1$
        has at most $n$ distinct complex roots.
    \end{solution}

    \paragraph{Exercise 6.} \mbox{}
    \begin{enumerate}
        \itemsep0em
        \item Find generators and relations analogous to (2.13) for the Klein four
        group.
        \item Find all subgroups of the Klein four group.
    \end{enumerate}
    \begin{solution} \mbox{}
        \begin{enumerate}
            \item Write \[
                e = \begin{bmatrix}
                    1 & 0 \\ 0 & 1
                \end{bmatrix}, \qquad
                x = \begin{bmatrix}
                    1 & 0 \\ 0 & -1
                \end{bmatrix}, \qquad
                y = \begin{bmatrix}
                    -1 & 0 \\ 0 & 1
                \end{bmatrix}, \qquad
                z = \begin{bmatrix}
                    -1 & 0 \\ 0 & -1
                \end{bmatrix}.
            \] It can be verified that $x^2 = y^2 = z^2 = e$. Also, $xy = z = yx$,
            $xz = y = zx$, $yz = x = zy$. Thus, this group is abelian. Also, any
            two of $x, y, z$ suffice to generate the third, and hence the entire
            Klein four group. Any one is not sufficient, since every element has
            order $2$.

            \item The Klein four group has two trivial subgroups, namely $\{e\}$ and
            the whole group. Also, each of $\{e, x\}$, $\{e, y\}$, $\{e, z\}$ form a
            subgroup, since all of them contain the identity $e$, each element is
            their own inverse, and multiplication is closed. This gives a total of
            $5$ subgroups.
        \end{enumerate}
    \end{solution}

    \paragraph{Exercise 7.} Let $a$ and $b$ be integers.
    \begin{enumerate}
        \itemsep0em
        \item Prove that the subset $a\Z + b\Z$ is a subgroup of $\Z^+$.
        \item Prove that $a$ and $b + 7a$ generate the subgroup $a\Z + b\Z$.
    \end{enumerate}
    \begin{solution} \mbox{}
        \begin{enumerate}
            \item We have \[
                G = a\Z + b\Z = \{ar + bs: r, s \in \Z\}.
            \] First note that $0 = a0 + b0 \in G$. Next, for any two elements $x, y
            \in G$, we can write $x = ar_1 + bs_1$ and $y = ar_2 + bs_2$ for
            integers $r_1, r_2, s_1, s_2$, so the sum $x + y = a(r_1 + r_2) + b(s_1
            + s_2) \in G$. Finally, for $x = ar + bs \in G$, we have $-x = a(-r) +
            b(-s) \in G$, with $x + (-x) = 0$. This proves that $G$ is a subgroup of
            $\Z^+$.

            \item Let $H$ be the group generated by $a$ and $b + 7a$. Note that $a$,
            $a + a = 2a$, $2a + a = 3a$, \dots, $6a + a = 7a$, \dots, $(n - 1)a + a
            = na$ are all in $H$. Similarly, if $na \in H$, then $-na \in H$ for all
            positive integers $n$. Thus, $0 = a + (-a) \in H$, and $b = (b + 7a) +
            (-7a) \in H$. We repeat this process with $b$ to see that $nb \in H$ for
            all integers $n$. Thus, $ar + bs \in H$ for all integers $r$ and $s$,
            which means that $H \subseteq a\Z + b\Z$. On the other hand, for all $ar
            + bs \in a\Z + b\Z$, we have $ar + bs = a(1 - 7s) + (b + 7a)s \in H$, so
            $a\Z + b\Z \subseteq H$. This establishes that $H = a\Z + b\Z$.
        \end{enumerate}
    \end{solution}

    \paragraph{Exercise 8.} Make a multiplication table for the quaternion group H.
    \begin{solution}
        Use $i^2 = j^2 = k^2 = ijk = -1$, $ij = k$, $ji = i^3j = -k$.
        \[
            \begin{array}{c|cccccccc}
                \times  & 1  & i  & j  & k  & -1 & -i & -j & -k \\\hline
                1       & 1  & i  & j  & k  & -1 & -i & -j & -k \\
                i       & i  & -1 & k  & -j & -i & 1  & -k & j  \\
                j       & j  & -k & -1 & i  & -j & k  & 1  & -i \\
                k       & k  & j  & -i & -1 & -k & -j & i  & 1  \\
                1       & -1 & -i & -j & -k &  1 &  i &  j &  k \\
                i       & -i & 1  & -k &  j &  i & -1 &  k & -j \\
                j       & -j & k  & 1  & -i &  j & -k & -1 &  i \\
                k       & -k & -j & i  & 1  &  k &  j & -i & -1 \\
            \end{array} 
        \] 
    \end{solution}

    \paragraph{Exercise 9.} Let $H$ be the subgroup generated by two elements $a$,
    $b$ of a group $G$. Prove that if $ab = ba$, then $H$ is an abelian group.
    \begin{solution}
        Since $H$ is generated by $a$ and $b$, every element in $H$ can be written
        in the form \[
            a^{m_1}b^{n_1}a^{m_2}b^{n_2} \dots a^{m_k}b^{n_k},
        \] where $k \in \N$ and $m_i, n_i \in \Z$. First, we claim that every such
        element can be simplified to the form $a^mb^m$. To see this, note that since
        $a$ and $b$ commute, we have $ab(b^{-1}a^{-1}) = 1 = ba(a^{-1}b^{-1}) =
        ab(a^{-1}b^{-1})$, so cancelling $ab$ gives $b^{-1}a^{-1} = a^{-1}b^{-1}$.
        Now, $ab = ba$ means $a = bab^{-1}$. Thus, for any positive power $a^m =
        ba^mb^{-1}$, and $a^{-m} = ba^{-m}b^{-1}$. Thus, $a^mb^n = ba^mb^{-1}b^n =
        ba^mb^{n - 1}$. Repeating this another $n - 1$ times gives $a^mb^n =
        b^na^{m}$ for positive $n$. For negative $n$, simply note that $a^m =
        b^{n}(b^{-n}a^m) = b^n(a^mb^{-n})$, hence $a^mb^n = b^na^m$.
        This means that the powers $a^m$ and $b^n$ commute. Thus, we can commute all
        powers $b^{n_i}$ in our general expression to the right, yielding \[
            a^{m_1 + \dots + m_k}b^{n_1 + \dots n_k} = a^mb^n. 
        \] Now pick arbitrary $x, y \in H$, and write $x = a^mb^n$, $y = a^rb^s$.
        Then, \[
            xy = a^mb^na^rb^s = a^m(b^n a^r)b^s = a^ma^r b^nb^s = a^{m + r}b^{n +
            s}, 
        \] \[
            yx = a^rb^sa^mb^n = a^r(b^s a^m)b^n = a^ra^m b^sb^n = a^{m + r}b^{n +
            s}. 
        \] This gives $xy = yx$ for all $x, y \in H$, which means that $H$ is abelian.
    \end{solution}

    \paragraph{Exercise 10.} \mbox{}
    \begin{enumerate}
        \itemsep0em 
        \item Assume that an element $x$ of a group has order $rs$. Find the order
        of $x^r$.
        \item Assuming that $x$ has arbitrary order $n$, what is the order of $x^r$?
    \end{enumerate}
    \begin{solution}
    \begin{enumerate}
        We first show that if $x$ has order $n$ and $x^m = 1$, then $n$ divides $m$.
        Note that $n \leq m$, since $n$ is chosen to be the least natural number
        satisfying $x^{n} = 1$. Thus, use Euclid's Division Lemma to write $m = nq +
        r$ for integers $q > 0$, $0 \leq r < n$. We now have \[
            1 = x^{m} = x^{nq + r} = (x^{n})^qx^r = x^r.
        \] Since $r < n$, the only possibility is $r = 0$, hence $m = nq$, proving
        that $n$ divides $m$.

        \item Note that $(x^{r})^s = x^{rs} = 1$, so the order of $x^r$ divides $s$.
        Also, if the order of $x^r$ were less then $s$, say $t < s$, then we would
        have $(x^r)^t = 1$, so $x^{rt} = 1$, with $rt < rs$. This would contradict
        the fact that $rs$ is the order of $x$. Thus, the order of $x^r$ must be
        $s$.

        \item We claim that the order of $x^r$ is $m = n / \gcd(n, r)$. Write
        $\gcd(n, r) = d$, so $rm = nr / d = nk$ for some integer $k$ since $d$
        divides $r$.
        Thus, \[
            (x^r)^m = x^{rm} = x^{nk} = (x^n)^k = 1.
        \] Suppose instead that the order of $x^r$ is some $m' < m$. Then $m'$ divides
        $m$, so $m = m'q$ for some integer $q > 1$. We have $1 = (x^r)^{m'} = x^{rm'}$, 
        therefore $n$ must divide $rm'$, say $rm' = nk'$ for some integer $k'$. Now,
        $nk = rm = rm'q = nk'q$, so $k = k'q$. Recall that $n = md$, $r = kd$. Now
        we have found $n = m'(qd)$, $r = k'(qd)$, thus $qd$ divides both $n$ and
        $r$. However, $qd > d$, which contradicts the fact that any common factor of
        $n$ and $r$ must divide $d$. Hence, the order of $x^r$ must be $m = n / d$.
    \end{enumerate}
    \end{solution}
    
    \paragraph{Exercise 11.} Prove that in any group the orders of $ab$ and $ba$ are
    equal.
    \begin{solution}
        Note that $ab = (b^{-1}b)ab = b^{-1}(ba)b$. It is easily seen by induction that
        $(ab)^n = b^{-1}(ba)^nb$ for all positive integers $n$. Therefore, if the
        order of $ab$ is some integer $n$, $1 = (ab)^n = b^{-1}(ba)^nb$,
        hence $(ba)^n = bb^{-1} = 1$. Thus, the order of $ba$ divides $n$. A similar
        argument using $(ba)^n = a^{-1}(ab)^na$ shows that if the order of $ba$ is
        $n'$, then the order of $ab$ divides $n'$. This forces $n = n'$ when either
        $ab$ or $ba$ has finite order.

        Note that we shown that if the order of $ab$ is finite, then the order of
        $ba$ must be finite, and vice versa. This immediately implies that if the
        order of any one of $ab$ or $ba$ is infinite, then the order of the other
        must also be infinite. Hence, the order of $ab$ and $ba$ are always the same
        in any group.
    \end{solution}
    
    \paragraph{Exercise 12.} Describe all groups $G$ which contain no proper
    subgroup.
    \begin{solution}
        Suppose that $G$ has no proper subgroups, i.e.\ the only subgroups of $G$
        are $\{1\}$ and $G$. The trivial group of one element $\{1\}$ satisfies
        this. Otherwise, let $G$ be a non-trivial group, with $x \in G$ such that $x
        \neq 1$. Then, the group generated by $x$, i.e.\ the group of elements
        $\{\dots x^{-2}, x^{-1}, 1, x, x^2, \dots\}$ is a subgroup of $G$. Since $G$
        has no proper subgroups, this forces this to be equal to $G$ itself.
        Thus, $G$ is a cyclic group.

        Suppose that $G$ has finite order, and furthermore suppose that this order
        is a composite number $ab$, where $a \geq b > 1$. Then, it can be shown that
        the group generated by $x^a$ is a proper subgroup of $G$, with $x$ not in
        this subgroup. Therefore, the order of $G$ must be prime.

        Suppose that $G$ has infinite order. Now note that the group generated by
        $x^2$ is a proper subgroup of $G$, with $x$ not in this subgroup. This is a
        contradiction, thus the order of $G$ cannot be infinite.
    \end{solution}

    \paragraph{Exercise 13.} Prove that every subgroup of a cyclic group is cyclic.
    \begin{solution}
        Let $G$ be a cyclic group generated by the single element $x$. If $x = 1$,
        then $G = \{1\}$, which has no proper subgroups. Otherwise, note that we can
        enumerate \[
            G = \{\dots x^{-2}, x^{-1}, 1, x, x^2, \dots\}.
        \] Let $H$ be a proper subgroup of $G$, and let $y \in H$ be a non-trivial
        element. This means that $y \in G$, so we can write $y = x^k$ for some
        positive integer $k$ (note that if $k$ were negative, then $y^{-1} = x^{-k}
        \in H$ too, so use $-k$ instead). Suppose that we have chosen $w = x^m \in
        H$ such that $m$ is the smallest possible, positive choice. This means that
        $m \leq k$, so using Euclid's Division Lemma, write $k = mq + r$ for $0 \leq
        r < m$. Thus, $x^{k} = (x^m)^q x^r = w^qx^r$, or $x^r = w^{-q}x^k$. Now, $w
        \in H$ means that all powers of $w$ are also in $H$, hence $w^{-q} \in H$.
        This means that $w^{-q}x^k = x^r \in H$. However, recall that $m$ was the
        smallest positive integer such that $x^m \in H$, which forces $r = 0$. Thus,
        for any $y \in H$, we see that $y = w^q$. This means that $H$ is generated
        by the element $w = x^m$, which makes it a cyclic subgroup.
    \end{solution}

    \paragraph{Exercise 14.} Let $G$ be a cyclic group of order $n$, and let $r$ be
    an integer dividing $n$. Prove that $G$ contains exactly one subgroup of order
    $r$.
    \begin{solution}
        Let $G$ be generated by the element $x$, so $G = \{1, x, \dots, x^{n -
        1}\}$, and let $H$ be a subgroup of order $r$, where $n = mr$ for some
        positive integer $m$.  We claim that the subgroup generated by $x^m$ is the
        only subgroup of order $r$. Note from the previous exercise that $H$ must be
        cyclic, and thus is generated by some element $x^k \in G$. Thus, if we pick
        $y \in H$, we must have $y = x^{kq}$ for some non-negative integer $q$.
        Since $H$ has order $r$, we require $y^r = 1$, i.e.\ $x^{kqr} = 1$.
        Now, $x$ has order $n$, hence $n$ must divide $kqr$, say $np = kqr$ for some
        positive integer $p$. Substitute $n = mr$, so $mp = kq$. Thus, $y = x^{kq} =
        x^{mp}$, so every element of $H$ is a power of $x^m$. In other words, $H
        \subseteq \{1, x^m, \dots, x^{m(r - 1)}\}$. Also note that $H$ contains
        exactly $r$ elements, and the right hand side also contains $r$ elements
        since all $x^{mp}$ are distinct. Thus, we must have an equality, which means
        that the only subgroup of $G$ with order $r$ is the one generated by $x^m$.
    \end{solution}

    \paragraph{Exercise 15.} \mbox{}
    \begin{enumerate}
        \itemsep0em
        \item In the definition of subgroup, the identity element in $H$ is required
        to be the identity of $G$. One might require only that $H$ have an identity
        element, not that it is the same as the identity in $G$. Show that if $H$
        has an identity at all, then it is the identity in $G$, so this definition
        would be equivalent to the one given.
        \item Show the analogous thing for inverses.
    \end{enumerate}
    \begin{solution} \mbox{}
    \begin{enumerate}
        \item Let $1$ be the identity element in $G$, and suppose that $1'$ is the
        identity element in $H$. Now, both $1' \in G$ and $1 \in G$, and we demand \[
            1x = x\text{ for all } x\in G, \qquad 1'x = x \text{ for all } x\in H.
        \] Combining these, we want \[
            1 = 1'1 = 1',
        \] so the identity in $H$ must be the same as the identity in $G$.
        
        \item Let $x \in H$, let $y$ be its inverse in $G$, and let $w$ be its
        inverse in $H$. We want \[
            xy = 1 = yx, \qquad xw = 1 = wx.
        \] Thus, \[
            y = y1 = y(xw) = (yx)w = 1w = w,
        \] so the inverse of $x$ in $G$ must be the same in $H$.
    \end{enumerate}
    \end{solution}

    \paragraph{Exercise 16.} \mbox{}
    \begin{enumerate}
        \itemsep0em
        \item Let $G$ be a cyclic group of order $6$. How many of its elements
        generate $G$?
        \item Answer the same question for cyclic groups of order $5$, $8$, and
        $10$.
        \item How many elements of a cyclic group of order $n$ are generators for
        that group?
    \end{enumerate}
    \begin{solution} \mbox{}
    \begin{enumerate}
        \item Let $G = \{1, x, \dots, x^5\}$. Then, the elements $x$ and $x^5$ (2
        elements) generate $G$. Note that $x^2$ and $x^4$ only generate $\{1, x^2,
        x^4\}$, and $x^3$ only generates $\{1, x^3\}$.

        \item Using the same notation as before, a cyclic group of order $5$ is
        generated by $x$, $x^2$, $x^3$, $x^4$ (4 elements). A cyclic group of order
        $8$ is generated by $x$, $x^3$, $x^5$, $x^7$ (4 elements). A cyclic group of
        order $10$ is generated by $x$, $x^3$, $x^7$, $x^9$ (4 elements).

        \item A cyclic group of order $n$ is generated by $\phi(n)$ elements, where
        the Euler totient function $\phi$ counts the number of positive integers
        less than $n$ which are co-prime with $n$. This is because a cyclic group
        $G = \{1, \dots, x^{n - 1}\}$ of order $n$ is generated by $x^r$ precisely
        when $\gcd(n, r) = 1$. Recall from Exercise~14 that if the order of a cyclic
        group $G$ is $n = ab$, then $x^a$ generates a subgroup of order $b$. Thus,
        when $b > 1$, the generated subgroup is a proper subgroup. This means that
        whenever $r$ divides $n$, $x^r$ cannot generate the entire group $G$.
        Similarly, if $r$ and $n$ share a common factor $m > 1$, then $m$ divides
        $n$ so $x^m$ cannot generate $G$, and neither can $x^r = x^{mk}$ for some
        integer $k$. This only leaves those $x^r$ such that $r$ and $n$ have no
        common factors. If so, then we can choose integers $p$ and $q$ such that
        $\gcd(n, r) = 1 = np + rq$, so $x^1 = x^{np + rq} = (x^r)^q$. Thus, any
        element $x^m \in G$ can be expressed as $(x^r)^{mq}$, so $x^r$ does indeed
        generate $G$.
    \end{enumerate}
    \end{solution}

    \paragraph{Exercise 17.} Prove that a group in which every element except the
    identity has order 2 is abelian.
    \begin{solution}
        Let $G$ be such a group, and let $x, y \in G$. Note that since every element
        has order $2$, we have $x^2 = y^2 = (xy)^2 = (yx)^2 = 1$. Also, \[
            (xy)(yx) = xy^2x = x^2 = 1, \qquad (xy)(xy) = (xy)^2 = 1.
        \] Equating and cancelling $xy$ from the left gives $yx = xy$, showing that
        $G$ is abelian.
    \end{solution} 
    
    \paragraph{Exercise 18.} \mbox{}
    \begin{enumerate}
        \itemsep0em
        \item Prove that the elementary matrices of the first and third types
        suffice to generate $GL_n(\R)$.
        \item The \textit{special linear group} $SL_n(\R)$ is the set of $n \times
        n$ matrices whose determinant is $1$. Show that $SL_n(\R)$ is a subgroup of
        $GL_n(\R)$.
        \item Use row reduction to prove that the elementary matrices of the first
        type generate $SL_n(\R)$. Do the $2 \times 2$ case first.
    \end{enumerate}
    \begin{solution} \mbox{}
    \begin{enumerate}
        \item First, we show that elementary matrices of the second type, i.e.\
        permutation matrices, can be generated by elementary matrices of the other
        two types, i.e.\ row adding and scaling. In order to transpose rows $i$ and
        $j$, consider the following row operations: $R_i \to R_i + R_j$, $R_j \to
        R_i - R_j$, $R_i \to R_i - R_j$. This can be achieved using only row
        addition and scaling, hence any transposition matrix can be obtained by
        applying this to the identity matrix. Furthermore, any permutation matrix
        can be expressed as the product of transposition matrices. This proves that
        elementary matrices of the first and third types generate those of the
        second type. Now, if $A$ is invertible, then the process of Gauss-Jordan
        elimination can be applied to reduce it to an identity matrix, hence
        $E_1\cdots E_kA = \mathbb{I}$. This means that $A = E_k^{-1} \cdots
        E_1^{-1}$, hence any matrix in $GL_n(\R)$ can be expressed as a product of
        elementary matrices. In addition, any product of elementary matrices is
        invertible, which means that the elementary matrices generate $GL_n(\R)$.

        \item First note that the identity matrix $\mathbb{I}_n \in SL_n(\R)$, since
        it has determinant $1$. Using $\det{AB} = \det{A}\det{B}$, conclude that if
        $A, B \in SL_n(\R)$, then $AB \in SL_n(\R)$. Finally, use $\det{A^{-1}} = 1 /
        \det{A}$ to conclude that if $A \in SL_n(\R)$, then $A^{-1} \in SL_n(\R)$.
        This proves that $SL_n(\R)$ is a subgroup of $GL_n(\R)$.

        \item Consider an arbitrary matrix \[
           \begin{bmatrix}
               a & b \\ c & d
           \end{bmatrix} \in SL_2(\R),
        \] where $ad - bc = 1$. Perform $R_1 \to R_1 - (b / d)R_2$ to get \[
            \begin{bmatrix}
                a - bc / d & 0 \\ c & d
            \end{bmatrix}.
        \] Note that $a - bc / d = 1 / d$. Next, perform $R_2 \to R_2 - cdR_1$. \[
            \begin{bmatrix}
                1 / d & 0 \\ 0 & d
            \end{bmatrix}.
        \] Note that if $d = 0$, we could have performed analogous operations using
        $b$ instead (both $b = d = 0$ is not possible, since we require $ad - bc =
        1$). Now, perform $R_2 \to R_2 + dR_1$. \[
            \begin{bmatrix}
                1 / d & 0 \\ 1 & d
            \end{bmatrix}.
        \] Next, perform $R_1 \to R_1 - R_2 / d$. \[
            \begin{bmatrix}
                0 & -1 \\ 1 & d
            \end{bmatrix}.
        \] Next, perform $R_2 \to R_2 + dR_1$. \[
            \begin{bmatrix}
                0 & -1 \\ 1 & 0
            \end{bmatrix}.
        \] Finally, perform $R_1 \to R_1 + R_2$, $R_2 \to R_2 - R_1$, $R_1 \to R_1 +
        R_2$. \[
            \begin{bmatrix}
                1 & 0 \\ 0 & 1
            \end{bmatrix}.
        \] Thus, we have shown that any matrix in $SL_2(\R)$ can be generated using
        only elementary matrices of the first type.

        Analogously, consider some $A = [a_{ij}] \in SL_n(\R)$. We have already seen
        that $R_i \to R_i + R_j$, $R_j \to R_j - R_i$, $R_i \to R_i + R_j$
        transposes rows with a change of sign. Thus, transpose rows such that let
        $a_{11} \neq 0$ (if all $a_{1j} = 0$, then $\det{A} = 0$). Performing $R_i
        \to R_i - (a_{i1} / a_{11})R_1$ for all $i = 2, \dots, n$ makes sure that
        all elements except the first one are zero. \[
            \begin{bmatrix}
                a_{11} & a_{12} & \cdots & a_{1n} \\
                0      & a_{22} - a_{12}a_{21} / a_{11} & \cdots & a_{2n} -
                a_{1n}a_{21} / a_{11} \\
                \vdots & \vdots & \ddots & \vdots \\
                0 & a_{n2} - a_{12}a_{n1} / a_{11} & \cdots & a_{nn} - a_{1n}a_{n1} / a_{11}
            \end{bmatrix}.
        \] Now, note that the determinant of $A$ is $a_{11} \det{M_{11}} \neq 0$, which
        means that the first column of $M_{11}$ contains a non-zero element. We thus
        repeat the above process, transposing rows if necessary $n$ times, finally
        giving us an upper triangular matrix. Relabel the entries as $b_{ij}$. \[
            \begin{bmatrix}
                b_{11} & b_{12} & \cdots & b_{1n} \\
                0 & b_{22} & \cdots & b_{2n} \\
                \vdots & \vdots & \ddots & \vdots \\
                0 & 0 & \cdots & b_{nn}
            \end{bmatrix}.
        \] Note that $b_{11} b_{22} \dots b_{nn} = 1$. Thus, for each $j = n, n-1,
        \dots, 2$, perform $R_i \to R_i - (b_{ij} / b_{jj})R_j$ for all $i = 1,
        \dots, j - 1$. This yields a diagonal matrix, \[
            \begin{bmatrix}
                b_{11} & 0 & \cdots & 0\\
                0 & b_{22} & \cdots & 0\\
                \vdots & \vdots & \ddots & \vdots \\
                0 & 0 & \cdots & b_{nn}
            \end{bmatrix}.
        \] Now, look at the upper left $2\times 2$ block. Performing $R_2 \to R_2 +
        R_1 / b_{11}$, $R_1 \to R_1 - b_{11}R_2$, $R_2 \to R_2 + R_1 / b_{11}$
        yields \[
            \begin{bmatrix}
                0 & -b_{11}b_{22} & \cdots & 0\\
                1 & 0 & \cdots & 0\\
                \vdots & \vdots & \ddots & \vdots \\
                0 & 0 & \cdots & b_{nn}
            \end{bmatrix}.
        \] Performing our transposition with sign change yields \[
            \begin{bmatrix}
                1 & 0 & \cdots & 0\\
                0 & b_{11}b_{22} & \cdots & 0\\
                \vdots & \vdots & \ddots & \vdots \\
                0 & 0 & \cdots & b_{nn}
            \end{bmatrix}.
        \] Repeating this down the line, we see that $b_{11}b_{22} \dots b_{nn}$
        bunches up at the lower right, but this is just $1$, so we get the identity
        matrix. Thus, $SL_{n}(\R)$ is generated by elementary matrices of the first
        type.
    \end{enumerate}

    \paragraph{Exercise 19.} Determine the number of elements of order $2$ in the
    symmetric group $S_4$.
    \begin{solution}
        Use the notation $(a\; b\; c\; d)$ to denote a cycle $a\rightsquigarrow b$,
        $b \rightsquigarrow c$, $c \rightsquigarrow d$, $d \rightsquigarrow a$.
        Note that any transposition of two elements has order $2$, hence all
        two-cycles \[
            (1\;2)\quad (1\;3)\quad (1\;4)\quad (2\;3)\quad (2\;4)\quad (3\;4)
        \] have order $2$. If any two of them commute, then their product will also
        be of order $2$ (Exercise~11), therefore the products of disjoint two-cycles
        \[
            (1\;2)(3\;4)\quad (1\;3)(2\;4)\quad (1\;4)(2\;3)
        \] also have order $2$. This gives a total of $9$ elements.
    \end{solution}

    \paragraph{Exercise 20.} \mbox{}
    \begin{enumerate}
        \itemsep0em
        \item Let $a$, $b$ be elements of an abelian group of orders $m$, $n$
        respectively. What can you say about the order of their product $ab$?
        \item Show by example that the product of elements of finite order in a
        non-abelian group need not have finite order.
    \end{enumerate}
    \begin{solution} \mbox{}
    \begin{enumerate}
        \item Set $d = \gcd(m, n)$. The order of $ab$ will always divide $k = mn / d$.
        Note that in an abelian group, \[
            (ab)^k = a^kb^k = a^{mn / d}b^{mn / d} = (a^m)^{n / d} (b^n)^{m / d} =
            1.
        \] Note that the order of $ab$ is not always $k$. Consider $a = b = i \in
        \C^\times$, and note that the orders of $a$ and $b$ are $4$. However, the
        order of their product $i^2 = -1$ is $2$.

        \item Select the following elements from $GL_2(\R)$. \[
            A = \begin{bmatrix}
                -1 & 1 \\ 0 & 1
            \end{bmatrix}, \qquad
            B = \begin{bmatrix}
                -1 & -1 \\ 0 & 1
            \end{bmatrix}.
        \] Note that $A^2 = B^2 = \mathbb{I}_2$. However, \[
            AB = \begin{bmatrix}
                1 & 2 \\ 0 & 1
            \end{bmatrix}, \qquad
            (AB)^2 = \begin{bmatrix}
                1 & 4 \\ 0 & 1
            \end{bmatrix}, \qquad
            (AB)^3 = \begin{bmatrix}
                1 & 6 \\ 0 & 1
            \end{bmatrix}, \qquad
            \dots
        \] In general for $n > 0$, we have \[
            (AB)^n = \begin{bmatrix}
                1 & 2n \\ 0 & 1
            \end{bmatrix}.
        \] Thus, $AB$ has infinite order.
    \end{enumerate}
    \end{solution}

    \paragraph{Exercise 21.} Prove that the set of elements of finite order in an
    abelian group is a subgroup.
    \begin{solution}
        Let $G$ be abelian, and let $H$ be the set of all elements with finite
        order. Note that $1 \in H$. Also, if $a$ and $b$ have finite order, so does
        $ab$ by the previous exercise. Finally, if $a$ has finite order $n$, then
        $a^n = 1$ forces $1 = a^{-n} = (a^{-1})^n$, hence $a^{-1}$ has finite order
        $n$. Thus, $H$ is a subgroup of $G$.
    \end{solution}

    \paragraph{Exercise 22.} Prove that the greatest common divisor of $a$ and $b$,
    as defined in the text, can be obtained by factoring $a$ and $b$ into primes and
    collecting the common factors.
    \begin{solution}
        Suppose that \[
            a = p_1^{\alpha_1}p_2^{\alpha_2} \cdots p_n^{\alpha_n} \cdots, \qquad
            b = p_1^{\beta_1}p_2^{\beta_2} \cdots p_n^{\beta_n} \cdots,
        \] where $p_i$ are the prime numbers, and $\alpha_i, \beta_i$ are
        non-negative integers.  We have $\alpha_{i \geq M} = \beta_{i \geq N} = 0$
        eventually, for $a$ and $b$ to be finite. Setting $\gamma_i =
        \min\{\alpha_i\, \beta_i\}$, we claim that the greatest common factor of $a$
        and $b$ is \[
            d = p_1^{\gamma_1}p_2^{\gamma_2} \cdots p_n^{\gamma_n} \cdots.
        \] First note that $d$ divides both $a$ and $b$, therefore $d$ must divide
        $d' = \gcd(a, b)$. Write \[
            d' / d = p_1^{\gamma_1'}p_2^{\gamma_2'}\cdots p_n^{\gamma_n'}\cdots.
        \] Suppose that $\gamma_k' > 0$ for some $k$. This means that the power of
        $p_k$ in $d'$ is $\gamma_k + \gamma_k' > \min\{\alpha_k, \beta_k\}$. This
        means that $d'$ cannot divide one or more of $a$ and $b$, so $\gamma_i' = 0$
        for all $i$. Thus, $d = d'$.
    \end{solution}

    

    \section{Isomorphisms}

    \paragraph{Exercise 1.} Prove that the additive group $\R^+$ of real numbers is
    isomorphic to the multiplicative group $P$ of positive reals.
    \begin{solution}
        We construct the isomorphism $\varphi\colon \R^+ \to P$, $x \mapsto e^x$.
        This is a homomorphism, because $\phi(0) = e^0 = 1$, and \[
            \varphi(x + y) = e^{x + y} = e^xe^y = \varphi(x)\,\varphi(y).
        \] This map is injective because for any $\varphi(x) = \varphi(y)$, we have
        $e^x = e^y$ or $e^{x - y} = 0$, hence $x - y = 0$ or $x = y$. This map is
        also surjective because $\varphi^{-1}(x) = \log(x)$ is well defined for all
        positive reals. Hence, $\varphi$ is a bijection between $\R^+$ and $P$
        respecting their laws of composition, which proves that they are isomorphic.
    \end{solution}
    
    \paragraph{Exercise 2.} Prove that the elements $ab$ and $ba$ are conjugate
    elements in a group.
    \begin{solution}
        Note that \[
            ab = (ab)aa^{-1} = a(ba)a^{-1}, \qquad
            ba = (ba)bb^{-1} = b(ab)b^{-1}.
        \] 
    \end{solution}
    
    \paragraph{Exercise 3.} Let $a$, $b$ be elements of a group $G$, and let $a' =
    bab^{-1}$. Prove that $a = a'$ is and only if $a$ and $b$ commute.
    \begin{solution}
        First, suppose that $a = a'$. This means that $a = b a ab^{-1}$, or $ab =
        ba$, hence $a$ and $b$ commute.

        Next, suppose that $a$ and $b$ commute. Then, $a' = ba a b^{-1} = abb^{-1} =
        a$.
    \end{solution}

    \paragraph{Exercise 4.} \mbox{}
    \begin{enumerate}
        \itemsep0em
        \item Let $b' = aba^{-1}$. Prove that $b'^n = a b^na^{-1}$.
        \item Prove that if $aba^{-1} = b^2$, then $a^3ba^{-3} = b^8$.
    \end{enumerate}
    \begin{solution} \mbox{}
    \begin{enumerate}
        \item The claim for $n \geq 1$ follows from induction. Note that the case $n
        = 1$ is true, and if it holds for some $k$ with $b'^k = ab^ka^{-1}$, then \[
            b'^{k + 1} = b'^k' = (ab^ka^{-1})(aba^{-1}) = ab^k(a^{-1}a)ba^{-1} =
            ab^{k + 1}a^{-1},
        \] thus proving the claim for all $n \geq 1$.
        Now, for any $n \leq -1$, note that $b^{-n} = ab^{-n}a^{-1}$ by the first
        part, hence taking inverses gives $b^n = ab^{n}a^{-1}$. This establishes the
        formula for all $n \in \Z$.

        \item We have been given the rule $ab = b^2a$. Thus, \[
            a^3ba^{-3} = a^2(ab)a^{-3} = a^2(b^2a)a^{-3} = a^2b^2a^{-2}.
        \] Proceeding in this fashion, this is the same as \[
            a(ab)ba^{-2} = a(b^2a)ba^{-2} = ab^2(ab)a^{-2} = ab^2(b^2a)a^{-2} =
            ab^4a^{-1}.
        \] Finally, this is the same as \[
            (ab)b^3a^{-1} = (b^2a)b^3a^{-1} = b^2(ab)b^2a^{-1} = b^2(b^2a)b^2a^{-1} =
        \] \[
            b^4(ab)ba^{-1} = b^4(b^2a)ba^{-1} = b^6(ab)a^{-1} = b^6(b^2a)a^{-1} =
            b^8.
        \] 


        Alternatively, note that $b^2 = aba^{-1}$, so \[
            b^8 = (b^2)^4 = ab^4a^{-1}, \qquad
            b^4 = (b^2)^2 = ab^2a^{-1}, \qquad
            b^2 = (b^2)^1 = aba^{-1}.
        \] Substituting each of these into the equation above gives \[
            b^8 = ab^4a^{-1} = a(ab^2a^{-1})a^{-1} = a(a(aba^{-1})a^{-1})a^{-1} =
            a^3ba^{-3}.
        \] 
    \end{enumerate}
    \end{solution}
    
    \paragraph{Exercise 5.} Let $\varphi\colon G \to G'$ be an isomorphism of groups.
    Prove that the inverse function $\varphi^{-1}$ is also an isomorphism.
    \begin{solution}
        First note that $\varphi$ is a bijection, which means that $\varphi^{-1}$ is
        also a bijection. Let $a', b' \in G'$ be arbitrary. We set $a =
        \varphi^{-1}(a')$, $b = \varphi^{-1}(b')$. Since \[
            \varphi(ab) = \varphi(a)\varphi(b),
        \] we have \[
            \varphi(ab) = a'b', \qquad \varphi^{-1}(a') \varphi^{-1}(b') = ab =
            \varphi^{-1}(a'b').
        \] This proves that $\varphi^{-1}\colon G' \to G$ is an isomorphism.
    \end{solution}

    \paragraph{Exercise 6.} Let $\varphi\colon G \to G'$ be an isomorphism of groups,
    let $x, y \in G$ and let $x' = \varphi(x)$ and $y' = \varphi(y)$.
    \begin{enumerate}
        \itemsep0em
        \item Prove that the orders of $x$ and $x'$ are equal.
        \item Prove that if $xyx = yxy$, then $x'y'x' = y'x'y'$.
        \item Prove that $\varphi(x^{-1}) = x'^{-1}$.
    \end{enumerate}
    \begin{solution}
        Recall that if $e$ and $e'$ are the identity elements in $G$ and $G'$
        respectively, then $\varphi(e) = e'$. This is because $ee = e$, so
        $\varphi(e)\varphi(e) = \varphi(e)$, hence cancellation gives $\varphi(e) =
        e'$.
        \begin{enumerate}
            \item Suppose that the orders of $x$ and $x'$ are $m$ and $n$
            respectively, hence $x^m = e$ and $x'^n = e'$. Note that $\varphi(x^2) =
            \varphi(x)\varphi(x) = x'^2$, and by repeating this process, it can be
            shown that $\varphi(x^k) = x'^{k}$ for any $k \geq 1$. This means that
            if $x^m = e$, we have $x'^m = \varphi(x^m) = \varphi(e) = e'$, hence $m
            \geq n$. Also, $\varphi^{-1}$ is also an isomorphism, so $x^n =
            \varphi^{-1}(x'^{n}) = \varphi^{-1}(e') = e$, hence $n \geq m$.
            Together, this implies that $m = n$.

            Note that if $m$ were infinite, this would force $n$ to be infinite as
            well, and vice versa.

            \item Write \[
                \varphi(xyx) = \varphi(xy)\varphi(x) =
                \varphi(x)\varphi(y)\varphi(x) = x'y'x'.
            \] Similarly, $\varphi(yxy) = y'x'y'$. Equating the two gives the
            desired result.

            \item Use the fact that $x x^{-1} = e$, so $\varphi(x)\varphi(x^{-1}) =
            \varphi(x x^{-1}) = \varphi(e) = e'$. This means $x' \varphi(x^{-1}) =
            e'$, which forces $\varphi(x^{-1}) = x'^{-1}$.
        \end{enumerate}
    \end{solution}
    

    
    \paragraph{Exercise 9.} Find an isomorphism from a group $G$ to its opposite
    group $G^\circ$.
    \begin{solution}
        Define $\varphi\colon G \to G^\circ$, $x \mapsto x^{-1}$. This is clearly a
        bijection because every element $x$ has exactly one inverse. Observe that 
        for any $x, y \in G$, \[
            \varphi(xy) = (xy)^{-1}= y^{-1}x^{-1} = \varphi(y)\varphi(x) =
            \varphi(x)\circ \varphi(y),
        \] as desired. Recall the nature of composition $\circ$ in $G^\circ$.
    \end{solution}
    
    \paragraph{Exercise 10.} Prove that the map $A \rightsquigarrow (A^t)^{-1}$ is
    an automorphism of $GL_n(\R)$.
    \begin{solution}
        First note that this map is a bijection, because every matrix in $GL_n(\R)$
        has exactly one inverse, and every matrix in $GL_n(\R)$ has exactly one
        transpose in $GL_n(\R)$. Also observe that for $A, B \in GL_n(\R)$, \[
            ((AB)^t)^{-1} = (B^tA^t)^{-1} = (A^t)^{-1}(B^t)^{-1},
        \] which proves that the given map is an automorphism.
    \end{solution}

    \paragraph{Exercise 11.} Prove that the set $\aut(G)$ of automorphisms of a
    group $G$ forms a group, the law of composition being the composition of
    functions.
    \begin{solution}
        Note that the identity map is trivially an automorphism. The composition of
        functions is inherently associative, and the composition of two
        automorphisms is another automorphism - suppose that $\varphi$ and $\psi$
        are automorphisms. Then, $\psi \circ \varphi$ is a bijection, with \[
            (\psi\circ\varphi)(xy) = \psi(\varphi(xy)) = \psi(\varphi(x)\,\varphi(y))
            = \psi(\varphi(x))\,\psi(\varphi(y)) =
            (\psi\circ\varphi)(x)\,(\psi\circ\varphi)(y).
        \] Finally, the inverse of an automorphism is also an automorphism. This
        shows that $\aut(G)$ forms a group.
    \end{solution}

    \paragraph{Exercise 12.} Let $G$ be a group and let $\varphi\colon G \to G$ be
    the map $\vaprhi(x) = x^{-1}$.
    \begin{enumerate}
        \itemsep0em 
        \item Prove that $\varphi$ is bijective.
        \item Prove that $\varphi$ is an automorphism if and only if $G$ is abelian.
    \end{enumerate}
    \begin{solution}
    \begin{enumerate}
        \item The map $\varphi$ is a bijection because every element $x$ in $G$ has
        a unique inverse.

        \item If $G$ is abelian, note that \[
            \varphi(xy) = (xy)^{-1} = y^{-1}x^{-1} = x^{-1}y^{-1} =
            \varphi(x)\varphi(y).
        \]

        If $\vaprhi$ is to be an automorphism, we have \[
            \varphi(x^{-1}y^{-1}) = (x^{-1}y^{-1})^{-1} = yx.
        \] for all $x, y \in G$. We also demand \[
            \varphi(x^{-1}y^{-1}) = \varphi(x^{-1})\varphi(y^{-1}) = xy,
        \] hence equating the two gives $xy = yx$, proving that $G$ is abelian.
    \end{enumerate}
    \end{solution}

    
    \paragraph{Exercise 13.}\mbox{}
    \begin{enumerate}
        \itemsep0em
        \item Let $G$ be a group of order $4$. Prove that every element has order
        $1$, $2$, or $4$.
        \item Classify groups of order $4$ by considering the following cases.
        \begin{enumerate}
            \itemsep0em
            \item $G$ contains an element of order $4$.
            \item Every element of $G$ has order $<4$.
        \end{enumerate}
    \end{enumerate}
    \begin{solution} \mbox{}
    \begin{enumerate}
        \item Note that every $x \in G$ obeys $x^4 = 1$, so the order of every
        element of $G$ must be either $1, 2, 3$ or $4$. Suppose that $x^3 = 1$.
        Writing $x^3 = 1 = x^4$ gives $x = 1$. Thus, no element of $G$ can have
        order $3$.

        In any group of $n$ elements, observe that every element obeys $x^n = 1$. To
        see this, consider the set $\{1, x, \dots, x^n\}$. All of these elements
        cannot be distinct since that would yield $n + 1$ elements in a group of $n$
        elements. Thus, we have $x^p = x^q$ for some $0 \leq p < q \leq n$, giving
        $x^{q - p} = 1$, so the order of $x$ is less than or equal to $n$.
        Suppose that $x$ has order $m \leq q - p$; write $n = am + b$ for $0 \leq b
        < m$. Now, $x^n = (x^m)^ax^b = x^b$, so the minimality of $m$ forces $b =
        0$. This yields $x^n = 1$.

        \item Let the elements of $G$ be $\{1, a, b, c\}$, where $a, b, c$ are distinct.
        Suppose that $a$ has order $4$. Now, $b$ and $c$ must have order $2$ or $4$.
        Now, $a^2 \neq 1$ and $a^2 \neq a$, so without loss of generality set $a^2 =
        b$. Thus, $b^2 = a^4 = 1$, so $b$ has order $2$. We now have $a^{-1} \neq a$
        since that would imply $a^2 = 1$, and also $a^{-1} \neq b$ because that
        would also give $a^{-2} = b^2 = 1$ hence $a^2 = 1$. Thus, $a^{-1} = c =
        a^3$, so $c$ also has order $4$. This means that $G$ is the cyclic group
        $\{1, a, a^2, a^3\}$.

        Now, suppose that no element has order $4$. None of $a, b, c$ can have order
        $1$, hence each of them has order $2$, with $a^2 = b^2 = c^2 = 1$. Examine
        $ab$, and note that $ab = b$ would imply $a = 1$, $ab = a$ would imply $b =
        1$. This forces $ab = c$. The same argument also forces $ba = c$, and
        similarly $ac = ca = b$, $bc = cb = a$. Thus, $G$ is the Klein four group
        $\{1, a, b, ab\}$.
    \end{enumerate}
    \end{solution}

    \paragraph{Exercise 14.} Determine the group of automorphisms of the following
    groups.
    \begin{enumerate}
        \itemsep0em
        \item $\Z^+$.
        \item A cyclic group of order $10$.
        \item $S_3$.
    \end{enumerate}


    \paragraph{Exercise 15.} Show that the functions $f = 1 / x$, $g = (x - 1) / x$
    generate a group of functions, the law of composition being composition of
    functions, which is isomorphic to the symmetric group $S_3$.
    \begin{solution}
        Observe that $f^2(x) = x$, $g^2(x) = ((x - 1) / x - 1) / ((x - 1) / x) = - 1
        / (x - 1)$, $g^3(x) = -1 / ((x - 1) / x - 1) = x$. This gives $fg(x) =
        (f\circ g)(x) = x / (x - 1)$, and $fg^2(x) = (f\circ g^2)(x) = 1 - x$.

        Next, $gf(x) = (1 / x - 1) / (1 / x) = 1 - x = fg^2(x)$, and $g^2f(x) = ((1
        - x) - 1) / (1 - x) = x / (1 - x) = fg(x)$. This means that compositions
        like $fgfg = f(gf)g = f(fg^2)g = f^2g^3 = 1$. Thus, the elements $f$ and $g$
        generate the elements $1, g, g^2, f, fg, fg^2$ which is closed under
        composition.

        Our group consists of the functions $\{1, g, g^2, f, fg, fg^2\}$. The map
        from this set to $S_3$, defined as $f \rightsquigarrow (1\,2)$ and $g
        \rightsquigarrow (1\, 2\, 3)$ is thus an isomorphism, because it is
        bijective, and the multiplication table of our given elements is identical
        to that of $S_3$.
    \end{solution}
    
    \paragraph{Exercise 16.} Give an example of two isomorphic groups such that
    there is more than one isomorphism between them.   
    \begin{solution}
        Recall that we constructed an isomorphism between the additive group $\R^+$
        of real numbers and the multiplicative group $P$ of positive reals, by
        considering the map $x \mapsto e^{x}$. Another isomorphism is defined by the
        map $x \mapsto e^{-x}$.


        Note that this map is motivated by the fact that the reflection
        $\varphi\colon \Z^{+} \to \Z^{+}$, $x \mapsto -x$ is an automorphism.
    \end{solution}


    \section{Homomorphisms}
    
    \paragraph{Exercise 1.} Let $G$ be a group, with law of composition written $x
    \,\#\, y$. Let $H$ be a group with law of composition $u \circ v$. What is the
    condition for a map $\varphi\colon G \to H$ to be a homomorphism?
    \begin{solution}
        We demand that for all $x, y \in G$, \[
            \varphi(x \,\#\, y) = \varphi(x) \circ \varphi(y).
        \] 
    \end{solution}

    \paragraph{Exercise 2.} Let $\varphi\colon G \to G'$ be a group homomorphism.
    Prove that for any elements $a_1, \dots, a_k$ of $G$, \[
        \varphi(a_1\cdots a_k) = \varphi(a_1)\cdots \varphi(a_k).
    \] 
    \begin{solution}
        This follows from induction, the base case $k = 2$ following from the
        definition of a homomorphism. If this formula holds for some $n \geq 2$,
        then \[
            \varphi(a_1\cdots a_n a_{n + 1}) = \varphi(a_1\cdots a_n)\,\varphi(a_n) =
            \varphi(a_1)\cdots \varphi(a_n)\,\varphi(a_{n + 1}).
        \] 
    \end{solution}

    \paragraph{Exercise 3.} Prove that the kernel and image of a homomorphism are
    subgroups.
    \begin{solution}
        Let $\varphi\colon G \to H$ be a homomorphism. The kernel of $\varphi$ is the set
        of all elements $x \in G$ such that $\varphi(x) = 1_H$. The image of
        $\varphi$ is the set of all elements $u \in H$ such that $u = \varphi(x)$
        for some $x \in G$. Denote the kernel as $K \subseteq G$, and the image as
        $I \subseteq H$. It suffices to show that $K < G$ and $I < H$ are subgroups.

        Note that $1_G \in K$ because a homomorphism must send the identity of $G$
        to the identity of $H$. Note that $\varphi(1_G) = \varphi(1_G 1_G) =
        \varphi(1_G)\varphi(1_G)$, and cancellation gives $\varphi(1_G) = 1_H$.
        Next, if $x, y \in K$, then $\varphi(xy) = \varphi(x)\varphi(y) = 1_H 1_H =
        1_H$, hence $xy \in K$. Finally, if $x \in K$, then $1_H = \varphi(1_G) =
        \varphi(x x^{-1}) = \varphi(x)\varphi(x^{-1}) = \varphi(x^{-1})$, so
        $x^{-1}$ in $K$. Thus, the kernel of $\varphi$ is a subgroup of $G$.

        Note that $1_H \in I$, because $\varphi(1_G) = 1_H$. If $u, v \in I$, then
        $u = \varphi(x)$ and $v = \varphi(y)$ for some $x, y \in G$, so $uv =
        \varphi(x)\varphi(y) = \varphi(xy)$. This means that $uv \in I$. Finally, if
        $u \in I$ with $u = \varphi(x)$, then $1_H = \varphi(1_G) = \varphi(x
        x^{-1}) = \varphi(x)\varphi(x^{-1}) = u \varphi(x^{-1})$, so $u^{-1} =
        \varphi(x^{-1})$ which means $u^{-1} \in I$. Thus the image of $\varphi$ is
        a subgroup of $H$.
    \end{solution}
    
    \paragraph{Exercise 4.} Describe all homomorphisms $\varphi\colon \Z^+ \to
    \Z^+$, and determine which are injective, which are surjective, and which are
    isomorphisms.
    \begin{solution}
        Note that for any such homomorphism, $\varphi(0) = 0$. Since $\Z^+$ is
        generated by $1$, the homomorphism $\varphi$ is fully determined by
        $\varphi(1)$. For any $n > 0$, write $n = 1 + \dots + 1$, hence $\varphi(n)
        = \varphi(1) + \dots + \varphi(1) = n\varphi(1)$. Thus, $\varphi(0) =
        \varphi(n + (-n)) = \varphi(n) + \varphi(-n)$, so $\varphi(-n) = -\varphi(n)
        = -n\varphi(1)$.

        If $\varphi(1) = 0$, then $\varphi(n = 0)$ for all $n$; this homomorphism is
        not injective, nor surjective. If $\varphi(1) = 1$ or $\varphi(1) = -1$,
        then $\varphi(n) = n$ or $\varphi(n) = -n$ respectively; these homomorphisms
        are isomorphisms, since they are both injective and surjective.

        If $\varphi(1) > 1$, this homomorphism is injective, because if $\varphi(m)
        = \varphi(n)$, then $\varphi(m - n) = 0$ or $(m - n)\varphi(1) = 0$, which
        is possible only if $m - n = 0$. This is not surjective however, since there
        is no $n$ such that $\varphi(n) = 1$. If $\varphi(n) = 1$, then note that
        $n\varphi(1) = 1$, which would require $\varphi(1) = 1 / n \notin \Z$. The
        same applies whenever $\varphi(1) < -1$.
    \end{solution}

    \paragraph{Exercise 5.} Let $G$ be an abelian group. Prove that the $n$th power
    map $\varphi\colon G \to G$ defined by $\varphi(x) = x^n$ is a homomorphism from
    $G$ to itself.
    \begin{solution}
        Note that for any $x, y \in G$, we have \[
            \varphi(xy) = (xy)^n = x^ny^n = \varphi(x)\, \varphi(y),
        \] which shows that $\varphi$ preserves the group structure, and is hence a
        homomorphism. Note that we have used the commutativity of elements in $G$ to
        conclude that $(xy)^n = x^ny^n$.
    \end{solution}

    \paragraph{Exercise 6.} Let $f\colon \R^+ \to \C^\times$ be the map $e^{ix}$.
    Prove that $f$ is a homomorphism, and determine its kernel and image.
    \begin{solution}
        For any $x, y \in \R$, note that \[
            f(x + y) = e^{i(x + y)} = e^{ix}e^{iy} = f(x)\,f(y),
        \] which establishes $f$ as a homomorphism.

        Note that the identity in $\C^\times$ is $1$; the solutions of $e^{ix}$ for
        real $x$ are precisely $2\pi k$, for $k \in \Z$. This set forms the kernel
        of $f$. Also, $f$ maps the interval $[0, 2\pi)$ precisely to the unit circle
        in $\C$, which thus forms the image of $f$.  There are no other points in
        the image of $f$, since $|f(x)| = |e^{ix}| = 1$ for all real $x$.
    \end{solution}
    
    \paragraph{Exercise 7.} Prove that the absolute value map $|~ ~|\colon
    \C^\times \to \R^\times$ sending $\alpha \rightsquigarrow |\alpha|$ is a
    homomorphism, and determine its kernel and image.
    \begin{solution}
        For any $\alpha, \beta \in \C$, note that \[
            |\alpha\beta| = |\alpha| |\beta|,
        \] which establishes the absolute value map as a homomorphism.

        The pre-image of $1$, the identity in $\R^\times$, is the unit circle in
        $\C$ described by $\{e^{ix}\colon x \in \R\}$. This is the kernel of the
        absolute value map. The image of this map is the entirety of $\R$, since
        $x = |x|$ for all $x \in \R$.
    \end{solution}

    \paragraph{Exercise 8.} \mbox{}
    \begin{enumerate}
        \itemsep0em 
        \item Find all subgroups of $S_3$, and determine which are normal.
        \item Find all subgroups of the quaternion group, and determine which are
        normal.
    \end{enumerate}
    \begin{solution} \mbox{}
    \begin{enumerate}
        \item Let $x$ denote the transposition $(1\, 2)$ and $y$ denote the cycle
        $(1\, 2\, 3)$. The elements of $S_3$ are $1, x, y, xy, y^2, xy^2$, with $x^2
        = 1$ and $y^3 = 1$. Now, $x$ generates the subgroup $\{1, x\}$ and $y$ or
        $y^2$ generate the subgroup $\{1, y, y^2\}$. Note that $x$ and $y$ together
        generate the whole of $S_3$. If a subgroup contains any of $xy, xy^2$ along
        with $x$, then $x(xy) = y$ and $x(xy^2) = y^2$ mean that $y$ is also in the
        subgroup, hence the subgroup is $S_3$ itself. Similarly, if a subgroup
        contains any of $xy, xy^2$ along with $y$, then $(xy)yy = x$ and $(xy^2)y =
        x$ mean that $y$ is also in the subgroup, again forcing the subgroup to be
        $S_3$. Also note that $xy$ generates the subgroup $\{1, xy\}$ and $xy^2$
        generates the subgroup $\{1, xy^2\}$. If a subgroup contains both $xy$ and
        $xy^2$, then it also contains $(xy)^{-1} = y^2x$ hence $(y^2x)(xy^2) = y$
        and $(xy^2)y = x$, which forces it to be equal to $S_3$.
        Thus, we have found all subgroups of $S_3$, namely $\{1\}$, $\{1,
        x\}$, $\{1, xy\}$, $\{1, xy^2\}$, $\{1, y, y^2\}$, and $S_3$.

        Consider the subgroup $\{1, x\}$. Note that $yxy^{-1} = yxy^2 = (xy^2)y^2 =
        xy \notin \{1, x\}$. Similarly for the other subgroups of order $2$,
        $y(xy)y^{-1} = yxy y^2 = yx = xy^2$, and $y(xy^2)y^{-1} = yxy = xy^2y = x$.
        Thus, none of these are normal subgroups. Looking at $\{1, y, y^2\}$,
        consider the homomorphism $\varphi\colon S_3 \to S_3$ defined by $\varphi(x)
        = x$, $\varphi(y) = 1$. This gives $\varphi(y^2) = 1$, $\varphi(xy) = x$,
        $\varphi(xy^2) = x$, hence $\{1, y, y^2\}$ is the kernel of the homomorphism
        $\varphi$, making it a normal subgroup. This is the only normal subgroup
        besides $\{1\}$ and $S_3$.

        \item The quaternion group contains the elements $1, i, j, k, -1, -i, -j,
        -k$. The only non-trivial subgroups are $\{1, -1\}$, $\{1, i, -1, -i\}$,
        $\{1, j, -1, -j\}$ and $\{1, k, -1, -k\}$. This is because any two other
        elements like $i$ and $j$ can generate the third, $ij = k$. The subgroup
        $\{1, -1\}$ is normal since $x(-1)x^{-1} = -x x^{-1} = -1$ for all other
        choices of $x$. It suffices to show that $\{1, i, -1, -i\}$ is normal to
        prove that all the subgroups are normal. We can check that $jij^{-1} =
        ji(-j) = -jij = -jk = -i$, $kik^{-1} = ki(-k) = -kik = -k(-j) = -i$, which
        is sufficient to show that $\{1, i, -1, -i\}$ is normal.
    \end{enumerate}
    \end{solution} 
    
    \paragraph{Exercise 9.} \mbox{}
    \begin{enumerate}
        \itemsep0em 
        \item Prove that the composition $\varphi\circ \psi$ of two homomorphisms
        $\varphi, \psi$ is a homomorphism.
        \item Describe the kernel of $\varphi\circ\psi$.
    \end{enumerate}
    \begin{solution} \mbox{}
    \begin{enumerate}
        \item Note that for all elements $x, y$, we have \[
            (\varphi\circ\psi)(xy) = \varphi(\psi(x)\psi(y)) =
            \varphi(\psi(x))\varphi(\psi(y)) = (\varphi\circ \psi)(x)\,
            (\varphi\circ \psi)(y).
        \] 
        \item The kernel of $\varphi\circ \psi$ is by definition
        $\psi^{-1}(\varphi^{-1}(1))$, where $1$ is the identity in the image of
        $\varphi \circ \psi$. This contains the kernel of $\psi$, since $\psi(x) = 1
        \implies \varphi(\psi(x)) = 1$.
    \end{enumerate}
    \end{solution}
    
    \paragraph{Exercise 10.} Let $\varphi\colon G \to G'$ be a group homomorphism.
    Prove that $\varphi(x) = \varphi(y)$ if and only if $xy^{-1} \in \ker{\varphi}$.
    \begin{solution}
        Recall that $\varphi(y^{-1}) = \varphi(y)^{-1}$. Suppose that $\varphi(x) =
        \varphi(y)$. Then, $\varphi(xy^{-1}) = \varphi(x)\varphi(y^{-1}) =
        \varphi(y)\varphi(y)^{-1} = 1$, so $xy^{-1} \in \ker{\varphi}$.

        Next, suppose that $xy^{-1} \in \ker{\varphi}$, i.e.\ $\varphi(xy^{-1}) =
        1$. This gives $\varphi(x)\varphi(y)^{-1} = 1$, hence $\varphi(x) =
        \varphi(y)$.
    \end{solution}

    \paragraph{Exercise 11.} Let $G$, $H$ be cyclic groups, generated by the
    elements $x$, $y$. Determine the condition on the orders $m$, $n$ of $x$ and $y$
    so that the map sending $x^i \rightsquigarrow y^i$ is a group homomorphism.
    \begin{solution}
        Note that $x^m = 1$, hence $x^{i} = x^{i + km}$ for all integers $k$. We
        want $y^i = \varphi(x^i) = \varphi(x^{i + m}) = y^{i + m}$, hence
        $1 = y^{m}$. This means that the order of $y$ must divide $m$, hence $n \,|\,
        m$ is a necessary condition for our map to be well-defined.

        We claim that $\varphi$ is now a homomorphism. Given any $x^p, x^q \in G$,
        we see that $\varphi(x^px^q) = \varphi(x^{p + q}) = y^{p + q} = y^p y^q =
        \varphi(x^p)\varphi(x^q)$. These operations are well-defined since there is
        a unique element $y^j \in H$ such that $y^j = y^{p + q}$, namely when $j = p
        + q \pmod{n}$.
    \end{solution}
    
    \paragraph{Exercise 12.} Prove that the $n\times n$ block matrices $M$ which
    have the block form \[
        \begin{bmatrix}
            A & B \\ 0 & D
        \end{bmatrix}
    \] with $A \in GL_r(\R)$ and $D \in GL_{n - r}(\R)$ form a subgroup $P$ of
    $GL_n(\R)$, and the map $P \to GL_r(\R)$ sending $M \rightsquigarrow A$ is a
    homomorphism. What is it's kernel?
    \begin{solution}
        Note that multiplication of such block matrices is closed, with \[
            \begin{bmatrix}
                A_1 & B_1 \\ 0 & D_1
            \end{bmatrix} \begin{bmatrix}
                A_2 & B_2 \\ 0 & D_2
            \end{bmatrix} = \begin{bmatrix}
                A_1A_2 & A_1B_2 + B_1D_2 \\ 0 & D_1D_2
            \end{bmatrix}.
        \] The identity matrix $\mathbb{I}_n \in P$, since it is of the required
        block form ($A = \mathbb{I}_r$, $D = \mathbb{I}_{n - r}$, $B = 0$). Because
        $A$ and $D$ are invertible, note that \[
            \begin{bmatrix}
                A & B \\ 0 & D
            \end{bmatrix}^{-1} = \begin{bmatrix}
                A^{-1} & -A^{-1}BD^{-1} \\ 0 & D^{-1}
            \end{bmatrix},
        \] which can be checked by \[
            \begin{bmatrix}
                A & B \\ 0 & D
            \end{bmatrix} \begin{bmatrix}
                A^{-1} & -A^{-1}BD^{-1} \\ 0 & D^{-1}
            \end{bmatrix} = \begin{bmatrix}
                A A^{-1} & A(-A^{-1}BD^{-1}) + B(D^{-1}) \\ 0 & D D^{-1}
            \end{bmatrix} = \begin{bmatrix}
                1 & 0 \\ 0 & 1
            \end{bmatrix}.
        \] Thus, $P$ forms a subgroup of $GL_n(\R)$. Labelling the matrices $M_A$,
        where $A$ is the top left block, we see that the map sends \[
            M_X \rightsquigarrow X, \qquad M_Y \rightsquigarrow Y, \qquad 
            M_XM_Y \rightsquigarrow XY
        \] since the top left block obeys the usual multiplication in $GL_r(\R)$.
        This shows that this map is a homomorphism. Its kernel is the pre-image of
        $\mathbb{I}_r$, which is the set of matrices $M_A \in P$ such that $A =
        \mathbb{I}_r$.
    \end{solution}

    \paragraph{Exercise 13.} \mbox{}
    \begin{enumerate}
        \itemsep0em
        \item Let $H$ be a subgroup of $G$, and let $g \in G$. The \textit{conjugate
        subgroup} $gHg^{-1}$ is defined to be the set of all conjugates $ghg^{-1}$,
        where $h \in H$. Prove that $gHg^{-1}$ is a subgroup of $G$.
        \item Prove that a subgroup $H$ of $G$ is normal if and only if $gHg^{-1} =
        H$ for all $g \in G$.
    \end{enumerate}
    \begin{solution} \mbox{}
    \begin{enumerate}
        \item Note that the identity $1 \in G$ is present in $H$, hence $g 1 g^{-1}
        = 1 \in gHg^{-1}$. Next, multiplication of elements in closed in the
        conjugate subgroup, because for any $h_1, h_2 \in H$, we have \[
            (gh_1g^{-1})(gh_2g^{-1}) = gh_1(g^{-1}g)h_2g^{-1} = g(h_1h_2)g^{-1},
        \] and $h_1 h_2 \in H$. Finally, for any $h \in H$, we have $(ghg^{-1})^{-1}
        = gh^{-1}g^{-1}$, with $h^{-1} \in H$. Thus, $gHg^{-1}$ is indeed a subgroup
        of $G$.

        \item Suppose that $H$ is a normal subgroup. For any $ghg^{-1} \in G$ with
        $h \in H$, we must have $ghg^{-1} \in H$ by definition. Also for any $h \in
        H$, we must have $ghg^{-1} \in H$, hence $gHg^{-1} = H$.

        Suppose that $gHg^{-1} = H$ for every $g \in G$. This means that given $h
        \in H$, we have $ghg^{-1} \in H$, hence $H$ must be a normal subgroup.
    \end{enumerate}
    \end{solution}
    
    \paragraph{Exercise 14.} Let $N$ be a normal subgroup of $G$, and let $g \in G$,
    $n \in N$. Prove that $g^{-1}ng \in N$.
    \begin{solution}
        Note that we must have $gng^{-1} \in N$ for every choice of $g \in G$. Thus,
        choose $g^{-1} \in G$, whence $(g^{-1})n(g^{-1})^{-1} = g^{-1}ng \in N$.
    \end{solution}

    \paragraph{Exercise 15.} Let $\varphi$ and $\psi$ be two homomorphisms from a
    group $G$ to another group $G'$, and let $H \subset G$ be the subset $\{x \in
    G\colon \phi(x) = \psi(x)\}$. Prove or disprove: $H$ is a subgroup of $G$.
    \begin{solution}
        Note that a homomorphism must map the identity to the identity, so
        $\barphi(1) = 1' = \psi(1)$ gives $1 \in H$. If $x, y \in H$, then
        $\varphi(x) = \psi(x)$ and $\varphi(y) = \psi(y)$, so $\varphi(xy) =
        \varphi(x)\varphi(y) = \psi(x)\psi(y) = \psi(xy)$ giving $xy \in H$.
        Finally, if $x \in H$ with $\varphi(x) = \psi(x)$, we have $\varphi(x^{-1}) =
        \varphi(x)^{-1} = \psi(x)^{-1} = \psi(x^{-1})$, so $x^{-1} \in H$. Thus, $H$
        forms a subgroup of $G$.
    \end{solution}
    
    \paragraph{Exercise 16.} Let $\varphi\colon G \to G'$ be a group homomorphism,
    and let $x \in G$ be an element of order $r$. What can you say about the order
    of $\varphi(x)$?
    \begin{solution}
        Note that the order of $\varphi(x)$ is at most $r$, since \[
            \varphi(x)^r = \varphi(x^r) = \varphi(1) = 1.
        \] Furthermore, if the order of $\varphi(x)$ is $r'$, we must have $r' \,|\,
        r$. Write $r = ar' + b$ for $0 \leq b < r'$, whence \[
            1 = \varphi(x^r) = \varphi(x^{ar' + b}) = \varphi(x^a)^{r'}\varphi(x)^b
            = \varphi(x)^b.
        \] The minimality of $r'$ forces $b = 0$, hence $r = ar'$.
    \end{solution}

    \paragraph{Exercise 17.} Prove that the center of a group is a normal subgroup.
    \begin{solution}
        Recall that the center $H$ of a group $G$ is the set of elements $x \in G$
        such that $xg = gx$ for all $g \in G$. This forms a subgroup of $G$ since $1
        \in H$, $(xy)g = x(yg) = x(gy) = (xg)y = g(xy)$ hence $xy \in H$ if $x, y
        \in H$, and $xg = gx$ implies $gx^{-1} = x^{-1}g$ hence $x^{-1}\in H$ if $x
        \in H$. Furthermore, $H$ is a normal subgroup since $H$ is abelian. Note
        that given any $x \in H$ and $g \in G$, we have $gxg^{-1} = xgg^{-1} = x \in
        H$.
    \end{solution}
    
    \paragraph{Exercise 18.} Prove that the center of $GL_n(\R)$ is the subgroup $Z
    = \{cI\colon c \in \R, c \neq 0\}$.
    \begin{solution}
        Note that for each element $cI \in Z$, we indeed observe that \[
            (cI)A = cA = Ac = (AI)c = A(cI)
        \] for every choice of $A \in GL_n(\R)$.
        Suppose that $B$ commutes with every matrix in $GL_n(\R)$, i.e.\ $BA = AB$
        for all $A \in GL_n(\R)$. Let $E_{ij}$ be the matrix with $1$ in the $i$,
        $j$th entry and zeroes everywhere else, and note that the product $BE_{ij}$
        is a matrix with its $j$th column equal to the $i$th column of $B$ and with
        zeroes everywhere else. Similarly, $E_{ij}B$ is the matrix whose $i$th row
        is the $j$th row of $B$, and with zeroes everywhere else. Now, $I + E_{ij}
        \in GL_n(\R)$, so by demanding $B(I + E_{ij}) = (I + E_{ij})B$, we want
        $BE_{ij} = E_{ij}B$, which forces $b_{jj} = b_{ii}$, with the remaining
        entries $b_{ki} = 0$ and $b_{jk} = 0$ for the remaining choices of $k$.
        Repeating this for every pair $1 \leq i < j \leq n$, we conclude that all
        the diagonal entries $b_{ii}$ are equal, and all off-diagonal entries of $B$
        are zero. Thus, $B$ must be of the form $b_{11}I = cI$.
    \end{solution}

    \paragraph{Exercise 19.} Prove that if a group contains exactly one element of
    order $2$, then that element is in the center of the group.
    \begin{solution}
        Let $x$ be the only element of order $2$ in the group $G$, i.e.\ $x^2 = 1$
        hence $x = x^{-1}$.
        Suppose that for some $g \in G$, we have $xg \neq gx$, i.e.\ $xgx^{-1}g^{-1}
        = y \neq 1$. Then we have $xy = x^{-1}y = gx^{-1}g^{-1}$, hence $(xy)^2 =
        gx^{-2}g^{-1} = gg^{-1} = 1$. Since $x$ is the only element of order $2$,
        either $xy = 1$ or $xy = x$. The former implies that $y = x^{-1} = x$, hence
        $xgx^{-1}g^{-1} = x \implies gx^{-1}g^{-1} = 1 \implies x = 1$, a
        contradiction. The latter implies that $y = 1$, another contradiction. Thus,
        we must have $xg = gx$ for all $g \in G$, which means that $x$ is in the
        center of $G$.
    \end{solution}
    
    \paragraph{Exercise 20.} Consider the set $U$ of real $3\times 3$ matrices of
    the form \[
        \begin{bmatrix}
            1 & * & * \\ 0 & 1 & * \\ 0 & 0 & 1
        \end{bmatrix}.
    \]
    \begin{enumerate}
        \itemsep0em
        \item Prove that $U$ is a subgroup of $SL_3(\R)$.
        \item Prove or disprove: $U$ is normal.
        \item Determine the center of $U$.
    \end{enumerate}
    \begin{solution} \mbox{}
    \begin{enumerate}
        \itemsep0em
        \item See Chapter 1, Miscellaneous Problems, Exercise 6.
        \item Consider \[
            \begin{bmatrix}
                1 & 0 & 0 \\ 1 & 1 & 0 \\ 0 & 0 & 1
            \end{bmatrix} \begin{bmatrix}
                1 & 0 & 0 \\ -1 & 1 & 0 \\ 0 & 0 & 1
            \end{bmatrix} = \begin{bmatrix}
                1 & 0 & 0 \\ 0 & 1 & 0 \\ 0 & 0 & 1
            \end{bmatrix},
        \] \[
             \begin{bmatrix}
                1 & 0 & 0 \\ 1 & 1 & 0 \\ 0 & 0 & 1
            \end{bmatrix} \begin{bmatrix}
                1 & 1 & 0 \\ 0 & 1 & 0 \\ 0 & 0 & 1
            \end{bmatrix}\begin{bmatrix}
                1 & 0 & 0 \\ -1 & 1 & 0 \\ 0 & 0 & 1
            \end{bmatrix} = \begin{bmatrix}
                1 & 1 & 0 \\ 1 & 2 & 0 \\ 0 & 0 & 1
            \end{bmatrix} \begin{bmatrix}
                1 & 0 & 0 \\ -1 & 1 & 0 \\ 0 & 0 & 1
            \end{bmatrix} = \begin{bmatrix}
                0 & 1 & 0 \\ -1 & 2 & 0 \\ 0 & 0 & 1
            \end{bmatrix}.
        \] Thus, this particular conjugate does not belong to $U$, which means that
        $U$ is not a normal subgroup of $SL_3(\R)$.

        \item Note that a general product of matrices in $U$ looks like \[
            \begin{bmatrix}
                1 & a & b \\ 0 & 1 & c \\ 0 & 0 & 1
            \end{bmatrix} \begin{bmatrix}
                1 & a' & b' \\ 0 & 1 & c' \\ 0 & 0 & 1
            \end{bmatrix} = \begin{bmatrix}
                1 & a' + a & b' + ac' + b \\ 0 & 1 & c' + c \\ 0 & 0 & 1
            \end{bmatrix}.
        \] Thus, for this product to commute for fixed $a, b, c$, we require $ac' =
        a'c$ for all choices of $a'$ and $c'$. This immediately gives $a = c = 0$,
        and this is sufficient. The center of $U$ is the set of all matrices of the
        form \[
            \begin{bmatrix}
                1 & 0 & b \\ 0 & 1 & 0 \\ 0 & 0 & 1
            \end{bmatrix}.
        \] 
    \end{enumerate}
    \end{solution}
    
    \paragraph{Exercise 21.} Prove by giving an explicit example that $GL_2(\R)$ is
    not a normal subgroup of $GL_2(\C)$.
    \begin{solution}
        Consider \[
            \begin{bmatrix}
                1 & i \\ 0 & 1
            \end{bmatrix} \begin{bmatrix}
                1 & 0 \\ 1 & 1
            \end{bmatrix} \begin{bmatrix}
                1 & -i \\ 0 & 1
            \end{bmatrix} = \begin{bmatrix}
                1 + i & i \\ 1 & 1
            \end{bmatrix} \begin{bmatrix}
                1 & -i \\ 0 & 1
            \end{bmatrix} = \begin{bmatrix}
                1 + i & -1 \\ 1 & 1 - i
            \end{bmatrix}.
        \] 
    \end{solution}

    \paragraph{Exercise 22.} Let $\varphi\colon G \to G'$ be a surjective homomorphism.
    \begin{enumerate}
        \itemsep0em
        \item Assume that $G$ is cyclic. Prove that $G'$ is cyclic.
        \item Assume that $G$ is abelian. Prove that $G'$ is abelian.
    \end{enumerate}
    \begin{solution} \mbox{}
    \begin{enumerate}
        \item Let $G$ be generated by the element $x$. For every element $u \in G'$,
        we must have $u = \varphi(x^r)$ for some $r \in \Z$ because of the
        surjectivity of $\varphi$. Thus, every $u \in G'$ is of the form $u =
        \varphi(x)^r$, which means that $G'$ is a cyclic group generated by
        $\varphi(x)$.

        \item For any $u, v \in G'$, we must have $u = \varphi(x)$ and $v =
        \varphi(y)$ for some $x, y \in G$. Then, $uv = \varphi(x)\varphi(y) =
        \varphi(xy) = \varphi(yx) = \varphi(y)\varphi(x) = vu$, which shows that
        $G'$ is abelian.
    \end{enumerate}    
    \end{solution}
    
    \paragraph{Exercise 23.} Let $\varphi\colon G \to G'$ be a surjective
    homomorphism, and let $N$ be a normal subgroup of $G$. Prove that $\varphi(N)$
    is a normal subgroup of $G'$.
    \begin{solution}
        Let $u \in \varphi(N)$ be fixed, and let $v \in G'$ be arbitrary. We find
        $x \in N$, $y \in G$ such that $u = \varphi(x)$, $v = \varphi(y)$. Since $x
        \in N$ is part of a normal subgroup, we have $yxy^{-1} \in N$, hence
        $\varphi(yxy^{-1}) \in \varphi(N)$. However, $\varphi(yxy^{-1}) = vuv^{-1}
        \in \varphi(N)$, which shows that $\varphi(N)$ is a normal subgroup of $G'$.
    \end{solution}
    

    \section{Equivalence Relations and Partitions}

    \paragraph{Exercise 1.} Prove that the non-empty fibres of a map form a
    partition of the domain.
    \begin{solution}
        Consider a map $f\colon X \to Y$, and without loss of generality let $f$ be
        surjective (if not, substitute $\im{Y}$ for $Y$). We wish to show that for
        any two $u, v \in Y$ such that $u \neq v$, the pre-images $f^{-1}(u)$ and
        $f^{-1}(v)$ are disjoint. Suppose these two sets had a common element $x$;
        this would imply $f(x) = u$ and $f(x) = v$ simultaneously, which is absurd.
        Furthermore, every $x \in X$ belongs to the fibre $f^{-1}(f(x))$ by
        construction, which means that the non-empty fibres of the map $f$ partition
        the domain $X$.
    \end{solution}

    \paragraph{Exercise 2.} Let $S$ be a set of groups. Prove that the relation
    $G\sim H$ if $G$ is isomorphic to $H$ is an equivalence relation in $S$.
    \begin{solution}
        Note that $\sim$ is reflexive, since every group is isomorphic to itself via
        the identity map. The relation $\sim$ is also symmetric because if $G\sim
        H$, there is an isomorphism $\varphi \colon G \to H$ whose inverse
        $\phi^{-1}\colon H \to G$ is also an isomorphism hence $H \sim G$. Finally,
        if $G_1 \sim G_2$ and $G_2 \sim G_3$, there exist isomorphisms
        $\varphi\colon G_1 \to G_2$ and $\psi\colon G_2 \to G_3$, whence the map
        $\psi\circ \varhi\colon G_1 \to G_3$ is an isomorphism giving $G_1 \sim
        G_3$, making $\sim$ transitive. Thus, $\sim$ is an equivalence relation in
        $S$.
    \end{solution}
    
    \paragraph{Exercise 3.} Determine the number of equivalence relations on a set
    of $5$ elements.
    \begin{solution}
        It can be shown that any partition $\{A_i\}$ of a set defines an equivalence
        relation, namely $x \sim y$ if $x$ and $y$ belong to the same set $A_i$.
        Additionally, every equivalence relation defines a partition of its set,
        which means that there is a one-to-one correspondence between partitions and
        equivalence relations.  There are $52$ partitions of a set of 5 elements,
        which means that there are $52$ possible equivalence relations.
    \end{solution}

    \paragraph{Exercise 4.} Is the intersection $R \cap R'$ of two equivalence
    relations $R, R' \subset S \times S$ an equivalence relation? Is the union?
    \begin{solution}
        Yes, the intersection $T = R\cap R'$ is an equivalence relation. Note that
        $xTx$ because $xRx$ and $xR'x$, $xTy$ means that $xRy$ and $xR'y$ which
        means $yRx$ and $yR'x$, giving $yTx$. Finally, if $xTy$ and $yTz$, that
        means that $xRy$ and $xR'y$, $yRz$ and $yR'z$, which together give $xRz$ and
        $xR'z$ giving $xTz$. This proves that $T$ is an equivalence relation.

        No, the union $R\cup R'$ is not necessarily an equivalence relation.
        Consider the relations on $S = \{1, 2, 3\}$, where \[
            R = \{(1, 1), (2, 2), (3, 3), (1, 2), (2, 1)\}, \qquad
            R' = \{(1, 1), (2, 2), (3, 3), (2, 3), (3, 2)\}. 
        \] It is clear that both $R$ and $R'$ are equivalence relation. Their union \[
            R \cup R' = \{(1, 1), (2, 2), (3, 3), (1, 2), (2, 1), (2, 3), (3, 2)\}
        \] is not however, since it contains $(1, 2)$ and $(2, 3)$, but not $(1, 3)$.
    \end{solution}
    
    \paragraph{Exercise 5.} Let $H$ be a subgroup of a group $G$. Prove that the
    relation defined by the rule $a \sim b$ if $b^{-1}a \in H$ is an equivalence
    relation on $G$.
    \begin{solution}
        For any $a \in G$, we have $a^{-1}a = 1 \in H$ so $a \sim a$. For any $a, b
        \in G$, if $a \sim b$ i.e.\ $b^{-1}a \in H$, its inverse $a^{-1}b \in H$
        i.e.\ $b \sim a$.  Finally, if for $a, b, c \in G$ we have $a \sim b$ and $b
        \sim c$, i.e.\ $b^{-1}a \in H$ and $c^{-1}b \in H$, we have the product
        $(c^{-1}b)(b^{-1}a) = c^{-1}a \in H$ so $c \sim a$.
    \end{solution}
    
    \paragraph{Exercise 6.} \mbox{}
    \begin{enumerate}
        \itemsep0em
        \item Prove that the relation $x$ conjugate to $y$ in a group $G$ is an
        equivalence relation on $G$.
        \item Describe the elements $a$ whose conjugacy class (equivalence class)
        consists of the element $a$ alone.
    \end{enumerate}
    \begin{solution} \mbox{}
    \begin{enumerate}
        \item Note that every element is self conjugate via $x = 1 x 1$. If $x$ is
        conjugate with $y$, we have $x = gyg^{-1}$ for some $g \in G$, hence $y =
        g^{-1}xg = (g^{-1})x(g^{-1})^{-1}$ so $y$ is conjugate with $y$. Finally, if
        $x$ is conjugate with $y$ and $y$ is conjugate with $z$, we have $x =
        gyg^{-1}$ and $y = hzh^{-1}$ for some $g, h \in G$, which gives $x =
        g(hzh^{-1})g^{-1} = (gh)x(gh)^{-1}$, hence $x$ is conjugate with $z$.

        \item Suppose that the conjugacy class of $a$ contains only $a$. This means
        that the only element $b$ such that $a = gbg^{-1}$ for some $g \in G$, i.e.\
        $b = g^{-1}ag$, is $a$ itself. Thus, the quantity $g^{-1}ag = a$ for all $g
        \in G$, i.e.\ $ag = ga$ for all $g \in G$, i.e.\ $a$ is in the center of
        $G$.
    \end{enumerate}
    \end{solution}
    
    \paragraph{Exercise 7.} Let $R$ be the relation on the set $\R$ of real numbers.
    We may view $R$ as a subset of the $(x, y)$ plane. Explain the geometric meaning
    of the reflexive and symmetric properties.
    \begin{solution}
        Reflexivity forces every pair $(x, x) \in R$, hence the diagonal line $x =
        y$ must be present in $R$. Symmetry forces $(y, x) \in R$ whenever $(x, y)
        \in R$, which means that $R$ has a reflection symmetry about the $x = y$
        line.
    \end{solution}
    
    \paragraph{Exercise 8.} With each of the following subsets $R$ of the $(x, y)$
    plane, determine which of the axioms (5.2) are satisfied and whether or not $R$
    is an equivalence relation on the set $\R$ of real numbers.
    \begin{enumerate}
        \itemsep0em
        \item $R = \{(s, s)\colon s \in \R\}$.
        \item $R = $ empty set.
        \item $R = $ locus $\{y = 0\}$.
        \item $R = $ locus $\{xy + 1 = 0\}$.
        \item $R = $ locus $\{x^2y - xy^2 - x + y = 0\}$.
        \item $R = $ locus $\{x^2 - xy + 2x - 2y = 0\}$.
    \end{enumerate}
    \begin{solution} \mbox{}
    \begin{enumerate}
        \item All three axioms are satisfied, $R$ being the `discrete' relation
        where every element is related only to itself.
        \item Symmetry and transitivity are vacuously satisfied, reflexivity is not.
        \item Only transitivity is (trivially) satisfied. Note that $(1, 1) \notin R$, $(1, 0)
        \in R$ but $(0, 1) \notin R$. Also, every element is only related to $0$,
        which means that the only configuration of $xRy$ and $yRz$ is $xR0$ and
        $0R0$.
        \item Only symmetry is satisfied. Note that $1\cdot 1 + 1 = 2 \neq 0$, and
        both $(1, -1) \in R$ and $(-1, 1) \in R$, yet $(1, 1) \notin R$.
        \item All three axioms are satisfied. Clearly, $(x, x) \in \R$ for all $x
        \in R$, whenever $(x, y) \in R$ we have $x^2y - xy^2 = x - y$, hence $y^2x -
        x^2y = y - x$ so $(y, x) \in R$. Rewrite our condition as $xy(x - y) - (x -
        y) = 0$ or $(xy - 1)(x - y) = 0$. Thus, $xRy$ if $x = y$ or $xy = 1$.
        Suppose that $xRy$ and $yRz$. We may have $x = y = z$, or $x = y$ and $yz =
        1$ hence $xz = 1$, $xy = 1$ and $y = z$ hence $xz = 1$, or $xy = 1$ and $yz
        = 1$, whence $x = z$. In all cases, $xRz$.
        \item Only symmetry is satisfied. Clearly $(x, x) \in R$ for all $x \in \R$.
        Note that our condition can be written as $x(x - y) + 2(x - y) = 0$, or $(x
        + 2)(x - y) = 0$. This gives $xRy$ whenever $x = -2$ or $x = y$. Thus, $(-2,
        0) \in R$ but $(0, -2) \notin R$. Now, if $xRy$ and $yRz$, we may have $x =
        -2$ and $y = -2$, or $x = -2$ and $y = z$, or $x = y$ and $y = -2$ hence $x
        = -2$, or $x = y$ and $y = z$ hence $x = z$. In all cases, $xRz$.
    \end{enumerate}
    \end{solution}

    \paragraph{Exercise 9.} Describe the smallest equivalence relation on the set of
    real numbers which contains the line $x - y = 1$ in the $(x, y)$ plane, and
    sketch it.
    \begin{solution}
        Reflexivity demands that we include the line $x = y$, symmetry demands that
        we include the line $y - x = 1$. Now, for all $x \in \R$, we have $(x - 1,
        x) \in R$ and $(x, x + 1) \in R$, so transitivity demands $(x - 1, x + 2)
        \in R$ which is equivalent to saying that $(x, x + 2) \in R$ for all $x \in
        \R$. By repeating this process arbitrarily many times and employing
        symmetry, we see that $(x, x + n) \in R$ for all $x \in \R$ and $n \in \Z$.
        This gives \[
            R = \text{locus }\{x - y = n\colon n \in \Z\}.
        \] This consists of all lines with slope $1$ which cut the axes at integral
        points.
    \end{solution}

    \paragraph{Exercise 10.} Draw the fibres of the map from the $(x, z)$ plane to
    the $y$ axis defined by the map $y = zx$.
    \begin{solution}
        Denote the map $f\colon \R^2 \to \R$, $(x, z) \mapsto xz$. Fixing $y \in R$,
        we see that $f^{-1}(y)$ is the collection of points $(x, z)$ such that
        $xz = y$, i.e.\ a rectangular hyperbola.
    \end{solution}
    
    \paragraph{Exercise 11.} Work out rules, obtained from the rules on the
    integers, for addition and multiplication on the set (5.8).
    \begin{solution}
        The required rules are those of addition and multiplication in the field
        $\Z_2$. Note that given any two elements $x, y \in \bar{0}$, we have $x + y
        = \bar{0}$ because the sum of even numbers is even. Similarly, the sum of an
        even number and an odd number is odd, and the sum of two odd numbers is
        even. This ensures that it makes sense to define $\bar{0} + \bar{0} =
        \bar{1} + \bar{1} = \bar{0}$, and $\bar{0} + \bar{1} = \bar{1} + \bar{0} =
        \bar{1}$. Similarly, the product of an even number with any other number is
        even, and the product of two odd numbers is odd. Thus, we can define
        $\bar{0}\cdot \bar{0} = \bar{0} \cdot \bar{1} = \bar{1} \cdot \bar{0} =
        \bar{0}$, and $\bar{1} \cdot \bar{1} = \bar{1}$.
    \end{solution}
    
    \paragraph{Exercise 12.} Prove that the cosets (5.14) are the fibres of the map
    $\varphi$.
    \begin{solution}
        Recall that we defined the group homomorphism $\varphi\colon G \to G'$ with
        kernel $N$, and described the cosets \[
            aN := \{g \in G \colon g = an \text{ for some }n \in N\}.
        \] Suppose that $b \in \im{\varphi}$. This means that $b = \varphi(a)$ for
        at least one $a \in G$. We claim that the fibre $\varphi^{-1}(b) = aN$.
        To see this, note that if $x \in \varphi^{-1}(b)$, then $\varphi(x) = b =
        \varphi(a)$, hence $\varphi(a^{-1}x) = 1$ so $a^{-1}x \in N$. Thus, $x =
        a(a^{-1}x) \in aN$. Next, pick $x \in aN$, i.e.\ $x = an$ where $\varphi(n)
        = 1$. Thus, $\varphi(x) = \varphi(an) = \varphi(a)\varphi(n) = b$, hence $x
        \in \varphi^{-1}(b)$. Together, we have shown that the coset $aN$ is
        precisely the fibre of $\varphi(a)$.
    \end{solution}
    
    
    
    \section{Cosets}

    \paragraph{Exercise 1.} Determine the index $[\Z : n\Z]$.
    \begin{solution}
        The left cosets of $n\Z$ are $n\Z, 1 + n\Z, \dots, (n - 1) + n\Z$, meaning
        that the index $[\Z\colon n\Z] = n$. Note that the coset $k + n\Z$ contains
        the elements $k + n\ell$, for all $\ell \in \Z$. This means that given any
        $m \in \Z$, set $m' = m\pmod{n}$, whereby $m = m' + n\ell'$ for some
        $\ell'$, so $m \in k + n\Z$. Finally, set $k' = k\pmod{n}$, and note that $0
        \leq k' < n$. Thus, $k \in k + n\Z$ and $k \in k' + n\Z$, so we already
        listed all possible cosets.
    \end{solution}

    \paragraph{Exercise 2.} Prove directly that distinct cosets do not overlap.
    \begin{solution}
        Let $H$ be a subgroup of $G$, and let $aH$ and $bH$ be cosets. Suppose that
        there is one element $ah_0 \in aH$ such that $ah_0 \notin bH$, i.e.\ the
        cosets are distinct. Then, for any $h \in aH$, if $ah \in bH$, that would
        imply that $ah = bh'$ for some $h' \in H$, hence $ah(h^{-1}h_0) =
        bh'(h^{-1}h_0)$, or $ah_0 = b(h'h^{-1}h_0)$. The parenthesized quantity is
        an element of $H$, hence $ah_0 \in bH$, which is a contradiction. Thus, no
        element of $aH$ is an element of $bH$. Similarly, for any $h \in bH$, if
        $bh \in aH$, we could find $bh = ah'$ for some $h' \in H$, hence
        $b(hh'^{-1}h_0) = ah_0$. This again implies that $ah_0 \in bH$, a
        contradiction. Thus, no element of $bH$ is an element of $aH$. Therefore,
        $aH$ and $bH$ are disjoint.
    \end{solution}

    \paragraph{Exercise 3.} Prove that for every group whose order is a power of a
    prime $p$ contains an element of order $p$.
    \begin{solution}
        Suppose that $G$ is a group such that $|G| = p^n$ for some natural number
        $n$. Pick $x \neq 1$ in $G$ and consider the cyclic group $H = \langle x\rangle$
        generated by $x$. Lagrange's Theorem guarantees that the order of $H$
        divides the order of $G$, i.e.\ $|H| = p^k$ for some natural number $k$
        (note that $|H| \neq 1$). Section~2.2,~Exercise~14.\ guarantees that since
        $p \,|\, p^k$, $H$ contains a cyclic subgroup of $H'$ of order $p$. The
        generator of $H'$ (specifically, $x^{p^{k - 1}}$) has order $p$ as desired.
    \end{solution}
    
    \paragraph{Exercise 4.} Give an example showing that left and right cosets of
    $GL_2(\R)$ in $GL_2(\C)$ are not always equal.
    \begin{solution}
        Consider the following cosets, and set $a = 1, d = 2, b = c = 0$. \[
            \begin{bmatrix}
                1 & i \\ i & 1
            \end{bmatrix} \begin{bmatrix}
                a & b \\ c & d
            \end{bmatrix} = \begin{bmatrix}
                a + ic & b + id \\ ai + c & bi + d
            \end{bmatrix} = \begin{bmatrix}
                1 & 2i \\ i & 2
            \end{bmatrix}
        \] \[
            \begin{bmatrix}
                a' & b' \\ c' & d'
            \end{bmatrix} \begin{bmatrix}
                1 & i \\ i & 1
            \end{bmatrix} = \begin{bmatrix}
                a' + ib' & a'i + b' \\ c' + id' & c'i + d'
            \end{bmatrix}
        \] For the matrix in the left coset to belong to the right coset, we require
        $a' = 1, b' = 0$ which immediately contradicts $a'i + b' = 2i$. Thus, the
        two cosets are not the same and are disjoint.
    \end{solution}

    \paragraph{Exercise 5.} Let $H, K$ be subgroups of a group $G$ of orders $3, 5$
    respectively. Prove that $H \cap K = \{1\}$.
    \begin{solution}
        Lagrange's Theorem forces the order of any element of a group to divide the
        order of the group, since the cyclic group generated by that element always
        forms a subgroup. Thus, for any element $x \in H \cap K$, the order of $x$
        divides both $3$ and $5$, which is possible only if $x$ has order $1$, i.e.\
        $x = 1$.
    \end{solution}
    
    \paragraph{Exercise 6.} Justify (6.15) carefully.
    \begin{quote}
        Let $\varphi\colon G \to G'$ be a homomorphism of finite groups. Then, \[
            |G| = |\ker{\varphi}| \cdot |\im{\varphi}|. 
        \] 
    \end{quote}
    \begin{solution}
        Recall that $\ker{\varphi}$ is a subgroup of $G$, so Lagrange's Theorem
        gives \[
            |G| = |\ker{\varphi}|\,[G: \ker{\varphi}].
        \] Again, the fibres of $\varphi$ correspond one-to-one with the left cosets
        of $\ker\varphi$, so \[
            [G:\ker{\varphi}] = |\im{\varphi}|.
        \] When $G$ and $G'$ are finite groups, all there quantities are finite, so
        combining them gives the desired formula.
    \end{solution}
    
    \paragraph{Exercise 7.} \mbox{}
    \begin{enumerate}
        \itemsep0em    
        \item Let $G$ be an abelian group of odd order. Prove that the map
        $\varphi\colon G \to G$ defined by $\varphi(x) = x^2$ is an automorphism.
        \item Generalize the result of (a).
    \end{enumerate}
    \begin{solution} \mbox{}
    \begin{enumerate}
        \item It is clear that $\varphi$ is a homomorphism in an abelian group
        because $(ab)^2 = a^2 b^2$ for all $a, b \in G$.  Pick $x \in G$. Since the
        order of $G$ is odd, the order of $x$ must also be odd (since it divides
        $|G|$), hence $x^{2k + 1} = 1$ for some $k$.  This gives $\varphi(x^{k + 1})
        = x^{2k + 2} = x^{2k + 1}x = x$. This means that $\varphi$ is surjective,
        which immediately gives $\varphi$ is injective and hence an isomorphism.

        The last fact can be verified by noting that $\im{\varphi} = G$, hence
        $|\ker{\varphi}| = |G| / |\im{\varphi}| = 1$.

        \item If $G$ is an abelian group of order $n$, the map $\varphi\colon G \to
        G$, $\varphi(x) = x^m$ where $m$ and $n$ are coprime is an automorphism.
        
        Since $n$ and $m$ are coprime, we can choose integers $a$ and $b$ such that
        $an + bm = 1$. Given any element $x \in G$, we must have $x^n = 1$.
        Thus, $\varphi(x^b) = x^{bm} = x^{an}x^{bm} = x^{an + bm} = x$, establishing
        that $\varphi$ is surjective. Thus, $\varphi$ is an automorphism.
        \end{enumerate}
    \end{solution}

    \paragraph{Exercise 8.} Let $W$ be the additive subgroup of $\R^m$ of solutions
    of a system of homogeneous linear equations $AX = 0$. Show that the solutions on
    an inhomogeneous system $AX = B$ form a coset of $W$.
    \begin{solution}
        Suppose $X$ satisfies $AX = B$. Then, for all $Y \in W$, we have $AY = 0$ so
        $A(Y + X) = B$. This means that all elements in the coset $W + X$ are
        solutions. Next, suppose that $X'$ is another solution, $AX' = B$. This
        gives $A(X - X') = 0$, so $X - X' \in W$, i.e.\ $X' = Y' + X$ for some $Y'
        \in W$. Thus, the solutions of $AX = B$ are precisely the elements of the
        coset $W + X$.

        Note that we may not even find any solution to $AX = B$, in which case the
        set of solutions is the empty set.
    \end{solution}
    
    \paragraph{Exercise 9.} Let $H$ be a subgroup of a group $G$. Prove that the
    number of left cosets is equal to the number of right cosets if (a) $G$ is
    finite and (b) in general.
    \begin{solution}
        Consider the map $x \rightsquigarrow x^{-1}$, which sends the cosets $aH
        \rightsquigarrow Ha^{-1}$. This is so because for any $x \in aH$,
        write $x = ah$ uniquely for $h \in H$, and note that $x \rightsquigarrow h^{-1}a^{-1}
        \in Ha^{-1}$. Similarly, for any $y \in Ha^{-1}$, write $y = ha^{-1}$
        uniquely for some $h \in H$, and note that $ah^{-1} \rightsquigarrow
        (ah^{-1})^{-1} = ha^{-1}$, with $ah^{-1} \in aH$. Thus, we have $aH
        \rightsquigarrow Ha^{-1}$ precisely, making this map a bijection between
        left and right cosets. This means that they must be equal in number.
    \end{solution}
    
    \paragraph{Exercise 10.} \mbox{}
    \begin{enumerate}
        \itemsep0em
        \item Prove that every subgroup of index $2$ is normal.
        \item Give an example of a subgroup of index $3$ which is not normal.
    \end{enumerate}
    \begin{solution} \mbox{}
    \begin{enumerate}
        \item Let $H$ be a subgroup of $G$, with $[G : H] = 2$. Note that $1H = H$
        is indeed a left coset of $H$. Let the other left coset of $H$ be $aH$. For
        $H \neq aH$, we have $1 \neq ah$ for any $h \in H$, i.e.\ $a \notin H$.
        Given any element $h \in H$, $g \in G$, either $g \in H$ in which case
        $ghg^{-1} \in H$ trivially, or $g \notin H$, so the other coset is $gH \neq
        H$. Now if $ghg^{-1} \notin H$, we have $ghg^{-1} \in gH$, so $ghg^{-1} =
        gh'$ for some $h' \in H$. This gives $hg^{-1} = h'$ or $g = h'^{-1}h \in H$,
        a contradiction.  Thus, we always have $ghg^{-1} \in H$, making $H$ a normal
        subgroup.

        \item Recall that the subgroup $\{1, x\} = H\subset S_3$ is not normal, and
        the index $[G:H] = |G| / |H| = 3$.
    \end{enumerate}
    \end{solution}

    \paragraph{Exercise 11.} Classify groups of order $6$ by analysing the the
    following three cases.
    \begin{enumerate}
        \itemsep0em
        \item $G$ contains an element of order $6$.
        \item $G$ contains an element of order $3$ but none of order $6$.
        \item All elements of $G$ have order $1$ or $2$.
    \end{enumerate}
    \begin{solution} \mbox{}
    \begin{enumerate}
        \item Let $x \in G$ be of order $6$. Then the elements $1, x, x^2, x^3, x^4,
        x^5$ are all distinct; if any two were equal, $x^m = x^n$, then $x^{m - n} =
        1$ with $|m - n| < 6$, a contradiction. These must be the six elements of
        $G$, hence $G$ is the cyclic group of order $6$.

        \item Let $x \in G$ be of order $3$. The elements $1, x, x^2$ are distinct.
        Pick $y \in G$, $y \neq x^k$. If $y$ also has order $3$, then the elements
        $1, y, y^2$ would be distinct, with $y \neq x$, $y \neq x^2$ so $y^2 \neq x$
        and $y^2 \neq x^2$ (if $y^2 = x$, then $x^2 = y^4 = y$ and if $y^2 = x^2$,
        then $x = x^4 = y^4 = y$). Thus, $1, x, x^2, y, y^2$ would be distinct. The
        element $xy$ is also distinct, since $xy \neq 1$ ($x \neq y^{-1} = y^2$),
        $xy \neq x$, $xy \neq y$ ($x, y \neq 1$), $xy \neq x^2$, $xy \neq y^2$ ($x
        \neq y$). The element $xy^2$ is also distinct, since $xy^2 \neq 1$ ($y^2
        \neq x^{-1} = x^2$), $xy^2 \neq x$ ($y^2 \neq 1$), $xy^2 \neq y$ ($xy \neq
        1$), $xy^2 \neq xy$ ($y \neq 1$), $xy^2 \neq x^2$ ($y^2 \neq x$), $xy^2 \neq
        y^2$ ($x \neq 1$). This gives too many elements in $G$.

        Thus, $y \neq 1$ must have order $2$. Thus, we have the elements $1, x, x^2,
        y, xy, xy^2$ which are all distinct. Note that $y^2 = 1$ gives $y^{-1} = y$.
        Examine $(xy)^2 = xyxy$. If $xyxy = x$ then $yxy = 1 \implies xy = y^{-1} =
        y \implies x = 1$, if $xyxy = x^2$ then $(xy)^3 = x^2(xy) = y \neq 1$, which
        contradicts the fact that the maximum order of any element in $G$ is $3$.
        If $xyxy = xy$, then $xyx = 1$, if $xyxy = y$ then $(xy)^3 = (xy)y = x \neq
        1$, again contradicting the maximum order. If $xyxy = xy^2$, then $yxy = y^2
        \implies x = 1$. Thus, we must have $(xy)^2 = 1$. A similar argument can be
        used to show that $(xy^2)^2 = 1$.

        Now, examine $yx$. Clearly, $yx \neq 1$, $yx \neq x$, $yx \neq y$, $yx \neq
        x^2$. If $yx = xy$, then $yxy = xy^2 = x$, which we have already seen gives
        a contradiction. Thus, $yx = xy^2$, and similarly we can show that $yx^2 =
        xy$. This is enough to complete the entire multiplication table of $G$,
        which we see must be precisely the symmetric group $S_3$, which is isomorphic
        to the dihedral group $D_3$.

        \item Let all $x \in G$, $x \neq 1$ have order $2$, and recall that we have
        already shown that a group where every element is of order $2$ at most is
        abelian. Pick $x, y \in G$ with $x, y \neq 1$ and $x \neq y$. Then note that
        $1, x, y, xy$ are distinct, with $xy = yx$. In fact, these four elements are
        closed under multiplication, each equal to their own inverse, and hence form
        a subgroup of order $4$. This contradicts Lagrange's Theorem, which requires
        the order of any subgroup of $G$ to divide the order of $G$.
    \end{enumerate}
    \end{solution}
    
    \paragraph{Exercise 12.} Let $G, H$ be the following subgroups of $GL_2(\R)$. \[
        G = \left\{ \begin{bmatrix}
            x & y \\ 0 & 1
        \end{bmatrix} \right\}, \qquad
        H = \left\{ \begin{bmatrix}
            x & 0 \\ 0 & 1
        \end{bmatrix} \right\}, \qquad x > 0.
    \] An element of $G$ can be represented by a point in the $(x, y)$ plane. Draw
    the partitions of the plane into left and right cosets of $H$.
    \begin{solution}
        Fix $A_{x_0y_0} \in G$, and examine the left coset \[
            \begin{bmatrix}
                x_0 & y_0 \\ 0 & 1
            \end{bmatrix} \begin{bmatrix}
                x & 0 \\ 0 & 1
            \end{bmatrix} = \begin{bmatrix}
                x_0x & y_0 \\ 0 & 1
            \end{bmatrix}.
        \] Thus, the elements of this coset correspond to the set of points $(x_0x,
        y_0) \equiv (x, y_0)$, which is the horizontal ray with $x > 0$ radiating
        from (but not intersecting) the $y$ axis at $y_0$.

        Now examine the right coset \[
            \begin{bmatrix}
                x & 0 \\ 0 & 1
            \end{bmatrix} \begin{bmatrix}
                x_0 & y_0 \\ 0 & 1
            \end{bmatrix} = \begin{bmatrix}
                x_0x & y_0x \\ 0 & 1
            \end{bmatrix}.
        \] Thus, the elements of this coset correspond to the set of points $(x_0x,
        y_0x)$, which is the ray with $x > 0$ radiating from the origin (but not
        including it) with slope $y_0 / x_0$.
    \end{solution}
    


    \section{Restriction of a Homomorphism to a Subgroup}

    \paragraph{Exercise 1.} Let $G$ and $G'$ be finite groups whose orders have no
    common factor. Prove that the only homomorphism $\varphi\colon G \to G'$ is the
    trivial one $\varphi(x) = 1$ for all $x$.
    \begin{solution}
        This follows directly from $|G| = |\ker{\varphi}|\cdot|\im{\varphi}|$. Note
        that $|\im{\varphi}|$ is a factor of $|G'|$, which forces $|\im{\varphi}| =
        1$, hence $|\ker{\varphi}| = |G|$. Thus, $\ker{\varphi} = G$, which means
        that $\varphi(x) = 1$ for all $x \in G$.
    \end{solution}

    \paragraph{Exercise 2.} Give an example of a permutation of even order which is
    odd and an example of one which is even.
    \begin{solution}
        The permutation $(12)$ has odd sign, but $(12)^2 = 1$.
        The permutation $(123)$ has even sign, but $(123)^3 = 1$.
    \end{solution}

    \paragraph{Exercise 3.} \mbox{}
    \begin{enumerate}
        \itemsep0em
        \item Let $H$ and $K$ be subgroups of a group $G$. Prove that the
        intersection $xH \cap yK$ of two cosets of $H$ and $K$ is either empty or
        else is a coset of the subgroup $H \cap K$.
        \item Prove that if $H$ and $K$ have finite index in $G$ then $H \cap K$
        also has finite index.
    \end{enumerate}
    \begin{solution} \mbox{}
    \begin{enumerate}
        \item The intersection $H \cap K$ contains elements of the form $z = xh =
        yk$, where $h \in H, k \in K$. Either this intersection is empty, or
        contains at least one such $z$. Such an element demands $x = zh^{-1}$, and
        $y = zk^{-1}$. In other words, $x$ and $y$ are elements of the coset
        $z(H\cap K)$ (recall that $H \cap K$ is indeed a subgroup). Now pick any
        element in $z(H \cap K)$, say $z' = zg$ where $g \in H \cap K$. Then, $z' =
        (xh)g = x(hg)$, and $z' = (yk)g = y(kg)$, with $hg \in H$ and $kg \in K$,
        hence $z' \in xH \cap yK$. Thus, we have shown that $xH \cap yK = z(H \cap
        K)$.

        \item If $H$ and $K$ have finite indices, that means that there are finitely
        many cosets $x_iH$ and $y_jK$. Every combination $x_iH \cap y_jK$
        corresponds to at most one coset $z_{ij}(H \cap K)$.
        Furthermore, given some coset $z(H \cap K)$, we have elements $z' = zg$ with
        $g \in H \cap K$, hence $z' \in zH \cap zK$. Thus, we have exhausted all
        cosets of $H \cap K$, which must be finite.
    \end{enumerate} 
    \end{solution}
    
    \paragraph{Exercise 4.} Prove Proposition~7.1.
    \begin{quote}
        The intersection $K \cap H$ of two subgroups is a subgroup of $G$. If $K$ is
        a normal subgroup of $G$, then $K \cap H$ is a normal subgroup of $H$.
    \end{quote}
    \begin{solution}
        First, note that $1 \in K$ and $1 \in H$, hence $1 \in K \cap H$. Next, pick
        $x, y \in K \cap H$. Then we have $xy \in H$ and $xy \in K$ due to the
        closure of multiplication in each of the groups, hence $xy \in K \cap H$.
        Finally, pick $x \in K \cap H$, whence $x^{-1} \in K$ and $x^{-1} \in H$
        hence $x^{-1} \in K \cap H$. This proves that $K \cap H$ is a subgroup of
        $G$.

        Suppose that $K$ is a normal subgroup of $G$. Pick some $x \in K \cap H$,
        and $g \in G$. Note that $x \in K$ implies that the conjugate $gxg^{-1} \in
        K$, hence $gxg^{-1} \in K \cap H$. This proves that $K \cap H$ is a normal
        subgroup of $G$.
    \end{solution}

    \paragraph{Exercise 5.} Let $H, N$ be subgroups of a group $G$, with $N$ normal.
    Prove $HN = NH$ and that this set is a subgroup.
    \begin{solution}
        Pick $x \in HN$, hence $x = hn$ for some $h \in H$, $n \in N$. Since $n \in
        N$ is in a normal subgroup, we must have $hnh^{-1} = n' \in N$. Thus, $hn =
        n'h$, hence $x = n'h$, so $HN \subseteq NH$. A similar argument can be used
        to show $NH \subseteq HN$, hence $HN = NH$.

        Note that $1 \in H$ and $1 \in N$, hence $1 \in HN$. Next, pick $x, y \in HN
        = NH$, thus we can choose $x = h_1n_1$ and $y = n_2h2$, so $xy = h_1n_1n2h_2
        = h_1n_3h_2$. Further note that $n_3h_2 \in NH = HN$, hence $n_3h_2 =
        h_4n_4$, so $xy = h_1h_4n_4 = (h_1h_4)n_4 \in HN$. Finally, pick $x \in HN$,
        so $x = hn$, whence $x^{-1} = n^{-1}h^{-1} \in NH = HN$. This proves that
        $HN = NH$ is a subgroup.
    \end{solution}
    
    \paragraph{Exercise 6.} Let $\varphi\colon G \to G'$ be a group homomorphism
    with kernel $K$, and let $H$ be another subgroup of $G$. Describe
    $\varphi^{-1}(\varphi(H))$ in terms of $H$ and $K$.
    \begin{solution}
        We claim that $\varphi^{-1}(\varphi(H)) = KH$. First, let $x \in KH$, i.e.\
        $x = kh$ for some $k \in K$, $h \in K$. Then, $\varphi(x) = \varphi(kh) =
        \varphi(k)\varphi(h) = \varphi(h)$, hence $x \in \varphi^{-1}(\varphi(H))$.
        Next, let $x \in \varphi^{-1}(\varphi(H))$, i.e.\ $\varphi(x) = \varphi(h)$
        for some choice of $h \in H$. Then, $\varphi(x)\varphi(h)^{-1} = 1$, which
        gives $\varphi(xh^{-1}) = 1$. Thus, $xh^{-1} \in K$ so $xh^{-1} = k$ for
        some $k \in K$. This gives $x = kh \in KH$.
    \end{solution}

    \paragraph{Exercise 7.} Prove that a group of order $30$ can have at most $7$
    subgroups of order $5$.
    \begin{solution}
        Consider the intersection $G_1 \cap G_2$ of two subgroups of $G$, both of
        order $5$. Note that $G_1 \cap G_2$ is also a subgroup of $G_1$ as well as
        $G_2$. Lagrange's Theorem gives \[
            |G_1 \cap G_2| \text{ divides } |G_1| = |G_2| = 5.
        \] This forces $|G_1 \cap G_2| = 1$ for distinct subgroups $G_1$ and $G_2$,
        i.e.\ distinct subgroups intersect only at the identity element. Thus, $n$
        subgroups require at least $1 + (5 - 1)n = 1 + 4k$ distinct elements in the
        group $G$. Eight or more subgroups require $1 + 4\times 8 = 33$ elements in
        $G$, a contradiction.
    \end{solution}
    
    \paragraph{Exercise 8.} Prove the \textit{Correspondence Theorem}: Let
    $\varphi\colon G \to G'$ be a surjective group homomorphism with kernel $N$. The
    set of subgroups $H'$ of $G'$ is in bijective correspondence with the set of
    subgroups $H$ of $G$ which contain $N$, the correspondence being defined by the
    maps $H \rightsquigarrow \varphi(H)$ and $H' \rightsquigarrow \varphi^{-1}(H')$.
    Moreover, normal subgroups of $G$ correspond to normal subgroups of $G'$.
    \begin{solution}
        First, we show that this map is injective. Let $\varphi(H_1) = \varphi(H_2)
        = H'$ where $H_1$ and $H_2$ are subgroups of $G$ containing $N$. Then,
        Exercise~6 shows that $\varphi^{-1}(H') = NH_1 = NH_2$. However, $H_1$
        contains $N$, so every $nh_1 \in H_1$ for every $n \in N$, $h_1 \in H_1$,
        and every $h_1 = 1h_1$ for $h_1 \in H_1$, so $NH_1 = H_1$. Similarly, $NH_2
        = H_2$, hence $H_1 = H_2$.

        Next, we show that this map is surjective. Let $H'$ be a subgroup of $G$.
        Since $\varphi$ is surjective, we see that every $h' \in G'$ has at least one
        $\varphi(h) = h'$ with $h \in G$. To see that $\varphi^{-1}(H')$ forms
        a subgroup of $G$, note that $\varphi(1) = 1$, $\varphi(h^{-1}) =
        h'^{-1}$, and if $\varphi(h_1) = h_1'$, $\varphi(h_2) = h_2'$, then 
        $\varphi(h_1h_2) = \varphi(h_1)\varphi(h_2) = h_1'h_2' \in H'$. Clearly, $H
        = \varphi^{-1}(H')$ contains the kernel $N$, since $N = \varphi^{-1}(1)
        \subseteq \varphi^{-1}(H')$.

        This proves that the described map between subgroups of $G$ containing the
        kernel $N$ and subgroups of $G'$ is indeed a bijection.

        Furthermore, suppose that $H \subseteq N$ is a normal subgroup. Then,
        $ghg^{-1} \in H$ for every $g \in G$. Now, set $H' = \varphi(H)$, and let
        $h' \in H'$, $g' \in G'$ be arbitrary. Pick $h \in H$, $g \in G$ such that
        $\varphi(h) = h'$, $\varphi(g) = g'$, whence $g'h'g'^{-1} =
        \varphi(ghg^{-1}) \in \varphi(H) = H'$. This shows that $H'$ is normal.

        Next, suppose that $H' \subseteq G'$ is a normal subgroup. Set $H =
        \varphi^{-1}(H')$ and let $h \in H$, $g \in G'$. Then, note that
        $\varphi(g)\varphi(h)\varphi(g)^{-1} \in H'$ so $\varphi(ghg^{-1}) \in H'$,
        hence $ghg^{-1} \in \varphi^{-1}(H') = H$. This shows that $H$ is normal.
    \end{solution}

    \paragraph{Exercise 9.} Let $G$ and $G'$ be cyclic groups of orders $12$ and $6$
    generated by elements $x$, $y$ respectively, and let $\varphi\colon G \to G'$ be
    the map defined by $\varphi(x^i) = y^i$. Exhibit the correspondence referred to
    in the previous problem explicitly.
    \begin{solution}
        It is clear that the kernel of $\varphi$ is the set $\{1, x^6\}$. We now
        illustrate the correspondence between those subgroups of $G$ containing this
        kernel, and the subgroups of $G'$ (whose entirety is the image of $\varphi$).
        \[
        \begin{array}{rl}
             \{1, x^6\} &\rightsquigarrow \{1\}, \\
             \{1, x^3, x^6, x^9\} &\rightsquigarrow \{1, y^3\}, \\
             \{1, x^2, x^4, x^6, x^8, x^{10}\} &\rightsquigarrow \{1, y^2, y^4\}, \\
             \{1, x, x^2, x^3, x^4, \dots, x^{10}, x^{11}\} &\rightsquigarrow \{1, y,
             y^2, y^3, y^4, y^5\}.
        \end{array}
        \] Note that we have exhausted all subgroups of $G'$ on the right. All
        subgroups here are normal, since cyclic groups are abelian.
    \end{solution}
    


    \section{Products of Groups}

    \paragraph{Exercise 1.} Let $G$, $G'$ be groups. What is the order of the
    product group $G \times G'$?
    \begin{solution}
        Clearly, the Cartesian product $G\times G'$ contains $|G|\,|G'|$ elements,
        which must then be the order of the product group.
    \end{solution}
    
    \paragraph{Exercise 2.} Is the symmetric group $S_3$ a direct product of
    non-trivial groups?
    \begin{solution}
        No. Suppose that $S_3$ is isomorphic to $G_1\times G_2$ for non-trivial
        groups $G_1$, $G_2$. Since $|S_3| = 6$, the previous exercise forces $|G_1|
        = 2$, $|G_2| = 3$ (or the other way around). These are cyclic groups (note
        that a group of order $3$ must contain the elements $1, b, b^{-1}$).
        Thus, if $G_1 = \{1, a\}$, $G_2 = \{1, b, b^2\}$, the element $(a, b)$ turns
        out to have order $6$. \[
            (a, b)^2 = (1, b^2), \qquad (a, b)^3 = (a, 1), \qquad (a, b)^4 = (1, b),
            \qquad (a, b)^5 = (a, b^2), \qquad (a, b)^6 = (1, 1).
        \] However, $S_3$ contains no element of order $6$.
    \end{solution}
    
    \paragraph{Exercise 3.} Prove that a finite cyclic group of order $rs$ is
    isomorphic to the product of cyclic groups of orders $r$ and $s$ if and only if
    $r$ and $s$ have no common factor.
    \begin{solution}
        Suppose that $C_n$ is a cyclic group of order $n$, and $n = rs$ with
        $\gcd(r, s) = 1$. Consider the homomorphism \[
            \varphi\colon C_n \to C_r\times C_s, \qquad x^i\rightsquigarrow (y^i,
            z^i)
        \] where $C_n, C_r, C_s$ are generated by $x, y, z$ respectively. Note that
        the power maps $x^i \rightsquigarrow y^i$ and $x^i \rightsquigarrow z^i$ are
        homomorphisms individually. The kernel of $\varphi$ is the pre-image of $(1,
        1)$, hence any $x^j$ in the kernel satisfies $y^j = 1$ and $z^j = 1$. Note
        that $y$ and $z$ have orders $r$ and $s$, so we have $r \,|\, j$ and $s\,|\,
        j$. Because $r$ and $s$ are coprime, the smallest positive solution for $j$
        is $rs = n$. Thus, the kernel of $\varphi$ contains only $x^n = 1$, hence
        this map is injective. The orders of $C_n$ and $C_r\times C_s$ are equal,
        hence $\varphi$ is an isomorphism.

        Now suppose that $C_n$ is isomorphic to $C_r\times C_s$, with $\varphi\colon
        C_n \to C_r\times C_s$ being such an isomorphism. We demand $n = rs$ by
        equating the orders of the groups. If $r$ and $s$ share a common factor $t >
        1$, note that \[
            (y, z)^{rs / t} = ((y^{r})^{s / t}, (z^{s})^{r / t}) = (1, 1),
        \] with $rs / t = n / t < n$. Thus, the kernel of $\varphi$ is non-trivial,
        meaning that $\varphi$ cannot have been an isomorphism.
    \end{solution}
    
    \paragraph{Exercise 4.} In each of the following cases, determine whether or not
    $G$ is isomorphic to the product of $H$ and $K$.
    \begin{enumerate}
        \itemsep0em
        \item $G = \R^\times$, $H = \{\pm 1\}$, $K = \{\text{positive real
        numbers}\}$.
        \item $G = \{\text{invertible upper triangular }2\times 2\text{
        matrices}\}$, $H = \{\text{invertible diagonal matrices}\}$, $K =
        \{\text{upper triangular matrices with diagonal entries }1\}$.
        \item $G = \C^\times$, $H = \{\text{unit circle}\}$, $K = \{\text{positive
        reals}\}$.
    \end{enumerate}
    \begin{solution} \mbox{}
    \begin{enumerate}
        \item Yes. Consider the map \[
            \varphi\colon \R^\times \to \{\pm 1\}\times \R^{pos}, \qquad
            x \rightsquigarrow (\sgn{x}, |x|).
        \] Clearly, if $z = xy$ where $x, y \in \R^\times$, we have \[
            (\sgn{x}, |x|)\cdot(\sgn{y}, |y|) = (\sgn{x}\sgn{y}, |x| |y|) =
            (\sgn{xy}, |xy|) = (\sgn{z}, |z|).
        \] Furthermore, every element $(s, r) \in \{\pm 1\}\times\R^{pos}$
        corresponds to $sr$ in $\R^\times$, and if $(s_1, r_1) = (s_2, r_2)$, then
        we demand $s_1 = s_2$ and $r_1 = r_2$, hence $s_1r1 = s_2r_2$. Thus,
        $\varphi$ is an isomorphism.

        \item No. Note that both $H$ and $K$ are abelian groups, but $G$ is not.

        Note that $(h_1, k_1)(h_2, k_2) = (h_1h_2, k_1k_2) = (h_2h_1, k_2k_1) = (h_2,
        k_2)(h_1, k_1)$ when $H$ and $K$ are both abelian, hence if $\varphi\colon
        H\times K \to G$ is to be an isomorphism, with $\varphi(h_i, k_i) = g_i$, we
        demand $g_1g_2 = g_2g_1$.

        \item Yes. Note that every point in the unit circle can be uniquely assigned
        an angle $x \in [0, 2\pi)$, and hence every point on the unit circle is
        associated with the unique complex number $e^{ix}$. Now, construct the map
        \[
            \varphi\colon \{e^{ix}\}\times \R^{pos} \to \C^\times, \qquad
            (e^{ix}, r) \rightsquigarrow re^{ix}.
        \] This map is clearly a bijection (every complex number has a unique
        positive norm, and can be assigned a unique argument/angle; two complex
        numbers that are equal have the same norm and the same argument).
        Furthermore, note that \[
            (e^{ix_1}, r_1)\cdot (e^{ix_2}, r_2) = (e^{i(x_1 + x_2)}, r_1r_2), \qquad 
            r_1e^{ix_1} r_2e^{ix_2} = r_1r_2 e^{i(x_1 + x_2)},
        \] which proves that $\varphi$ is an isomorphism.
    \end{enumerate}
    \end{solution}
    
    \paragraph{Exercise 5.} Prove that the product of two infinite cyclic groups is
    not infinite cyclic.
    \begin{solution}
        Suppose that $G$ and $H$ are infinite cyclic groups generated by $g$
        and $h$ respectively, and suppose that $G_1 \times G_2$ is generated by
        the element $(g^i, h^j)$. This requires every element of $G\times H$ to be
        of the form $(g^{in}, h^{jn})$ for integers $n$. Now, $(g, 1)$ is in the
        desired group, so $g^{in} = g$, and $h^{jn} = 1$. Thus, $j = 0$. A similar
        argument with $(1, h)$ forces $i = 0$, but clearly $(g^i, h^j) = (1, 1)$
        cannot generate $(g, h)$. Thus, the product $G\times H$ is not cyclic.
    \end{solution}
    
    \paragraph{Exercise 6.} Prove that the centre of the product of two groups is
    the product of their centres.
    \begin{solution}
        Suppose that some $(x, y)$ commutes with all choices of $(a, b) \in G\times
        H$. This immediately gives $xa = ax$, $yb = by$ for all choices of $a \in
        G$, $b \in H$ hence $x, y$ are in the centres of $G$, $H$ respectively.

        Similarly, if $xa = ax$ for all $a \in G$ and $yb = by$ for all $b \in G$,
        we must have $(x, y)(a, b) = (xa, yb) = (ax, by) = (a, b)(x, y)$ for all
        $(a, b) \in G\times H$, which means that $(x, y)$ is in the centre of
        $G\times H$.
    \end{solution}
    
    \paragraph{Exercise 7.} \mbox{}
    \begin{enumerate}
        \itemsep0em
        \item Let $H$, $K$ be subgroups of a group $G$. Show that the set of
        products $HK = \{hk: h \in H, k \in K\}$ is a subgroup if and only if $HK =
        KH$.
        \item Give an example of a group $G$ and two subgroups $H$, $K$ such that
        $HK$ is not a subgroup.
    \end{enumerate}
    \begin{solution} \mbox{}
    \begin{enumerate}
        \item Suppose that $HK = KH$. Clearly, $HK$ contains $1 1 = 1$, and if $x
        \in HK$, then $x = hk$ for some $h\in H$, $k \in K$ so $x^{-1} =
        k^{-1}h^{-1} \in KH = HK$. Finally if $x = h_1k_1$, $y = h_2k_2$ for $h_i
        \in H, k_i \in K$, then $xy = h_1k_1h_2k_2 = h_1(k_1h_2)k_2 =
        h_1(h_2'k_1')k_2 = (h_1h_2')(k_1'k_2) \in HK$, since $k_1h_2 \in KH = HK$.
        This proves that $HK$ is a subgroup of $G$.

        \item Consider $G = S_4$, $H = \{1, (14)\}$, $K = \{1, (123), (132)\}$. We
        compute \[
            HK = \{1, (14), (123), (1234), (132), (1324)\},
        \] \[
            KH = \{1, (14), (123), (1423), (132), (1432)\}.
        \] It is clear that $HK$ is not a subgroup; note that the inverse
        $(1234)^{-1} = (1432)$ is not present in $HK$.
    \end{enumerate}
    \end{solution}

    \paragraph{Exercise 8.} Let $G$ be a group containing normal subgroups of orders
    3 and 5 respectively. Prove that G contains an element of order 15.
    \begin{solution}
        Recall that in Section~2.7,~Exercise~5, we have shown that for subgroups $H$
        and $K$ of $G$ with either one normal, $HK = KH$ and hence $HK$ is a
        subgroup of $G$. Section~2.6,~Exercise~3 shows that each of the
        groups $H$ and $K$ of prime orders $3$ and $5$ contain an element each of
        order $3$ and $5$, say $x$ and $y$. In other words, $x$ and $y$ generate
        $H$ and $K$, so they are cyclic, abelian groups. Section~2.6,~Exercise~5
        shows that $H$ and $K$ are disjoint. Proposition~8.6~(c) shows that $HK$ is
        isomorphic to $H \times K$. Section~2.8,~Exercise~3 shows that $H\times K$
        is isomorphic to the finite cyclic group of order $15$. The generator of
        this group has order $15$, thus $G$ contains an element of order $15$.
    \end{solution}

    \paragraph{Exercise 9.} Let $G$ be a finite group whose order is a product of
    two integers: $n = ab$. Let $H$, $K$ be subgroups of $G$ of orders $a$ and $b$
    respectively. Assume that $H \cap K = \{1\}$. Prove that $HK = G$. Is $G$
    isomorphic to the product group $H \times K$?
    \begin{solution}
        It is clear that $HK \subseteq G$. We claim that all $ab$ possible products
        $h_ik_j$ in $HK$ are distinct, which would force $HK = G$. Suppose that
        $h_1k_1 = h_2k_2$, where either $h_1 \neq h_2$ or $k_1 \neq k_2$ or both.
        The possibility that exactly one of $h_1 = h_2$ and $k_1 = k_2$ can be
        discarded immediately by cancellation. Thus, $h_1 \neq h_2$ and $k_1 \neq
        k_2$. We have $h_1 = h_2k_2k_1^{-1} = h_2 (k_2k_1^{-1})$, thus $h_2^{-1}h_1
        = k_2k_1^{-1}$, with neither side equal to $1$. This is a contradiction,
        since $H \cap K = \{1\}$.

        Consider $G = D_3$, $H = \{1, s\}$, $K = \{1, r, r^2\}$. Note that $H \times
        K$ is the cyclic group of order $6$, and hence cannot be isomorphic to
        $D_3$.
    \end{solution}

    \paragraph{Exercise 10.} Let $x \in G$ have order $m$, and let $y \in G'$ have
    order $n$. What is the order of $(x, y)$ in $G \times G'$?
    \begin{solution}
        We claim that the desired order is $\lcm(m, n) = mn / d$, where $d = \gcd(m,
        n)$. Clearly, \[
            (x, y)^{mn / d} = ((x^m)^{n / d}, (y^n)^{m / d}) = (1, 1),
        \] so the order of $(x, y)$ must divide $mn / d$. Suppose that the order of
        $(x, y)$ is $mn / dt$ for $t > 1$. This means that both $x^{mn / dt} = 1$
        and $y^{mn / dt} = 1$, so $m$ and $n$ both divide $mn / dt$. Thus, $n / dt$
        and $m / dt$ are both integers, meaning that $dt$ is a common factor of $m$
        and $n$, contradicting the maximality of $d$ as the greatest common factor.
    \end{solution}
    
    \paragraph{Exercise 11.} Let $H$ be a subgroup of a group $G$, and let
    $\varphi\colon G \to H$ be a homomorphism whose restriction to $H$ is the
    identity map: $\varphi(h) = h$, if $h \in H$. Let $N = \ker{\varphi}$.
    \begin{enumerate}
        \itemsep0em
        \item Prove that if $G$ is abelian then it is isomorphic to the product
        group $H \times N$.
        \item Find a bijective map $G \to H \times N$ without the assumption that
        $G$ is abelian, but show by an example that $G$ need not be isomorphic to
        the product group.
    \end{enumerate}
    \begin{solution} \mbox{}
    \begin{enumerate}
        \item We claim that $HN = G$. Recall that $N$ is a normal subgroup, so $HN =
        NH$ hence $HN$ is a subgroup of $G$. Now, suppose that $g \in G$ such that
        $g \notin HN$, i.e.\ there is no choice of $h \in H$, $n \in N$ such that $g
        = hn$. Thus, $g' = \varphi(g) \notin H$ (if it were, then $\varphi(g) = h =
        \varphi(h)$, so $\varphi(gh^{-1}) = 1$ hence $gh^{-1} = n$ for some $n \in
        N$). This is a contradiction, since $\im{\varphi} = H$, proving that $HN =
        G$. Furthermore, $H \cap N = \{1\}$; to see this, note that for all $x \in H
        \cap N$, $\varphi(x) = x$ and $\varphi(x) = 1$. Thus, every element $g \in
        G$ has a unique decomposition $g = hn$ where $h \in H$, $n \in N$. If $g =
        hn = h'n'$, then $h'^{-1}h = n'n^{-1}$ with the left side in $H$ and the
        right in $N$, forcing both to be equal to $1$ hence $h = h'$, $n = n'$.

        The above establishes that the map $\psi\colon G \to H\times N$, $g
        \rightsquigarrow (h, n)$ where $g = hn$ with $h \in H$, $n \in N$ is a
        bijection. To show that this is an isomorphism, note that for $g_1 = h_1n_1$
        and $g_2 = h_2n2$, we have $g_1g_2 = h_1n_1h_2n_2 = h_1h_2n_1n_2$ hence
        $\psi(g_1g_2) = (h_1h_2, n_1n_2) = \psi(g_1)\psi(g_2)$, using the fact that
        $G$ is abelian. Thus, $G$ is isomorphic to $H \times N$.

        \item The map $\psi$ described earlier is the required bijection. It need
        not be an isomorphism when $G$ is not abelian; choose $G$, $H$, $N$ as in
        Exercise~4(b), i.e.\ let $G$ be the set of invertible upper triangular
        matrices, let $H$ be the set of invertible diagonal matrices, and let $N$ be
        the set of upper triangular matrices with diagonal entries $1$. With the
        homomorphism $\varphi\colon G \to H$, \[
            \begin{bmatrix}
                a & b \\ 0 & d
            \end{bmatrix} \rightsquigarrow \begin{bmatrix}
                a & 0 \\ 0 & d
            \end{bmatrix},
        \] where $ad \neq 0$, we see that $\varphi$ fixes $H$, with $N$ being the
        kernel. Note that $\varphi$ is indeed a homomorphism, because \[
            \begin{bmatrix}
                a_1 & b_1 \\ 0 & d_1
            \end{bmatrix} \begin{bmatrix}
                a_2 & b_2 \\ 0 & d_2
            \end{bmatrix} = \begin{bmatrix}
                a_1a_2 & a_1b_2 + b_1d_2 \\ 0 & d_1d_2
            \end{bmatrix} \rightsquigarrow \begin{bmatrix}
                a_1a_2 & 0 \\ 0 & d_1d_2
            \end{bmatrix},
        \] \[
            \begin{bmatrix}
                a_1 & 0 \\ 0 & d_1
            \end{bmatrix} \begin{bmatrix}
                a_2 & 0 \\ 0 & d_2
            \end{bmatrix} = \begin{bmatrix}
                a_1a_2 & 0 \\ 0 & d_1d_2
            \end{bmatrix}.
        \] However, as we have already noted earlier, the product group $H \times N$
        is not isomorphic to $G$, the former being abelian while the latter is not.
    \end{enumerate}
    \end{solution}  


    \section{Modular Arithmetic}

    \paragraph{Exercise 1.} Compute $(7 + 14)(3 - 16)$ modulo 17.
    \begin{solution}
        We have \[
            (7 + 14)(3 - 16) \equiv (21)(-13) \equiv (4)(4) \equiv 16 \pmod{17}.
        \] 
    \end{solution}

    \paragraph{Exercise 2.} \mbox{}
    \begin{enumerate}
        \itemsep0em
        \item Prove that the square $a^2$ of an integer $a$ is congruent to 0 or 1
        modulo 4.
        \item What are the possible values of $a^2$ modulo $8$?
    \end{enumerate}
    \begin{solution} \mbox{}
    \begin{enumerate}
        \item Note that $a \bmod{4}$ is exactly one of $0, 1, 2, 3$, hence the
        corresponding possibilities for $a^2 \bmod{4}$ are $0, 1, 4, 9$, which reduce
        to just $0$ or $1$ modulo 4.
        \item Again, the possibilities for $a\bmod{8}$ are $0, \pm 1, \pm 2, \pm 3,
        4$ only, hence $a^2 \bmod{8}$ must be one of $0, 1, 4, 9, 16$ which reduce to
        $0, 1, 4$ modulo 8.
    \end{enumerate}
    \end{solution}

    \paragraph{Exercise 3.} \mbox{}
    \begin{enumerate}
        \itemsep0em
        \item Prove that $2$ has no inverse modulo $6$.
        \item Determine all integers $n$ such that $2$ has an inverse modulo $n$.
    \end{enumerate}
    \begin{solution} \mbox{}
    \begin{enumerate}
        \item We need only check the products of $2$ with $0, 1, 2, 3, 4, 5$, which
        are $0, 2, 4, 6, 8, 10$, reducing to just $0, 2, 4$ modulo $6$. Thus, $2$ has
        no inverse modulo $6$.
        \item For all odd natural numbers $n = 2k - 1$, we have $2\cdot k \equiv 1
        \pmod{n}$.

        For all even integers $n = 2k$, note that $2\cdot m \equiv 1 \pmod{n}$ would
        imply that $2m - 1 = n\ell = 2k\ell$ for some multiplier $\ell$, which is
        absurd because the left side is odd while the right is even.
    \end{enumerate}
    \end{solution}

    \paragraph{Exercise 4.} Prove that every integer $a$ is congruent to the sum of
    its decimal digits modulo 9.
    \begin{solution}
        Note that $10 \equiv 1 \pmod{9}$, hence $10^k \equiv 1 \pmod{9}$ for all
        powers $10^k$. Thus, given any integer $a$, we can expand it in base $10$ as
        \[
            a = \sum_{k = 0}^n a_k 10^k \equiv \sum_{k = 1}^n a_k \pmod{9},
        \] but the latter sum is simply the sum of all the decimal digits of $a$.
        Here, $n$ denotes the highest power of $10$ in $a$.
    \end{solution}
    
    \paragraph{Exercise 5.} Solve the congruence $2x = 5$ (a) modulo 9 and (b) modulo
    6.
    \begin{solution} \mbox{}
    \begin{enumerate}
        \item We can have $x = 0, \pm 1, \pm 2, \pm 3, \pm 4$, hence $2x = 0, \pm 1,
        \pm 4, \pm 6, \pm 8$ $\equiv 0, \pm 1, \pm 4, \mp 3, \mp 1 \pmod{9}$. Now, $5
        \equiv -4 \pmod{9}$, so the only solutions are $x = -2 \equv 7 \pmod{9}$.
        Thus, the complete class of solutions is $x = 9k + 7$, where $k \in \Z$.

        \item There are no solutions, since $2x \equiv 5 \pmod{6}$ requires $2x = 5 +
        6k$ for some $k$, which is absurd since the left side is even but the right
        is odd.
    \end{enumerate}
    \end{solution}
    
    \paragraph{Exercise 6.} Determine the integers $n$ for which the congruences $x +
    y = 2$, $2x - 3y = 3$ (modulo $n$) have a solution.
    \begin{solution}
        By adding and subtracting the equations, we obtain the congruences $5x = 9$,
        $5y = 1$ modulo $n$. Thus, $5x + pn = 9$, $5y + qn = 1$ for integers $p, q$.
        The second Diophantine equation admits solutions if and only if $\gcd(5, n)$
        divides $1$, i.e.\ $n$ is co-prime with $5$ (this is made obvious by the fact
        that if $5$ divided $n$, then $5x + qn$ would be a multiple of $5$ but $1$ is
        not).

        We thus examine the four possibilities of $n = 1, 2, 3, 4$ modulo 5. In each,
        we can find integers $a$, $b$ such that $5a + nb = 1$, i.e.\ $5a \equiv 1$
        modulo $n$; $5$ is invertible in $\Z_n^\times$. Thus immediately gives the
        solution $x = 9a$, $y = a$. Thus, our equations have solutions precisely when
        $n$ is not a multiple of $5$.
    \end{solution}
    
    \paragraph{Exercise 7.} Prove the associative and commutative laws for
    multiplication in $\Z / n\Z$.
    

    \paragraph{Exercise 8.} Use Proposition~(2.6) to prove the \textit{Chinese
    Remainder Theorem}: Let $m, n, a, b$ be integers, and assume that the greatest
    common divisor of $m$ and $n$ is $1$. Then there is an integer $x$ such that $x =
    a \pmod{m}$ and $x = b \pmod{n}$.
    \begin{solution}
        Since $\gcd(m, n) = 1$, we find integers $p, q$ such that $pm + qn = 1$.
        Thus, with $p' = (b - a)p$ and $q' = (b - a)q$, we have $p'm + q'n = b - a$,
        so set $a + p'm = b - q'n = x$. This clearly satisfies the desired
        conditions.
    \end{solution}



    \section{Quotient Groups}
    
    \paragraph{Exercise 1.} Let $G$ be the group of invertible real upper triangular
    $2 \times 2$ matrices. Determine whether or not the following conditions describe
    normal subgroups $H$ of $G$. If they do, use the First Isomorphism Theorem to
    identify the quotient group $G/H$. \\
    (a) $a_{11} = 1$ (b) $a_{12} = 0$ (c) $a_{11} = a_{22}$ (d) $a_{11} = a_{22} =
    1$.
    \begin{solution}
        It is useful to note that multiplication in $G$ is of the form \[
            \begin{bmatrix}
                a & b \\ 0 & d
            \end{bmatrix} \begin{bmatrix}
                a' & b' \\ 0 & d'
            \end{bmatrix} = \begin{bmatrix}
                aa' & ab' + bd' \\ 0 & dd'
            \end{bmatrix}, \qquad
            \begin{bmatrix}
                a & b \\ 0 & d
            \end{bmatrix}^{-1} = \begin{bmatrix}
                1 / a & -b / ad \\ 0 & 1 / d
            \end{bmatrix}.
        \] Specifically, the $a_{11}$ and $a_{22}$ terms all multiply without
        interference from the other terms. Invertibility guarantees $ad \neq 0$.
        \begin{enumerate}
            \item Consider the general conjugate \[
                \begin{bmatrix}
                    a & b \\ 0 & d
                \end{bmatrix} \begin{bmatrix}
                    1 & x \\ 0 & y
                \end{bmatrix} \begin{bmatrix}
                    1 / a & -b / ad \\ 0 & 1 / d
                \end{bmatrix}.
            \] Since the $a_{11}$ and $a_{22}$ elements multiply in order, the result
            also obeys $a_{11} = (a)(1)(1 / a) = 1$, hence all conjugates are in $H$
            making it a normal subgroup of $G$.

            Consider the map $\varphi\colon G \to G$, defined by \[
                \begin{bmatrix}
                    a & b \\ 0 & d
                \end{bmatrix} \rightsquigarrow \begin{bmatrix}
                    a & 0 \\ 0 & 1
                \end{bmatrix}.
            \] This is a homomorphism, since \[
                \begin{bmatrix}
                    a & b \\ 0 & d
                \end{bmatrix} \begin{bmatrix}
                    a' & b' \\ 0 & d'
                \end{bmatrix} = \begin{bmatrix}
                    aa' & ab' + bd' \\ 0 & dd'
                \end{bmatrix} \rightsquigarrow \begin{bmatrix}
                    aa' & 0 \\ 0 & 1
                \end{bmatrix} = \begin{bmatrix}
                    a & 0 \\ 0 & 1
                \end{bmatrix} \begin{bmatrix}
                    a' & 0 \\ 0 & 1
                \end{bmatrix}.
            \] The kernel of this map is clearly $H$, and the image is the set of all
            diagonal matrices with $a_{22} = 1$. Call this subgroup $H'$, and note
            that this is isomorphic to $\R^\times$ (the multiplicative group on
            $\R\setminus\{0\}$) via the isomorphism \[
                \begin{bmatrix}
                    a & 0 \\ 0 & 1
                \end{bmatrix} \rightsquigarrow a.
            \] Thus, the First Isomorphism Theorem produces the identification $G /H
            \sim H'$, and $H' \sim \R^\times$. This means that the quotient group
            $G/H$ is simply $\R^\times$ upto isomorphism.
            
            \item Consider the general conjugate \[
                \begin{bmatrix}
                    a & b \\ 0 & d
                \end{bmatrix} \begin{bmatrix}
                    x & 0 \\ 0 & y
                \end{bmatrix} \begin{bmatrix}
                    1 / a & -b / ad \\ 0 & 1 / d
                \end{bmatrix} = \begin{bmatrix}
                    a & b \\ 0 & d
                \end{bmatrix} \begin{bmatrix}
                    x / a & -xb / ad \\ 0 & y / d
                \end{bmatrix} = \begin{bmatrix}
                    x & -xb / d + yb / d \\ 0 & y
                \end{bmatrix}.
            \] In general, the resultant $a_{12} = (y - x) b / d \neq 0$.
            Specifically, choose $a = b = d = 1$, and $x = 1$, $y = 2$. Thus, $H$ is
            not a normal subgroup of $G$.
            
            \item Consider the general conjugate \[
                \begin{bmatrix}
                    a & b \\ 0 & d
                \end{bmatrix} \begin{bmatrix}
                    x & y \\ 0 & x
                \end{bmatrix} \begin{bmatrix}
                    1 / a & -b / ad \\ 0 & 1 / d
                \end{bmatrix}.
            \] Again, the diagonal elements in the result must be $a_{11} = (a)(x)(1
            / a) = x$, and $a_{22} = (d)(x)(1 / d) = x$, which means that the
            conjugate is in $H$. This makes $H$ a normal subgroup of $G$.

            Consider the map $\varphi \colon G \to G$, defined by \[
                \begin{bmatrix}
                    a & b \\ 0 & d
                \end{bmatrix} \rightsquigarrow \begin{bmatrix}
                    a / d & 0 \\ 0 & 1
                \end{bmatrix}.
            \] This is a homomorphism because \[
                \begin{bmatrix}
                    a & b \\ 0 & d
                \end{bmatrix} \begin{bmatrix}
                    a' & b' \\ 0 & d'
                \end{bmatrix} = \begin{bmatrix}
                    aa' & ab' + bd' \\ 0 & dd'
                \end{bmatrix} \rightsquigarrow \begin{bmatrix}
                    aa' / dd' & 0 \\ 0 & 1
                \end{bmatrix} = \begin{bmatrix}
                    a / d & 0 \\ 0 & 1
                \end{bmatrix} \begin{bmatrix}
                    a' / d' & 0 \\ 0 & 1
                \end{bmatrix}.
            \] The kernel of $\varphi$ is clearly $H$, and the image is the set of
            all diagonal matrices with $a_{22} = 1$, which we have already seen is
            isomorphic to $\R^\times$. Thus, $G / H$ is $\R^\times$ upto isomorphism.

            \item Following the previous part, we see that any conjugate of the form \[
                \begin{bmatrix}
                    a & b \\ 0 & d
                \end{bmatrix} \begin{bmatrix}
                    1 & y \\ 0 & 1
                \end{bmatrix} \begin{bmatrix}
                    1 / a & -b / ad \\ 0 & 1 / d
                \end{bmatrix}
            \] belongs to $H$, with both diagonal elements equal to $1$. This makes
            $H$ a normal subgroup of $G$.

            Consider the map $\varphi\colon G \to G$, defined by \[
                \begin{bmatrix}
                    a & b \\ 0 & d
                \end{bmatrix} \rightsquigarrow \begin{bmatrix}
                    a & 0 \\ 0 & d
                \end{bmatrix}.
            \] This is a homomorphism, because \[
                \begin{bmatrix}
                    a & b \\ 0 & d
                \end{bmatrix} \begin{bmatrix}
                    a' & b' \\ 0 & d'
                \end{bmatrix} = \begin{bmatrix}
                    aa' & ab' + bd' \\ 0 & dd'
                \end{bmatrix} \rightsquigarrow \begin{bmatrix}
                    aa' & 0 \\ 0 & dd'
                \end{bmatrix} = \begin{bmatrix}
                    a & 0 \\ 0 & d
                \end{bmatrix} \begin{bmatrix}
                    a' & 0 \\ 0 & d'
                \end{bmatrix}.
            \] The kernel of $\varphi$ is $H$, and the image is the set of all
            ($2\times 2$) diagonal matrices. This in turn is isomorphic to $\R^\times
            \times \R^\times$,
            via the isomorphism \[
                \begin{bmatrix}
                    a & 0 \\ 0 & d
                \end{bmatrix} \rightsquigarrow (a, d).
            \] Thus, $G / H$ is the product group $\R^\times \times \R^\times$ upto
            isomorphism.
        \end{enumerate}
    \end{solution}

    \paragraph{Exercise 2.} Write out the proof of (10.1) in terms of elements.
    \begin{quote}
        Let $N$ be a normal subgroup of a group $G$. Then the product of two cosets
        $aN$, $bN$ is again a coset, in fact \[
            (aN)(bN) = abN.
        \]
    \end{quote}
    \begin{solution}
        Fix $a, b \in G$.

        First, pick an element from the left hand side, which must be of the form
        $x = an_1bn_2$ for some $n_1, n_2 \in G$. From the normality of $N$, we can
        choose $n_1b = bn_1'$, hence $x = abn_1'n_2 = (ab)(n_1'n_2)$, which is an
        element of $anN$. This gives $(aN)(bN) \subseteq abN$.

        Next, pick an element from the right hand side, which must be of the form
        $abn$ for some $n \in N$> Simply rewriting this as $(a1)(bn)$ shows that this
        is an element of $(aN)(bN)$, giving $(aN)(bN) \supseteq abN$.
    \end{solution}

    \paragraph{Exercise 3.} Let $P$ be a partition of a group $G$ with the property
    that for any pair of elements $A$, $B$ of the partition, the product $AB$ is
    contained entirely within another element $C$ of the partition. Let $N$ be the
    element of $P$ which contains $1$. Prove that $N$ is a normal subgroup of $G$ and
    that $P$ is the set of its cosets.
    \begin{solution}
        To show that $N$ is a subgroup of $G$, first note that $1 \in N$ by
        construction. Next, if $x \in N$ but $x^{-1} \notin N$, suppose $x^{-1} \in
        A$ for some $A \in P$. Then, $1 = x^{-1}x \in AN \subseteq C$ for some $C
        \in P$, but $N$ is the only element of the partition containing $1$, forcing
        $C = N$. This immediately gives $x^{-1}1 \in N$, a contradiction. Thus, we
        have $x^{-1} \in N$. Finally, pick $x, y \in N$ and suppose that the product
        $xy \notin A \neq N$ for some $A \in P$. Then, $x = (xy)y^{-1} \in AN
        \subseteq C$ for some $C \in P$, but as before, the only element of $P$
        containing $x$ is $N$. This means that $C = N$, so $(xy)1 \in N$, a
        contradiction. Thus, we have $xy \in N$.

        Next, to show that $N$ is a normal subgroup of $G$, pick $x\in N$ and
        consider the conjugate $axa^{-1}$ for some $a \in A \in P$. Note that this
        choice of $A$ is unique; if $A = N$, clearly $axa^{-1} \in N$ as desired.
        Otherwise, note that $ax \in AN \subseteq C$ for some $C \in P$. This set $C$
        also contains $a 1 \in AN$, hence we must have $C = A$. Thus, $axa^{-1} \in
        A A' \subseteq D$, where $A'\in P$ is chosen to contain $a^{-1}$ and $D \in
        P$. Now, the element $1 = a a^{-1} \in A A' \subseteq D$, which forces $D =
        N$. As a result, $axa^{-1} \in N$, making $N$ a normal subgroup of $G$.

        To show that every element of $P$ is a coset of $N$, pick $A \in P$. If $A =
        N$, we are done, otherwise pick some $a \in A$. We claim that $A = aN$.
        Recall that we have already shown that $AN \subseteq A$, hence $aN \subseteq
        AN \subseteq A$. Next, pick some $b \in A$, and let $A' \in P$ be the set
        containing $a^{-1}$. As before, $1 \in A A' \subseteq C$ for some $C \in P$,
        forcing $C = N$. Thus, $ba^{-1} \in A A' \subseteq N$, hence $b \in aN$.
        This proves that $A \subseteq aN$, thus $A = aN$.

        Note that we have shown that every element in $P$ is a coset of $N$.
        Furthermore, $P$ must contain every coset of $N$, since every $a \in G$
        belongs to exactly one $A \in P$, hence $aN = A \in P$ for every coset $aN$.
    \end{solution}

    \paragraph{Exercise 4.} \mbox{}
    \begin{enumerate}
        \itemsep0em
        \item Consider the presentation (1.17) of the symmetric group $S_3$. Let $H$
        be the subgroup $\{1, y\}$. Compute the product sets $(1H)(xH)$ and
        $(1H)(x^2H)$, and verify that they are not cosets.
        \item Show that a cyclic group of order $6$ has two generators satisfying the
        rules $x^3 = 1$, $y^2 = 1$, $yx = xy$.
        \item Repeat the computation of (a), replacing the relations (1.18) by the
        relations given in part (b). Explain.
    \end{enumerate}
    \begin{solution} \mbox{}
    \begin{enumerate}
        \item Recall that we have $x^3 = 1$, $y^2 = 1$, $yx = x^2y$. For $(1H)(xH)$,
        compute $1x1 = x$, $1xy = xy$, $yx1 = x^2y$, $yxy = x^2y y = x^2$ hence
        $(1H)(xH) = \{x, x^2, xy, x^2y\}$. Similarly for $(1H)(x^2H)$, compute $1x^21
        = x^2$, $1x^2y = x^2y$, $yx^21 = x^2yx = x^4y = xy$, $yx^2y = xy y = x$ hence
        $(1H)(x^2H) = \{x, x^2, xy, x^2y\}$. This cannot be a coset of any subgroup
        of $S_3$, since any coset must contain the same number of elements as the
        corresponding subgroup but Lagrange's Theorem prevents any subgroup of $S_3$
        from having 4 elements.

        \item Let $C_6 = \langle a \rangle$, and set $x = a^2$, $y = a^3$. Then we
        clearly have $x^3 = y^2 = 1$ and $yx = xy = a^5$. Furthermore, $x$ and $y$ do
        indeed generate $C_6$, because $yx^{-1} = a$ generates $C_6$.

        \item For $(1H)(xH)$, compute $1x1 = x$, $1xy = xy$, $yx1 = xy$, $yxy = xy y
        = x$ hence $(1H)(xH) = \{x, xy\}$. Similarly for $(1H)(x^2H)$, compute $1x^21
        = x^2$, $1x^2y = x^2y$, $yx^21 = xyx = x^2y$, $yx^2y = x^2y y = x^2$ hence
        $(1H)(x^2H) = \{x^2, x^2y\}$. Now, the first product is the coset $xH$ and
        the second is the product $x^2H$.

        Note that $C_6$ is abelian, which guarantees that $H$ is a normal subgroup.
        Thus, Proposition~$10.1$ ensures that the product of cosets of $H$ is another
        coset of $H$.
    \end{enumerate}
    \end{solution}

    \paragraph{Exercise 5.} Identify the quotient group $\R^\times / P$, where $P$
    denotes the subgroup of positive real numbers.
    \begin{solution}
        Note that the $\sgn$ map where $\sgn{x} = x / |x|$ is a homomorphism whose
        kernel is precisely $P$. Thus, the First Isomorphism Theorem shows that
        $\R^\times /P$ is isomorphic to $\im(\sgn) = \{\pm 1\}$, the cyclic group of
        two elements.
    \end{solution}

    \paragraph{Exercise 6.} Let $H = \{\pm 1, \pm i\}$ be the subgroup of $G =
    \C^\times$ of fourth roots of unity. Describe the cosets of $H$ in $G$
    explicitly, and prove that $G / H$ is isomorphic to $G$.
    \begin{solution}
        Given some $z \in \C^\times$, the coset $zH = \{\pm z, \pm iz\}$. This
        represents all four rotations of $z$ in the complex plane by multiples of
        $\pi / 2$.

        Consider the map $\varphi\colon \C^\times \to \C^\times$, defined by $z
        \rightsquigarrow z^4$. Note that this is a homomorphism, since
        $\varphi(x)\varphi(y) = x^4y^4 = (xy)^4 = \varphi(xy)$ for all $x, y \in
        \C^\times$. It's kernel consists of the solutions of $z^4 = 1$, which is
        precisely $H$. Furthermore, $\varphi$ is surjective since every $z \in
        \C^\times$ can be written in the form $re^{i\theta}$ for $r > 0$, hence
        $\varphi(r^{1 / 4}e^{i\theta / 4}) = z$. Thus, the First Isomorphism Theorem
        shows that $G / H$ is isomorphic to $G$.
    \end{solution}

    \paragraph{Exercise 7.} Find all normal subgroups $N$ of the quaternion group
    $H$, and identify the quotients $H / N$.
    \begin{solution}
        Recall that in Section~4,~Exercise~8(b), we have shown that all subgroups of
        the quaternion group are normal. We consider all possible cases below.
        \begin{itemize}
            \itemsep0em
            \item For $N = \{1\}$, we trivially have $H / N \sim H$.

            \item For $N = \{\pm 1\}$, note that the cosets of $N$ are $\{\pm 1\}$,
            $\{\pm i\}$, $\{\pm j\}$ and $\{\pm k\}$. It is clear that $H / N$ is
            isomorphic to the Klein four group, since we have the relations$\{\pm
            i\}\cdot \{\pm j\} = \{\pm k\} = \{\pm j\}\cdot \{\pm k\}$ and all other
            similar ones.

            \item For $N = \{\pm 1, \pm i\}$, note that the cosets of $N$ are $\{\pm
            1, \pm i\}$ and $\{\pm j, \pm k\}$. Thus, $H / N$ is isomorphic to $C_2$.
            The cases replacing $i$ with $j$ or $k$ are identical.

            \item For $N = H$, we trivially have $H / N \sim \{1\}$.
        \end{itemize}
    \end{solution}
    
    \paragraph{Exercise 8.} Prove that the subset $H$ of $G = GL_n(\R)$ or matrices
    whose determinant is positive forms a normal subgroup, and describe the quotient
    group $G / H$.
    \begin{solution}
        Consider the map $\varphi\colon GL_n(\R) \to \{\pm 1\}$, defined by $X
        \rightsquigarrow \sgn(\det{X})$. This is clearly a homomorphism, being the
        composition of two homomorphisms. Furthermore, the kernel of $\varphi$ is
        precisely $H$, and $\varphi$ is surjective because $\pm \mathbb{I}
        \rightsquigarrow \pm 1$. Thus, $H$ must be a normal subgroup, and $G / H$
        must be isomorphic to $\{\pm 1\}$.
    \end{solution}

    \paragraph{Exercise 9.} Prove that the subset $G \times 1$ of the product group
    $G \times G'$ is a normal subgroup isomorphic to $G$ and that $(G \times G') / (G
    \times 1)$ is isomorphic to $G'$.
    \begin{solution}
        Consider the map $\varphi\colon G \times G' \to G'$, defined by $(x, y)
        \rightsquigarrow y$. This is a homomorphism since it is a projection
        map; furthermore, it's kernel is precisely the set $G \times 1$, and
        $\varphi$ is surjective because every $y \in G'$ has the pre-image $(1, y)
        \in G\times G'$. Thus, $G \times 1$ is a normal subgroup of $G \times G'$,
        and the quotient group $(G \times G') / (G \times 1)$ is isomorphic to $G'$.
    \end{solution}

    \paragraph{Exercise 10.} Describe the quotient groups $\C^\times / P$ and
    $\C^\times / U$, where $U$ is the subgroup of complex numbers of absolute value
    $1$ and $P$ denotes the positive reals.
    \begin{solution}
        First, let $\varphi\colon \C^\times \to U$ be the map defined by $z
        \rightsquigarrow z / |z|$. It is clear that this is a homomorphism, and its
        kernel is the set of solutions of $z = |z|$, i.e.\ precisely the set $P$.
        Furthermore, this map is surjective, since every $e^{ix} \in U$ has itself as
        a pre-image. Thus, the quotient group $\C^\times / P$ is isomorphic to $U$,
        i.e.\ the unit circle/set of rotations in 2D.

        Next, let $\varphi\colon \C^\times \to P$ be the absolute value map. We have
        already shown that this is a homomorphism. It's kernel is the set of
        solutions of $|z|= 1$, i.e.\ precisely the set $U$. Furthermore, this map is
        surjective, fixing $P$. Thus, the quotient group $\C^\times / U$ is isomorphic
        to $P$, i.e.\ the multiplicative group of positive reals.
    \end{solution}

    \paragraph{Exercise 11.} Prove that the groups $\R^+ / \Z^+$ and $\R^+ / 2\pi\Z^+$
    are isomorphic.
    \begin{solution}
        We claim that all groups of the form $\R^+ / t\Z^+$ for non-zero $t$ are
        isomorphic (note that without loss of generality, we can choose positive
        $t$). Let $\varphi_t\colon\colon \R^+ \to U$ be the map define by $x
        \rightsquigarrow e^{2\pi i x / t}$. This is clearly a homomorphism since
        $e^{x + y} = e^{x}e^{y}$. Furthermore, the kernel of $\varphi_t$ is the set
        of solutions to $e^{2\pi ix / t} = 1$, i.e.\ $x = tn$ for integers $n$, which
        is precisely the set $t\Z^+$; we also see that $\varphi_t$ is surjective
        since any $e^{ix} \in U$ has the pre-image $tx / 2\pi \in \R^+$. Thus, all
        quotient groups of the form $\R^/t\Z^+$ are isomorphic to the unit circle,
        and hence to each other.
    \end{solution}
    
    

\end{document}
% vim: set tabstop=4 shiftwidth=4 softtabstop=4:
