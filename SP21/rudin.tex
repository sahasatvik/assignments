\documentclass[11pt]{report}

\usepackage[T1]{fontenc}
\usepackage{geometry}
\usepackage{amsmath, amssymb, amsthm}
\usepackage{bm}
\usepackage[scr]{rsfso}
\usepackage{xcolor}
\usepackage{fancyhdr}
\usepackage{hyperref}

\geometry{a4paper, margin=1in, headheight=14pt}

\pagestyle{fancy}
\renewcommand\headrulewidth{0.4pt}
\fancyhead[L]{\it Principles of Mathematical Analysis}
\rfoot{\footnotesize\it Updated on \today}
\cfoot{\thepage}
\renewcommand{\chaptermark}[1]{\markboth{#1}{}}

\renewcommand{\labelenumi}{(\alph{enumi})}
\renewcommand{\labelenumii}{(\roman{enumii})}

\def\C{\mathbb{C}}
\def\R{\mathbb{R}}
\def\Q{\mathbb{Q}}
\def\Z{\mathbb{Z}}
\def\N{\mathbb{N}}

\def\O{\mathscr{O}}

\renewcommand\vec\boldsymbol
\def\vx{\vec{x}}
\def\vy{\vec{y}}
\def\vz{\vec{z}}
\def\va{\vec{a}}
\def\vb{\vec{b}}
\def\vc{\vec{c}}
\def\vr{\vec{r}}

\theoremstyle{remark}
\newtheorem*{remark}{Remark}
\newtheorem*{example}{Example}
\newtheorem*{solution}{Solution}

\title{
    \Large\textsc{Summer Programme 2021} \\
    \vspace{10pt}
    \huge Solutions to exercises from Walter Rudin's \\
    \textit{Principles of Mathematical Analysis} 
}
\author{
    \large Satvik Saha%
    % \thanks{Email: \tt ss19ms154@iiserkol.ac.in}
    \\\textsc{\small 19MS154}
}
\date{\normalsize
    \textit{Indian Institute of Science Education and Research, Kolkata, \\
    Mohanpur, West Bengal, 741246, India.} \\
    % \vspace{10pt}
    % \today
}

\begin{document}
    \maketitle

    \chapter{The Real and Complex Number Systems}

    \paragraph{Exercise 1.} If $r$ is rational ($r \neq 0$) and $x$ is irrational,
    prove that $r + x$ and $rx$ are irrational.
    \begin{solution}
        Use the fact that the field of rationals is closed under additiona and
        multiplication, as well as the existence of the additive inverse $-r$ and
        the multiplicative inverse $1 / r$. If $r + x$ and $rx$ were rational, then
        both \[
            (-r) + r + x = x, \qquad (1 / r) rx = x
        \] must also be rational. These are contradictions.
    \end{solution}

    \paragraph{Exercise 2.} Prove that there is no rational number whose square is
    $12$.
    \begin{solution}
        Suppose that $x \in \Q$, $x^2 = 12$, and $x = p / q$ where $q \neq 0$ and
        $p$ and $q$ are coprime integers. This would imply that \[
            p^2 = 12q^2 = 3(2q)^2,
        \] so $3$ divides $p^2$, hence $3$ divides $p$. Write $p = 3m$ for some
        integer $m$, giving \[
            3(2q)^2 = p^2 = (3m)^2 = 9m^2, \qquad (2q)^2 = 3m^2.
        \] This means that $3$ divides $(2q)^2$, hence $3$ divides $2q$, hence $3$
        divides $q$. This contradicts the fact that $p$ and $q$ are coprime, which
        means that there is no rational number whose square is $12$.
    \end{solution}

    \paragraph{Exercise 3.} Prove that the axioms of multiplication in a field imply
    the following statements.
    \begin{enumerate}
        \itemsep0em
        \item If $x \neq 0$ and $xy = xz$, then $y = z$.
        \item If $x \neq 0$ and $xy = x$, then $y = 1$.
        \item If $x \neq 0$ and $xy = 1$, then $y = 1 / x$.
        \item If $x \neq 0$ then $1 / (1 / x) = x$.
    \end{enumerate}
    \begin{solution}
        The axioms of multiplication guarantee the existence of an element $1 / x$
        such that $x (1 / x) = 1$. Left multiply on both sides of $xy = xz$, use
        associativity and $1w = w$ for all $w$ in the field to get \[
            (1 / x)xy = (1 / x)xz, \qquad y = z.
        \] This proves (a). Setting $z = 1$ proves (b), and setting $z = 1 / x$
        proves (c). Using $x(1 / x) = 1$, replace $x$ with $1 / x$ in (c)
        to give \[
            (1 / x)(1 / (1 / x)) = 1,
        \] then left multiply with $x$ yielding \[
            x(1 / x)(1 / (1 / x)) = x, \qquad 1 / (1 / x) = x.
        \] 
    \end{solution}
    
    \paragraph{Exercise 4.} Let $E$ be a non-empty subset of an ordered set; suppose
    that $\alpha$ is a lower bound of $E$ and $\beta$ is an upper bound of $E$.
    Prove that $\alpha \leq \beta$.
    \begin{solution}
        By definition, $\alpha \leq x$ for all $x \in E$ and $x \leq \beta$ for all
        $\x \in E$. Since $E$ is non-empty, simply select some $x \in E$, whence
        $\alpha \leq x \leq \beta$. Thus, we either have $\alpha = x = \beta$,
        $\alpha = x < \beta$, $\alpha < x = \beta$, or $\alpha < x < \beta$.
        In the last case, transitivity gives $\alpha < \beta$. Hence, $\alpha \leq
        \beta$.
    \end{solution}

    \paragraph{Exercise 5.} Let $A$ be a non-empty subset of the real numbers which
    is bounded below. Let $-A$ be the set of all numbers $-x$, where $x \in A$.
    Prove that \[
        \inf A = - \sup(-A).
    \] 
    \begin{solution}
        Fix $\alpha = -\sup(-A)$. We claim that $\alpha = \inf A$, i.e.\ $\beta \leq
        \alpha \leq x$ for all lower bounds $\beta$ of $A$ and for all $x \in A$.

        First, note that $-\alpha = \sup(-A)$, which means that $-\alpha \geq x$ for
        all $x \in -A$, whence $\alpha \leq -x$ for all $-x \in A$. However, for
        each $x \in A$, we have $-x \in -A$ so $\alpha \leq x$ for all $x \in A$.

        Now, let $\beta$ be a lower bound of $A$. This means that $\beta \leq x$ for
        all $x \in A$, so $-\beta \geq -x$ for all $x \in A$. Again, $-x \in -A$ for
        all $x \in A$, so $-\beta \geq x$ for all $x \in -A$. This means that
        $\beta$ is an upper bound of $-A$, which means $-\beta \geq \sup(-A) =
        -\alpha$. Thus, $\beta \leq \alpha$.

        This proves that $\inf A = -\sup(-A)$.
    \end{solution}

    \paragraph{Exercise 6.} Fix $b > 1$.
    \begin{enumerate}
        \item If $m, n, p, q$ are integers, $n > 0$, $q > 0$, and $r = m / n = p /
        q$, prove that \[
            (b^m)^{1 / n} = (b^p)^{1 / q}.
        \] Hence it makes sense to define $b^r = (b^m)^{1 / n}$.
        \item Prove that $b^{r + s} = b^rb^s$ if $r$ and $s$ are rational.
        \item If $x$ is real, define $B(x)$ to be the set of all numbers $b^t$,
        where $t$ is rational and $t \leq x$. Prove that \[
            b^r = \sup B(r).
        \] Hence is makes sense to define $b^x = \sup B(x)$ for every real $x$.
        \item Prove that $b^{x + y} = b^xb^y$ for all real $x$ and $y$.
    \end{enumerate}
    \begin{solution} \mbox{}
        \begin{enumerate}
            \item Write $r$ with the common denominator $s = nq$, so $r = mq / s =
            pn / s$. Now, note that \[
                \left((b^m)^{1 / n}\right)^s = (b^m)^{q} = b^{mq}, \qquad
                \left((b^p)^{1 / q}\right)^s = (b^p)^{n} = b^{np},
            \] but $mq = np = rs$. Setting $b^{rs} = x$, use Theorem~1.21 to conclude
            that there is a unique $y$ such that $y^s = x = b^{rs}$. However, we
            have just verified two such $y$, hence \[
                (b^m)^{1 / n} = (b^p)^{1 / q}.
            \]
            \item Set $r = m / n$, $s = p / q$ with $n > 0$, $q > 0$. Then, \[
                b^{r + s} = b^{(mq + np) / nq} = (b^{mq + np})^{1 / nq} =
                (b^{mq}b^{np})^{1 / nq}.
            \] The corollary of Theorem~1.21 lets us distribute the integer root
            over the product, giving \[
                b^{r + s} = b^{mq / nq} b^{np / nq} = b^{m / n}b^{p / q} = b^rb^s.
            \] 
            \item First, we show that $b^n - 1 \geq n(b - 1)$ for all positive integers
            $n$. This is trivially true for $n = 1$. For $n > 1$, write $b = 1 + a$
            where $a > 0$. Hence the Binomial Theorem gives \[
                b^n = (1 + a)^n = 1 + na + \frac{1}{2}n(n - 1)a^2 + \dots + a^n > 1
                + na,
            \] hence \[
                b^n - 1 > na = n(b - 1).
            \] Note that this inequality becomes strict for $n > 1$.
            Replacing $b$ with $b^{1 / n} > 1$, we have $b - 1 > n(b^{1 / n} - 1)$ 
            for all positive integers $n$.

            Now, given some $t > 1$, we can choose a positive integer 
            $n > (b - 1) / (t - 1)$, which implies $n(t - 1) > b - 1 > n(b^{1 / n} -
            1)$, hence $t > b^{1 / n}$.

            Now, note that for all $x \in B(r)$, $x = b^t$ for some rational $t$.
            First, note that for all rational $t \leq r$, we have $b^t \leq b^r$.
            This is because if we write $t$ and $r$ with a common positive integer 
            denominator, $t = m / q$, $r = n / q$, then $m \leq n$ so $(b^{1 / q})^m
            \leq (b^{1 / q})^n$. Thus, $b^r$ is an upper bound for $B(r)$.

            Next, we show that $b^r$ is the least upper bound to $B(r)$. Suppose
            that $\alpha = \sup B(r)$, and $b^t \leq \alpha < b^r$ for all $t \leq
            r$. Using the previously proven inequality, find a large enough integer
            $n$ such that $b^{1 / n} < b^r / \alpha$. Thus, $\alpha < b^{r - 1 /
            n}$, and $r - 1 / n < r$ so $b^{r - 1 / n} \in B(r)$, which contradicts
            the fact that $\alpha$ is the supremum of $B(r)$. Hence, $b^r$ is the
            least upper bound of $B(r)$, so \[
                b^r = \sup B(r).
            \] 
            \item We have been given \[
                b^x = \sup B(x), \qquad b^y = \sup B^y, \qquad b^{x + y} = \sup B(x + y)
            \] by definition for real $x$ and $y$. Choose some rational $t \leq x +
            y$, so $b^t \in B(x + y)$. By choosing a rational $r$ such that $t - y <
            r < x$ and setting $s = t - r$, we have $t = r + s$ and $r < x$, $s <
            y$. Thus, $b^r \in B(x)$ and $b^s \in B(y)$, so every element $b^t \in
            B(x + y)$ can be written as $b^{r + s} = b^rb^s$, which is the product
            of an element each from $B(x)$ and $B(y)$. Conversely, given elements
            $b^r \in B(x)$ and $b^s \in B(y)$, we have $r \leq x$ and $s \leq y$ so
            $t = r + s \leq x + y$, hence $b^{r + s} = b^t \in B(x + y)$. Thus,
            we have \[
                B(x + y) = \{wz: w\in B(x), z \in B(y)\}.
            \]

            Thus, for any element $wz \in B(x + y)$, $w \in B(x)$, $z \in B(y)$, we
            have $w \leq \sup B(x) = b^x$ and $z \leq \sup B(y) = b^y$, so $wz \leq
            b^x b^y$. This means that $b^xb^y$ is an upper bound of $B(x + y)$.

            Now suppose that $\alpha = \sup B(x + y)$ such that 
            $wz \leq \alpha < b^xb^y$ for all $wz \in B(x + y)$, where $w \in B(x)$
            and $z \in B(y)$. Then, $\alpha / b^x < b^y$, so choose $\beta$ such
            that $\alpha / b^x < \beta < b^y$. In other words, $\alpha / \beta <
            b^x$ and $\beta < b^y$, so we can choose rational $r < x$, $s < y$ such
            that $\alpha / \beta \leq b^r \in B(x)$ and $\beta \leq b^s \in B(y)$.
            Note that $r \neq x$ and $s \neq y$.
            Thus, the product $(\alpha / \beta)\beta = \alpha \leq b^rb^s \in B(x +
            y)$. However, recall that we chose $\alpha$ such that $b^rb^s \leq \alpha$
            for all $b^r \in B(x)$, $b^s \in B(y)$, so we must have $\alpha =
            b^rb^s$ for our choice of $r$ and $s$. Now, we can choose rational $r'$ and 
            $s'$ such that $r < r' < x$ and $s < s' < y$, hence $b^r < b^{r'} \in
            B(x)$ and $b^s < b^{s'} \in B(y)$. This gives $\alpha = b^rb^s <
            b^{r'}b^{s'} \in B(x + y)$, which contradicts the fact that $\alpha$ is
            an upper bound. Thus, $b^xb^y$ must be the least upper bound of $B(x +
            y)$, so \[
                b^{x + y} = b^xb^y.
            \] 
        \end{enumerate}
    \end{solution}
    
    \paragraph{Exercise 7.} Fix $b > 1$, $y > 0$, and show the following.
    \begin{enumerate}
        \itemsep0em
        \item For any positive integer $n$, $b^n - 1 \geq n(b - 1)$.
        \item Hence, $b - 1 \geq n(b^{1 / n} - 1)$.
        \item If $t > 1$ and $n > (b - 1)/(t - 1)$, then $b^{1 / n} < t$.
        \item If $w$ is such that $b^w < y$, then $b^{w + 1 / n} < y$ for
        sufficiently large $n$.
        \item If $b^w > y$, then $b^{w - 1 / n} > y$ for sufficiently large $n$.
        \item Let $A$ be the set of all $w$ such that $b^w < y$, and show that $x =
        \sup A$ satisfies $b^x = y$.
        \item Prove that this $x$ is unique.
    \end{enumerate}
    \begin{solution} \mbox{}
        \begin{enumerate}
            \item See Exercise 1 (c).
            \item See Exercise 1 (c).
            \item See Exercise 1 (c).
            \item Set $t = yb^{-w} > 1$, and using the previous inequality, choose
            sufficiently large $n$ such that $b^{1 / n} < t = yb^{-w}$. Thus, \[
                b^{w + 1 / n} < y.        
            \]
            \item Set $t = (1 / y)b^{w} > 1$, and using the inequality in (c),
            choose sufficiently large $n$ such that $b^{1 / n} < t = (1 / y)b^w$.
            Thus, \[
                y < b^{w - 1 / n}.
            \] 
            \item Exactly one of the following must be true; $b^x < y$, $b^x = y$,
            $b^x > y$. If $b^x < y$, then $x \in A$ by definition. Using (d), we can
            find sufficiently large $n$ such that \[
                b^{x + 1 / n} < y,
            \] hence $x < x + 1 / n \in A$, contradicting the fact that $x$ is an
            upper bound of $A$. If $b^x > y$, then using (e), we can find
            sufficiently large $n$ such that \[
                y < b^{x - 1 / n},
            \] which means that $x - 1 / n$ is also an upper bound of $A$,
            contradicting the fact that $x$ is the lowest upper bound of $A$.
            This leaves us with $b^x = y$.
            \item Suppose that $x \neq x'$, and without loss of generality $x < x'$.
            Set $x' - x = h > 0$, and note that $b^{x'} = b^{x + h} = b^x b^h$.
            Now, $b > 1$ and $h > 0$, so $b^h > 1$. Thus, $b^{x'} > b^x$, which
            means that $b^{x'} \neq b^x$ for $x' \neq x$. Thus, if $b^x = y$, then
            $x$ is unique.
        \end{enumerate}
    \end{solution}

    \paragraph{Exercise 8.} Prove that no order can be defined in the complex field
    that turns it into an ordered field.
    \begin{solution}
        In an ordered field, if $x > 0$, then we must have $-x < 0$, and vice versa
        by Proposition~1.18. The same proposition gives that if $x \neq 0$, then
        $x^2 > 0$. This forces $i^2 = -1 > 0$. Applying the same proposition again,
        this forces $(-1)^2 = 1 > 0$, which is a contradiction because we cannot
        have both $-1 > 0$ and $1 > 0$.
    \end{solution}
    
    \paragraph{Exercise 9.} Suppose $z = a + bi$, $w = c + di$. Define $z < w$ if $a
    < c$, and also if $a = c$ but $b = d$. Prove that this turns the set of all
    complex numbers into an ordered set. Does this ordered set have the
    least-upper-bound property?
    \begin{solution}
        First, we show that for arbitrary $z = a + bi$ and $w = c + di$, exactly one
        of the following is true: $z < w$, $z = w$, $z > w$. To do this, note that
        the real numbers are ordered, so either $a < c$, $a = c$, or $a > c$.
        In the case $a < c$, we have $z < w$ and since $a \neq c$, $z \neq w$. Also,
        this excludes $w < z$. In the case $a > c$, the roles of $z$ and $w$ are
        interchanged, so $z > w$. In the case $a = c$, we note that either $b < d$,
        $b = d$, or $b > d$; when $b < d$, $z < w$ and when $b > d$, $z > w$.
        Finally, when $a = c$ and $b = d$, we have $z = w$.

        Next, we show that transitivity holds, i.e.\ if $z < w$ and $w < x$, then $z
        < x$. Write $z = a + bi$, $w = c + di$ and $x = e + fi$. Note that the
        conditions $z < w$ and $w < x$ imply $a \leq c$ and $c \leq e$. This has to
        be further split into four cases.
        \begin{quote}
        \begin{description}
            \itemsep0em
            \item[Case 1] If $a < c$ and $c < e$, then $a < e$ so $z < x$.
            \item[Case 2] If $a = c$ and $c < e$, then $a < e$ again so $z < x$.
            \item[Case 3] If $a < c$ and $c = e$, then $a < e$ again so $z < x$.
            \item[Case 4] If $a = c$ and $c = e$, then we must have had $b < d$
            and $d < f$, so $a = e$ and $b < f$ gives $z < x$.
        \end{description}
        \end{quote}

        No, this ordered set does not have the least upper bound property. Consider
        the set of complex numbers $S = \{a + bi: 0 < a < 1, b = 0\}$. If $w = c +
        di$ is to be an upper bound of $S$, i.e.\ $z \leq w$ for all $z \in S$, then 
        either $z = w$ for some $z \in S$ or $z < w$ for all $z \in S$.
        The former implies that $w = a + 0i$ for some $0 < a < 1$, in which case we
        have $w = a + 0i < (a + 1) / 2 + 0i \in S$, a contradiction. The latter
        implies that $a \leq c$ for all $0 < a < 1$, which forces $1 \leq c$.
        If $w = c + di$ is the least upper bound of $S$ with $1 < c$, then note that
        $(1 + c) / 2 + di < c + di = w$ is smaller upper bound of $S$. Otherwise, if
        $w = 1 + di$ is the least upper bound of $S$, then $1 + (d - 1)i < 1 + di =
        w$ is a smaller upper bound. This means that the set $S$ have no least upper
        bound.
    \end{solution}
    
    \paragraph{Exercise 10.} Suppose $z = a + bi$, $w = u + vi$, and \[
        a = \left(\frac{|w| + u}{2}\right)^{1 / 2}, \qquad
        b = \left(\frac{|w| - u}{2}\right)^{1 / 2}.
    \] Prove that $z^2 = w$ if $v \geq 0$ and $(\overline{z})^2 = w$ if $v \leq 0$.
    Conclude that every complex number (with one exception) has two complex square
    roots.
    \begin{solution}
        Write \[
            z^2 = (a + bi)^2 = a^2 - b^2 + 2abi, \qquad
            \overline{z}^2 = (a - bi)^2 = a^2 - b^2 - 2abi.
        \] Now, \[
            a^2 - b^2 = \frac{1}{2}(|w| + u) - \frac{1}{2}(|w| - v) = u,
        \] and 
        \begin{align*}
            2ab &= 2\left(\frac{|w| + u}{2}\right)^{1 / 2}\left(\frac{|w| -
                    u}{2}\right)^{1 / 2} \\
                &= 2\left(\frac{(|w| + u)(|w| - u)}{4}\right)^{1 / 2} \\
                &= 2\left(\frac{|w|^2 - u^2}{4}\right)^{1 / 2} \\
                &= 2\left(\left(\frac{v}{2}\right)^2\right)^{1 / 2}.
        \end{align*}
        Recall that $(x^2)^{1 / 2} = x$ if $x \geq 0$ and $(x^2)^{1 / 2} = -x$ if $x
        \leq 0$. Thus, when $v \geq 0$, we have $2ab = v$ and when $v \leq 0$, we
        have $2ab = -v$. This means that $w = u + 2abi = z^2$ when $v \geq 0$ and $w =
        u - 2abi = (\overline{z})^2$ when $v \leq 0$.

        Note that when $w = 0$, it has only one square root, namely $0$. Otherwise,
        every non-zero complex number $w = u + iv$ has two square roots, either $z,
        -z$ or $\overline{z}, -\overline{z}$ depending on the sign of $v$.
    \end{solution}

    \paragraph{Exercise 11.} If $z$ is a complex number, prove that there exists an
    $r \geq 0$, a complex number $w$ with $|w| = 1$ such that $z = rw$. Are $w$ and
    $r$ always uniquely determined by $z$?
    \begin{solution}
        Write $z = a + bi$, and if $z \neq 0$ define \[
            r = \sqrt{a^2 + b^2}, \qquad
            w = z / r = \frac{a}{\sqrt{a^2 + b^2}} + \frac{bi}{\sqrt{a^2 + b^2}}.
        \] If $z = 0$, simply take $r = 0$ and $w = 1$. Thus, $z = rw$.

        When $z \neq 0$, this choice is unique, since $z = rw$ forces $|z| = |rw| =
        |r| |w| = r$, hence $r = |z| = \sqrt{a^2 + b^2}$ and $w = z / r$. Otherwise
        for $z = 0$, we can choose any $w$ (say $w = \pm 1$) as long as $r = 0$.
    \end{solution}

    \paragraph{Exercise 12.} If $z_1, \dots, z_n$ are complex, prove that \[
        |z_1 + z_2 + \dots + z_n| \leq |z_1| + |z_2| + \dots + |z_n|.
    \]
    \begin{solution}
        We prove this by induction. The case $n = 1$ is trivially true. For $n = 2$,
        see Theorem~1.33. If this holds for some $n \geq 1$, then use the $n = 2$
        case on $z_1 + \dots + z_n$ and $z_{n + 1}$, then the induction hypothesis
        to get \[
            |z_1 + \dots + z_n + z_{n + 1}| \leq |z_1 + \dots + z_n| + |z_{n + 1}|
            \leq |z_1| + \dots + |z_n| + |z_{n + 1}|.
        \] This proves the desired statement by induction.
    \end{solution}

    \paragraph{Exercise 13.} If $x$ and $y$ are complex, prove that \[
         | |x| - |y| | \leq |x - y|.
    \]
    \begin{solution}
        Use the triangle inequality to write \[
            |x| = |x - y + y| \leq |x - y| + |y|, \qquad
            |y| = |y - x + x| \leq |x - y| + |x|.
        \] Thus, if $|x| > |y|$, then $| |x| - |y| | = |x| - |y| \leq |x - y|$ by
        the first inequality. If $|x| < |y|$, then $| |x| - |y| | = |y| - |x| \leq
        |x - y|$ by the second inequality. If $|x| = |y|$, then $| |x| - |y| | = 0$,
        so the inequality holds trivially.
    \end{solution}
    
    \paragraph{Exercise 14.} If $z$ is a complex number such that $|z| = 1$, that
    is, such that $z\overline{z} = 1$, compute \[
        |1 + z|^2 + |1 - z|^2.
    \]
    \begin{solution}
        Write $z = a + bi$, so $a^2 + b^2 = 1$. Now, $|1 + z|^2 = (a + 1)^2 + b^2$,
        and $|1 - z|^2 = (a - 1)^2 + b^2$. Adding, \[
            |1 + z|^2 + |1 - z|^2 = 2(a^2 + b^2 + 1) + 2a - 2a = 4.
        \] 
    \end{solution}
    
    \paragraph{Exercise 15.} Under what conditions does equality hold in the Schwarz
    inequality?
    \begin{solution}
        In Theorem~1.35, recall that \[
            A = \sum |a_i|^2, \qquad B = \sum |b_i|^2, \quad C = \sum a_i
            \overline{b_i},   
        \] and the desired inequality was $AB \geq C^2$.
        If $B = 0$, then all $b_i = 0$ so equality holds. Otherwise, we concluded
        that with $B > 0$, \[
            \sum |Ba_i - Cb_i|^2 = B(AB - |C|^2) \geq 0.
        \] Here, equality means $AB = |C|^2$, so every $|Ba_i - Cb_i| = 0$, hence
        $a_i = (C / B)b_i$ for all $i$.
    \end{solution}
    
    \paragraph{Exercise 16.} Suppose $k \geq 3$, $\vx, \vy \in \R^k$,
    $|\vx - \vy| = d > 0$ and $r > 0$. Prove the following.
    \begin{enumerate}
        \itemsep0em
        \item If $2r > d$, then there are infinitely many $\vz \in \R^k$ such
        that \[
            |\vz - \vx| = |\vz - \vy| = r.
        \] 
        \item If $2r = d$, there is exactly one such $\vz$.
        \item If $2r < d$, there is no such $\vz$.
    \end{enumerate}
    \begin{solution}
        Note that by translating all the variables $\vx' = \vx - \vy$,
        $\vy' = \vec{0}$, our system of equations looks identical, with $|\vx' -
        \vy'| = d$ and the solutions are related by $\vz' = \vz -
        \vy$. Thus, we may instead consider the system $|\vx| = d$, \[
            |\vz - \vx| = |\vz| = r.
        \]
        Consider an arbitrary solution $\vz$ and write $\vec{v} = \vz - 
        \frac{1}{2}\vx$. Now, \[
            |\vz|^2 = (\frac{1}{2}\vx + \vec{v})\cdot
            (\frac{1}{2}\vx + \vec{v}) = \frac{1}{4}|\vx|^2 +
            |\vec{v}|^2 + \vx\cdot\vec{v}.
        \] Also, \[
            |\vz - \vx|^2 = 
            (-\frac{1}{2}\vx+\vec{v})\cdot(-\frac{1}{2}\vx+\vec{v}) =
            \frac{1}{4}|\vx|^2 + |\vec{v}|^2 - \vx\cdot\vec{v}. 
        \] Adding the above equations gives \[
            |\vz|^2 + |\vz - \vx|^2 = \frac{1}{2}|\vx|^2 +
            2|\vec{v}|^2, \qquad
            |\vec{v}|^2 = r^2 - \frac{d^2}{4}.
        \] Subtracting the two equations gives $\vec{v}\cdot\vx = 0$.

        These conditions on $\vec{v}$ are necessary and sufficient to generate
        solutions $\vz = \frac{1}{2}\vx + \vec{v}$.

        \begin{enumerate}
            \item Pick a unit vector $\hat{\vec{v}}$ perpendicular to $\vx$,
            i.e.\ $\hat{\vec{v}}\cdot\vx = 0$. Note that the components satisfy
            \[
                v_1x_1 + \dots + v_kx_k = 0.
            \] Since $d > 0$, we have $\vx\neq \vec{0}$, so without loss of generality
            let $x_1 \neq 0$. Then we have \[
                v_1 = -\frac{1}{x_1}(v_2x_2 + \dots + v_kx_k).
            \] Therefore, we may choose the components $v_2, \dots, v_k$
            arbitrarily.  For example, fix $v_2 = 1$, vary $v_3 = 0, 1, 2, \dots$
            and vary the remaining components arbitrarily, then normalize. All of
            the generated unit vectors are distinct, because the ratio of components
            $v_2$ and $v_3$ is different in each case. Thus, we have generated
            infinitely many unit vectors $\hat{\vec{v}}$ this way.

            Now define the real number $v \geq 0$, $v^2 = r^2 - d^2 / 4$. Then,
            all the vectors $\vz = \frac{1}{2}\vx + \vec{v}$ are
            solutions, where $\vec{v} = v\hat{\vec{v}}$.

            \item We have $|\vx| = d = 2r$, which means \[
                |\vec{v}|^2 = r^2 - \frac{1}{4}(2r)^2 = 0,
            \] forcing $|\vec{v}| = 0$, $\vec{v} = \vec{0}$. Thus, there is only one
            solution, namely $\vz = \frac{1}{2}\vx$.

            \item When $2r < d$ \[
                |\vec{v}|^2 = r^2 - \frac{d^2}{4} < 0,
            \] which is impossible. Thus, there are no solutions $\vz$ of this
            system.
        \end{enumerate}
        Note that when $k = 2$, we can only generate 2 unit vectors $\hat{\vec{v}}$
        such that $\hat{\vec{v}}\cdot\vx = 0$. This is because we want \[
            v_1x_1 + v_2x_2 = 0, \qquad v_1 = -\frac{v_2x_2}{x_1}, \qquad v_1^2 = 1
            - v_2^2.
        \] Thus, there are only two solutions, when $2r > d$. When $k = 1$, the
        condition $vx = 0$ with $x \neq 0$ forces $v = 0$, yet we require
        $v^2 = r^2 - d^2 / 4 > 0$ when $2r > d$, so there are no solutions.

        The remaining parts (b) and (c) remain identical for $k = 1, 2$.
    \end{solution}
    
    \paragraph{Exercise 17.} Prove that \[
        |\vx + \vy|^2 + |\vx - \vy|^2 = 2|\vx|^2 + 2|\vy|^2
    \] if $\vx, \vy \in \R^k$. Interpret this geometrically, as a statement about
    parallelograms.
    \begin{solution}
        Calculate \[
            |\vx + \vy|^2 = (\vx + \vy)\cdot(\vx + \vy) = |\vx|^2 + |\vy|^2 +
            2\vx\cdot\vy,
        \] \[
            |\vx - \vy|^2 = (\vx - \vy)\cdot(\vx - \vy) = |\vx|^2 + |\vy|^2 -
            2\vx\cdot\vy.
        \] Adding the two gives the desired equation.

        If we interpret $\vx$ and $\vy$ to be two adjacent legs of a parallelogram,
        then $\vx + \vy$ and $\vx - \vy$ represent its diagonals. Thus, the sum of
        squares of the diagonals of a parallelogram is equal to twice the sum of
        squares of two adjacent sides.
    \end{solution}

    \paragraph{Exercise 18.} If $k \geq 2$ and $\vx \in \R^k$, prove that there
    exists $\vy \in \R^k$ such that $\vy \neq \vec{0}$ but $\vx\cdot\vy = 0$.
    Is this also true if $k = 1$?
    \begin{solution}
        If $\vx = \vec{0}$, then any non-zero vector in $\vy \in \R^k$ satisfies
        $\vx\cdot\vy = 0$. Otherwise, $\vx = (x_1, x_2, \dots, x_k) \neq\vec{0}$ so
        without loss of generality let the component $x_1 \neq 0$. Set \[
            \vec{y} = (-x_2, x_1, 0, \dots, 0) \in \R^k,
        \] so \[
            \vx\cdot\vy = x_1(-x_2) + x_2(x_1) + 0 + \dots + 0 = 0.
        \] This is clearly not possible in $\R$ unless $x = 0$, because the product
        of any two non-zero real numbers is also non-zero.
    \end{solution}

    \paragraph{Exercise 19.} Suppose $\va, \vb \in \R^k$. Find $\vc \in
    \R^k$ such that \[
        |\vx - \va| = 2|\vx - \vb|
    \] if and only if $|\vx - \vc| = r$.
    \begin{solution}
        Write $\vx' = \vx - \va$, $\vb' = \vb - \va$, $\vc' = \vc - \va$, so we want
        to find $\vc'$ such that \[
            |\vx'| = 2|\vx' - \vb'|
        \] if and only if $|\vx' - \vc'| = r$.

        Write $\vx' = \frac{4}{3}\vb' + \vr$. Then \[
            |\vx'|^2 = \frac{16}{9}|\vb'|^2 + |\vr|^2 + \frac{8}{3}\vb'\cdot\vr,
        \] and \[
            |\vx' - \vb'|^2 = |\frac{1}{3}\vb' + \vr|^2 = \frac{1}{9}|\vb'|^2 +
            |\vr|^2 + \frac{2}{3}\vb'\cdot\vr.
        \] Using $|\vx'|^2 = 4|\vx' - \vb'|^2$, we have \[
            \frac{12}{9}|\vb'|^2 = 3|\vr|^2, \qquad |\vr| = \frac{2}{3}|\vb'|.
        \] Thus, $|\vx' - \frac{4}{3}\vb'| = \frac{2}{3}|\vb'|$, which is both
        necessary and sufficient. This means that  $\vc' = \frac{4}{3}\vb'$ and $r =
        \frac{2}{3}|\vb'|$. Translating everything back by $\va$, we have \[
            \vc = \frac{4}{3}\vb - \frac{1}{3}\va, \qquad r = \frac{2}{3}|\vb - \va|.
        \] 
    \end{solution}

    \paragraph{Exercise 20.} With reference to the Appendix, suppose that property
    (III) were omitted from the definition of a cut. Keep the same definitions of
    order and addition. Show that the resulting ordered set has the
    least-upper-bound property, that addition satisfies axioms (A1) to (A4) (with
    a slightly different zero-element!) but that (A5) fails.
    \begin{solution}
        We define a cut as any set $\alpha \subset \Q$ with the following properties.
        \begin{itemize}
            \itemsep0em
            \item[(I)] $\alpha$ is not empty, $\alpha \neq \Q$.
            \item[(II)] If $p \in \alpha$, $q \in \Q$, and $q < p$, then $q \in \alpha$.
        \end{itemize}
        Property (III) used to state that if $p \in \alpha$, then $p < r$ for some
        $r \in \alpha$, which meant that $\alpha$ had no maximal element.
        Property (II) implies that if $p \in \alpha$ and $q \notin \alpha$, then $p
        < q$ (take the contrapositive, and note that $p \neq q$). It also implies
        that if $r \notin \alpha$ and $r < s$, then $s \notin \alpha$ ($s \in
        \alpha$ would have forced $r \in \alpha$).

        Call the set of all these cuts $\R'$.
        Like before, the order $\alpha < \beta$ is defined to mean $\alpha \subset
        \beta$, for $\alpha, \beta \in \R'$. Again, $\R'$ has the least upper bound
        property.

        To see this, let $A$ be any non-empty subset of $\R'$ bounded above by
        $\beta \in \R'$, and let $\gamma$ be the union of all $\alpha \in A$. Thus,
        $p \in \gamma$ if and only if $p \in \alpha$ for some $\alpha \in A$. To
        verify that $\gamma$ is indeed a cut, note that $A$ is non-empty so there is
        at least one element $\alpha_0 \in A$ which is non-empty, so $\alpha_0
        \subset \gamma$ with $\gamma$ non-empty. Also, $\gamma \subset \beta$ since
        $\beta$ being an upper bound means that $\alpha < \beta$ for all $\alpha \in
        A$, which in turn means $\alpha \subset \beta$ for all $\alpha \in A$, hence
        $\gamma = \cup_{\alpha \in A} \alpha \subset \beta$. This verifies property
        (I). To verify property (II), pick $p \in \gamma$, and suppose that $p \in
        \alpha_1$ for some $\alpha \in A$. If $q \in \Q$ with $q < p$, this gives $q
        \in \alpha_1$, hence $q \in \gamma$. Thus, $\gamma$ is indeed a cut, i.e.\
        $\gamma \in \R'$.

        Now, we claim that $\gamma = \sup A$. Clearly, for any $\alpha \in A$, we
        have $\alpha \subset \gamma$ by definition to $\alpha \leq \gamma$ for all
        $\alpha \in A$, meaning $\gamma$ is an upper bound of $A$. Now suppose that
        $\delta \in \R'$, and $\delta < \gamma$. This means that $\delta$ is a
        proper subset of $\gamma$, so there is some $p \in \gamma$ such that $\p
        \notin \delta$. However, we must have $p \in \alpha_1$ for some $\alpha_1
        \in A$, so $\alpha$ cannot be a proper subset of $\delta$, meaning that
        $\delta$ is not an upper bound of $A$. Thus, $\gamma$ is the least upper
        bound of $A$.

        Like before, for $\alpha, \beta \in \R'$, define addition $\alpha + \beta$
        as the set of sums $r + s$ with $r \in \alpha$, $s \in \beta$. We must now
        verify the axioms of addition. 
        \begin{itemize}
            \item[(A1)] We demand closure, which is easily seen because $\alpha +
            \beta$ is a non-empty proper subset of $\Q$, and if $p \in \alpha +
            \beta$, then we must be able to write $p = r + s$ for some $r \in
            \alpha$, $s \in \beta$. Now if $q \in \Q$ and $q < p$, then $q - s < p -
            s = r$, so $q - s \in \alpha$, hence $q = (q - s) + s \in \alpha +
            \beta$. 

            \item[(A2)] We demand commutativity, which follows trivially. $\alpha +
            \beta = \beta + \alpha$, both being the set of $r + s = s + r$ with $r
            \in \alpha$, $s \in \beta$. 

            \item[(A3)] We demand associativity, which follows again from the
            associativity of the rational numbers. Note that if $\alpha, \beta,
            \gamma \in \R'$, with $r \in \alpha$, $s \in \beta$, $t \in \gamma$,
            then $r + (s + t) = (r + s) + t$. 

            \item[(A4)] Here, select $0' = \{r \in \Q: r \leq 0\}$. To show that for
            any $\alpha \in \R'$, $0' + \alpha = \alpha$, note that $0' + \alpha$ is
            the set of all rational numbers $r + s$ with $r \leq 0$ and $s \in
            \alpha$, so $r + s \leq s \in \alpha$ hence $0' + \alpha \subseteq
            \alpha$. Now, if $s \in \alpha$, then $0 + s \in 0' + \alpha$ since $0
            \in 0'$ and $s \in \alpha$, so $\alpha \subseteq 0' + \alpha$. This
            proves $0' + \alpha = \alpha$. 

            \item[(A5)] We demand the existence of an additive inverse $-\alpha$ for
            every $\alpha$, such that $\alpha + (-\alpha) = 0'$. This fails with the
            choice $\alpha = 0^* = \{r \in \Q: r < 0\}$. Note that if $0^* + (-0^*)
            = 0'$, we require $r + s \leq 0$ for all $r \in 0^*$, $s \in -0^*$.
            There must also be some $r_0 \in 0^*$, $s_0 \in -0^*$ such that $r_0 +
            s_0 = 0$. Since $r_0 \in 0^*$, $r_0 < 0$, so $s_0 = -r_0 > 0$.  Now,
            note that $-s_0 / 2 < 0$ so $-s_0 / 2 \in 0^*$, but the sum $(-s_0 / 2)
            + s_0 = s_0 / 2 > 0$, which is a contradiction.
        \end{itemize}

        In addition, note that $0^*$ does not serve as a zero element, since $0^* +
        0' = 0'$, not $0^*$. Furthermore, there is no choice of a zero element, say
        $\alpha_0$, which makes (A1-4) hold as well as (A5), since our choice of the
        zero element $0'$ is forced (we have already shown that $0' + \alpha_0 =
        \alpha_0$, not $0'$ if $\alpha_0 \neq 0'$; the field axioms imply that the
        zero element once found is unique).
    \end{solution}
    

    \chapter{Basic Topology}

    \paragraph{Exercise 1.} Prove that the empty set is a subset of every set.
    \begin{solution}
        Suppose that there exists a set $A$ such that $\emptyset \nsubseteq A$. Note
        that $\emptyset \neq A$, so $\emptyset \subset A$ is a proper subset. Thus,
        there must be some element $x \in \emptyset$, $x \notin A$ which is absurd
        since the empty set $\emptyset$ contains no elements.
    \end{solution}

    \paragraph{Exercise 2.} A complex number $z$ is said to be
    \textit{algebraic} if there are integers $a_0, \dots, a_n$, not all zero,
    such that \[
        a_0z^n + a_1z^{n - 1} + \dots + a_{n - 1}z + a_n = 0.
    \] Prove that the set of all algebraic numbers is countable.
    \begin{solution}
        Let $A_n$ be the set of all algebraic numbers which are the roots of a
        polynomial with integral coefficients of degree $n$. Each of these
        algebraic numbers can be mapped to a tuple $(a_0, \dots, a_n) \in
        \Z^n$, denoting the coefficients of the polynomial it is a root of. This
        particular polynomial can have at most $n$ distinct roots, hence any such
        tuple can have at most $n$ algebraic numbers in its pre-image. Thus, we can
        map each $x \in A_n$ injectively to a tuple $(a_0, \dots, a_n, k) \in \Z^n
        \times \Z$, where $k$ is an index. Theorem~2.13 guarantees that
        $\Z^n\times\Z$ is countable, and Theorem~2.8 guarantees that the range of
        this map, which is an infinite subset of $\Z^n\times\Z$, is also countably
        infinite. Thus, $A_n$ is countably infinite.  Finally, Theorem~2.12
        guarantees that the countable union \[
            A = \bigcup_{n = 1}^\infty A_n
        \] is countably infinite. Noting that $A$ is the set of all algebraic
        numbers gives the desired result.
    \end{solution}

    \paragraph{Exercise 3.} Prove that there exist real numbers which are not
    algebraic.
    \begin{solution}
        Let $S = \R\setminus A$ be the set of all real numbers which are not
        algebraic. Note that the set of algebraic numbers $A$ is countable. If $S$ were 
        empty, or even countable, then the union $\R = S \cup A$ would also be
        countable, which is a contradiction. Thus, there are an uncountably infinite
        number of real numbers which are not algebraic.
    \end{solution}
    
    \paragraph{Exercise 4.} Is the set of all irrational real numbers countable?
    \begin{solution}
        We use the same argument as before, noting that if $\R\setminus\Q$ were
        countable, then the union $\R = (\R\setminus\Q) \cup \Q$ would also be
        countable, which is a contradiction.
    \end{solution}

    \paragraph{Exercise 5.} Construct a bounded set of real numbers with exactly
    three limit points.
    \begin{solution}
        For $x \in \R$, define the set \[
           S_x = \{x + 1 / n : n \in \N\}.
        \] Then, $S = S_0 \cup S_1 \cup S_2$ has exactly three limit points, namely
        $0, 1, 2$.
    \end{solution}

    \paragraph{Exercise 6.} Let $E'$ be the set of all limit points of a set $E$.
    Prove that $E'$ is closed. Prove that $E$ and $\overline{E}$ have the same limit
    points. Do $E$ and $E'$ always have the same limit points?
    \begin{solution}
        We must show that $E'$ contains all its limit points. Let $x'$ be a limit
        point of $E'$, which means that given $r > 0$, we can find $y \in E'$ such
        that $0 < d(x, y) < r$. Let $h$ be the positive real number such that $d(x,
        y) = r - h$. Now, $y \in E'$ is a limit point of $E$, which means that we
        can find $z \in E$ such that $0 < d(y, z) < h$.  Thus, $d(x, z) \leq d(x, y)
        + d(y, z) < r$. Also, we can choose $z \neq x$ since there are infinitely
        many $z \in E$ to choose from for each neighbourhood of $y$. Thus, $x$ is a
        limit point of $E$, so $x \in E'$.

        Suppose that $x$ is a limit point of $E$. This means that every
        deleted neighbourhood of $x$ contains some point $y \in E$, hence $y \in E
        \cup E' = \overline{E}$, which means that $x$ is a limit point of
        $\overline{E}$. Next, suppose that $x$ is a limit point of $\overline{E}$,
        which means that for any $r > 0$, we can find some point $y \in E \cup E'$
        such that $0 < d(x, y) < r$. Let $h > 0$ such that $d(x, y) = r - h$. If $y
        \in E'$, then $y$ is a limit point of $E$ which means that we can find $z
        \in E$ such that $z \neq x$ and $0 < d(y, z) < r - h$. Thus, $0 < d(x, z)
        \leq d(x, y) + d(y, z) < r$. If $y \in E$, then simply set $z = y$. In
        either case, we have shown that $x$ is a limit point of $E$. Thus, $E$ and
        $\overline{E}$ have precisely the same limit points.

        Note that $E$ and $E'$ need not have the same limit points. If $E = \{1 / n
        : n \in \N\}$, then $E' = \{0\}$ and $E'' = \emptyset$.
    \end{solution}

    \paragraph{Exercise 7.} Let $A_1, A_2, A_3, \dots$ be subsets of a metric space.
    \begin{enumerate}
        \itemsep0em
        \item If $B_n = \cup_{i = 1}^n A_i$, prove that $\overline{B}_n = \cup_{i =
        1}^n \overline{A}_i$ for $n = 1, 2, 3, \dots$.
        \item If $B = \cup_{i = 1}^\infty A_i$, prove that $\overline{B} \supset
        \cup_{i = 1}^\infty \overline{A}_i$.
    \end{enumerate}
    \begin{solution} \mbox{}
        \begin{enumerate}
            \item Note that each $\overline{A}_i$ is closed, hence the finite union
            $S = \cup_{i = 1}^n \overline{A}_i$ is closed. We wish to show that $
            \overline{B}_n = S$. First, each $A_i \subseteq \overline{A}_i$ so $B_n
            = \cup_{i = 1}^n A_i \subseteq \cup_{i = 1}^n \overline{A}_i \subseteq
            S$. Theorem~2.27~(c) guarantees that $\overline{B}_n \subseteq S$. Next,
            each $A_i \subseteq B_n \subseteq \overline{B}_n$ which is closed,  so
            Theorem~2.27~(c) guarantees that $\overline{A}_i \subseteq
            \overline{B}_n$. Thus, the union $S \subseteq \overline{B}_n$.
            Together, this proves that $\overline{B}_n = S$.

            \item Note that each $A_i \subseteq B \subseteq \overline{B}$ and
            $\overline{B}$ is closed, so Theorem~2.27~(c) guarantees that
            $\overline{A}_i \subseteq \overline{B}$. Thus, the union $\cup_{i =
            1}^\infty \overline{A}_i \subseteq \overline B$.
        \end{enumerate}
    \end{solution}

    \paragraph{Exercise 8.} Is every point of every open set $E \subseteq \R^2$ a limit
    point of $E$? Answer the same question for closed sets in $\R^2$.
    \begin{solution}
        In $\R^2$, every interior point of a set $E$ is a limit point of $E$. This
        is because given every $x \in E^\circ$, we can find a neighbourhood $N_r(x)
        \subseteq E$. Therefore, all the points $x + (h, 0) \in E$ where $0 < h <
        r$. For any $\epsilon > 0$, we set $x' = x + (\min\{\epsilon, r\}, 0)$ so
        $x' \in N_r(x) \subseteq E$ and $0 < d(x, x') < \epsilon$. When $E$ is open,
        $E = E^\circ$ hence every point of $E$ is a limit point of $E$.

        This is false for closed subsets in $\R^2$. Consider the finite set $\{(0,
        0)\}$, which has no limit points.
    \end{solution}
    
    \paragraph{Exercise 9.} Let $E^\circ$ denote the set of all interior points of a
    set E.
    \begin{enumerate}
        \itemsep0em
        \item Prove that $E^\circ$ is always open.
        \item Prove that $E$ is open if and only if $E^\circ = E$.
        \item If $G \subseteq E$ and $G$ is open, prove that $G \subseteq E^\circ$.
        \item Prove that the complement of $E^\circ$ is the closure of the
        complement of $E$.
        \item Do $E$ and $\overline{E}$ always have the same interiors?
        \item Do $E$ and $E^\circ$ always have the same closures?
    \end{enumerate}
    \begin{solution} \mbox{}
        \begin{enumerate}
            \itemsep0em
            \item We wish to show that every point of $E^\circ$ is an interior point
            of $E^\circ$. Suppose that $x \in E^\circ$, which means that for some
            $r > 0$, there is a neighbourhood $N_{r}(x) \subseteq E$. Now, $N_r(x)$
            is open, so for any $y \in N_r(x)$, we there is a neighbourhood $N_s(y)
            \subseteq N_r(x) \subseteq E$. Thus, $y$ is an interior point of
            $E$, i.e.\ $y \in E^\circ$. This gives $N_r(x) \subseteq E^\circ$, hence
            $x$ is an interior point of $E^\circ$.

            \item First suppose that $E$ is open. This means that every point of $E$
            is an interior point of $E$, so $E \subseteq E^\circ$. However, every $x
            \in E^\circ$ has a neighbourhood $N_r(x) \subseteq E$, so $x \in E$,
            thus $E^\circ \subseteq E$. This gives $E^\circ = E$.

            Next, suppose that $E^\circ = E$. Part (a) directly gives $E$ is open.

            \item Pick $g \in G$, hence $g \in E$. Since $E$ is open, $E = E^\circ$,
            so $g \in E^\circ$. Thus, $G \subseteq E^\circ$.

            \item First, pick $x \in (E^\circ)^c$. This means that $x$ is not an
            interior point of $E$, so every neighbourhood of $x$ will contain some
            point $y \notin E$. If $x \in E^c$, then $x \in \overline{E^c}$ as
            required. Otherwise, $x \in E$, so $x \neq y$ for each chosen
            neighbourhood, which means that $x$ is a limit point of $E^c$. Thus, $x
            \in (E^c)'$, so $x \in \overline{E^c}$. This gives $(E^\circ)^c
            \subseteq \overline{E^c}$.

            Next, pick $x \in \overline{E^c}$. If $x \in E^c$, then $x \notin E$ so
            $x \notin E^\circ \subseteq E$. Otherwise, $x$ must be a limit point of
            $E^c$, so every neighbourhood of $x$ contains a point $y \in E^c$. In
            other words, no neighbourhood of $x$ is wholly contained within $E$, so
            $x$ cannot be an interior point of $E$, i.e.\ $x in (E^\circ)^c$. This
            gives $\overline{E^c} \subseteq (E^\circ)^c$.

            \item Consider $E = (0, 1) \cup (1, 2)$, with $\overline{E} = [0, 2]$.
            Note that $E^\circ = (0, 1) \cup (1, 2)$ but $(\overline{E})^\circ = (1,
            2)$.

            \item Consider $E = \{1\}$, with $E^\circ = \emptyset$. Note that
            $\overline{E} = \{1\}$ but $\overline{E^\circ} = \emptyset$.
        \end{enumerate}
    \end{solution}
    
    \paragraph{Exercise 10.} Let $X$ be an infinite set. For $p \in X$ and $q \in
    X$, define \[
        d(p, q) = \begin{cases}
            1, &\text{ if }p \neq q, \\
            0, &\text{ if }p = q. 
        \end{cases}    
    \] Prove that this is a metric. Which subsets of the resulting metric space are
    open? Which subsets closed? Which are compact?
    \begin{solution}
        Clearly, $d(p, q) = 1 > 0$ if $p \neq q$ and $d(p, p) = 0$ for $p, q \in X$.
        Also, if $p = q$, $d(p, q) = 0 = d(p, q)$, otherwise if $p \neq q$, then $q
        \neq p$, so $d(p, q) = 1 = d(q, p)$. Finally, $0 \leq d(p, q) \leq 1$, so
        for any $p, q, r \in X$, \[
            d(p, q) + d(q, r) = \begin{cases}
                2 > d(p, r), &\text{ if }p \neq q, q \neq r, \\
                1 = d(p, r), &\text{ if }p \neq q, q = r \text{ or } p = q, q \neq r, \\
                0 = d(p, r), &\text{ if }p = q = r,
            \end{cases}
        \] This proves that $X$ is a metric space under this distance function.

        All subsets of this metric space are open. To see this, note that every
        singleton set $\{x\}$ for $x \in X$ is open, because $\{x\} = N_{1 / 2}(x)$.
        Thus, any subset $E \subseteq X$ is the union $\cup_{x \in E} \{x\}$, and is
        hence open. The empty set is vacuously open.

        All subsets of this metric space are closed. This is because their
        complements are open by the previous remark.

        All finite sets are compact, since for any open cover $\cup \{\O_n\}$ of a
        finite set, we can choose an open set covering each element $x_n$ of the
        finite set, which means that their union makes a finite sub-cover.
        
        Any infinite set $E \subseteq X$ cannot be compact, since the open cover
        $\cup_{x \in E} \{N_{1 / 2}(x)\} = E$ has no finite sub-cover. This is
        because each $x \in E$ is covered by precisely one neighbourhood, $N_{1 /
        2}(x)$.
    \end{solution}

    \paragraph{Exercise 11.} For $x \in \R^1$ and $y \in \R^1$, define \begin{align*}
        d_1(x, y) &= (x - y)^2 \\
        d_2(x, y) &= \sqrt{|x - y|} \\
        d_3(x, y) &= |x^2 - y^2| \\
        d_4(x, y) &= |x - 2y| \\
        d_5(x, y) & = \frac{|x - y|}{1 + |x - y|}
    \end{align*} Determine, for each of these, whether it is a metric or not.
    \begin{solution}
        The first distance function fails the triangle inequality. Note that \[
            d_1(1, 0) + d_1(0, -1) = 2 < 4 = d(1, -1).
        \] 

        The second distance function is a metric. Note that $d_2(x, y) = \sqrt{|x -
        y|} > 0$ when $x \neq y$ and $d_2(x, x) = 0$. Furthermore, let $x, y, z \in
        \R$. We also claim that \[
            d_2(x, y) + d_2(y, z) \geq d_2(x, z).
        \] Set $a = |x - y|$, $b = |y - z|$, and $c = |x - z| \leq |x - y| + |y - z|
        = a + b$ by the triangle inequality. The desired inequality is equivalent to
        showing \[
            \sqrt{a} + \sqrt{b} \geq \sqrt{c}, \qquad a + b +  2\sqrt{ab} \geq c,
        \] the last of which is clearly true, since $2\sqrt{ab} \geq 0$.

        The third distance function is not a metric, since $1 \neq -1$ yet $d_3(1, -1)
        = |1 - 1| = 0$.

        The fourth distance function is not a metric, since $d_4(1, 1) = |1 - 2| = 1
        \neq 0$.

        The fifth distance function is a metric. Note that $d_5(x, y) \neq 0$ when
        $x \neq y$, and $d_5(x, x) = 0$. Furthermore, for $x, y, z \in \R$, we claim
        that \[
            d_2(x, y) + d_2(y, z) \geq d_2(x, z).
        \] Set $a = |x - y|$, $b = |y - z|$, and $c = |x - z| \leq |x - y| + |y - z|
        = a + b$ by the triangle inequality. The desired inequality is equivalent to
        showing \[
            \frac{a}{1 + a} + \frac{b}{1 + b} \geq \frac{c}{1 + c}.
        \] Since $d(x, y) \geq 0$, this is equivalent to \begin{align*}
            a(1 + b)(1 + c) + b(1 + a)(1 + c) &\geq c(1 + a)(1 + b), \\
            a + ab + ac + abc + b + ab + bc + abc &\geq c + ac + bc + abc, \\
            a + b + 2ab + abc &\geq c,
        \end{align*} which is true since $2ab + abc \geq 0$.
    \end{solution}

    \paragraph{Exercise 12.} Let $K \subset \R^1$ consist of $0$ and the numbers $1
    / n$, for $n = 1, 2, 3, \dots$. Prove that $K$ is compact directly from the
    definition.
    \begin{solution}
        Let $\{\O_\alpha\}$ be an open cover of $K$. This means that \[
            K \subseteq \bigcup_{\alpha}\, \O_\alpha,
        \] so there is some $\alpha_0$ such that $0 \in \O_{\alpha_0}$. Since
        $\O_{\alpha_0}$ is an open set, there exists $r > 0$ such that $N_r(0)
        \subseteq \O_{\alpha_0}$, therefore $x \in \O_{\alpha_0}$ whenever $0 \leq x
        < r$. This means that $\O_{\alpha_0}$ covers all elements $1 / n \in K$
        where $n > 1 / r$. This leaves finitely many elements $1 / n > r$, since there
        are only finitely many $n = 1, 2, 3, \dots$ such that $n < 1 / r$.
        For all such $n < 1 / r$, pick $\alpha_n$ such that $1 / n \in
        \O_{\alpha_n}$. As a result, \[
            K \subseteq \O_{\alpha_0} \cup \O_{\alpha_1} \cup \dots \cup
            \O_{\alpha_n}
        \] hence we have constructed a finite sub-cover of $K$, proving that $K$ is
        compact.
    \end{solution}

    \paragraph{Exercise 13.} Construct a compact set of real numbers whose limit
    points form a countable set.
    \begin{solution}
        Write \[
            S = \left\{\frac{1}{n} + \frac{1}{m} : n, m \in \N\right\}.
        \] It can be shown that the set of limit points of $S$ is \[
            S' = \left\{\frac{1}{n} : n \in \N\right\} \cup \{0\}.
        \] Thus, $\overline{S} = S \cup S'$ has countably many limit points.
        Furthermore, $\overline{S}$ is bounded and closed, hence compact.
    \end{solution}
    
    \paragraph{Exercise 14.} Give an example of an open cover of the segment $(0,
    1)$ which has no finite sub-cover.
    \begin{solution}
        For every $n \in \N$, define \[
            \O_n = (1 / n, 1).
        \] We claim that $(0, 1) = \cup_{n = 1}^\infty \O_n$. Note that for any $x
        \in (0, 1)$, we can choose $n \in \N$ such that $nx > 1$, hence $x \in (1 /
        n, 1) = \O_n$. If $\{\O_n\}_{n \in S}$ were a finite sub-cover, set $N =
        \max{S}$ and note that $1 / 2N < 1 / N$, hence $1 / 2N \notin \O_{n}$ for
        all $n \in S$, which is a contradiction.
    \end{solution}

    \paragraph{Exercise 15.} Show that Theorem~2.36 and its Corollary become false
    if the word ``compact'' is replaced by ``closed'' or ``bounded''.
    \begin{solution}
        Consider the collection of closed sets in $\R$, \[
            \mathscr{C}_n = [n, \infty),\quad \text{ for all }n \in \N.
        \] The intersection of any finite number of them is non-empty and
        $\mathscr{C}_n \supset \mathscr{C}_{n + 1}$, but the intersection of all of
        them must be empty, since for any $x \in R$, there exists $n \in N$ such
        that $x < n$, hence $x \notin \mathscr{C}_n$. \\

        Consider the collection of bounded sets in $\R$, \[
            \mathscr{B}_n = (0, 1 / n),\quad \text{ for all }n \in \N.
        \] The intersection of any finite number of them is non-empty and
        $\mathscr{B}_n \supset \mathscr{B}_{n + 1}$, but the intersection of all of
        them must be empty, since for any $x > 0$, there exists $n \in N$ such that
        $nx > 1$, so $x > 1 / n$, hence $x \notin \mathscr{B}_n$.
    \end{solution}

    \paragraph{Exercise 16.} Regard $\Q$, the set of all rational numbers, as a
    metric space, with $d(p, q) = |p - q|$. Let $E$ be the set of all $p \in \Q$
    such that $2 < p^2 < 3$. Show that $E$ is closed and bounded in $\Q$, but that
    $E$ is not compact. Is $E$ open in $\Q$?
    \begin{solution}
        It is clear that $E$ is bounded, because $E \subset N_3(0)$. If $p \in E$, 
        then $p^2 < 3$, hence $p^2 < 9$ or $|p - 0| < 3$, so $p \in N_3(0)$.

        In order to show that $E$ is closed, we'll show that it contains all its
        limit points. If $x \notin E$ is a limit point of $E$, then $x^2 \leq 2$ or
        $x^2 \geq 3$. Since $x$ is rational, we have $x \neq 2$, $x \neq 3$. If $x^2
        < 2$, set $r = \sqrt{2} - |x| > 0$. We claim that $N_r(x) \cap E =
        \emptyset$. This is because for $y \in N_r(x)$, we have $|y - x| < r$ hence
        $|y| \leq |x| + |y - x| < |x| + r \leq \sqrt{2}$, so $y^2 < 2$. Similarly,
        if $x^2 > 3$, set $r = |x| - \sqr{3} > 0$, whence $N_r(x) \cap E =
        \emptyset$. This is because for $y \in N_r(x)$, $|y - x| < r$ hence $|x|
        \leq |x - y| + |y|$, or $|y| \geq |x| - |x - y| > |x| - r \geq \sqrt{3}$, so
        $y^2 > 3$. In either case, we have found a neighbourhood of $x$ not
        containing any point of $E$, which contradicts the fact that $x$ is a limit
        point of $E$. Therefore, $E$ contains all its limit points and is closed in
        $\Q$.

        In order to show that $E$ is not compact, consider the open cover
        $\{\O_n\}_{n \in \N}$, where \[
            \O_n = \Q \cap (-\infty, \sqrt{3 - 1 / n}).
        \] Each $\O$ is open relative to $\Q$ because $\Q \subset \R$, and
        $(-\infty, \sqrt{3 - 1 / n})$ is open in $\R$. $\O_1$ already covers all $p
        < 0$. For any $p \in E$, $p > 0$, note that $p^2 < 3$, so there exists some
        $n \in N$ such that $n(3 - p^2) > 1$, whence $p^2 < 3 - 1 / n$ or $p <
        \sqrt{3 - 1 / n}$. This means that $p$ is covered by $\O_n$.
        On the other hand, if this cover contains a finite sub-cover $\{\O_n\}_{n
        \in J}$, let $m = \max{J}$. Since $\O_n \subset \O_{n + 1}$, we see that the
        union $\cup_{n \in J}\O_n = \O_m$. Now, $\sqrt{3 - 1 / m} < \sqrt{3}$, so
        there exists a rational number $p$ between these real numbers such that $3 -
        1 / m < p^2 < 3$, which means that $p \notin \O_m$ but $p \in E$, which is a
        contradiction.

        In order to show that $E$ is open in $\Q$, note that $E = \Q \cap
        ((-\sqrt{3}, -\sqrt{2}) \cup (\sqrt{2}, \sqrt{3}))$.
    \end{solution}
    
    \paragraph{Exercise 17.} Let $E$ be the set of all $x \in [0, 1]$ whose decimal
    expansion contains only the digits $4$ and $7$. Is $E$ countable? Is $E$ dense
    in $[0, 1]$? Is $E$ compact? Is $E$ perfect?
    \begin{solution}
        The set $E$ is uncountable, which can be shown by applying Cantor's
        diagonalization argument with the symbols $4$ and $7$ in lieu of $0$ and
        $1$. The set $E$ is not dense in $[0, 1]$, since the neighbourhood $(-0.1,
        +0.1) \cap [0, 1] = [0, 0.1)$ contains no elements of $E$.

        In order to show that the bounded set $E$ is compact, it is sufficient to
        show that it is closed and then apply the Heine-Borel Theorem.
        Suppose that the number $a = 0.a_1a_2 \dots a_n \dots \in [0, 1]\setminus E$,
        where all $a_{k < n}$ being either $4$ or $7$ and $a_n$ being neither $4$
        nor $7$. This means that for any $b = 0.b_1b_2 \dots\in E$, we have $a \neq b$, 
        and the first digit which differs from $a$, say $b_m \neq a_m$ must be such
        that $m \leq n$ (clearly the $n$th digit of $a$ differs from that of $b$).
        Thus, \[
            a - b = \sum_{k = m}^{\infty} a_k10^{-k}, \qquad
            |a - b| \geq |a_m - b_m| 10^{-m} - \sum_{k = m + 1}^\infty |a_k -
            b_k|10^{-k}.
        \] Now, clearly $|a_m - b_m| \geq 1$. Also, each $|a_k - b_k| \leq 7$; this
        is because $b_k$ is limited to either $4$ or $7$, and each of these digits
        is at most $7$ away from $0 \leq a_k < 10$. Thus, the infinite sum is
        bounded above by \[
            \sum_{k = m + 1}^\infty 7 \cdot 10^{-k} = \frac{7}{9}10^{-m}, \qquad |a
            - b| \geq \frac{2}{9}10^{-m} \geq \frac{2}{9}10^{-n}.
        \] Note that $|a - b|$ is bounded below by the same quantity regardless of
        our choice of $b \in E$, hence the corresponding neighbourhood of $a$
        contains no points from $E$. Thus, $a$ cannot be a limit point of $E$. In
        other words, $E$ contains all its limit points, and is hence closed. This
        further shows that $E \subset [0, 1]$ is compact.

        In order to show that $E$ is perfect, it is sufficient to show that every
        point in $E$ is a limit point. Again, given $b = 0.b_1 b_2 \dots \in E$,
        toggling the digit $b_n$ between $4$ and $7$ creating the number $b'\in E$ 
        gives a difference of $3\cdot 10^{-n}$. Since $10^{-n} \to 0$,
        given any $\epsilon > 0$ we can find sufficiently large $n$ such that
        $3\cdot 10^{-n} < \epsilon$, hence $|b - b'| < \epsilon$. Thus, every point
        $b \in E$ is a limit point of $E$.
    \end{solution}
    
    \paragraph{Exercise 18.} Is there a non-empty perfect set in $\R^1$ which
    contains no rational number?


    \paragraph{Exercise 19.} \mbox{}
    \begin{enumerate}
        \itemsep0em
        \item If $A$ and $B$ are disjoint closed sets in some metric space $X$,
        prove that they are separated.
        \item Prove the same for disjoint open sets.
        \item Fix $p \in X$, $\delta > 0$, define $A$ to be the set of all $q \in X$
        for which $d(p, q) < \delta$, define $B$ similarly with $>$ in place of $<$.
        Prove that $A$ and $B$ are separated.
        \item Prove that every connected metric space with at least two points is
        uncountable.
    \end{enumerate}
    \begin{solution} \mbox{}
    \begin{enumerate}
        \item Note that since $A$ and $B$ are closed, we have $A = \overline{A}$ and
        $B = \overline{B}$. This immediately gives $A \cap \overline{B} = B \cap
        \overline{A} = \emptyset$.
        \item Suppose that $A \cap \overline{B} \neq \emptyset$, i.e.\ there is a
        limit point $x$ of $B$ which is contained in $A$. Thus, there is an open
        ball $B_r(x) \subseteq A$, and since $x$ is a limit point of $B$ there are
        an infinite number of point $b_i \in B \cap B_r(x) \subseteq A$, which
        contradicts the assumption that $A \cap B = \emptyset$. A symmetric argument
        shows that $B \cap \overline{A} = \epmtyset$.
        \item Note that $A$ and $B$ are open sets; $A$ is an open ball, and $B$ is
        the complement of the closed ball described by $d(p, q) \leq \delta$. In
        addition, $A$ and $B$ are disjoint, since we cannot have both $\delta < d(p,
        q) < \delta$. Thus, $A$ and $B$ are separated by the previous exercise.
        \item Choose distinct $p, q \in X$, set $\delta = d(p, q) > 0$. Now, let $0
        \leq \delta' \leq \delta$, and examine all points $x$ such that $d(p, x) =
        \delta$. If no such $x$ existed, then the open sets described by $d(p, x) <
        \delta'$ and $d(p, x) > \delta'$ would cover $X$, thus separating it. In
        this manner, we choose $x_{\delta'}$ for every $\delta' \in [0, 1]$. Note
        that all such $x_{\delta'}$ are distinct by construction, thus we have a
        bijection between $\{x_{\delta'}\}$ and the uncountable interval $[0, 1]$,
        hence $\{x_{\delta'}\} \subseteq X$ is uncountable.
    \end{enumerate}
    \end{solution}

    \paragraph{Exercise 20.} Are closures and interiors of connected sets always
    connected?
    \begin{solution}
        The closures of connected sets are always connected. Suppose that $A$ is
        connected but $\overline{A}$ is not, i.e.\ we can choose non-empty disjoint
        open sets $X, Y$ such that $\overline{A} = X \cup Y$. Consider $X_0 = X \cap
        A$ and $Y_0 = Y \cap A$. Suppose that $A \subseteq X$; this would imply that
        all the points in the non-empty set $Y$ are limit points of $A$. However,
        since $A \subseteq X$, these are also limit points of $X$, which contradicts
        the fact that $X$ is open. Similarly, we cannot have $A \subseteq Y$.
        Therefore, $X_0$ and $Y_0$ are both non-empty. Furthermore, $A = X_0 \cup
        Y_0$, because $X_0 \cup Y_0 = (X \cap A) \cup (Y \cap A) = (X \cup Y) \cap A
        = \overline{A} \cap A = A$. Thus, $X_0$ and $Y_0$ separate $A$, which is a
        contradiction.

        Consider the sets $A$ and $B$ in $\R^2$ described by $d(x, (0, 0)) \leq
        1$ and $d(x, (2, 0)) \leq 1$. These are closed discs with the common point
        $(1, 0)$. This means that the union $A \cup B$ is connected. However, their
        interiors are open discs which are disjoint, and hence their union is
        disconnected.
    \end{solution}

    \paragraph{Exercise 21.} Let $A$ and $B$ be separated subsets of some $\R^k$,
    suppose $\va \in A$, $\vb \in B$, and define \[
        p(t) = (1 - t)\va + t\vb
    \] for $t \in \R^1$. Put $A_0 = p^{-1}(A)$, $B_0 = p^{-1}(B)$.
    \begin{enumerate}
        \itemsep0em
        \item Prove that $A_0$ and $B_0$ are separated subsets of $\R^1$.
        \item Prove that there exists $t_0 \in (0, 1)$ such that $p(t_0) \notin A
        \cup B$.
        \item Prove that every convex subset of $\R^k$ is connected.
    \end{enumerate}
    \begin{solution} \mbox{}
    \begin{enumerate}
        \item Suppose that $A_0 \cap
        \overline{B_0} \neq \emptyset$, i.e.\ let $x \in A_0$ be a limit point of
        $B_0$. Thus, for any $r > 0$, we can find $y \in B_0$ such that $d(x, y) <
        r$. Note that $p(x) = (1 - x)\va + x\vb \in A$, and $p(y) = (1 - y)\va +
        y\vb \in B$ by construction, and \[
            d(p(x), p(y)) = |(y - x)\va + (x - y)\vb| = |x - y| |\va - \vb|.
        \] Since $|\va - \vb|$ is fixed and positive ($A \cap B = \emptyset$), for
        any $\epsilon > 0$, we can choose sufficiently small $r$ such that $|x - y|
        < r$ gives $d(p(x), p(y)) < \epsilon$. Thus, we have shown that $p(x)$ is a
        limit point of $B$, with every $\epsilon$ neighbourhood of $p(x) \in A$
        containing the points $p(y) \in B$. This contradicts the fact that $A$ and
        $B$ are separated, so $A \cap \overline{B} = \emptyset$. A symmetric
        argument shows that we must have $\B_0 \cap \overline{A_0} = \emptyset$,
        which proves that $A_0$ and $B_0$ are separated.

        \item Suppose that $p(t) \in A \cup B$ for all $t \in [0, 1]$. Set $X = [0,
        1] \cap A_0$, $Y = [0, 1] \cap B_0$. Note that $A_0 \cup B_0 = p^{-1}(A)
        \cup p^{-1}(B) = [0, 1]$, since for every $t \in [0, 1]$, either $t \in A$
        or $t \in B$. Thus, $X \cup Y = [0, 1]$, with $X \cap Y = \emptyset$. This
        constitutes a disconnection of the connected set $[0, 1]$, which is a
        contradiction. Thus, there must exist some point $t_0 \in [0, 1]$ such that
        $p(t_0) \notin A \cup B$. Furthermore, $p(0) = \va \in A$ and $p(1) = \vb
        \in B$, so we must have $t_0 \in (0, 1)$.

        \item A convex set $E$ is such that if $\va, \vb \in E$, then the line
        segment joining $\va$ and $\vb$ is contained in $E$, i.e.\ $p([0, 1])
        \subseteq E$. Thus, if a convex set $E$ were separated as $E = A \cup B$,
        then we could use the previous exercise to find $t_0$ such that $p(t_0)
        \notin E$, which is a contradiction.
    \end{enumerate} 
    \end{solution}

    \paragraph{Exercise 22.} A metric space is called separable if it contains a
    countable dense set. Show that $\R^k$ is separable.
    \begin{solution}
        We claim that $\Q^k$ is a countable dense set in $\R^k$. This follows from
        the fact that $\Q$ is countable and dense in $\R$. Given any point $\vx \in
        \R^k$ and $\epsilon > 0$, we can always find points $y_i \in \Q$ such that
        $|x_i - y_i| < \epsilon / \sqrt{k}$ for each component $x_i \in \R$ of
        $\vx$. Thus, we have found $\vy \in \Q^k$ where \[
            d(\vx, \vy)^2 = \sum_{i = 1}^k |x_i - y_i|^2 < k\cdot
            \frac{\epsilon^2}{k} = \epsilon^2.
        \] 
    \end{solution}

    \paragraph{Exercise 23.} A collection $\{V_\alpha\}$ of open sets of $X$ is
    called a \textit{base} for $X$ if the following is true: For every $x \in X$ and
    every open set $G \subset X$ such that $x \in G$, we have $x \in V_\alpha
    \subset G$ for some $\alpha$. In other words, every open set in $X$ is the union
    of a sub-collection of $\{V_\alpha\}$.

    Prove that every separable metric space has a countable base.
    \begin{solution}
        Let $\{x_i\}_{i \in \N}$ be a countable dense subset of the separable metric
        space $X$. Define $V_{ij} = B_{q_j}(x_i)$ for every positive rational $q_j$.
        Note that $\{V_{ij}\}$ is countable, since the indexing set $\N \times \N$
        is countable. We claim that $\{V_{ij}\}$ is a base for $X$. Indeed, for
        every $x \in X$ and every open set $G \subset X$ with $x \in G$, we can find
        a point $x_i \in G$ such that $d(x, x_i) < r/2$ for any $r$; choose this $r$
        such that the open ball $B_r(x) \subset G$ (this can be done since $G$ is
        open). Now choose a positive rational $q_j$ such that $d(x, x_i) < q_j < r/2$.
        This immediately gives $x \in V_{ij} = B_{q_j}(x_i) \subset B_r(x) \subset
        G$.
    \end{solution}

    \paragraph{Exercise 24.} Let $X$ be a metric space in which every infinite set
    has a limit point. Prove that $X$ is separable.
    \begin{solution}
        Fix $\delta > 0$ and $x_i \in X$. Construct the sequence of points $x_1,
        x_2, \dots$ such that every new point $x_{j + i} \in X$ is chosen to satisfy
        $d(x_{j + 1}, x_i) \geq \delta$ for all previous $i = 1, \dots, j$. We claim
        that this process terminates, i.e.\ $X$ can be covered by finitely many many
        neighbourhoods of radius $\delta$. If this process did not terminate, then
        the infinite set $\{x_i\}$ would not have any limit point; any point $x \in
        X$ cannot have more than one point of $\{x_i\}$ in the neighbourhood
        $B_{\delta / 2}(x)$ since every point $x_i$ is separated from the others by
        at least $\delta$. Thus, we have found a finite set $X_\delta$ of open balls
        $B_\delta(x_i)$ which cover $X$. Consider the collection of the centres
        $x_{in}$ of all such covers $X_{1 / n}$. This is a countable set which is
        dense in $X$. This is because given any $x \in X$ and $\epsilon > 0$, we can
        find $k$ such that $0 < 1 /k < \epsilon$, and furthermore $X$ is covered by
        $X_{1 / k}$ so there is an open ball $B_{1 / k}(x_{in})$ containing the
        point $x$.
    \end{solution}

    \paragraph{Exercise 25.} Prove that every compact metric space $K$ has a
    countable base, and that $K$ is therefore separable.
    \begin{solution}
        Given $n \in N$, consider the set of open balls $B_{1 / n}(x)$ for all $x
        \in K$. This is an open cover of $K$, and hence must contain a finite
        sub-cover $X_{1 / n}$ of $K$. Repeating this for all $n \in N$, we see that
        the collection of the centres $x_{in}$ of all such $X_{1 / n}$ is a
        countable dense subset of $X$, by the previous exercise. This proves that
        $X$ is separable. The union of all $X_{1 / n}$, i.e.\ the collection of all
        open balls $B_{1 / n}(x_{in})$, is a countable base of $X$.
    \end{solution}

    \paragraph{Exercise 26.} Let $X$ be a metric space in which every infinite set
    set has a limit point. Show that $X$ is compact.
    \begin{solution}
        By Exercise~24, $X$ is separable, and by Exercise~23, $X$ has a countable
        base $\{V_i\}$. Suppose that $\{X_\alpha\}$ is an open cover of $X$. Since
        for every $x \in X$ we can choose $V_i$ such that $x \in V_i \subset
        X_\alpha$, we can manufacture a countable sub-cover $\{X_{\alpha_i}\} \equiv
        \{X_j\}$. Now suppose that no \textit{finite} collection of such $X_j$
        covers $X$. This means that for every finite union $X_1 \cup \dots \cup
        X_n$, the complement $Y_n$ is non-empty. However, the intersection of all
        such $Y_n$ is empty, since the infinite union $X_1 \cup \dots$ covers $X$.
        Thus, choose $y_n \in Y_n$ for each $n$, note that the infinite set $E =
        \{y_n\}$ must have a limit point $y \in X$. Since $\{X_i\}$ covers $X$, we
        have $y \in X_n$ for some $n$, and since $X_n$ is
        open, we can find an open ball $B_r(y) \subset X_n$. Thus, $B_r(y) \cap Y_n
        = \emptyset$, and since $Y_{k} \supset Y_{k + 1}$ we have $B_r(y) \cap
        Y_{k\geq n} = \emptyset$. Thus, we have found a neighbourhood of $y$ not
        containing any points in $Y_{k \geq n}$, specifically not containing any
        $y_n$. This contradicts the fact that $y$ is a limit point of $E$.
    \end{solution}
    
    \paragraph{Exercise 27.} Define a point $p$ in a metric space $X$ to be a
    \textit{condensation point} of a set $E \subset X$ if every neighbourhood of $p$
    contains uncountably many points of $E$.

    Suppose $E \subset \R^k$, $E$ is uncountable, and let $P$ be the set of all
    condensation points of $E$. Prove that $P$ is perfect and that at most countably
    many points of $E$ are not in $P$. In other words, show that $P^c \cap E$ is at
    most countable.
    \begin{solution}
        Let $\{V_n\}$ be a countable base of $\R^k$, and let $W$ be the union of
        those $V_n$ such that $E \cap V_n$ is at most countable. Then, $E \cap W =
        \bigcup_n (E \cap V_n)$ is at most countable since it is countable union of
        at most countable sets. We claim that $P = W^c$.

        First, let $x \in W^c$. Since $\{V_n\}$ is a countable base, given any open
        ball $B_r(x)$, we can find $x \in V_k \subset B_r(x)$. Now, $x \notin W$
        hence $x \notin V_n$ where $E \cap V_n$ is countable. Hence, $x \in V_k$
        where $E \cap V_k$ is uncountable. Thus, the neighbourhood $B_r(x)$ contains
        uncountably many points of $E$, so $x$ is a condensation point of $E$, i.e.\
        $x \in P$. This gives $W^c \subseteq P$.

        Next, let $x \in W$. This means that $x \in V_n$ where $E\cap V_n$ is
        countable. Thus, any neighbourhood of $x$ contained within $V_n$ (recall
        that the countable base in a separable metric space can consist of open
        balls) can contain at most countably many points of $E$, so $x$ is not a
        condensation point of $E$. Thus, $x \notin P$, i.e.\ $W \subseteq P^c$.
        Together, we have $W^c = P$.

        To show that $P$ is perfect, we must show that it is closed and that every
        point of $P$ is a limit point. Note that the complement $P^c = W$ is open,
        being a union of open sets $V_n$, immediately showing that $P$ is closed.
        Now, let $x \in P$, and let $B_r(x)$ be a neighbourhood of $x$, with $B_r(x)
        \cap E$ being uncountable. Suppose that there is no point $y \in B_r(x) \cap
        P$, $y \neq x$. This means that all the points $y \in B_r(x)\setminus\{x\}$
        are contained in $P^c = W$, hence all such $y$ belong to some $V_n$ where
        $E \cap V_n$ is countable. Thus, $B_r(x) \setminus\{x\} \subset W$. Since $E
        \cap W$ contains at most countably many points, we have $E \cap (B_r(x)
        \setminus\{x\})$ containing at most countably many points, hence the
        addition of the single point $x$ gives $E \cap B_r(x)$ containing at most
        countably many points. This contradicts the fact that $x$ is a condensation
        point of $E$. Thus, every point in $P$ must be a limit point of $P$, proving
        that $P$ is perfect.
    \end{solution}

    \paragraph{Exercise 28.} Prove that every closed set in a metric space is the
    union of a (possibly empty) perfect set and a set which is at most countable.
    As a corollary, every countable closed set in $\R^k$ has isolated points.
    \begin{solution}
        Given any closed set $E$, note that its set of condensation points $P$ is
        perfect, and the set $P^c \cap E$ is at most countable, with their union $P
        \cup (P^c \cap E) = E$. This is because $P \subseteq \overline{E} = E$, since
        every condensation point is also a limit point.

        If $E$ were countable, then $P$ must be empty (there are no uncountably many
        points of $E$ to choose from). Thus, the set $P^c \cap E = E$ is countable
        and hence not perfect. This means that at least one of its points is not a
        limit point, and is hence an isolated point.
    \end{solution}
    
    \paragraph{Exercise 29.} Prove that every open set in $\R^1$ is the union of an
    at most countable collection of disjoint segments.
    \begin{solution}
        Note that $\R$ is separable and hence has a countable base $\{V_i\}$ of open
        intervals $(a_i, b_i)$. Thus, given any open set $O \subseteq \R$, for each
        $x \in O$ choose $V_i$ such that $x \in V_j \subset O$. The countable union
        of all such $V_j$ gives precisely $O$. This is because any $x \in O$ belongs
        to some $V_j$, and any $x$ in the union of $V_j$ belongs to some $V_j$ which
        is contained in $O$.

        Now we refine our collection of $V_j$ based on the following rules. If $V_j
        \subset V_k$, we eliminate $V_j$. If $V_j$ and $V_k$ overlap such that $a_j
        < a_k < b_j < b_k$, then replace $V_j$ and $V_k$ with $V_j \cup V_k$. This
        new collection of open intervals is thus by construction disjoint as
        required.
    \end{solution}

    \paragraph{Exercise 30.} Imitate the proof of Theorem~2.43 to obtain the
    following result.
    \begin{quote}
        If $\R^{k} = \bigcup_1^\infty F_n$, where each $F_n$ is a closed subset of
        $\R^k$, then at least one $F_n$ has a non-empty interior.

        \textit{Equivalent statement:} If $G_n$ is a dense open subset of $\R^k$,
        for $n = 1, 2, 3, \dots$, then $\bigcap_1^\infty G_n$ is not empty (in fact,
        it is dense in $\R^k$).
    \end{quote}
    \begin{solution}
        First we show that the statements are equivalent. If the first statement is
        true, and we have been given dense open subsets $G_n \subseteq \R^k$, set
        $F_n = G_n^c$ for all $n$. All $F_n$ must have empty interiors since every
        $G_n$ is dense in $\R^k$. Thus, their union cannot be the entirety of $R^k$,
        hence the union of all $G_n$ must be non-empty.

        If the second statement is true, and we have been given closed subsets $F_n
        \subseteq \R^n$ whose union is $\R^k$, set $G_n = F_n^c$. Note that the
        intersection of all such $G_n$ is empty, which means that at least one such
        $G_k$ is not dense in $\R^k$. The corresponding $F_k$ thus has a non-empty
        interior.

        We prove the second statement. Let $G_n$ be dense open sets of $\R^k$. Pick
        any open non-empty set $O \subseteq \R^k$, and note that since $G_1$ is
        dense, the set $O \cap G_1$ is non-empty and open (note that $\overline{G_1}
        = \R^k$, so $G_1^c$ is a collection of limit points of $G_1$, which is thus
        closed). Thus, pick $x_1 \in O \cap G_1$ and construct the open ball
        $B_{r_1}(x_1)$ contained within it. This is another non-empty open set,
        hence $O_1 = B_{r_1}(x_1) \cap G_2$ is non-empty and open. Pick a point
        $x_2$ in this set, and construct the open ball $B_{r_2}(x_2) \subset O_1
        \subseteq B_{r_1}(x_1)$. Continuing in this manner, we have generated nested
        sequence of open balls such that $B_{r_k}(x_k) \supseteq B_{r_{k + 1}}(x_{k
        + 1})$.  Now, the closure of each of these balls is compact by the
        Heine-Borel Theorem, hence the intersection of all these compact balls must
        be non-empty by Theorem~2.36. Any point $x$ in this intersection is a common
        element of all the compact balls, but every one of these compact balls is
        contained within its parent open set $O_n = B_{r_n}(x_n) \cap G_{n + 1}$.
        Hence, $x \in O \cap G_n$ for all $G_n$, thus the set $G = \bigcap_1^\infty
        G_n$ intersects with every open set $O$ in at least one point, i.e.\ every
        non-empty neighbourhood $O$ contains some point of $G$, which means that $G$
        is dense in $\R^k$.
    \end{solution}
    
    
    
\end{document}
% vim: set tabstop=4 shiftwidth=4 softtabstop=4:
