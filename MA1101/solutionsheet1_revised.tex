\documentclass[10pt]{article}

\usepackage[T1]{fontenc}
\usepackage{geometry}
\usepackage{amsmath, amssymb, amsthm}

\geometry{a4paper, margin=1in}
\setlength\parindent{0pt}
\renewcommand{\labelenumi}{(\roman{enumi})}
% \renewcommand\qedsymbol{$\blacksquare$}

\renewcommand{\land}{\text{ and }}
\renewcommand{\lor}{\text{ or }}
\newcommand{\compl}{\mathsf{c}}

\newcommand\blfootnote[1]{%
    \begingroup
    \renewcommand\thefootnote{}\footnote{#1}%
    \addtocounter{footnote}{-1}%
    \endgroup
}


\title{
    \textsc{
        \large
        MA1101: Mathematics I
    }
    \\\vspace{0.2em}
    \textbf{
        \huge
        Some Solutions for Problem Sheet I
    }
    \vspace{-1em}
}
\author{}
\date{}

\begin{document}

    \maketitle
    \blfootnote{
        Satvik Saha (\texttt{ss19ms154@iiserkol.ac.in})
    }

    \paragraph{Problem 1.}
    \begin{enumerate}
        \item[(i)] We prove $A \cup B = B \cup A$ by first showing that $A
        \cup B \subseteq B \cup A$, then showing that $B \cup A \subseteq A
        \cup B$.

        Consider \begin{align*}
            x \in A \cup B
            &\implies (x \in A) \lor (x \in B) \\
            &\implies (x \in B) \lor (x \in A) \\
            &\implies x \in B \cup A.
        \end{align*}
        This proves that $A \cup B \subseteq B \cup A$. Similarly, consider
        \begin{align*}
            x \in B \cup A
            &\implies (x \in B) \lor (x \in A) \\
            &\implies (x \in A) \lor (x \in B) \\
            &\implies x \in A \cup B.
        \end{align*}
        This proves that $B \cup A \subseteq A \cup B$. \qed


        \item[(ii)] We prove that $(A \cup B) \cup C = A \cup (B \cup C)$ by
        first showing that $(A \cup B) \cup C \subseteq A \cup (B \cup C)$,
        then showing that $A \cup (B \cup C) \subseteq (A \cup B) \cup C$.

        Consider \begin{align*}
            x \in (A \cup B) \cup C
            &\implies (x \in A \cup B) \lor (x \in C) \\
            &\implies ((x \in A) \lor (x \in B)) \lor (x \in C) \\
            &\implies (x \in A) \lor (x \in B) \lor (x \in C) \\
            &\implies (x \in A) \lor ((x \in B) \lor (x \in C)) \\
            &\implies (x \in A) \lor (x \in B \cup C) \\
            &\implies x \in A \cup (B \cup C).
        \end{align*}
        This proves that $(A \cup B) \cup C \subseteq A \cup (B \cup C)$.
        Similarly, consider \begin{align*}
            x \in A \cup (B \cup C)
            &\implies (x \in A) \lor (x \in B \cup C) \\
            &\implies (x \in A) \lor ((x \in B) \lor (x \in C)) \\
            &\implies (x \in A) \lor (x \in B) \lor (x \in C) \\
            &\implies ((x \in A) \lor (x \in B)) \lor (x \in C) \\
            &\implies (x \in A \cup B) \lor (x \in C) \\
            &\implies x \in (A \cup B) \cup C.
        \end{align*}
        This proves that $A \cup (B \cup C) \subseteq (A \cup B) \cup C$. \qed


        \item[(iii)] ($\Rightarrow$) Suppose that $A \subseteq B$. To show
        that $A \cup B = B$, we first show that $A \cup B \subseteq B$, then
        show that $B \subseteq A \cup B$.

        Consider \begin{align*}
            x \in (A \cup B)
            &\implies (x \in A) \lor (x \in B) \\
            &\implies (x \in B) \lor (x \in B) \tag{Using $A \subseteq B$} \\
            &\implies x \in B.
        \end{align*}
        This proves that $A \cup B \subseteq B$. Similarly, consider
        \begin{align*}
            x \in B
            &\implies (x \in A) \lor (x \in B) \\
            &\implies x \in A \cup B.
        \end{align*}
        This proves that $B \subseteq A \cup B$.

        ~

        ($\Leftarrow$) Suppose that $A \cup B = B$. Consider \begin{align*}
            x \in A
            &\implies (x \in A) \lor (x \in B) \\
            &\implies x \in A \cup B \\
            &\implies x \in B. \tag{Using $A \cup B = B$}
        \end{align*}
        This proves that $A \subseteq B$.


        \item[(vii)] Recall that $X \setminus Y = X \cap Y^\compl$. We compute
        \begin{align*}
            A \setminus (B \cup C)
            &= A \cap (B \cup C)^\compl \\
            &= A \cap (B^\compl \cap C^\compl) \tag{De Morgan's Law} \\
            &= (A \cap A) \cap (B^\compl \cap C^\compl) \\
            &= (A \cap B^\compl) \cap (A \cap C^\compl) \\
            &= (A \setminus B) \cap (A \setminus C). \tag*{\qed}
        \end{align*}


        \item[(xi)] First, note that $X \Delta Y = Y \Delta X$, since
        \begin{align*}
            X \Delta Y
            &= (X \setminus Y) \cup (Y \setminus X) \\
            &= (Y \setminus X) \cup (X \setminus Y) \tag{Using (i)} \\
            &= Y \Delta X.
        \end{align*} Next, observe that \begin{align*}
            X \Delta Y
            &= (X \setminus Y) \cup (Y \setminus X) \\
            &= (X \cap Y^\compl) \cup (Y \cap X^\compl).
        \end{align*}

        With this, we compute \begin{align*}
            A \Delta (B \Delta C)
            &= A \Delta ((B \cap C^\compl) \cup (B^\compl \cap C)) \\
            &= (A \cap ((B \cap C^\compl) \cup (B^\compl \cap C))^\compl) \cup
               (A^\compl \cap ((B \cap C^\compl) \cup (B^\compl \cap C))) \\
            &= (A \cap ((B \cap C^\compl)^\compl \cap (B^\compl \cap C)^\compl)) \cup
               (A^\compl \cap ((B \cap C^\compl) \cup (B^\compl \cap C))) \tag{De Morgan's Law}\\
            &= (A \cap ((B^\compl \cup C) \cap (B \cup C^\compl))) \cup
               (A^\compl \cap ((B \cap C^\compl) \cup (B^\compl \cap C))) \tag{De Morgan's Law}\\
            &= (A \cap 
                 (
                   ((B^\compl \cup C) \cap B) \cup
                   ((B^\compl \cup C) \cap C^\compl)
                 )
               ) \cup
               (A^\compl \cap ((B \cap C^\compl) \cup (B^\compl \cap C))) \tag{Distributive Law}\\
            &= (A \cap
                  (
                    ((B^\compl \cap B) \cup (C \cap B)) \cup 
                    ((B^\compl \cap C^\compl) \cup (C \cap C^\compl))
                  )
               ) \cup
               (A^\compl \cap ((B \cap C^\compl) \cup (B^\compl \cap C))) \tag{Distributive Law}\\
            &= (A \cap
                  (
                    (\emptyset \cup (B \cap C)) \cup
                    ((B^\compl \cap C^\compl) \cup \emptyset)
                  )
               ) \cup
               (A^\compl \cap ((B \cap C^\compl) \cup (B^\compl \cap C))) \\
            &= (A \cap
                  ((B \cap C)) \cup (B^\compl \cap C^\compl))
               ) \cup
               (A^\compl \cap ((B \cap C^\compl) \cup (B^\compl \cap C))) \\
            &= (A \cap B \cap C) \cup (A \cap B^\compl \cap C^\compl) \cup
               (A^\compl \cap B \cap C^\compl) \cup (A^\compl \cap B^\compl \cap C)
                \tag{Distributive Law} \\
        \end{align*}

        Interchanging the roles of $A$ and $C$ in the previous argument, we obtain \begin{align*}
            C \Delta (B \Delta A)
            &= (C \cap B \cap A) \cup (C \cap B^\compl \cap A^\compl) \cup
               (C^\compl \cap B \cap A^\compl) \cup (C^\compl \cap B^\compl \cap A) \\
            &= (A \cap B \cap C) \cup (A^\compl \cap B^\compl \cap C) \cup
               (A^\compl \cap B \cap C^\compl) \cup (A \cap B^\compl \cap C^\compl) \\
            &= (A \cap B \cap C) \cup (A \cap B^\compl \cap C^\compl) \cup
               (A^\compl \cap B \cap C^\compl) \cup (A^\compl \cap B^\compl \cap C) \\
            &= A \Delta (B \Delta C).
        \end{align*}
        Thus, we have \begin{align*}
            A \Delta (B \Delta C)
            &= C \Delta (B \Delta A) \\
            &= (B \Delta A) \Delta C \tag{Using $X \Delta Y = Y \Delta X$} \\
            &= (A \Delta B) \Delta C \tag{Using $X \Delta Y = Y \Delta X$}.
        \end{align*}

    \end{enumerate}


    \clearpage
    \paragraph{Problem 2.}
    \begin{enumerate}
        \item[(iii)] We prove $A \times (B \setminus C) = (A \times B)
        \setminus (A \times C)$ by showing that each side is a subset of the
        other. Consider \begin{align*}
            (x, y) \in A \times (B \setminus C)
            &\implies (x \in A) \land (y \in B\setminus C) \\
            &\implies (x \in A) \land (y \in B \cap C^\compl) \\
            &\implies (x \in A) \land ((y \in B) \land (y \in C^\compl)) \\
            &\implies (x \in A) \land (y \in B) \land (y \in C^\compl) \\
            &\implies ((x \in A) \land (y \in B)) \land (y \notin C) \\
            &\implies ((x, y) \in A\times B) \land ((x, y) \notin A \times C) \\
            &\implies (x, y) \in (A\times B) \cap (A \times C)^\compl \\
            &\implies (x, y) \in (A\times B) \setminus (A \times C).
        \end{align*}
        This proves that $A \times (B \setminus C) \subseteq (A \times B)
        \setminus (A \times C)$. Similarly, consider \begin{align*}
            (x, y) \in (A \times B) \setminus (A \times C)
            &\implies ((x, y) \in A \times B) \land ((x, y) \notin A \times C) \\
            &\implies ((x \in A) \land (y \in B)) \land ((x \notin A) \lor (y \notin C)) \\
            &\implies ((x \in A) \land (y \in B) \land (x \notin A)) \lor
                      ((x \in A) \land (y \in B) \land (y \notin C)) \\
            &\implies (x \in A) \land ((y \in B) \land (y \notin C)) \\
            &\implies (x \in A) \land (y \in B \setminus C) \\
            &\implies (x, y) \in A \times (B \setminus C).
        \end{align*}
        This proves that $(A \times B) \setminus (A \times C) \subseteq A
        \times (B \setminus C)$. \qed


        \item[(iv)] No. Consider the following counterexample.

        Let $A = \{0\}, B = \{1\}$. Then, \begin{align*}
            A\times B &= \{(0, 1)\}, \\
            \mathcal{P}(A\times B) &= \{\emptyset, \{(0, 1)\}\}, \\
            \mathcal{P}(A) &= \{\emptyset, \{0\}\}, \\
            \mathcal{P}(B) &= \{\emptyset, \{1\}\}, \\
            \mathcal{P}(A) \times \mathcal{P}(B) &= \{(\emptyset, \emptyset),
            (\emptyset, \{1\}), (\{0\}, \emptyset), (\{0\}, \{1\})\}.
        \end{align*}


        \item[(v)] Yes. Consider \begin{align*}
            (x, y) \in (A \cap C) \times (B \cap D)
            &\iff (x \in A \cap C) \land (y \in B \cap D) \\
            &\iff ((x \in A) \land (x \in C)) \land ((y \in B) \land (y \in D)) \\
            &\iff ((x \in A) \land (y \in B)) \land ((x \in C) \land (y \in D)) \\
            &\iff ((x, y) \in A\times B) \land ((x, y) \in C \times D) \\
            &\iff (x, y) \in (A\times B) \cap (C \times D). \tag*{\qed}
        \end{align*}


        \item[(vi)] No. Consider the following counterexample.

        Let $A = \{0\}, B = \{1\}, C = \{2\}, D = \{3\}$. Then,
        \begin{align*}
            A \cup C &= \{0, 2\}, \\
            B \cup D &= \{1, 3\}, \\
            (A \cup C) \times (B \cup D) &= \{(0, 1), (0, 3), (2, 1), (2, 3)\}, \\
            A \times B &= \{(0, 1)\}, \\
            C \times D &= \{(2, 3)\}, \\
            (A \times B) \cup (C \times D) &= \{(0, 1), (2, 3)\}.
        \end{align*}
    \end{enumerate}


    \clearpage
    \paragraph{Problem 3.}
    \begin{enumerate}
        \item The number of subsets of $X$ is $2^n$.

        To prove this, note that for each $x \in X$, we can either choose it
        or leave it aside when forming a subset of $X$. In other words, each
        of the $n$ elements in $X$ presents us with $2$ choices, giving us a
        total of $2^n$ ways of forming subsets of $X$. Moreover, every subset
        of $X$ can be formed in this manner.


        \item There are $2^n - 1$ non-empty subsets of $X$.

        There is precisely one empty subset out of the $2^n$ subsets of $X$.


        \item There are $(3^n + 1) / 2$ ways of choosing two disjoint subsets
        of $X$.

        For each $x \in X$, we can either place it in one subset, a second
        subset, or leave it aside. This gives us a total of $3^n$ ways of
        forming an ordered pair $(A, B)$ of disjoint subsets $A, B$ of $X$.
        However, we are looking for the number of unordered pairs of disjoint
        subsets. Thus, we have double-counted all cases where $A \neq B$, of
        which there are $3^n - 1$; the only case where $A = B$ is when $A = B
        = \emptyset$. This leaves us with $3^n - (3^n - 1) / 2 = (3^n + 1) /
        2$ ways.


        \item There are $(3^n - 2^{n + 1} + 1) / 2$ ways of choosing two
        non-empty disjoint subsets of $X$.

        Of the $(3^n + 1) / 2$ ways of choosing two disjoint subsets of $X$,
        consider the case where one of them is empty. This means that the
        other subset is simply an arbitrary subset of $X$, of which there are
        $2^n$. Removing these from our count leaves precisely all disjoint
        non-empty pairs of subsets of $X$. Thus, we have $(3^n + 1) / 2 - 2^n
        = (3^n - 2^{n + 1} + 1) / 2$ ways.
    \end{enumerate}


\end{document}
