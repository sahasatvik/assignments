\documentclass[10pt]{article}

\usepackage[T1]{fontenc}
\usepackage{geometry}
\usepackage{amsmath, amssymb, amsthm}
\usepackage{enumitem}

\title{Mathematics I - Problem Sheet V}
\author{Satvik Saha}
\date{}

\geometry{a4paper, margin=1in}
% \setlength\parindent{0pt}
% \setlength\parskip{5pt}
\renewcommand{\labelenumi}{(\roman{enumi})}
\newtheorem{theorem}{Theorem}[section]
\newtheorem{lemma}[theorem]{Lemma}
\newtheorem{corollary}{Corollary}[theorem]
\theoremstyle{definition}
\newtheorem*{definition}{Definition}
\theoremstyle{remark}
\newtheorem*{notation}{Notation}
% \renewcommand\qedsymbol{$\blacksquare$}
\makeatletter
\def\th@plain{%
  \thm@notefont{}% same as heading font
  \itshape % body font
}
\def\th@definition{%
  \thm@notefont{}% same as heading font
  \normalfont % body font
}
\makeatother
\newcommand{\N}{\mathbb{N}}
\newcommand{\Z}{\mathbb{Z}}
\newcommand{\Q}{\mathbb{Q}}
\newcommand{\Zm}{\Z\times(\Z\setminus\{0\})}
\newcommand{\simZ}{\sim_{\Z}} 
\newcommand{\simQ}{\sim_{\Q}} 
\newcommand{\IdN}{I_{\N}} 
\newcommand{\IdZ}{I_{\Z}} 

\begin{document}
        \par\textbf{IISER Kolkata} \hfill \textbf{Problem Sheet V}
        \vspace{3pt}
        \hrule
        \vspace{3pt}
        \begin{center}
                \LARGE{\textbf{MA 1101 : Mathematics I}}
        \end{center}
        \vspace{3pt}
        \hrule
        \vspace{3pt}
        Satvik Saha, \texttt{19MS154}\hfill\today
        \vspace{20pt}

        \section{Integers}
        \begin{theorem}
        Define a relation $\simZ$ on $\N\times \N$ as
        \[(m, n) \simZ (p, q) \quad\text{if}\quad m + q = n + p. \]
        Then, $\simZ$ is an equivalence relation on $\N\times \N$.
        \end{theorem}
        \begin{proof}
                For an arbitrary $(m, n) \in \N\times \N$, clearly $(m, n) \simZ (m, n)$, hence $\simZ $ is reflexive.

                Again, for arbitrary $(m, n), (p, q) \in \N\times \N$, if
                $(m, n) \simZ (p, q)$, we have $m + q = n + p$.
                By the commutativity of addition on natural numbers, $p + n = q + m$, so 
                $(p, q) \simZ (m, n) $, hence $\simZ$ is symmetric.

                For $(m, n), (p, q), (r, s) \in \N\times \N$, if
                $(m, n) \simZ (p, q)$ and $(p, q) \simZ (r, s)$, we have
                $m + q = n + p$ and $p + s = q + r$. Thus, 
                $m + q + p + s = n + p + q + r$, so $m + s = n + r$.
                Thus, $(m, n) \simZ (r, s)$, hence $\simZ$ is transitive.

                Therefore, $\simZ$ is an equivalence relation on $\N\times \N$.
        \end{proof}

        \begin{notation}
        Let us set
        \[
                \Z \;:=\; (\N \times \N)/\simZ,\\
        \]
        \[
                \Z^+ \;:=\; \{[(n + 1, 1)] : n \in \N\}, \quad
                \bar{0} \;:=\; [(1, 1)], \quad
                \bar{1} \;:=\; [(2, 1)].
        \]
        \end{notation}

        \begin{definition}[Addition]
        For $a = [(m, n)],\; b = [(p, q)] \in \Z$, we define
        \[
                a + b \;:=\; [(m + p, n + q)].
        \]
        \end{definition}
        
        \begin{theorem}
                Addition $(+)$ is well-defined, associative and commutative.
        \end{theorem}
        \begin{proof}
                First, we show that $+$ is well-defined.
                Let $a = [(m, n)] = [(m', n')]$, $b = [(p, q)] = [(p', q')] \in \Z$.
                We claim that $a + b = [(m + p, n + q)] = [(m' + p', n' + q')]$, i.e. $(m + p, n + q) \simZ (m' + p', n' + q')$, i.e $m + p + n' + q' = n + q + m'  + p'$.
                Now, $(m, n) \simZ (m' , n')$ and $(p, q) \simZ (p', q')$, from which we have $m + n' = n + m'$ and $p + q' = q + p'$.
                Adding these gives the desired result.

                For $a,b,c \in \Z$, let $a = [(m, n)], b = [(p, q)], c = [(r, s)]$.
                From the associativity of addition in $\N$,
                \begin{align*}
                        (a + b) + c \;&=\; [(m + p, n + q)] + [(r, s)] \\
                                \;&=\; [((m + p) + r, (n + q) + s)] \\
                                \;&=\; [(m + (p + r), n + (q + s))] \\
                                \;&=\; [(m, n)] + [(p + r, q + s)] \\
                                \;&=\; a + (b + c)
                \end{align*}
                Therefore, $+$ is associative.

                From the commutativity of addition in $\N$,
                \begin{align*}
                        a + b \;&=\; [(m + p, n + q)] \\
                                \;&=\; [(p + m, q + n)] \\
                                \;&=\; b + a
                \end{align*}
                Therefore, $+$ is commutative.
        \end{proof}
        
        \begin{lemma}
                For all $m, n, k \in \N$, $[(m, n)] = [(m + k, n + k)] \in \Z$.
        \end{lemma}
        \begin{proof}
                It is sufficient to show that $(m, n) \simZ (m + k, n + k)$, i.e.
                $m + n + k = n + m + k$, which is certainly true.
        \end{proof}
        \begin{lemma}
                For all $n \in \N$, $[(n, n)] = \bar{0}$.
        \end{lemma}
        \begin{proof}
                It is sufficient to show that $(n, n) \simZ (1, 1)$, i.e.
                $n + 1 = n + 1$, which is certainly true.
        \end{proof}

        \begin{theorem}
                For all $a \in \Z$, $\bar{0} + a = a = a + \bar{0}$.
        \end{theorem}
        \begin{proof}
                Let $a = [(m, n)] \in \Z$.
                \begin{align*}
                        a + \bar{0} \;&=\; [(m, n)] + [(1, 1)] \\
                                \;&=\; [(m + 1, n + 1)] \\
                                \;&=\; [(m, n)] \\
                                \;&=\; a \\
                        a + \bar{0} \;&=\; a \;=\; \bar{0} + a \qedhere
                \end{align*}
        \end{proof}
        
        \begin{theorem}
                        For all $a \in \Z$, there exists a unique $x \in \Z$, satisfying $a + x = \bar{0} = x + a$.
        \end{theorem}
        \begin{proof}
                For $a = [(m, n)] \in \Z$, construct $x = [(n, m)] \in \Z$.
                Clearly, $a + x = [(m + n, n + m)] = \bar{0}$.
                From commutativity of $+$, $a + x = \bar{0} = x + a$.

                We now show that $x$ is unique. Let $x' \in \Z$, $a + x' = \bar{0} = x' + a$.
                \begin{align*}
                        a + x' \;&=\; \bar{0} \\
                        x + (a + x') \;&=\; x + \bar{0} \\
                        (x + a) + x' \;&=\; x \\
                        \bar{0} + x' \;&=\; x \\
                        x' \;&=\; x     \qedhere
                \end{align*}
        \end{proof}
        \begin{notation}
                We denote $x$ as $-a$ and say that $-a$ is the \textit{negative} of $a$.
        \end{notation}
        \begin{corollary}
                If $a = [(m, n)] \in \Z$, then $-a = [(n, m)]$.
        \end{corollary}
        \begin{notation}
                For $a, b \in \Z$, we write
                \[
                a - b \;:=\; a + (-b).\quad\quad\quad
                \]
        \end{notation}

        \begin{theorem}
                        For all $a, b \in \Z$, there exists a unique $x \in \Z$ satisfying $a + x = b$.
        \end{theorem}
        \begin{proof}
                From the well-defined nature of $+$, there exists a unique $x = b - a = b + (-a) \in \Z$.
                \begin{align*}
                        a + x \;&=\; a + (b + (-a)) \\
                                \;&=\; a + ((-a) + b) \\
                                \;&=\; (a + (-a)) + b \\
                                \;&=\; \bar{0} + b \\
                                \;&=\; b
                \end{align*}
                Let $x' \in \Z$, $a + x' = b$.
                \begin{align*}
                        a + x' \;&=\; b \\
                        x + (a + x') \;&=\; x + b \\
                        (x + a) + x' \;&=\; x + b \\
                        b + x' \;&=\; b + x \\
                        x' \;&=\; x \qedhere
                \end{align*}
        \end{proof}

        \begin{definition}[Multiplication]
        For $a = [(m, n)],\; b = [(p, q)] \in \Z$, we define
        \[
                a \cdot b \;:=\; [(mp + nq, mq + np)].
        \]
        \end{definition}
        \begin{theorem}
                Multiplication $(\cdot)$ is well-defined, associative and commutative.
        \end{theorem}
        \begin{proof}
                First, we show that $\cdot$ is well-defined.
                Let $a = [(m, n)] = [(m', n')]$, $b = [(p, q)] = [(p', q')] \in \Z$.
                We claim that $a\cdot b = [(mp + nq, mq + np)] = [(m'p' + n'q', m'q' + n'p')]$,
                i.e. $(mp + nq, mq + np) \simZ (m'p' + n'q', m'q' + n'p')$.

                From $(p, q) \simZ (p', q')$,
                \begin{align*}
                        p + q' \;&=\; q + p' \\
                        mp + mq' \;&=\; mq + mp' \\
                        np + nq' \;&=\; nq + np' \\
                        mp + nq + mq' + np' \;&=\; mq + np + mp' + nq' \\
                        (mp + nq, mq + np) &\simZ (mp' + nq', mq' + np')
                \end{align*}

                From $(m, n) \simZ (m', n')$,
                \begin{align*}
                        m + n' \;&=\; n + m' \\
                        mp' + n'p' \;&=\; np' + m'p' \\
                        mq' + n'q' \;&=\; nq' + m'q' \\
                        mp' + nq' + m'q' + n'p' \;&=\; mq' + np' + m'p' + n'q' \\
                        (mp' + nq', mq' + np') &\simZ (m'p' + n'q', m'q' + n'p')
                \end{align*}
                Transitivity of $\simZ$ yields the desired result.

                For $a,b,c \in \Z$, let $a = [(m, n)], b = [(p, q)], c = [(r, s)]$.
                \begin{align*}
                        (a\cdot b)\cdot c \;&=\; [(mp + nq, mq + np)] \cdot [(r, s)] \\
                                \;&=\; [((mp + nq)r + (mq + np)s, (mp + nq)s + (mq + np)r)]\\
                                \;&=\; [(mpr + nqr + mqs + nps, mps + nqs + mqr + npr)]\\
                        a\cdot (b\cdot c) \;&=\; [(m, n)] \cdot [(pr + qs, ps + qr)] \\
                                \;&=\; [(m(pr + qs) + n(ps + qr), m(ps + qr) + n(pr + qs))] \\
                                \;&=\; [(mpr + mqs + nps + nqr, mps + mqr + npr + nqs)]
                \end{align*}
                Therefore, $(a\cdot b)\cdot c = a\cdot (b\cdot c)$, i.e. $\cdot$ is associative.
                
                \begin{align*}
                        a\cdot b \;&=\; [(mp + nq, mq + np)] \\
                                \;&=\; [(pm + qn, pn + qm)] \\
                                \;&=\; b\cdot a
                \end{align*}
                Therefore, $\cdot$ is commutative.
        \end{proof}

        \begin{theorem}
                For all $a \in \Z$, $a\cdot\bar{1} = a = \bar{1}\cdot a$.
        \end{theorem}
        \begin{proof}
                Let $a = [(m, n)] \in \Z$.
                \begin{align*}
                        a\cdot \bar{1} \;&=\; [(m, n)] \cdot [(2, 1)] \\
                                \;&=\; [(2m + n, m + 2n)] \\
                                \;&=\; [(m + (m + n), (m + n) + n)] \\
                                \;&=\; [(m, n)] \\
                                \;&=\; a \\
                        a\cdot \bar{1} \;&=\; a \;=\; \bar{1}\cdot a \qedhere
                \end{align*}
        \end{proof}
        \begin{theorem}
                For all $a \in \Z$, $a\cdot \bar{0} = \bar{0} = \bar{0}\cdot a$.
        \end{theorem}
        \begin{proof}
                Let $a = [(m, n)] \in \Z$.
                \begin{align*}
                        a\cdot \bar{0} \;&=\; [(m, n)]\cdot [(1, 1)]\\
                                \;&=\; [(m + n, m + n)] \\
                                \;&=\; \bar{0} \\
                        a\cdot \bar{0} \;&=\; \bar{0} \;=\; \bar{0}\cdot a \qedhere
                \end{align*}
        \end{proof}

        \begin{theorem}[Distributivity]
                For all $a,b,c \in \Z$, $a\cdot(b + c) = a\cdot b + a\cdot c$.
        \end{theorem}
        \begin{proof}
                For $a,b,c \in \Z$, let $a = [(m, n)], b = [(p, q)], c = [(r, s)]$.
                \begin{align*}
                        a\cdot (b + c) \;&=\; [(m, n)] \cdot [(p + r, q + s)] \\
                                \;&=\; [(m(p + r) + n(q + s), m(q + s) + n(p + r))] \\
                                \;&=\; [(mp + mr + nq + ns, mq + ms + np + nr)] \\
                                \;&=\; [(mp + nq, mq + np)] + [(mr + ns, ms + nr)] \\
                                \;&=\; a\cdot b + a\cdot c \qedhere
                \end{align*}
        \end{proof}
        \begin{theorem}
                For all $a, b \in \Z$, $(-a)\cdot b = -(a\cdot b)$.
        \end{theorem}
        \begin{proof}
                \begin{align*}
                        (-a)\cdot b + a\cdot b \;&=\; ((-a) + a)\cdot b \\
                                \;&=\; \bar{0} \cdot b \\
                                \;&=\; \bar{0}\\
                        (-a)\cdot b \;&=\; -(a\cdot b) \qedhere
                \end{align*}
        \end{proof}
        \begin{theorem}
                For all $a, b \in \Z$, $(-a)\cdot(-b) = a\cdot b$.
        \end{theorem}
        \begin{proof}
                \begin{align*}
                        (-a)\cdot(-b) + (-(a\cdot b)) \;&=\; (-a)\cdot(-b) + (-a)\cdot b\\
                                \;&=\; (-a)\cdot((-b) + b) \\
                                \;&=\; (-a)\cdot \bar{0} \\
                                \;&=\; \bar{0}\\
                        (-a)\cdot(-b) \;&=\; a\cdot b \qedhere
                \end{align*}
        \end{proof}

        \begin{lemma}
                If $a = [(m, n)] \in \Z$, $a \neq \bar{0}$, then $m \neq n$.
        \end{lemma}
        \begin{proof}
                Assume that $m = n$. Then, we have $(m, n) \simZ \bar{0}$,
                contradicting our premise. Hence, we must have $m\neq n$.
        \end{proof}
        \begin{theorem}[No zero divisors]
                For all $a, b \in \Z$ with $a,b \neq \bar{0}$, we have $a\cdot b \neq \bar{0}$.
        \end{theorem}
        \begin{proof}
                Let $a = [(m, n)], b = [(p, q)] \in \Z$.
                Note that $m\neq n$, $p\neq n$, since $a, b \neq \bar{0}$.

                Assume that our theorem is false, i.e. $a\cdot b = \bar{0}$.
                Then $[(mp + nq, mq + np)] = \bar{0} \Rightarrow mp + nq = mq + np$.
                One of the following must be true.
                
                \par\textbf{Case I:} If $m > n$, there exists $u \in \N$, such that $m = n + u$. Thus, $(n + u)p + nq = (n + u)q + np \Rightarrow np + up + nq = nq + uq + np$.
                This implies that $up = uq \Rightarrow p = q$, contradicting $p\neq q$.
                \par\textbf{Case II:} If $n > m$, there exists $v \in \N$, such that $n = m + v$. Thus, $mp + (m + v)q = mq + (m + v)p\Rightarrow mp + mq + vq = mq + mp + vp$.
                This implies that $vp = vq \Rightarrow p = q$, contradicting $p\neq q$.

                Hence, $a\cdot b \neq \bar{0}$.
        \end{proof}
        \begin{corollary}
        \label{cor:zero}
                For all $a, b \in \Z$, if $a\cdot b = \bar{0}$, then $a = \bar{0}$ or 
                $b = \bar{0}$.
        \end{corollary}

        \begin{theorem}[Cancellation]
                For $a, b, c \in \Z$ with $a\neq \bar{0}$, we have $a\cdot b = a\cdot c
                \Rightarrow b = c$.
        \end{theorem}
        \begin{proof}
                For $a,b,c \in \Z$, let $a = [(m, n)], b = [(p, q)], c = [(r, s)]$.
                We have $m \neq n$.
                \begin{align*}
                a\cdot b \;&=\; a\cdot c\\
                [(mp + nq, mq + np)] \;&=\; [(mr + ns, ms + nr)] \\
                mp + nq + ms + nr \;&=\; mq + np + mr + ns \\
                m(p + s) + n (q + r) \;&=\; m(q + r) + n(p + s)
                \end{align*}
                
                Assume that our theorem is false.
                Thus, $b\neq c$, i.e. $b + (-c) = [(p + s, q + r)] \neq \bar{0} \Rightarrow p + s \neq q + r$.
                Without loss of generality, let $p + s > q + r$, i.e. $p + s = q + r + x$ for some $x \in \N$.

                Thus, $m(q + r + x) + n(q + r) = m(q + r) + n(q + r + x)$. This implies that
                $mx = nx \Rightarrow m = n$, which contradicts $m \neq n$.

                Hence, $b = c$.
        \end{proof}
        
        \begin{definition}[Order]
                For all $a, b \in \Z$, we say that $a > b$ if $a - b \in \Z^+$.
        \end{definition}
        
        \begin{lemma}
        \label{lem:Zplus}
                If $m, n \in \N$, $m > n$, i.e. $m = n + x$ for $x \in \N$,
                then $a = [(m, n)] \in \Z^+$.
        \end{lemma}
        \begin{proof}
                We must show that $a = [(n + x, n)] \in \Z^+$, i.e.
                for some $k \in \N$, $(n + x, n) \simZ (k + 1, 1)$, i.e.
                $n + x + 1 = n + k + 1$. This is clearly true for $k = x$.
        \end{proof}
        \begin{theorem}
                For all $a, b \in \Z$, we have $a\cdot b > \bar{0}$ if $a, b > \bar{0}$ or $a, b < \bar{0}$.
        \end{theorem}
        \begin{proof}
                If $a, b > \bar{0}$, then $a, b \in \Z^+$. Thus, $a = [(m + 1, 1)]$ and
                $b = [(n + 1, 1)]$ for some $m, n \in \N$.
                \begin{align*}
                        a\cdot b \;&=\; [((m + 1)(n + 1) + (1)(1), (m + 1)1 + 1(n + 1))] \\
                                \;&=\; [(mn + m + n + 1 + 1, m + 1 + n + 1)] \\
                                \;&=\; [((m + n + 2) + mn, (m + n + 2))] \in \Z^+ \qedhere
                \end{align*}
                Therefore, $a\cdot b > \bar{0}$.

                If $a, b < \bar{0}$, then $\bar{0} - a, \bar{0} - b \in \Z^+$, i.e.
                $-a, -b > \bar{0}$. Therefore, $(-a)\cdot(-b) > \bar{0} \implies a\cdot b > \bar{0}$
        \end{proof}

        \begin{definition}[Identification map]
                Define $\IdN\colon \N\to \Z$ by 
                \[
                        \IdN(n) \;:=\; [(n + 1, 1)],\quad\text{for all }n\in \N.
                \]
        \end{definition}

        \begin{theorem}
                $\IdN$ is injective.
        \end{theorem}
        \begin{proof}
                Let $m, n \in \N$.
                \begin{align*}
                        \IdN(m) \;&=\; \IdN(n) \\
                        [(m + 1, 1)] \;&=\; [(n + 1, 1)] \\
                        (m + 1, 1) &\simZ (n + 1, 1) \\
                        m + 1 + 1 \;&=\; n + 1 + 1 \\
                        m \;&=\; n
                \end{align*}
                Hence, $\IdN$ is injective.
        \end{proof}

        \begin{theorem}
                $\IdN(\N) = \Z^+$.
        \end{theorem}
        \begin{proof}
                We first show that $\IdN(\N) \subseteq \Z^+$.
                Let $x \in \IdN(\N)$. Thus, there exists at least one
                $k \in \N$ such that $x = \IdN(k) = [(k + 1, 1)]$, which implies that $x \in \Z^+$ by definition.

                Next, we show that $\Z^+ \subseteq \IdN(\N)$.
                Let $x \in \Z^+$. By definition, $x = [(k + 1, 1)]$ for some $k \in \N$. Clearly, $x = \IdN(k) \in \IdN(\N)$.

                Hence, we conclude that $\IdN(\N) = \Z^+$.
        \end{proof}

        \begin{theorem}
                $\IdN(1) = \bar{1}$.
        \end{theorem}
        \begin{proof}
                \[\IdN(1) = [(1 + 1, 1)] = [(2, 1)] = \bar{1} \qedhere\]
        \end{proof}

        \begin{theorem}
                For all $m, n \in \N$, $\IdN(m + n) = \IdN(m) + \IdN(n)$.
        \end{theorem}
        \begin{proof}
                \begin{align*}
                        \IdN(m) + \IdN(n) \;&=\; [(m + 1, 1)] + [(n + 1, 1)] \\
                                \;&=\; [(m + 1 + n + 1, 1 + 1)] \\
                                \;&=\; [((m + n) + 1, 1)] \\
                                \;&=\; \IdN(m + n) \qedhere
                \end{align*}
        \end{proof}

        \begin{theorem}
                For all $m, n \in \N$, $\IdN(m \cdot n) = \IdN(m) \cdot \IdN(n)$.
        \end{theorem}
        \begin{proof}
                \begin{align*}
                        \IdN(m) \cdot \IdN(n) \;&=\; [(m + 1, 1)] \cdot [(n + 1, 1)] \\
                                \;&=\; [((m + 1)(n + 1) + (1)(1), (m + 1)1 + 1(n + 1))] \\
                                \;&=\; [(mn + m + n + 1 + 1, m + n + 1 + 1)] \\
                                \;&=\; [(mn + 1, 1)] \\
                                \;&=\; \IdN(m \cdot n) \qedhere
                \end{align*}
        \end{proof}

        \begin{theorem}
                For all $m, n \in \N$ with $m > n$, $\IdN(m) > \IdN(n)$.
        \end{theorem}
        \begin{proof}
                \begin{align*}
                \IdN(m) - \IdN(n) \;&=\; [(m + 1, 1)] + (-[(n + 1, 1)]) \\
                        \;&=\; [(m + 1, 1)] + [(1, n + 1)] \\
                        \;&=\; [(m + 1 + 1, 1 + n + 1)] \\
                        \;&=\; [(m, n)].
                \end{align*}
                From \ref{lem:Zplus}, $[(m, n)] \in \Z^+$.
                Therefore, $\IdN(m) - \IdN(n) \in \Z^+ \implies \IdN(m) > \IdN(n)$, as desired.
        \end{proof}

        \subsection*{Identification}
        For all $n \in \N$, we shall identify $\IdN(n)$ with $n$.
        With this identification,
        \[0 \leftrightarrow \bar{0}\]
        \[1 \leftrightarrow \bar{1}\]
        \[\N = \Z^+ \subset \Z\]
        \[
        \Z \;=\; \{\,n : n \in \N\,\} \cup
                \{\,-n : n \in \N\,\} \cup
                \{\,\bar{0}\,\}
        \]
        
        \section{Rationals}
        \begin{theorem}
        Define a relation $\simQ$ on $\Zm)$ as
        \[(m, n) \simQ (p, q) \quad\text{if}\quad mq = np. \]
        Then, $\simQ$ is an equivalence relation on $\Zm)$.
        \end{theorem}
        \begin{proof}
                For an arbitrary $(m, n) \in \Zm$, clearly $(m, n) \simQ (m, n)$, hence $\simQ $ is reflexive.

                Again, for arbitrary $(m, n), (p, q) \in \Zm$, if
                $(m, n) \simQ (p, q)$, we have $mq = np$.
                By the commutativity of multiplication on integers, $pn = qm$, so 
                $(p, q) \simQ (m, n) $, hence $\simQ$ is symmetric.

                For $(m, n), (p, q), (r, s) \in \Zm$, if
                $(m, n) \simQ (p, q)$ and $(p, q) \simQ (r, s)$, we have
                $mq = np$ and $ps = qr$. Thus, 
                $mqps = npqr$, so $ms = nr$.
                Thus, $(m, n) \simQ (r, s)$, hence $\simQ$ is transitive.

                Therefore, $\simQ$ is an equivalence relation on $\Zm)$.
        \end{proof}

        \begin{notation}
        Let us set
        \[
                \Q \;:=\; (\Zm)/\simQ,\\
        \]
        \[
                \bar{0} \;:=\; [(0, 1)], \quad
                \bar{1} \;:=\; [(1, 1)].
        \]
        \end{notation}

        \begin{definition}[Addition]
        For $a = [(m, n)],\; b = [(p, q)] \in \Q$, we define
        \[
                a + b \;:=\; [(mq + np, nq)].
        \]
        \end{definition}
        
        \begin{theorem}
                Addition $(+)$ is well-defined, associative and commutative.
        \end{theorem}
        \begin{proof}
                First, we show that $+$ is well-defined.
                Let $a = [(m, n)] = [(m', n')]$, $b = [(p, q)] = [(p', q')] \in \Q$.
                Now, $(m, n) \simQ (m' , n')$ and $(p, q) \simQ (p', q')$, from which we have 
                $mn' = m'n$ and $pq' = p'q$.
                We claim 
                \begin{align*}
                a + b = [(mq + np, nq)] \;&=\; [(m'q' + n'p', n'q')] \\
                (mq + np)(n'q') \;&=\; (m'q' + n'p')(nq) \\
                mn'qq' + nn'pq' \;&=\; m'nqq' + nn'p'q \\
                qq'(mn' - m'n) \;&=\; nn'(p'q - pq') \\
                qq'(0) \;&=\; nn'(0)
                \end{align*}
                which is clearly true.

                For $a,b,c \in \Z$, let $a = [(m, n)], b = [(p, q)], c = [(r, s)]$.
                \begin{align*}
                        (a + b) + c \;&=\; [(mq + np, nq)] + [(r, s)] \\
                                \;&=\; [((mq + np)s + nq(r), nqs)] \\
                                \;&=\; [(mqs + nps + nqr, nqs)] \\
                                \;&=\; [(m)qs + n(ps + qr), nqs] \\
                                \;&=\; [(m, n)] + [(ps + qr, qs)] \\
                                \;&=\; a + (b + c)
                \end{align*}
                Therefore, $+$ is associative.

                \begin{align*}
                        a + b \;&=\; [(mq + np, nq)] \\
                                \;&=\; [(pn + qm, qn)] \\
                                \;&=\; b + a
                \end{align*}
                Therefore, $+$ is commutative.
        \end{proof}
        
        \begin{lemma}
                For all $(m, n) \in S$, $k \in \Z\setminus\{0\}$,
                $[(m, n)] = [(mk, nk)] \in \Q$.
        \end{lemma}
        \begin{proof}
                It is sufficient to show that $(m, n) \simQ (mk, nk)$, i.e.
                $mnk = nmk$, which is certainly true.
        \end{proof}
        \begin{lemma}
                For all $n \in \Z\setminus\{0\}$, $[(n, n)] = \bar{1}$.
        \end{lemma}
        \begin{proof}
                It is sufficient to show that $(n, n) \simQ (1, 1)$, i.e.
                $n\cdot 1 = n\cdot 1$, which is certainly true.
        \end{proof}

        \begin{theorem}
                For all $a \in \Q$, $\bar{0} + a = a = a + \bar{0}$.
        \end{theorem}
        \begin{proof}
                Let $a = [(m, n)] \in \Q$.
                \begin{align*}
                        a + \bar{0} \;&=\; [(m, n)] + [(0, 1)] \\
                                \;&=\; [(m \cdot 1 + n\cdot 0, n \cdot 1)] \\
                                \;&=\; [(m, n)] \\
                                \;&=\; a \\
                        a + \bar{0} \;&=\; a \;=\; \bar{0} + a \qedhere
                \end{align*}
        \end{proof}
        
        \begin{theorem}
                        For all $a \in \Q$, there exists a unique $x \in \Q$, satisfying $a + x = \bar{0} = x + a$.
        \end{theorem}
        \begin{proof}
                For $a = [(m, n)] \in \Q$, construct $x = [(-m, n)] \in \Q$.
                Clearly, $a + x = [(mn + n(-m), nn)] = \bar{0}$.
                From commutativity of $+$, $a + x = \bar{0} = x + a$.

                We now show that $x$ is unique. Let $x' \in \Q$, $a + x' = \bar{0} = x' + a$.
                \begin{align*}
                        a + x' \;&=\; \bar{0} \\
                        x + (a + x') \;&=\; x + \bar{0} \\
                        (x + a) + x' \;&=\; x \\
                        \bar{0} + x' \;&=\; x \\
                        x' \;&=\; x     \qedhere
                \end{align*}
        \end{proof}
        \begin{notation}
                We denote $x$ as $-a$ and say that $-a$ is the \textit{negative} of $a$.
        \end{notation}
        \begin{corollary}
                If $a = [(m, n)] \in \Q$, then $-a = [(-m, n)]$.
        \end{corollary}
        \begin{notation}
                For $a, b \in \Q$, we write
                \[
                a - b \;:=\; a + (-b).\quad\quad\quad
                \]
        \end{notation}

        \begin{theorem}
                        For all $a, b \in \Q$, there exists a unique $x \in \Q$ satisfying $a + x = b$.
        \end{theorem}
        \begin{proof}
                From the well-defined nature of $+$, there exists a unique $x = b - a = b + (-a) \in \Q$.
                \begin{align*}
                        a + x \;&=\; a + (b + (-a)) \\
                                \;&=\; a + ((-a) + b) \\
                                \;&=\; (a + (-a)) + b \\
                                \;&=\; \bar{0} + b \\
                                \;&=\; b
                \end{align*}
                Let $x' \in \Q$, $a + x' = b$.
                \begin{align*}
                        a + x' \;&=\; b \\
                        x + (a + x') \;&=\; x + b \\
                        (x + a) + x' \;&=\; x + b \\
                        b + x' \;&=\; b + x \\
                        -b + (b + x') \;&=\; -b + (b + x) \\
                        (-b + b) + x' \;&=\; (-b + b) + x \\
                        \bar{0} + x' \;&=\; \bar{0} + x \\
                        x' \;&=\; x \qedhere
                \end{align*}
        \end{proof}

        \begin{definition}[Multiplication]
        For $a = [(m, n)],\; b = [(p, q)] \in \Q$, we define
        \[
                a \cdot b \;:=\; [(mp, nq)].
        \]
        \end{definition}
        \begin{theorem}
                Multiplication $(\cdot)$ is well-defined, associative and commutative.
        \end{theorem}
        \begin{proof}
                First, we show that $\cdot$ is well-defined.
                Let $a = [(m, n)] = [(m', n')]$, $b = [(p, q)] = [(p', q')] \in \Q$.
                Now, $(m, n) \simQ (m', n')$ and $(p, q) \simQ (p', q')$, from which we have
                $mn' = m'n$ and $pq' = p'q$.
                We claim
                \begin{align*}
                        a\cdot b = [(mp, nq)] \;&=\; [(m'p', n'q')] \\
                                (mp)(n'q') \;&=\; (nq)(m'p') \\
                                (mn')(pq') \;&=\; (m'n)(p'q)
                \end{align*}
                which is clearly true.

                For $a,b,c \in \Z$, let $a = [(m, n)], b = [(p, q)], c = [(r, s)]$.
                \begin{align*}
                        (a\cdot b)\cdot c \;&=\; [(mp, nq)] \cdot [(r, s)] \\
                                \;&=\; [((mp)r, (nq)s)]\\
                                \;&=\; [(mpr, nqs)]\\
                        a\cdot (b\cdot c) \;&=\; [(m, n)] \cdot [(pr, qs)] \\
                                \;&=\; [(m(pr), n(qs))] \\
                                \;&=\; [(mpr, nqs)]
                \end{align*}
                Therefore, $(a\cdot b)\cdot c = a\cdot (b\cdot c)$, i.e. $\cdot$ is associative.
                
                \begin{align*}
                        a\cdot b \;&=\; [(mp, nq)] \\
                                \;&=\; [(pm, qn)] \\
                                \;&=\; b\cdot a
                \end{align*}
                Therefore, $\cdot$ is commutative.
        \end{proof}

        \begin{theorem}
                For all $a \in \Q$, $a\cdot\bar{1} = a = \bar{1}\cdot a$.
        \end{theorem}
        \begin{proof}
                Let $a = [(m, n)] \in \Q$.
                \begin{align*}
                        a\cdot \bar{1} \;&=\; [(m, n)] \cdot [(q, 1)] \\
                                \;&=\; [(m\cdot 1, n\cdot 1)] \\
                                \;&=\; [(m, n)] \\
                                \;&=\; a \\
                        a\cdot \bar{1} \;&=\; a \;=\; \bar{1}\cdot a \qedhere
                \end{align*}
        \end{proof}
        \begin{theorem}
                For all $a \in \Z$, $a\cdot \bar{0} = \bar{0} = \bar{0}\cdot a$.
        \end{theorem}
        \begin{proof}
                Let $a = [(m, n)] \in \Q$.
                \begin{align*}
                        a\cdot \bar{0} \;&=\; [(m, n)]\cdot [(0, 1)]\\
                                \;&=\; [(m\cdot 0, n)] \\
                                \;&=\; \bar{0} \\
                        a\cdot \bar{0} \;&=\; \bar{0} \;=\; \bar{0}\cdot a \qedhere
                \end{align*}
        \end{proof}
        
        \begin{theorem}
                For all $a \in \Q\setminus\{\bar{0}\}$, there exists a unique $x \in \Q$ satisfying $a\cdot x = \bar{1} = x\cdot a$.
        \end{theorem}
        \begin{proof}
                For $a = [(m, n)] \in \Q\setminus\{\bar{0}\}$, construct $x = [(n, m)] \in \Q$.
                Clearly, $a\cdot x = [(mn, nm)] = \bar{1}$.
                From commutativity of $\cdot$, $a\cdot x = \bar{1} = x\cdot a$.

                We now show that $x$ is unique. Let $x' \in \Q$, $a\cdot x' = \bar{1} = x' \cdot a$.
                \begin{align*}
                        a\cdot x' \;&=\; \bar{1} \\
                        x\cdot (a\cdot x') \;&=\; x\cdot\bar{1}\\
                        (x\cdot a)\cdot x' \;&=\; x \\
                        \bar{1}\cdot x' \;&=\; x\\
                        x' \;&=\; x
                \end{align*}
        \end{proof}
        \begin{notation}
                We denote $x$ as $a^{-1}$ and say that $a^{-1}$ is the \emph{inverse} of $a$.
        \end{notation}

        \begin{theorem}
                        For all $a, b \in \Q\setminus\{\bar{0}\}$, there exists a unique $x \in \Q$ satisfying $a \cdot x = b$.
        \end{theorem}
        \begin{proof}
                From the well-defined nature of $\cdot$, there exists a unique $x = a^{-1}\cdot b \in \Q$.
                \begin{align*}
                        a \cdot x \;&=\; a \cdot (a^{-1} \cdot b) \\
                                \;&=\; (a\cdot a^{-1}) \cdot b \\
                                \;&=\; \bar{1}\cdot b \\
                                \;&=\; b
                \end{align*}
                Let $x' \in \Q$, $a \cdot x' = b$.
                \begin{align*}
                        a\cdot x' \;&=\; b \\
                        x \cdot (a\cdot x') \;&=\; x\cdot b \\
                        (x\cdot a)\cdot x' \;&=\; x\cdot b \\
                        b\cdot x' \;&=\; b\cdot x \\
                        b^{-1}\cdot (b\cdot x') \;&=\; b^{-1}\cdot (b\cdot x) \\
                        (b^{-1}\cdot b)\cdot x' \;&=\; (b^{-1}\cdot b)\cdot x\\
                        \bar{1}\cdot x' \;&=\; \bar{1}\cdot x\\
                        x' \;&=\; x \qedhere
                \end{align*}
        \end{proof}

        \begin{theorem}[Distributivity]
                For all $a,b,c \in \Q$, $a\cdot(b + c) = a\cdot b + a\cdot c$.
        \end{theorem}
        \begin{proof}
                For $a,b,c \in \Q$, let $a = [(m, n)], b = [(p, q)], c = [(r, s)]$.
                \begin{align*}
                        a\cdot (b + c) \;&=\; [(m, n)] \cdot [(ps + qr, qs)] \\
                                \;&=\; [(m(ps + qr), nqs)] \\
                                \;&=\; [(mps + nqr, nqs)] \\
                        a\cdot b + a\cdot c \;&=\; [(mp, nq)] + [(mr, ns)] \\
                                \;&=\; [((mp)(ns) + (nq)(mr), (nq)(ns))] \\
                                \;&=\; [(mnps + mnqr, nnqs)] \\
                                \;&=\; [(n(mps + mqr), n(nqs))] \\
                                \;&=\; [(mps + mqr, nqs)]
                \end{align*}
                Hence, $a\cdot (b + c) = a\cdot b + a\cdot c$.
        \end{proof}
        \begin{theorem}
                For all $a, b \in \Q$, $(-a)\cdot b = -(a\cdot b)$.
        \end{theorem}
        \begin{proof}
                \begin{align*}
                        (-a)\cdot b + a\cdot b \;&=\; ((-a) + a)\cdot b \\
                                \;&=\; \bar{0} \cdot b \\
                                \;&=\; \bar{0}\\
                        (-a)\cdot b \;&=\; -(a\cdot b) \qedhere
                \end{align*}
        \end{proof}
        \begin{theorem}
                For all $a, b \in \Q$, $(-a)\cdot(-b) = a\cdot b$.
        \end{theorem}
        \begin{proof}
                \begin{align*}
                        (-a)\cdot(-b) + (-(a\cdot b)) \;&=\; (-a)\cdot(-b) + (-a)\cdot b\\
                                \;&=\; (-a)\cdot((-b) + b) \\
                                \;&=\; (-a)\cdot \bar{0} \\
                                \;&=\; \bar{0}\\
                        (-a)\cdot(-b) \;&=\; a\cdot b \qedhere
                \end{align*}
        \end{proof}

        \begin{lemma}
                If $a = [(m, n)] \in \Q$, $a \neq \bar{0}$, then $m \neq 0$.
        \end{lemma}
        \begin{proof}
                Assume that $m = 0$. Then, we have $(m, n) \simQ \bar{0}$,
                contradicting our premise. Hence, we must have $m\neq 0$.
        \end{proof}
        \begin{theorem}[No zero divisors]
                For all $a, b \in \Q$ with $a,b \neq \bar{0}$, we have $a\cdot b \neq \bar{0}$.
        \end{theorem}
        \begin{proof}
                Let $a = [(m, n)], b = [(p, q)] \in \Q$.
                Note that $m\neq 0$, $p\neq 0$, since $a, b \neq \bar{0}$.

                Assume that our theorem is false, i.e. $a\cdot b = \bar{0}$.
                Then $[(mp, nq)] = \bar{0} \Rightarrow mp = 0$.
                
                From \ref{cor:zero}, $m = 0$ or $p = 0$, which contradicts our premise.

                Hence, $a\cdot b \neq \bar{0}$.
        \end{proof}
        \begin{corollary}
        \label{cor:zeroQ}
                For all $a, b \in \Q$, if $a\cdot b = \bar{0}$, then $a = \bar{0}$ or 
                $b = \bar{0}$.
        \end{corollary}

        \begin{theorem}[Cancellation]
                For $a, b, c \in \Q$ with $a\neq \bar{0}$, we have $a\cdot b = a\cdot c
                \Rightarrow b = c$.
        \end{theorem}
        \begin{proof}
                \begin{align*}
                        a\cdot b \;&=\; a\cdot c\\
                        a^{-1}\cdot (a\cdot b) \;&=\; a^{-1}\cdot (a\cdot c) \\
                        (a^{-1}\cdot a)\cdot b \;&=\; (a^{-1}\cdot a)\cdot c\\
                        b \;&=\; c \qedhere
                \end{align*}
        \end{proof}
        
        \begin{lemma}
                For all $a = [(m, n)] \in \Q$, $a = [(-m, -n)]$.
        \end{lemma}
        \begin{proof}
                It is sufficient to show that $(m, n) \simQ (-m, -n)$, i.e.
                $m(-n) = n(-m)$, which is certainly true.
        \end{proof}

        \begin{definition}[Order]
                For all $a = [(m, n)], b = [(p, q)] \in \Q$, $n, q \in \N$, we say that
                $a > b$ if $mq > np$.
        \end{definition}
        
        \begin{theorem}
                For all $a, b \in \Q$, we have $a\cdot b > \bar{0}$ if $a, b > \bar{0}$ or $a, b < \bar{0}$.
        \end{theorem}
        \begin{proof}
                Let $a = [(m, n)], b = [(p, q)] \in \Q$, $n, q \in \N$.
                From $n, q \in \N = \Z^+$ we have $n > 0$ and $q > 0$, so $nq > 0\Rightarrow nq \in \N$.

                If $a, b > \bar{0}$, then $m > 0$ and $p > 0$.
                Thus, $mp > 0$ which gives $a\cdot b = [(mp, nq)] > 0$.

                If $a, b < 0$, then $0 > a$ and $0 > b$ so $0 > m$ and $0 > p$.
                Thus, $-m, -n > 0$, so $(-m)(-n) = mn > 0$, which gives $a\cdot b > 0$.
        \end{proof}

        \begin{definition}[Identification map]
                Define $\IdZ\colon \Z\to \Q$ by 
                \[
                        \IdZ(n) \;:=\; [(n, 1)],\quad\text{for all }n\in \Z.
                \]
        \end{definition}

        \begin{theorem}
                $\IdZ$ is injective.
        \end{theorem}
        \begin{proof}
                Let $m, n \in \Z$.
                \begin{align*}
                        \IdZ(m) \;&=\; \IdZ(n) \\
                        [(m, 1)] \;&=\; [(n, 1)] \\
                        m\cdot 1\;&=\; n\cdot 1 \\
                        m \;&=\; n
                \end{align*}
                Hence, $\IdZ$ is injective.
        \end{proof}

        \begin{theorem}
                $\IdZ(0) = \bar{0}$.
        \end{theorem}
        \begin{proof}
                \[\IdZ(0) = [(0, 1)] = \bar{0} \qedhere\]
        \end{proof}

        \begin{theorem}
                $\IdZ(1) = \bar{1}$.
        \end{theorem}
        \begin{proof}
                \[\IdZ(1) = [(1 + 1, 1)] = [(2, 1)] = \bar{1} \qedhere\]
        \end{proof}

        \begin{theorem}
                For all $m, n \in \Z$, $\IdZ(m + n) = \IdZ(m) + \IdZ(n)$.
        \end{theorem}
        \begin{proof}
                \begin{align*}
                        \IdZ(m) + \IdZ(n) \;&=\; [(m, 1)] + [(n, 1)] \\
                                \;&=\; [(m\cdot 1 + 1\cdot n, 1\cdot 1)] \\
                                \;&=\; [(m + n, 1)] \\
                                \;&=\; \IdZ(m + n) \qedhere
                \end{align*}
        \end{proof}

        \begin{theorem}
                For all $m, n \in \Z$, $\IdZ(m \cdot n) = \IdZ(m) \cdot \IdZ(n)$.
        \end{theorem}
        \begin{proof}
                \begin{align*}
                        \IdZ(m) \cdot \IdZ(n) \;&=\; [(m, 1)] \cdot [(n, 1)] \\
                                \;&=\; [(m\cdot n, 1\cdot 1)] \\
                                \;&=\; [(mn, 1)] \\
                                \;&=\; \IdZ(m \cdot n) \qedhere
                \end{align*}
        \end{proof}

        \begin{theorem}
                For all $m, n \in \Z$ with $m > n$, $\IdZ(m) > \IdZ(n)$.
        \end{theorem}
        \begin{proof}
                We claim $\IdZ(m) > \IdZ(n)$, i.e. $[(m, 1)] > [(n, 1)]$.
                This is equivalent to $m > n$, which is true.
        \end{proof}
        
        \subsection*{Identification}
        For all $n \in \Z$, we shall identify $\IdZ(n)$ with $n$.
        With this identification,
        \[0 \leftrightarrow \bar{0}\]
        \[1 \leftrightarrow \bar{1}\]
        \[\Z \subset \Q\]
\end{document}
