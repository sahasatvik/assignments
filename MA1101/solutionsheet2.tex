\documentclass[10pt]{article}

\usepackage[T1]{fontenc}
\usepackage{geometry}
\usepackage{amsmath, amssymb, amsthm}

\title{Mathematics I - Problem Sheet II}
\author{Satvik Saha}
\date{}

\geometry{a4paper, margin=1in}
\setlength\parindent{0pt}
\renewcommand{\labelenumi}{(\roman{enumi})}
% \renewcommand\qedsymbol{$\blacksquare$}

\begin{document}
        \par\textbf{IISER Kolkata} \hfill \textbf{Problem Sheet II}
        \vspace{3pt}
        \hrule
        \vspace{3pt}
        \begin{center}
                \LARGE{\textbf{MA 1101 : Mathematics I}}
        \end{center}
        \vspace{3pt}
        \hrule
        \vspace{3pt}
        Satvik Saha, \texttt{19MS154}\hfill\today
        \vspace{20pt}

        \textbf{Solution 1.}\\
        Let $R$ be a relation on $\mathbb{R}^2$ such that
        \[(x_1, x_2)\,R\,(y_1, y_2) \quad\text{if}\quad x_1 = y_1.\]
        \begin{enumerate}
                \item For an arbitrary $(x, y)\in\mathbb{R}^2$, $(x, y)\,R\,(x, y)$, since $x = x$. Therefore, $R$ is reflexive.

                For $(x_1, x_2), (y_1, y_2) \in \mathbb{R}^2$, if $(x_1, x_2)\,R\,(y_1, y_2)$, we can write $x_1 = y_1 \Rightarrow y_1 = x_1$.
                Thus, we have $(y_1, y_2)\,R\,(x_1, x_2)$. Therefore, $R$ is symmetric.

                For $(x_1, x_2), (y_1, y_2), (z_1, z_2) \in \mathbb{R}^2$, if $(x_1, x_2)\,R\,(y_1, y_2)$ and $(y_1, y_2)\,R\,(z_1, z_2)$,
                we can write $x_1 = y_1$ and $y_1 = z_1$, from which we have $x_1 = z_1 \Rightarrow (x_1, x_2)\,R\,(z_1, z_2)$.
                Therefore, $R$ is transitive.

                Hence, $R$ is an equivalence relation.\qed

                \item For $(x_1, x_2) \in \mathbb{R}^2$, we have
                \begin{align*}
                [(x_1, x_2)] \;&=\; \{(y_1, y_2) \in \mathbb{R}^2 : (x_1, x_2)\,R\,(y_1, y_2)\} \\
                        \;&=\; \{(y_1, y_2) \in \mathbb{R}^2 : x_1 = y_1\}\\
                        \;&=\; \{(x_1, y) : y \in \mathbb{R}\}
                \end{align*}
                Therefore, the quotient set of $R$ is given by $$\mathbb{R}/R \;=\; \{L_x : x \in \mathbb{R}\},$$
                where $L_x = \{(x, y) : y \in \mathbb{R}\}$.
                Clearly, each equivalence class $L_x \in \mathbb{R}/R$ is a vertical line in the Cartesian plane, passing through $(x, 0)$.
        \end{enumerate}

        \textbf{Solution 2.}\\
        Let $R$ be a relation on $\mathbb{R}^2$ such that
        \[(x_1, x_2)\,R\,(y_1, y_2) \quad\text{if}\quad x_1^2 + x_2^2 = y_1^2 + y_2^2\]

        \begin{enumerate}
                \item For an arbitrary $(x, y) \in \mathbb{R}^2$, $(x, y)\,R\,(x, y)$, since $x^2 + y^2 = x^2 + y^2$. Therefore, $R$
                is reflexive.

                For $(x_1, x_2), (y_1, y_2) \in \mathbb{R}^2$, if $(x_1, x_2)\,R\,(y_1, y_2)$, we can write 
                $x_1^2 + x_2^2 = y_1^2 + y_2^2 \Rightarrow y_1^2 + y_2^2 = x_1^2 + x_2^2$. Thus, we have $(y_1, y_2)\,R\,(x_1, x_2)$.
                Therefore, $R$ is symmetric.

                For $(x_1, x_2), (y_1, y_2), (z_1, z_2) \in \mathbb{R}^2$, if $(x_1, x_2)\,R\,(y_1, y_2)$ and $(y_1, y_2)\,R\,(z_1, z_2)$,
                we can write $x_1^2 + x_2^2 = y_1^2 + y_2^2$ and $y_1^2 + y_2^2 = z_1^2 + z_2^2$, from which we have
                $x_1^2 + x_2^2 = z_1^2 + z_2^2 \Rightarrow (x_1, x_2)\,R\,(z_1, z_2)$. Therefore, $R$ is transitive.

                Hence, $R$ is an equivalence relation.\qed
                
                \item For $(x_1, x_2) \in \mathbb{R}^2$, we have
                \begin{align*}
                [(x_1, x_2)] \;&=\; \{(y_1, y_2) \in \mathbb{R}^2 : (x_1, x_2)\,R\,(y_1, y_2)\} \\
                        \;&=\; \{(y_1, y_2) \in \mathbb{R}^2 : x_1^2 + x_2^2 = y_1^2 + y_2^2\}
                \end{align*}
                Clearly, each equivalence class is a circle of radius $r = \sqrt{x_1^2 + x_2^2}$ centred at the origin.
                Such a circle can be denoted by $C_r = \{(x, y) \in \mathbb{R}^2 : x^2 + y^2 = r^2\}$.
                Therefore, the quotient set of $R$ is given by $$\mathbb{R}/R \;=\; \{C_r : r \geq 0\}.$$
                
        \end{enumerate}

        \textbf{Solution 5.}\\
        Let $n \in \mathbb{N}$ and $X$ be a set of $n$ elements.
        An arbitrary relation $R$ on $X$ is a subset of the Cartesian product $X\times X = X^2$.
        Note that for $(a, b) \in X^2$, $a$ can be any of the $n$ elements in $X$, and $b$ can be independently any
        of the $n$ elements in $X$. Thus, we have a total of $n^2$ elements in $X^2$.
        \begin{enumerate}
                \item Since $R$ is any subset $R \subseteq X^2$, we can say that a relation on $X$ is any $R \in \mathcal{P}(X^2)$.
                Thus, the total number of possible relations $R$ is the number of elements in $\mathcal{P}(X^2)$, i.e., $2^{n^2}$.

                \item Let $D = \{(x, x) : x \in X\}$ be the set of the diagonal elements of $X^2$. Clearly, there are $n$ elements in $D$.
                A reflexive relation $R$ must have $D \subseteq R$. Thus, of the $n^2$ elements of $X^2$, the $n$ diagonal elements
                are fixed -- the remaining $n^2 - n$ elements can be chosen to be or not to be in $R$, giving us a total of
                $2^{n^2 - n}$ such relations.

                \item Since $xRy \Rightarrow yRx$ if $x = y$, each of the $n$ diagonal elements of $X^2$ may or may not be present in a
                symmetric relation $R$ on $X$. Also, the presence of $(x, y) \in X^2\setminus D$ in $R$ forces the presence
                of $(y, x)$ in $R$. Thus, we have $(n^2 - n)/2$ choices for the non-diagonal elements, giving a total of
                $2^n \cdot 2^{(n^2 - n)/2} = 2^{(n^2 + n)/2}$ such relations.

                \item As before, we have $(n^2 - n)/2$ choices for non-diagonal elements to fulfil symmetry. The remaining diagonal
                elements are fixed to fulfil reflexivity, giving a total of $2^{(n^2 - n)/2}$ such relations.
        \end{enumerate}

\end{document}
