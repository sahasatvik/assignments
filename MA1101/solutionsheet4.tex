\documentclass[10pt]{article}

\usepackage[T1]{fontenc}
\usepackage{geometry}
\usepackage{amsmath, amssymb, amsthm}

\title{Mathematics I - Problem Sheet IV}
\author{Satvik Saha}
\date{}

\geometry{a4paper, margin=1in}
\setlength\parindent{0pt}
\renewcommand{\labelenumi}{(\roman{enumi})}
% \renewcommand\qedsymbol{$\blacksquare$}

\begin{document}
        \par\textbf{IISER Kolkata} \hfill \textbf{Problem Sheet IV}
        \vspace{3pt}
        \hrule
        \vspace{3pt}
        \begin{center}
                \LARGE{\textbf{MA 1101 : Mathematics I}}
        \end{center}
        \vspace{3pt}
        \hrule
        \vspace{3pt}
        Satvik Saha, \texttt{19MS154}\hfill\today
        \vspace{20pt}

        \textbf{Solution 1.}
        \begin{enumerate}
                \item Let $P(n)$ be the statement
                \[1 + 2 + \dots + n \;=\; \frac{1}{2}n(n + 1) \quad\quad\text{for all }n \in \mathbb{N}.\]
                \paragraph{Base step}
                We establish $P(1)$. Clearly, $1 = \frac{1}{2}1(1 + 1)$. Thus, $P(1)$ is true.
                \paragraph{Inductive step}
                We assume $P(k)$ is true. We will show that $P(k + 1)$ is true.
                \begin{align*}
                        1 + 2 + \dots + k + (k + 1)
                                \;&=\; [1 + 2 + \dots + k] + (k + 1) \\
                                \;&=\; \frac{1}{2}k(k + 1) + (k + 1) \tag{From $P(k)$}\\
                                \;&=\; \frac{1}{2}(k + 2)(k + 1) \\
                                \;&=\; \frac{1}{2}(k + 1)((k + 1) + 1)
                \end{align*}

                Hence, by the principle of mathematical induction, $P(n)$ is true for all $n \in \mathbb{N}$.
                
                \item Let $P(n)$ be the statement
                \[1^2 + 2^2 + \dots + n^2 \;=\; \frac{1}{6}n(n + 1)(2n + 1) \quad\quad\text{for all }n \in \mathbb{N}.\]
                \paragraph{Base step}
                We establish $P(1)$. Clearly, $1 = \frac{1}{6}1(1 + 1)(2 + 1)$. Thus, $P(1)$ is true.
                \paragraph{Inductive step}
                We assume $P(k)$ is true. We will show that $P(k + 1)$ is true.
                \begin{align*}
                        1^2 + 2^2 + \dots + k^2 + (k + 1)^2
                                \;&=\; [1^2 + 2^2 + \dots + k^2] + (k + 1)^2 \\
                                \;&=\; \frac{1}{6}k(k + 1)(2k + 1) + (k + 1)^2 \tag{From $P(k)$}\\
                                \;&=\; \frac{1}{6}(k + 1)(2k^2 + k + 6k + 6) \\
                                \;&=\; \frac{1}{6}(k + 1)(2k^2 + 7k + 6) \\
                                \;&=\; \frac{1}{6}(k + 1)(k + 2)(2k + 3) \\
                                \;&=\; \frac{1}{6}(k + 1)((k + 1) + 1)(2(k + 1) + 1)
                \end{align*}

                Hence, by the principle of mathematical induction, $P(n)$ is true for all $n \in \mathbb{N}$.
                
                \item Let $P(n)$ be the statement
                \[1^2 + 3^2 + \dots + (2n - 1)^2 \;=\; \frac{1}{3}(4n^3 - n) \quad\quad\text{for all }n \in \mathbb{N}.\]
                \paragraph{Base step}
                We establish $P(1)$. Clearly, $1 = \frac{1}{3}1(4 - 3)$. Thus, $P(1)$ is true.
                \paragraph{Inductive step}
                We assume $P(k)$ is true. We will show that $P(k + 1)$ is true.
                \begin{align*}
                        1^2 + 3^2 + \dots + (2k - 1)^2 + (2k + 1)^2
                                \;&=\; [1^2 + 3^2 + \dots + (2k - 1)^2] + (2k + 1)^2 \\
                                \;&=\; \frac{1}{3}(4k^3 - k) + (2k + 1)^2 \tag{From $P(k)$}\\
                                \;&=\; \frac{1}{3}(4k^3 - k + 12k^2 + 12k + 3)\\
                                \;&=\; \frac{1}{3}(4(k^3 + 3k^2 + 3k + 1) - k - 1)) \\
                                \;&=\; \frac{1}{3}(4(k + 1)^3 - (k + 1))
                \end{align*}

                Hence, by the principle of mathematical induction, $P(n)$ is true for all $n \in \mathbb{N}$.

                \item Let $P(n)$ be the statement
                \[1^3 + 2^3 + \dots + n^3 \;=\; \frac{1}{4}n^2(n + 1)^2 \quad\quad\text{for all }n \in \mathbb{N}.\]
                \paragraph{Base step}
                We establish $P(1)$. Clearly, $1 = \frac{1}{4}1(1 + 1)^2$. Thus, $P(1)$ is true.
                \paragraph{Inductive step}
                We assume $P(k)$ is true. We will show that $P(k + 1)$ is true.
                \begin{align*}
                        1^3 + 2^3 + \dots + k^3 + (k + 1)^3
                                \;&=\; [1^3 + 2^3 + \dots + k^3] + (k + 1)^3 \\
                                \;&=\; \frac{1}{4}k^2(k + 1)^2 + (k + 1)^3 \tag{From $P(k)$}\\
                                \;&=\; \frac{1}{4}(k + 1)^2(k^2 + 4k + 4)\\
                                \;&=\; \frac{1}{4}(k + 1)^2(k + 2)^2\\
                                \;&=\; \frac{1}{4}(k + 1)^2((k + 1) + 1)^2
                \end{align*}

                Hence, by the principle of mathematical induction, $P(n)$ is true for all $n \in \mathbb{N}$.
                
                \item Let $P(n)$ be the statement
                \[\sum_{r = 1}^n r(r + 1)\dots(r + 9) \;=\; \frac{1}{11}n(n + 1)\dots(n + 10) \quad\quad\text{for all }n \in \mathbb{N}.\]
                \paragraph{Base step}
                We establish $P(1)$. Clearly, 
                \[1(1 + 1)\dots(1 + 9) \;=\; \frac{1}{11}1(1 + 1)\dots(1 + 9)(1 + 10)\]
                Thus, $P(1)$ is true.
                \paragraph{Inductive step}
                We assume $P(k)$ is true. We will show that $P(k + 1)$ is true.
                \begin{align*}
                \sum_{r = 1}^{k + 1} r(r + 1)\dots(r + 9)
                                \;&=\; \left[\sum_{r = 1}^{k} r(r + 1)\dots(r + 9)\right] \;+\; (k + 1)(k + 2)\dots(k + 1 + 9) \\
                                \;&=\; \frac{1}{11}k(k + 1)\dots(k + 10) \;+\; (k + 1)(k + 2)\dots(k + 1 + 9) \tag{From $P(k)$}\\
                                \;&=\; \frac{1}{11}(k + 1)\dots(k + 10)(k + 11)\\
                                \;&=\; \frac{1}{11}(k + 1)\dots((k + 1) + 9)((k + 1) + 10)
                \end{align*}

                Hence, by the principle of mathematical induction, $P(n)$ is true for all $n \in \mathbb{N}$.
        \end{enumerate}
        \textbf{Solution 2.}
        \begin{enumerate}
                \item Let $P(n)$ be the statement that for all $n \in \mathbb{N}$,
                \[3^n > n^2\]
                \paragraph{Base step}
                We establish $P(1)$ and $P(2)$. Clearly, $3^1 > 1^2$. Thus, $P(1)$ is true.
                Again, $3^2 = 9 > 8 = 2^2$. Thus, $P(2)$ is true.
                \paragraph{Inductive step}
                We assume $P(k)$ is true. We will show that $P(k + 1)$ is true.
                \begin{align*}
                        3^{k + 1} \;=\; 3\cdot 3^k
                                \;>\; 3\cdot k^2
                \end{align*}
                We must show $3k^2 > (k + 1)^2 \Leftrightarrow 3k^2 - (k + 1)^2 > 0$.
                \begin{align*}
                        3k^2 - (k + 1)^2 \;=\; 2k^2 - 2k - 1
                                \;=\; k^2 + (k - 1)^2  - 2
                \end{align*}
                Clearly, for $k \geq 2$, $k^2 > 2$, so $k^2 + (k - 1)^2 > 2$, and we are done.

                Hence, by the principle of mathematical induction, $P(n)$ is true for all $n \in \mathbb{N}$.

                \item Let $P(n)$ be the statement that for all $n \in \mathbb{N}$ and $x > -1$,
                \[(1 + x)^n \geq 1 + nx. \tag{Bernoulli's Inequality}\]
                \paragraph{Base Step}
                We establish $P(1)$. 
                Clearly, $(1 + x)^1 \geq (1 + 1\cdot x)$, thus $P(1)$ is true.
                \paragraph{Inductive Step}
                We assume $P(k)$ is true. We will show that $P(k + 1)$ is true.
                \begin{align*}
                        (1 + x)^{k + 1} \;&=\; (1 + x)^k\cdot (1 + x) \\
                                \;&\geq\; (1 + kx)\cdot (1 + x) \tag{$x + 1 > 0$}\\
                                \;&=\; (1 + x + kx + kx^2)\\
                                \;&\geq\; (1 + (k + 1)x) \tag{$k > 0$ and $x^2 \geq 0$}
                \end{align*}
                
                Hence, by the principle of mathematical induction, $P(n)$ is true for all $n \in \mathbb{N}$.

                \item Let $P(n)$ be the statement that for all $n \geq 5$, $n \in \mathbb{N}$,
                \[\binom{2n}{n} \;<\; 2^{2n - 2}.\]
                \paragraph{Base Step}
                We establish $P(5)$. Now, $\binom{2n}{n} = 252$, while $2^{10 - 2} = 256$. Thus, $P(5)$ is true.
                \paragraph{Inductive Step}
                We assume $P(k)$ is true. We will show that $P(k + 1)$ is true.
                \begin{align*}
                \binom{2(k + 1)}{k + 1} \;&=\; \frac{(2k + 2)!}{(k + 1)!^2} \\
                        \;&=\; \frac{(2k + 2)(2k + 1)}{(k + 1)^2} \binom{2n}{n} \\
                        \;&<\; 2\cdot \frac{2k + 1}{k + 1} \cdot 2^{2k - 2} \\
                        \;&<\; 2\cdot \frac{2k + 2}{k + 1} \cdot 2^{2k - 2} \\
                        \;&<\; 2^{2(k + 1) - 2}
                \end{align*}
                
                Hence, by the principle of mathematical induction, $P(n)$ is true for all $n \geq 5$, $n \in \mathbb{N}$.
        \end{enumerate}
        
        \clearpage
        \textbf{Solution 3.}
        \begin{enumerate}
                \item Let $P(n)$ be the statement that every $n \geq 2$, $n \in \mathbb{N}$ has a prime divisor.
                We prove this using the principle of strong mathematical induction.
                \paragraph{Base Step}
                We establish $P(2)$. Clearly, $2$ is a prime divisor of itself, so $P(2)$ is true.
                \paragraph{Inductive Step}
                We assume that the statements $P(2), P(3), \dots, P(k-1)$ are all true. We will show that $P(k)$ is true.

                If $k \geq 2$ is prime, then we are done, as $k$ is a prime divisor of itself. Otherwise, if
                $k$ is not prime, then $k = ab$ for some $1 < a, b < k$ and $a, b \in \mathbb{N}$. We see that $a \geq 2$,
                so by the induction hypothesis, $a$ has a prime divisor $p \in \mathbb{N}$, i.e.,  $a = pc$ for some $c \in \mathbb{N}$.
                Thus, $k = (pc)b = p(cb)$, and $cb \in \mathbb{N}$, so $p$ is a prime factor of $k$. This proves $P(k)$.

                Hence, by the principle of strong induction, $P(n)$ is true for all $n \geq 2$, $n \in \mathbb{N}$.
                
                \item We define the Fibonacci sequence $(f_n)_{n\geq 0}$ as follows.
                \begin{align*}
                        f_0 \;&:=\; 0 \\
                        f_1 \;&:=\; 1 \\
                        f_n \;&:=\; f_{n - 1} + f_{n - 2}, \quad\quad\text{for all }n \geq 2
                \end{align*}
                \begin{enumerate}
                        \item We wish to show that for all $n \in \mathbb{N}$,
                        \[f_n \;=\; \frac{1}{\sqrt{5}}\left[ \left(\frac{1 + \sqrt{5}}{2}\right)^{\!\!n} - \left(\frac{1 - \sqrt{5}}{2}\right)^{\!\!n} \right] 
                        \tag{Binet's formula} \]
                        We prove this using the principle of strong mathematical induction.
                        Let $P(n)$ be the aforementioned statement, and let $\varphi = (1 + \sqrt{5})/2$ and $\psi = (1 - \sqrt{5})/2$.
                        Note that $\varphi$ and $\psi$ both satisfy $x^2 = x + 1$.
                        \[
                        \left(\frac{1 \pm \sqrt{5}}{2}\right)^2 \;=\; \frac{6 \pm 2\sqrt{5}}{4} \;=\; \frac{1 \pm \sqrt{5}}{2} + 1
                        \]
                        \paragraph{Base Step}
                        We establish $P(1)$. Clearly, $f_1 = 1 = (\varphi - \psi)/\sqrt{5}$. Thus, $P(1)$ is true.
                        \paragraph{Inductive Step}
                        We assume that the statements $P(2), P(3), \dots, P(k)$ are all true. We will show that $P(k + 1)$ is true.
                        \begin{align*}
                                f_{n + 1} \;&=\; f_{n} + f_{n - 1} \\
                                        \;&=\; \frac{1}{\sqrt{5}}(\varphi^n - \psi^n) + \frac{1}{\sqrt{5}}(\varphi^{n - 1} + \psi^{n - 1}) \\
                                        \;&=\; \frac{1}{\sqrt{5}}(\varphi^{n - 1}(\varphi + 1) - \psi^{n - 1}(\psi + 1)) \\
                                        \;&=\; \frac{1}{\sqrt{5}}(\varphi^{n - 1}(\varphi^2) - \psi^{n - 1}(\psi^2)) \\
                                        \;&=\; \frac{1}{\sqrt{5}}(\varphi^{n + 1} - \psi^{n + 1})
                        \end{align*}
                        Hence, by the principle of strong induction, $P(n)$ is true for all $n \in \mathbb{N}$.
                        
                \end{enumerate}
        \end{enumerate}
\end{document}
