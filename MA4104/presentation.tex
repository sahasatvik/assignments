\documentclass[11pt]{article}

\usepackage[T1]{fontenc}
\usepackage{geometry}
\usepackage{amsmath, amssymb, amsthm}
\usepackage{bm}
\usepackage[scr]{rsfso}
\usepackage{graphicx}
\usepackage{float}
\usepackage{xcolor}
\usepackage{hyperref}

\geometry{a4paper, margin=1in, headheight=14pt}

\setlength{\parindent}{0em}
\setlength{\parskip}{0.8em}

\renewcommand{\labelenumi}{(\alph{enumi})}
\renewcommand{\labelenumii}{(\roman{enumii})}

\def\C{\mathbb{C}}
\def\R{\mathbb{R}}
\def\Q{\mathbb{Q}}
\def\Z{\mathbb{Z}}
\def\N{\mathbb{N}}

\newtheorem{theorem}{Theorem}[section]
\newtheorem{corollary}{Corollary}[theorem]
\newtheorem{lemma}[theorem]{Lemma}

\theoremstyle{definition}
\newtheorem{definition}{Definition}[section]

\theoremstyle{remark}
\newtheorem*{remark}{Remark}
\newtheorem*{example}{Example}

\title{
    \Large\textsc{MA4104: Algebraic Topology} \\
    \vspace{10pt}
    % \textit{\Large } \\
    % \vspace{6pt}
    {\huge\bf The fundamental group of $\C\times\C\setminus\Delta$.}
    \vspace{-1em}
}
\author{
    \large Satvik Saha%
    % \thanks{Email: \tt ss19ms154@iiserkol.ac.in}
    \\\textsc{\small 19MS154}
}
\date{\normalsize
    \textit{Indian Institute of Science Education and Research, Kolkata, \\
    Mohanpur, West Bengal, 741246, India.} \\
    % \vspace{10pt}
    % \today
}


\begin{document}
    \maketitle

    Consider the space $\C\times\C\setminus\Delta$, where $\Delta = \{(z, z): z \in
    \C\}$. We may identify $\R^2 \cong \C$ via the usual map $(x, y) \mapsto z + iy$;
    with this, the space under question may be identified with \[
        \R^4\setminus\{(x, y, x, y): x, y \in \R\} \;=\;
        \R^4\setminus\operatorname{span}\{e_1 + e_3, e_2 + e_4\}.
    \] The homeomorphism \[
        \varphi\colon \R^4 \to \R^4, \qquad
        (x, y, z, w) \mapsto (x + z, y + w, x - z, y - w)
    \] restricts to the homeomorphism \[
        \R^4\setminus\operatorname{span}\{e_1, e_2\} \cong
        \R^4\setminus\operatorname{span}\{e_1 + e_3, e_2 + e_4\}.
    \] In other words, our original space is homeomorphic to \[
        \R^4\setminus(\R^2\times\{0\}\times\{0\}).
    \] However, this can be identified again with \[
        (\C\times\C) \setminus(\C\times\{0\}) \;=\; \C \times (\C\setminus\{0\}).
    \] Now, $\C$ is contractible, and $\C\setminus\{0\}$ deformation retracts to the
    unit circle $S^1$. With this, we have established the homotopy equivalence \[
        \C\times\C\setminus\Delta \;\sim\; \{0\} \times S^1 \cong S^1.
    \] In particular, this demonstrates that the space which we have been examining
    is path connected. As a result, we can safely discuss its first fundamental group
    without reference to a particular basepoint (if insisted upon, pick $(1, -1) \in
    \C\times\C\setminus\Delta$, which gets mapped to $e_3 \in \R^4\setminus(\R^2
    \times \{0\} \times \{0\})$, hence $(0, 1) \in \{0\}\times S^1$). Thus, \[
        \pi_1(\C\times\C\setminus\Delta) \cong \pi_1(S^1) \cong \Z.
    \]


    \subsection*{Identification of $\C$ with $\R^2$}

    We use the standard identification \[
        \phi\colon \R^2 \to \C, \qquad x + iy \mapsto x + iy
    \] and its inverse to interchangeably talk about $\C$ and $\R^2$ here.


    \subsection*{Homeomorphism of $\R^4$}

    The map $\varphi\colon \R^4 \to \R^4$ described earlier can be written in the
    following manner. \[
        \begin{bmatrix}x\\ y\\ z\\ w\end{bmatrix} \mapsto
        \begin{bmatrix}
            1 & 0 & 1 & 0 \\
            0 & 1 & 0 & 1 \\
            1 & 0 & -1 & 0 \\
            0 & 1 & 0 & -1
        \end{bmatrix}
        \begin{bmatrix}x\\ y\\ z\\ w\end{bmatrix}
    \] The $4\times 4$ matrix here is of full rank; all four columns are orthogonal,
    hence linearly independent. As a result, $\varphi$ is a bijective linear map,
    hence a homeomorphism.

    By restricting $\varphi$ to $\R^4\setminus\operatorname{span}\{e_1, e_2\}$, we
    obtain a homeomorphism onto its image. Since $\varphi(e_1) = e_1 + e_3$ and
    $\varphi(e_2) = e_2 + e_4$, we have removed precisely $\operatorname{span}\{e_1 +
    e_3, e_2 + e_4\}$ from the image of $\varphi$. Thus, \[
        \R^4\setminus\operatorname{span}\{e_1, e_2\} \cong
        \R^4\setminus\operatorname{span}\{e_1 + e_3, e_2 + e_4\}
    \] via $\varphi$, as desired.


    \subsection*{Deformation retracts}

    The deformation retract of $\C$ onto the point $0$ looks like \[
        h_1\colon [0, 1]\times \C \to \C, \qquad
        (t, z) \mapsto (1 - t)z.
    \] Similarly, the deformation retract of $\C\setminus\{0\}$ onto the circle
    $S^1$ looks like \[
        h_2\colon [0, 1]\times \C\setminus\{0\} \to \C\setminus\{0\}, \qquad
        (t, z) \mapsto (1 - t)z + tz/|z|.
    \] Note that $(1 - t)z + tz/|z| \neq 0$; if it were, then $|z| = t / (t - 1) <
    0$, a contradiction.

    These can be performed on $\C\times(\C\setminus\{0\})$ one after another on the
    corresponding slots, or in one go as \[
        h\colon [0, 1] \times \C \times \C\setminus\{0\} \to \C \times
        \C\setminus\{0\}, \qquad
        (t, z_1, z_2) \mapsto (t, (1 - t)z_1, (1 - t)z_2 + tz_2 / |z_2|).
    \] Note that $h(0, \cdot, \cdot) = \operatorname{id}_{\C\times\C\setminus\{0\}}$
    and $h(1, z_1, z_2) = (0, z_2 / |z_2|) \in \{0\} \times S^1$. Also, it is clear
    that each $h(t, \cdot, \cdot)$ fixes $\{0\}\times S^1$; when $z_2 \in S^1$, we
    have \[
        (1 - t)z_2 + tz_2 / |z_2| = (1 - t)z_2 + tz_2 = z_2.
    \] Thus, $h$ describes
    a deformation retraction of $\C\times \C\setminus\{0\}$ onto $\{0\} \times S^1$,
    which is homeomorphic to just $S^1$.


    \subsection*{Path connectedness of a space and its deformation retract}

    Suppose that $h\colon I \times X \to X$ is a deformation retract of $X$ onto
    $A\subseteq X$. Then, given $x \in X$, we have a path $h(\cdot, x)\colon I \to X$
    joining $h(0, x) = x$ with $h(1, x) \in A$. If in addition we know that $A$ is
    path connected, then given any $x, x' \in X$, we can pick a path $\gamma$ joining
    $h(1, x)$ and $h(1, x')$ in $A$. Thus, $h(\cdot, x) * \gamma * \overline{h(\cdot,
    x')}$ describes a path joining $x$ and $x'$, proving that $X$ is path connected.


    \subsection*{* Analogy with $\R^3\setminus(\R\times \{0\}\times\{0\})$}

    Here, we have shown that removing a 2-plane from $\R^4$ keeps it path connected.
    However, this is a bit difficult to visualize. An analogous construction involves
    removing a line from $\R^3$. Now, it is clear that this space is path connected;
    indeed, it deformation retracts to a circle once again. Additionally, it is easy
    to see that each based homotopy class of loops is completely determined by the
    number of times it winds around the line that has been removed.



\end{document}
