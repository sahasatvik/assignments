\documentclass[11pt]{article}

\usepackage[T1]{fontenc}
\usepackage{geometry}
\usepackage{amsmath, amssymb, amsthm}
\usepackage[scr]{rsfso}
\usepackage[%
    hidealllines=true,%
    innerbottommargin=15,%
    nobreak=true,%
]{mdframed}
\usepackage{xcolor}
\usepackage{graphicx}
\usepackage{fancyhdr}
\usepackage{hyperref}

\geometry{a4paper, margin=1in, headheight=14pt}

\pagestyle{fancy}
\fancyhf{}
\renewcommand\headrulewidth{0.4pt}
\fancyhead[L]{\scshape MA3104: Linear Algebra II}
\fancyhead[R]{\scshape \leftmark}
\rfoot{\footnotesize\it Updated on \today}
\cfoot{\thepage}

\newcommand{\C}{\mathbb{C}}
\newcommand{\R}{\mathbb{R}}
\newcommand{\Q}{\mathbb{Q}}
\newcommand{\Z}{\mathbb{Z}}
\newcommand{\N}{\mathbb{N}}
\newcommand{\F}{\mathbb{F}}

\newcommand{\I}{I}

\renewcommand{\vec}[1]{\boldsymbol{#1}}
\newcommand{\vx}{\vec{x}}
\newcommand{\vy}{\vec{y}}
\newcommand{\vv}{\vec{v}}
\newcommand{\vw}{\vec{w}}

\newcommand{\alg}[1]{\mathscr{#1}}
\newcommand{\algL}{\alg{L}}

\newcommand{\ip}[2]{\langle #1, #2 \rangle}

\renewcommand{\ker}{\operatorname{ker}}
\newcommand{\im}{\operatorname{im}}
\newcommand{\dim}{\operatorname{dim}}

\newmdtheoremenv[%
    backgroundcolor=blue!10!white,%
]{theorem}{Theorem}[section]
\newmdtheoremenv[%
    backgroundcolor=violet!10!white,%
]{corollary}{Corollary}[theorem]
\newmdtheoremenv[%
    backgroundcolor=teal!10!white,%
]{lemma}[theorem]{Lemma}

\theoremstyle{definition}
\newmdtheoremenv[%
    backgroundcolor=green!10!white,%
]{definition}{Definition}[section]
\newmdtheoremenv[%
    backgroundcolor=red!10!white,%
]{exercise}{Exercise}[section]

\theoremstyle{remark}
\newtheorem*{remark}{Remark}
\newtheorem*{example}{Example}
\newtheorem*{solution}{Solution}

\surroundwithmdframed[%
    linecolor=black!20!white,%
    hidealllines=false,%
    innertopmargin=5,%
    innerbottommargin=10,%
    skipabove=0,%
    skipbelow=0,%
]{example}

\numberwithin{equation}{section}

\title{
    \Large\textsc{MA3104} \\
    \Huge \textbf{Linear Algebra II} \\
    \vspace{5pt}
    \Large{Autumn 2021}
}
\author{
    \large Satvik Saha
    \\\textsc{\small 19MS154}
}
\date{\normalsize
    \textit{Indian Institute of Science Education and Research, Kolkata, \\
    Mohanpur, West Bengal, 741246, India.} \\
}

\begin{document}
    \maketitle

    \tableofcontents

    \section{Linear operators on a vector space}

    \subsection{Preliminaries}
    We discuss finite dimensional vector spaces $V$ over some field $\F$, along with
    linear operators $T\colon V \to V$. We also assume that $V$ has the inner
    product $\ip{\cdot}{\cdot}$.

    \begin{theorem}
        Let $\algL(V)$ be the set of all linear operators on the vector space $V$.
        Then, $\algL(V)$ is a linear algebra over the field $\F$.
    \end{theorem}

    \subsection{Field ideals}
    \begin{definition}
        Let $\F$ be a field, and let $\F[x]$ be the ring of polynomials with
        coefficients from $\F$. An ideal in $\F[x]$ is a subspace $I$ such that $fg
        \in I$ for all $f \in \F[x]$, $g \in I$.
    \end{definition}

    \begin{definition}
        Given $p \in \F[x]$, the set \[
            I_p = \F[x]p = \{fp : f \in \F[x]\}
        \] is called the principal ideal generated by $p$.
    \end{definition}

    \begin{theorem}
        Every ideal in $\F[x]$ is a principal ideal.
        \begin{remark}
            This is analogous to the theorem which states that every subgroup of a
            cyclic group is cyclic. Both lead to a precise definition of the greatest
            common divisor.
        \end{remark}
    \end{theorem}
    \begin{corollary}
        Let $M$ be a non-trivial ideal in $\F[x]$. Then, there exists a unique monic
        polynomial $p \in \F[x]$ (leading coefficient $1$) such that $M$ is precisely
        the principal ideal generated by $p$.
    \end{corollary}

    \subsection{Eigenvalues and eigenvectors}
    \begin{definition}
        Let $T \in \algL(V)$ and $c \in \F$. We say that $c$ is an eigenvalue or
        characteristic value of $T$ if $T\vv = c\vv$ for some non-zero $\vv \in V$.
        The vector $\vv$ is called an eigenvector of $T$.
    \end{definition}

    \begin{theorem}
        Let $T \in \algL(V)$ and $c \in \F$. The following are equivalent.
        \begin{enumerate}
            \itemsep0em 
            \item $c$ is an eigenvalue of $T$.
            \item $T - c \I$ is singular.
            \item $\det(T - c \I) = 0$.
        \end{enumerate}
    \end{theorem}

    \begin{definition}
        The polynomial $\det(T - x\I)$ is called the characteristic polynomial
        of $T$.
    \end{definition}
    
    \begin{definition}
        Two linear operators $S, T \in \algL(V)$ are similar if there exists an
        invertible operator $X \in \algL(V)$ such that $S = X^{-1} T X$.

        \begin{remark}
            Similarity is an equivalence relation on $\algL(V)$, thus partitioning it
            into similarity classes.
        \end{remark}
    \end{definition}

    \begin{lemma}
        Similar linear operators have the same characteristic polynomial.
    \end{lemma}
    \begin{proof}
        Let $S, T$ be similar with $S = X^{-1}TX$. Then,
        \begin{align*}
            \det(S \,-\, x\I) &= \det(X^{-1}TX \,-\, xX^{-1}X) \\
                &= \det(X^{-1})\,\det(T \,-\, x\I)\,\det(X) \\
                &= \det(T \,-\, x\I). \qedhere
        \end{align*}
    \end{proof}
    
    \begin{definition}
        A linear operator $T \in \algL(V)$ is diagonalizable if there is a basis of
        $V$ consisting of eigenvectors of $T$.
        \begin{remark}
            The matrix of $T$ with respect to such a basis is diagonal.
        \end{remark}
    \end{definition}

    \subsection{Annihilating polynomials}
    
    \begin{definition}
        An polynomial $p$ such that $p(T) = 0$ for a given linear operator $T \in
        \algL(V)$ is called an annihilating polynomial of $T$.
    \end{definition}

    \begin{lemma}
        Every linear operator $T\in\algL(V)$, where $V$ is finite dimensional, has
        a non-trivial annihilating polynomial.
    \end{lemma}
    \begin{proof}
        Note that the operators $\I, T, T^2, \dots, T^{n^2} \in \algL(V)$, of
        which there are $n^2 + 1$, are linearly dependent, since $\dim{\algL(V)} =
        n^2$.
    \end{proof}

    \begin{lemma}
        The annihilating polynomials of $T$ form an ideal in $\F[x]$.
    \end{lemma}

    \begin{definition}
        The minimal polynomial of $T$ is the unique monic generator of the
        annihilating polynomials of $T$.
        \begin{remark}
            The minimal polynomial of $T$ divides all its annihilating polynomials.
        \end{remark}
    \end{definition}

    \begin{theorem}
        The minimal polynomial and characteristic polynomial of $T$ share the same
        roots, except for multiplicities.
    \end{theorem}
    \begin{proof}
        Let $p$ be the minimal polynomial of $T$ and let $f$ be its characteristic
        polynomial.

        First, let $c \in \F$ be a root of the minimal polynomial, i.e.\ $p(c) = 0$.
        The Division Algorithm guarantees \[
            p(x) = (x - c)\, q(x)
        \] for some monic polynomial $q$. By the minimality of the degree of $p$, we
        have $q(T) \neq 0$, hence there exists non-zero $\vv \in V$ such that $\vw =
        q(T)\,\vv \neq \vec{0}$. Thus, $p(T)\,\vv = \vec{0}$ gives \[
            (T - c \I)\, q(T)\, \vv = \vec{0}, \qquad T\vw = c\vw,
        \] which shows that $c$ is an eigenvalue, i.e.\ a root of the characteristic
        polynomial $f$.

        Next, suppose that $c$ is a root of the characteristic polynomial, i.e.\
        $f(c) = 0$. Thus, $c$ is an eigenvalue of $T$, hence there exists non-zero
        $\vv \in V$ such that $T\vv = c\vv$. This gives $p(T)\,\vv = p(c)\,\vv$, but
        $p(T) = 0$ identically, forcing $p(c) = 0$.
    \end{proof}

    \begin{theorem}[Cayley-Hamilton]
        The characteristic polynomial of $T$ annihilates $T$.
    \end{theorem}
    \begin{corollary}
        The minimal polynomial of $T$ divides its characteristic polynomial.
    \end{corollary}
    \begin{corollary}
        The minimal polynomial of $T$ in a finite-dimensional vector space $V$ is
        at most $\dim{V}$.
    \end{corollary}

    
\end{document}
