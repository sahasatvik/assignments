\documentclass[11pt]{article}

\usepackage[T1]{fontenc}
\usepackage{geometry}
\usepackage{amsmath, amssymb, amsthm}
\usepackage[scr]{rsfso}
\usepackage{bm}
\usepackage[%
    hidealllines=true,%
    innerbottommargin=15,%
    nobreak=true,%
]{mdframed}
\usepackage{xcolor}
\usepackage{graphicx}
\usepackage{fancyhdr}
\usepackage{hyperref}

\geometry{a4paper, margin=1in, headheight=14pt}

\pagestyle{fancy}
\fancyhf{}
\renewcommand\headrulewidth{0.4pt}
\fancyhead[L]{\scshape MA3101: Analysis III}
\fancyhead[R]{\scshape \leftmark}
\rfoot{\footnotesize\it Updated on \today}
\cfoot{\thepage}

\newcommand{\C}{\mathbb{C}}
\newcommand{\R}{\mathbb{R}}
\newcommand{\Q}{\mathbb{Q}}
\newcommand{\Z}{\mathbb{Z}}
\newcommand{\N}{\mathbb{N}}

\newcommand{\ip}[2]{\langle #1, #2 \rangle}
\newcommand{\norm}[1]{\Vert #1 \Vert}
\renewcommand{\vec}[1]{\boldsymbol{#1}}

\newcommand{\vx}{\vec{x}}
\newcommand{\vy}{\vec{y}}
\newcommand{\vv}{\vec{v}}
\newcommand{\vw}{\vec{w}}
\newcommand{\ve}{\vec{e}}

\newmdtheoremenv[%
    backgroundcolor=blue!10!white,%
]{theorem}{Theorem}[section]
\newmdtheoremenv[%
    backgroundcolor=violet!10!white,%
]{corollary}{Corollary}[theorem]
\newmdtheoremenv[%
    backgroundcolor=teal!10!white,%
]{lemma}[theorem]{Lemma}

\theoremstyle{definition}
\newmdtheoremenv[%
    backgroundcolor=green!10!white,%
]{definition}{Definition}[section]
\newmdtheoremenv[%
    backgroundcolor=red!10!white,%
]{exercise}{Exercise}[section]

\theoremstyle{remark}
\newtheorem*{remark}{Remark}
\newtheorem*{example}{Example}
\newtheorem*{solution}{Solution}

\surroundwithmdframed[%
    linecolor=black!20!white,%
    hidealllines=false,%
    innertopmargin=5,%
    innerbottommargin=10,%
    skipabove=0,%
    skipbelow=0,%
]{example}

\numberwithin{equation}{section}

\title{
    \Large\textsc{MA3101} \\
    \Huge \textbf{Analysis III} \\
    \vspace{5pt}
    \Large{Autumn 2021}
}
\author{
    \large Satvik Saha
    \\\textsc{\small 19MS154}
}
\date{\normalsize
    \textit{Indian Institute of Science Education and Research, Kolkata, \\
    Mohanpur, West Bengal, 741246, India.} \\
}

\begin{document}
    \maketitle

    \tableofcontents

    \section{Euclidean spaces}

    \subsection{$\R^n$ as a vector space}
    
    We are familiar with the vector space $\R^n$, with the standard inner product \[
        \ip{\vx}{\vy} = x_1y_1 + \dots + x_ny_n.
    \] The standard norm is defined as \[
        \norm{\vx - \vy}^2 = \ip{\vx - \vy}{\vx - \vy} = \sum_{k = 1}^n (x_i -
        y_i)^2.
    \]

    \begin{exercise}
        What are all possible inner products on $\R^n$?
        \begin{solution}
            Note that an inner product is a bilinear, symmetric map such that
            $\ip{\vx}{\vx} \geq 0$, and $\ip{\vx}{\vx} = 0$ if and only if $\vx =
            \vec{0}$. Thus, an product map on $\R^n$ is completely and uniquely
            determined by the values $\ip{\ve_i}{\ve_j} = a_{ij}$. Let $A$ be the
            $n \times n$ matrix with entries $a_{ij}$. Note that $A$ is a real
            symmetric matrix with positive entries. Now, \[
                \ip{\vx}{\ve_j} = x_1a_{1j} + \dots + x_na_{nj} = \vx^\top \vec{a}_j,
            \] where $\vec{a}_j$ is the $j^\text{th}$ column of $A$. Thus, \[
                \ip{\vx}{\vy} = \vx^\top \vec{a}_1 y_1 + \dots + \vx^\top
                \vec{a}_n y_n = \vx^\top A \vy.
            \] Furthermore, any choice of real symmetric $A$ with positive
            entries produces an inner product.
        \end{solution}
    \end{exercise}

    \begin{theorem}[Cauchy-Schwarz]
        Given two vectors $\vv, \vw \in \R^n$, we have \[
            |\ip{\vv}{\vw}| \leq \norm{\vv}\norm{\vw}.
        \] 
    \end{theorem}
    \begin{proof}
        This is trivial when $\vw = \vec{0}$. When $\vw \neq \vec{0}$, set $\lambda =
        \ip{\vv}{\vw} / \norm{\vw}^2$. Thus, \[
            0 \leq \norm{\vv - \lambda \vw}^2 = \norm{\vv}^2 - 2\lambda \ip{\vv}{\vw}
            + \lambda^2\norm{\vw}^2.
        \] Simplifying, \[
            0 \leq \norm{\vv}^2 - \frac{|\ip{\vv}{\vw}|^2}{\norm{\vw}^2}.
        \] This gives the desired result. Clearly, equality holds if and only if $\vv
        = \lambda \vw$.
    \end{proof}
    
    \begin{theorem}[Triangle inequality]
        Given two vectors $\vv, \vw \in \R^n$, we have \[
            \norm{\vv + \vw} \leq \norm{\vv} + \norm{\vw}.
        \] 
    \end{theorem}
    \begin{proof}
        Write \[
            \norm{\vv + \vw}^2 = \norm{\vv}^2 + 2\ip{\vv}{\vw} + \norm{\vw}^2
            \leq \norm{\vv}^2 + 2|\ip{\vv}{\vw}| + \norm{\vw}^2.
        \] Applying Cauchy-Schwarz gives \[
            \norm{\vv + \vw}^2 \leq (\norm{\vv} + \norm{\vw})^2.
        \] 
        Equality holds if and only if $\vv = \lambda \vw$ for $\lambda \geq 0$.
    \end{proof}

    \subsection{$\R^n$ as a metric space}

    Our previous observations allow us to define the standard metric on $\R^n$, seen
    as a point set. \[
        d(\vx, \vy) = \norm{\vx - \vy}.
    \] 

    \begin{definition}
        For any $\delta > 0$, the set \[
            B_\delta(\vx) = \{\vy \in \R^n : d(\vx, \vy) < \delta\}
        \] is called the open ball centred at $\vx \in \R^n$ with radius $\delta$.
        This is also called the $\delta$ neighbourhood of $\vx$.
    \end{definition}

    \begin{definition}
        A set $U$ is open in $\R^n$ if for every $\vx \in U$, there exists an
        open ball $B_\delta(\vx) \subset U$.
        \begin{remark}
            Every open ball in $\R^n$ is open.
        \end{remark}
        \begin{remark}
            Both $\emptyset$ and $\R^n$ are open.
        \end{remark}
    \end{definition}
    
    \begin{definition}
        A set $F$ is closed in $\R^n$ if its complement $\R^n \setminus F$ is open in
        $\R^n$.
        \begin{remark}
            Both $\emptyset$ and $\R^n$ are closed.
        \end{remark}
        \begin{remark}
            Finite sets in $\R^n$ are closed.
        \end{remark}
    \end{definition}

    \begin{theorem}
        Unions and finite intersections of open sets are open.
    \end{theorem}
    \begin{corollary}
        Intersections and finite unions of closed sets are closed.
    \end{corollary}

    \begin{definition}
        An interior point $x$ of a set $S \subseteq \R^n$ is such that there is a
        neighbourhood of $x$ contained within $S$.
    \end{definition}
    \begin{definition}
        An exterior point $x$ of a set $S \subseteq \R^n$ is an interior point of the
        complement $\R^n \setminus S$.
    \end{definition}
    \begin{definition}
        A boundary point of a set is neither an interior point, nor an exterior point.
    \end{definition}

    \begin{definition}
        A limit point $x$ of a set $S \subseteq \R^n$ is such that every
        neighbourhood of $x$ contains a point from $S$.
    \end{definition}
    \begin{definition}
        The closure of a set $S \subseteq \R^n$ is the union of $S$ and its limit
        points.
        \begin{remark}
            The closure of a set is the smallest closed set containing it.
        \end{remark}
    \end{definition}

    \subsection{Continuous maps}
    \begin{definition}
        A map $f\colon X \to Y$ is continuous if the pre-image of every open set from
        $Y$ is open in $X$.
    \end{definition}
    \begin{lemma}
        A map $f\colon X \to Y$ is continuous if the pre-image of every closed set from
        $Y$ is closed in $X$.
    \end{lemma}

    \begin{theorem}
        The projection maps $\pi_i\colon \R^n \to \R$, $\vx \mapsto x_i$ are continuous.
    \end{theorem}
    \begin{proof}
        Let $U \subseteq \R$ be open; we claim that $\pi_i^{-1}(U)$ is open. Pick
        $\vx \in \pi_i^{-1}(U)$, and note that $\pi_i(\vx) = x_i \in U$. Thus, there
        exists $\delta > 0$ such that $(x_i - \delta, x_i + \delta) \subset U$. Now
        examine $B_\delta(\vx)$; for any point $\vy$ within this open ball, we have
        $d(\vx, \vy) < \delta$ hence \[
            |x_i - y_i|^2 \leq \sum_{k = 1}^n (x_k - y_k)^2 = d(\vx, \vy)^2 <
            \delta^2.
        \] In other words, $\pi_i(\vy) = y_i \in (x_i - \delta, x_i + \delta)$, hence
        $\pi_i \,B_\delta(\vx) \subseteq (x_i - \delta, x_i + \delta) \subset
        U$. Thus, given arbitrary $\vx \in \pi_i^{-1}(U)$, we have found an open ball
        $B_\delta(\vx) \subset \pi_i^{-1}(U)$.
    \end{proof}

    \begin{lemma}
        Finite sums, products, and compositions of continuous functions are continuous.
    \end{lemma}

    \begin{theorem}
        All polynomial functions of the coordinates in $\R^n$ are continuous.
    \end{theorem}
    \begin{example}
        The unit sphere $S^{n - 1} \subset \R^n$ is closed. It is by definition the
        pre-image of the singleton closed set $\{1\}$ under the continuous map \[
            \vx \mapsto x_1^2 + \dots + x_n^2.
        \] 
    \end{example}
    
\end{document}
