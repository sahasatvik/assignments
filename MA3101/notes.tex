\documentclass[11pt]{article}

\usepackage[T1]{fontenc}
\usepackage{geometry}
\usepackage{amsmath, amssymb, amsthm}
\usepackage[scr]{rsfso}
\usepackage{bm}
\usepackage[%
    hidealllines=true,%
    innerbottommargin=15,%
    nobreak=true,%
]{mdframed}
\usepackage{xcolor}
\usepackage{graphicx}
\usepackage{fancyhdr}
\usepackage{hyperref}

\geometry{a4paper, margin=1in, headheight=14pt}

\pagestyle{fancy}
\fancyhf{}
\renewcommand\headrulewidth{0.4pt}
\fancyhead[L]{\scshape MA3101: Analysis III}
\fancyhead[R]{\scshape \leftmark}
\rfoot{\footnotesize\it Updated on \today}
\cfoot{\thepage}

\newcommand{\C}{\mathbb{C}}
\newcommand{\R}{\mathbb{R}}
\newcommand{\Q}{\mathbb{Q}}
\newcommand{\Z}{\mathbb{Z}}
\newcommand{\N}{\mathbb{N}}

\newcommand{\ip}[2]{\langle #1, #2 \rangle}
\newcommand{\norm}[1]{\Vert #1 \Vert}
\renewcommand{\vec}[1]{\boldsymbol{#1}}

\newcommand{\vx}{\vec{x}}
\newcommand{\vy}{\vec{y}}
\newcommand{\vv}{\vec{v}}
\newcommand{\vw}{\vec{w}}
\newcommand{\ve}{\vec{e}}

\newmdtheoremenv[%
    backgroundcolor=blue!10!white,%
]{theorem}{Theorem}[section]
\newmdtheoremenv[%
    backgroundcolor=violet!10!white,%
]{corollary}{Corollary}[theorem]
\newmdtheoremenv[%
    backgroundcolor=teal!10!white,%
]{lemma}[theorem]{Lemma}

\theoremstyle{definition}
\newmdtheoremenv[%
    backgroundcolor=green!10!white,%
]{definition}{Definition}[section]
\newmdtheoremenv[%
    backgroundcolor=red!10!white,%
]{exercise}{Exercise}[section]

\theoremstyle{remark}
\newtheorem*{remark}{Remark}
\newtheorem*{example}{Example}
\newtheorem*{solution}{Solution}

\surroundwithmdframed[%
    linecolor=black!20!white,%
    hidealllines=false,%
    innertopmargin=5,%
    innerbottommargin=10,%
    skipabove=0,%
    skipbelow=0,%
]{example}

\numberwithin{equation}{section}

\title{
    \Large\textsc{MA3101} \\
    \Huge \textbf{Analysis III} \\
    \vspace{5pt}
    \Large{Autumn 2021}
}
\author{
    \large Satvik Saha
    \\\textsc{\small 19MS154}
}
\date{\normalsize
    \textit{Indian Institute of Science Education and Research, Kolkata, \\
    Mohanpur, West Bengal, 741246, India.} \\
}

\begin{document}
    \maketitle

    \tableofcontents

    \section{Euclidean spaces}

    \subsection{$\R^n$ as a vector space}
    
    We are familiar with the vector space $\R^n$, with the standard inner product \[
        \ip{\vx}{\vy} = x_1y_1 + \dots + x_ny_n.
    \] The standard norm is defined as \[
        \norm{\vx - \vy}^2 = \ip{\vx - \vy}{\vx - \vy} = \sum_{k = 1}^n (x_i -
        y_i)^2.
    \]

    \begin{exercise}
        What are all possible inner products on $\R^n$?
        \begin{solution}
            Note that an inner product is a bilinear, symmetric map such that
            $\ip{\vx}{\vx} \geq 0$, and $\ip{\vx}{\vx} = 0$ if and only if $\vx =
            \vec{0}$. Thus, an product map on $\R^n$ is completely and uniquely
            determined by the values $\ip{\ve_i}{\ve_j} = a_{ij}$. Let $A$ be the
            $n \times n$ matrix with entries $a_{ij}$. Note that $A$ is a real
            symmetric matrix with positive entries. Now, \[
                \ip{\vx}{\ve_j} = x_1a_{1j} + \dots + x_na_{nj} = \vx^\top \vec{a}_j,
            \] where $\vec{a}_j$ is the $j^\text{th}$ column of $A$. Thus, \[
                \ip{\vx}{\vy} = \vx^\top \vec{a}_1 y_1 + \dots + \vx^\top
                \vec{a}_n y_n = \vx^\top A \vy.
            \] Furthermore, any choice of real symmetric $A$ with positive
            entries produces an inner product.
        \end{solution}
    \end{exercise}

    \begin{theorem}[Cauchy-Schwarz]
        Given two vectors $\vv, \vw \in \R^n$, we have \[
            |\ip{\vv}{\vw}| \leq \norm{\vv}\norm{\vw}.
        \] 
    \end{theorem}
    \begin{proof}
        This is trivial when $\vw = \vec{0}$. When $\vw \neq \vec{0}$, set $\lambda =
        \ip{\vv}{\vw} / \norm{\vw}^2$. Thus, \[
            0 \leq \norm{\vv - \lambda \vw}^2 = \norm{\vv}^2 - 2\lambda \ip{\vv}{\vw}
            + \lambda^2\norm{\vw}^2.
        \] Simplifying, \[
            0 \leq \norm{\vv}^2 - \frac{|\ip{\vv}{\vw}|^2}{\norm{\vw}^2}.
        \] This gives the desired result. Clearly, equality holds if and only if $\vv
        = \lambda \vw$.
    \end{proof}
    
    \begin{theorem}[Triangle inequality]
        Given two vectors $\vv, \vw \in \R^n$, we have \[
            \norm{\vv + \vw} \leq \norm{\vv} + \norm{\vw}.
        \] 
    \end{theorem}
    \begin{proof}
        Write \[
            \norm{\vv + \vw}^2 = \norm{\vv}^2 + 2\ip{\vv}{\vw} + \norm{\vw}^2
            \leq \norm{\vv}^2 + 2|\ip{\vv}{\vw}| + \norm{\vw}^2.
        \] Applying Cauchy-Schwarz gives \[
            \norm{\vv + \vw}^2 \leq (\norm{\vv} + \norm{\vw})^2.
        \] 
        Equality holds if and only if $\vv = \lambda \vw$ for $\lambda \geq 0$.
    \end{proof}

    \subsection{$\R^n$ as a metric space}

    Our previous observations allow us to define the standard metric on $\R^n$, seen
    as a point set. \[
        d(\vx, \vy) = \norm{\vx - \vy}.
    \] 

    \begin{definition}
        For any $\delta > 0$, the set \[
            B_\delta(\vx) = \{\vy \in \R^n : d(\vx, \vy) < \delta\}
        \] is called the open ball centred at $\vx \in \R^n$ with radius $\delta$.
        This is also called the $\delta$ neighbourhood of $\vx$.
    \end{definition}

    \begin{definition}
        A set $U$ is open in $\R^n$ if for every $\vx \in U$, there exists an
        open ball $B_\delta(\vx) \subset U$.
        \begin{remark}
            Every open ball in $\R^n$ is open.
        \end{remark}
        \begin{remark}
            Both $\emptyset$ and $\R^n$ are open.
        \end{remark}
    \end{definition}
    
    \begin{definition}
        A set $F$ is closed in $\R^n$ if its complement $\R^n \setminus F$ is open in
        $\R^n$.
        \begin{remark}
            Both $\emptyset$ and $\R^n$ are closed.
        \end{remark}
        \begin{remark}
            Finite sets in $\R^n$ are closed.
        \end{remark}
    \end{definition}

    \begin{theorem}
        Unions and finite intersections of open sets are open.
    \end{theorem}
    \begin{corollary}
        Intersections and finite unions of closed sets are closed.
    \end{corollary}

    \begin{definition}
        An interior point $x$ of a set $S \subseteq \R^n$ is such that there is a
        neighbourhood of $x$ contained within $S$.
    \end{definition}
    \begin{example}
        Every point in an open set is an interior point by definition. The interior
        of a set is the largest open set contained within it.
    \end{example}

    \begin{definition}
        An exterior point $x$ of a set $S \subseteq \R^n$ is an interior point of the
        complement $\R^n \setminus S$.
    \end{definition}

    \begin{definition}
        A boundary point of a set is neither an interior point, nor an exterior point.
    \end{definition}
    \begin{example}
        The boundary of the unit open ball $B_1(0) \subset \R^n$ is the sphere $S^{n
        - 1}$.
    \end{example}

    \begin{definition}
        A limit point $x$ of a set $S \subseteq \R^n$ is such that every
        neighbourhood of $x$ contains a point from $S$ other than itself.
    \end{definition}
    \begin{definition}
        The closure of a set $S \subseteq \R^n$ is the union of $S$ and its limit
        points.
        \begin{remark}
            The closure of a set is the smallest closed set containing it.
        \end{remark}
    \end{definition}

    \begin{lemma}
        Every open set in $\R^n$ is a union of open balls.
    \end{lemma}
    \begin{proof}
        Let $U\subseteq \R^n$ be open. Thus, for every $\vx\in\R^n$, we can choose
        $\delta_x > 0$ such that $B_{\delta_x}(\vx) \subset U$. The union of all such
        open balls is precisely the set $U$.
    \end{proof}

    \subsection{$\R^n$ as a topological space}
    \begin{definition}
        A topology on a set $X$ is a collection $\tau$ of subsets of $X$ such that
        \begin{enumerate}
            \itemsep0em
            \item $\emptyset \in \tau$
            \item $X \in \tau$
            \item Arbitrary union of sets from $\tau$ belong to $\tau$.
            \item Finite intersections of sets from $\tau$ belong to $\tau$.
        \end{enumerate}
        Sets from $\tau$ are called open sets.
    \end{definition}
    \begin{example}
        The Euclidean metric induces the standard topology on $\R^n$.
    \end{example}
    \begin{example}
        The discrete topology on a set $X$ is one where every singleton set is open.
        This is the topology induced by the discrete metric, \[
            d_\text{discrete}\colon X\times X \to \R, \qquad
            (x, y) \mapsto \begin{cases}
                0, &\text{ if }x = y, \\
                1, &\text{ if }x \neq y.
            \end{cases}
        \] 
    \end{example}
    \begin{example}
        Let $X$ be an infinite set. The collection of sets consisting of $\emptyset$
        along with all sets $A$ such that $X\setminus A$ is finite is a topology on
        $X$. This is called the Zariski topology.
    \end{example}
    \begin{example}
        Consider the set of real numbers, and let $\tau$ be the collection
        $\emptyset$, $\R$, and all intervals $(-x, +x)$ for $x > 0$. This constitutes
        a topology on $\R$, very different from the usual one.

        This topology cannot be induced by a metric; it is not metrizable.

        Consider the constant sequence of zeros. In this topology $(\R, \tau)$, this
        sequence converges to \emph{every} point in $\R$. Given any $\ell \in \R$, the
        open neighbourhoods of $\ell$ are precisely the sets $\R$ and the open intervals
        $(-x, +x)$ for $x > |\ell|$. The tail of the constant sequence of zeros is
        contained within every such neighbourhood of $\ell$, hence $0 \to \ell$.
        Indeed, the element zero belongs to every open set apart from $\emptyset$ in
        this topology.
    \end{example}
    \begin{definition}
        A topological space is called Hausdorff if for every distinct $x, y \in X$,
        there exist disjoint neighbourhoods of $x$ and $y$.
    \end{definition}
    \begin{example}
        Every metric space is Hausdorff.
        Given distinct $x, y$ in a metric space $(X, d)$, set $\delta = d(x, y) / 3$
        and consider the open balls $B_{\delta}(x)$ and $B_\delta(y)$.
    \end{example}
    
    \begin{lemma}
        Every convergent sequence in a Hausdorff space has exactly one limit.
    \end{lemma}
    \begin{proof}
        Consider a sequence $\{x_n\}_{n \in \N}$, and suppose that it converges to
        distinct $x_1$ and $x_2$. Construct disjoint neighbourhoods $U_1$ and $U_2$
        around $x_1$ and $x_2$. Now, convergence implies that both $U_1$ and $U_2$
        contain the tail of $\{x_n\}$, which is impossible since they are disjoint
        and hence contain no elements in common.
    \end{proof}

    \begin{definition}
        Given a topological space $(X, \tau)$ and a subset $Y\subseteq X$, the
        collection of sets $U \cap Y$ where $U \in \tau$ is a topology $\tau_Y$ on
        $Y$. We call this collection the subspace topology on $Y$, induced by the
        topology on $X$.
    \end{definition}

    \subsection{Compact sets in $\R^n$}
    
    \begin{definition}
        A set $K \subset X$ in a topological space is compact if every open cover of
        $K$ has a finite sub-cover. That is, for every collection if
        $\{U_\alpha\}_{\alpha \in A}$ of open sets such that $K$ is contained in
        their union, there exists a finite sub-collection $U_{\alpha_1}, \dots,
        U_{\alpha_k}$ such that $K$ is also contained in their union.
    \end{definition}
    \begin{example}
        All finite sets are compact.
    \end{example}
    \begin{example}
        Given a convergent sequence of real numbers $x_n \to x$, the collection
        $\{x_n\}_{n \in \N} \cup \{x\}$ is compact.
    \end{example}
    \begin{example}
        In $\R^n$, compact sets are precisely those sets which are closed and
        bounded. This is the Heine-Borel Theorem.
    \end{example}

    \begin{theorem}
        The closed intervals $[a, b] \subset \R$ are compact.
        \begin{remark}
            This can be extended to show that any $k$-cell $[a_1, b_1]\times \dots
            \times [a_n, b_n] \subset \R^n$ is compact.
        \end{remark}
    \end{theorem}
    \begin{proof}
        Let $\{U_\alpha\}_{\alpha \in A}$ be an open cover of $[a, b]$, and suppose
        that $I_1 = [a, b]$ has no finite sub-cover. Then, at least one of the
        intervals $[a, (a + b) / 2]$ and $[(a + b)/ 2, b]$ must not have a finite
        sub-cover; pick one and call it $I_2$. Similarly, one of the halves of $I_2$
        must not have a finite sub-cover; call it $I_3$. In this process, we generate
        a sequence of closed intervals $I_1 \supset I_2 \supset \dots$, none of which
        have a finite sub-cover. The length of each interval is given by \[
            |I_n| = 2^{-n + 1}\norm{b - a} \to 0.
        \] Now, pick a sequence of points $\{x_n\}$ where each $x_n \in I_n$. Then,
        $\{x_n\}$ is a Cauchy sequence. To see this, given any $\epsilon > 0$, we can
        find sufficiently large $n_0$ such that $2^{-n_0 + 1}\norm{b - a} <
        \epsilon$. Thus, $x_n \in I_n \subset I_{n_0}$ for all $n \geq n_0$, which
        means that for any $m, n \geq n_0$, we have $x_m, x_n \in I_{n_0}$ forcing\footnote{
            If $x_1, x_2 \in [a, b]$ with $x_1 < x_2$, note that $a \leq x_1 < x_2
            \leq b$, so\[
                |x_2 - x_1| = x_2 - x_1 \leq b - a.
            \] 
        } \[
            \norm{x_m - x_n} \leq |I_{n_0}| = 2^{-n_0 + 1}\norm{b - a} < \epsilon.
        \] From the completeness of $\R$, this sequence must converge in $\R$,
        specifically in $[a, b]$. Thus, $x_n \to x$ for some $x \in [a, b]$. It can
        also be seen that  the limit $x \in I_n$ for all $n \in \N$; if not, say $x
        \notin I_{n_0}$, then $x \in [a, b] \setminus I_{n_0}$ which is open, hence
        there is an open interval such that $(x - \delta, x + \delta) \cap I_{n_0} =
        \emptyset$.  However, $I_{n_0}$ contains all $x_{n \geq n_0}$, thus this
        $\delta$-neighbourhood of $x$ would miss out a tail of $\{x_n\}$.

        Now, pick the open set $U \in \{U_\alpha\}$ which covers the point $x$. Thus,
        $x \in U$ so $U$ contains some non-empty open interval $(x - \delta, x +
        \delta)$ around $x$. Choose $n_0$ such that $2^{-n_0 + 1} \norm{b - a} <
        \delta$; this immediately gives $I_{n_0} \subseteq (x - \delta, x + \delta)
        \subset U$.  This contradicts that fact that $I_{n_0}$ has no finite
        sub-cover from $\{U_\alpha\}$, completing the proof.
    \end{proof}
    \begin{remark}
        The fact that Cauchy sequences in $\R^n$ converge isn't immediately obvious;
        it is a consequence of the completeness of $\R^n$. Start by noting that $\R$
        has the Least Upper Bound property, from which the Monotone Convergence
        Theorem follows; every monotonic, bounded sequence of reals converges. It
        can also be shown that any sequence of reals with contain a monotone
        subsequence, from which it follows that every bounded sequence contains a
        convergent subsequence (Bolzano-Weierstrass). Finally, it can be shown that
        if a subsequence of a Cauchy sequence converges, then the entire sequence
        also converges to the same limit, giving us the desired result for $\R$. For
        sequence in $\R^n$, we may apply this coordinate-wise to obtain the result.
    \end{remark}

    \begin{lemma}
        Compact sets in $\R^n$ are closed and bounded.
    \end{lemma}
    \begin{proof}
        Consider a compact set $K \subset \R^n$. Let $x \in \R^n \setminus K$, and
        let $y \in K$. Since $x \neq y$, we choose open balls $U_y$ around $y$ and
        $V_y$ around $x$ such that $U_y\cap V_y = \emptyset$. Repeating this for all
        $y \in K$, we generate an open cover $\{U_y\}$ of $K$ consisting of open balls. The
        compactness of $K$ guarantees that this has a finite sub-cover, i.e.\ there
        is a finite set $Y$ such that the collection $\{U_y\}_{y \in Y}$ covers $X$.
        As a result, the finite intersection of all $V_y$ for $y \in Y$ is contained
        within $\R^n \setminus K$. Thus, $x$ is in the exterior of $K$. Since $x$ was
        chosen arbitrarily from $\R^n\setminus K$, we see that $K$ is closed.

        Now, consider the open cover $\{B_1(x)\}_{x \in K}$, and extract a finite
        sub-cover of unit open balls. The distance between any two points in $K$ is
        at most the maximum distance between the centres of any two balls in our
        sub-cover, plus two.
    \end{proof}

    \begin{lemma}
        The intersection of a closed set and a compact set is compact.
    \end{lemma}
    \begin{proof}
        Let $F \subseteq \R^n$ be closed and let $K \subseteq \R^n$ be compact.
        Suppose that the open cover $\{U_\alpha\}$ of $F \cap K$ has no finite
        sub-cover. Now the complement $U = F^c$ is open in $\R^n$, hence the
        collection $\{U_\alpha\} \cup \{U\}$ is an open cover of $K$, and hence must
        admit a finite sub-cover of $K$. In particular, this must be a finite
        sub-cover of $F \cap K$. However, we can remove the set $U$ from this
        sub-cover since it shares no element with $F \cap K$; as a result, our
        sub-cover must be a finite sub-collection of sets $U_\alpha$, contradicting
        our assumption. This shows that $F \cap K$ is compact.
    \end{proof}

    \begin{lemma}[Finite intersection property]
        Let $\{K_\alpha\}$ be a collection of compact sets in $\R^n$ which have the
        property that any finite intersection of them is non-empty. Then, \[
            \bigcap_{\alpha} K_\alpha \neq \emptyset.
        \] 
    \end{lemma}
    \begin{proof}
        Suppose to the contrary that the intersection of all $K_\alpha$ is empty. Fix
        an index $\beta$, and note that no element of $K_\beta$ lies in every
        $K_\alpha$. Set $J_\alpha = K_\alpha^c$, whence the collection $\{J_\alpha :
        \alpha \neq \beta\}$ is an open cover of $K_\beta$. This must admit a finite
        sub-cover $\{J_{\alpha_1},\dots, J_{\alpha_k}\}$ of $K_\beta$. Thus, we must
        have \[
            K_\beta^c \cup J_{\alpha_1} \cup \dots \cup J_{\alpha_k} = \R^n.
        \] This immediately gives the contradiction \[
            K_\beta \cap K_{\alpha_1} \cap \dots \cap K_{\alpha_k} = \emptyset.
            \qedhere
        \] 
    \end{proof}

    \begin{theorem}[Heine-Borel]
        Compact sets in $\R^n$ are precisely those that are closed and bounded.
    \end{theorem}
    \begin{proof}
        Given a compact set in $\R^n$, we have already shown that it must be closed
        and bounded. Next, if $F \subset \R^n$ is closed and bounded, it can be
        enclosed within a $k$-cell which we know is compact. Thus, $F$ is the
        intersection of the closed set $F$ and the compact $k$-cell, proving that
        $F$ must be compact.
    \end{proof}

    \subsection{Continuous maps}
    \begin{definition}
        A map $f\colon X \to Y$ is continuous if the pre-image of every open set from
        $Y$ is open in $X$.
    \end{definition}
    \begin{lemma}
        A map $f\colon X \to Y$ is continuous if the pre-image of every closed set from
        $Y$ is closed in $X$.
    \end{lemma}

    \begin{theorem}
        The projection maps $\pi_i\colon \R^n \to \R$, $\vx \mapsto x_i$ are continuous.
    \end{theorem}
    \begin{proof}
        Let $U \subseteq \R$ be open; we claim that $\pi_i^{-1}(U)$ is open. Pick
        $\vx \in \pi_i^{-1}(U)$, and note that $\pi_i(\vx) = x_i \in U$. Thus, there
        exists $\delta > 0$ such that $(x_i - \delta, x_i + \delta) \subset U$. Now
        examine $B_\delta(\vx)$; for any point $\vy$ within this open ball, we have
        $d(\vx, \vy) < \delta$ hence \[
            |x_i - y_i|^2 \leq \sum_{k = 1}^n (x_k - y_k)^2 = d(\vx, \vy)^2 <
            \delta^2.
        \] In other words, $\pi_i(\vy) = y_i \in (x_i - \delta, x_i + \delta)$, hence
        $\pi_i \,B_\delta(\vx) \subseteq (x_i - \delta, x_i + \delta) \subset
        U$. Thus, given arbitrary $\vx \in \pi_i^{-1}(U)$, we have found an open ball
        $B_\delta(\vx) \subset \pi_i^{-1}(U)$.
    \end{proof}

    \begin{lemma}
        Finite sums, products, and compositions of continuous functions are continuous.
    \end{lemma}

    \begin{theorem}
        All polynomial functions of the coordinates in $\R^n$ are continuous.
    \end{theorem}
    \begin{example}
        The unit sphere $S^{n - 1} \subset \R^n$ is closed. It is by definition the
        pre-image of the singleton closed set $\{1\}$ under the continuous map \[
            \vx \mapsto x_1^2 + \dots + x_n^2.
        \] 
    \end{example}
    
    \begin{theorem}
        The continuous image of a compact set is compact.
    \end{theorem}
    \begin{proof}
        Let $f\colon X \to Y$ be continuous, where $Y$ is the image of the compact
        set $X$, and let $\{U_\alpha\}$ be an open cover of $Y$. Then, the collection
        $\{f^{-1}(U_\alpha)\}$ is an open cover of $X$. Using the compactness of $X$,
        extract a finite sub-cover $f^{-1}(U_{\alpha_1}), \dots,
        f^{-1}(U_{\alpha_k})$ of $X$. It follows that the collection $U_{\alpha_1},
        \dots, U_{\alpha_k}$ is a finite sub-cover of $Y$.
    \end{proof}

    \subsection{Connectedness}
    \begin{definition}
        Let $X$ be a topological space. A separation of $X$ is a pair $U, V$ of
        non-empty disjoint open subsets such that $X = U \cup V$.
    \end{definition}

    \begin{definition}
        A connected topological space is one which cannot be separated.
    \end{definition}
    \begin{lemma}
        A topological space $X$ is connected if and only if the only sets which are
        both open and closed are $\emptyset$ and $X$.
    \end{lemma}

    \begin{example}
        The intervals $(a, b) \subset \R$ are connected. To see this, suppose that
        $U$, $V$ is a separation of $(a, b)$. Pick $x \in U$, $y \in V$, and without
        loss of generality let $x < y$. Define $S = [x, y] \cap U$, and set $c =
        \sup{S}$. It can be argued that $c \in (a, b)$, but $c \notin U$, $c \notin
        V$, using the properties of the supremum.
    \end{example}

    \begin{theorem}
        The continuous image of a connected set is connected.
    \end{theorem}
    \begin{proof}
        Let $f$ be a continuous map on the connected set $X$, and let $Y$ be the
        image of $X$. If $U$, $V$ is a separation of $Y$, then it can be shown that
        $f^{-1}(U)$, $f^{-1}(V)$ constitutes a separation of $X$, which is a
        contradiction.
    \end{proof}

    \begin{definition}
        A path $\gamma$ joining two points $x, y \in X$ is a continuous map
        $\gamma\colon [a, b] \to X$ such that $\gamma(a) = x$, $\gamma(b) = y$.
    \end{definition}

    \begin{definition}
        A set in $X$ is path connected if given any two distinct points in $X$, there
        exists a path joining them.
    \end{definition}

    \begin{lemma}
        Every path connected set is connected.
    \end{lemma}
    \begin{proof}
        Let $X$ be path connected, and suppose that $U$, $V$ is a separation of $X$.
        Then, pick $x \in U$, $y \in V$, and choose a path $\gamma\colon [0, 1] \to
        X$ between $x$ and $y$. The sets $f^{-1}(U)$ and $f^{-1}(V)$ separate the
        interval $[0, 1]$, which is a contradiction.
    \end{proof}
    \begin{example}
        All connected sets are not path connected. Consider the topologist's sine
        curve, \[
            \left\{\left(x, \sin{\frac{1}{x}}\right): 0 < x \leq 1\right\} \cup \{(0,
            0)\}.
        \] 
    \end{example}


    \begin{definition}
        The $\epsilon$ neighbourhood of a set $K$ in a metric space $X$ is defined as \[
             \bigcup_{a \in K} B_\epsilon(a) = \bigcup_{a \in K} \{x \in X: d(x, a) <
             \epsilon\}.
        \] 
    \end{definition}

    \begin{exercise}
        Let $K \subseteq \R^n$ be compact, and define $f\colon \R^n \to \R$, \[
            f(x) = \operatorname{dist}(x, K) = \inf_{a \in K} d(x, a).
        \] Show that $f$ is continuous on $\R^n$, and $f^{-1}(\{0\}) = K$.
    \end{exercise}

    \begin{exercise}
        If $K \subseteq \R^n$ is compact and $K \cap L = \emptyset$, then \[
            \operatorname{dist}(K, L) = \inf_{a \in K} \operatorname{dist}(a, L) > 0.
        \] 
    \end{exercise}

    \begin{exercise}
        If $K \subseteq \R^n$ is compact and $U$ is an open set containing $K$, then
        there exists $\epsilon > 0$ such that $U$ contains the $\epsilon$
        neighbourhood of $K$.

        Is the compactness of $K$ necessary?
    \end{exercise}
    

    
\end{document}
