\documentclass[11pt]{article}

\usepackage[T1]{fontenc}
\usepackage{geometry}
\usepackage{amsmath, amssymb, amsthm}
\usepackage[scr]{rsfso}
\usepackage{bm}
\usepackage[%
    hidealllines=true,%
    innerbottommargin=15,%
    nobreak=true,%
]{mdframed}
\usepackage{xcolor}
\usepackage{graphicx}
\usepackage{fancyhdr}
\usepackage{hyperref}

\geometry{a4paper, margin=1in, headheight=14pt}

\pagestyle{fancy}
\fancyhf{}
\renewcommand\headrulewidth{0.4pt}
\fancyhead[L]{\scshape MA3206: Statistics I}
\fancyhead[R]{\scshape \leftmark}
\rfoot{\footnotesize\it Updated on \today}
\cfoot{\thepage}

\newcommand{\C}{\mathbb{C}}
\newcommand{\R}{\mathbb{R}}
\newcommand{\Q}{\mathbb{Q}}
\newcommand{\Z}{\mathbb{Z}}
\newcommand{\N}{\mathbb{N}}

\DeclareMathOperator{\var}{Var}
\DeclareMathOperator{\cov}{Cov}

\newmdtheoremenv[%
    backgroundcolor=blue!10!white,%
]{theorem}{Theorem}[section]
\newmdtheoremenv[%
    backgroundcolor=violet!10!white,%
]{corollary}{Corollary}[theorem]
\newmdtheoremenv[%
    backgroundcolor=teal!10!white,%
]{lemma}[theorem]{Lemma}

\theoremstyle{definition}
\newmdtheoremenv[%
    backgroundcolor=green!10!white,%
]{definition}{Definition}[section]
\newmdtheoremenv[%
    backgroundcolor=red!10!white,%
]{exercise}{Exercise}[section]

\theoremstyle{remark}
\newtheorem*{remark}{Remark}
\newtheorem*{example}{Example}
\newtheorem*{solution}{Solution}

\surroundwithmdframed[%
    linecolor=black!20!white,%
    hidealllines=false,%
    innertopmargin=5,%
    innerbottommargin=10,%
    skipabove=0,%
    skipbelow=0,%
]{example}

\numberwithin{equation}{section}

\title{
    \Large\textsc{MA3206} \\
    \Huge \textbf{Statistics I} \\
    \vspace{5pt}
    \Large{Spring 2022}
}
\author{
    \large Satvik Saha
    \\\textsc{\small 19MS154}
}
\date{\normalsize
    \textit{Indian Institute of Science Education and Research, Kolkata, \\
    Mohanpur, West Bengal, 741246, India.} \\
}

\begin{document}
    \maketitle

    \tableofcontents

    \section{Introduction}

    We are interested in two types of data: \emph{categorical} and \emph{numerical}.
    Categorical data used named qualities to describe a particular observation. This
    can be further categorized into \emph{nominal} and \emph{ordinal}; the latter
    admit a natural ordering. Numerical data uses numbers, and can be further
    categorized into \emph{discrete} and \emph{continuous}.
    
    \section{Measures of central tendency}

    \subsection{Arithmetic mean}
    
    Suppose that we have been given a collection of $n$ numeric observations, denoted
    $x_1, x_2, \dots, x_n$. These may be concentrated around some specific point, or
    spread out over some range; regardless, we wish to identify one particular point
    around which our observations are `balanced' or aggregate in some sense. In other
    words, we want to identify a point $\bar{x}$ such that the net deviation $|x_i -
    \bar{x}|$ is minimized. For convenience, we consider the square deviations $(x_i
    - \bar{x})^2$; thus, we wish to minimize the loss function defined by \[
        t \mapsto \sum_{i = 1}^n (x_i - t)^2.
    \] It is easy to check that our loss function attains its minimum at \[
        \bar{x} = \frac{1}{n}\sum_{i = 1}^n x_i.
    \] This quantity $\bar{x}$ is called the \emph{arithmetic mean} of our data.
    Note that this is not the only choice of loss function measuring central
    tendency, but it is certainly quite convenient.

    If our data is summarized in terms of frequencies, i.e.\ each $x_i$ has been
    recorded $f_i$ times, we may write \[
        \bar{x} = \frac{1}{N}\sum_{i = 1}^n f_ix_i, \qquad N = \sum_{i = 1}^n f_i.
    \] The quantities $f_i / N$ are often referred to as the \emph{weights} of the
    observations $x_i$. The arithmetic mean can thus be interpreted as their `centre
    of mass'. \\

    Now suppose that our data values have not been explicitly presented: instead, we
    have been given the data classes $(x_{i - 1}, x_{i}]$ and the number of
    observations $f_i$ falling within each class. We can make an estimate of the true
    mean by identifying each data class with some value, say $(x_{i - 1}, x_{i}]$ gets
    associated with $x_i^* = (x_{i - 1} + x_{i}) / 2$. Then we calculate the usual
    arithmetic mean using these values. This gives us the estimate \[
        \bar{x}^* = \frac{1}{N}\sum_{i = 1}^n f_i x_i^*, \qquad 
        N = \sum_{i = 1}^n f_i.
    \] Note that the true mean must lie within the bounds \[
        \frac{1}{N}\sum_{i = 1}^{n} f_i x_{i - 1} \;\leq \bar{x} \leq\;
        \frac{1}{N}\sum_{i = 1}^{n} f_i x_{i}.
    \] Suppose that each data class has width $h$. We may estimate the error in our
    mean by observing that within a particular class $(x_{i - 1}, x_i]$ with
    frequency $f_i$, the deviation between any of the true data points and $x_i^*$ is
    at most $h / 2$. Thus, the net deviation accumulated over a particular class is
    at most $f_ih / 2$, and the net deviation overall is at most $Nh / 2$. Putting
    everything together, we have \[
        |\bar{x} - \bar{x}^*| \leq \frac{h}{2}.
    \]

    
    \subsection{Geometric mean}

    Another measure of central tendency is the geometric mean $G$, calculated \[
        G = \sqrt[n]{x_1x_2\cdots x_n}.
    \] Note that \[
        \log{G} = \frac{1}{n}\sum_{i = 1}^n \log{x_i}.
    \] Consider $k$ sets of observations, with $n_i$ observations in each set. Then,
    the geometric mean of the combined data is related with the geometric means $G_I$
    of the sets as \[
        \log{G} = \frac{1}{N}\sum_{i = 1}^k n_i\log{G_i}, \qquad N = \sum_{i = 1}^k
        n_i.
    \]
    
    \subsection{Harmonic mean}

    Another measure of central tendency is the harmonic mean $G$, calculated \[
        \frac{1}{H} = \frac{1}{n}\sum_{i = 1}^n \frac{1}{x_i}.
    \] The Harmonic means of combined data and sets of data are related as \[
        \frac{N}{H} = \sum_{i = 1}^k \frac{n_i}{H_i}, \qquad N = \sum_{i = 1}^k n_i.
    \] 

    \begin{exercise}
        Given two positive numbers, their arithmetic, geometric, and harmonic means
        all lie between them.
    \end{exercise}
    \begin{proof}
        Without loss of generality, let $x \geq y > 0$. Then for any $a, b$, we have
        \[
            x = \frac{ax + bx}{a + b} \geq \frac{ax + by}{a + b} \geq \frac{ay +
            by}{a + b} = y.
        \] Setting $a = b = 1 / 2$ give the result for the arithmetic mean. Now, the
        logarithm function is monotonic for positive reals, so $\log{x} \geq
        \log{y}$. Applying the above gives \[
            \log{x} \geq \frac{1}{2}(\log{x} + \log{y}) \geq \log{y},
        \] and taking exponentials yields \[
            x \geq \sqrt{xy} \geq y.
        \] Finally, applying the result to $1 / y \geq 1 / x$, we have \[
            \frac{1}{y} \geq \frac{a / y + b / x}{a + b} \geq \frac{1}{x},
        \] which we can rearrange and set $a = b = 1 / 2$ to get \[
            x \geq \frac{2}{1 / x + 1 / y} \geq y. \qedhere
        \] 
    \end{proof}
    \begin{remark}
        The same proof applies for weighted means.
    \end{remark}

    \begin{theorem}
        For $n$ observations $x_1, \dots, x_n$, the arithmetic mean, geometric mean,
        and harmonic mean are in descending order, i.e.\ \[
            AM \geq GM \geq HM.
        \] 
    \end{theorem}
    \begin{proof}
        We assume that all $x_i > 0$. Consider the case $n = 2$. Then, \[
            (\sqrt{x_1} - \sqrt{x_2})^2 \geq 0, \qquad x_1 + x_2 \geq 2\sqrt{x_1 x_2}
        \] is precisely $AM \geq GM$. Applying the same on the reciprocals, \[
            \frac{1}{x_1} + \frac{1}{x_2} \geq 2\sqrt{\frac{1}{x_1 x_2}}, \qquad
            \sqrt{x_1 x_2} \geq \frac{2}{1 / x_1 + 1 / x_2}
        \] is precisely $GM \geq HM$. \\

        Suppose that the result holds for some $n$. Now consider a collection of $2n$
        observations $x_1, \dots, x_{2n}$. Then, applying $AM \geq GM$ on both
        halves, then the two variable case gives \[
            \sum_{i = 1}^{2n} x_i \geq n\sqrt[n]{x_1\cdot x_n} + n\sqrt[n]{x_{n + 1}\cdots
            x_{2n}} \geq 2n\sqrt[2n]{x_1\cdots x_n x_{n + 1}\cdots x_{2n}}
        \] which is precisely $AM \geq GM$ for $2n$ observations.  Now suppose that
        $AM \geq GM$ holds for some $n + 1$. Consider a collection of $n$
        observations $x_1, \dots, x_n$, set $\bar{x} = (x_1 + \dots + x_n) / n$, and
        note that \[
            \sum_{i = 1}^n x_i + \bar{x} \geq (n + 1)\sqrt[n + 1]{x_1\cdots x_n
            \bar{x}}.
        \] The left-hand side is simply $(n + 1)\bar{x}$, so \[
            \bar{x} \geq \sqrt[n + 1]{x_1\cdots x_n \bar{x}}, \qquad \bar{x}^{n / n +
            1} \geq (x_1\cdots x_n)^{1 / n + 1}, \qquad \bar{x} \geq
            \sqrt[n]{x_1\cdots x_n},
        \] which is precisely $AM \geq GM$ for $n$ observations. Therefore, $AM \geq
        GM$ holds for all $n \geq 2$ by induction.

        Now that we have $AM \geq GM$ for $n$ observations, use it on their
        reciprocals to get \[
            \sum_{i = 1}^n \frac{1}{x_i} \geq n\sqrt[n]{\frac{1}{x_1\cdots x_n}}, \qquad 
            \sqrt[n]{x_1\cdots x_n} \geq \frac{n}{\sum_{i = 1}^n 1 / x_i}
        \] which is precisely $GM \geq HM$.
    \end{proof}


    \subsection{Median}
    
    The median of a collection of ordered observations $x_1 \leq x_2 \leq \dots \leq
    x_n$ is defined to be their middle value: $x_{k + 1}$ if $n = 2k + 1$ is odd, and
    the mean $(x_{k} + x_{k + 1}) / 2$ if $n = 2k$ is even.

    For grouped data, we assume that the observations are evenly distributed over
    the median class $(l, u]$ with frequency $f_m$, width $h$. If the total frequency
    is denoted by $N$, we write \[
        \frac{M - l}{h} = \frac{N / 2 - n_l}{f_m}.
    \] Here, $n_l$ is the cumulative frequency of the preceding classes. This will
    give \[
        M = l + \frac{N / 2 - n_l}{f_m}\cdot h.
    \] Another way of estimating the median of grouped data is by drawing the more than
    and less than ogives, and picking the abscissa of their intersection point. In
    the median class, the ogives have the equations \[
        y = n_l + \frac{f_m}{h}(x - l), \qquad y = N - n_l - \frac{f_m}{h}(x - l).
    \] Solving for their intersection, we recover our formula.

    \begin{theorem}
        Let $\varphi$ be a monotone function, and let two variables be related as $y
        = \varphi(x)$. Then their medians are related as $M_y = \varphi(M_x)$.
    \end{theorem}
    
    \begin{theorem}
        The median of a combination of two sets of observations lies in between the
        individual medians.
    \end{theorem}


    \subsection{Mode}
    
    The mode of a collection of observations $x_1, \dots, x_n$ is the value with the
    highest frequency.

    For grouped data, we pick the value with the highest frequency density. Let $f_m$
    denote the frequency of the modal class $(l, u]$. We approximate \[
        M_0 = l + \frac{f_m - f_{m - 1}}{2f_m - f_{m - 1} - f_{m + 1}} \cdot h.
    \] 


    \begin{theorem}
        An empirical relation between these measures of central tendency is given by
        \[
            \text{mean} - \text{mode} \,\approx\, 3(\text{mean} - \text{median}).
        \] 
    \end{theorem}


    \section{Measures of dispersion}
    
    \subsection{Range}

    The range is a simple way of measuring how \emph{dispersed} or scattered a set of
    observations is. This is simply the difference between the maximum and the
    minimum value in the set.

    \begin{theorem}
        If two variables are related by $y = a + bx$, then their ranges are related
        by \[
            R_Y = |b|\cdot R_X.
        \] 
    \end{theorem}

    \subsection{Mean deviation}
    
    The mean deviation about some value $\alpha$ is defined by \[
        MD(\alpha) = \frac{1}{n}\sum_{i = 1}^n |x_i - \alpha|.
    \] 
    

    \begin{theorem}
        If two variables are related by $y = a + bx$, then \[
            MD_Y(\alpha) = |b|\cdot MD_X(\alpha).
        \] 
    \end{theorem}

    \begin{theorem}
        The mean deviation about a point is minimized at the median.
    \end{theorem}

    \begin{exercise}
        The mean deviation is given by \[
            n\cdot MD(\alpha) = S_2 - S_1 + (n_1 - n_2)\alpha.
        \] Here, $n_1$ is the number of values less than $\alpha$ and $S_1$ is their
        sum, and $n_2$ is the number of values more than $\alpha$ and $S_2$ is their
        sum.
    \end{exercise}
    \begin{proof}
        Calculate \begin{align*}
            n\cdot MD(\alpha) &= \sum_{i = 1}^n |x_i - \alpha| \\
                &= \sum_{x_i < \alpha} \alpha - x_i + \sum_{x_i \geq \alpha} x_i -
                \alpha \\
                &= n_1\alpha - S_1 + S_2 - n_2\alpha \\
                &= S_2 - S_1 + (n_1 - n_2)\alpha. \qedhere
        \end{align*}
        By denoting $n_\alpha$ to be the number of values less than $\alpha$,
        $S_\alpha$ to be their sum, and $S$ to be the sum of all elements, we have \[
            n\cdot MD(\alpha) = S - 2S_\alpha + (2n_\alpha - n)\alpha.
        \] 
    \end{proof}

    \subsection{Root mean square deviation}
    
    The RMS deviation about some value $\alpha$ is defined by \[
        RMS(\alpha) = \sqrt{\frac{1}{n} \sum_{i = 1}^n (x_i - \alpha)^2}.
    \] We call $RMS(\bar{x})$ the standard deviation $\sigma$, and its square the
    variance. We can calculate \[
        \sigma^2 = \frac{1}{n} \sum_{i = 1}^n x_i^2 - \bar{x}^2.
    \] 

    \begin{theorem}
        The root mean square deviation about a point is minimized at the mean.
    \end{theorem}

    \begin{theorem}
        The standard deviations of two sets of observations are related by \[
            \sigma^2 = \frac{n_1\sigma_1^2 + n_2\sigma_2^2}{n_1 + n_2} +
            \frac{n_1(\bar{x}_1 - \bar{x})^2 + n_2(\bar{x}_2 - \bar{x})^2}{n_1 +
            n_2}.
        \] In general, for $k$ sets of observations, we have \[
            \sigma^2 = \frac{1}{N}\sum_{i = 1}^k n_i\sigma_i^2 + n_i(\bar{x}_i -
            \bar{x})^2.
        \] 
    \end{theorem}

    \begin{theorem}
        If two variables are related by $y = a + bx$, then \[
            \sigma_Y = |b|\cdot \sigma_X.
        \] 
    \end{theorem}

    \begin{exercise}
        If a single observation $\alpha$ is added to a set of $n$ values, then the
        standard deviation increases only if \[
            |\bar{x} - \alpha| > \sqrt{\frac{n + 1}{n}}\cdot \sigma.
        \] 
    \end{exercise}
    \begin{proof}
        Without loss of generality, let $\bar{x} = 0$; this can be done by
        relabelling the data $x_i - \bar{x}$, putting the mean at zero without
        affecting the variance. Thus, we have \[
            n\sigma^2 = \sum_{i = 1}^n x_i^2.
        \] Upon adding the point $\alpha$ to our data, the new mean is \[
            \bar{x}_n = \frac{\alpha}{n + 1},
        \] so our new variance is related as \begin{align*}
            (n + 1)\sigma_n^2 &= \sum_{i = 1}^n x_i^2 + \alpha^2 - (n + 1)\bar{x}_n^2 \\
                &= n\sigma^2 + \alpha^2 - \frac{1}{n + 1}\alpha^2, \\
            (n + 1)(\sigma_n^2 - \sigma^2) &= -\sigma^2 + \frac{n}{n + 1}\alpha^2.
        \end{align*}
        For the standard deviation to increase, this must be positive, hence \[
            \alpha^2 > \frac{n + 1}{n}\sigma^2
        \] as desired.
    \end{proof}

    \begin{theorem}
        The mean deviation about the mean cannot exceed the standard deviation.
    \end{theorem}
    \begin{theorem}
        The difference between the mean and median cannot exceed the standard
        deviation.
    \end{theorem}

    \begin{exercise}
        The range and standard deviation obey \[
            \frac{R^2}{2n} \leq \sigma^2 \leq \frac{R^2}{4}.
        \] 
    \end{exercise}
    \begin{proof}
        Without loss of generality, let $\bar{x} = 0$. Then, \[
            n\sigma^2 = \sum_{i = 1}^n x_i^2, \qquad R = x_n - x_1.
        \] Set $\alpha = (x_n + x_1) / 2$. Since the RMS deviation is minimized at
        the mean, we have \begin{align*}
            n\sigma^2 &\leq \sum_{i = 1}^n (x_i - \alpha)^2 \\
                &= \sum_{x_i < \alpha} (x_i - \alpha)^2 + \sum_{x_i \geq \alpha} (x_i
                - \alpha)^2 \\
                &\leq \sum_{x_i < \alpha} (x_1 - \alpha)^2 + \sum_{x_i \geq \alpha}
                (x_n - \alpha)^2 \\
                &= \sum_{x_i < \alpha} \frac{R^2}{4} + \sum_{x_i \geq \alpha}
                \frac{R^2}{4} \\
                &= \frac{nR^2}{4}.
        \end{align*}
        Finally note that $RMS \geq AM$ gives \[
            n\sigma^2 \geq x_n^2 + x_1^2 \geq \frac{(|x_n| + |x_1|)^2}{2} =
            \frac{R^2}{2}. \qedhere
        \] 
    \end{proof}

    \begin{lemma}
        The standard deviation is given by \[
            \sigma^2 = \frac{1}{2n^2}\sum_{i, j} (x_i - x_j)^2.
        \] 
    \end{lemma}
    \begin{proof}
        Observe that \[
            \sum_{ij} (x_i - x_j)^2 = 2\sum_{ij}x_i^2 - \sum_{ij}2x_jx_j = 2n\sum_i x_i^2
            - 2\sum_i x_i\sum_j x_j = 2n\sum_i x_j^2 - 2n^2\bar{x}^2. \qedhere
        \] 
    \end{proof}
    

    \subsection{Quartile deviation}
    
    A \emph{quantile} of order $p$ is such a value of the variable such that a
    proportion $p$ of all the values are less than or equal to it. For grouped data,
    we estimate \[
        z_p = l + \frac{np - n_l}{f_m}\cdot h.
    \] The quartile deviation, or semi-interquartile range is defined \[
        Q = \frac{Q_3 - Q_1}{2} = \frac{z_{3 / 4} - z_{1 / 4}}{2}.
    \] 


    \subsection{Coefficient of variation}

    Unlike the previous measures, the coefficient of variation is a \emph{relative}
    measure of dispersion, expressed as a percentage. \[
        CV = \frac{\sigma}{\bar{x}}.
    \] A variable having a lower coefficient of variation is considered to be more
    stable. Similar coefficients are \[
        CV(\alpha) = \frac{MD(\alpha)}{\alpha}
    \] where $\alpha$ is the mean or median.


    \begin{exercise}
        Suppose that the deviations $x_i - \bar{x}$ are small, so that $((x_i -
        \bar{x}) / \bar{x})^3$ and higher powers can be neglected. Then, \begin{enumerate}
            \item $GM \approx \bar{x}(1 - \sigma^2 / 2\bar{x}^2)$.
            \item $HM \approx \bar{x}(1 - \sigma^2 / \bar{x}^2)$.
            \item $\bar{x}^2 - GM^2 \approx \sigma^2$.
            \item $\bar{x} - 2GM + HM \approx 0$.
            \item $E(\sqrt{X}) \approx \sqrt{\bar{x}}(1 - \sigma^2 / 8\bar{x}^2)$.
        \end{enumerate}
    \end{exercise}
    \begin{proof} \mbox{}
        \begin{enumerate}
            \item Write \[
                \log{GM} = \frac{1}{n}\sum \log{x_i} = \frac{1}{n}\sum
                \log\left[\bar{x}\left(1 + \frac{x_i -
                \bar{x}}{\bar{x}}\right)\right].
            \] Using the series expansion of the logarithm, this is approximately \[
                \log{\bar{x}} + \frac{1}{n}\sum \frac{x_i - \bar{x}}{\bar{x}} +
                \frac{(x_i - \bar{x})^2}{2\bar{x}^2} = 
                \log{\bar{x}} - \frac{\sigma^2}{2\bar{x}^2}.
            \] Finally, use $e^\alpha \approx 1 + \alpha$ to write \[
                GM \approx \bar{x}\left(1 - \frac{\sigma^2}{2\bar{x}^2}\right).
            \] 


            \item Write \[
                \frac{1}{HM} = \frac{1}{n}\sum \frac{1}{x_i} = \frac{1}{n\bar{x}}\sum
                \left[1 + \frac{x_i - \bar{x}}{\bar{x}}\right]^{-1}.
            \] Using the series expansion of $1 / (1 + x)$, this is approximately \[
                \frac{1}{n\bar{x}}\sum 1 - \frac{x_i - \bar{x}}{\bar{x}} + \frac{(x_i
                - \bar{x})^2}{\bar{x}^2} = \frac{1}{\bar{x}} \left(1 +
                \frac{\sigma^2}{\bar{x}^2}\right).
            \] Taking the reciprocal and approximating $(1 + \alpha)^{-1} \approx 1 -
            \alpha$ gives \[
                HM \approx \bar{x}\left(1 - \frac{\sigma^2}{\bar{x}^2}\right).
            \] 

            \item Use the first approximation to estimate \[
                \bar{x}^2 - GM^2 \approx \bar{x}^2\left[1 - \left(1 -
                \frac{\sigma^2}{2\bar{x}^2}\right)^2\right].
            \] Use $(1 - x)^2 \approx 1 - 2x$ to write \[
                \bar{x}^2 - GM^2 \approx \bar{x}^2\left[1 - 1 +
                \frac{\sigma^2}{\bar{x}^2}\right] = \sigma^2.
            \] 


            \item Use the first two approximations to write \[
                \bar{x} - 2GM + HM \approx 0.
            \] 


            \item Write \[
                E[\sqrt{X}] = \frac{1}{n}\sum \sqrt{x_i} = \frac{\sqrt{\bar{x}}}{n}\sum
                \left[1 + \frac{x_i - \bar{x}}{\bar{x}}\right]^{1 / 2}.
            \] Using the series expansion of the square root, this is approximately
            \[
                E[\sqrt{X}] \approx \frac{\sqrt{\bar{x}}}{n}\sum 1 + \frac{x_i -
                \bar{x}}{2\bar{x}} - \frac{(x_i - \bar{x})^2}{8\bar{x}^2} =
                \sqrt{\bar{x}} \left(1 - \frac{\sigma^2}{8\bar{x}^2}\right).
            \] 
        \end{enumerate}
    \end{proof}


    \subsection{Moments}
    
    The $r$th order moment about $\alpha$ is given by \[
        m_r(\alpha) = \frac{1}{n}\sum_{i = 1}^n (x - \alpha)^r.
    \] The corresponding central moment is simply \[
        m_r = \frac{1}{n}\sum_{i = 1}^n (x - \bar{x})^r.
    \] The $r$th order raw moment is simply $m_r' = m_r(0)$.

    \begin{lemma}
        If two variables are related by $y = a + bx$, then \[
            m_{r, Y} = b^r\cdot m_{r, X}.
        \] 
    \end{lemma}

    \begin{lemma}
        The central moments can be expressed in terms of raw moments as \[
            m_r = \sum_{k = 0}^r \binom{n}{k} (-1)^k m'_{r - k}(m'_1)^k.
        \]
    \end{lemma}

    \begin{definition}
        Define \[
            b_1 = \frac{m_3^2}{m_2^3}, \qquad b_2 = \frac{m_4}{m_2^2}.
        \] 
    \end{definition}

    \begin{theorem}
        For any frequency distribution, $b_1 \geq 1$, $b_2 > b_1$, $b_2 \geq b_1 +
        1$. Equality holds only when the variable takes two values with equal
        frequency.
    \end{theorem}
    
    The absolute $r$th order moment about $\alpha$ is given by \[
        \nu_r(\alpha) = \frac{1}{n}\sum_{i = 1}^n |x - \alpha|^r.
    \] For absolute central moments, we have $\nu_r = \nu'_r(\bar{x})$.


    \begin{theorem}
        \[
            \nu_{b}^{a - c} \leq \nu_{c}^{a - b}\,\nu_{a}^{b - c}, \qquad
            a > b > c \geq 0.
        \] 
    \end{theorem}
    \begin{corollary}
        \[
            \nu_{k + l}^2 < \nu_{2k}\,\nu_{2l}, \qquad
            m_{k + l}^2 < m_{2k}\,m_{2l}.
        \] 
    \end{corollary}
    
    \begin{theorem}[Liapunov]
        $\nu_{x}^{1 / x}$ is increasing in $x$.
    \end{theorem}


    \section{Skew and kurtosis}

    \subsection{Skew}
    
    Skewness is a measure of lack of symmetry in a frequency distribution. A
    positively skewed distribution has a longer tail to the right.

    The odd central moments of a symmetric distribution are all zero for a symmetric
    distribution, positive for a positively skewed distribution. Thus, one measure of
    skewness is \[
        g_1 = \frac{m_3}{m_2^{3 / 2}}.
    \]

    \begin{lemma}
        For a positively skewed distribution, we have \[
            \text{mean} > \text{median} > \text{mode}.
        \] 
    \end{lemma}

    Thus, \[
        s_k = \frac{\text{mean} - \text{mode}}{\sigma}
    \] is considered a measure of skewness. We have seen the empirical relation
    between mean, median, mode, and standard deviation, hence we typically have \[
        -3 \leq s_k \leq 3.
    \] This measure $s_k$ is called Pearson's coefficient of skewness.

    For a positively skewed distribution, $Q_1$ is nearer to $Q_2$ than $Q_3$. Thus,
    Bowley's coefficient of skewness is \[
        s_k = \frac{Q_3 - 2Q_2 + Q_1}{Q_3 - Q_1} \in [-1, 1].
    \]


    \subsection{Kurtosis}
    
    Kurtosis is a measure of the peakedness of a frequency distribution. We look at
    $m_4$, normalized as $b_2 = m_4 / m_2^2$ as a measure of kurtosis.

    \begin{lemma}
        For a normal distribution, $b_2 = 3$.
    \end{lemma}

    Thus, $g_2 = b_2 - 3$ is also a measure of kurtosis. A distribution with $g_2 =
    0$ is called \emph{mesokurtic}, $g_2 > 0$ is called \emph{leptokurtic}, and $g_2
    < 0$ is called \emph{platykurtic}.


    
    \section{Bivariate data}
    
    Here, out data items are in the form of points $(x_i, y_i)$. We are typically
    interested in predicting the values of one of these variables (called the
    \emph{study} variable) given knowledge of the other (called the \emph{auxiliary}
    variable).


    \subsection{Correlation}
    Correlation is a measure of how change in one variable is associated with change
    in the other. Here, we only examine linear correlation. Two variables are said to
    be \emph{positively} correlated if one variable increases with average increase
    in the other, or \emph{negatively} correlated if one variable decreases with
    average increase in the other.

    We define the covariance between two variables as \[
        \cov(x, y) = \frac{1}{n}\sum_{i = 1}^n (x_i - \bar{x})(y_i - \bar{y}).
    \] The Pearson's product moment correlation coefficient is now defined as \[
        r(x, y) = \frac{\cov(x, y)}{\sigma_x\cdot \sigma_y}.
    \]

    \begin{lemma}
        \[
            \cov(x, y) = \frac{1}{n}\sum_{i = 1}^n x_iy_i - \bar{x}\bar{y}.
        \] 
    \end{lemma}

    \begin{lemma}
        \[
            \cov(ax + by, cx + dy) = ac\sigma_x^2 + bd\sigma_y^2 + (ad + bc)\cov(x,
            y).
        \]
    \end{lemma}

    \begin{lemma}
        The numerical value of $r$ is invariant under shifting and scaling.
    \end{lemma}
    
    \begin{lemma}
        We have \[
            -1 \leq r(x, y) \leq 1.
        \] Furthermore, equality holds if and only if $(x_i - \bar{x}) = k(y_i -
        \bar{y})$, $k = \pm \sigma_x / \sigma_y$.
    \end{lemma}


    \subsection{Least square regression}
    
    Given two variables $X, Y$, we want to relate them as \[
        Y = \phi(X) + \epsilon = \beta_1 + \beta_2X + \epsilon.
    \] Here, our choice of the regression function $\phi$ is linear. $X$ is called
    the \emph{auxiliary} variable, and $Y$ is called the \emph{response} variable.

    In order to determine $\beta_1, \beta_2$, we minimize the sum of squares of
    errors $\epsilon_i = y_i - (\beta_1 + \beta_2x_i)$. This yields \[
        \hat{\beta}_1 = \bar{y} - \hat{\beta}_2\bar{x}, \qquad
        \hat{\beta}_2 = r_{xy} \cdot \frac{\sigma_y}{\sigma_x} = \frac{\sum_i x_iy_i
        - n\bar{x}\bar{y}}{\sum_i x_i^2 - n\bar{x}^2}.
    \] The regression line of $Y$ on $X$ thus is \[
        \hat{y} = \bar{y} + r\frac{\sigma_y}{\sigma_x}(x - \bar{x}).
    \] The slope $b_{yx}$ is called the regression coefficient of $Y$ on $X$. Note
    that we assume that $X$ is free of error.

    The regression line of $X$ on $Y$ is \[
        \hat{x} = \bar{x} + r\frac{\sigma_x}{\sigma_y}(y - \bar{y}).
    \] 

    \begin{lemma}
        If $u = (x - a) / c$, $v = (y - b) / d$, then \[
            b_{yx} = \frac{d}{c}b_{vu}, \qquad
            b_{xy} = \frac{c}{d}b_{uv}.
        \] 
    \end{lemma}

    \begin{lemma}
        The mean of the predicted values $\hat{y}$ is equal to the mean $\bar{y}$ of
        the values.
    \end{lemma}

    The residuals are \[
        \hat{\epsilon}_i = y_i - \hat{y}_i = y_i - \bar{y} - b_{yx}(x_i - \bar{x}).
    \] 

    \begin{lemma}
        The sum of residuals is zero.
    \end{lemma}

    \begin{lemma}
        \[
            |r_{xy}| = \frac{\sigma_{\hat{y}}}{\sigma{y}}, \qquad
            \sigma_{\hat{\epsilon}} = \sigma_y\sqrt{1 - r^2}.
        \] 
    \end{lemma}

    Thus, the coefficient of determination $r^2 = b_{yx}b_{xy}$ is a measure of the
    usefulness of the linear regression. If $r = 0$, then $\sigma_{\hat{\epsilon}} =
    \sigma_y$, making the regression pointless.

    \begin{lemma}
        \[
            \cov(\hat{y}, \hat{\epsilon}) = 0.  
        \]
    \end{lemma}

    \begin{lemma}
        The angle between both regression lines satisfies \[
            \tan\theta = \left|\frac{1 - r^2}{r} \cdot
            \frac{\sigma_x\sigma_y}{\sigma_x^2 + \sigma_y^2}\right|.
        \] 
    \end{lemma}


    Now suppose that our variables are empirically related as \[
        y = f(x, \beta_1, \beta_2, \dots, \beta_n) = f(x; \beta).
    \] Here, $\beta = (\beta_1, \dots, \beta_n)$ are unknown constants. Make an
    initial guess $\beta^{(0)}$, and note that we wish to minimize $\beta -
    \beta^{(0)}$. Taylor's theorem gives the approximation \[
        y = f(x; \beta) \approx f(x; \beta^{(0)}) + (\beta - \beta^{(0)})\cdot
        \left[\frac{\partial f}{\partial \beta_i}(\beta^{(0)})\right].
    \] Since this is now linear, the values $\beta - \beta^{(0)}$ can be determined
    using the least squares method as before. This gives us a new approximation
    $\beta^{(1)}$; we can repeat this process getting better and better solutions. \\


    Suppose that both variables are subject to error. To perform a linear regression
    \[
        ax + by + 1 = 0,
    \] we minimize the sum of squares of distances from this line to the points
    $(x_i, y_i)$, i.e.\ we want to minimize \[
        \sum_{i = 1}^n \frac{(ax_i + by_i + 1)^2}{a^2 + b^2}.
    \] Setting the partial derivatives to zero yields \[
        ab^2\sum_i (x_i^2 - y_i^2) + b(b^2 - a^2)\sum_i x_iy_i + (b^2 - a^2)\sum_ix_i
        - 2ab\sum_i y_i = na,
    \] \[
        a^2b\sum_i (y_i^2 - x_i^2) + a(a^2 - b^2)\sum_i x_iy_i + (a^2 - b^2)\sum_iy_i
        - 2ab\sum_i x_i = nb,
    \] \[
        a\bar{x} + b\bar{y} + 1 = 0.
    \] 

    
    \subsection{Orthogonal polynomials}
    
    A sequence of polynomials $\{p_i\}$ is said to be orthogonal if \[
        \sum_x p_i(x)\cdot p_j(x) \;\begin{cases}
            = 0, &\text{ if } i \neq j \\
            \neq 0, &\text{ if } i = j
        \end{cases}.
    \] Consider the polynomials \[
        p_n(x) = \sum_{k = 0}^n c_{nk}x^k.
    \] When considering the first $N + 1$ such polynomials, we have $(N + 1)(N + 2) /
    2$ unknown constants $c_{nk}$. For these to be orthogonal, we have $N(N + 1) / 2$
    equations from the orthogonality equations. Thus, we need $N + 1$ additional
    constraints to fully determine the coefficients. Typically, we take the
    coefficients of the highest power, $c_{nn} = 1$. Thus, we can calculate such
    orthogonal polynomials, given our data $x_1, \dots, x_n$.

\end{document}
