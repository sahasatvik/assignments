\documentclass[11pt]{article}

\usepackage[T1]{fontenc}
\usepackage{geometry}
\usepackage{amsmath, amssymb, amsthm}
\usepackage[scr]{rsfso}
\usepackage{bm}
\usepackage[%
    hidealllines=true,%
    innerbottommargin=15,%
    nobreak=true,%
]{mdframed}
\usepackage{xcolor}
\usepackage{graphicx}
\usepackage{fancyhdr}
\usepackage{hyperref}

\geometry{a4paper, margin=1in, headheight=14pt}

\pagestyle{fancy}
\fancyhf{}
\renewcommand\headrulewidth{0.4pt}
\fancyhead[L]{\scshape MA3206: Statistics I}
\fancyhead[R]{\scshape \leftmark}
\rfoot{\footnotesize\it Updated on \today}
\cfoot{\thepage}

\newcommand{\C}{\mathbb{C}}
\newcommand{\R}{\mathbb{R}}
\newcommand{\Q}{\mathbb{Q}}
\newcommand{\Z}{\mathbb{Z}}
\newcommand{\N}{\mathbb{N}}

\newmdtheoremenv[%
    backgroundcolor=blue!10!white,%
]{theorem}{Theorem}[section]
\newmdtheoremenv[%
    backgroundcolor=violet!10!white,%
]{corollary}{Corollary}[theorem]
\newmdtheoremenv[%
    backgroundcolor=teal!10!white,%
]{lemma}[theorem]{Lemma}

\theoremstyle{definition}
\newmdtheoremenv[%
    backgroundcolor=green!10!white,%
]{definition}{Definition}[section]
\newmdtheoremenv[%
    backgroundcolor=red!10!white,%
]{exercise}{Exercise}[section]

\theoremstyle{remark}
\newtheorem*{remark}{Remark}
\newtheorem*{example}{Example}
\newtheorem*{solution}{Solution}

\surroundwithmdframed[%
    linecolor=black!20!white,%
    hidealllines=false,%
    innertopmargin=5,%
    innerbottommargin=10,%
    skipabove=0,%
    skipbelow=0,%
]{example}

\numberwithin{equation}{section}

\title{
    \Large\textsc{MA3206} \\
    \Huge \textbf{Statistics I} \\
    \vspace{5pt}
    \Large{Spring 2022}
}
\author{
    \large Satvik Saha
    \\\textsc{\small 19MS154}
}
\date{\normalsize
    \textit{Indian Institute of Science Education and Research, Kolkata, \\
    Mohanpur, West Bengal, 741246, India.} \\
}

\begin{document}
    \maketitle

    \tableofcontents

    \section{Analysing data}

    \subsection{Categorizing data}
    
    We are interested in two types of data: \emph{categorical} and \emph{numerical}.
    Categorical data used named qualities to describe a particular observation. This
    can be further categorized into \emph{nominal} and \emph{ordinal}; the latter
    admit a natural ordering. Numerical data uses numbers, and can be further
    categorized into \emph{discrete} and \emph{continuous}.
    
    \subsection{Measures of central tendency}
    
    Suppose that we have been given a collection of $n$ numeric observations, denoted
    $x_1, x_2, \dots, x_n$. These may be concentrated around some specific point, or
    spread out over some range; regardless, we wish to identify one particular point
    around which our observations are `balanced' or aggregate in some sense. In other
    words, we want to identify a point $\bar{x}$ such that the net deviation $|x_i -
    \bar{x}|$ is minimized. For convenience, we consider the square deviations $(x_i
    - \bar{x})^2$; thus, we wish to minimize the loss function defined by \[
        t \mapsto \sum_{i = 1}^n (x_i - t)^2.
    \] It is easy to check that our loss function attains its minimum at \[
        \bar{x} = \frac{1}{n}\sum_{i = 1}^n x_i.
    \] This quantity $\bar{x}$ is called the \emph{arithmetic mean} of our data.
    Note that this is not the only choice of loss function measuring central
    tendency, but it is certainly quite convenient.

    If our data is summarized in terms of frequencies, i.e.\ each $x_i$ has been
    recorded $f_i$ times, we may write \[
        \bar{x} = \frac{1}{N}\sum_{i = 1}^n f_ix_i, \qquad N = \sum_{i = 1}^n f_i.
    \] The quantities $f_i / N$ are often referred to as the \emph{weights} of the
    observations $x_i$. \\

    Now suppose that our data values have not been explicitly presented: instead, we
    have been given the data classes $(x_{i - 1}, x_{i}]$ and the number of
    observations $f_i$ falling within each class. We can make an estimate of the true
    mean by identifying each data class with some value, say $(x_{i - 1}, x_{i}]$ gets
    associated with $x_i^* = (x_{i - 1} + x_{i}) / 2$. Then we calculate the usual
    arithmetic mean using these values. This gives us the estimate \[
        \bar{x}^* = \frac{1}{N}\sum_{i = 1}^n f_i x_i^*, \qquad 
        N = \sum_{i = 1}^n f_i.
    \] Note that the true mean must lie within the bounds \[
        \frac{1}{N}\sum_{i = 1}^{n} f_i x_{i - 1} \;\leq \bar{x} \leq\;
        \frac{1}{N}\sum_{i = 1}^{n} f_i x_{i}.
    \] Suppose that each data class has width $h$. We may estimate the error in our
    mean by observing that within a particular class $(x_{i - 1}, x_i]$ with
    frequency $f_i$, the deviation between any of the true data points and $x_i^*$ is
    at most $h / 2$. Thus, the net deviation accumulated over a particular class is
    at most $f_ih / 2$, and the net deviation overall is at most $Nh / 2$. Putting
    everything together, we have \[
        |\bar{x} - \bar{x}^*| \leq \frac{h}{2}.
    \] 

\end{document}
