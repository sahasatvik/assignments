\documentclass[11pt]{article}

\usepackage[T1]{fontenc}
\usepackage{geometry}
\usepackage{amsmath, amssymb, amsthm}
\usepackage[scr]{rsfso}
\usepackage{bm}
\usepackage[%
    hidealllines=true,%
    innerbottommargin=15,%
    nobreak=true,%
]{mdframed}
\usepackage{xcolor}
\usepackage{graphicx}
\usepackage{fancyhdr}
\usepackage{hyperref}

\geometry{a4paper, margin=1in, headheight=14pt}

\pagestyle{fancy}
\fancyhf{}
\renewcommand\headrulewidth{0.4pt}
\fancyhead[L]{\scshape MA3206: Statistics I}
\fancyhead[R]{\scshape \leftmark}
\rfoot{\footnotesize\it Updated on \today}
\cfoot{\thepage}

\newcommand{\C}{\mathbb{C}}
\newcommand{\R}{\mathbb{R}}
\newcommand{\Q}{\mathbb{Q}}
\newcommand{\Z}{\mathbb{Z}}
\newcommand{\N}{\mathbb{N}}

\newmdtheoremenv[%
    backgroundcolor=blue!10!white,%
]{theorem}{Theorem}[section]
\newmdtheoremenv[%
    backgroundcolor=violet!10!white,%
]{corollary}{Corollary}[theorem]
\newmdtheoremenv[%
    backgroundcolor=teal!10!white,%
]{lemma}[theorem]{Lemma}

\theoremstyle{definition}
\newmdtheoremenv[%
    backgroundcolor=green!10!white,%
]{definition}{Definition}[section]
\newmdtheoremenv[%
    backgroundcolor=red!10!white,%
]{exercise}{Exercise}[section]

\theoremstyle{remark}
\newtheorem*{remark}{Remark}
\newtheorem*{example}{Example}
\newtheorem*{solution}{Solution}

\surroundwithmdframed[%
    linecolor=black!20!white,%
    hidealllines=false,%
    innertopmargin=5,%
    innerbottommargin=10,%
    skipabove=0,%
    skipbelow=0,%
]{example}

\numberwithin{equation}{section}

\title{
    \Large\textsc{MA3206} \\
    \Huge \textbf{Statistics I} \\
    \vspace{5pt}
    \Large{Spring 2022}
}
\author{
    \large Satvik Saha
    \\\textsc{\small 19MS154}
}
\date{\normalsize
    \textit{Indian Institute of Science Education and Research, Kolkata, \\
    Mohanpur, West Bengal, 741246, India.} \\
}

\begin{document}
    \maketitle

    \tableofcontents

    \section{Introduction}

    We are interested in two types of data: \emph{categorical} and \emph{numerical}.
    Categorical data used named qualities to describe a particular observation. This
    can be further categorized into \emph{nominal} and \emph{ordinal}; the latter
    admit a natural ordering. Numerical data uses numbers, and can be further
    categorized into \emph{discrete} and \emph{continuous}.
    
    \section{Measures of central tendency}

    \subsection{Arithmetic mean}
    
    Suppose that we have been given a collection of $n$ numeric observations, denoted
    $x_1, x_2, \dots, x_n$. These may be concentrated around some specific point, or
    spread out over some range; regardless, we wish to identify one particular point
    around which our observations are `balanced' or aggregate in some sense. In other
    words, we want to identify a point $\bar{x}$ such that the net deviation $|x_i -
    \bar{x}|$ is minimized. For convenience, we consider the square deviations $(x_i
    - \bar{x})^2$; thus, we wish to minimize the loss function defined by \[
        t \mapsto \sum_{i = 1}^n (x_i - t)^2.
    \] It is easy to check that our loss function attains its minimum at \[
        \bar{x} = \frac{1}{n}\sum_{i = 1}^n x_i.
    \] This quantity $\bar{x}$ is called the \emph{arithmetic mean} of our data.
    Note that this is not the only choice of loss function measuring central
    tendency, but it is certainly quite convenient.

    If our data is summarized in terms of frequencies, i.e.\ each $x_i$ has been
    recorded $f_i$ times, we may write \[
        \bar{x} = \frac{1}{N}\sum_{i = 1}^n f_ix_i, \qquad N = \sum_{i = 1}^n f_i.
    \] The quantities $f_i / N$ are often referred to as the \emph{weights} of the
    observations $x_i$. The arithmetic mean can thus be interpreted as their `centre
    of mass'. \\

    Now suppose that our data values have not been explicitly presented: instead, we
    have been given the data classes $(x_{i - 1}, x_{i}]$ and the number of
    observations $f_i$ falling within each class. We can make an estimate of the true
    mean by identifying each data class with some value, say $(x_{i - 1}, x_{i}]$ gets
    associated with $x_i^* = (x_{i - 1} + x_{i}) / 2$. Then we calculate the usual
    arithmetic mean using these values. This gives us the estimate \[
        \bar{x}^* = \frac{1}{N}\sum_{i = 1}^n f_i x_i^*, \qquad 
        N = \sum_{i = 1}^n f_i.
    \] Note that the true mean must lie within the bounds \[
        \frac{1}{N}\sum_{i = 1}^{n} f_i x_{i - 1} \;\leq \bar{x} \leq\;
        \frac{1}{N}\sum_{i = 1}^{n} f_i x_{i}.
    \] Suppose that each data class has width $h$. We may estimate the error in our
    mean by observing that within a particular class $(x_{i - 1}, x_i]$ with
    frequency $f_i$, the deviation between any of the true data points and $x_i^*$ is
    at most $h / 2$. Thus, the net deviation accumulated over a particular class is
    at most $f_ih / 2$, and the net deviation overall is at most $Nh / 2$. Putting
    everything together, we have \[
        |\bar{x} - \bar{x}^*| \leq \frac{h}{2}.
    \]

    
    \subsection{Geometric mean}

    Another measure of central tendency is the geometric mean $G$, calculated \[
        G = \sqrt[n]{x_1x_2\cdots x_n}.
    \] Note that \[
        \log{G} = \frac{1}{n}\sum_{i = 1}^n \log{x_i}.
    \] Consider $k$ sets of observations, with $n_i$ observations in each set. Then,
    the geometric mean of the combined data is related with the geometric means $G_I$
    of the sets as \[
        \log{G} = \frac{1}{N}\sum_{i = 1}^k n_i\log{G_i}, \qquad N = \sum_{i = 1}^k
        n_i.
    \]
    
    \subsection{Harmonic mean}

    Another measure of central tendency is the harmonic mean $G$, calculated \[
        \frac{1}{H} = \frac{1}{n}\sum_{i = 1}^n \frac{1}{x_i}.
    \] The Harmonic means of combined data and sets of data are related as \[
        \frac{N}{H} = \sum_{i = 1}^k \frac{n_i}{H_i}, \qquad N = \sum_{i = 1}^k n_i.
    \] 

    \begin{theorem}
        Given two positive numbers, their arithmetic, geometric, and harmonic means
        all lie between them.
    \end{theorem}
    \begin{proof}
        Without loss of generality, let $x \geq y > 0$. Then for any $a, b$, we have
        \[
            x = \frac{ax + bx}{a + b} \geq \frac{ax + by}{a + b} \geq \frac{ay +
            by}{a + b} = y.
        \] Setting $a = b = 1 / 2$ give the result for the arithmetic mean. Now, the
        logarithm function is monotonic for positive reals, so $\log{x} \geq
        \log{y}$. Applying the above gives \[
            \log{x} \geq \frac{1}{2}(\log{x} + \log{y}) \geq \log{y},
        \] and taking exponentials yields \[
            x \geq \sqrt{xy} \geq y.
        \] Finally, applying the result to $1 / y \geq 1 / x$, we have \[
            \frac{1}{y} \geq \frac{a / y + b / x}{a + b} \geq \frac{1}{x},
        \] which we can rearrange and set $a = b = 1 / 2$ to get \[
            x \geq \frac{2}{1 / x + 1 / y} \geq y. \qedhere
        \] 
    \end{proof}
    \begin{remark}
        The same proof applies for weighted means.
    \end{remark}

    \begin{theorem}
        For $n$ observations $x_1, \dots, x_n$, the arithmetic mean, geometric mean,
        and harmonic mean are in descending order, i.e.\ \[
            AM \geq GM \geq HM.
        \] 
    \end{theorem}
    \begin{proof}
        We assume that all $x_i > 0$. Consider the case $n = 2$. Then, \[
            (\sqrt{x_1} - \sqrt{x_2})^2 \geq 0, \qquad x_1 + x_2 \geq 2\sqrt{x_1 x_2}
        \] is precisely $AM \geq GM$. Applying the same on the reciprocals, \[
            \frac{1}{x_1} + \frac{1}{x_2} \geq 2\sqrt{\frac{1}{x_1 x_2}}, \qquad
            \sqrt{x_1 x_2} \geq \frac{2}{1 / x_1 + 1 / x_2}
        \] is precisely $GM \geq HM$. \\

        Suppose that the result holds for some $n$. Now consider a collection of $2n$
        observations $x_1, \dots, x_{2n}$. Then, applying $AM \geq GM$ on both
        halves, then the two variable case gives \[
            \sum_{i = 1}^{2n} x_i \geq n\sqrt[n]{x_1\cdot x_n} + n\sqrt[n]{x_{n + 1}\cdots
            x_{2n}} \geq 2n\sqrt[2n]{x_1\cdots x_n x_{n + 1}\cdots x_{2n}}
        \] which is precisely $AM \geq GM$ for $2n$ observations.  Now suppose that
        $AM \geq GM$ holds for some $n + 1$. Consider a collection of $n$
        observations $x_1, \dots, x_n$, set $\bar{x} = (x_1 + \dots + x_n) / n$, and
        note that \[
            \sum_{i = 1}^n x_i + \bar{x} \geq (n + 1)\sqrt[n + 1]{x_1\cdots x_n
            \bar{x}}.
        \] The left-hand side is simply $(n + 1)\bar{x}$, so \[
            \bar{x} \geq \sqrt[n + 1]{x_1\cdots x_n \bar{x}}, \qquad \bar{x}^{n / n +
            1} \geq (x_1\cdots x_n)^{1 / n + 1}, \qquad \bar{x} \geq
            \sqrt[n]{x_1\cdots x_n},
        \] which is precisely $AM \geq GM$ for $n$ observations. Therefore, $AM \geq
        GM$ holds for all $n \geq 2$ by induction.

        Now that we have $AM \geq GM$ for $n$ observations, use it on their
        reciprocals to get \[
            \sum_{i = 1}^n \frac{1}{x_i} \geq n\sqrt[n]{\frac{1}{x_1\cdots x_n}}, \qquad 
            \sqrt[n]{x_1\cdots x_n} \geq \frac{n}{\sum_{i = 1}^n 1 / x_i}
        \] which is precisely $GM \geq HM$.
    \end{proof}


    \subsection{Median}
    
    The median of a collection of ordered observations $x_1 \leq x_2 \leq \dots \leq
    x_n$ is defined to be their middle value: $x_{k + 1}$ if $n = 2k + 1$ is odd, and
    the mean $(x_{k} + x_{k + 1}) / 2$ if $n = 2k$ is even.

    For grouped data, we assume that the observations are evenly distributed over
    the median class $(l, u]$ with frequency $f_0$, width $h$. If the total frequency
    is denoted by $N$, we write \[
        \frac{M - l}{h} = \frac{N / 2 - n_l}{f_0}.
    \] Here, $n_l$ is the cumulative frequency of the preceding classes. This will
    give \[
        M = l + \frac{N / 2 - n_l}{f_0}\cdot h.
    \] 

    \begin{theorem}
        Let $\varphi$ be a monotone function, and let two variables be related as $y
        = \varphi(x)$. Then their medians are related as $M_y = \varphi(M_x)$.
    \end{theorem}
    
    \begin{theorem}
        The median of a combination of two sets of observations lies in between the
        individual medians.
    \end{theorem}


    \subsection{Mode}
    
    The mode of a collection of observations $x_1, \dots, x_n$ is the value with the
    highest frequency.

    For grouped data, we pick the value with the highest frequency density. Let $f_m$
    denote the frequency of the modal class $(l, u]$. We approximate \[
        M_0 = l + \frac{f_m - f_{m - 1}}{2f_m - f_{m - 1} - f_{m + 1}} \cdot c.
    \] 


\end{document}
