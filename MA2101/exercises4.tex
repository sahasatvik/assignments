\documentclass[10pt]{article}

\usepackage[T1]{fontenc}
\usepackage{geometry}
\usepackage{amsmath, amssymb, amsthm}
\usepackage{bm}

\title{Analysis I - Assignment IV}
\author{Satvik Saha}
\date{}

\geometry{a4paper, margin=1in}
\setlength\parindent{0pt}
\renewcommand{\labelenumi}{(\alph{enumi})}
% \renewcommand\qedsymbol{$\blacksquare$}
\newcounter{prob}
\def\problem{\stepcounter{prob}\paragraph{Exercise \arabic{prob}}}
\def\solution{\paragraph{Solution}}
\def\cl{\operatorname{cl}}

\begin{document}
        \par\textbf{IISER Kolkata} \hfill \textbf{Assignment IV}
        \vspace{3pt}
        \hrule
        \vspace{3pt}
        \begin{center}
                \LARGE{\textbf{MA 2101 : Analysis I}}
        \end{center}
        \vspace{3pt}
        \hrule
        \vspace{3pt}
        Satvik Saha, \texttt{19MS154}, Group C\hfill\today
        \vspace{20pt}

        \problem Show that no interval in $\mathbb{R}$ is a union of two disjoint nonempty open sets.

        \solution Let $I \subseteq \mathbb{R}$ be an interval such that $I = A \cup B$ where $A$ and $B$ are disjoint, nonempty open sets.
        Pick $a \in A$ and $b \in B$, and without loss of generality\footnote{If $a > b$, just swap the roles of $A$ and $B$.
        Note that $a \neq b$ since $A$ and $B$ share no common elements.} let $a < b$.
        We construct the set
        \[
                S = \{x: a \leq x \leq b, x \in A\} = [a, b] \cap A.
        \]
        Note that $S \subseteq \mathbb{R}$ is bound above and below, and $a \in S$, so $S$ has a supremum, say $\gamma = \sup{S}$.
        We have the restriction $a \leq \gamma \leq b$.
        Since $a, b \in I$, every element in between them must be in $I$ since it is an interval. Thus, $\gamma \in I$. This means that
        $\gamma$ must be in exactly one of $A$ and $B$. \\

        Suppose $\gamma \in A$. This means that $\gamma \neq b$, so $b - \gamma > 0$. Also, from the openness of $A$, we find $r > 0$ such that
        $(\gamma - r, \gamma + r) \subseteq A$. Setting $\epsilon = \min(r, b - \gamma)$, we see that $\gamma' = \gamma + \epsilon /2 \in A$,
        and $a \leq \gamma < \gamma' < b$, so $\gamma' \in S$. This is a contradicts the fact that $\gamma = \sup{S}$. \\

        Similarly, suppose $\gamma \in B$. This means that $\gamma \neq a$, so $\gamma - a > 0$. Also, from the openness of $B$, we find $r > 0$ such that
        $(\gamma - r, \gamma + r) \subseteq B$. Setting $\epsilon = \min(r, \gamma - a)$, we see that $\gamma' = \gamma - \epsilon /2 \in B$,
        and $a < \gamma' < \gamma \leq b$, so $\gamma' \in S$. This means that there are no elements of $A$ between $\gamma'$ and $\gamma$,
        which means that $\gamma'$ is also an upper bound of $S$. This again contradicts the fact that $\gamma = \sup{S}$ is the lowest upper bound. \\

        Thus, we conclude that $\gamma \notin A$ and $\gamma \notin B$, so $\gamma \notin A \cup B = I$, which is absurd.
        Thus, it is impossible to choose such $A$ and $B$, and this proves the desired statement.


        \problem Given $n \in \mathbb{N}$, construct a bounded set $S_n \in \mathbb{R}$ which has exactly $n$ limit points.

        \solution Let $A_m = \{m + \frac{1}{n}: n > 1, n \in \mathbb{N}\}$. Clearly, $A_m$ is bound by $m$ and $m + 1$. We claim that
        \[
                S_n = \bigcup_{m = 1}^n A_m
        \]
        has exactly $n$ limit points. \\
        
        Suppose $x \in \{1, \dots, n\}$. Then, for any neighbourhood $(x - \epsilon, x + \epsilon)$, we find $k \in \mathbb{N}$ such that
        $k\epsilon > 1$ using the Archimedean principle. Thus, $x - \epsilon < x + \frac{1}{k} < x + \epsilon$, so this neighbourhood
        of $x$ contains the point $x + \frac{1}{k} \in A_x \subseteq S_n$, which means that $x$ is a limit point of $S_n$. \\

        Suppose $x \in S_n\setminus\{1, \dots, n\}$. We find $m, n \in \mathbb{N}$, $n > 1$ such that $x = m + 1 /n$. Now,
        \[
                \left|\frac{1}{n} - \frac{1}{n+1}\right| = \frac{1}{n(n + 1)} < \frac{1}{n(n-1)} = \left|\frac{1}{n-1} + \frac{1}{n}\right|.
        \]
        Thus, setting $\epsilon = \frac{1}{n} - \frac{1}{n + 1}$, we see that $(x - \epsilon, x + \epsilon) \cap S_n = \{x\}$, which 
        means that $x$ is not a limit point of $S_n$. \\

        Suppose $x \notin S_n$. Note that the largest element of $S_n$ is $n + \frac{1}{2}$, so if $x > n + \frac{1}{2}$, we set
        $\epsilon = (x - n - \frac{1}{2}) /2$. Every element of $S_n$ is greater than $1$, so if $x < 1$, set $\epsilon = (1 - x)/2$.
        In both cases, we find that $(x - \epsilon, x + \epsilon) \cap S_n = \emptyset$.
        Otherwise, note that $x$ cannot be an integer, since the only integers between $1$ and $n + \frac{1}{2}$ are already in $S_n$.
        Thus, we find $m \in \mathbb{N}$ such that $m < x < m + 1$, i.e.\ $m = \lfloor x\rfloor$. Note that $m \in \{1, \dots n\}$.
        If $x \geq m + \frac{1}{2}$, set $\epsilon = (m + 1 - x) /3$, and note that $(x - \epsilon, x + \epsilon) \cap S_n = \emptyset$.
        Otherwise, $m < x < m + \frac{1}{2}$, so set $\epsilon = (x - m) /2$.
        Any points of $S_n$ in the interval $(x - \epsilon, x + \epsilon)$ must be of the form $m + \frac{1}{k}$,
        where $m + \frac{1}{k} > x - \epsilon = (x + m) /2$, so $1 /k > (x - m)/2$, so $k < 2 /(x - m)$. Thus, there are finitely many
        such $k$, so $(x - \epsilon, x + \epsilon) \cap S_n$ is finite, which means that $x$ is not a limit point of $S_n$. \\
        
        This covers all possible cases of $x \in \mathbb{R}$, so we have proved that $S_n$ has exactly $n$ limit points.
        

        \problem Let $(M, d)$ be a metric space and $A, B$ be two subsets of $M$. Show that $\cl(A \cap B) \subseteq \cl(A) \cap \cl(B)$.
        Is it always true that $\cl(A \cap B) = \cl(A) \cap \cl(B)$?

        \solution Given $A, B \subseteq M$, let $x\in \cl(A \cap B)$. Thus, every neighbourhood of $x$ contains at least one point of $A \cap B$.
        This point is in both $A$ and $B$.
        This means that every neighbourhood of $x$ contains a point in $A$, so $x$ is a closure point of $A$. Similarly, every neighbourhood
        of $x$ contains a point in $B$, so $x$ is a closure point of $B$. Thus, $x \in \cl(A)$ and $x \in \cl(B)$, so $x \in \cl(A) \cap \cl(B)$.
        This proves that $\cl(A \cap B) \subseteq \cl(A) \cap \cl(B)$. \\

        It is not always true that $\cl(A \cap B) = \cl(A) \cap \cl(B)$. For example, let $A = (0, 1)$ and $B = (1, 2)$ be subsets of $\mathbb{R}$
        with the usual topology. Then $\cl(A) = [0, 1]$, $\cl(B) = [1, 2]$ and $\cl(A \cap B) = \cl(\emptyset) = \emptyset$,
        while $\cl(A) \cap \cl(B) = \{1\}$.

        \problem Let $(M, d)$ be a metric space and $A, B$ be two subsets of $M$. Show that $\cl(A \cup B) = \cl(A) \cup \cl(B)$.
        
        \solution First, let $x \in \cl(A)$. Then, $x$ is a closure point of $A$, so every neighbourhood of $x$ contains
        some point in $A$, which is also a point in $A \cup B$. Thus, $x$ is a closure point of $A \cup B$, so $\cl(A) \subseteq \cl(A \cup B)$.
        A similar argument with $B$ shows that $\cl(B) \subseteq \cl(A \cup B)$. Thus, $\cl(A) \cup \cl(B) \subseteq \cl(A \cup B)$. \\

        Now, let $x \in \cl(A \cup B)$. Let $(A \cup B)'$ denote the set of limit points of $A \cup B$.
        We thus have $\cl(A \cup B) = A \cup B \cup (A \cup B)'$.
        If $x \in A \cup B$, then we are done, since in that case, $x$ is in one of $A$ or $B$, which are included in their closures,
        so either $x \in \cl(A)$ or $x \in \cl(B)$.

        Otherwise, $x \in (A \cup B)'$, i.e.\ $x$ is a limit point of $ A\cup B$. 
        In this case, every neighbourhood of $x$ contains infinitely many elements of $A \cup B$.
        Consider the neighbourhoods $N_{\epsilon_n}(x)$ of size $\epsilon_n = 1 /n > 0$.
        In each of these neighbourhoods, we find a point $y_n \in A \cup B$.
        Suppose that $x \notin \cl(A)$. This means that for some $k \in \mathbb{N}$, the $\epsilon_k$ neighbourhood of
        $x$ contains no points in $A$. Thus, $y_k \in B$.
        However, since $N_{\epsilon_{n + 1}} \subseteq N_{\epsilon_{n}}$ for all $n \in \mathbb{N}$, this forces $y_{k + 1} \in B$.
        By induction, all $y_{n \geq k} \in B$, so $x \in \cl(B)$. This follows because every neighbourhood larger than $\epsilon_k$
        contains $y_k$, and for all smaller $\epsilon$ neighbourhoods, there exists $n \in \mathbb{N}$ such that 
        $1 /k = \epsilon_k \geq \epsilon > 1 /n$, so $N_{\epsilon} \supseteq N_{\epsilon_n}$ which contains $y_n \in B$, since $n \geq k$.
        Thus, in all cases, $x \in \cl(A) \cup \cl(B)$, so $\cl(A\cup B) \subseteq \cl(A) \cup \cl(B)$. \\

        Hence, both inclusions prove that $\cl(A\cup B) = \cl(A) \cup \cl(B)$.



\end{document}
