\documentclass[10pt]{article}

\usepackage[T1]{fontenc}
\usepackage{geometry}
\usepackage{amsmath, amssymb, amsthm}
\usepackage{bm}

\title{Analysis I - Assignment II}
\author{Satvik Saha}
\date{}

\geometry{a4paper, margin=1in}
\setlength\parindent{0pt}
\renewcommand{\labelenumi}{(\alph{enumi})}
% \renewcommand\qedsymbol{$\blacksquare$}
\newcounter{prob}
\def\problem{\stepcounter{prob}\paragraph{Exercise \arabic{prob}}}
\def\solution{\paragraph{Solution}}

\begin{document}
        \par\textbf{IISER Kolkata} \hfill \textbf{Assignment II}
        \vspace{3pt}
        \hrule
        \vspace{3pt}
        \begin{center}
                \LARGE{\textbf{MA 2101 : Analysis I}}
        \end{center}
        \vspace{3pt}
        \hrule
        \vspace{3pt}
        Satvik Saha, \texttt{19MS154}, Group C\hfill\today
        \vspace{20pt}

        \problem Show that for every real number $r$, there exists an integer such that $n \leq r < n + 1$.
        \solution Supoose that there is no $n \in \mathbb{Z}$ such that $n \leq r < n + 1$ for some $r \in \mathbb{R}$.
        Note that the integers are unbounded below, so there exists some $m \in \mathbb{Z}$ such that $m \leq r$.
        By our assumption, we cannot have $r < m + 1$, so instead $m + 1 \leq r$. Let this be our base case.

        Now, let $k \in \mathbb{Z}$ be such that $m \leq k \leq r$. Again, $r < k + 1$ would contradict our assumption, so $k + 1 \leq r$.

        Thus, we have shown by induction that all integers $n \geq m$ are bounded above by $r$. Additionally, the other integers $n' < m < r$
        anyways. Thus, $n \leq r$ for all $n \in \mathbb{Z}$, which is absurd since the integers are unbounded above.
        This proves the given statement.

        \problem Show that between any two rational numbers, there exists an irrational number.
        \solution Without loss of generality, let $p, q \in \mathbb{Q}$ such that $p > q$.
        Note that $2 > \sqrt{2} > 1$, so $0 < 1 /\sqrt{2} < 1$ and $0 < (p - q) /\sqrt{2} < p - q$.
        Adding $q$ to both sides,
        \[
                q < q \,+\, \frac{p - q}{\sqrt{2}} < p.
        \]
        Note that the irrationality of $q + (p - q)/\sqrt{2}$ follows directly from the irrationality of $\sqrt{2}$.

        \problem Show that in a group, every element has a unique inverse.
        \solution Let $(G, *)$ be a group with identity $e \in G$, and let $a \in G$ be arbitrary. Clearly, $a$ must have an inverse
        in $G$. Suppose $a', a'' \in G$ are two such inverses. Thus,
        \[
                a' * a = e = a * a', \quad\text{ and }\quad a'' * a = e = a * a''.
        \]
        Now, we evaluate
        \begin{align*}
                a' \;&=\; a' * e \tag{Identity} \\
                        \;&=\; a' * (a * a'') \tag{Composition with inverse} \\
                        \;&=\; (a' * a) * a'' \tag{Associativity} \\
                        \;&=\; e * a''          \tag{Composition with inverse} \\
                        \;&=\; a''      \tag{Identity}
        \end{align*}
        Thus, $a' = a''$ for all inverses of $a$. In other words, the inverse of $a$ is unique.

        \problem Let $T \subset \mathbb{R}$ be bounded and let $S = \{|x - y| : x, y \in T\}$. Show that $\sup S = \sup T - \inf T$.
        \solution We assume that $T$ is non-empty.
        Note that $T$ is a bounded subset of $\mathbb{R}$, so $\sup T$ and $\inf T$ exist by the completeness of $\mathbb{R}$.
        Without loss of generality, let $x, y \in T$ such that $x \geq y$. Then, $|x - y| \leq x - y \leq \sup T - \inf T$,
        since $x \leq \sup T$ and $y \geq \inf T$\footnote{The analogous case with $x < y$ shows that $|x - y| = -x + y \leq -\inf T + \sup T$.}.
        Hence, $S$ is a subset of $\mathbb{R}$ bounded above, so $\sup S$ exists. We claim that $\sup S = \sup T - \inf T$.
        Thus, for any $\epsilon > 0$, we must find $s \in S$ such that $\sup T - \inf T - \epsilon < s \leq \sup T - \inf T$. \\
        Now, from the properties of the supremum and infinum, we choose $x', y' \in T$ such that $\sup T - \epsilon /2 < x' \leq \sup T$
        and $\inf T \leq y' < \inf T + \epsilon /2$. Thus, $x' - y' > \sup T - \inf T - \epsilon$. Thus,
        without loss of generality\footnote{If $y' > x'$, we can simply swap the roles of $x'$ and $y'$, since
        $\sup T - \epsilon /2 < x' < y' \leq \sup T$ and $\inf T \leq x' < y' < \inf T + \epsilon /2$.}, we have $ s = |x' - y'| \in S$
        and $\sup T - \inf T - \epsilon < s \leq \sup T - \inf T$. Thus, $\sup T - \inf T$ is indeed the least upper bound of $S$,
        and is thus equal to its supremum.

        \problem Find the supremum and infimum of the set $S = \{m/(m + n) : m, n \in \mathbb{N}\}$.
        \solution We claim that $\inf S = 0$ and $\sup S = 1$. First, note that 
        \[
                0 \;<\; \frac{m}{m + n} \;<\; \frac{m}{m} = 1,
        \]
        for all $m, n \in \mathbb{N}$. Thus, $S$ is bounded, so its supremum and infimum exist by the completeness of $\mathbb{R}$.
        Also, we must have $\sup S \leq 1$ and $\inf S \geq 0$. We must now show that for any upper bound $1 > \alpha \in \mathbb{R}$
        and for any lower bound $0 < \beta \in \mathbb{R}$ of $S$, there exist $x, y \in S$ such that
        \[
                0 \;<\; x \;<\; \beta, \quad\text{ and }\quad \alpha \;<\; y \;<\; 1.
        \]
        Clearly, $1 /2 = 1 /(1 + 1) \in S$, so $\beta < 1 /2 < 1$ and $\alpha > 1 /2 > 0$.

        On the other hand, the rationals $\mathbb{Q}$ are dense in the reals, so between any two real numbers, there exists a rational
        number $p /q$ for $p, q \in \mathbb{Z}$, $q \neq 0$. Thus, we find rationals $0 < a /b < \beta < 1$ and $0 < \alpha < c /d < 1$,
        so $0 < a < b$ and $0 < c < d$ for $a, b, c, d \in \mathbb{N}$. Thus, $0 < b - a \in \mathbb{N}$ and $0 < d - c \in \mathbb{N}$.
        We thus set
        \[
                x = \frac{a}{a + (b - a)} \in S, \quad\quad\quad y = \frac{c}{c + (d - c)} \in S,
        \]
        completing the proof.
\end{document}
