\documentclass[10pt]{article}

\usepackage[T1]{fontenc}
\usepackage{geometry}
\usepackage{amsmath, amssymb, amsthm}
\usepackage{bm}

\title{Analysis I - Assignment VII}
\author{Satvik Saha}
\date{}

\geometry{a4paper, margin=1in}
\setlength\parindent{0pt}
\renewcommand{\labelenumi}{(\alph{enumi})}
% \renewcommand\qedsymbol{$\blacksquare$}
\newcounter{prob}
\def\problem{\stepcounter{prob}\paragraph{Exercise \arabic{prob}}}
\def\solution{\paragraph{Solution}}
% \def\cl{\operatorname{cl}}
\newcommand\cl[1]{\overline{#1}}
% \def\int{\operatorname{int}}
\newcommand\inte[1]{{#1}^\circ}
\newcommand\norm[1]{\lVert #1 \rVert}
\def\O{\mathcal{O}}
\def\C{\mathcal{C}}

\begin{document}
        \par\textbf{IISER Kolkata} \hfill \textbf{Assignment VII}
        \vspace{3pt}
        \hrule
        \vspace{3pt}
        \begin{center}
                \LARGE{\textbf{MA 2101 : Analysis I}}
        \end{center}
        \vspace{3pt}
        \hrule
        \vspace{3pt}
        Satvik Saha, \texttt{19MS154}, Group C\hfill\today
        \vspace{20pt}

        \problem Show that if $\{\alpha_n\}$ is a Cauchy sequence in the Euclidean space $\mathbb{R}$, then so is $\{\alpha_n^2\}$.
        Does the converse also hold?.

        \solution Since $\{\alpha_n\}$ is a Cauchy sequence in $\mathbb{R}$, it converges to a real number $L$.
        We claim that $\alpha_n^2 \to L^2$, which in turn means that $\{\alpha_n^2\}$ is a Cauchy sequence\footnote{
                Cauchy sequences and convergent sequences are precisely the same in the Euclidean space $\mathbb{R}$.
        }. \\

        Let $\epsilon > 0$ be arbitrary.
        Note that the convergent sequence $\{\alpha_n\}$ must be bounded, i.e.\ $|\alpha_n| < M$ for some positive $M \in \mathbb{R}$.
        This means that for all $n \in \mathbb{N}$,
        \[
                |\alpha_n + L| \leq |\alpha_n| + |L| < M + |L|.
        \]
        Furthermore, we can choose $N \in \mathbb{N}$ such that for all $n \geq N$, 
        \[
                |\alpha_n - L| < \frac{\epsilon}{M + |L|}.
        \]
        Thus, for all $n \geq N$, we have
        \[
                |\alpha_n^2 - L^2| = |\alpha_n - L| |\alpha_n + L| < (M + |L|)\frac{\epsilon}{M + |L|} = \epsilon.
        \]
        This establishes that $\alpha_n^2 \to L^2$. \\

        The converse is not true. Consider the sequence defined by $\alpha_n = (-1)^n$. Clearly, the constant sequence $\{\alpha_n^2\} = \{1\}$
        converges to $1$ and hence is Cauchy, but the sequence $\{\alpha_n\}$ does not converge and hence is not Cauchy.

        \problem Let $(M, d)$ be a metric space and let $\{\alpha_n\}$ and $\{\beta_n\}$ be two sequences in $M$ such that
        \[
                \lim_{n \to \infty} d(\alpha_n, \beta_n) = 0.
        \]
        Prove the following statements.
        \begin{enumerate}
                \item The sequence $\{\alpha_n\}$ is convergent if and only if the sequence $\{\beta_n\}$ is convergent.
                \item The sequence $\{\alpha_n\}$ is a Cauchy sequence if and only if $\{\beta_n\}$ is a Cauchy sequence.
        \end{enumerate}

        \solution
        \begin{enumerate}
                \item Suppose $\alpha_n \to p$, where $p \in M$. This means that for all reals $r > 0$, there exists $N \in \mathbb{N}$
                such that $\alpha_n \in B_r(p)$ for all $n \geq N$. We claim that $\beta_n \to p$. \\

                Let $r > 0$ be arbitrary, and let $N_1 \in \mathbb{N}$ such that for all $n \geq N_1$, $\alpha_n \in B_{r /2}(p)$.
                Because $d(\alpha_n, \beta_n) \to 0$, we can choose $N_2 \in \mathbb{N}$ such that for all $n \geq N_2$,
                $d(\alpha_n, \beta_n) < r /2$. Thus, setting $N = N_1 + N_2$, we observe that for all $n \geq N$, the triangle inequality gives
                \[
                        d(\beta_n, p) \leq d(\beta_n, \alpha_n) + d(\alpha_n, p) < \frac{r}{2} + \frac{r}{2} = r.
                \]
                Thus, $\beta_n \in B_r(p)$ for all $n \geq N$, which proves that the sequence $\{\beta_n\}$ converges. \\

                The converse follows trivially by swapping the roles of $\{\alpha_n\}$ and $\{\beta_n\}$.

                \item Suppose $\{\alpha_n\}$ is a Cauchy sequence. This means that for all reals $r > 0$, there exists $N \in \mathbb{N}$
                such that $d(\alpha_m, \alpha_n) < r$ when both $m, n \geq N$. \\

                Let $r > 0$ be arbitrary, and let $N_1$ be such that for all $m, n \geq N_1$, $d(\alpha_m, \alpha_n) < r /3$.
                Because $d(\alpha_n, \beta_n) \to 0$, we can choose $N_2 \in \mathbb{N}$ such that for all $n \geq N_2$,
                $d(\alpha_n, \beta_n) < r /3$. Thus, whenever both $m, n \geq N_1 + N_2$, repeated application of the 
                triangle inequality gives
                \[
                        d(\beta_m, \beta_n) \leq d(\beta_m, \alpha_m) + d(\alpha_m, \alpha_n) + d(\alpha_n, \beta_n)
                                < \frac{r}{3} + \frac{r}{3} + \frac{r}{3} = r.
                \]
                This shows that $\{\beta_n\}$ is Cauchy. \\

                Again, the converse follows trivially by swapping the roles of $\{\alpha_n\}$ and $\{\beta_n\}$.
        \end{enumerate}

        \problem Let $\{\alpha_n\}$ be a sequence of positive real numbers such that 
        \[
                |\alpha_{n + 2} - \alpha_{n + 1}| < |\alpha_{n + 1} - \alpha_n|,
        \]
        for all $n \in \mathbb{N}$. Can we conclude from the above condition that the sequence $\{\alpha_n\}$ converges?

        \solution No. Consider the sequence defined by $\alpha_n = \sqrt{n}$. We see that
        \[
                \sqrt{n + 2} - \sqrt{n + 1} = \frac{1}{\sqrt{n + 2} + \sqrt{n + 1}} < \frac{1}{\sqrt{n} + \sqrt{n + 1}} = \sqrt{n + 1} - \sqrt{n}.
        \]
        Thus, the sequence $\{\alpha_n\}$ satisfies the given condition. However, the sequence is clearly unbounded in the reals, and
        hence cannot converge. \\

        Another counterexample is the sequence defined by
        \[
                \alpha_n = 1 + \frac{1}{2} + \dots + \frac{1}{n}.
        \]
        Clearly, 
        \[
                |\alpha_{n + 2} - \alpha_{n + 1}| = \frac{1}{n + 2} < \frac{1}{n + 1} = |\alpha_{n + 1} - \alpha_n|.
        \]
        On the other hand, the harmonic series diverges, so the sequence $\{\alpha_n\}$ does not converge.

\end{document}
