\documentclass[11pt]{article}

\usepackage[T1]{fontenc}
\usepackage{geometry}
\usepackage{amsmath, amssymb, amsthm}
\usepackage[scr]{rsfso}
\usepackage{bm}
\usepackage[%
    hidealllines=true,%
    innerbottommargin=15,%
    nobreak=true,%
]{mdframed}
\usepackage{xcolor}
\usepackage{graphicx}
\usepackage{fancyhdr}
\usepackage{hyperref}

\geometry{a4paper, margin=1in, headheight=14pt}

\pagestyle{fancy}
\fancyhf{}
\renewcommand\headrulewidth{0.4pt}
\fancyhead[L]{\scshape MA3203: Analysis IV}
\fancyhead[R]{\scshape \leftmark}
\rfoot{\footnotesize\it Updated on \today}
\cfoot{\thepage}

\newcommand{\C}{\mathbb{C}}
\newcommand{\R}{\mathbb{R}}
\newcommand{\Q}{\mathbb{Q}}
\newcommand{\Z}{\mathbb{Z}}
\newcommand{\N}{\mathbb{N}}

\newcommand{\M}{\mathcal{M}}
\newcommand{\MN}{\mathcal{N}}
\renewcommand{\L}{\mathcal{L}}

\newcommand{\ip}[2]{\langle #1, #2 \rangle}
\newcommand{\norm}[1]{\Vert #1 \Vert}
\renewcommand{\vec}[1]{\boldsymbol{#1}}

\newcommand{\vx}{\vec{x}}
\newcommand{\vy}{\vec{y}}
\newcommand{\vv}{\vec{v}}
\newcommand{\vw}{\vec{w}}
\newcommand{\ve}{\vec{e}}

\newcommand{\dd}[2]{\frac{d #1}{d #2}}
\newcommand{\pp}[2]{\frac{\partial #1}{\partial #2}}
\newcommand{\ddn}[3][]{\frac{d^{#1} #2}{d #3^{#1}}}
\newcommand{\ppn}[3][]{\frac{\partial^{#1} #2}{\partial #3^{#1}}}
\newcommand{\grad}{\nabla}

\newmdtheoremenv[%
    backgroundcolor=blue!10!white,%
]{theorem}{Theorem}[section]
\newmdtheoremenv[%
    backgroundcolor=violet!10!white,%
]{corollary}{Corollary}[theorem]
\newmdtheoremenv[%
    backgroundcolor=teal!10!white,%
]{lemma}[theorem]{Lemma}

\theoremstyle{definition}
\newmdtheoremenv[%
    backgroundcolor=green!10!white,%
]{definition}{Definition}[section]
\newmdtheoremenv[%
    backgroundcolor=red!10!white,%
]{exercise}{Exercise}[section]

\theoremstyle{remark}
\newtheorem*{remark}{Remark}
\newtheorem*{example}{Example}
\newtheorem*{solution}{Solution}

\surroundwithmdframed[%
    nobreak=false,%
    linecolor=black!20!white,%
    hidealllines=false,%
    innertopmargin=5,%
    innerbottommargin=10,%
    skipabove=0,%
    skipbelow=0,%
]{example}

\numberwithin{equation}{section}

\title{
    \Large\textsc{MA3203} \\
    \Huge \textbf{Analysis IV} \\
    \vspace{5pt}
    \Large{Spring 2022}
}
\author{
    \large Satvik Saha
    \\\textsc{\small 19MS154}
}
\date{\normalsize
    \textit{Indian Institute of Science Education and Research, Kolkata, \\
    Mohanpur, West Bengal, 741246, India.} \\
}

\begin{document}
    \maketitle

    \tableofcontents

    \section{Measure theory}

    \subsection{Introduction}
    Measure theory seeks to generalize the notions of \emph{length}, \emph{area},
    \emph{volume} to more general sets: this new notion is called a \emph{measure}.
    This also allows us to generalize the notion of Riemann integration to a broader
    class of functions.

    Recall that continuous functions, or at least functions with finitely many
    discontinuities on a closed interval are Riemann integrable. The Dirichlet
    function, which is discontinuous everywhere, is not. \[
        f\colon \R \to \R, \qquad x \mapsto \begin{cases}
            1, &\text{ if } x \in \Q, \\
            0, &\text{ if } x \notin \Q.
        \end{cases}
    \] This is simply because every non-empty interval contains at least one rational
    and one irrational number, so the Darboux lower sum is always 0 and the upper sum
    is always 1 regardless of the choice of partition.

    On the other hand, if we had to assign a particular value to this integral,
    intuition tells us that it ought to be zero. After all, the function $f$ attains
    a non-zero value only on the countable set $\Q \cap [0, 1]$; it is zero almost
    everywhere. Formally, we will show that $f$ is non-zero on a set of zero
    Lebesgue measure, which will allow us to set this Lebesgue integral to zero. We
    will see that with this new formulation of integration, we end up partitioning
    the range of $f$ rather than it's domain, and write \[
        \int f = 0\cdot \mu([0, 1]\setminus \Q) + 1\cdot \mu([0, 1] \cap \Q) = 0.
    \] 

    \begin{theorem}[Lebesgue criterion]
        A function $f\colon [a, b] \to \R$ is Riemann integrable if and only if it is
        bounded and its set of discontinuities has Lebesgue measure zero. This means
        that the set of discontinuities of $f$ must be coverable by countably many
        intervals $(x_i, y_i)$ such that the sum of lengths $y_i - x_i$ can be made
        arbitrarily small.
    \end{theorem}
    
    \subsection{Basic definitions}

    \begin{definition}
        Let $X$ be a set, and let $\M$ be a collection of subsets of $X$. We say
        that $\M$ is a $\sigma$-algebra over $X$ if it satisfies the following.
        \begin{enumerate}
            \itemsep0em
            \item $\M$ contains $X$.
            \item $\M$ is closed under complementation.
            \item $\M$ is closed under countable unions.
        \end{enumerate}
        \begin{remark}
            The first condition can be replaced by forcing $\M$ to be non-empty.
        \end{remark}
        \begin{remark}
            The following properties follow immediately.
            \begin{enumerate}
                \itemsep0em
                \item $\M$ contains $\emptyset$.
                \item $\M$ is closed under countable intersections.
                \item $\M$ is closed under differences.
            \end{enumerate}
        \end{remark}
    \end{definition}
    
    \begin{example}
        Given a set $X$, its power set forms a $\sigma$-algebra over $X$.
    \end{example}
    \begin{example}
        Given a set $X$, the set $\{\emptyset, X\}$ forms a $\sigma$-algebra over
        $X$.
    \end{example}
    \begin{example}
        Given an uncountable set $X$, the following set forms the co-countable
        $\sigma$-algebra over $X$. \[
            \{E \subseteq X : E\text{ is countable or } E^c\text{ is countable}\}.
        \] 
    \end{example}


    \begin{definition}
        Let $X$ be a set, and let $\M$ be a $\sigma$-algebra over $X$. We say
        that a function $\mu\colon \M \to \R \cup \{-\infty, +\infty\}$ is called
        a measure if it satisfies the following.
        \begin{enumerate}
            \itemsep0em
            \item $\mu$ is non-negative.
            \item $\mu(\emptyset) = 0$.
            \item $\mu$ is additive over countable unions of disjoint sets, i.e.\ for
            any countable collection $\{E_i\}_{i = 1}^\infty$ such that $E_i \cap E_j
            = \emptyset$ for all pairs, we have \[
                \mu\left(\bigcup_{i = 1}^\infty E_i\right) = \sum_{i = 1}^\infty
                \mu(E_i).
            \] 
        \end{enumerate}
    \end{definition}

    \begin{example}
        The trivial zero measure sends every set to zero.
    \end{example}
    \begin{example}
        In probability theory, we look at the event space $\mathcal{E}$ as a
        $\sigma$-algebra over the sample space $\Omega$. The probability function $P$
        is a measure on this event space such that $P(\Omega) = 1$.
    \end{example}
    \begin{example}
        Let $X$ be a set, and let $\M$ be its power set as a $\sigma$-algebra
        over $X$. Fix $x_0 \in X$, and define \[
            \mu\colon \M \to [0, \infty], \qquad E \mapsto \begin{cases}
                1, &\text{ if } x_0 \in E, \\
                0, &\text{ if } x_0 \notin E. \\
            \end{cases}
        \] This is called the Dirac measure.
    \end{example}

    \begin{example}
        Let $X$ be a set, and let $\M$ be its power set as a $\sigma$-algebra
        over $X$. Define \[
            \mu\colon \M \to [0, \infty], \qquad E \mapsto \begin{cases}
                0, &\text{ if } E = \emptyset, \\
                |E|, &\text{ if } E\text{ is finite},\\
                \infty, &\text{ otherwise}.
            \end{cases}
        \] This is called the counting measure.
    \end{example}

    \begin{example}
        Let $X$ be an uncountable set, and let $\M$ be its co-countable sigma
        algebra. Define \[
            \mu\colon \M \to [0, \infty], \qquad E \mapsto \begin{cases}
                0, &\text{ if } E \text{ is countable},\\
                1, &\text{ if } E^c \text{ is countable}.\\
            \end{cases}
        \] 
    \end{example}


    \begin{definition}
        Let $(X, \M, \mu)$ be a measure space. \begin{enumerate}
            \itemsep0em
            \item We say that $\mu$ is finite if $\mu(E)$ is finite for all $E \in
            \M$.
            \item We say that $\mu$ is $\sigma$-finite if given $E \in \M$, we
            can write \[
                E = \bigcup_{i = 1}^\infty E_i
            \] for $E_i \in \M$ such that each $\mu(E_i)$ is finite.
        \end{enumerate}
    \end{definition}


    \subsection{Basic properties}

    \begin{lemma}
        Let $(X, \M, \mu)$ be a measure space. Then, the following properties
        hold. \begin{enumerate}
            \itemsep0em
            \item If $A, B \in \M$ such that $A \subseteq B$, then $\mu(A) \leq
            \mu(B)$.
            \item If $A, B \in \M$ such that $A \subseteq B$ and $\mu(A)$ is
            finite, then $\mu(B - A) = \mu(B) - \mu(A)$.
            \item If $\{E_i\}_{i = 1}^\infty$ such that $E_i \in \M$, then \[
                \mu\left(\bigcup_{i = 1}^\infty E_i\right) \leq \sum_{i = 1}^\infty
                \mu(E_i).
            \] 
        \end{enumerate}
    \end{lemma}

    \begin{corollary}
        A measure $\mu$ is finite if and only if $\mu(X)$ is finite.
    \end{corollary}

    \begin{theorem}[Continuity from below]
        Let $(X, \M, \mu)$ be a measure space, and let $\{E_i\}_{i = 1}^\infty$
        be a sequence of measurable sets such that $E_i \subseteq E_j$ for all $i <
        j$. Then, \[
            \mu\left(\bigcup_{i = 1}^\infty E_i\right) = \lim_{n \to \infty}
            \mu(E_n).
        \] 
    \end{theorem}
    \begin{proof}
        Define $F_i = E_i - E_{i - 1}$, denoting $E_0 = \emptyset$. Thus, \[
            \mu\left(\bigcup_{i = 1}^\infty E_i\right) = \mu\left(\bigcup_{i =
            1}^\infty F_i\right) = \sum_{i = 1}^\infty \mu(F_i).
        \] Also note that \[
            \sum_{i = 1}^n \mu(F_i) = \mu\left(\bigcup_{i = 1}^\infty F_n\right) =
            \mu(E_n).
        \] Since the infinite sum in the first part is the limit of partial sums, we
        have our result.
    \end{proof}
    
    \begin{theorem}[Continuity from above]
        Let $(X, \M, \mu)$ be a measure space, and let $\{E_i\}_{i = 1}^\infty$
        be a sequence of measurable sets such that $E_i \supseteq E_j$ for all $i <
        j$. Further assume that $\mu(E_1)$ is finite. Then, \[
            \mu\left(\bigcap_{i = 1}^\infty E_i\right) = \lim_{n \to \infty}
            \mu(E_n).
        \] 
    \end{theorem}
    \begin{proof}
        Define $F_i = E_1 - E_n$, and note that $F_i \subseteq F_j$ for all $i < j$.
        Thus, \[
            \mu\left(\bigcup_{i = 1}^\infty F_i\right) = \lim_{n \to \infty}
            \mu(F_n).
        \] This can be rewritten as \[
            \mu\left(E_1 - \bigcap_{i = 1}^\infty E_i\right) = \lim_{n \to \infty}
            \mu(E_1 - E_n).
        \] Using the subtractive property and the fact that each $\mu(E_i)$ is
        finite, \[
            \mu(E_1) - \mu\left(\bigcap_{i = 1}^\infty E_i\right) = \lim_{n \to \infty}
            \mu(E_1) - \mu(E_n).
        \] Pulling the constant $\mu(E_1)$ out from the limit and subtracting from
        both sides gives our result.
    \end{proof}
    \begin{example}
        Consider the counting measure $\mu$ on $(\N, \mathcal{P}(\N))$, and define
        $E_n = \{n, n + 1, \dots \}$. Then, \[
            \mu\left(\bigcap_{i = 1}^\infty E_i\right) = 0, \qquad \lim_{n \to
            \infty} \mu(E_n) = \infty.
        \] 
    \end{example}


    \subsection{The Borel $\sigma$-algebra}

    \begin{theorem}
        Let $X$ be a set, and let $S$ be a collection of subsets of $X$. Then, there
        exists a smallest $\sigma$-algebra containing $S$. This is called the
        $\sigma$ algebra generated by $S$, denoted $\M(S)$.
    \end{theorem}
    \begin{proof}
        Let $\Omega$ be the collection of all $\sigma$-algebras on $X$ containing
        $S$. Note that $\Omega \neq \emptyset$, since it contains the power set of
        $X$. Consider the intersection of all the sigma algebras in $\Omega$, \[
            \M = \bigcap_{\M_\lambda \in \Omega} \M_\lambda.
        \] We claim that $\M$ is indeed a $\sigma$-algebra. To see this, first note
        that $X \in \M$. Next, pick $E \in \M \subseteq \M_\lambda$, so $E^c \in
        \M_\lambda$ for all $\M_\lambda \in \Omega$, hence $E^c \in \M$. Finally,
        pick $\{E_i\}_{i = 1}^\infty$ where $E_i \in \M \subseteq \M_\lambda$, which
        shows that the union of these $E_i$ is in every $\M_\lambda$, hence in $\M$.
    \end{proof}

    \begin{definition}
        Let $(X, \tau)$ be a topological space. The $\sigma$ algebra generated by
        $\tau$ is called the Borel $\sigma$-algebra, $\mathcal{B}_X = \M(\tau)$.
        \begin{remark}
            The Borel $\sigma$-algebra $\mathcal{B}_X$ contains all open as well as
            all closed sets in $X$, as well as their countable unions and
            intersections.
        \end{remark}
    \end{definition}

    \begin{theorem}
        Consider the collection $\beta$ of open intervals in $\R$, and the standard
        topology $\tau$ on $\R$. Then, both $\beta$ and $\tau$ generate the same
        Borel $\sigma$-algebra $\mathcal{B}_\R$.
    \end{theorem}
    \begin{proof}
        This relies on the fact that every open set $U \subseteq \R$ can be written
        as a countable union of open intervals. To see this, pick $x \in U$, and an
        open interval $x \in (a, b) \subset U$. Now pick $p, q \in \Q$ such that $a <
        p < x < q < b$, hence $x \in (p, q) \subset U$. Now, $U$ is precisely the
        union of all such intervals $(p, q)$. This collection is countable, due to
        the countability of the rationals.
    \end{proof}
    \begin{remark}
        The same holds if we consider the collection $\beta'$ of closed intervals in
        $\R$. This can be shown using the standard trick \[
            (a, b) = \bigcup_{n = 1}^\infty \left[a + \frac{1}{n}, b -
            \frac{1}{n}\right].
        \] Indeed, we may also consider the collection of intervals of the form $[a,
        b)$, or the collection of intervals $(a, b]$, or even the collection of
        intervals $(a, \infty)$, or $(-\infty, b)$, or $[a, \infty)$, or $(-\infty,
        b]$.
    \end{remark}

    \begin{definition}
        A countable union of closed sets is called an $F_\sigma$ set. A countable
        intersection of open sets is called a $G_\delta$ set.
    \end{definition}

    
    \subsection{Measurable functions}

    \begin{definition}
        Let $(X, \M_X)$, $(Y, \M_Y)$ be measure spaces. We say that a function
        $f\colon X \to Y$ is $(\M_X, \M_Y)$ measurable if for every $E \in \M_Y$, we
        have $f^{-1}(E) = \M_X$.
    \end{definition}
    \begin{example}
        Consider the Borel $\sigma$-algebra $\mathcal{B}_\R$ on $\R$, and fix $E \in
        \mathcal{B}_\R$. Define the characteristic function \[
            \chi_E\colon \R \to \R, \qquad x \mapsto \begin{cases}
                1, &\text{ if } x \in E, \\
                0, &\text{ if } x \notin E.
            \end{cases}
        \] Then, $\chi_E$ is measurable. However, we can choose $E$ to be closed and
        not open, so that $\chi_E$ is not continuous.
    \end{example}

    \begin{lemma}
        Let $f\colon (X, \M_X) \to (Y, \M_Y)$, and let $\M_Y$ be generated by $S$.
        Then $f\colon X\to Y$ is measurable if for every $E \in S$, we have
        $f^{-1}(E) \in \M_X$.
    \end{lemma}
    \begin{proof}
        Define \[
            \M = \{E \subseteq Y: f^{-1}(E) \in \M_X\}.
        \] Clearly, $S \subseteq \M$. We now claim that $\M$ is a $\sigma$-algebra
        over $Y$. First, $Y \in \M$ since $f^{-1}(Y) = X \in \M_X$. Next if $E \in
        \M$, we have $f^{-1}(E^c) = f^{-1}(E)^c \in \M_X$. Finally, if $\{E_i\}_{i =
        1}^\infty$ such that each $E_i \in \M$, set $E$ to be their union, whence \[
            f^{-1}(E) = \bigcup_{i = 1}^\infty f^{-1}(E_i) \in \M.
        \] Thus, $\M$ is indeed a $\sigma$-algebra. Since $S$ generates $\M_Y$, we
        have $\M_Y \subseteq \M$, completing the proof.
    \end{proof}
    \begin{example}
        A function $f\colon X \to \R$ is measurable if and only if $f^{-1}((a,
        \infty))$ is measurable for all $a \in \R$.
    \end{example}


    \begin{theorem}
        Let $f\colon X \to Y$ be continuous. Then, $f$ is $(\mathcal{B}_X,
        \mathcal{B}_Y)$ measurable.
    \end{theorem}

    \begin{lemma}
        The composition of measurable functions is measurable. In other words, if
        $f\colon (X, \M_X) \to (Y, \M_Y)$ is surjective and $(\M_X, \M_Y)$
        measurable, and $g\colon (Y, \M_Y) \to (Z, \M_Z)$ is $(\M_Y, \M_Z)$
        measurable, then $g\circ f$ is $(\M_X, \M_Z)$ measurable.
    \end{lemma}

    \begin{lemma}
        Let $u, v\colon (X, \M) \to (\R, \mathcal{B}_\R)$ be $(\M, \mathcal{B}_\R)$
        measurable. Then, $f\colon (X, \M) \to (\R^2, \mathcal{B}_{\R^2})$, defined
        by $x \mapsto (u(x), v(x))$, is $(\M, \mathcal{B}_{\R^2})$ measurable.
    \end{lemma}
    \begin{proof}
        Basic open sets in $\R^2$ can be chosen as the open rectangles $(a, b) \times
        (c, d)$. The pre-image of such an open set under $f$ is $u^{-1}((a, b)) \cap
        v^{-1}((c, d))$, which is clearly a measurable set in $X$.
    \end{proof}
    
    \begin{corollary}
        Let $f, g\colon (X, \M) \to (\R, \mathcal{B}_\R)$ be $(\M, \mathcal{B}_\R)$
        measurable. Then the sum $f + g$ and the product $f\cdot g$ are also $(\M,
        \mathcal{B}_\R)$ measurable.
    \end{corollary}
    \begin{proof}
        The maps $(x, y) \mapsto x + y$ and $(x, y) \mapsto xy$ are continuous, hence
        measurable. Thus, the composite maps $x \mapsto (f(x), g(x)) \mapsto f(x) +
        g(x)$ and $x \mapsto (f(x), g(x)) \mapsto f(x)g(x)$ are also measurable.
    \end{proof}

    \begin{example}
        Let $(X, \M)$ be a measurable space, and let $A_1, \dots, A_n \in \M$. Then
        the map \[
            s\colon X \to \R, \qquad x \mapsto \sum_{i = 1}^n c_i \chi_{A_i}(x)
        \] is measurable. Such functions are called simple functions.
    \end{example}
    
    \begin{lemma}
        The maximum and minimum of measurable functions are measurable.
    \end{lemma}
    \begin{corollary}
        The positive and negative parts of a measurable function are measurable.
        \begin{remark}
            Recall that \[
                f^+ = \max\{f, 0\}, \qquad f^{-1} = -\min\{f, 0\}, \qquad f = f^+ -
                f^-.
            \] Thus, in order to show that a result holds for all measurable
            functions, it suffices to show the result only for all non-negative
            measurable functions.
        \end{remark}
    \end{corollary}
    

    \begin{theorem}
        Let $\{f_n\}_{n = 1}^\infty$ be a collection of measurable functions
        $f_n\colon X \to \R \cup \{-\infty, +\infty\}$. Then, their supremum and
        infimum are measurable.
    \end{theorem}
    
    \begin{theorem}
        Let $\{f_n\}_{n = 1}^\infty$ be a sequence of measurable functions $f_n\colon
        X \to \R$. Then, $\limsup_{n \to \infty} f_n$ and $\liminf_{n \to \infty}
        f_n$ are measurable.
    \end{theorem}
    
    \begin{theorem}
        Let $\{f_n\}_{n = 1}^\infty$ be a sequence of measurable functions $f_n\colon
        X \to \R$, and let $f_n \to f$ pointwise on $X$. Then, $f$ is measurable.
        \begin{remark}
            This is a stronger result than the corresponding one regarding limits of
            continuous functions.
        \end{remark}
    \end{theorem}

    

    \subsection{Lebesgue integration}

    \begin{theorem}
        Let $f\colon X \to [0, \infty]$ be measurable. Then, there exists a sequence
        of simple functions $s_n\colon X \to [0, \infty)$ such that \[
            0 \leq s_1 \leq s_2 \leq \dots \leq s_n \leq f
        \] for all $n \in \N$, and $s_n \to f$.
    \end{theorem}
    \begin{corollary}
        Any measurable function $f\colon X \to [-\infty, \infty]$ can be written as
        the limit of a sequence of simple functions $\{s_n\}_{n = 1}^\infty$, with
        $s_n \to f$.
    \end{corollary}

    \begin{definition}
        When dealing with the extended reals in measure theory, we use the convention
        $0\cdot \infty = \infty \cdot 0 = 0$.
        \begin{remark}
            We want to have \[
                \int_0^\infty 0 = 0\cdot \mu(f^{-1}(0)) = 0\cdot \mu([0, \infty)) =
                0.
            \] 
        \end{remark}
    \end{definition}

    
    \begin{definition}
        Let $(X, \M, \mu)$ be a measure space, and let $s\colon X \to [0, \infty)$ be
        a simple, measurable function, of the form \[
            s = \sum_{i = 1}^n c_i\chi_{A_i}, \qquad A_i \in \M
        \] where $c_1, \dots, c_n$ are distinct values of $s$. Then, the Lebesgue
        integral of $s$ on $E \in \M$ is defined as \[
            \int_E s\:d\mu = \sum_{i = 1}^n c_i\cdot \mu(E \cap A_i).
        \] 
    \end{definition}
    \begin{example}
        The Dirichlet function is the simple function $\chi_\Q$. Thus, upon assigning
        a $\sigma$-algebra and a measure $\mu$ on $\R$, we will be able to assign its
        Lebesgue integral on $\R$ as the value $\mu(\Q)$.
    \end{example}
    
    \begin{definition}
        Let $(X, \M, \mu)$ be a measure space, and let $f\colon X \to [0, \infty)$ be
        a measurable function. Then, the Lebesgue integral of $f$ on $E \in \M$ is
        defined as \[ 
            \int_E f\:d\mu = \sup\left\{\int_E s\:d\mu, \text{ for all simple
            functions }s, \text{where }0 \leq s \leq f\right\}.
        \] 
    \end{definition}

    \begin{theorem}
        Let $f, g\colon X \to [0, \infty]$ be measurable functions.
        \begin{enumerate}
            \item If $0 \leq f \leq g$, then \[
                \int_E f\:d\mu \leq \int_E g\:d\mu.
            \] 
            \item If $A \subset B$, then \[
                \int_A f\:d\mu \leq \int_B f\:d\mu.
            \] 
            \item For $c \in \R$, \[
                \int_E cf\:d\mu = c\int_E f\:d\mu.
            \] 
            \item If $f = 0$, then \[
                \int_E f\:d\mu = 0.
            \] 
            \item If $\mu(E) = 0$, then \[
                \int_E f\:d\mu = 0.
            \] 
        \end{enumerate}
    \end{theorem}

    \begin{definition}
        We say that a statement is true \emph{almost everywhere} on $X$ if it is true
        everywhere on $X\setminus E$ for a measure zero set $E$.
    \end{definition}
    
    \begin{lemma}
        Let $f\colon X \to [0, \infty]$ be a measurable function. If \[
            \int_X f\:d\mu = 0,
        \] then $f = 0$ almost everywhere.
    \end{lemma}
    \begin{proof}
        We wish to show that the set $E = f^{-1}(0, \infty]$ has measure zero. Now,
        note that this is the union of the measurable sets $E_n = f^{-1}(1 / n,
        \infty]$. Since $E_n \subset E \subseteq X$, we have \[
            \int_{E_n} \frac{1}{n}\:d\mu \leq \int_{E_n} f\:d\mu \leq \int_X f\:d\mu.
        \] However, this is just \[
            0 \leq \frac{1}{n}\mu(E_n) \leq \int_{E_n} f\:d\mu \leq 0,
        \] hence each $\mu(E_n) = 0$. Continuity from below gives $\mu(E) = 0$.
    \end{proof}


    \begin{lemma}
        Let $(X, \M, \mu)$ be a measure space, and let $s\colon X \to \R$ be a
        non-negative simple function. Define \[
            \nu(E) = \int_E s\:d\mu, \qquad E \in \M.
        \] Then, $\nu$ is a measure on $\M$.
    \end{lemma}

    
    \subsubsection{Monotone convergence}

    \begin{theorem}[Monotone convergence]
        Let $\{f_n\}_{n = 1}^\infty$ be a sequence of non-negative measurable
        functions $f_n\colon X \to \R$, such that $f_n \leq f_{n + 1}$, and $f_n \to
        f$ pointwise. Then, \[
            \lim_{n \to \infty} \int_X f_n\:d\mu = \int_X f\:d\mu.
        \] 
    \end{theorem}

    \begin{lemma}
        Let $f,g\colon X \to \R$ be measurable functions. Then, $f + g$ is measurable.
    \end{lemma}

    \begin{theorem}
        Let $\{f_n\}_{n = 1}^\infty$ be a sequence of non-negative measurable
        functions $f_n\colon X \to \R$. Define \[
            f\colon X \to \R, \qquad f(x) = \sum_{n = 1}^\infty f_n(x).
        \] Then, $f$ is measurable and \[
            \int_X f\:d\mu = \sum_{n = 1}^\infty \int_X f_n\:d\mu.
        \] 
    \end{theorem}

    \begin{theorem}[Fatou]
        Let $\{f_n\}_{n = 1}^\infty$ be a sequence of non-negative measurable
        functions $f_n\colon X \to \R$. Then, \[
            \int_X \liminf_{n \to \infty} f_n\:d\mu \leq \liminf_{n\to \infty} \int_X
            f_n\:d\mu.
        \] 
    \end{theorem}
    \begin{corollary}
        If $f_n \to f$, then \[
            \int_X f\:d\mu \leq \liminf_{n\to \infty} \int_X f_n\:d\mu.
        \] 
    \end{corollary}
    \begin{corollary}
        If $f_n \to f$ and each $0 \leq f_n \leq f$, then \[
            \int_X f\:d\mu = \liminf_{n\to \infty} \int_X f_n\:d\mu.
        \] 
    \end{corollary}
    \begin{example}
        Consider the functions $f_n = \chi_{[n, n + 1)}$. Now, $f_n \to 0$ pointwise,
        but the Lebesgue integrals of $f_n$ are all $1$.
    \end{example}

    \begin{lemma}
        Let $\{a_{ij}\}_{i,j \in \N}$ be sequences of non-negative terms. Then, \[
            \sum_i\sum_j a_{ij} = \sum_j\sum_i a_{ij}.
        \] 
    \end{lemma}
    \begin{proof}
        Define $f_i(j) = a_{ij}$, and $f = \sum_i f_i$. Using the counting measure, \[
            \int_\N f_i\:d\mu = \sum_j a_{ij}.
        \] Also, \[
            f(j) = \sum_i a_{ij}, \qquad
            \sum_j\sum_i a_{ij} = \sum_j f(j) = 
            \int_\N f\:d\mu = \sum_i\int_\N f_i\:d\mu = \sum_i\sum_j a_{ij}. \qedhere
        \] 
    \end{proof}


    \begin{theorem}
        Let $f\colon X \to [0, \infty]$ be a non-negative measurable function. Define
        \[
            \nu(E) = \int_E f\:d\mu, \qquad E \in \M.
        \] Then, $\nu$ is a measure on $\M$. Furthermore, if $g\colon X \to [0,
        \infty]$ is measurable, then \[
            \int_X g\:d\nu = \int_X g\,f\:d\mu.
        \] 
    \end{theorem}

    
    \subsubsection{Dominated convergence}

    \begin{definition}
        For a measurable function $f\colon X \to [-\infty, \infty]$, we may define \[
            \int_X f\:d\mu = \int_X f^+\:d\mu - \int_X f^-\:d\mu,
        \] as long as at least one of these terms is finite.
    \end{definition}

    \begin{definition}
        Let \[
            L^1(\mu) = \{f\colon X \to \C : f\text{ is measurable, and }\int_X
            |f|\:d\mu \text{ is finite}\}.
        \] For $f \in L^1(\mu)$, we may write $f = u + iv$ where $u, v$ are real
        valued measurable functions and define \[
            \int_X f\:d\mu = \int_X u\:d\mu + i\int_X v\:d\mu.
        \]
    \end{definition}

    \begin{lemma}
        When $f, g \in L^1(\mu)$, $\alpha, \beta \in \C$, we have \[
            \int_X \alpha f + \beta g \:d\mu = \alpha\int_X f\:d\mu + \beta\int_X
            g\:d\mu.
        \] \[
            \left|\int_X f \:d\mu \right| \leq \int_X |f| \:d\mu.
        \] 
    \end{lemma}
    
    \begin{theorem}
        The space of functions $L^1(\mu)$ is a vector space over $\C$.
        The map \[
            T\colon L^1(\mu) \to \C, \qquad f \mapsto \int_X f\:d\mu
        \] is a linear map. Furthermore, $L^1(\mu)$ is a metric space, with \[
            d(f, g) = \int_X |f - g|\:d\mu.
        \] 
        \begin{remark}
            Observe that \[
                |T(f) - T(g)| \leq d(f, g),
            \] making $T$ a Lipschitz continuous map.
        \end{remark}
    \end{theorem}


    \begin{theorem}[Dominated convergence]
        Let $\{f_n\}_{n = 1}^\infty$ be a sequence of complex measurable functions
        $f_n\colon X \to \C$, such that $f_n \to f$ pointwise on $X$. Furthermore,
        let $g\colon X \to [0, \infty)$, $g \in L^1(\mu)$ such that $|f_n| \leq g$.
        Then, $f \in L^1(\mu)$, \[
            \lim_{n \to \infty} \int_X |f_n - f| \:d\mu = 0, \qquad
            \lim_{n \to \infty} \int_X f_n \:d\mu = \int_X f \:d\mu.
        \] 
    \end{theorem}

    \begin{corollary}[Bounded convergence]
        Let $\{f_n\}_{n = 1}^\infty$ be a sequence of complex measurable functions
        $f_n\colon X \to \C$, such that $f_n \to f$ pointwise on $X$.
        Furthermore let $|f_n| \leq M$ for some $M \in \R$, and let $\mu(X) <
        \infty$. Then, $f \in L^1(\mu)$, \[
            \lim_{n \to \infty} \int_X |f_n - f| \:d\mu = 0, \qquad
            \lim_{n \to \infty} \int_X f_n \:d\mu = \int_X f \:d\mu.
        \] 
    \end{corollary}

    \begin{example}
        Consider \[
            f_n \colon [0, 1] \to \R, \qquad x \mapsto \frac{nx}{1 + n^2x^2}.
        \] Then, it can be shown that $f_n \to 0$ pointwise on $[0, 1]$. Furthermore,
        each $|f_n| < 1$ and $\mu([0, 1]) = 1$, hence \[
            \lim_{n \to \infty} \int_{[0, 1]} \frac{nx}{1 + n^2x^2}\:d\mu = 0.
        \] 
    \end{example}


    \begin{theorem}[Fundamental theorem]
        Let $F\colon [a, b] \to \R$ such that $F' = f$ where $f \in L^1(\mu)$, $|f| <
        c$. Then, \[
            \int_{[a, b]} f \:d\mu = F(b) - F(a).
        \] 
    \end{theorem}
    \begin{proof}
        Define \[
            f_n\colon [a, b] \to \R, \qquad x \mapsto n(F(x + 1 / n) - F(x)),
        \] and note that $f_n \to f$.
    \end{proof}

    \begin{theorem}
        Let $f\colon X \times [a, b] \to \C$, where $-\infty < a < b < \infty$, and
        let each $f(\cdot, t) = f_t \in L^1(\mu, X)$. Define \[
            F(t) = \int_X f(x, t)\:d\mu(x).
        \] Suppose that $\partial f / \partial t$ exists, with $g \in L^1(\mu, X)$
        such that \[
            \left|\pp{f}{t}(x, t)\right| \leq g(x).
        \] Then, $F$ is differentiable, with \[
            F'(t) = \int_X \pp{f}{t}(x, t)\:d\mu(x).
        \] 
    \end{theorem}

    \begin{theorem}
        Let $f\colon X \to \C$, where $f \in L^1(\mu)$. Then for every $\epsilon >
        0$, there exists a $\delta > 0$ such that for all $E \in \M$ with $\mu(E) <
        \delta$, we have \[
            \int_E |f|\:d\mu < \epsilon.
        \] In other words, \[
            \lim_{\mu(E) \to 0} \int_E |f|\:d\mu = 0.
        \] 
    \end{theorem}
    \begin{proof}
        Define \[
            f_n\colon X \to \C, \qquad x \mapsto \begin{cases}
                |f(x)|, &\text{ if } |f(x)| \leq n, \\
                n, &\text{ if } |f(x)| > n.
            \end{cases}
        \] Then each $|f_n| \leq n$, $f_n \to |f|$ monotonically. Thus, the monotone
        convergence theorem will give \[
            \lim_{n \to \infty} \int_X f_n \:d\mu = \int_X |f|\:d\mu.
        \] For sufficiently large $N$, we have \[
            \int_X |f|\:d\mu - \int_X f_N\:d\mu < \frac{\epsilon}{2}.
        \] By restricting the domain of integration to $E \subseteq X$, \[
            \int_E |f|\:d\mu - \int_E f_N \:d\mu < \frac{\epsilon}{2}.
        \] Finally, set $\delta = \epsilon / 2N$, so that for $\mu(E) < \delta$, we
        have \[
            \int_E |f|\:d\mu < \int_E f_N\:d\mu + \frac{\epsilon}{2} < N\mu(E) +
            \frac{\epsilon}{2} = \epsilon. \qedhere
        \] 
    \end{proof}

    \begin{theorem}
        Let $f\colon X \to \C$, where $f \in L^1(\mu)$. Suppose that for all $E \in
        \M$, we have \[
             \int_E f\:d\mu = 0.
        \] Then, $f = 0$ almost everywhere on $X$.
    \end{theorem}

    \begin{theorem}
        Let $f\colon [a, b] \to \R$ be bounded on the compact interval $[a, b]$. If
        $f$ is Riemann integrable, then $f \in L^1(\mu)$ with \[
            \int_a^b f = \int_{[a, b]} f\:d\mu.
        \] 
        \begin{remark}
            The converse fails, since $\chi_\Q$ is not Riemann integrable, but it is
            Lebesgue integrable.
        \end{remark}
    \end{theorem}
    \begin{example}
        Suppose that we wish to compute the integral \[
            \int_{(0, 1)} \frac{1}{\sqrt{x}}\:d\mu.
        \] Note that the corresponding Riemann integral is improper. Thus, we define
        the functions \[
            f_n\colon (0, 1) \to \R, \qquad 
            x \mapsto \frac{1}{\sqrt{x}}\chi_{[1 / n, 1)}(x),
        \] and note that $f_n(x) \to 1 / \sqrt{x}$ on $(0, 1)$ monotonically.
        Thus, the monotone convergence theorem guarantees that \[
            \int_{(0, 1)} \frac{1}{\sqrt{x}}\:d\mu = \lim_{n \to \infty} \int_{1 /
            n}^1 \frac{1}{\sqrt{x}}\:dx = \lim_{n \to \infty} 2 \left( 1 -
            \frac{1}{\sqrt{n}} \right) = 2.
        \]
    \end{example}
    \begin{example}
        Suppose that we wish to compute \[
            \int_{(1, \infty)} \frac{1}{x^2}\:d\mu.
        \] Again, define \[
            f_n\colon (1, \infty) \to \R, \qquad x \mapsto \frac{1}{x^2}\chi_{(1,
            n)}(x).
        \] Then $f_n(x) \to 1 / x^2$ on $(1, \infty)$ monotonically, so \[
            \int_{(0, 1)} \frac{1}{x^2}\:d\mu = \lim_{n \to \infty} \int_1^n
            \frac{1}{x^2}\:dx = \lim_{n \to \infty} 1 - \frac{1}{n} = 1.
        \] 
    \end{example}
    \begin{example}
        Suppose that we wish to compute \[
            \lim_{n \to \infty} \int_0^n \left( 1 - \frac{x}{n} \right)^n
            e^{-2x}\:dx.
        \] By setting \[
            f_n\colon [0, \infty) \to \R, \qquad x \mapsto \left( 1 - \frac{x}{n}
            \right)^n e^{-2x}\chi_{[0, n]}(x),
        \] we have \[
            \lim_{n \to \infty} \int_0^n \left( 1 - \frac{x}{n} \right)^n
            e^{-2x}\:dx = \lim_{n \to \infty} \int_{[0, \infty)} f_n\:d\mu.
        \] Furthermore, $|f_n(x)| \leq e^{-2x}$ and the latter is in $L^1(\mu)$.
        Thus, the dominated convergence theorem guarantees that this limit is \[
            \int_{[0, \infty)} \lim_{n \to \infty} f_n\:d\mu = \int_0^\infty
            e^{-3x}\:dx = \frac{1}{3}.
        \] 
        \begin{remark}
            For any $x \in \R$, the sequence \[
                e_n = \left(1 + \frac{x}{n}\right)^n
            \] is increasing, with $e_n \to e^x$. To show this, consider the ratio \[
                \frac{e_{n + 1}}{e_n} = \frac{(1 + \frac{x}{n + 1})^{n + 1}}{(1 +
                \frac{x}{n})^n}
                = \left(\frac{1 + \frac{x}{n + 1}}{1 + \frac{x}{n}}\right)^{n + 1}
                \left(1 + \frac{x}{n}\right)
                = \left(1 - \frac{x}{(n + 1)(n + x)}\right)^{n + 1} \left(1 +
                \frac{x}{n}\right).
            \]
            Applying Bernoulli's inequality, noting that $|x / (n + x)| < 1$, \[
                \frac{e_{n + 1}}{e_n} \geq \left(1 - \frac{x}{n + x}\right)\left(1 +
                \frac{x}{n}\right) = 1.
            \] 
        \end{remark}
    \end{example}

    \begin{example}
        Suppose that we wish to compute \[
            \lim_{n \to \infty} \int_1^\infty \frac{\log(1 + nx)}{1 + x^2\log{n}}\:dx.
        \] Observe that for $x\geq 1, n\geq 1$, we have $nx \leq 1 + nx \leq 2nx$ so
        \[
            \frac{\log{n} + \log{x}}{1 + x^2\log{n}} \leq \frac{\log(1 + nx)}{1 +
            x^2\log{n}} \leq \frac{\log{2} + \log{n} + \log{x}}{1 + x^2\log{n}}.
        \] The Squeeze Theorem immediately gives \[
            \lim_{n \to \infty} \frac{\log(1 + nx)}{1 + x^2\log{n}} = \frac{1}{x^2}.
        \] Furthermore, \[
            \frac{\log(1 + nx)}{1 + x^2\log{n}} \leq \frac{\log{2}}{x^2} +
            \frac{\log{x}}{x^2} + \frac{\log{n}}{x^2},
        \] and the latter is in $L^1(\mu)$. Indeed, $\log{x} < x$ yields
        $\log{\sqrt{x}} \leq \sqrt{x}$, hence $\log{x}/x^2 \leq 2 x^{-3 / 2}$. Thus,
        the dominated convergence theorem guarantees that our limit is \[
            \lim_{n \to \infty} \int_1^\infty \frac{\log(1 + nx)}{1 + x^2\log{n}}\:dx
            = \int_1^\infty \frac{1}{x^2}\:dx = 1.
        \]
    \end{example}

    \begin{example}
        Suppose that we wish to compute \[
            \lim_{n \to \infty} \int_0^\infty \frac{n^{1 / 4}e^{-nx^2}}{1 + x^2}\:dx.
        \] Observe that the map $t \mapsto te^{-t^4}$ is bounded on $(0, \infty)$,
        attaining a maximum at $t = 1 / \sqrt{2}$ with the maximum $M = e^{-1 / 4} /
        \sqrt{2}$.  Thus, putting $t = n^{1 / 4}x^{1 / 2}$, we have \[
            n^{1 / 4}\sqrt{x}e^{-nx^4} \leq M, \qquad
            \frac{n^{1 / 4}e^{-nx^2}}{1 + x^2} \leq \frac{M / \sqrt{x}}{1 + x^2}.
        \] On the interval $(0, 1)$, we have \[
            \int_0^1 \frac{1}{\sqrt{x}(1 + x^2)}\:dx \leq \int_0^1
            \frac{1}{\sqrt{x}}\:dx = 2,
        \] and on $(1, \infty)$, we have \[
            \int_1^\infty \frac{1}{\sqrt{x}(1 + x^2)}\:dx \leq \int_1^\infty
            \frac{1}{x^2}\:dx = 1.
        \] Furthermore, $n^{1 / 4}e^{-nx^2} \to 0$ pointwise as $n \to \infty$. Thus,
        the dominated convergence theorem guarantees that \[
            \lim_{n \to \infty} \int_0^\infty \frac{n^{1 / 4}e^{-nx^2}}{1 + x^2}\:dx
            = 0.
        \] 

        \begin{remark}
            Consider $n^{1 / m}e^{-nx^k}$; to bound this, we examine the map
            $t \mapsto te^{-t^m}$. This attains a maximum when $e^{-t^m} -
            mt^me^{-t^m} = 0$, i.e.\ $t = 1 / m^{1 / m}$, hence a maximum value of $M
            = e^{-1 / m} / m^{1 / m}$. Thus, putting $t = n^{1 / m}x^{k / m}$, \[
                n^{1 / m}x^{k / m} e^{-nx^k} \leq M, \qquad n^{1 / m}e^{-nx^k} \leq
                Mx^{-k / m}.
            \] 
        \end{remark}
    \end{example}

    \subsection{$L^p$ spaces}
    \begin{definition}
        Let $(X, \M, \mu)$ be a measure space. For $1 \leq p < \infty$, we define \[
            L^p(\mu) = \{f\colon X \to \C : f\text{ is measurable, and }\int_X
            |f|^p\:d\mu \text{ is finite}\}.
        \]
        \begin{remark}
            We denote \[
                \norm{f}_p = \left(\int_X |f|^p\:d\mu\right)^{1 / p}.
            \] 
        \end{remark}
    \end{definition}

    \begin{definition}
        Let $(X, \M, \mu)$ be a measure space, and let $f\colon X \to \C$ be
        measurable. We define the essential supremum of $f$ as \[
            \norm{f}_\infty = \inf \{M : \mu\{x \in X: |f(x)| > M\} = 0\}.
        \]
        \begin{remark}
            We have $f \leq \norm{f}_\infty$ almost everywhere in $X$.
        \end{remark}
    \end{definition}
    \begin{example}
        Note that for \[
            f\colon (0, 1) \to \R, \qquad x \mapsto 1 / x,
        \] we have $\norm{f}_\infty = \infty$. By convention, $\inf\emptyset =
        \infty$.
    \end{example}

    \begin{definition}
        Let $(X, \M, \mu)$ be a measure space. We define the space of essentially
        bounded functions as \[
            L^\infty(\mu) = \{f\colon X \to \C : f\text{ is measurable, and }
            \norm{f}_\infty \text{ is finite}\}.
        \]
    \end{definition}

    \begin{lemma}[Young]
        For $a, b \geq 0$, $p, q > 1$, \[
            ab \leq \frac{a^p}{p} + \frac{b^q}{q}, \qquad \frac{1}{p} + \frac{1}{q} =
            1.
        \] 
    \end{lemma}
    \begin{proof}
        Using Jensen's inequality on the logarithm, \[
            \log(ta^p + (1 - t)b^q) \geq t\log(a^p) + (1 - t)\log(b^q).
        \] Putting $t = 1 / p$ and exponentiating immediately gives the result.
    \end{proof}

    \begin{lemma}[H\"older]
        For $1 \leq p, q \leq \infty$, $f, g$ measurable, \[
            \norm{fg}_1 \leq \norm{f}_p\, \norm{g}_q, \qquad \frac{1}{p}
            + \frac{1}{q} = 1.
        \] 
    \end{lemma}
    \begin{proof}
        This is trivial when either $f, g = 0$, or $p, q = 1$.  Otherwise, applying
        Young's inequality and integrating gives
        \[
            \int_X |f(x)g(x)|\:d\mu \leq \frac{1}{p}\int_X |f(x)|^p\:d\mu +
            \frac{1}{q}\int_X |g(x)|^q\:d\mu.
        \] When $\norm{f}_p = \norm{g}_q = 1$, this immediately gives the desired
        inequality since the right hand side is $1$. Otherwise, define $F = f /
        \norm{f}_p$, $G = g / \norm{g}_q$, upon which $\norm{F}_p = \norm{G}_q = 1$,
        hence \[
            \norm{fg}_1 = \int_X |f(x)g(x)|\:d\mu \leq \norm{f}_p\norm{g}_q. \qedhere
        \] 
    \end{proof}

    \begin{example}
        Let $f \in L^2(\R)$, and let each $\mu([n, n + 1)) = 1$. Then, \[
            \lim_{n \to \infty} \int_{[n, n + 1)} f\:d\mu = 0.
        \] To show this, note that H\"older's inequality for $p = 1 = 2$, also known
        as the Cauchy-Schwarz inequality, gives \[
            \int_{[n, n + 1)} |f|\:d\mu \leq \left(\int_{[n, n +
            1)}|f|^2\:d\mu\right)^{1 / 2}\left(\int_{[n, n + 1)}d\mu\right)^{1 / 2} =
            \left(\int_{[n, n + 1)} |f|^2\:d\mu\right)^{1 / 2}.
        \] Now, note that the following defines a measure $\nu$ on $\R$. \[
            \nu(E) = \int_E |f|^2\:d\mu.
        \] Furthermore, this is a finite measure because $\nu(\R) = \norm{f}_2^2 <
        \infty$. Now, each $\nu([n, n + 1)) \leq \nu([n, \infty))$; continuity
        from above now gives \[
            \lim_{n \to \infty} \int_{[n, n + 1)} |f|^2\:d\mu \leq \lim_{n \to
            \infty} \nu([n, \infty)) = \nu(\emptyset) = 0.
        \] 
    \end{example}
    
    \begin{lemma}[Minkowski]
        For $1 \leq p \leq \infty$, $f, g$ measurable, \[
            \norm{f + g}_p \leq \norm{f}_p + \norm{g}_p.
        \] 
    \end{lemma}
    \begin{proof}
        This is trivial when $p = 1, \infty$. Otherwise, for $1 < p < \infty$, note
        that \[
            |f(x) + g(x)|^p \leq \left(|f(x) + g(x)\right)^p \leq \left(2\max(f(x),
            g(x))\right)^p \leq 2^p\left(|f(x)|^p + |g(x)|^p\right).
        \] This shows that when $f, g \in L^p(\mu)$, we have $f + g \in L^p(\mu)$.
        Set $F = |f + g|^{p - 1}$, when the triangle inequality followed by
        H\"older's inequality gives \[
            \norm{f + g}_p^p \leq \int_X F(x)|f(x)|\:d\mu + \int_X F(x)|g(x)|\:d\mu
            \leq \norm{F(x)}_q\norm{f}_p + \norm{F(x)}_q\norm{g}_p,
        \] where $q = 1 - 1 /p$. Using $(p - 1)q = p$, \[
            \norm{F(x)}_q = \left(\int_X |f(x) + g(x)|^{(p - 1)^q} \:d\mu\right)^{1 /
            q} = \left(\int_X |f(x) + g(x)|^p \:d\mu\right)^{1 - 1 / p} =
            \frac{\norm{f + g}_p^p}{\norm{f + g}_p}.
        \] This immediately gives the result.
    \end{proof}

    \begin{theorem}
        The spaces of functions $L^p(\mu)$ and $L^\infty(\mu)$ are complete metric
        spaces.
        \begin{remark}
            Two functions in such a space are identified if they are equal almost
            everywhere.
        \end{remark}
    \end{theorem}

    \begin{corollary}
        Let $f_n \to f$ in $L^p(\mu)$, where each $f_n \in L^p(\mu)$, and $1 \leq p
        \leq \infty$. Then, there exists a subsequence $\{f_{n_k}\}_{k \in \N}$ such
        that $f_{n_k} \to f$ pointwise in $\C$, almost everywhere in $X$. Moreover,
        there exists $h \in L^p(\mu)$ such that $|f_{n_k}| \leq h$ almost everywhere
        in $X$.
    \end{corollary}

    \begin{theorem}
        Let $(X, \M, \mu)$ be a finite measure space, and let $1 \leq p < q \leq
        \infty$. Then, $L^p(\mu) \supseteq L^q(\mu)$.
    \end{theorem}
    \begin{proof}
        Set $u = q / p > 1$, $v = 1 - 1 / u$, whence \[
            \norm{|f|^p}_1 \leq \norm{|f|^p}_u\norm{1}_v = \norm{1}_v\left(\int_X
            |f|^q\:d\mu\right)^{p / q} = \mu(X)^{1 / v} \norm{f}_q^p.
        \] Thus, \[
            \norm{f}_p = \norm{|f|^p}_1^{1 / q} \leq \mu(X)^{1 / p - 1 /
            q}\norm{f}_q. \qedhere
        \] 
    \end{proof}
    \begin{example}
        Note that the map $x \mapsto 1 / \sqrt{x}$ is in $L^1(0, 1)$, but not in
        $L^2(0, 1)$.
    \end{example}
    \begin{example}
        Note that the map $x \mapsto 1 / x(|\log{x}| + 1)^2$ is in $L^1(0, 1)$, but
        not in any $L^p(0, 1)$ where $p > 1$.
    \end{example}

    \begin{theorem}
        Let $(X, \M, \mu)$ be a finite measure space. Then, $L^\infty(\mu) \subseteq
        L^p(\mu)$ for all $1 \leq p \leq \infty$, and \[
            \lim_{p \to \infty} \norm{f}_p = \norm{f}_\infty.
        \] 
    \end{theorem}
    \begin{example}
        Consider the function \[
            f\colon (0, 1) \to \R, \qquad x \mapsto \log{x}.
        \] Then for $1 \leq p < \infty$, we have \[
            \norm{f}_p^p = \int_{(0, 1)} |\log{x}|^p\:d\mu = \lim_{n \to \infty}
            \int_{(0, 1)} |\log{x}|^p\chi_{(1/n, 1)}\:d\mu.
        \] The latter follows from the monotone convergence theorem. Now, make the
        substitution $u = \log{x}$ to evaluate the Riemann integral \[
            \int_{1 / n}^1 |\log{x}|^p\:dx = \int_{-\log{n}}^0 |u|^p e^{u}\:du 
            = \int_0^{\log{n}} u^p e^{-u}\:du.
        \] Indeed, the monotone convergence theorem immediately shows that this
        converges to $\Gamma(p + 1)$. Another way to show that this converges is to
        see that $e^u > u^k / k!$, $u^pe^{-u} \leq k!u^{p - k}$. Choosing
        sufficiently large $k$ so that $k - p > 2$, we have \[
            \int_0^{\log{n}} u^p e^{-u}\:du \leq \int_0^1 du + k!\int_1^{\log{n}}
            \frac{1}{u^2}\:du = 1 + k! - \frac{k!}{\log{n}}.
        \] Thus, we have \[
            \norm{f}_p^p = \int_{(0, 1)}|\log{x}|^p\:dx \leq 1 + k! < \infty.
        \] This shows that $f \in L^p(\mu)$ for all $1 \leq p < \infty$. However, it
        is clear that $f \notin L^\infty(\mu)$.
    \end{example}

    \begin{theorem}
        Let $S$ be the set of all simple, measurable, complex valued functions which
        are non-zero on a set of finite measure. Then, the closure of $S$ in
        $L^p(\mu)$ is the whole of $L^p(\mu)$, for $1 \leq p < \infty$.
    \end{theorem}
    
    \begin{theorem}[Lusin]
        Let $f\colon \R^n \to \C$ be measurable, and let $A\subseteq \R^n$ have
        finite measure, with $f = 0$ on $\R^n \setminus A$. Given $\epsilon > 0$,
        there exists a continuous function $g$ on $\R^n$ with compact support, such
        that \[
            \mu\{x: f(x) \neq g(x)\} < \epsilon.
        \] Moreover, $g$ can be chosen such that \[
            \sup_{x \in \R^n} |g| \leq \norm{f}_\infty.
        \] 
    \end{theorem}

    \begin{theorem}
        The set of all continuous functions on $\R^n$ with compact support is dense
        in $L^p(\mu)$, for $1 \leq p < \infty$.
    \end{theorem}
    \begin{example}
        The set of all continuous functions on $\R^n$ with compact support is
        \emph{not} dense in $L^\infty(\mu)$; recall that the uniform limit of
        continuous functions is always continuous, i.e.\ sequences of continuous
        functions can only converge to continuous functions in $L^\infty(\mu)$.
        Instead, the closure of this set in $L^\infty$ consists of all continuous
        functions on $\R^n$ such such given $\epsilon > 0$, there exists compact $K
        \subset \R^n$ where $|f| < \epsilon$ on $\R^n\setminus K$.
    \end{example}

    \begin{example}
        Let $f \in L^2[0, 1]$, and suppose that \[
            \int_0^1 f(x)\, x^n\:dx = 0
        \] for all integers $n \geq 0$. Then, $f = 0$ almost everywhere. To show
        this, note that we have \[
            \int_0^1 f(x)(a_0 + a_1x + \dots + a_nx^n)\:dx = 0, \qquad
            \int_0^1 f(x)p(x)\:dx.
        \] for all polynomials $p(x)$. By the Weierstrass Approximation Theorem, we
        have \[
            \int_0^1 f(x) g(x)\:dx = 0
        \] for all continuous functions $h$ on $[0, 1]$. Now, $f \in L^2[0, 1]$,
        hence there exists a sequence of continuous functions $\{h_n\}_{n =
        1}^\infty$ on $[0, 1]$ so that $h_n \to f$ in $L^2[0, 1]$. By Cauchy-Schwarz,
        \[
            \left|\int_0^1 f(x)(h_n(x) - f(x))\:dx\right| \leq \norm{f}_2\, \norm{h_n
            - f}_2 \to 0.
        \] Thus, \[
            \int_0^1 f^2\:dx = 0,
        \] forcing $f = 0$ almost everywhere.
    \end{example}

    \begin{theorem}
        The set of all smooth functions on $\R^n$ with compact support is dense in
        $L^p(\mu)$, for $1 \leq p < \infty$.
    \end{theorem}

    \begin{theorem}[Egoroff]
        Let $(X, \M, \mu)$ be a finite measure space, and let $\{f_n\}_{n =
        1}^\infty$ be a sequence of complex measurable functions $f_n \colon X \to
        \C$ such that $f_n \to f$ pointwise almost everywhere on $X$. Then for every
        $\epsilon > 0$, there exists $E \subseteq X$ such that $f_n \to f$ uniformly
        on $E$ and $\mu(X \setminus E) < \epsilon$.

        \begin{remark}
            The converse holds even without assuming $\mu(X) < \infty$.
        \end{remark}
    \end{theorem}

    \begin{example}
        Consider the counting measure on $\N$, and the sequence of functions
        $\{f_n\}_{n = 1}^\infty$ described by \[
            f_n\colon \N \to \R, \qquad k \mapsto \begin{cases}
                1, &\text{ if } 1 \leq k \leq n, \\
                0, &\text{ otherwise}.
            \end{cases}
        \] Then, $f_n \to 1$ pointwise. Setting $\epsilon = 1 / 2$, we demand $f_n
        \to f$ uniformly on a set $E$ with $\mu(\N\setminus E) < 1 / 2$. This forces
        $\N\setminus E = \emptyset$, i.e.\ $E = \N$. However, the convergence $f_n
        \to 1$ is not uniform on $\N$.
    \end{example}
    \begin{example}
        Consider the sequence of functions $\{f_n\}_{n = 1}^\infty$ described by \[
            f_n\colon [0, 1] \to \R, \qquad x \mapsto x^n.
        \] Then, $f_n \to \chi_{\{1\}}$ pointwise, and the convergence $f_n \to 0$ is
        uniform on every interval $[0, 1 - \delta]$ for $\delta > 0$. However, we
        cannot have uniform convergence on any measure zero set.
    \end{example}


    \begin{theorem}
        Let $1 \leq p < \infty$, and let $\{f_n\}_{n = 1}^\infty$ be a sequence of
        measurable functions on $[0, 1]$ such that $f_n \to f$ pointwise and $f_n, f
        \in L^p(\mu)$. Then, $f_n \to f$ converges in $L^p(\mu)$ if and only if
        $\norm{f_n}_p \to \norm{f}_p$.
    \end{theorem}
    \begin{proof}
        First suppose that $f_n \to f$ in $L^p(\mu)$, i.e.\ $\norm{f_n - f}_p \to 0$.
        Now, Minkowski's inequality will show that \[
            |\norm{f_n}_p - \norm{f}_p| \leq \norm{f_n - f}_p,
        \] hence we have $\norm{f_n}_p \to \norm{f}_p$ by the squeeze theorem.

        Next, suppose that $\norm{f_n}_p \to \norm{f}_p$. We first show that for
        $\alpha, \beta \in \R$, \[
            (\alpha + \beta)^p \leq 2^{p - 1}(|\alpha|^p + |\beta|^p).
        \] Indeed, this follows immediately from the convexity of the map $t \mapsto
        t^p$ and Jensen's inequality, whence $(\alpha / 2 + \beta / 2)^p \leq
        (|\alpha|^p + |\beta|^p) / 2$. Thus, \[
            |f_n - f|^p \leq 2^{p - 1}|f_n|^p + 2^{p - 1}|f|^p.
        \] Now, $|f_n \to f|^p \to 0$ pointwise, and the right hand side of our
        inequality is in $L^1(\mu)$. Thus, the dominated convergence theorem
        guarantees that \[
            \lim_{n \to \infty} \int_{[0, 1]} |f_n - f|^p\:d\mu = \int_{[0, 1]}
            \lim_{n \to \infty} |f_n - f|^p\:d\mu = 0.
        \] Thus, $\norm{f_n - f}_p \to 0$.
    \end{proof}


    \subsection{The Lebesgue measure}
    
    \begin{definition}
        Let $X$ be a set, and let $\mathcal{P}(X)$ denote its power set. We say that
        a function $\mu^*\colon \mathcal{P}(X) \to [0, \infty]$ is called an outer
        measure if it satisfies the following.
        \begin{enumerate}
            \itemsep0em
            \item $\mu^*(\emptyset) = 0$.
            \item $\mu^*(A) \leq \mu^*(B)$ whenever $A \subseteq B$.
            \item \[
                \mu^*\left(\bigcup_{i = 1}^\infty A_i\right) \leq \sum_{i = 1}^\infty
                \mu^*(A_i).
            \] 
        \end{enumerate}
    \end{definition}

    \begin{example}
        Let $\xi \subseteq \mathcal{P}(X)$ such that $\emptyset, X \in \xi$, and let
        $f\colon \xi \to [0, \infty]$ with $f(\emptyset) = 0$. For $A \subseteq X$,
        we may define \[
            \mu^*(A) = \inf\left\{\sum_{i = 1}^\infty f(E_i) : A \subseteq \bigcup_{i
            = 1}^\infty E_i, E_i \in \xi\right\}.
        \] It can be verified that this is indeed an outer measure. The first
        property follows immediately, and the second follows from the fact that if
        $A\subseteq B$, all covers of $B$ are also covers of $A$. For the third, let
        \[
            A = \bigcup_{i = 1}^\infty A_i.
        \] Then, for arbitrary $\epsilon > 0$, we can choose $E_{ij} \in \xi$ such
        that \[
            \mu^*(A_i) + \frac{\epsilon}{2^i} \geq \sum_j f(E_{ij}), \qquad 
            A_i \subseteq \bigcup_j E_{ij}.
        \] Now, the sets $E_{ij}$ all cover $A$, so \[
            \mu^*(A) \leq \sum_{ij} f(E_{ij}) \leq \sum_i \left[\mu^*(A_i) +
            \frac{\epsilon}{2^i}\right] = \sum_i \mu^*(A_i) + \epsilon.
        \] Since $\epsilon > 0$ is arbitrary, we have the desired result, \[
            \mu^*(A) \leq \sum_i \mu^*(A_i).
        \] 
    \end{example}

    \begin{definition}
        A set $A \subseteq X$ is called $\mu^*$-measurable if for all $E \subseteq
        X$, we have \[
            \mu^*(A) = \mu^*(E \cap A) + \mu^*(E \cap A^c).
        \]
        \begin{remark}
            We need only check the $\geq$ direction, and we need only check sets $E$
            of finite outer measure.
        \end{remark}
    \end{definition}

    \begin{theorem}[Carath\'eodory]
        Let $X$ be non-empty with an outer measure $\mu^*$, and let $\M$ be the set
        of all $\mu^*$-measurable subsets of $X$. Then, $\M$ forms a
        $\sigma$-algebra, and the restriction of $\mu^*$ to $\M$ is a measure $\mu$.
        Moreover, $\mu$ is a complete measure.

        \begin{remark}
            A complete measure $\mu$ is such that given a measurable set $A$ with
            $\mu(A) = 0$, all the subsets $B \subseteq A$ are measurable, with
            $\mu(B) = 0$.
        \end{remark}
    \end{theorem}

    \begin{definition}
        Let $\ell$ be the length function for intervals in $\R$. Extend this to an
        outer measure $m_1^*$ as outlined previously, and use the Carath\'eodory
        Theorem to find a $\sigma$-algebra $\L \subset \mathcal{P}(\R)$ and a
        complete measure $m_1$ on $\L$. Then, we call $\L$ the Lebesgue
        $\sigma$-algebra, and $m_1$ the Lebesgue measure on $\R$. Note that \[
            m_1^*(A) = \inf\left\{\sum_{i = 1}^\infty |b_i - a_i|: A \subseteq
            \bigcup_{i = 1}^\infty (a_i, b_i)\right\}.
        \] 
        \begin{remark}
            A similar process can be carried out using the volume function $\ell_n$
            for rectangles in $\R^n$, yielding the Lebesgue $\sigma$-algebra $\L_n$
            and the complete Lebesgue measure $m_n$.
        \end{remark}
    \end{definition}

    \begin{example}
        All singletons have Lebesgue measure zero. As a result, all countable sets
        also have Lebesgue measure zero.
    \end{example}

    \begin{theorem}
        We have the inclusion $\mathcal{B}_\R \subset \L$.
        
        \begin{remark}
            The restriction of $m_1$ to $\mathcal{B}_\R$ is called the Borel measure
            $m_{\mathcal{B}}$.
        \end{remark}
    \end{theorem}
    \begin{proof}
        It is enough to show that sets of the form $A = (a, \infty) \in \L$. Pick a
        subset $E \subseteq \R$, and suppose that $\{I_n\}_{n = 1}^\infty$ is a cover
        of $E$ with intervals $(a_n, b_n)$. Then, by the property of infimums, \[
            \sum_{n = 1}^\infty \ell(I_n \cap (a, \infty)) + \ell(I_n \cap (-\infty,
            a)) \geq m_1^*(E \cap A) + m_1^*(E \cap A^c)
        \] However, each \[
            \ell(I_n) \geq \ell(I_n \cap (a, \infty)) + \ell(I_n \cap (-\infty, a)).
        \] Thus, summing and taking infimums again, \[
            m_1^*(A) \geq m_1^*(E \cap A) + m_1^*(E \cap A^c),
        \] proving that $A$ is $m_1^*$ measurable, hence $A \in \L$.
    \end{proof}

    \begin{theorem}
        Let $E \subseteq \R$ be Lebesgue measurable. Then, \begin{align*}
            m_1(E) &= \inf\{m_1(U): E \subseteq U,\, U\text{ is open}\}, \\
            &= \sup\{m_1(K): K \subseteq E,\, K\text{ is compact}\}.
        \end{align*}
        \begin{remark}
            The above relations describe the Lebesgue measure as a limit of sorts of
            the Borel measure.
        \end{remark}
    \end{theorem}
    \begin{proof}
        For the first part, note that for any open set $U$ such that $E \subseteq U$,
        we immediately have $m_1(E) \leq m_1(U)$, hence \[
            m_1(E) \leq \inf\{m_1(U): E \subseteq U,\, U\text{ is open}\}.
        \] We now show the reverse inequality. Note that if $m_1(E) = \infty$, this
        is trivial. Otherwise, $m_1(E)$ is finite, hence for $\epsilon > 0$ we can
        find an open cover $\{I_n\}_{n = 1}^\infty$ of $E$ such that \[
            m_1(E) + \epsilon \geq \sum_{n = 1}^\infty \ell((a_n, b_n)) = \sum_{n =
            1}^\infty m_1(I_n) \geq m_1\left(\bigcup_{n = 1}^\infty I_n\right).
        \] Since $U = \bigcup_{n = 1}^\infty I_n$ is an open set with $E \subseteq
        U$, we are done.

        Note that when $m_1(E)$ is finite, we have found open $U$ such that $E
        \subseteq U$, and $m_1(U \setminus E) < \epsilon$. But $U\setminus E = U \cap
        E^c = E^c\setminus U^c$, and $U^c \subseteq E^c$. Relabelling, we have shown
        that given Lebesgue measurable $E \subseteq \R$, we can find a closed set $F$
        such that $F \subseteq E$, and $m_1(E \setminus F) < \epsilon$. \\

        For the next part, note that for any compact set $K$ such that $K \subseteq
        E$, we immediately have $m_1(E) \geq m_1(K)$, hence \[
            m_1(E) \geq \sup\{m_1(K): K \subseteq E,\, K\text{ is compact}\}.
        \] We now show the reverse inequality. First, consider the case where $E$ is
        bounded, so $m_1(E)$ is finite. If $E$ is closed, it is also compact, hence
        the result is trivial. Otherwise, the inclusion $E \subset \overline{E}$ is
        strict, hence $\overline{E}\setminus E$ is non-empty and open. This gives \[
            m_1(\overline{E}\setminus E) = \inf\{m_1(U): \overline{E}\setminus E
            \subseteq U, U\text{ is open}\}.
        \] Thus for $\epsilon > 0$, there exists open $U$ such that
        $\overline{E}\setminus E \subseteq U$ and \[
            m_1(\overline{E}\setminus E) + \epsilon \geq m_1(U), \qquad
            m_1(E) - \epsilon \leq m_1(\overline{E}\setminus U).
        \] Note that we could perform this rearrangement since the sets $E,
        \overline{E}, U$ all have finite measure. This, we have found a suitable
        compact set $K = \overline{E}\setminus U$ with $E \subseteq K$, hence we are
        done.

        If $E$ is unbounded and $m_1(E) = \infty$, set $E_n = E \cap [n, n + 1)$,
        whence $E = \bigcup_{n \in \Z} E_n$. Now the $E_n$ are bounded, measurable,
        and disjoint. For $\epsilon > 0$, find compact sets $K_n$ such that each \[
            m_1(E_n) - \frac{\epsilon}{3\cdot 2^{|n|}} \leq m_1(K_n).
        \] Set \[
            E^n = \bigcup_{-n \leq i \leq n} E_n, \qquad
            K^n = \bigcup_{-n \leq i \leq n} K_i
        \] Note that each $K^n$ is compact. Summing our inequality gives \[
            m_1(E^n) - \epsilon \leq m_1(K^n).
        \] Continuity from below gives \[
            m_1(E) = \lim_{n \to \infty} m_1(E^n).
        \] Thus, if $m_1(E) = \infty$, we have $m_1(K^n) \to \infty$ with each $K^n
        \subseteq E$ and $K^n$ compact, hence we are done.

        Otherwise, $m_1(E)$ is finite. Find open $U$ such that $E \subseteq U$,
        $m_1(U\setminus E) < \epsilon/3$. Setting $U^n = U \cap (-n, n)$, we have \[
            U = \bigcup_{n = 1}^\infty U^n, \qquad 
            m_1(U) = \lim_{n \to \infty} m_1(U^n),
        \] hence $m_1(U\setminus U^m) < \epsilon/3$ for sufficiently large $m$. This
        in turn gives $m_1(E\setminus U^m) < \epsilon/3$. Next, since $m_1(E)$,
        $m_1(U^m)$ are finite, we can find closed sets $F_1, F_2$ such that $F_1
        \subseteq E$, $F_2 \subseteq U^m$, and $m_1(E\setminus F_1) < \epsilon/3$,
        $m_1(U^m\setminus F_2) < \epsilon/3$. Furthermore, $F_2$ is closed and
        bounded, hence compact. Set $K = F_1 \cap F_2$, which is closed and bounded
        hence compact. Now, $K \subseteq E$, and \[
            E\setminus K \subseteq (E\setminus U^m) \cup (E \setminus F_1) \cup
            (U^m\setminus F_2).
        \] This immediately gives $m_1(E\setminus K) < \epsilon$, or $m_1(E) -
        \epsilon < m_1(K)$, hence we are done.
    \end{proof}

    \begin{corollary}
        If $E \subseteq \R$ is Lebesgue measurable and $\epsilon > 0$, then there
        exists an open set $U$ such that $E \subseteq U$ and $m_1(U \setminus E) <
        \epsilon$.
    \end{corollary}
    \begin{proof}
        We have already dealt with the case where $m_1(E)$ is finite. Otherwise, set
        $E_n = E \cap [n, n + 1)$; each of these has finite measure, hence we can
        find open sets $U_n$ such that $E_n \subseteq U_n$ and $m_1(U_n\setminus E_n)
        < \epsilon / 3\cdot 2^{|n|}$. Set $U = \bigcup_{n \in \Z} U_n$, whence $U$ is
        open with $E \subseteq U$ and \[
            m_1(U\setminus E) \leq \sum_{n \in \Z} m_1(U_n\setminus E_n) < \epsilon.
            \qedhere
        \] 
    \end{proof}
    
    \begin{corollary}
        If $E \subseteq \R$ is Lebesgue measurable with $m_1(E) < \infty$ and
        $\epsilon > 0$, then there exists a compact set $K$ such that $K \subseteq E$
        and $m_1(E \setminus K) < \epsilon$.
    \end{corollary}

    \begin{example}
        Note that we cannot necessarily find open sets $U$ with $U \subseteq E$, or
        closed sets $F$ with $E \subseteq K$ such that the differences have
        arbitrarily small measure. Note that the set $\Q^c$ of irrationals has empty
        interior although $m_1(\Q^c) = \infty$, and the closure of the set $\Q$ of
        rationals is the entirety of $\R$ although $m_1(\Q) = 0$.
    \end{example}

    \begin{example}
        If $U \subseteq R$ is Lebesgue measurable and non-empty, then $m_1(U) > 0$.
        This is because the non-empty open set $U$ must contain a basic open interval
        of the form $(a, b) \subseteq U$, hence $m_1(U) \geq m_1((a, b)) = b - a >
        0$.
    \end{example}

    \begin{example}
        If $E \subseteq \R$ is Lebesgue measurable with $m_1(E) = 0$, then
        $\overline{E^c} = \R$. To see this, pick $x \in \R$ and an open neighbourhood
        $(x - \delta, x + \delta)$. If $(x - \delta, x + \delta) \cap E^c = (x -
        \delta, x + \delta)\setminus E$ were empty, that would force $(x - \delta, x
        + \delta) \subseteq E$, hence $m_1(E) \geq 2\delta > 0$, a contradiction.
    \end{example}

    \begin{theorem}
        A set $E \subseteq \R$ is Lebesgue measurable if and only if we can write \[
            E = G \setminus N_1 = F \cup N_2,
        \] where $G$ is a $G_\delta$ set, $F$ is an $F_\sigma$ set, and $N_1, N_2$
        have measure zero.

        \begin{remark}
            A $G_\delta$ set is a countable intersection of closed sets, and an
            $F_\sigma$ set is a countable union of open sets.
        \end{remark}
    \end{theorem}

    \begin{proof}
        Let $E \subseteq \R$ be Lebesgue measurable. Then, we can find a sequence of
        open sets $\{U_n\}_{n = 1}^\infty$ such that each $E \subseteq U_n$ and
        $m_1(U_n\setminus E) < 1 / n$. Set \[
            G = \bigcap_{n = 1}^\infty U_n, \qquad
            m_1(G\setminus E) \leq m_1(U_n\setminus E) < \frac{1}{n}.
        \] This forces $m_1(G\setminus E) = 0$. Thus, we write $E = G \setminus
        (G\setminus E)$.

        Next, we can find a sequence of closed sets $\{F_n\}_{n = 1}^\infty$ such
        that each $F_n \subseteq E$ and $m_1(E\setminus F_n) < 1 / n$. Again, set \[
            F = \bigcup_{n = 1}^\infty F_n, \qquad
            m_1(E\setminus F) \leq m_1(E\setminus F_n) < \frac{1}{n}.
        \] This forces $m_1(E\setminus F) = 0$. Thus, we write $E = F\cup(E\setminus
        F)$.
    \end{proof}

    \begin{corollary}
        Given any Lebesgue measurable set $E \subseteq \R$, we can find Borel
        measurable sets $F, G$ such that $F \subseteq E \subseteq G$, and
        $m_1(G\setminus F) = 0$.
    \end{corollary}

    \begin{theorem}
        Let $(X, \M, \mu)$ be a measure space. Then, there exists an extended
        $\sigma$-algebra $\bar{\M}$ and an extended measure $\bar{\mu}$ such that
        $\bar{\mu}$ is a complete measure. Furthermore, \[
            \bar{\M} = \{S \cup N : S \in \M, N \subseteq N', \mu(N) = 0\},
        \] and \[
            \bar{\mu}(S \cup N) = \mu(S).
        \] 
    \end{theorem}

    \begin{theorem}
        The completion of the Borel measure space $(\R, \mathcal{B}_\R,
        m_\mathcal{B})$ is the Lebesgue measure space $(\R, \L, m_1)$.
    \end{theorem}

    
    \begin{lemma}
        If $E \subseteq \R$ is Lebesgue measurable, so are the translations $x + E$
        and the dilations $rE$. Moreover, \[
            m_1(x + E) = m_1(E), \qquad
            m_1(rE) = |r|\,m_1(E).
        \] 
    \end{lemma}

    \begin{example}
        The Cantor set defined as \[
            \mathcal{C} = [0, 1]\setminus \bigcup_{n = 0}^\infty \bigcup_{k = 0}^{3n
            - 1} \left(\frac{3k + 1}{3^{n + 1}}, \frac{3k + 2}{3^{n + 1}}\right),
        \] or equivalently \[
            \mathcal{C} = \bigcap_{n = 1}^\infty C_n, \qquad 
            C_n = \bigcup_{k = 0}^{3^n - 1}\left(\left[\frac{3k}{3^n}, \frac{3k +
            1}{3^n}\right] \cup \left[\frac{3k + 2}{3^n}, \frac{3k +
            3}{3^n}\right]\right)
        \] is compact, and uncountable. Indeed it is Borel measurable, hence Lebesgue
        measurable with $m_1(\mathcal{C}) = 0$.
    \end{example}

    \begin{example}
        Consider $\R$ as an additive group, and examine the quotient group $\R/\Q$.
        Pick exactly one representative element from each coset, ensuring that it
        belongs to the interval $[0, 1]$, and call this set $\mathcal{V}$. This is a
        Vitali set, and it it not Lebesgue measurable. To see this, suppose that it
        were. Enumerate the rationals in $[-1, 1]$ as $\{q_i\}_{i \in \N}$, and set
        $\mathcal{V}_i = q_i + \mathcal{V}$. Note that each $\mathcal{V}_i$ must also
        be Lebesgue measurable, with $m_1(\mathcal{V}_i) = m_1(\mathcal{V})$ due to
        translation invariance. We claim that \[
            [0, 1] \subseteq \bigcup_{i = 1}^\infty
            \mathcal{V}_i \subseteq [-1, 2].
        \] To see the former inclusion, pick arbitrary $x \in [0, 1]$. Then, $x$ must
        belong to one of the cosets of $\R/\Q$, say $r + \Q$ with $r \in
        \mathcal{V}$.  Thus, $x - r \in \Q$, but $-1 \leq x - r \leq 1$ hence $x - r
        = q_i$ for some $i \in \N$. It immediately follows that $x = q_i + r \in q_i
        + \mathcal{V} = \mathcal{V}_i$.

        Our set of inclusions implies that \[
            m_1([0, 1]) \leq \sum_{i = 1}^\infty m_1(\mathcal{V}_i) \leq m_1([-1, 3]).
        \] Thus, $1 \leq \sum_{i = 1}^\infty m_1(\mathcal{V}) \leq 3$, which is
        absurd.

        \begin{remark}
            The construction of $\mathcal{V}$ invokes the Axiom of Choice.
        \end{remark}
    \end{example}

    \begin{lemma}
        The following inclusions are strict. \[
            \mathcal{B}_\R \subset \L \subset \mathcal{P}(\R).
        \] 
        \begin{remark}
            The Borel $\sigma$-algebra has the cardinality of the continuum,
            $\mathfrak{c}$. However, note that the uncountable Cantor set
            $\mathcal{C}$ is Lebesgue measurable with $m_1(\mathcal{C}) = 0$, hence
            all of its subsets are also Lebesgue measurable. This shows that the
            Lebesgue $\sigma$-algebra has cardinality $2^\mathfrak{c}$, strictly
            greater than that of the Borel $\sigma$-algebra.
        \end{remark}
    \end{lemma}
    

    \begin{theorem}
        If $E \subseteq \R$ is Lebesgue measurable with $m_1(E) > 0$, then $E$
        contains a non-measurable subset.
    \end{theorem}
    \begin{proof}
        First, we show that any measurable subset of a Vitali set $\mathcal{V}$ has
        measure zero. Indeed if $A \subseteq \mathcal{V}$ is Lebesgue measurable,
        then set $A_i = q_i + A$ for all rationals $q_i \in [-1, 1]$. Furthermore,
        the sets $A_i$ are all mutually disjoint. From $A_i \subseteq [-1, 2]$, we
        have \[
            \sum_{i = 1}^\infty m_1(A) = \sum_{n = 1}^\infty m_1(A_i) \leq 3,
        \] hence $m_1(A) = 0$.

        Now, let $E \subseteq [0, 1]$ be Lebesgue measurable, and $m_1(E) > 0$. Then,
        if all the sets $E_i = E \cap \mathcal{V}_i$ were to be measurable, each
        $m_1(E_i) = 0$ hence the union \[
            m_1(E) = m_1\left(\bigcap_{i = 1}^\infty E_i\right) = 0,
        \] a contradiction. Thus, at least one of the $E_i$ must be non-measurable.

        Finally, given $E \subseteq \R$ Lebesgue measurable with $m_1(E) > 0$, we
        must have some $m_1(E \cap [n, n + 1)) > 0$, whence we apply the same
        argument on the shifted set $(E \cap [n, n + 1)) - n$.
    \end{proof}
    \begin{corollary}
        Let $E \subseteq \R$ be Lebesgue measurable, such that all of its subsets
        are also Lebesgue measurable. Then, $m_1(E) = 0$.
    \end{corollary}

    

    \subsection{Product measures}

    \begin{definition}
        Let $(X, \M, \mu)$ and $(Y, \MN, \nu)$ be measure spaces.
        \begin{enumerate}
            \itemsep0em
            \item The sets $A \times B$ with $A \in \M$, $B \in \MN$ are called
            measurable rectangles.
            \item Finite unions of disjoint measurable rectangles are called
            elementary sets.
            \item Let $\mathcal{E}$ be the collection of all elementary rectangles.
            The product $\sigma$-algebra $\M \times \MN$ is the $\sigma$-algebra
            generated by $\mathcal{E}$.
        \end{enumerate}
    \end{definition}

    \begin{definition}
        A collection $\mathcal{A} \subseteq \mathcal{P}(X)$ is called a monotone
        class if the following hold.
        \begin{enumerate}
            \itemsep0em
            \item Given $\{A_n\}_{n \in \N}$ with each $A_n \in \mathcal{A}$ and $A_n
            \subseteq A_{n + 1}$, the union \[
                A = \bigcup_{n = 1}^\infty A_n \in \mathcal{A}.
            \] 
            \item Given $\{B_n\}_{n \in \N}$ with each $B_n \in \mathcal{A}$ and $B_n
            \supseteq B_{n + 1}$, the intersection \[
                B = \bigcap_{n = 1}^\infty B_n \in \mathcal{A}.
            \] 
        \end{enumerate}
    \end{definition}

    \begin{definition}
        The smallest monotone class containing a collection $S$ of subsets of $X$ is
        called the monotone class generated by $S$, denoted $\mathcal{A}(S)$.
    \end{definition}

    \begin{lemma}
        Given a collection $S$ of subsets of $X$, we have \[
            S \subseteq \mathcal{A}(S) \subseteq \M(S) \subseteq \mathcal{P}(X).
        \] 
    \end{lemma}

    \begin{definition}
        A collection $\mathcal{F} \subseteq \mathcal{P}(X)$ is called an algebra over
        $X$ if the following hold.
        \begin{enumerate}
            \itemsep0em
            \item $\mathcal{F}$ contains $X$.
            \item $\mathcal{F}$ is closed under complementation.
            \item $\mathcal{F}$ is closed under finite unions.
        \end{enumerate}
        \begin{remark}
            The following properties follow immediately.
            \begin{enumerate}
                \itemsep0em
                \item $\mathcal{F}$ contains $\emptyset$.
                \item $\mathcal{F}$ is closed under finite intersections.
                \item $\mathcal{F}$ is closed under differences.
            \end{enumerate}
        \end{remark}
    \end{definition}

    \begin{example}
        The collection of elementary sets $\mathcal{E}$ defined earlier forms an
        algebra over $X \times Y$.
    \end{example}
    

    \begin{theorem}[Monotone Class Theorem]
        If $\mathcal{F}$ is an algebra of sets over $X$, then
        $\mathcal{A}(\mathcal{F}) = \mathcal{M}(\mathcal{F})$. In other words, the
        monotone class generated by $\mathcal{F}$ is precisely the $\sigma$-algebra
        generated by $\mathcal{F}$.
    \end{theorem}
    \begin{proof}
        For each $P \in \mathcal{F}$, define \[
            \mathcal{A}_P = \{Q \in \mathcal{A}(\mathcal{F}): P\cup Q, P\setminus Q,
            Q\setminus P \in \mathcal{A}(\mathcal{F})\}.
        \] We claim that each $\mathcal{A}_P$ is a monotone class. Indeed, let
        $\{Q_n\}_{n \in \N}$ be an increasing sequence in $\mathcal{A}_P$, and let
        $Q$ be their union. Then, $\{P\cup Q_n\}_{n \in \N}$, $\{Q_n \setminus P\}_{n
        \in \N}$ are increasing sequences in $\mathcal{A}(\mathcal{F})$, hence \[
            P \cup Q = \bigcup_{n \in \N} P\cup Q_n \in \mathcal{A}(\mathcal{F}), \qquad
            Q\setminus P = \bigcup_{n \in \N} Q_n\setminus P \in
            \mathcal{A}(\mathcal{F}).
        \] Also, $\{P\setminus Q_n\}_{n \in \N}$ is a decreasing sequence in
        $\mathcal{A}(\mathcal{F})$, hence \[
            P\setminus Q = \bigcap_{n \in \N} P\setminus Q_n \in
            \mathcal{A}(\mathcal{F}).
        \] This shows that the union $Q \in \mathcal{A}_P$. The case for decreasing
        sequences is analogous. \\

        Now, note that given $P, Q \in \mathcal{F}$, we have $P \cup Q, P\setminus Q,
        Q\setminus P \in \mathcal{F} \subseteq \mathcal{A}(\mathcal{F})$. Thus, $Q
        \in \mathcal{A}_P$, $P \in \mathcal{A}_Q$. This shows that $\mathcal{F}
        \subseteq \mathcal{A}_P, \mathcal{A}_Q$, so $\mathcal{A}(\mathcal{F})
        \subseteq \mathcal{A}_P, \mathcal{A}_Q$.

        Next, for $P, Q \in \mathcal{A}(\mathcal{F})$, we have $P \in \mathcal{A}_Q$,
        so $P \cup Q, P\setminus Q \in \mathcal{A}_Q \supseteq
        \mathcal{A}(\mathcal{F})$. This is enough to show that
        $\mathcal{A}(\mathcal{F})$ is an algebra.

        Finally, let $\{E_n\}_{n \in \N}$ be a countable collection of sets from
        $\mathcal{A}(\mathcal{F})$. Since the latter is an algebra, the finite unions
        \[
            F_n = \bigcup_{i = 1}^n E_i \in \mathcal{A}(\mathcal{F}).
        \] Now, $\{F_n\}_{n \in \N}$ is an increasing sequence in
        $\mathcal{A}(\mathcal{F})$, so \[
            \bigcup_{n \in \N} E_n = \bigcup_{n \in \N} F_n \in
            \mathcal{A}(\mathcal{F}).
        \] This shows that $\mathcal{A}(\mathcal{F})$ is a $\sigma$-algebra
        containing $\mathcal{F}$, so $\M(\mathcal{F}) \subseteq
        \mathcal{A}(\mathcal{F})$. Thus, $\mathcal{A}(\mathcal{F}) =
        \M(\mathcal{F})$.
    \end{proof} 

    \begin{corollary}
        The monotone class generated by the algebra of elementary sets $\mathcal{E}$
        is the product $\sigma$-algebra $\M \times \MN$.
    \end{corollary}

    \begin{theorem}
        Let $E \subseteq \M \times \MN$. Define the sections \[
            E_x = \{y \in Y: (x, y) \in E\}, \qquad
            E_y = \{x \in X: (x, y) \in E\}.
        \] Then, $E_x \in \M$, $E_y \in \MN$.
    \end{theorem}
    \begin{proof}
        Let \[
            \Omega = \{E \in \M \times \MN: E_x \in \M\text{ for all } x \in X\}
            \subseteq \M \times \MN.
        \] Note that $\Omega$ trivially contains all measurable rectangles, hence we
        have $\mathcal{E} \subseteq \Omega$. We claim that $\Omega$ is a
        $\sigma$-algebra, whence $\M(\mathcal{E}) \subseteq \Omega$ forces $\Omega =
        \M \times \MN$. To show this, first we clearly have $X \times Y \in \Omega$.
        Next given $E \in \Omega$, we have $E_x \in \MN$, hence \begin{align*}
            (E^c)_x &= \{y \in Y: (x, y) \in E^c\} \\
            &= \{y \in Y: (x, y) \notin E\} \\
            &= Y \setminus \{y \in Y: (x, y) \in E\} \\
            &= Y\setminus E_x
        \end{align*} gives $(E^c)_x = (E_x)^c \in \MN$. Finally, if $\{E_i\}_{i \in
        \N}$ are such that $E_i \in \Omega$, then each $(E_i)_x \in \MN$, hence
        \begin{align*}
            \left(\bigcup_{i = 1}^\infty E_i\right)_x &= \{y \in Y: (x, y) \in
            \bigcup_{i = 1}^\infty E_i\} \\
            &= \bigcup_{i = 1}^\infty \{y \in Y: (x, y) \in E_i\} \\
            &= \bigcup_{i = 1}^\infty (E_i)_x
        \end{align*} gives $(\bigcup_{i = 1}^\infty E_i)_x = \bigcup_{i = 1}^\infty
        (E_i)_x \in \MN$.
    \end{proof}


    \begin{theorem}
        Let $Z$ be a topological space, and let $f\colon X \times Y \to Z$ be
        $(\M\times \MN, \mathcal{B}_Z)$ measurable. Define the sections \begin{align*}
            f_x&\colon Y \to Z, \qquad y \mapsto f(x, y), \\
            f_y&\colon X \to Z, \qquad x \mapsto f(x, y).
        \end{align*}
        Then, $f_x, f_y$ are measurable functions.
    \end{theorem}
    \begin{proof}
        Let $U\subseteq Z$ be open. Then, $E = f^{-1}(U) \in \M \times \MN$ by the
        measurability of $f$. It can be shown that $f_x^{-1}(U) = E_x \in \MN$ and
        $f_y^{-1}(U) = E_y \in \M$, hence $f_x, f_y$ are indeed measurable.
    \end{proof}

    \begin{definition}
        A measure space $(X, \M, \mu)$ is called $\sigma$-finite if $X$ can be
        written as the countable, disjoint union of measurable sets of finite
        measure.
    \end{definition}

    \begin{definition}
        Let $(X, \M, \mu)$ and $(Y, \MN, \nu)$ be $\sigma$-finite measure spaces.
        The product measure on $\M \times \MN$ is defined as \[
            (\mu \times \nu)(E) = \int_X \nu(E_x)\:d\mu = \int_Y \mu(E_y)\:d\nu.
        \] 
    \end{definition}

    \begin{lemma}
        The product measure is well-defined, and is indeed a measure on $\M\times
        \MN$.
    \end{lemma}

    \begin{example}
        Consider $X = [0, 1]$ with the Lebesgue measure, and $Y = [0, 1]$ with the
        counting measure. Set $D = \{(x, x): x \in [0, 1]\}$, whence \[
            \int_{X}\nu(D_x)\:d\mu = \int_{[0, 1]} 1\:d\mu = 1, \qquad
            \int_{Y}\nu(D_y)\:d\nu = \int_{[0, 1]} 0\:d\mu = 0.
        \] 
    \end{example}

    \begin{example}
        The product measure obtained from two complete measure spaces may not be
        complete. Consider the product $m_1 \times m_1$ on $[0, 1] \times [0, 1]$,
        and the set $\{0\} \times [0, 1]$. This clearly has zero measure, but the
        subset $\{0\} \times \mathcal{V}$ is not $\L\times \L$ measurable; if it
        were, the section $\mathcal{V}$ would have to be Lebesgue measurable.

        This shows in particular that $m_1 \times m_1 \neq m_2$, where $m_2$ is the
        Lebesgue measure defined on $R^2$ (via the Carath\'eodory process). However,
        it is true that the completion $\overline{m_1 \times m_1} = m_2$.
    \end{example}


    \begin{theorem}[Fubini-Tonelli]
        Let $(X, \M, \mu)$ and $(Y, \MN, \nu)$ be $\sigma$-finite measure spaces, and
        let $f\colon X\times Y \to \C$ be measurable.
        \begin{enumerate}
            \item Let $f\geq 0$. Then the functions \begin{align*}
                \varphi&\colon X \to [0, \infty], \qquad x \mapsto \int_Y f_x\:d\nu,
                \\
                \psi&\colon Y \to [0, \infty], \qquad y \mapsto \int_X f_y\:d\mu
            \end{align*}
            are measurable, and \[
                \iint_{X\times Y} f\:d(\mu\times \nu) = \int_X \varphi\:d\mu = \int_Y
                \psi\:d\nu.
            \] 

            \item If $f \in L^1(\mu\times \nu)$, then $\varphi \in L^1(\mu)$, $\psi
            \in L^1(\nu)$, and \[
                \iint_{X\times Y} f\:d(\mu\times \nu) = \int_X \varphi\:d\mu = \int_Y
                \psi\:d\nu < \infty.
            \] 
        \end{enumerate}

        \begin{remark}
            If $f\colon X \times Y \to \C$ is measurable, we have $|f| \geq 0$
            measurable, hence \begin{align*}
                \varphi^*&\colon X \to [0, \infty], \qquad x \mapsto \int_Y |f_x|\:d\nu,
                \\
                \psi^*&\colon Y \to [0, \infty], \qquad y \mapsto \int_X |f_y|\:d\mu
            \end{align*}
            measurable. This immediately gives \[
                \iint_{X \times Y} |f|\:d(\mu\times\nu) = \int_X \varphi^*\:d\mu =
                \int_Y \psi^*\:d\nu.
            \] If either \[
                \int_X \varphi^*\:d\mu < \infty, \qquad
                \int_Y \psi^*\:d\nu < \infty,
            \] we obtain $f \in L^1(\mu\times\nu)$, and can use part \emph{2}.
        \end{remark}
    \end{theorem}

    \begin{corollary}
        Let $\{a_{mn}\}_{(m, n) \in \N\times\N}$ be a doubly-indexed sequence.
        \begin{enumerate}
            \item If each $a_{mn} \geq 0$, then \[
                 \sum_{m = 1}^\infty \sum_{n = 1}^\infty a_{mn} = \sum_{n = 1}^\infty
                 \sum_{m = 1}^\infty a_{mn}.
            \]
            \item If each $a_{mn} \in \C$ but either \[
                \sum_{m = 1}^\infty \sum_{n = 1}^\infty |a_{mn}| < \infty, \qquad
                \sum_{n = 1}^\infty \sum_{m = 1}^\infty |a_{mn}| < \infty,
            \] then the interchange of summations can be performed as in part 1.
        \end{enumerate}
    \end{corollary}

    \begin{example}
        Consider the sequence defined as \[
            a_{mn} = \begin{cases}
                1, &\text{ if } m = n, \\
                -1, &\text{ if } m + 1 = n, \\
                0, &\text{ otherwise}.
            \end{cases}
        \] Then, \[
            \sum_{m = 1}^\infty\sum_{n = 1}^\infty a_{mn} = 0, \qquad
            \sum_{n = 1}^\infty\sum_{m = 1}^\infty a_{mn} = 1.
        \] 
    \end{example}


    \begin{theorem}
        Let $(X, \M, \mu)$ be a finite measure space and let $f\colon X \to \C$ be
        measurable. Set \[
            E(t) = \{x \in X: |f(x)| > t\},
        \] and define the distribution function of $f$ as \[
            F\colon [0, \infty) \to [0, \infty], \qquad t \mapsto \mu(E(t)).
        \] Then, \[
            \int_{[0, \infty)} F\:dm_1 = \int_X |f|\:d\mu.
        \] 
    \end{theorem}
    \begin{proof}
        Using Fubini-Tonelli, write \[
            \int_{[0, \infty)} F\:dm_1 = \int_{[0, \infty)} \int_X
            \chi_{E(t)}(x)\:d\mu(x)\,dm_1(t) = \int_X \int_{[0, \infty)}
            \chi_{E(t)}(x)\:dm_1(t)\,d\mu(x).
        \] Now, note that \[
            \chi_{E(t)}(x) = \chi_{[0, |f(x)|)}(t) = \begin{cases}
                1, &\text{ if } |f(x)| > t, \\
                0, &\text{ otherwise}.
            \end{cases}
        \] Thus, \[
            \int_{[0, \infty)} F\:dm_1 = \int_{X}\int_{[0, \infty)} \chi_{[0,
            |f(x)|)}(t) \:dm_1(t)\,d\mu(x) = \int_X |f(x)|\:d\mu(x). \qedhere
        \] 
    \end{proof}

    \begin{theorem}
        Let \[
            \Phi\colon \R^n\setminus\{0\} \to (0, \infty) \times S^{n - 1}, \qquad
            x \mapsto (\norm{x}, x / \norm{x}).
        \] This is clearly a homeomorphism. Define the measure \[
            m_n'\colon \mathcal{B}_{(0, \infty) \times S^{n - 1}} \to [0, \infty],
            \qquad E \mapsto m_n(\Phi^{-1}(E)).
        \] Then, there exists a measure $\rho$ on $(0, \infty)$ and a surface measure
        $\sigma$ on $S^{n - 1}$ such that $m_n'$ factors as the product \[
            m_n' = \rho \times \sigma.
        \] Furthermore, \[
            \rho\colon \mathcal{B}_{(0, \infty)}\to [0, \infty], \qquad
            E \mapsto \int_E r^{n - 1}\:dm_1(r),
        \] and \[
            \sigma\colon \mathcal{B}_{S^{n - 1}}\to [0, \infty], \qquad
            F \mapsto n\cdot m_n\{rx: x \in F, 0 \leq r \leq 1\}.
        \] 
    \end{theorem}

    \begin{lemma}
        If $f\colon \R^n \to \C$ is Borel measurable, and either $f \geq 0$ or $f\in
        L^1(m_n)$, then \[
            \int_{\R^n} f\:dm_n = \int_{(0, \infty)} \int_{S^{n - 1}} f(r\hat{x})
            \,r^{n - 1}\:d\sigma(\hat{x})\,dm_1(r).
        \] Furthermore, if $f$ is radial, i.e.\ $f(x) = g(|x|)$ for some $g\colon (0,
        \infty) \to \C$, then \[
            \int_{\R^n} f\:dm_n = \sigma(S^{n - 1}) \int_{(0, \infty)} g(r)\,r^{n -
            1}\:dm_1(r).
        \] 
    \end{lemma}

    \begin{example}
        Consider the integrals \[
            I_n = \int_{\R^n} e^{-\norm{x}^2}\:dm_n.
        \] For $n = 2$, we can apply the polar formula to write \[
            I_2 = \sigma(S^1)\int_0^\infty e^{-r^2} r\:dr = \pi.
        \] Note that we have used $\sigma(S^1) = 2m_2(B^2) = 2\pi$. Now, by repeated
        application of Fubini-Tonelli, we have \[
            I_n = \underbrace{\int_\R \dots \int_\R}_{n\text{ times}}
            e^{-\sum_{i = 1}^n x_i^2}\:\underbrace{dm_1 \dots dm_1}_{n\text{ times}}
            = \prod_{i = 1}^n \int_\R e^{-x_i^2} \:dm_1(x_i) = I_1^n.
        \] Thus, $I_2 = \pi$ gives $I_1 = \sqrt{\pi}$, hence $I_n = \pi^{n / 2}$. The
        polar formula also gives \[
            I_n = \sigma(S^{n - 1})\int_0^\infty e^{-r^2} r^{n - 1}\:dx.
        \] Making the substitution $r = x^2$ yields \[
            \pi^{n / 2} = \sigma(S^{n - 1}) \cdot \frac{1}{2} \int_0^\infty e^{-u}
            u^{n /2 - 1}\:du = \frac{1}{2} \sigma(S^{n - 1})
            \,\Gamma\left(\frac{n}{2}\right).
        \] Thus, \[
            \sigma(S^{n - 1}) = \frac{2\pi^{n / 2}}{\Gamma(n / 2)}.
        \] Using $\sigma(S^{n - 1}) = nm_n(B^n)$, we have the volume of the unit
        $n$-ball \[
            m_n(B^n) = \frac{\pi^{n / 2}}{(n / 2)\Gamma(n / 2)} = \frac{\pi^{n /
            2}}{\Gamma(n / 2 + 1)}.
        \] Putting $n = 1$, $m_1(B^1) = 2$, we compute $\Gamma(1 / 2) = \sqrt{\pi}$.
    \end{example}

\end{document}
