\documentclass[11pt]{article}

\usepackage[T1]{fontenc}
\usepackage{geometry}
\usepackage{amsmath, amssymb, amsthm}
\usepackage[scr]{rsfso}
\usepackage{bm}
\usepackage[%
    hidealllines=true,%
    innerbottommargin=15,%
    nobreak=true,%
]{mdframed}
\usepackage{xcolor}
\usepackage{graphicx}
\usepackage{fancyhdr}
\usepackage{hyperref}

\geometry{a4paper, margin=1in, headheight=14pt}

\pagestyle{fancy}
\fancyhf{}
\renewcommand\headrulewidth{0.4pt}
\fancyhead[L]{\scshape MA3203: Analysis IV}
\fancyhead[R]{\scshape \leftmark}
\rfoot{\footnotesize\it Updated on \today}
\cfoot{\thepage}

\newcommand{\C}{\mathbb{C}}
\newcommand{\R}{\mathbb{R}}
\newcommand{\Q}{\mathbb{Q}}
\newcommand{\Z}{\mathbb{Z}}
\newcommand{\N}{\mathbb{N}}

\newcommand{\M}{\mathcal{M}}

\newcommand{\ip}[2]{\langle #1, #2 \rangle}
\newcommand{\norm}[1]{\Vert #1 \Vert}
\renewcommand{\vec}[1]{\boldsymbol{#1}}

\newcommand{\vx}{\vec{x}}
\newcommand{\vy}{\vec{y}}
\newcommand{\vv}{\vec{v}}
\newcommand{\vw}{\vec{w}}
\newcommand{\ve}{\vec{e}}

\newcommand{\dd}[2]{\frac{d #1}{d #2}}
\newcommand{\pp}[2]{\frac{\partial #1}{\partial #2}}
\newcommand{\ddn}[3][]{\frac{d^{#1} #2}{d #3^{#1}}}
\newcommand{\ppn}[3][]{\frac{\partial^{#1} #2}{\partial #3^{#1}}}
\newcommand{\grad}{\nabla}

\newmdtheoremenv[%
    backgroundcolor=blue!10!white,%
]{theorem}{Theorem}[section]
\newmdtheoremenv[%
    backgroundcolor=violet!10!white,%
]{corollary}{Corollary}[theorem]
\newmdtheoremenv[%
    backgroundcolor=teal!10!white,%
]{lemma}[theorem]{Lemma}

\theoremstyle{definition}
\newmdtheoremenv[%
    backgroundcolor=green!10!white,%
]{definition}{Definition}[section]
\newmdtheoremenv[%
    backgroundcolor=red!10!white,%
]{exercise}{Exercise}[section]

\theoremstyle{remark}
\newtheorem*{remark}{Remark}
\newtheorem*{example}{Example}
\newtheorem*{solution}{Solution}

\surroundwithmdframed[%
    linecolor=black!20!white,%
    hidealllines=false,%
    innertopmargin=5,%
    innerbottommargin=10,%
    skipabove=0,%
    skipbelow=0,%
]{example}

\numberwithin{equation}{section}

\title{
    \Large\textsc{MA3203} \\
    \Huge \textbf{Analysis IV} \\
    \vspace{5pt}
    \Large{Spring 2022}
}
\author{
    \large Satvik Saha
    \\\textsc{\small 19MS154}
}
\date{\normalsize
    \textit{Indian Institute of Science Education and Research, Kolkata, \\
    Mohanpur, West Bengal, 741246, India.} \\
}

\begin{document}
    \maketitle

    \tableofcontents

    \section{Measure theory}

    \subsection{Introduction}
    Measure theory seeks to generalize the notions of \emph{length}, \emph{area},
    \emph{volume} to more general sets: this new notion is called a \emph{measure}.
    This also allows us to generalize the notion of Riemann integration to a broader
    class of functions.

    Recall that continuous functions, or at least functions with finitely many
    discontinuities on a closed interval are Riemann integrable. The Dirichlet
    function, which is discontinuous everywhere, is not. \[
        f\colon \R \to \R, \qquad x \mapsto \begin{cases}
            1, &\text{ if } x \in \Q, \\
            0, &\text{ if } x \notin \Q.
        \end{cases}
    \] This is simply because every non-empty interval contains at least one rational
    and one irrational number, so the Darboux lower sum is always 0 and the upper sum
    is always 1 regardless of the choice of partition.

    On the other hand, if we had to assign a particular value to this integral,
    intuition tells us that it ought to be zero. After all, the function $f$ attains
    a non-zero value only on the countable set $\Q \cap [0, 1]$; it is zero almost
    everywhere. Formally, we will show that $f$ is non-zero on a set of zero
    Lebesgue measure, which will allow us to set this Lebesgue integral to zero. We
    will see that with this new formulation of integration, we end up partitioning
    the range of $f$ rather than it's domain, and write \[
        \int f = 0\cdot \mu([0, 1]\setminus \Q) + 1\cdot \mu([0, 1] \cap \Q) = 0.
    \] 

    \begin{theorem}[Lebesgue criterion]
        A function $f\colon [a, b] \to \R$ is Riemann integrable if and only if it is
        bounded and its set of discontinuities has Lebesgue measure zero. This means
        that the set of discontinuities of $f$ must be coverable by countably many
        intervals $(x_i, y_i)$ such that the sum of lengths $y_i - x_i$ can be made
        arbitrarily small.
    \end{theorem}
    
    \subsection{Basic definitions}

    \begin{definition}
        Let $X$ be a set, and let $\M$ be a collection of subsets of $X$. We say
        that $\M$ is a $\sigma$-algebra over $X$ if it satisfies the following.
        \begin{enumerate}
            \itemsep0em
            \item $\M$ contains $X$.
            \item $\M$ is closed under complementation.
            \item $\M$ is closed under countable unions.
        \end{enumerate}
        \begin{remark}
            The first condition can be replaced by forcing $\M$ to be non-empty.
        \end{remark}
        \begin{remark}
            The following properties follow immediately.
            \begin{enumerate}
                \itemsep0em
                \item $\M$ contains $\emptyset$.
                \item $\M$ is closed under countable intersections.
                \item $\M$ is closed under differences.
            \end{enumerate}
        \end{remark}
    \end{definition}
    
    \begin{example}
        Given a set $X$, its power set forms a $\sigma$-algebra over $X$.
    \end{example}
    \begin{example}
        Given a set $X$, the set $\{\emptyset, X\}$ forms a $\sigma$-algebra over
        $X$.
    \end{example}
    \begin{example}
        Given an uncountable set $X$, the following set forms the co-countable
        $\sigma$-algebra over $X$. \[
            \{E \subseteq X : E\text{ is countable or } E^c\text{ is countable}\}.
        \] 
    \end{example}


    \begin{definition}
        Let $X$ be a set, and let $\M$ be a $\sigma$-algebra over $X$. We say
        that a function $\mu\colon \M \to \R \cup \{-\infty, +\infty\}$ is called
        a measure if it satisfies the following.
        \begin{enumerate}
            \itemsep0em
            \item $\mu$ is non-negative.
            \item $\mu(\emptyset) = 0$.
            \item $\mu$ is additive over countable unions of disjoint sets, i.e.\ for
            any countable collection $\{E_i\}_{i = 1}^\infty$ such that $E_i \cap E_j
            = \emptyset$ for all pairs, we have \[
                \mu\left(\bigcup_{i = 1}^\infty E_i\right) = \sum_{i = 1}^\infty
                \mu(E_i).
            \] 
        \end{enumerate}
    \end{definition}

    \begin{example}
        The trivial zero measure sends every set to zero.
    \end{example}
    \begin{example}
        In probability theory, we look at the event space $\mathcal{E}$ as a
        $\sigma$-algebra over the sample space $\Omega$. The probability function $P$
        is a measure on this event space such that $P(\Omega) = 1$.
    \end{example}
    \begin{example}
        Let $X$ be a set, and let $\M$ be its power set as a $\sigma$-algebra
        over $X$. Fix $x_0 \in X$, and define \[
            \mu\colon \M \to [0, \infty], \qquad E \mapsto \begin{cases}
                1, &\text{ if } x_0 \in E, \\
                0, &\text{ if } x_0 \notin E. \\
            \end{cases}
        \] This is called the Dirac measure.
    \end{example}

    \begin{example}
        Let $X$ be a set, and let $\M$ be its power set as a $\sigma$-algebra
        over $X$. Define \[
            \mu\colon \M \to [0, \infty], \qquad E \mapsto \begin{cases}
                0, &\text{ if } E = \emptyset, \\
                |E|, &\text{ if } E\text{ is finite},\\
                \infty, &\text{ otherwise}.
            \end{cases}
        \] This is called the counting measure.
    \end{example}

    \begin{example}
        Let $X$ be an uncountable set, and let $\M$ be its co-countable sigma
        algebra. Define \[
            \mu\colon \M \to [0, \infty], \qquad E \mapsto \begin{cases}
                0, &\text{ if } E \text{ is countable},\\
                1, &\text{ if } E^c \text{ is countable}.\\
            \end{cases}
        \] 
    \end{example}


    \begin{definition}
        Let $(X, \M, \mu)$ be a measure space. \begin{enumerate}
            \itemsep0em
            \item We say that $\mu$ is finite if $\mu(E)$ is finite for all $E \in
            \M$.
            \item We say that $\mu$ is $\sigma$-finite if given $E \in \M$, we
            can write \[
                E = \bigcup_{i = 1}^\infty E_i
            \] for $E_i \in \M$ such that each $\mu(E_i)$ is finite.
        \end{enumerate}
    \end{definition}


    \subsection{Basic properties}

    \begin{lemma}
        Let $(X, \M, \mu)$ be a measure space. Then, the following properties
        hold. \begin{enumerate}
            \itemsep0em
            \item If $A, B \in \M$ such that $A \subseteq B$, then $\mu(A) \leq
            \mu(B)$.
            \item If $A, B \in \M$ such that $A \subseteq B$ and $\mu(A)$ is
            finite, then $\mu(B - A) = \mu(B) - \mu(A)$.
            \item If $\{E_i\}_{i = 1}^\infty$ such that $E_i \in \M$, then \[
                \mu\left(\bigcup_{i = 1}^\infty E_i\right) \leq \sum_{i = 1}^\infty
                \mu(E_i).
            \] 
        \end{enumerate}
    \end{lemma}

    \begin{corollary}
        A measure $\mu$ is finite if and only if $\mu(X)$ is finite.
    \end{corollary}

    \begin{theorem}[Continuity from below]
        Let $(X, \M, \mu)$ be a measure space, and let $\{E_i\}_{i = 1}^\infty$
        be a sequence of measurable sets such that $E_i \subseteq E_j$ for all $i <
        j$. Then, \[
            \mu\left(\bigcup_{i = 1}^\infty E_i\right) = \lim_{n \to \infty}
            \mu(E_n).
        \] 
    \end{theorem}
    \begin{proof}
        Define $F_i = E_i - E_{i - 1}$, denoting $E_0 = \emptyset$. Thus, \[
            \mu\left(\bigcup_{i = 1}^\infty E_i\right) = \mu\left(\bigcup_{i =
            1}^\infty F_i\right) = \sum_{i = 1}^\infty \mu(F_i).
        \] Also note that \[
            \sum_{i = 1}^n \mu(F_i) = \mu\left(\bigcup_{i = 1}^\infty F_n\right) =
            \mu(E_n).
        \] Since the infinite sum in the first part is the limit of partial sums, we
        have our result.
    \end{proof}
    
    \begin{theorem}[Continuity from above]
        Let $(X, \M, \mu)$ be a measure space, and let $\{E_i\}_{i = 1}^\infty$
        be a sequence of measurable sets such that $E_i \supseteq E_j$ for all $i <
        j$. Further assume that $\mu(E_1)$ is finite. Then, \[
            \mu\left(\bigcap_{i = 1}^\infty E_i\right) = \lim_{n \to \infty}
            \mu(E_n).
        \] 
    \end{theorem}
    \begin{proof}
        Define $F_i = E_1 - E_n$, and note that $F_i \subseteq F_j$ for all $i < j$.
        Thus, \[
            \mu\left(\bigcup_{i = 1}^\infty F_i\right) = \lim_{n \to \infty}
            \mu(F_n).
        \] This can be rewritten as \[
            \mu\left(E_1 - \bigcap_{i = 1}^\infty E_i\right) = \lim_{n \to \infty}
            \mu(E_1 - E_n).
        \] Using the subtractive property and the fact that each $\mu(E_i)$ is
        finite, \[
            \mu(E_1) - \mu\left(\bigcap_{i = 1}^\infty E_i\right) = \lim_{n \to \infty}
            \mu(E_1) - \mu(E_n).
        \] Pulling the constant $\mu(E_1)$ out from the limit and subtracting from
        both sides gives our result.
    \end{proof}
    \begin{example}
        Consider the counting measure $\mu$ on $(\N, \mathcal{P}(\N))$, and define
        $E_n = \{n, n + 1, \dots \}$. Then, \[
            \mu\left(\bigcap_{i = 1}^\infty E_i\right) = 0, \qquad \lim_{n \to
            \infty} \mu(E_n) = \infty.
        \] 
    \end{example}


    \subsection{The Borel $\sigma$-algebra}

    \begin{theorem}
        Let $X$ be a set, and let $S$ be a collection of subsets of $X$. Then, there
        exists a smallest $\sigma$-algebra containing $S$. This is called the
        $\sigma$ algebra generated by $S$, denoted $\M(S)$.
    \end{theorem}
    \begin{proof}
        Let $\Omega$ be the collection of all $\sigma$-algebras on $X$ containing
        $S$. Note that $\Omega \neq \emptyset$, since it contains the power set of
        $X$. Consider the intersection of all the sigma algebras in $\Omega$, \[
            \M = \bigcap_{\M_\lambda \in \Omega} \M_\lambda.
        \] We claim that $\M$ is indeed a $\sigma$-algebra. To see this, first note
        that $X \in \M$. Next, pick $E \in \M \subseteq \M_\lambda$, so $E^c \in
        \M_\lambda$ for all $\M_\lambda \in \Omega$, hence $E^c \in \M$. Finally,
        pick $\{E_i\}_{i = 1}^\infty$ where $E_i \in \M \subseteq \M_\lambda$, which
        shows that the union of these $E_i$ is in every $\M_\lambda$, hence in $\M$.
    \end{proof}

    \begin{definition}
        Let $(X, \tau)$ be a topological space. The $\sigma$ algebra generated by
        $\tau$ is called the Borel $\sigma$-algebra, $\mathcal{B}_X = \M(\tau)$.
        \begin{remark}
            The Borel $\sigma$-algebra $\mathcal{B}_X$ contains all open as well as
            all closed sets in $X$, as well as their countable unions and
            intersections.
        \end{remark}
    \end{definition}

    \begin{theorem}
        Consider the collection $\beta$ of open intervals in $\R$, and the standard
        topology $\tau$ on $\R$. Then, both $\beta$ and $\tau$ generate the same
        Borel $\sigma$-algebra $\mathcal{B}_\R$.
    \end{theorem}
    \begin{proof}
        This relies on the fact that every open set $U \subseteq \R$ can be written
        as a countable union of open intervals. To see this, pick $x \in U$, and an
        open interval $x \in (a, b) \subset U$. Now pick $p, q \in \Q$ such that $a <
        p < x < q < b$, hence $x \in (p, q) \subset U$. Now, $U$ is precisely the
        union of all such intervals $(p, q)$. This collection is countable, due to
        the countability of the rationals.
    \end{proof}
    \begin{remark}
        The same holds if we consider the collection $\beta'$ of closed intervals in
        $\R$. This can be shown using the standard trick \[
            (a, b) = \bigcup_{n = 1}^\infty \left[a + \frac{1}{n}, b -
            \frac{1}{n}\right].
        \] Indeed, we may also consider the collection of intervals of the form $[a,
        b)$, or the collection of intervals $(a, b]$, or even the collection of
        intervals $(a, \infty)$, or $(-\infty, b)$, or $[a, \infty)$, or $(-\infty,
        b]$.
    \end{remark}

    \begin{definition}
        A countable union of closed sets is called an $F_\sigma$ set. A countable
        intersection of open sets is called a $G_\delta$ set.
    \end{definition}

    
    \subsection{Measurable functions}

    \begin{definition}
        Let $(X, \M_X)$, $(Y, \M_Y)$ be measure spaces. We say that a function
        $f\colon X \to Y$ is $(\M_X, \M_Y)$ measurable if for every $E \in \M_Y$, we
        have $f^{-1}(E) = \M_X$.
    \end{definition}
    \begin{example}
        Consider the Borel $\sigma$-algebra $\mathcal{B}_\R$ on $\R$, and fix $E \in
        \mathcal{B}_\R$. Define the characteristic function \[
            \chi_E\colon \R \to \R, \qquad x \mapsto \begin{cases}
                1, &\text{ if } x \in E, \\
                0, &\text{ if } x \notin E.
            \end{cases}
        \] Then, $\chi_E$ is measurable. However, we can choose $E$ to be closed and
        not open, so that $\chi_E$ is not continuous.
    \end{example}

    \begin{lemma}
        Let $f\colon (X, \M_X) \to (Y, \M_Y)$, and let $\M_Y$ be generated by $S$.
        Then $f\colon X\to Y$ is measurable if for every $E \in S$, we have
        $f^{-1}(E) \in \M_X$.
    \end{lemma}
    \begin{proof}
        Define \[
            \M = \{E \subseteq Y: f^{-1}(E) \in \M_X\}.
        \] Clearly, $S \subseteq \M$. We now claim that $\M$ is a $\sigma$-algebra
        over $Y$. First, $Y \in \M$ since $f^{-1}(Y) = X \in \M_X$. Next if $E \in
        \M$, we have $f^{-1}(E^c) = f^{-1}(E)^c \in \M_X$. Finally, if $\{E_i\}_{i =
        1}^\infty$ such that each $E_i \in \M$, set $E$ to be their union, whence \[
            f^{-1}(E) = \bigcup_{i = 1}^\infty f^{-1}(E_i) \in \M.
        \] Thus, $\M$ is indeed a $\sigma$-algebra. Since $S$ generates $\M_Y$, we
        have $\M_Y \subseteq \M$, completing the proof.
    \end{proof}
    \begin{example}
        A function $f\colon X \to \R$ is measurable if and only if $f^{-1}((a,
        \infty))$ is measurable for all $a \in \R$.
    \end{example}


    \begin{theorem}
        Let $f\colon X \to Y$ be continuous. Then, $f$ is $(\mathcal{B}_X,
        \mathcal{B}_Y)$ measurable.
    \end{theorem}

    \begin{lemma}
        The composition of measurable functions is measurable. In other words, if
        $f\colon (X, \M_X) \to (Y, \M_Y)$ is surjective and $(\M_X, \M_Y)$
        measurable, and $g\colon (Y, \M_Y) \to (Z, \M_Z)$ is $(\M_Y, \M_Z)$
        measurable, then $g\circ f$ is $(\M_X, \M_Z)$ measurable.
    \end{lemma}

    \begin{lemma}
        Let $u, v\colon (X, \M) \to (\R, \mathcal{B}_\R)$ be $(\M, \mathcal{B}_\R)$
        measurable. Then, $f\colon (X, \M) \to (\R^2, \mathcal{B}_{\R^2})$, defined
        by $x \mapsto (u(x), v(x))$, is $(\M, \mathcal{B}_{\R^2})$ measurable.
    \end{lemma}
    \begin{proof}
        Basic open sets in $\R^2$ can be chosen as the open rectangles $(a, b) \times
        (c, d)$. The pre-image of such an open set under $f$ is $u^{-1}((a, b)) \cap
        v^{-1}((c, d))$, which is clearly a measurable set in $X$.
    \end{proof}
    
    \begin{corollary}
        Let $f, g\colon (X, \M) \to (\R, \mathcal{B}_\R)$ be $(\M, \mathcal{B}_\R)$
        measurable. Then the sum $f + g$ and the product $u\cdot g$ are also $(\M,
        \mathcal{B}_\R)$ measurable.
    \end{corollary}
    \begin{proof}
        The maps $(x, y) \mapsto x + y$ and $(x, y) \mapsto xy$ are continuous, hence
        measurable. Thus, the composite maps $x \mapsto (f(x), g(x)) \mapsto f(x) +
        g(x)$ and $x \mapsto (f(x), g(x)) \mapsto f(x)g(x)$ are also measurable.
    \end{proof}

    \begin{example}
        Let $(X, \M)$ be a measurable space, and let $A_1, \dots, A_n \in \M$. Then
        the map \[
            s\colon X \to \R, \qquad x \mapsto \sum_{i = 1}^n c_i \chi_{A_i}(x)
        \] is measurable. Such functions are called simple functions.
    \end{example}
    
    \begin{lemma}
        The maximum and minimum of measurable functions are measurable.
    \end{lemma}
    \begin{corollary}
        The positive and negative parts of a measurable function are measurable.
        \begin{remark}
            Recall that \[
                f^+ = \max\{f, 0\}, \qquad f^{-1} = -\min\{f, 0\}, \qquad f = f^+ -
                f^-.
            \] Thus, in order to show that a result holds for all measurable
            functions, it suffices to show the result only for all non-negative
            measurable functions.
        \end{remark}
    \end{corollary}
    

    \begin{theorem}
        Let $\{f_n\}_{n = 1}^\infty$ be a collection of measurable functions $f\colon
        X \to \R \cup \{-\infty, +\infty\}$, and let $f_n \to f$ pointwise on $X$.
        Then, their supremum and infinimum are measurable.
    \end{theorem}
    
    \begin{theorem}
        Let $\{f_n\}_{n = 1}^\infty$ be a sequence of measurable functions $f\colon X
        \to \R$. Then, $\limsup_{n \to \infty} f_n$ and $\liminf_{n \to \infty} f_n$
        are measurable.
    \end{theorem}
    
    \begin{theorem}
        Let $\{f_n\}_{n = 1}^\infty$ be a sequence of measurable functions $f\colon X
        \to \R$, and let $f_n \to f$ pointwise on $X$. Then, $f$ is measurable.
        \begin{remark}
            This is a stronger result than the corresponding one regarding limits of
            continuous functions.
        \end{remark}
    \end{theorem}

    

    \subsection{Lebesgue integration}

    \begin{theorem}
        Let $f\colon X \to [0, \infty]$ be measurable. Then, there exists a sequence
        of simple functions $s_n\colon X \to [0, \infty)$ such that \[
            0 \leq s_1 \leq s_2 \leq \dots \leq s_n \leq f
        \] for all $n \in \N$, and $s_n \to f$.
    \end{theorem}
    \begin{corollary}
        Any measurable function $f\colon X \to [-\infty, \infty]$ can be written as
        the limit of a sequence of simple functions $\{s_n\}_{n = 1}^\infty$, $s_n
        \to \infty$.
    \end{corollary}

    \begin{definition}
        When dealing with the extended reals in measure theory, we use the convention
        $0\cdot \infty = \infty \cdot 0 = 0$.
        \begin{remark}
            We want to have \[
                \int_0^\infty 0 = 0\cdot \mu(f^{-1}(0)) = 0\cdot \mu([0, \infty)) =
                0.
            \] 
        \end{remark}
    \end{definition}

    
    \begin{definition}
        Let $(X, \M, \mu)$ be a measure space, and let $s\colon X \to [0, \infty)$ be
        a simple, measurable function, of the form \[
            s = \sum_{i = 1}^n c_i\chi_{A_i}, \qquad A_i \in \M
        \] where $c_1, \dots, c_n$ are distinct values of $s$. Then, the Lebesgue
        integral of $s$ on $E \in \M$ is defined as \[
            \int_E s\:d\mu = \sum_{i = 1}^n c_i\cdot \mu(E \cap A_i).
        \] 
    \end{definition}
    \begin{example}
        The Dirichlet function is the simple function $x\mapsto \chi_\Q$. Thus, upon
        assigning a $\sigma$-algebra and a measure $\mu$ on $\R$, we will be able to
        assign its Lebesgue integral on $\R$ as the value $\mu(\Q)$.
    \end{example}
    
    \begin{definition}
        Let $(X, \M, \mu)$ be a measure space, and let $f\colon X \to [0, \infty)$ be
        a measurable function. Then, the Lebesgue integral of $f$ on $E \in \M$ is
        defined as \[ 
            \int_E f\:d\mu = \sup\left\{\int_E s\:d\mu, \text{ for all simple
            functions }s, \text{where }0 \leq s \leq f\right\}.
        \] 
    \end{definition}

    \begin{theorem}
        Let $f, g\colon X \to [0, \infty]$ be measurable functions.
        \begin{enumerate}
            \item If $0 \leq f \leq g$, then \[
                \int_E f\:d\mu \leq \int_E g\:d\mu.
            \] 
            \item If $A \subset B$, then \[
                \int_A f\:d\mu \leq \int_B f\:d\mu.
            \] 
            \item For $c \in \R$, \[
                \int_E cf\:d\mu = c\int_E f\:d\mu.
            \] 
            \item If $f = 0$, then \[
                \int_E f\:d\mu = 0.
            \] 
            \item If $\mu(E) = 0$, then \[
                \int_E f\:d\mu = 0.
            \] 
        \end{enumerate}
    \end{theorem}

    \begin{definition}
        We say that a statement is true \emph{almost everywhere} on $X$ if it is true
        everywhere on $X\setminus E$ for a measure zero set $E$.
    \end{definition}
    
    \begin{lemma}
        Let $f\colon X \to [0, \infty]$ be a measurable function. If \[
            \int_X f\:d\mu = 0,
        \] then $f = 0$ almost everywhere.
    \end{lemma}
    \begin{proof}
        We wish to show that the set $E = f^{-1}(0, \infty]$ has measure zero. Now,
        note that this is the union of the measurable sets $E_n = f^{-1}(1 / n,
        \infty]$. Since $E_n \subset E \subseteq X$, we have \[
            \int_{E_n} \frac{1}{n}\:d\mu \leq \int_{E_n} f\:d\mu \leq \int_X f\:d\mu.
        \] However, this is just \[
            0 \leq \frac{1}{n}\mu(E_n) \leq \int_{E_n} f\:d\mu \leq 0,
        \] hence each $\mu(E_n) = 0$. Continuity from below gives $\mu(E) = 0$.
    \end{proof}


\end{document}
