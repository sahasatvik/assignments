\documentclass[11pt]{article}

\usepackage[T1]{fontenc}
\usepackage{geometry}
\usepackage{amsmath, amssymb, amsthm}
\usepackage[scr]{rsfso}
\usepackage{bm}
\usepackage[%
    hidealllines=true,%
    innerbottommargin=15,%
    nobreak=true,%
]{mdframed}
\usepackage{xcolor}
\usepackage{graphicx}
\usepackage{fancyhdr}
\usepackage{hyperref}

\geometry{a4paper, margin=1in, headheight=14pt}

\pagestyle{fancy}
\fancyhf{}
\renewcommand\headrulewidth{0.4pt}
\fancyhead[L]{\scshape MA3203: Analysis IV}
\fancyhead[R]{\scshape \leftmark}
\rfoot{\footnotesize\it Updated on \today}
\cfoot{\thepage}

\newcommand{\C}{\mathbb{C}}
\newcommand{\R}{\mathbb{R}}
\newcommand{\Q}{\mathbb{Q}}
\newcommand{\Z}{\mathbb{Z}}
\newcommand{\N}{\mathbb{N}}

\newcommand{\ip}[2]{\langle #1, #2 \rangle}
\newcommand{\norm}[1]{\Vert #1 \Vert}
\renewcommand{\vec}[1]{\boldsymbol{#1}}

\newcommand{\vx}{\vec{x}}
\newcommand{\vy}{\vec{y}}
\newcommand{\vv}{\vec{v}}
\newcommand{\vw}{\vec{w}}
\newcommand{\ve}{\vec{e}}

\newcommand{\dd}[2]{\frac{d #1}{d #2}}
\newcommand{\pp}[2]{\frac{\partial #1}{\partial #2}}
\newcommand{\ddn}[3][]{\frac{d^{#1} #2}{d #3^{#1}}}
\newcommand{\ppn}[3][]{\frac{\partial^{#1} #2}{\partial #3^{#1}}}
\newcommand{\grad}{\nabla}

\newmdtheoremenv[%
    backgroundcolor=blue!10!white,%
]{theorem}{Theorem}[section]
\newmdtheoremenv[%
    backgroundcolor=violet!10!white,%
]{corollary}{Corollary}[theorem]
\newmdtheoremenv[%
    backgroundcolor=teal!10!white,%
]{lemma}[theorem]{Lemma}

\theoremstyle{definition}
\newmdtheoremenv[%
    backgroundcolor=green!10!white,%
]{definition}{Definition}[section]
\newmdtheoremenv[%
    backgroundcolor=red!10!white,%
]{exercise}{Exercise}[section]

\theoremstyle{remark}
\newtheorem*{remark}{Remark}
\newtheorem*{example}{Example}
\newtheorem*{solution}{Solution}

\surroundwithmdframed[%
    linecolor=black!20!white,%
    hidealllines=false,%
    innertopmargin=5,%
    innerbottommargin=10,%
    skipabove=0,%
    skipbelow=0,%
]{example}

\numberwithin{equation}{section}

\title{
    \Large\textsc{MA3203} \\
    \Huge \textbf{Analysis IV} \\
    \vspace{5pt}
    \Large{Spring 2022}
}
\author{
    \large Satvik Saha
    \\\textsc{\small 19MS154}
}
\date{\normalsize
    \textit{Indian Institute of Science Education and Research, Kolkata, \\
    Mohanpur, West Bengal, 741246, India.} \\
}

\begin{document}
    \maketitle

    \tableofcontents

    \section{Measure theory}

    \subsection{Introduction}
    Measure theory seeks to generalize the notions of \emph{length}, \emph{area},
    \emph{volume} to more general sets: this new notion is called a \emph{measure}.
    This also allows us to generalize the notion of Riemann integration to a broader
    class of functions.

    Recall that continuous functions, or at least functions with finitely many
    discontinuities on a closed interval are Riemann integrable. The Dirichlet
    function, which is discontinuous everywhere, is not. \[
        f\colon \R \to \R, \qquad x \mapsto \begin{cases}
            1, &\text{ if } x \in \Q, \\
            0, &\text{ if } x \notin \Q.
        \end{cases}
    \] This is simply because every non-empty interval contains at least one rational
    and one irrational number, so the Darboux lower sum is always 0 and the upper sum
    is always 1 regardless of the choice of partition.

    On the other hand, if we had to assign a particular value to this integral,
    intuition tells us that it ought to be zero. After all, the function $f$ attains
    a non-zero value only on the countable set $\Q \cap [0, 1]$; it is zero almost
    everywhere. Formally, we will show that $f$ is non-zero on a set of zero
    Lebesgue measure, which will allow us to set this Lebesgue integral to zero. We
    will see that with this new formulation of integration, we end up partitioning
    the range of $f$ rather than it's domain, and write \[
        \int f = 0\cdot \mu([0, 1]\setminus \Q) + 1\cdot \mu([0, 1] \cap \Q) = 0.
    \] 

    \begin{theorem}[Lebesgue criterion]
        A function $f\colon [a, b] \to \R$ is Riemann integrable if and only if it is
        bounded and its set of discontinuities has Lebesgue measure zero. This means
        that the set of discontinuities of $f$ must be coverable by countably many
        intervals $(x_i, y_i)$ such that the sum of lengths $y_i - x_i$ can be made
        arbitrarily small.
    \end{theorem}
    
    \subsection{Basic definitions}

    \begin{definition}
        Let $X$ be a set, and let $\Sigma$ be a collection of subsets of $X$. We say
        that $\Sigma$ is a $\sigma$-algebra over $X$ if it satisfies the following.
        \begin{enumerate}
            \itemsep0em
            \item $\Sigma$ contains $X$.
            \item $\Sigma$ is closed under complementation.
            \item $\Sigma$ is closed under countable unions.
        \end{enumerate}
        \begin{remark}
            The first condition can be replaced by forcing $\Sigma$ to be non-empty.
        \end{remark}
        \begin{remark}
            The following properties follow immediately.
            \begin{enumerate}
                \itemsep0em
                \item $\Sigma$ contains $\emptyset$.
                \item $\Sigma$ is closed under countable intersections.
                \item $\Sigma$ is closed under differences.
            \end{enumerate}
        \end{remark}
    \end{definition}
    
    \begin{example}
        Given a set $X$, its power set forms a $\sigma$-algebra over $X$.
    \end{example}
    \begin{example}
        Given a set $X$, the set $\{\emptyset, X\}$ forms a $\sigma$-algebra over
        $X$.
    \end{example}
    \begin{example}
        Given an uncountable set $X$, the following set forms the co-countable
        $\sigma$-algebra over $X$. \[
            \{E \subseteq X : E\text{ is countable or } E^c\text{ is countable}\}.
        \] 
    \end{example}


    \begin{definition}
        Let $X$ be a set, and let $\Sigma$ be a $\sigma$-algebra over $X$. We say
        that a function $\mu\colon \Sigma \to \R \cup \{-\infty, +\infty\}$ is called
        a measure if it satisfies the following.
        \begin{enumerate}
            \itemsep0em
            \item $\mu$ is non-negative.
            \item $\mu(\emptyset) = 0$.
            \item $\mu$ is additive over countable unions of disjoint sets, i.e.\ for
            any countable collection $\{E_i\}_{i = 1}^\infty$ such that $E_i \cap E_j
            = \emptyset$ for all pairs, we have \[
                \mu\left(\bigcup_{i = 1}^\infty E_i\right) = \sum_{i = 1}^\infty
                \mu(E_i).
            \] 
        \end{enumerate}
    \end{definition}

    \begin{example}
        The trivial zero measure sends every set to zero.
    \end{example}
    \begin{example}
        In probability theory, we look at the event space $\mathcal{E}$ as a
        $\sigma$-algebra over the sample space $\Omega$. The probability function $P$
        is a measure on this event space such that $P(\Omega) = 1$.
    \end{example}
    \begin{example}
        Let $X$ be a set, and let $\Sigma$ be its power set as a $\sigma$-algebra
        over $X$. Fix $x_0 \in X$, and define \[
            \mu\colon \Sigma \to [0, \infty], \qquad E \mapsto \begin{cases}
                1, &\text{ if } x_0 \in E, \\
                0, &\text{ if } x_0 \notin E. \\
            \end{cases}
        \] This is called the Dirac measure.
    \end{example}

    \begin{example}
        Let $X$ be a set, and let $\Sigma$ be its power set as a $\sigma$-algebra
        over $X$. Define \[
            \mu\colon \Sigma \to [0, \infty], \qquad E \mapsto \begin{cases}
                0, &\text{ if } E = \emptyset, \\
                |E|, &\text{ if } E\text{ is finite},\\
                \infty, &\text{ otherwise}.
            \end{cases}
        \] This is called the counting measure.
    \end{example}

    \begin{example}
        Let $X$ be an uncountable set, and let $\Sigma$ be its co-countable sigma
        algebra. Define \[
            \mu\colon \Sigma \to [0, \infty], \qquad E \mapsto \begin{cases}
                0, &\text{ if } E \text{ is countable},\\
                1, &\text{ if } E^c \text{ is countable}.\\
            \end{cases}
        \] 
    \end{example}

    \begin{lemma}
        Let $(X, \Sigma, \mu)$ be a measure space. Then, the following properties
        hold. \begin{enumerate}
            \itemsep0em
            \item If $A, B \in \Sigma$ such that $A \subseteq B$, then $\mu(A) \leq
            \mu(B)$.
            \item If $A, B \in \Sigma$ such that $A \subseteq B$ and $\mu(A)$ is
            finite, then $\mu(B - A) = \mu(B) - \mu(A)$.
            \item If $\{E_i\}_{i = 1}^\infty$ such that $E_i \in \Sigma$, then \[
                \mu\left(\bigcup_{i = 1}^\infty E_i\right) \leq \sum_{i = 1}^\infty
                \mu(E_i).
            \] 
        \end{enumerate}
    \end{lemma}

    \begin{theorem}[Continuity from below]
        Let $(X, \Sigma, \mu)$ be a measure space, and let $\{E_i\}_{i = 1}^\infty$
        be a sequence of measurable sets such that $E_i \subseteq E_j$ for all $i <
        j$. Then, \[
            \mu\left(\bigcup_{i = 1}^\infty E_i\right) = \lim_{n \to \infty}
            \mu(E_n).
        \] 
    \end{theorem}
    \begin{proof}
        Define $F_i = E_i - E_{i - 1}$, denoting $E_0 = \emptyset$. Thus, \[
            \mu\left(\bigcup_{i = 1}^\infty E_i\right) = \mu\left(\bigcup_{i =
            1}^\infty F_i\right) = \sum_{i = 1}^\infty \mu(F_i).
        \] Also note that \[
            \sum_{i = 1}^n \mu(F_i) = \mu\left(\bigcup_{i = 1}^\infty F_n\right) =
            \mu(E_n).
        \] Since the infinite sum in the first part is the limit of partial sums, we
        have our result.
    \end{proof}
    
    \begin{theorem}[Continuity from above]
        Let $(X, \Sigma, \mu)$ be a measure space, and let $\{E_i\}_{i = 1}^\infty$
        be a sequence of measurable sets such that $E_i \supseteq E_j$ for all $i <
        j$. Further assume that $\mu(E_1)$ is finite. Then, \[
            \mu\left(\bigcap_{i = 1}^\infty E_i\right) = \lim_{n \to \infty}
            \mu(E_n).
        \] 
    \end{theorem}
    \begin{proof}
        Define $F_i = E_1 - E_n$, and note that $F_i \subseteq F_j$ for all $i < j$.
        Thus, \[
            \mu\left(\bigcup_{i = 1}^\infty F_i\right) = \lim_{n \to \infty}
            \mu(F_n).
        \] This can be rewritten as \[
            \mu\left(E_1 - \bigcap_{i = 1}^\infty E_i\right) = \lim_{n \to \infty}
            \mu(E_1 - E_n).
        \] Using the subtractive property and the fact that each $\mu(E_i)$ is
        finite, \[
            \mu(E_1) - \mu\left(\bigcap_{i = 1}^\infty E_i\right) = \lim_{n \to \infty}
            \mu(E_1) - \mu(E_n).
        \] Pulling the constant $\mu(E_1)$ out from the limit and subtracting from
        both sides gives our result.
    \end{proof}
    \begin{example}
        Consider the counting measure $\mu$ on $(\N, \mathcal{P}(\N))$, and define
        $E_n = \{n, n + 1, \dots \}$. Then, \[
            \mu\left(\bigcap_{i = 1}^\infty E_i\right) = 0, \qquad \lim_{n \to
            \infty} \mu(E_n) = \infty.
        \] 
    \end{example}

    


\end{document}
