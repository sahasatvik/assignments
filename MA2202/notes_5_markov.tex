\documentclass[11pt]{article}

\usepackage[T1]{fontenc}
\usepackage{geometry}
\usepackage{amsmath, amssymb, amsthm}
\usepackage[%
    hidealllines=true,%
    innerbottommargin=15,%
    nobreak=true,%
]{mdframed}
\usepackage{xcolor}
\usepackage{graphicx}
\usepackage{fancyhdr}
\usepackage{lipsum}
\usepackage{bm}

\geometry{a4paper, margin=1in, headheight=14pt}

\pagestyle{fancy}
\fancyhf{}
\renewcommand\headrulewidth{0.4pt}
\fancyhead[L]{\scshape MA2201: Probability I}
\fancyhead[R]{\scshape Markov chains}
\rfoot{\footnotesize\it Updated on \today}
\cfoot{\thepage}

\def\C{\mathbb{C}}
\def\R{\mathbb{R}}
\def\Q{\mathbb{Q}}
\def\Z{\mathbb{Z}}
\def\N{\mathbb{N}}
\newcommand\ddx[1]{\frac{d #1}{d x}}
\newcommand\ddt[1]{\frac{d #1}{d t}}
\newcommand\dd[3][]{\frac{d^{#1}{#2}}{d {#3}^{#1}}}
\newcommand\ppx[1]{\frac{\partial #1}{\partial x}}
\newcommand\ppt[1]{\frac{\partial #1}{\partial t}}
\newcommand\pp[3][]{\frac{\partial^{#1}{#2}}{\partial {#3}^{#1}}}
\newcommand\norm[1]{\left\lVert#1\right\rVert}
\newcommand\E[1]{E\left[#1\right]}
\newcommand\var[1]{\operatorname{Var}[#1]}
\newcommand\cov[1]{\operatorname{Cov}[#1]}
\renewcommand\vec[1]{\boldsymbol{#1}}

\def\vX{\vec{X}}
\def\vt{\vec{t}}
\def\mP{\mathbb{P}}

\newcounter{module}
\setcounter{module}{5}

\newmdtheoremenv[%
    backgroundcolor=blue!10!white,%
]{theorem}{Theorem}[module]
\newmdtheoremenv[%
    backgroundcolor=violet!10!white,%
]{corollary}{Corollary}[theorem]
\newmdtheoremenv[%
    backgroundcolor=teal!10!white,%
]{lemma}[theorem]{Lemma}

\theoremstyle{definition}
\newmdtheoremenv[%
    backgroundcolor=green!10!white,%
]{definition}{Definition}[module]
\newmdtheoremenv[%
    backgroundcolor=red!10!white,%
]{exercise}{Exercise}[module]

\theoremstyle{remark}
\newtheorem*{remark}{Remark}
\newtheorem*{example}{Example}
\newtheorem*{solution}{Solution}

\surroundwithmdframed[%
    linecolor=black!20!white,%
    hidealllines=false,%
    innertopmargin=5,%
    innerbottommargin=10,%
    skipabove=0,%
    skipbelow=0,%
]{example}

\numberwithin{equation}{module}

\title{
    \Large\textsc{MA2202: Probability I} \\
    % \vspace{10pt}
    \Huge \textbf{Markov chains} \\
    \vspace{5pt}
    \Large{Spring 2021}
}
\author{
    \large Satvik Saha%
    % \thanks{Email: \tt ss19ms154@iiserkol.ac.in}
    \\\textsc{\small 19MS154}
}
\date{\normalsize
    \textit{Indian Institute of Science Education and Research, Kolkata, \\
    Mohanpur, West Bengal, 741246, India.} \\
    % \vspace{10pt}
    % \today
}

\begin{document}
    \maketitle

    \begin{definition}[Markov chain]
        Let $\{X_n\}_{n = 0}^\infty$ be a sequence of random variables taking values
        in $\{0, 1, \dots, N\}$ such that \[
            P(X_{n + 1} = s_{n + 1} | X_{n} = s_n, \dots, X_0 = s_0) = P(X_{n + 1} =
            s_{n + 1} | X_n = s_n).
        \] Then, the sequence $\{X_n\}$ is a Markov chain.
    \end{definition}

    \begin{definition}[Transition probabilities]
        We define \[
            p_{ij} = P(X_{n + 1} = j \,|\, X_n = i).
        \] If $p_{ij}$ does not depend on $n$, then it is called a stationary
        transition probability from the $i$\textsuperscript{th} state to the
        $j$\textsuperscript{th} state.
        \begin{remark}
            Note that \[
                \sum_{j = 0}^N P(X_{n + 1} = j \,|\, X_n = i) = \sum_{j = 0}^N
                p_{ij} = 1.
            \] 
        \end{remark}
    \end{definition}

    \begin{definition}[Stochastic matrix]
        The stochastic matrix or transition probability matrix $\mP$ is
        defined such that $\mP_{ij} = p_{ij}$, where $0 \leq i, j \leq N$.
        \begin{remark}
            Note that setting $\vec{1} = (1 \dots 1)^\top$, we have $\mP\vec{1} =
            \vec{1}$.
        \end{remark}
    \end{definition}

    \begin{lemma}
        For a Markov chain with stationary transition probabilities,
        \[
            P(X_n = s_n, \dots, X_0 = s_0) = p_{s_{n - 1}s_n}p_{s_{n - 2}s_{n -
            1}}\dots p_{s_0s_1}P(X_0 = s_0).
        \] 
    \end{lemma}

    \begin{definition}[$m$-stage probability matrix]
        We define $\mP^{(m)}$ such that \[
            \mP^{(m)}_{ij} = P(X_m = j \,|\, X_0 = i).
        \] 
    \end{definition}

\end{document}
% vim: set tabstop=4 shiftwidth=4 softtabstop=4:
