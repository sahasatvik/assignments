\documentclass[10pt]{article}

\usepackage[T1]{fontenc}
\usepackage{geometry}
\usepackage{amsmath, amssymb, amsthm}
\usepackage{graphicx}
\usepackage{hyperref}
\usepackage{cancel}
\usepackage{xcolor}

\title{Probability I - Assignment VIII}
\author{Satvik Saha}
\date{}

\geometry{a4paper, margin=1in}
\setlength\parindent{0pt}
\renewcommand{\labelenumi}{(\roman{enumi})}
\renewcommand\CancelColor{\color{red}}
% \renewcommand\qedsymbol{$\blacksquare$}
\newcounter{prob}
\def\problem{\stepcounter{prob}\paragraph{Exercise \arabic{prob}}}
\def\solution{\paragraph{Solution}}

\newcommand\op[1]{\operatorname{#1}}
\newcommand\E[1]{\op{E}\left[#1\right]}
\newcommand\V[1]{\op{Var}[#1]}
\newcommand\CV[1]{\op{Cov}[#1]}

\newtheorem*{lemma}{Lemma}

\begin{document}
        \par\textbf{IISER Kolkata} \hfill \textbf{Assignment VIII}
        \vspace{3pt}
        \hrule
        \vspace{3pt}
        \begin{center}
                \LARGE{\textbf{MA 2202 : Probability I}}
        \end{center}
        \vspace{3pt}
        \hrule
        \vspace{3pt}
        Satvik Saha, \texttt{19MS154}, Group D\hfill\today
        \vspace{20pt}

        \problem Let $X$, $Y$ and $Z$ be independent identically distributed
        positive random variables. Determine \[
            \E{\frac{2X + 3Y}{X + Y + Z}}.
        \] 

        \solution Using the fact that $X$, $Y$, $Z$ are identically distributed, we
        use a symmetry argument and permute them cyclically, which gives \[
            \E{\frac{2X + 3Y}{X + Y + Z}} = \E{\frac{2Y + 3Z}{X + Y + Z}} =
            \E{\frac{2Z + 3X}{X + Y + Z}}.
        \] Adding these up and using linearity of expectation, we have \[
            3\E{\frac{2X + 3Y}{X + Y + Z}} = \E{\frac{2(X + Y + Z) + 3(Y + Z + X)}{X
            + Y + Z}} = \E{5} = 5.
        \] Hence, \[
            \E{\frac{2X + 3Y}{X + Y + Z}} = \frac{5}{3}.
        \] 

        \problem A gambler plays a game in which on each play he wins Rs.\ 1 with
        probability $p$ and loses Rs.\ 1 with probability $q = 1 - p$. Once he loses
        all his money, he cannot gamble any more. The gambler starts with Rs.\ $R$,
        where $R \in \mathbb{N}$.

        \begin{enumerate}
            \item To evade the possibility of losing all his money, he chooses an
            integer $M > R$ and decides to quit gambling as soon as he has Rs.\ $M$.
            Find the probability that he loses all his money if 
            \begin{enumerate}
                \item $p \neq q$.
                \item $p = 1 = 1 /2$.
            \end{enumerate}

            \item Show that if $p \leq q$ and if instead of setting a target $M$ to
            quit gambling, the gambler keeps on gambling irrespective of his gains
            or losses, then eventually he will lose all his money.
        \end{enumerate}

        \solution 
        Let $\{X_n\}_{n = 1}^\infty$ be a Markov chain where $X_n$ denotes the money
        the gambler has after gambling $n$ times (we fix $X_0 = R$ for notational
        purposes). Recall that the Markov property is indeed satisfied -- the money
        the gambler has at stage $n + 1$ depends only on how much he had at stage
        $n$, and whether he wins or loses.
        \begin{enumerate}
            \item The gambler must either win or lose money at each round. Say he
            currently has money $X_n = k$. If $1 < k < M - 1$, he could have reached
            this point by winning Rs.\ 1 from $X_{n - 1} = k - 1$, or by losing Rs.\
            1 from $X_{n - 1} = k + 1$. 

            Let $P_k$ be the probability that the gambler loses all his money
            given that he currently has Rs.\ $k$. If $0 < k < M$, note that \[
                P_{k} = pP_{k + 1} + qP_{k - 1}.
            \] This is because the gambler wins on the next turn with probability
            $p$, getting Rs.\ $k + 1$. Now, the Markov property means that his odds
            of ruin henceforth only depend on the current state, which we have
            defined as $P_{k + 1}$. The same applies for the $qP_{k - 1}$ term.
            Also, since the game terminates whenever $k = 0$ or $k = M$ \[
                P_0 = 1,\qquad P_M = 0.
            \] Note that $R \neq 0, M$. Now, write \[
                P_k = (p + q)P_k = pP_{k + 1} + qP_{k - 1}, \qquad p(P_{k + 1} -
                P_k) = q(P_k - P_{k - 1}).
            \] Therefore, using $P_0 = 1$, \[
                P_{k + 1} - P_k = \frac{q}{p}\left(P_k - P_{k - 1}\right) =
                \dots = \frac{q^k}{p^k}(P_1 - 1).
            \] Now by telescoping the series, \[
                P_{k + 1} - 1 = (P_{k + 1} - P_k) + (P_{k} - P_{k - 1}) + \dots +
                (P_1 - 1) = \sum_{n = 0}^k \frac{q^n}{p^n}(P_1 - 1). 
            \] This is a geometric series, which for $p \neq q$ is \[
                P_{k + 1} - 1 = \frac{1 - (q / p)^{k + 1}}{1 - (q /p)}(P_1 - 1)
            \] and for $p = q$ is \[
                P_{k + 1} - 1 = (k + 1)(P_1 - 1).
            \] To evaluate $1 - P_1$, set $k + 1 = M$. Demanding $P_M = 0$ gives \[
                1 - P_1 = \begin{cases}
                    \frac{1 - (q / p)}{1 - (q / p)^M}, &\text{ if }p \neq q, \\
                    \frac{1}{M}, &\text{ if } p = q = 1 /2.
                \end{cases}
            \] Thus, setting $k + 1 = R$, we have the desired probability of ruin \[
                P_R = \begin{cases}
                    1 - \frac{1 - (q / p)^R}{1 - (q / p)^M}, &\text{ if }p \neq q, \\
                    1 - \frac{R}{M}, &\text{ if } p = q = 1 /2.
                \end{cases}
            \] 

            \item Note that when $p \leq q$, setting $1 < \alpha = q / p$, we have \[
                \frac{1 - \alpha^R}{1 - \alpha^M} = \frac{\alpha^{-M} - \alpha^{R
                -M}}{\alpha^{-M} - 1}.
            \] Since $\alpha > 1$, we have $\alpha^{-M} \to 0$ as $M \to \infty$.
            Also, $\alpha^{R - M} \to 0$ since we always have $R < M$. Thus, \[
                1 - \frac{1 - \alpha^R}{1 - \alpha^M} \to 1
            \] as $M \to \infty$. Similarly, we have $R / M \to 0$, so \[
                1 - \frac{R}{M} \to 1.
            \] In any case, $P_R \to 1$ as $M \to \infty$. This means that if the
            gambler keeps playing irrespective of gains or losses, he will be
            eventually lose all his money (ruin is a probability one event).
        \end{enumerate}

\end{document}
