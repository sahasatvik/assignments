\documentclass[11pt]{article}

\usepackage[T1]{fontenc}
\usepackage{geometry}
\usepackage{amsmath, amssymb, amsthm}
\usepackage[%
    hidealllines=true,%
    innerbottommargin=15,%
    nobreak=true,%
]{mdframed}
\usepackage{xcolor}
\usepackage{graphicx}
\usepackage{fancyhdr}
\usepackage{lipsum}

\geometry{a4paper, margin=1in, headheight=14pt}

\pagestyle{fancy}
\fancyhf{}
\renewcommand\headrulewidth{0.4pt}
\fancyhead[L]{\scshape MA2201: Probability I}
\fancyhead[R]{\scshape Introduction to probability}
\rfoot{\footnotesize\it Updated on \today}
\cfoot{\thepage}

\def\C{\mathbb{C}}
\def\R{\mathbb{R}}
\def\Q{\mathbb{Q}}
\def\Z{\mathbb{Z}}
\def\N{\mathbb{N}}
\newcommand\ddx[1]{\frac{d #1}{d x}}
\newcommand\ddt[1]{\frac{d #1}{d t}}
\newcommand\dd[3][]{\frac{d^{#1}{#2}}{d {#3}^{#1}}}
\newcommand\ppx[1]{\frac{\partial #1}{\partial x}}
\newcommand\ppt[1]{\frac{\partial #1}{\partial t}}
\newcommand\pp[3][]{\frac{\partial^{#1}{#2}}{\partial {#3}^{#1}}}
\newcommand\norm[1]{\left\lVert#1\right\rVert}
\newcommand\E[1]{E[#1]}

\newcounter{module}
\setcounter{module}{2}

\newmdtheoremenv[%
    backgroundcolor=blue!10!white,%
]{theorem}{Theorem}[module]
\newmdtheoremenv[%
    backgroundcolor=violet!10!white,%
]{corollary}{Corollary}[theorem]
\newmdtheoremenv[%
    backgroundcolor=teal!10!white,%
]{lemma}[theorem]{Lemma}

\theoremstyle{definition}
\newmdtheoremenv[%
    backgroundcolor=green!10!white,%
]{definition}{Definition}[module]
\newmdtheoremenv[%
    backgroundcolor=red!10!white,%
]{exercise}{Exercise}[module]

\theoremstyle{remark}
\newtheorem*{remark}{Remark}
\newtheorem*{example}{Example}

\surroundwithmdframed[%
    linecolor=black!20!white,%
    hidealllines=false,%
    innertopmargin=5,%
    innerbottommargin=10,%
    skipabove=0,%
    skipbelow=0,%
]{example}

\numberwithin{equation}{module}

\title{
    \Large\textsc{MA2202: Probability I} \\
    % \vspace{10pt}
    \Huge \textbf{Conditional probability} \\
    \vspace{5pt}
    \Large{Spring 2021}
}
\author{
    \large Satvik Saha%
    % \thanks{Email: \tt ss19ms154@iiserkol.ac.in}
    \\\textsc{\small 19MS154}
}
\date{\normalsize
    \textit{Indian Institute of Science Education and Research, Kolkata, \\
    Mohanpur, West Bengal, 741246, India.} \\
    % \vspace{10pt}
    % \today
}

\begin{document}
    \maketitle

    Suppose a fair die is rolled and the outcome is even. The probability that the
    outcome was prime is $1 /3$. This is because the only even prime is $2$, and we
    know that we have rolled one of $2, 4$ or $6$, all of which are equally likely.
    \begin{definition}[Conditional probability]
        Suppose a random experiment is performed, and we know that an event $B$
        occurred. Let $A \in \mathcal{E}$. The probability of occurrence of $A$
        given $B$ is called the conditional probability $P(A \,|\, B)$.

        \begin{remark}
            Note that the conditional probability $A$ given $B$ is equivalent to the
            occurrence of $A \cap B$ computed relative to the probability of the
            occurrence of $B$. \[
                P(A\,|\, B) \,P(B) = P(A \cap B).
            \] 
        \end{remark}
    \end{definition}
    \begin{theorem}[Bayes' Theorem]
        If $A, B \in \mathcal{E}$ with $P(B) \neq 0$, we have \[
            P(A \,|\, B) = \frac{P(A)}{P(B)}P(B \,|\, A).
        \]
    \end{theorem}
    \begin{proof}
        This follows directly from \[
            P(A \cap B) = P(A \,|\, B) \,P(B) = P(B \,|\, A) \,P(A). \qedhere
        \] 
    \end{proof}
    \begin{example}
        Suppose a fair die is rolled and the outcome is even. The probability that
        the outcome is prime is $1 /3$. This is because the only even prime is $2$,
        the available even numbers are $2, 4, 6$, and the available primes are 
        $2, 3, 5$. Thus, \[
            P(\text{prime} \,|\, \text{even}) =
            \frac{P(\text{even})}{P(\text{prime})} P(\text{even} \,|\, \text{prime})
            = \frac{3 / 6}{3 / 6} \cdot \frac{1}{3} = \frac{1}{3}.
        \] 
    \end{example}
    
    \begin{definition}[Independent events]
        If $A, B \in \mathcal{E}$ such that $P(A), P(B) \neq 0$, and we have \[
            P(A \,|\, B) = P(A), \qquad P(B \,|\, A) = P(B),
        \] then we say that
        $A$ and $B$ are independent. \\

        \noindent Equivalently, $P(A \cap B) = P(A) P(B)$.
    \end{definition}

\end{document}
% vim: set tabstop=4 shiftwidth=4 softtabstop=4:
