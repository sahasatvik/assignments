\documentclass[11pt]{article}

\usepackage[T1]{fontenc}
\usepackage{geometry}
\usepackage{amsmath, amssymb, amsthm}
\usepackage[%
    hidealllines=true,%
    innerbottommargin=15,%
    nobreak=true,%
]{mdframed}
\usepackage{xcolor}
\usepackage{graphicx}
\usepackage{fancyhdr}
\usepackage{lipsum}
\usepackage{bm}

\geometry{a4paper, margin=1in, headheight=14pt}

\pagestyle{fancy}
\fancyhf{}
\renewcommand\headrulewidth{0.4pt}
\fancyhead[L]{\scshape MA2201: Probability I}
\fancyhead[R]{\scshape Random variables}
\rfoot{\footnotesize\it Updated on \today}
\cfoot{\thepage}

\def\C{\mathbb{C}}
\def\R{\mathbb{R}}
\def\Q{\mathbb{Q}}
\def\Z{\mathbb{Z}}
\def\N{\mathbb{N}}
\newcommand\ddx[1]{\frac{d #1}{d x}}
\newcommand\ddt[1]{\frac{d #1}{d t}}
\newcommand\dd[3][]{\frac{d^{#1}{#2}}{d {#3}^{#1}}}
\newcommand\ppx[1]{\frac{\partial #1}{\partial x}}
\newcommand\ppt[1]{\frac{\partial #1}{\partial t}}
\newcommand\pp[3][]{\frac{\partial^{#1}{#2}}{\partial {#3}^{#1}}}
\newcommand\norm[1]{\left\lVert#1\right\rVert}
\newcommand\E[1]{E\left[#1\right]}
\newcommand\var[1]{\operatorname{Var}[#1]}
\newcommand\cov[1]{\operatorname{Cov}[#1]}
\renewcommand\vec[1]{\boldsymbol{#1}}

\def\vX{\vec{X}}
\def\vt{\vec{t}}

\newcounter{module}
\setcounter{module}{4}

\newmdtheoremenv[%
    backgroundcolor=blue!10!white,%
]{theorem}{Theorem}[module]
\newmdtheoremenv[%
    backgroundcolor=violet!10!white,%
]{corollary}{Corollary}[theorem]
\newmdtheoremenv[%
    backgroundcolor=teal!10!white,%
]{lemma}[theorem]{Lemma}

\theoremstyle{definition}
\newmdtheoremenv[%
    backgroundcolor=green!10!white,%
]{definition}{Definition}[module]
\newmdtheoremenv[%
    backgroundcolor=red!10!white,%
]{exercise}{Exercise}[module]

\theoremstyle{remark}
\newtheorem*{remark}{Remark}
\newtheorem*{example}{Example}
\newtheorem*{solution}{Solution}

\surroundwithmdframed[%
    linecolor=black!20!white,%
    hidealllines=false,%
    innertopmargin=5,%
    innerbottommargin=10,%
    skipabove=0,%
    skipbelow=0,%
]{example}

\numberwithin{equation}{module}

\title{
    \Large\textsc{MA2202: Probability I} \\
    % \vspace{10pt}
    \Huge \textbf{Random vectors} \\
    \vspace{5pt}
    \Large{Spring 2021}
}
\author{
    \large Satvik Saha%
    % \thanks{Email: \tt ss19ms154@iiserkol.ac.in}
    \\\textsc{\small 19MS154}
}
\date{\normalsize
    \textit{Indian Institute of Science Education and Research, Kolkata, \\
    Mohanpur, West Bengal, 741246, India.} \\
    % \vspace{10pt}
    % \today
}

\begin{document}
    \maketitle

    \begin{definition}[Random vector]
        A random vector $\vX\colon \Omega \to \R^n$ is a tuple of random 
        variables $X_i\colon \Omega \to \R$.
    \end{definition}
    \begin{definition}[Joint cumulative distribution function]
        The joint cumulative distribution function of a random vector $\vX$ is
        the map $F_{\vX}\colon \R^n \to [0, 1]$, given as \[
            F_{\vX}(\vec{s}) = P(X_1 \leq s_1, \dots, X_n \leq s_n).
        \] 
    \end{definition}

    \begin{definition}[Joint probability mass function]
        If $X_i$ are discrete random variables, their joint probability mass 
        function is the map $p_{\vX} \colon \R^n \to [0, 1]$, \[
            p_{\vX}(\vec{s}) = P(X_1 = s_1, \dots, X_n = s_n).
        \] 
    \end{definition}

    \begin{definition}[Joint probability density function]
        Suppose that \[
            F_{\vX}(\vec{s}) = \int_{-\infty}^{s_n} \dots \int_{-\infty}^{s_1}
            f_{\vX}(t_1, \dots, t_n)\:dt_1\dots\:dt_n,
        \] then $f_{\vX}\colon \R^n \to [0, 1]$ is the probability density
        function corresponding to the joint cumulative distribution function
        $F_{\vX}$.

        \begin{remark}
            If $f_{\vX}$ is continuous, then \[
                f_{\vX} = \frac{\partial F_{\vX}(t_1, \dots, t_n)}{\partial
                t_1\dots\partial t_n}.
            \] 
        \end{remark}
    \end{definition}

    \begin{definition}[Joint moment generating function]
        Let $\vX$ be a random vector. Then, its joint moment generating function
        is defined as \[
            M_{\vX}(\vt) = \E{e^{\vt^\top \vX}} = \E{e^{t_1X_1 +
            \dots + t_nX_n}}.
        \] 
        \begin{remark}
            If $X_1, \dots, X_n$ are independent, then \[
                M_{\vX}(\vt) = \prod M_{X_i}(t_i).
            \] 
        \end{remark} 
    \end{definition}
    
\end{document}
% vim: set tabstop=4 shiftwidth=4 softtabstop=4:
