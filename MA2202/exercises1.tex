\documentclass[10pt]{article}

\usepackage[T1]{fontenc}
\usepackage{geometry}
\usepackage{amsmath, amssymb, amsthm}

\title{Probability I - Assignment I}
\author{Satvik Saha}
\date{}

\geometry{a4paper, margin=1in}
\setlength\parindent{0pt}
\renewcommand{\labelenumi}{(\alph{enumi})}
% \renewcommand\qedsymbol{$\blacksquare$}
\newcounter{prob}
\def\problem{\stepcounter{prob}\paragraph{Exercise \arabic{prob}}}
\def\solution{\paragraph{Solution}}

\begin{document}
        \par\textbf{IISER Kolkata} \hfill \textbf{Assignment I}
        \vspace{3pt}
        \hrule
        \vspace{3pt}
        \begin{center}
                \LARGE{\textbf{MA 2202 : Probability I}}
        \end{center}
        \vspace{3pt}
        \hrule
        \vspace{3pt}
        Satvik Saha, \texttt{19MS154}, Group D\hfill\today
        \vspace{20pt}


        \problem Toss a coin a hundred times and write down the outcomes 
        sequentially.
        \solution Tails, Tails, Heads, Tails, Heads, Tails, Tails, Tails, Tails,
        Tails, Heads, Heads, Heads, Heads, Tails, Heads, Tails, Heads, Tails, Heads,
        Heads, Tails, Tails, Tails, Tails, Tails, Heads, Tails, Heads, Tails, Tails,
        Tails, Heads, Tails, Heads, Heads, Tails, Heads, Heads, Tails, Tails, Tails,
        Heads, Heads, Heads, Tails, Tails, Tails, Heads, Heads, Tails, Tails, Heads,
        Heads, Tails, Tails, Heads, Heads, Heads, Tails, Tails, Heads, Heads, Heads,
        Tails, Tails, Heads, Heads, Heads, Tails, Heads, Tails, Tails, Heads, Heads,
        Tails, Tails, Tails, Tails, Heads, Tails, Tails, Tails, Tails, Tails, Tails,
        Tails, Heads, Tails, Heads, Heads, Tails, Heads, Tails, Heads, Tails, Heads,
        Heads, Tails, Tails.

        \problem Let $(\Omega, \mathcal{E}, P)$ be a probability space.
        Let $A, B \in \mathcal{E}$. Show that
        \begin{enumerate}
                \item $P(B \cap A^c) = P(B) - P(B \cap A)$.
                \item $P(A \cup B) = P(A) + P(B) - P(A \cap B)$.
                \item If $A \subseteq B$, then $P(A) \leq P(B)$.
        \end{enumerate}
        \solution Here, we denote $X^c = \Omega\setminus X$.
        We first note that for any $A \in \mathcal{E}$, the complement $A^c \in
        \mathcal{E}$. Also, $A \cap A^c = \emptyset$ and $A \cup A^c = \Omega$.
        Thus, the probability of the countable union can be broken as \[
                P(\Omega) = P(A \cup A^c) = P(A) + P(A^c).
        \] Since $P(\Omega) = 1$, we have $P(A^c) = 1 - P(A)$. \\
        
        Also, since $\mathcal{E}$ is closed under countable unions as well as
        complements, we use de Morgan's Laws to conclude that if $A, B \in
        \mathcal{E}$, then $A^c \cup B^c = (A \cap B)^c \in \mathcal{E}$,
        so $A \cap B \in \mathcal{E}$.
        \begin{enumerate}
                \item Note that \[
                        (B \cap A^c) \cap (B \cap A) = B \cap B \cap A^c \cap A
                        = B \cap \emptyset = \emptyset,
                \] which means that $B \cap A^c$ and $B \cap A$ are disjoint.
                Also, \[
                        (B\cap A^c) \cup (B \cap A) = B \cap (A^c \cup A) = 
                        B \cap \Omega = B.
                \] 
                Thus, the probability of the union must be the sum of the individual
                probabilities, so \[
                        P(B \cap A^c) + P(B \cap A) = P(B),
                \] as desired.

                \item Note that \[
                        (A \cap B) \cap (A \setminus B) = (A \cap B) \cap (A \cap
                        B^c) = \emptyset,
                \]
                \[
                        (A \cap B) \cap (B \setminus A) = (A \cap B) \cap (B \cap
                        A^c) = \emptyset,
                \] 
                \[
                        (A \setminus B) \cap (B \setminus A) = (A \cap B^c) \cap (B
                        \cap A^c) = \emptyset,
                \]
                This means that the sets $A\cap B$, $A\setminus B$ and $B\setminus
                A$ are pairwise disjoint. 
                Also, note that \[
                        A = A \cap \Omega = A \cap (B \cup B^c) = (A \cap B) \cup (A
                        \cap B^c) = (A \cap B) \cup (A \setminus B),
                \] \[
                        B = B \cap \Omega = B \cap (A \cup A^c) = (B \cap A) \cup (B
                        \cap A^c) = (A \cap B) \cup (B \setminus A).
                \] 
                Additionally, \[
                        (A \cap B) \cup (A \setminus B) \cup (B \setminus A) = A
                        \cup (B \setminus A) = A \cup (B \cap A^c) = (A \cup B) \cap
                        (A \cup A^c) = A \cup B.
                \] 
                Thus, we can write 
                \begin{align*}
                P(A) + P(B) \,&=\, \left[P(A\cap B) + P(A\setminus B)\right] + 
                            \left[P(A \cap B) + P(B \setminus A)\right] \\
                        \,&=\, P(A \cap B) + P((A \cap B) \cup (A \setminus B) 
                            \cup (B\setminus A)) \\
                        \,&=\, P(A \cap B) + P(A \cup B),
                \end{align*}
                as desired.

                \item If $A \subseteq B$, we set $C = B \setminus A = B \cap A^c \in
                \mathcal{E}$.
                Also note that $B \cap A = A$, so part (a) gives \[
                        P(C) = P(B) - P(B \cap A) = P(B) - P(A).
                \] 
                On the other hand, $C \in \mathcal{E}$ which means that $P(C) \geq
                0$. This directly gives $P(B) \geq P(A)$, as desired.

        \end{enumerate}

        \problem Let $\Omega$ be a countably infinite sample space of a random
        experiment, none of whose outcomes are expected to occur in preference to
        the others. Justify whether the set of events $\mathcal{E}$ can be identical
        with $2^\Omega$ for such a random experiment.
        \solution We cannot have $\mathcal{E} = 2^\Omega$. We have been given that
        none of the outcomes $n \in \Omega$ occur in preference to another.
        This means that we can write $P(\{n\}) = x$, for some $x \in \mathbb{R}$.
        Now, note that \[
                \bigcup_{n \in \Omega} \{n\} = \Omega.
        \] This is a countable union because $\Omega$ is countably infinite, so we
        can write \[
                P(\Omega) = \sum_{n \in \Omega} x.
        \] The right hand side is an infinite sum with constant terms. If $x = 0$,
        then the sum evaluates to $0$, otherwise the sum diverges and cannot be
        assigned any meaningful value. On the other hand, we demand $P(\Omega) = 1$,
        which is a contradiction.

        \problem To the choice of each $n \in \mathbb{N}$, could you assign a 
        probability $P(n) > 0$ such that the following conditions hold?
        \begin{enumerate}
                \item $P(m) \neq P(n)$ for all $m, n \in \mathbb{N}$, $m \neq n$.
                \item The probability of choosing an odd positive integer is the
                same as the probability of choosing an even positive integer.
        \end{enumerate}
        \solution We assign the probabilities \[
                P(2k - 1) = \frac{1}{3^k}, \qquad 
                P(2k) = \frac{3}{2}\cdot\frac{1}{4^k},
        \] where $k \in \mathbb{N}$. Note that this covers all $n \in \mathbb{N}$,
        where each $n$ is either even or odd. Odd $n$ can be uniquely
        written as $n = 2k - 1$ and even $n$ can be uniquely written as $n = 2k$. \\

        Now, pick $m, n \in \mathbb{N}$, $m \neq n$. If both are odd, say $m = 2k -
        1$ and $n = 2\ell - 1$, then \[
                \frac{1}{3^k} = \frac{1}{3^\ell}, \qquad 3^k = 3^\ell
        \] forces $k = \ell$, hence $m = n$. If both are even, say $m = 2k$ and 
        $n = 2\ell$, then \[
                \frac{3}{2}\cdot \frac{1}{4^k} = \frac{3}{2}\cdot \frac{1}{4^\ell},
                \qquad 4^k = 4^\ell
        \] also forces $k = \ell$, hence $m = n$.
        If one of them is even and the other odd, without loss of generality let $m
        = 2k - 1$ be odd and $n = 2\ell$ be even. Then, \[
                \frac{1}{3^k} = \frac{3}{2}\cdot \frac{1}{4^\ell}, \qquad
                3^{k + 1} = 2^{2\ell + 1}
        \] can only happen if $k + 1 = 2\ell + 1 = 0$ since $2$ and $3$ are prime.
        This forces $k = -1$, which is absurd. Thus, we obtain contradictions in all
        cases, which means that $P(m) \neq P(n)$ for any $m \neq n$.

        Also note that each of the events $\{m\}$ and $\{n\}$ are pairwise disjoint,
        so the probability of their countable union is the sum of their individual
        probabilities. Thus,
        \begin{align*}
                P(n\text{ is odd}) \,&=\, P(\{1, 3, 5, \dots\}) \\ 
                  \,&=\, P(\{1\}) + P(\{3\}) + P(\{5\}) + \dots \\
                  \,&=\, \sum_{k = 1}^{\infty} \frac{1}{3^k} \\
                  \,&=\, \frac{1}{3}\cdot \frac{1}{1 - 1 /3} \,=\, \frac{1}{2}.
        \end{align*}
        \begin{align*}
                P(n\text{ is even}) \,&=\,  P(\{2, 4, 6, \dots\}) \\
                  \,&=\, P(\{2\}) + P(\{4\}) + P(\{6\}) + \dots \\ 
                  \,&=\, \frac{3}{2}\sum_{k = 1}^{\infty} \frac{1}{4^k} \\ 
                  \,&=\, \frac{3}{8}\cdot \frac{1}{1 - 1 /4} \,=\, \frac{1}{2}.
        \end{align*}
        This means that $P(\mathbb{N}) = P(\{1, 3, 5, \dots\} \cup \{2, 4, 6, 
        \dots\}) = 1 /2 + 1 /2 = 1$, as desired.

\end{document}
