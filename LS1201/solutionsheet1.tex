\documentclass[10pt]{article}

\usepackage[T1]{fontenc}
\usepackage{geometry}
\usepackage{amsmath, amssymb, amsthm}
\usepackage{array} 
\usepackage{enumitem}
\usepackage{siunitx}

\geometry{a4paper, margin=1in}
% \setlength\parindent{0pt}
% \renewcommand\qedsymbol{$\blacksquare$}
\newcolumntype{L}{l@{\quad\quad}}
\newcounter{prob}
\def\problem{\stepcounter{prob}\paragraph{Problem \arabic{prob}}}
\def\solution{\\\\\textbf{Solution }}

\begin{document}
        \par\textbf{IISER Kolkata} \hfill \textbf{Why Evolution is True}
        \vspace{3pt}
        \hrule
        \vspace{3pt}
        \begin{center}
                \LARGE{\textbf{LS1201 : Introduction to Biology II}}
        \end{center}
        \vspace{3pt}
        \hrule
        \vspace{3pt}
        Satvik Saha, \texttt{19MS154}\hfill\today
        \vspace{20pt}

        \problem Can you think of two reasons why it took us, as a species, centuries to come up with a theory of evolution,
        in spite of having rather advanced brains and thought processes?
        \solution A theory of evolution demands immensely long periods of time, measured in millions of years, in order to be able to
        explain how life reached its present form. It is somewhat astounding that the weight of the Earth was determined to within 1\%
        of its correct value by Cavendish as early as 1798, yet it took until 1956 for Clair Patterson to correctly assess the Earth's age --
        around 4.5 billion years. It is difficult to fully comprehend such timescales even today. Hence, it seems apparent that
        the sheer magnitude of time required, as well as the incredible slowness of the processes involved, might have been a barrier towards
        our understanding of evolution.

        Another probable reason is the influence of religion. Given the seeming availability of good explanations for life's diversity,
        most thinkers of the past had little reason to search for something better.
        Such theories include Plato's ``principle of plenitude'', the idea that all things in the world are fixed by divine design,
        the Oriental concept of Tao, in which all living beings are in a constant state of change in response to their environment,
        and the idea of spontaneous generation, in which living things may arise from non-living things.
        Particluarly in the Western world, there was immense pressure for scientific discovery to somehow align with Christian theology,
        which explicitly lays out a Creation story. Notably, a large amount of opposition towards the ideas of Lamarck, or even Darwin,
        were based on religious grounds. Such objections, involving the rejection of new theories in favour of older established ideas,
        may also have played a factor in our gradual understanding of evolution.

        \problem Can you find one example of Lamarckism?
        \solution The principle of `use and disuse' may be illustrated with Lamarck's giraffe. According to Lamarckism, giraffes must have
        developed their long necks in response to their environment, due to the neccesity of feeding off tall plants and trees. Not only
        did giraffes develop this characteristic by physically stretching their necks, such a trait was also heritable, with
        offspring retaining those long necks. Such a process continued until giraffes reached their present form.

        Today, of course, we understand that the evolution of organisms such as giraffes has a precise molecular basis, with little to 
        none evidence supporting Lamarck's giraffe. We may argue that a giraffe cannot substantially lengthen its neck within its
        lifetime. We may also note that Lamarck's theory offers no mechanism to explain how an organism would `choose' beneficial changes
        from harmful ones.

        \problem Do you think Lamarck was wrong? Justify your answer.
        \solution Lamarck's theory of use and disuse describes phenomena which occurs only in a minority of systems. Today, we have
        established a molecular basis of inheritance, together with Darwin's theory of evolution by naural selection. Such theories
        are well supported by evidence, and show that it is not enough for an organism to acquire a characteristic within its lifetime.
        Rather, in order for a trait to be heritable, it must have a molecular basis, via DNA. The only known way for genetic code
        to change is essentially by random mutation, which is contrary to Lamarck's idea that only beneficial changes are acquired, feeding the 
        drive towards complexity. Hence, we may conclude that Lamarck's theory was wrong, in light of better and more accurate modern theories.

\end{document}
