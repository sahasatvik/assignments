\documentclass[11pt]{article}

\usepackage[T1]{fontenc}
\usepackage{geometry}
\usepackage{amsmath, amssymb, amsthm}
\usepackage[scr]{rsfso}
\usepackage{bm}
\usepackage[%
    hidealllines=true,%
    innerbottommargin=15,%
    nobreak=true,%
]{mdframed}
\usepackage{xcolor}
\usepackage{graphicx}
\usepackage{fancyhdr}
\usepackage{hyperref}

\geometry{a4paper, margin=1in, headheight=14pt}

\pagestyle{fancy}
\fancyhf{}
\renewcommand\headrulewidth{0.4pt}
\fancyhead[L]{\scshape MA3205: Geometry of Curves and Surfaces}
\fancyhead[R]{\scshape \leftmark}
\rfoot{\footnotesize\it Updated on \today}
\cfoot{\thepage}

\newcommand{\C}{\mathbb{C}}
\newcommand{\R}{\mathbb{R}}
\newcommand{\Q}{\mathbb{Q}}
\newcommand{\Z}{\mathbb{Z}}
\newcommand{\N}{\mathbb{N}}

\newcommand{\ip}[2]{\langle #1, #2 \rangle}
\newcommand{\norm}[1]{\Vert #1 \Vert}
\renewcommand{\vec}[1]{\boldsymbol{#1}}

\newcommand{\vx}{\vec{x}}
\newcommand{\vy}{\vec{y}}
\newcommand{\vv}{\vec{v}}
\newcommand{\vw}{\vec{w}}
\newcommand{\ve}{\vec{e}}

\newcommand{\dd}[2]{\frac{d #1}{d #2}}
\newcommand{\pp}[2]{\frac{\partial #1}{\partial #2}}
\newcommand{\ddn}[3][]{\frac{d^{#1} #2}{d #3^{#1}}}
\newcommand{\ppn}[3][]{\frac{\partial^{#1} #2}{\partial #3^{#1}}}
\newcommand{\grad}{\nabla}

\newmdtheoremenv[%
    backgroundcolor=blue!10!white,%
]{theorem}{Theorem}[section]
\newmdtheoremenv[%
    backgroundcolor=violet!10!white,%
]{corollary}{Corollary}[theorem]
\newmdtheoremenv[%
    backgroundcolor=teal!10!white,%
]{lemma}[theorem]{Lemma}

\theoremstyle{definition}
\newmdtheoremenv[%
    backgroundcolor=green!10!white,%
]{definition}{Definition}[section]
\newmdtheoremenv[%
    backgroundcolor=red!10!white,%
]{exercise}{Exercise}[section]

\theoremstyle{remark}
\newtheorem*{remark}{Remark}
\newtheorem*{example}{Example}
\newtheorem*{solution}{Solution}

\surroundwithmdframed[%
    linecolor=black!20!white,%
    hidealllines=false,%
    innertopmargin=5,%
    innerbottommargin=10,%
    skipabove=0,%
    skipbelow=0,%
    nobreak=false,%
]{example}

\numberwithin{equation}{section}

\title{
    \Large\textsc{MA3205} \\
    \Huge \textbf{Geometry of Curves and Surfaces} \\
    \vspace{5pt}
    \Large{Spring 2022}
}
\author{
    \large Satvik Saha
    \\\textsc{\small 19MS154}
}
\date{\normalsize
    \textit{Indian Institute of Science Education and Research, Kolkata, \\
    Mohanpur, West Bengal, 741246, India.} \\
}

\begin{document}
    \maketitle

    \tableofcontents
    
    \section{Curves}
    
    \subsection{Introduction}
    
    \begin{definition}
        A curve is a continuous map $\gamma\colon \R \to \R^n$.
    \end{definition}

    \begin{definition}
        A smooth curve $\gamma\colon \R \to \R^n$ is $C^\infty$, i.e.\ differentiable
        arbitrarily times.
    \end{definition}

    \begin{definition}
        A closed curve $\gamma\colon \R \to \R^n$ is periodic, i.e\ there exists some
        $c$ such that $\gamma(t + c) = \gamma(t)$ for all $t \in \R$.
    \end{definition}
    \begin{example}
        Alternatively, a closed curve can be thought of as a continuous map
        $\gamma\colon S^1 \to \R^n$. For instance, given a closed curve $\gamma\colon
        \R \to \R^n$ with period $c$, we can define the corresponding map \[
            \tilde{\gamma}\colon S^1 \to \R^n, \qquad \tilde{\gamma}(e^{it}) =
            \gamma(ct / 2\pi).
        \] 
    \end{example}

    \begin{definition}
        A simple curve $\gamma\colon \R \to \R^n$ is injective on its period.
    \end{definition}

    \begin{theorem}[Four Vertex Theorem]
        The curvature of a simple, closed, smooth plane curve has at least two local
        minima and two local maxima.
    \end{theorem}


    \begin{definition}
        A knot is a simple closed curve in $\R^3$.
    \end{definition}

    \begin{definition}
        The total absolute curvature of a knot $K$ is the integral of the absolute
        value of the curvature, taken over the curve, i.e.\ it is the quantity \[
            \oint_K |\kappa(s)|\:ds.
        \] 
    \end{definition}
    \begin{example}
        The total absolute curvature of a circle is always $2\pi$.
    \end{example}

    \begin{theorem}[F\'ary-Milnor Theorem]
        If the total absolute curvature of a knot $K$ is at most $4\pi$, then $K$ is
        an unknot.
    \end{theorem}

    \begin{definition}
        An immersed loop $\gamma$ is such that $\gamma'$ is never zero.
    \end{definition}

    \begin{definition}
        Two loops are isotopic if there exists an interpolating family of loops
        between them. Two immersed loops are isotopic if we can choose such an
        interpolating family of immersed loops.
    \end{definition}
    \begin{example}
        Without the restriction of immersion, any two loops $\gamma, \eta\colon S^1
        \to \R^n$ would be isotopic, since we can always construct the linear
        interpolations \[
            H\colon S^1 \times [0, 1] \to \R^n, \qquad H(e^{i\theta}, t) = (1 -
            t)\gamma(e^{i\theta}) + t\eta(e^{i\theta}).
        \] 
    \end{example}

    \begin{theorem}[Hirsch-Smale Theory]
        \mbox{}
        \begin{enumerate}
            \itemsep0em
            \item Any two immersed loops in $\R^2$ are isotopic if and only if their
            turning numbers match.
            \item Any two immersed loops in $S^2$ are isotopic if and only if their
            turning numbers modulo 2 match.
        \end{enumerate}
    \end{theorem}


    \subsection{Whitney's theorem}
    
    \begin{lemma}
        Let $\Omega \subset \R^n$ be open and let $C \subseteq \Omega$ be closed.
        Then there exists a continuous function $f\colon \Omega \to \R$ such that
        $f^{-1}(0) = C$.
        \begin{remark}
            The converse, i.e.\ $f^{-1}(0) = C$ implies $C$ is closed, where $f$ is
            continuous on $\Omega$, is trivial.
        \end{remark}
    \end{lemma}
    \begin{proof}
        Set $f$ to be the distance function from $C$, i.e.\ \[
            f(x) = \inf_{y \in C} d(x, y). \qedhere
        \] 
    \end{proof}

    \begin{theorem}[Whitney's Theorem]
        Let $\Omega \subset \R^n$ be open and let $C \subseteq \Omega$ be closed.
        Then there exists a smooth function $f\colon \Omega \to \R$ such that
        $f^{-1}(0) = C$.
    \end{theorem}
    \begin{proof}
        Set $V = \Omega\setminus C$, and cover $V$ by a countable collection of open
        balls, \[
            V = \bigcup_{i = 1}^\infty B(q_k, r_k).
        \] This can always be done since $V$ is open, and using the density of $\Q$
        in $\R$ to pick only rational $q_k, r_k$. Now for each open ball $B(q_k,
        r_k)$, we can construct a smooth bump functions $f_k$ such that $f^{-1}(0) =
        \R^n\setminus B(q_k, r_k)$, $f^{-1}(1) = \overline{B(q_k, r_k / 2)}$, and all
        derivatives of $f_k$ vanish on $\R^n\setminus B(q_k, r_k)$.

        Define the weights \[
            c_k = \max_{\substack{|\alpha| \leq k \\ y \in \overline{B(q_k, r_k)}}}
            \left|\ppn[\alpha]{f_k}{x}(y)\right|.
        \] Note that each $c_k$ is well-defined: there are finitely many multi-indices
        $\alpha$ given $k$, and each of the partials $\partial^\alpha f / \partial
        x^\alpha$ is a smooth function over a compact set, hence bounded.
        Furthermore, each $c_k \geq 1$. Finally, set \[
            f = \sum_{k = 1}^\infty \frac{f_k}{2^k c_k}.
        \] It is clear that $f^{-1}(0) = C$. We can show that the partial sums $s_n$
        converge; let $\epsilon > 0$ and choose sufficiently large $N$ such that $1 /
        2^N < \epsilon$. Now for $m > n \geq N$, examine \[
            |s_m(x) - s_n(x)| = \sum_{k = n + 1}^m \frac{f_k(x)}{2^k c_k} \leq
            \sum_{k = n + 1}^m \frac{1}{2^k} \leq \frac{1}{2^N} < \epsilon
        \] Thus, the convergence is uniform, and $f$ is $C^0$. For higher
        derivatives, we examine some $\alpha$ partial of the sums, and use the same
        argument; at each stage, $|\partial^\alpha f/ \partial x^\alpha| < c_k$
        whenever $|\alpha| < k$.
    \end{proof}

    
    \subsection{Parametrized curves}
    
    \begin{definition}
        A parametrized curve in $\R^n$ is a smooth map $\gamma\colon (\alpha, \beta)
        \to \R^n$ for some $\alpha, \beta$ with $-\infty \leq \alpha < \beta \leq
        \infty$.
        \begin{remark}
            Here, we will always implicitly assume that maps are continuous.
        \end{remark}
        \begin{remark}
            Such a curve is called regular if $\gamma'(t) \neq 0$ for all $t \in
            (\alpha, \beta)$.
        \end{remark}
    \end{definition}
    \begin{example}
        The curve defined by \[
            \gamma\colon \R \to \R^n, \qquad t \mapsto a + tb
        \] is a straight line through the point $a$, in the direction $b$.
    \end{example}
    \begin{example}
        The curve defined by \[
            \gamma\colon \R \to \R^2, \qquad t \mapsto (\cos{t}, \sin{t})
        \] is the unit circle in $\R^2$, counter-clockwise.
    \end{example}
    \begin{example}
        The curve defined by \[
            \gamma\colon \R \to \R^3, \qquad t \mapsto (t, \cos{t}, \sin{t})
        \] is a helix in $\R^3$, wrapped around the $x$-axis.
    \end{example}

    \begin{definition}
        A diffeomorphism is a smooth map with a smooth inverse.
    \end{definition}

    \begin{example}
        Suppose that $\gamma\colon (\alpha, \beta) \to \R^n$ is a smooth curve. If we
        have a diffeomorphism $\varphi\colon (\alpha', \beta') \to (\alpha, \beta)$,
        then the smooth curve $\eta = \gamma\circ \varphi$ is a reparametrization of
        $\gamma$. Note that \[
            \eta'(t) = \gamma'(\varphi(t))\, \varphi'(t).
        \] 
    \end{example}

    \begin{lemma}
        If $\varphi\colon (\alpha', \beta') \to (\alpha, \beta)$ is a diffeomorphism,
        then $\varphi'(t) \neq 0$ for all $t \in (\alpha', \beta')$.
    \end{lemma}

    \begin{definition}
        If the diffeomorphism $\varphi' > 0$, we say that it it orientation
        preserving. If $\varphi' < 0$, we say that it is orientation reversing.
    \end{definition}

    \begin{definition}
        The arc length of a differentiable curve $\gamma\colon (\alpha, \beta) \to
        \R^n$, starting at $t_0$, is defined as \[
            s(t) = \int_{t_0}^t \norm{\gamma'(u)}\:du.
        \] We call $s$ the arch length parameter.

        \begin{remark}
            If $\gamma'(t) \neq 0$, then $s'(t) > 0$.
        \end{remark}
    \end{definition}

    \begin{definition}
        A unit speed curve $\gamma$ is one where $\norm{\gamma'} = 1$
    \end{definition}

    \begin{lemma}
        Let $\gamma$ be a regular smooth curve. Then its arc length parameter is a
        smooth function.
    \end{lemma}
    \begin{proof}
        Note that $\gamma'$ and $\ip{\cdot}{\cdot}$ are smooth functions. Thus, \[
            \dd{s}{t} = \norm{\gamma'(t)} = \sqrt{\ip{\gamma'(t)}{\gamma'(t)}} > 0,
        \] showing that $ds / dt$ is smooth.
    \end{proof}

    \begin{lemma}
        The arc length function $s$ is a diffeomorphism onto its image.
    \end{lemma}
    \begin{proof}
        This follows from the Inverse Function Theorem, using the smoothness of $s$.
    \end{proof}

    \begin{lemma}
        Let $\varphi$ denote $s^{-1}\colon (\alpha', \beta') \to (\alpha, \beta)$.
        Then, $\gamma\circ \varphi$ is a unit speed reparametrization of $\gamma$.

        \begin{remark}
            Any other unit speed reparametrization is related to $s$ by shifts and
            reflections.
        \end{remark}
    \end{lemma}
    \begin{proof}
        Note that $s$ is strictly increasing, so $s', \varphi' > 0$. Now, \[
            \norm{(\gamma\circ \varphi)'(t)} = \norm{\gamma'(\varphi(t))}\cdot
            |\varphi'(t)| = s'(\varphi(t))\varphi'(t) = (s\circ \varphi)'(t) = 1. \qedhere
        \] 
    \end{proof}


    \subsection{Curvature}
    
    \begin{definition}
        Let $\gamma\colon (\alpha, \beta) \to \R^n$ be a regular curve. Let $\Delta
        s$ be the length of the curve from $\gamma(t)$ to $\gamma(t + \Delta t)$, and
        let $\Delta\theta$ be the angle between these two vectors. Then, the
        curvature of $\gamma$ at $\gamma(t)$ is defined as \[
            \lim_{\Delta t \to 0} \frac{\Delta\theta}{\Delta s}.
        \] \begin{remark}
            For a unit speed curve, the curvature is precisely $\norm{\gamma''(s)}$.
        \end{remark}
    \end{definition}
    \begin{example}
        For a straight line $a + bt$, the curvature vanishes identically.
    \end{example}
    \begin{example}
        For a circle of radius $R$, the curvature is $1 / R$. Note that we
        parametrize \[
            \gamma(s) = (x_0 + R\cos(t / R), y_0 + R\sin(t / R)).
        \] 
    \end{example}

    \begin{definition}
        Let $\gamma\colon (\alpha, \beta) \to \R^3$ be a regular $C^2$ curve. Its
        curvature is defined as \[
            \kappa(t) = \frac{\norm{\gamma'(t) \times
            \gamma''(t)}}{\norm{\gamma'(t)}^3}.
        \] 
        \begin{remark}
            It is easy to check that the curvature at a point is independent of
            parametrization.
        \end{remark}
    \end{definition}

    \begin{definition}
        Given a $C^2$ plane curve $\gamma$ such that $\ddot{\gamma}(0) \neq 0$, it is
        said to turn to the right when $\det(\dot{\gamma}(0), \ddot{\gamma}(0))$ is
        negative.
    \end{definition}
    
    \begin{definition}
        Let $\gamma\colon (\alpha, \beta) \to \R^n$ be a regular $C^2$ curve. Its
        curvature is defined as \[
            \kappa(t) = \frac{\norm{\gamma''\ip{\gamma'}{\gamma'} -
            \gamma'\ip{\gamma'}{\gamma''}}}{\norm{\gamma'(t)}^4}.
        \] 
    \end{definition}

    \begin{definition}
        Consider a regular smooth curve $\gamma$, such that $\ddot{\gamma}(s) \neq 0$ at
        $s$. Then, $\dot{\gamma}(s)$ and $\ddot{\gamma}(s)$ are perpendicular, and
        span the osculating plane at $\gamma(s)$.
    \end{definition}

    \begin{theorem}
        Consider a regular smooth curve $\gamma\colon (\alpha, \beta) \to \R^n$, such
        that $\ddot{\gamma}(s) \neq 0$ at $s$. \begin{enumerate}
            \itemsep0em
            \item For $s_1, s_2, s_3$ sufficiently close to $s$, the points
            $\gamma(s_i)$ are not colinear.  
            \item As $s_1, s_2, s_3$ tend to $s$, the planes $A(s_1, s_2, s_3)$ tend
            to the osculating plane at $\gamma(s)$.
            \item The circumcircle $C(s_1, s_2, s_3)$ associated with these points
            tend to a circle $C(s)$ lying in the osculating plane which passes
            through $\gamma(s)$. Furthermore, this has radius $1 /
            \norm{\ddot{\gamma}(s)}$.
            \item For $s_1$ sufficiently close to $s$, there is a unique plane
            containing $\gamma(s_1)$ and the tangent line to $\gamma$ at $s$. As
            $s_1$ tends to $s$, these planes converge to the osculating plane at
            $\gamma(s)$.
        \end{enumerate}
    \end{theorem}
    \begin{proof}
        Let $\{(s_{1n}, s_{2n}, s_{3n})\}_{n = 1}^\infty$ be a sequence converging to
        $(s, s, s)$; assume that $s_{1n} < s_{2n} < s_{3n}$. Suppose that
        $\gamma(s_{1n})$, $\gamma(s_{2n})$, $\gamma(s_{3n})$ line on a line $\ell_n$,
        for every $n$. Define the planes $V_n = \ell_n^\perp$, and look at the
        functions \[
            f_n^v(t) = \ip{\gamma(t) - \gamma(s_{1n})}{v}, \qquad v \in V_n.
        \] Notice that $s_{1n}, s_{2n}, s_{3n}$ are zeroes of $f_n^v$. Thus, we can
        choose $s_{12n}, s_{23n}$, where $s_{1n} \neq s_{12n} \leq s_{2n} \leq
        s_{23n} \leq s_{3n}$, such that $(f_n^v)'(s_{12n}) = (f_n^v)'(s_{23n}) = 0$.
        This gives \[
            \ip{\gamma'(s_{12n})}{v} = \ip{\gamma'(s_{23n})}{v} = 0.
        \] Repeating yields a point $s_{n}$ such that $\ip{\gamma''(s_n)}{v} = 0$.
        Now, there is a neighbourhood of $s$ on which \[
            \norm{\gamma'(u) - \gamma'(s)} < \epsilon, \qquad
            \norm{\gamma''(u) - \gamma''(s)} < \epsilon.
        \] As a result, \[
            \ip{\gamma'(s)}{v} \leq \norm{v}\epsilon, \qquad
            \ip{\gamma''(s)}{v} \leq \norm{v}\epsilon.
        \] Thus, given a vector in the osculating plane, $w = a\gamma'(s) +
        b\gamma''(s)$, we have $\norm{w}^2 = a^2 + b^2k^2$, and \[
            |\ip{w}{v}| \leq (|a| + |b|)\norm{v}\epsilon \leq
            c\norm{w}\norm{v}\epsilon.
        \] This means that the osculating plane is part of the
        $\epsilon$-perpendicular region to $V_n$.
    \end{proof}

    \begin{lemma}
        Let $d$ be the Euclidean distance between $\gamma(0)$ and $\gamma(s)$, and
        $s$ be the arc length between these two points. Then, \[
            \lim_{s \to 0} \frac{d}{s} = 1, \qquad \lim_{s \to 0} \frac{d - s}{s^3} =
            -\frac{1}{24}\norm{\ddot{\gamma}(0)}^2
        \] 
    \end{lemma}


    \subsection{Torsion}

    \begin{definition}
        Let $\gamma\colon (\alpha, \beta) \to \R^3$ be a $C^2$, regular space curve
        such that $\ddot{\gamma}(s) \neq 0$. The torsion of $\gamma(s)$ is defined as
        \[
            \tau(s) = \lim_{\Delta s \to 0} \frac{\Delta \theta}{\Delta s},
        \] where $\Delta \theta$ is the angle between the osculating planes to
        $\gamma$ at $s$ and $s + \Delta s$.

        \begin{remark}
            When we talk of the angle between two planes, we examine the normals \[
                n(s) = \frac{\dot{\gamma}(s) \times
                \ddot{\gamma}(s)}{\norm{\dot{\gamma}(s) \times \ddot{\gamma}(s)}}.
            \] 
        \end{remark}
    \end{definition}
    \begin{lemma}
        The torsion at $\gamma(s)$ can be expressed at \[
            \tau(s) = \frac{\norm{(\dot{\gamma}(s) \times \ddot{\gamma}(s))\cdot
            \dddot{\gamma}(s)}}{\norm{\ddot{\gamma}(s)}^2}.
        \] 
    \end{lemma}
    \begin{lemma}
        The torsion at $\gamma(s)$ can be expressed at \[
            \tau(t) = \frac{\det(\gamma'(t)\; \gamma''(t)\;
            \gamma'''(t))}{\norm{\gamma'(t) \times \gamma''(t)}^2}
        \] 
    \end{lemma}


    \subsection{Frenet-Serret formulas}

    \begin{definition}
        The tangent, principal normal, and binormal at $\gamma(s)$ are \[
            t(s) = \dot{\gamma}(s), \qquad
            n(s) = \frac{\ddot{\gamma}(s)}{\norm{\ddot{\gamma}(s)}}, \qquad
            b(s) = t(s) \times n(s).
        \] 
    \end{definition}

    \begin{theorem}[Frenet-Serret]
        For a unit speed curve with nowhere vanishing curvature, the following holds.
        \[
            \begin{bmatrix}
                t' \\ n' \\ b'
            \end{bmatrix} = \begin{bmatrix}
                0 & \kappa & 0 \\
                -\kappa  & 0 & \tau \\
                0 & -\tau & 0 
            \end{bmatrix} \begin{bmatrix}
                t \\ n \\ b
            \end{bmatrix}
        \]
    \end{theorem}
    \begin{corollary}
        A regular space curve with non-zero curvature everywhere is planar if and
        only if $\tau = 0$.
    \end{corollary}
    \begin{corollary}
        Under suitable conditions, a space curve is completely determined by $\kappa,
        \tau$. Two unit speed curves having the same curvature and torsion functions
        are identical, up to a rotation and a shift.
    \end{corollary}

    \begin{example}
        Suppose that a unit speed space curve $\gamma$ lies entirely on a sphere of
        radius $r$. Then we have \begin{align*}
            \norm{\gamma - a}^2 &= r^2, \\
            2\dot{\gamma}\cdot (\gamma - a) &= 0, \\
            t\cdot (\gamma - a) &= 0, \\
            \dot{t}\cdot (\gamma - a) + t\cdot\dot{\gamma} &= 0, \\
            \kappa n\cdot(\gamma - a) + 1 &= 0, \\
            n\cdot(\gamma - a) &= -1 / \kappa, \\
            \dot{n}\cdot(\gamma - a) + n\cdot\dot{\gamma} &= \dot{\kappa}/\kappa^2,
            \\
            -\kappa t\cdot(\gamma - a) + \tau b\cdot(\gamma - a) &=
            \dot{\kappa}/\kappa^2, \\
            b\cdot(\gamma - a) &= \dot{\kappa}/\kappa^2\tau, \\
            \dot{b}\cdot(\gamma - a) + b\cdot\dot{\gamma} &=
            (\dot{\kappa}/\kappa^2\tau)', \\
            -\tau n\cdot(\gamma - a) &= (\dot{\kappa}/\kappa^2\tau)', \\
            \tau / \kappa &= (\dot{\kappa}/\kappa^2\tau)'.
        \end{align*}
        Thus, our curve satisfies \[
            \frac{\tau}{\kappa} =
            \dd{}{s}\left(\frac{\dot{\kappa}}{\kappa^2\tau}\right).
        \] By setting $\rho = 1 / \kappa$, $\sigma = 1 / \tau$, this reads \[
            \rho = -\sigma \dd{}{s}(\dot{\rho}\sigma).
        \] Conversely, assume that the above holds. Consider the quantity $\rho^2 +
        (\dot{\rho}\sigma)^2$; differentiating this gives \[
            2\rho \dot{\rho} + 2\dot{\rho}\sigma\dd{}{s}(\dot{\rho}\sigma) =
            2\rho\dot{\rho} + 2\dot{\rho}(-\rho) = 0.
        \] Thus, we have \[
            \rho^2 + (\dot{\rho}\sigma)^2 = r^2
        \] for some positive constant $r$. Now, define the curve \[
            \alpha = \gamma + \rho n + \dot{\rho}\sigma b.
        \] Then, \[
            \dot{\alpha} = \dot{\gamma} + \dot{\rho}n + \rho\dot{n} +
            \dd{}{s}(\dot{\rho}\sigma)b + (\dot{\rho}\sigma)\dot{b} = t + \dot{\rho}n
            - \rho\kappa t + \rho\tau b - \rho / \sigma b - \dot{\rho}\sigma \tau n =
            0.
        \] This means that the curve $\alpha$ is constant, say $\alpha = a$, whence
        \[
            \norm{\gamma - a}^2 = \norm{\rho n + \dot{\rho}\sigma b}^2 = \rho^2 +
            (\dot{\rho}\sigma)^2 = r^2,
        \] hence $\gamma$ lies on a sphere.
    \end{example}


    \subsection{Evolutes and involutes}
    
    \begin{definition}
        The evolute of a curve is the locus of the centres of its osculating circles.
        \begin{remark}
            Consider a curve $\gamma$, with normal $n$ and radius of curvature $\rho
            = 1 / \kappa$. Then the equation of its evolute is given by $\gamma +
            \rho n$.
        \end{remark}
    \end{definition}
    \begin{example}
        The evolute of a circle is a constant curve, namely its centre.
    \end{example}
    \begin{example}
        The evolute of a cycloid is a shifted copy of itself.
    \end{example}

    \begin{definition}
        The involute of a curve is the locus of a point on a piece of taut string, as
        the string is wrapped around the curve.
        \begin{remark}
            Consider a curve $\gamma$, with tangent $t$. Then the equations of its
            involutes are given by $\gamma - t(s - a)$, for different choices of $a$.
            All such involutes are merely shifted copies of themselves.
        \end{remark}
    \end{definition}

    \begin{theorem}
        The evolute of an involute of a curve is the curve itself.
    \end{theorem}



    \section{Surfaces}

    \subsection{Introduction}

    \begin{definition}
        A surface $\Sigma \subseteq \R^3$ is a subset satisfying the property that
        for any $p \in \Sigma$, there exists an open set $W \subseteq \R^3$, an
        open set $U \subseteq \R^2$, and a homeomorphism $\varphi\colon U \to W \cap
        \Sigma$.
        \begin{remark}
            The pair $(W\cap \Sigma, \varphi^{-1})$ is called a chart around $p$.
        \end{remark}
        \begin{remark}
            Without loss of generality, we can demand that the open set $U\subseteq
            \R^2$ be the unit disc centred at $0$.
        \end{remark}
    \end{definition}

    \begin{example}
        Any affine plane in $\R^3$ is a surface.
    \end{example}
    \begin{example}
        The unit sphere $S^2$ is a surface. Note that the north hemisphere is
        homeomorphic to the unit disc via a projection map; this gives us a chart for
        $(0, 0, 1)$. By symmetry, we can produce similar charts for any point on the
        sphere.
        \begin{remark}
            The sphere cannot be realized as a surface using only a single chart.
            This is because $S^2$ is compact while the unit disc is not, hence they
            cannot be homeomorphic.
        \end{remark}
    \end{example}
    \begin{example}
        The cylinder defined by $x^2 + y^2 = 1$ is a surface. Note that we can
        produce the homeomorphism $(1, \theta, z) \mapsto (e^z, \theta)$, which maps
        the cylinder to $\R^2\setminus 0$.
        \begin{remark}
            Note that the cylinder is just $S^1\times \R$, and the plane minus the
            origin is just $S^1\times (0, \infty)$. Thus we need only find a
            homeomorphism mapping $\R \to (0, \infty)$.
        \end{remark}
    \end{example}
    \begin{example}
        The following is a parametrization of a torus. \[
            (u, v) \mapsto ((a + b\cos{u})\cos{v}, (a + b\cos{u})\sin{v}, b\sin{v}).
        \] The torus is also the zero set of the map \[
            (x, y, z) \mapsto (\sqrt{x^2 + y^2} - a)^2 + z^2 - b^2.
        \] 
    \end{example}

    \begin{example}
        Let $U\subseteq \R^2$ be open, and let $f\colon U \to \R$ be continuous. The
        graph of $f$, which is the set $\Gamma_f = \{(x, y, f(x, y)): (x, y) \in
        U\}$, is a surface.
    \end{example}

    \begin{lemma}
        All homeomorphisms $\psi\colon \R^3 \to \R^3$ take surfaces to surfaces.
    \end{lemma}
    \begin{example}
        In particular, all rigid motions take surfaces to surfaces.
    \end{example}

    \begin{theorem}
        Let $a$ be a regular value of the smooth map $f\colon U \to \R$, where
        $U\subseteq \R^3$ is open. Then, $S = f^{-1}(a)$ is a surface.
    \end{theorem}

    \begin{definition}
        A surface $\Sigma \subset \R^3$ is called smooth if it admits a covering by
        charts $\{(U_\alpha, \varphi_\alpha)\}_{\alpha \in \Lambda}$ such that
        $\varphi_\alpha^{-1}\colon \varphi_\alpha(U_\alpha) \to \R^3$ is smooth and
        $D_u(\varphi_\alpha^{-1})$, $D_v(\varphi_\alpha^{-1})$ are linearly
        independent everywhere.
        \begin{remark}
            Such charts are called regular.
        \end{remark}
    \end{definition}

    \begin{lemma}
        Let $\Sigma$ be a smooth surface. For every $p \in \Sigma$, there exists a
        smooth function $\varphi\colon U \to \R$ where $U\subseteq \R^2$ is open,
        such that $\Gamma_\varphi$ defines a regular chart of $\Sigma$ around $p$.
    \end{lemma}
    
    


\end{document}
