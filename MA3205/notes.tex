\documentclass[11pt]{article}

\usepackage[T1]{fontenc}
\usepackage{geometry}
\usepackage{amsmath, amssymb, amsthm}
\usepackage[scr]{rsfso}
\usepackage{bm}
\usepackage[%
    hidealllines=true,%
    innerbottommargin=15,%
    nobreak=true,%
]{mdframed}
\usepackage{xcolor}
\usepackage{graphicx}
\usepackage{fancyhdr}
\usepackage{hyperref}

\geometry{a4paper, margin=1in, headheight=14pt}

\pagestyle{fancy}
\fancyhf{}
\renewcommand\headrulewidth{0.4pt}
\fancyhead[L]{\scshape MA3205: Geometry of Curves and Surfaces}
\fancyhead[R]{\scshape \leftmark}
\rfoot{\footnotesize\it Updated on \today}
\cfoot{\thepage}

\newcommand{\C}{\mathbb{C}}
\newcommand{\R}{\mathbb{R}}
\newcommand{\Q}{\mathbb{Q}}
\newcommand{\Z}{\mathbb{Z}}
\newcommand{\N}{\mathbb{N}}

\newcommand{\ip}[2]{\langle #1, #2 \rangle}
\newcommand{\norm}[1]{\Vert #1 \Vert}
\renewcommand{\vec}[1]{\boldsymbol{#1}}

\newcommand{\vx}{\vec{x}}
\newcommand{\vy}{\vec{y}}
\newcommand{\vv}{\vec{v}}
\newcommand{\vw}{\vec{w}}
\newcommand{\ve}{\vec{e}}

\newcommand{\dd}[2]{\frac{d #1}{d #2}}
\newcommand{\pp}[2]{\frac{\partial #1}{\partial #2}}
\newcommand{\ddn}[3][]{\frac{d^{#1} #2}{d #3^{#1}}}
\newcommand{\ppn}[3][]{\frac{\partial^{#1} #2}{\partial #3^{#1}}}
\newcommand{\grad}{\nabla}

\newmdtheoremenv[%
    backgroundcolor=blue!10!white,%
]{theorem}{Theorem}[section]
\newmdtheoremenv[%
    backgroundcolor=violet!10!white,%
]{corollary}{Corollary}[theorem]
\newmdtheoremenv[%
    backgroundcolor=teal!10!white,%
]{lemma}[theorem]{Lemma}

\theoremstyle{definition}
\newmdtheoremenv[%
    backgroundcolor=green!10!white,%
]{definition}{Definition}[section]
\newmdtheoremenv[%
    backgroundcolor=red!10!white,%
]{exercise}{Exercise}[section]

\theoremstyle{remark}
\newtheorem*{remark}{Remark}
\newtheorem*{example}{Example}
\newtheorem*{solution}{Solution}

\surroundwithmdframed[%
    linecolor=black!20!white,%
    hidealllines=false,%
    innertopmargin=5,%
    innerbottommargin=10,%
    skipabove=0,%
    skipbelow=0,%
]{example}

\numberwithin{equation}{section}

\title{
    \Large\textsc{MA3205} \\
    \Huge \textbf{Geometry of Curves and Surfaces} \\
    \vspace{5pt}
    \Large{Spring 2022}
}
\author{
    \large Satvik Saha
    \\\textsc{\small 19MS154}
}
\date{\normalsize
    \textit{Indian Institute of Science Education and Research, Kolkata, \\
    Mohanpur, West Bengal, 741246, India.} \\
}

\begin{document}
    \maketitle

    \tableofcontents
    
    \section{Introduction}
    
    \subsection{Curves}
    
    \begin{definition}
        A curve is a continuous map $\gamma\colon \R \to \R^n$.
    \end{definition}

    \begin{definition}
        A smooth curve $\gamma\colon \R \to \R^n$ is $C^\infty$, i.e.\ differentiable
        arbitrarily times.
    \end{definition}

    \begin{definition}
        A closed curve $\gamma\colon \R \to \R^n$ is periodic, i.e\ there exists some
        $c$ such that $\gamma(t + c) = \gamma(t)$ for all $t \in \R$.
    \end{definition}
    \begin{example}
        Alternatively, a closed curve can be thought of as a continuous map
        $\gamma\colon S^1 \to \R^n$. For instance, given a closed curve $\gamma\colon
        \R \to \R^n$ with period $c$, we can define the corresponding map \[
            \tilde{\gamma}\colon S^1 \to \R^n, \qquad \tilde{\gamma}(e^{it}) =
            \gamma(ct / 2\pi).
        \] 
    \end{example}

    \begin{definition}
        A simple curve $\gamma\colon \R \to \R^n$ is injective on its period.
    \end{definition}

    \begin{theorem}[Four Vertex Theorem]
        The curvature of a simple, closed, smooth plane curve has at least two local
        minima and two local maxima.
    \end{theorem}


    \begin{definition}
        A knot is a simple closed curve in $\R^3$.
    \end{definition}

    \begin{definition}
        The total absolute curvature of a knot $K$ is the integral of the absolute
        value of the curvature, taken over the curve, i.e.\ it is the quantity \[
            \oint_K |\kappa(s)|\:ds.
        \] 
    \end{definition}
    \begin{example}
        The total absolute curvature of a circle is always $2\pi$.
    \end{example}

    \begin{theorem}[F\'ary-Milnor Theorem]
        If the total absolute curvature of a knot $K$ is at most $4\pi$, then $K$ is
        an unknot.
    \end{theorem}

    \begin{definition}
        An immersed loop $\gamma$ is such that $\gamma'$ is never zero.
    \end{definition}

    \begin{definition}
        Two loops are isotopic if there exists an interpolating family of loops
        between them. Two immersed loops are isotopic if we can choose such an
        interpolating family of immersed loops.
    \end{definition}
    \begin{example}
        Without the restriction of immersion, any two loops $\gamma, \eta\colon S^1
        \to \R^n$ would be isotopic, since we can always construct the linear
        interpolations \[
            H\colon S^1 \times [0, 1] \to \R^n, \qquad H(e^{i\theta}, t) = (1 -
            t)\gamma(e^{i\theta}) + t\eta(e^{i\theta}).
        \] 
    \end{example}

    \begin{theorem}[Hirsch-Smale Theory]
        \mbox{}
        \begin{enumerate}
            \itemsep0em
            \item Any two immersed loops in $\R^2$ are isotopic if and only if their
            turning numbers match.
            \item Any two immersed loops in $S^2$ are isotopic if and only if their
            turning numbers modulo 2 match.
        \end{enumerate}
    \end{theorem}

    \begin{definition}
        A parametrized curve in $\R^n$ is a smooth map $\gamma\colon (\alpha, \beta)
        \to \R^n$ for some $\alpha, \beta$ with $-\infty \leq \alpha < \beta \leq
        \infty$.
        \begin{remark}
            Here, we will always implicitly assume that maps are continuous.
        \end{remark}
        \begin{remark}
            Such a curve is called regular if $\gamma'(t) \neq 0$ for all $t \in
            (\alpha, \beta)$.
        \end{remark}
    \end{definition}
    \begin{example}
        The curve defined by \[
            \gamma\colon \R \to \R^n, \qquad t \mapsto a + tb
        \] is a straight line through the point $a$, in the direction $b$.
    \end{example}
    \begin{example}
        The curve defined by \[
            \gamma\colon \R \to \R^2, \qquad t \mapsto (\cos{t}, \sin{t})
        \] is the unit circle in $\R^2$, counter-clockwise.
    \end{example}
    \begin{example}
        The curve defined by \[
            \gamma\colon \R \to \R^3, \qquad t \mapsto (t, \cos{t}, \sin{t})
        \] is a helix in $\R^3$, wrapped around the $x$-axis.
    \end{example}


    \subsection{Whitney's theorem}
    
    \begin{lemma}
        Let $\Omega \subset \R^n$ be open and let $C \subseteq \Omega$ be closed.
        Then there exists a continuous function $f\colon \Omega \to \R$ such that
        $f^{-1}(0) = C$.
        \begin{remark}
            The converse, i.e.\ $f^{-1}(0) = C$ implies $C$ is closed, where $f$ is
            continuous on $\Omega$, is trivial.
        \end{remark}
    \end{lemma}
    \begin{proof}
        Set $f$ to be the distance function from $C$, i.e.\ \[
            f(x) = \inf_{y \in C} d(x, y). \qedhere
        \] 
    \end{proof}

    \begin{theorem}[Whitney's Theorem]
        Let $\Omega \subset \R^n$ be open and let $C \subseteq \Omega$ be closed.
        Then there exists a smooth function $f\colon \Omega \to \R$ such that
        $f^{-1}(0) = C$.
    \end{theorem}
    \begin{proof}
        Set $V = \Omega\setminus C$, and cover $V$ by a countable collection of open
        balls, \[
            V = \bigcup_{i = 1}^\infty B(q_k, r_k).
        \] This can always be done since $V$ is open, and using the density of $\Q$
        in $\R$ to pick only rational $q_k, r_k$. Now for each open ball $B(q_k,
        r_k)$, we can construct a smooth bump functions $f_k$ such that $f^{-1}(0) =
        \R^n\setminus B(q_k, r_k)$, $f^{-1}(1) = \overline{B(q_k, r_k / 2)}$, and all
        derivatives of $f_k$ vanish on $\R^n\setminus B(q_k, r_k)$.

        Define the weights \[
            c_k = \max_{|\alpha| \leq k, y \in \overline{B(q_k, r_k)}}
            \left|\ppn[\alpha]{f_k}{x}(y)\right|.
        \] Note that each $c_k$ is well-defined: there are finitely many multi-indices
        $\alpha$ given $k$, and each of the partials $\partial^\alpha f / \partial
        x^\alpha$ is a smooth function over a compact set, hence bounded.
        Furthermore, each $c_k \geq 1$. Finally, set \[
            f = \sum_{k = 1}^\infty \frac{f_k}{2^k c_k}.
        \] It is clear that $f^{-1}(0) = C$. We can show that the partial sums $s_n$
        converge; let $\epsilon > 0$ and choose sufficiently large $N$ such that $1 /
        2^N < \epsilon$. Now for $m > n \geq N$, examine \[
            |s_m(x) - s_n(x)| = \sum_{k = n + 1}^m \frac{f_k(x)}{2^k c_k} \leq
            \sum_{k = n + 1}^m \frac{1}{2^k} \leq \frac{1}{2^N} < \epsilon
        \] Thus, the convergence is uniform, and $f$ is $C^0$. For higher
        derivatives, we examine some $\alpha$ partial of the sums, and use the same
        argument; at each stage, $|\partial^\alpha f/ \partial x^\alpha| < c_k$
        whenever $|\alpha| < k$.
    \end{proof}

\end{document}
