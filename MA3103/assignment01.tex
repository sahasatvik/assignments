\documentclass[10pt]{article}

\usepackage[T1]{fontenc}
\usepackage{geometry}
\usepackage{amsmath, amssymb, amsthm}

\geometry{a4paper, margin=1in}

\renewcommand{\labelenumi}{(\roman{enumi})}

\newcounter{prob}
\newcommand{\problem}{\stepcounter{prob}\paragraph{Exercise \arabic{prob}}}
\newcommand{\solution}{\paragraph{Solution}}

\newcommand{\C}{\mathbb{C}}
\newcommand{\R}{\mathbb{R}}
\newcommand{\Q}{\mathbb{Q}}
\newcommand{\Z}{\mathbb{Z}}
\newcommand{\N}{\mathbb{N}}

\title{MA3103: Introduction to Graph Theory and Combinatorics}
\author{Satvik Saha}
\date{}

\begin{document}
    \noindent\textbf{IISER Kolkata} \hfill \textbf{Assignment I}
    \vspace{3pt}
    \hrule
    \vspace{3pt}
    \begin{center}
    \LARGE{\textbf{MA3101 : Introduction to Graph Theory and Combinatorics}}
    \end{center}
    \vspace{3pt}
    \hrule
    \vspace{3pt}
    Satvik Saha, \texttt{19MS154} \hfill \today
    \vspace{20pt}

    \problem Show that if $n + 1$ integers are chosen from the set $\{1, 2, \dots,
    mn\}$, then there are always two which differ by less than $m$.

    \solution Consider the set $S_a = \{am + b: 1 \leq b \leq m\}$ for each $0 \leq a
    < n$. These are disjoint, and their union is precisely $\{1, 2, \dots, mn\}$.
    Since there are a total of $n$ sets, the Pigeonhole Principle guarantees that
    upon choosing $n + 1$ integers from $\{1, 2, \dots, mn\}$, there are at least two
    which belong to the same set; say $p, q \in S_a$ for some $a$.  Thus write $p =
    am + b_p$, $q = am + b_q$, let $p > q$ without loss of generality, and note that
    $p - q = b_p - b_q \leq m - 1 < m$.

    \problem From the integers $1, 2, ... , 200$, if we choose 101 integers, show
    that among the integers chosen, there are two such that one of them is divisible
    by the other.

    \solution Note that every positive integer can be written uniquely in the form $n
    = 2^ab$, where $b$ is odd (this is a consequence of the uniqueness of the prime
    factorization of natural numbers). Define the sets $S_b = \{2^ab : a \in \Z_{\geq
    0}\}$ for odd $b$. Then, the numbers $1, 2, \dots, 200$ are covered by the sets
    $S_1, S_3, \dots, S_{199}$ of which there are $100$. Thus, upon choosing $101$
    integers from $1, 2, \dots, 200$, two of them must lie in the same set $S_b$ by
    the Pigeonhole Principle; say $p, q \in S_b$ with $p > q$. Thus, $p = 2^{a_p}b$,
    $q = 2^{a_q}b$ with $a_p > a_q$, giving $q | p$ (with quotient $2^{a_p - a_q}$).

    \problem $n + 2$ numbers between $0$ and $2n + 1$ are chosen. Show that two of
    them sum to $2n + 1$.

    \solution Consider the sets $\{0, 2n + 1\}, \{1, 2n\}, \dots \{n, n + 1\}$ of
    which there are $n + 1$. Each of them is of the form $\{j, 2n + 1 - j\}$ where $0
    \leq j \leq n$, hence the sum of the elements in each set is precisely $2n + 1$.
    Also, their union is precisely $\{0, 1, \dots, 2n + 1\}$.  The Pigeonhole
    Principle guarantees that upon choosing $n + 2$ (distinct) numbers from $0, 1,
    \dots, 2n + 1$, two must lie in the same set, completing the proof.

    \problem Suppose 9 points are randomly placed inside a unit square. Show that 3
    of the points form a triangle whose area is at most $1 / 8$ units.

    \solution Cut the square into four congruent squares, each of side length $1 / 2$
    units (we do not care how the common boundaries are distributed among the
    squares). The Pigeonhole Principle (strong form) guarantees that upon choosing $9
    = 2\cdot 4 + 1$ points from the unit square, 3 must lie within one of our four
    smaller squares. We state without proof that any triangle enclosed by a square
    has area at most half that of the square. Thus, these three selected points have
    area at most $(1 / 2)^2 / 2 = 1 / 8$ square units.

    \problem 51 points are placed, in a random way, into a square of side 1
    unit. Prove or disprove that 3 of these points can be covered by a circle of
    radius $1 / 7$ units.

    \solution Note that a circle of radius $1 / 7$ contains an inscribed square of
    side length $\sqrt{2} / 7$. Now, consider a $5\times 5$ grid of such squares;
    since $5\cdot \sqrt{2} / 7 > 1$ (note that $25\cdot 2 > 49$), our unit square can
    be placed within this grid, which thus divides the unit square into $25$ parts.
    The Pigeonhole Principle guarantees that upon choosing $51 = 2\cdot 25 + 1$
    points from our unit square, 3 must lie within the same part, i.e.\ a square of
    side length $\sqrt{2} / 7$. Simply inscribe this square in a circle of radius $1
    / 7$, completing the proof.

    \problem There are 13 points randomly placed on the periphery of a circle. Now
    construct a graph where the vertices are the given 13 points and two points are
    connected by an edge if and only if the angle formed by extending two points to
    the center is less than $\pi / 2$. Show that there are at least 7 triangles in
    that graph.

    \solution Divide the periphery of the triangle into 4 equal arcs, each subtending
    $\pi / 2$. Thus, any two points on the same arc subtend an angle less than $\pi /
    2$, meaning that any three points on the same arc form a triangle in our graph.
    The Pigeonhole Principle guarantees that upon choosing $13 = 3\cdot 4 + 1$ points
    on the circumference, at least 4 must lie on the same arc. Thus, without loss of
    generality, let the number of points on the four arcs be $w \geq x \geq y \geq
    z$. If $w \geq 5$, we already have $\binom{5}{3} = 10$ triangles so we are done.
    Otherwise, we are forced to have $w = 4$ giving us $\binom{4}{3} = 4$ triangles.
    Now, $x + y + z = 13 - 4 = 9$, forcing $x \geq 3$ (otherwise $x + y + z < 3 + 3 +
    3 = 9$). If $x = 4$, we have another $\binom{4}{3} = 4$ triangles, bringing the
    total to $4 + 4 = 8$ so we are done. Otherwise, $x = 3$ gives us $1$ additional
    triangle, bringing the total to $4 + 1 = 5$. Now, $y + z = 9 - 3 = 6$, so $y \geq
    3$ (otherwise $y + z < 3 + 3 = 6$). This forces $y = 3, z = 3$, giving us $1 + 1
    = 2$ additional triangles, bringing the total to $5 + 2 = 7$, as desired.

    \problem In how many ways can you write the numbers $1, 2, \dots, n$ in a
    different order such that any number $i$ does not occupy the
    $i$\textsuperscript{th} position?

    \solution We want to count the number of permutations of $1, 2, \dots, n$ that do
    not fix any of the numbers. Let $P_i$ be the set of permutations of $1, 2, \dots,
    n$ that fix the number $i$. It is clear that such permutations send $i
    \rightsquigarrow i$, and the remaining numbers in all possible ways amongst
    themselves, giving $|P_i| = (n - 1)!$. Now, any intersection of the form $P_A =
    P_{i_1} \cap \dots \cap P_{i_k}$ where $A = \{i_1, \dots, i_k\}$ contains those
    permutations which fix $k$ numbers, hence there are $(n - k)!$ of them. Further
    note that there are $\binom{n}{k}$ such subsets of the form $A$. We want to
    calculate the number of permutations of $1, 2, \dots, n$ which fix \emph{none} of
    the elements, i.e.\ $n! - |P_1 \cup \dots P_n|$ (note that there are $n!$
    permutations in general). The Principle of Inclusion and Exclusion gives 
    \begin{align*}
        |P_1 \cup \dots \cup P_n| &= \sum_{k=1}^n (-1)^{k - 1}\sum_{\substack{A
        \subseteq \{1, 2, \dots, n\} \\ |A| = k}} |P_A| \\
        &= \sum_{k = 1}^n (-1)^{k - 1} \binom{n}{k} (n - k)! \\
        &= \sum_{k = 1}^n (-1)^{k - 1} \frac{n!}{k!}.
    \end{align*}
    Thus, the number of permutations under consideration are 
    \begin{align*}
        n! - |P_1 \cup \dots \cup P_n| &= n! - \sum_{k = 1}^n (-1)^{n - 1}
        \frac{n!}{k!} \\
        &= n!\sum_{k = 0}^n \frac{(-1)^{k}}{k!}.
    \end{align*}
    \emph{Remark.} As $n \to \infty$, the ratio of such permutations to the total
    number of permutations $n!$ approaches $e^{-1} \approx 0.37$.

    \problem Let $S = \{1, 2, \dots, 280\}$. Find the smallest natural number $n$
    such that every $n$ element subset of $S$ contains 5 pairwise relatively prime numbers.

    \problem There are 1990 scientists. Every one of them has collaborated with at
    least 1327 others. Show that we can find 4 scientists, every pair of them
    collaborated with each other.

    \solution Construct a graph $G$ on 1990 vertices representing these scientists,
    such that two vertices are connected by ad edge if and only if the corresponding
    scientists have collaborated. We claim that this graph contains $K_4$. Note that
    each vertex has degree at least $1327$, hence the number of edges is \[
        |E| = \frac{1}{2}\sum_{v \in V} d(v) \geq \frac{1}{2}\cdot 1327|V| =
        1,320,365.
    \] Now, note that \[
        \frac{|V|^2}{2}\left(1 - \frac{1}{3}\right) = 1,320,033 \frac{1}{3}.
    \] Thus, Turan's Theorem guarantees the existence of a 4-clique, as
    desired\footnote{It is perhaps easier to note that $1327 > 2\cdot 1990 / 3$.}.

    \problem There are five boxes with colours red, blue, yellow, green, and white
    respectively. There are also five balls with the colours red, blue, yellow,
    green, and white, respectively. How many ways can one arrange one ball in each
    box such that no box contains the ball with the same colour.

    \solution We use the result from Exercise 7 to see that the number of ways is \[
        5!\left[1 - \frac{1}{1!} + \frac{1}{2!} - \frac{1}{3!} + \frac{1}{4!} -
        \frac{1}{5!}\right] = 44.
    \]

    \problem Let $X$ be the set of $n$ points randomly located on a closed disc of
    radius $1 / \sqrt{2}$. Construct a graph with all these points as vertices; two
    points are connected by an edge if and only if the distance between them is
    greater than $1$. Prove or disprove that there is no $K_4$ in that graph.

    \solution None of these graphs contains $K_4$. This is clear for $n < 4$. In
    general, we show that given any four points on the unit disc, the minimum
    distance between them is at most $1$. This in turn implies that given any four
    vertices from our graph, there is at least one pair which is not connected by an
    edge, proving the result.

    We first show that two points in the same quadrant are separated by at most $1$
    unit. In other words, if $\theta$ is the (smaller) angle subtended by the line
    segment joining the points with the center, we claim that when $\theta \leq \pi /
    2$, the length of this line segment is at most $1$ unit. This is clear from the
    cosine rule and the triangle inequality: let the distances between the points and
    the center of the disc be $r_1$, $r_2$. Clearly $0 \leq r_1, r_2 \leq 1 /
    \sqrt{2}$, and when $0 \leq \theta \leq \pi / 2$, we have $0 \leq \cos\theta \leq
    1$ so the square of the distance between the points is \[
        r_1^2 + r_2^2 - 2r_1r_2\cos\theta \leq \frac{1}{2} + \frac{1}{2} -
        \frac{2}{2}\cdot 0 = 1.
    \] The contrapositive says that if the distance between two points is greater
    than $1$, then their angular separation (with respect to the center) is strictly
    greater than $\pi / 2$.  Now, if four points $P, Q, R, S$ are such that their
    pairwise distances are all greater than $1$ unit, then let $\theta_1, \theta_2,
    \theta_3, \theta_4$ be the angles between $P$ and $Q$, $Q$ and $R$, $R$ and $S$,
    $S$ and $P$, where $P, Q, R, S$ have been labelled anticlockwise about the
    center. This ensures that each $0 \leq \theta_i < 2\pi$, and that the angles
    $\theta_i$ must sum to exactly one full turn $2\pi$.  We now demand $\theta_1 +
    \theta_2 + \theta_3 + \theta_4 > 4\cdot \pi /2 = 2\pi$ due to the distance
    property, which is absurd.

    \problem Let $e(n)$ denote the largest number of edges among all triangle-free
    graphs that are non-bipartite. For all $n \geq 5$, prove that 
    \begin{enumerate}
        \itemsep0em
        \item $e(n) \leq ((n - 1)^2 + 4) / 4$
        \item $e(n) = ((n - 1)^2 + 4) / 4$, if $n$ is odd.
    \end{enumerate}

    \solution \mbox{}
    \begin{enumerate}
        \item \textbf{(Attempt)}
        Let $G$ be a triangle-free non-bipartite graph on $n$ vertices and $m$ edges.
        Furthermore, let $A$ be a maximal independent set of vertices within $G$ with
        $|A| = r$, and let $B$ be the set of remaining $n - r$ vertices.  If $n = r$,
        we are done ($G$ has no edges). Now, $B$ \emph{cannot} be independent since
        $G$ is not bipartite; this also ensures that $n - r \geq 2$ or $r \leq n -
        2$. Let $e_{AB}$ denote the number of edges between $A$ and $B$, and $e_{B}$
        denote the number of edges within $B$ (clearly $e_A = 0$), hence $m = e_{AB}
        + e_B$. Since $A$ is the maximal independent set, the degree of any vertex
        (which is the number of elements in its neighbouring set, which in turn is
        independent since $G$ is triangle-free) is at most $|A| = r$. Also, every
        edge in $G$ has an endpoint in $B$ (both endpoints cannot be in $A$ since it
        is an independent set).

        First note that every vertex $b \in B$ must be connected to at least one
        vertex in $A$; otherwise, $A\cup \{b\}$ is an independent set, contradicting
        the maximality of $A$. Given any edge $\{x, y\} \subset B$, both $x$ and $y$
        cannot be connected to the same vertex in $A$, since that would form a
        triangle.

        Suppose that there is a vertex $v_0$ with degree $0$. Let $G_0$ be the graph
        on $n - 1$ vertices and $m$ edges obtained by removing $v_0$ from $G$. Since
        $G_0$ is triangle-free, Mantel's Theorem guarantees that $m \leq (n - 1)^2 /
        4 < (n - 1)^2 / 4 + 1$ as desired.

        Similarly, suppose that there is a vertex $v_1$ with degree $1$. Let $G_1 =
        G\setminus\{v_1\}$ be the induced subgraph on $n - 1$ vertices and $m - 1$
        edges, whence $m - 1 \leq (n - 1)^2 / 4$, or $m \leq (n - 1)^2 / 4 + 1$ as
        desired.

        \item Set $s = \lfloor n / 2\rfloor = (n - 1) / 2$, and construct the graph
        $G$ by taking the complete bipartite graph, $K_{s, s}$, removing an edge
        $\{x, y\}$ (note that $x$ and $y$ belong to separate partites) and inserting
        a vertex $v$ and the edges $\{x, v\}$, $\{v, y\}$. In other words, we have
        inserted the vertex $v$ `inside' one of the edges bridging the two partites.
        This graph has $2s + 1 = n$ vertices, $s^2 + 1 = ((n - 1)^2 + 4) / 4$ edges.
        Furthermore, this graph is non-bipartite; the vertices in the already
        existing partites must remain there, since every vertex from one partite has
        an edge connecting it to the other. This leaves no choice for the vertex $v$,
        which is connected to both partites. Thus, we have achieved the upper bound.
    \end{enumerate} 

    \problem Let $G$ be a graph on $n$ vertices and $m$ edges. Then, show that
    \begin{enumerate}
        \itemsep0em
        \item $G$ contains at least $4m/3n \cdot (m - n^2 / 4)$ triangles.
        \item $G$ contains at least $\lfloor n / 2\rfloor$ triangles if $m \geq
        \lfloor n^2 / 4\rfloor + 1$.
    \end{enumerate}

    \solution \mbox{}
    \begin{enumerate}
        \item Let $G$ be a graph on $n$ vertices and $m$ edges. Pick an arbitrary
        edge $\{x, y\}$ and let $A$ and $B$ be the neighbouring sets of $x$ and $y$
        respectively, both excluding the vertices $x$ and $y$. We denote the degree of
        a vertex $v$ by $d(v)$. Now, for every vertex $z \in A \cap B$, observe that
        $x,y,z$ form a triangle. Furthermore, $|A| = d(x) - 1$, $|B| = d(y) - 1$, $|A
        \cup B| \leq n - 2$, so \[
            |A \cap B| \geq d(x) - 1 + d(y) - 1 - n + 2 = d(x) + d(y) - n.
        \] We sum this over all $m$ edges of the given form. On the left, we have
        counted the total number of triangles thrice (each triangle has been included
        once for each of its three sides). On the right, each $d(x)$ appears as many
        times as $x$ is in an edge, i.e.\ $d(x)$ many times. Thus, thrice the number
        of triangles is at least \[
            \sum_{x \in V} d(x)^2 - mn \geq \frac{1}{n}\left(\sum_{x \in
            V}d(x)\right)^2 - mn = \frac{(2m)^2}{n} - mn = \frac{4m^2}{n} - mn.
        \] The number of triangles is at least \[
            \frac{1}{3}\left(\frac{4m^2}{n} - mn\right) = \frac{4m}{3n}\left(m -
            \frac{n^2}{4}\right).
        \] 

        \item We show this by induction on $n$. The base cases $n < 5$ are clear
        by inspection, so we let $n \geq 5$ and suppose that the statement holds for
        all lower $n$.

        \textbf{Case I} When $n = 2k + 1$ is odd, we want to show that there are
        $\lfloor n / 2\rfloor = k$ triangles if $m = \lfloor n^2 / 4\rfloor + 1 =
        k^2 + k + 1$; clearly, adding more edges will only increase the number of
        triangles. Now, we claim that there is a vertex $v$ such that $d(v) \leq k$;
        if not, then all vertices have degree greater than $k$ hence the sum of the
        degrees is at least $n(k + 1) = 2k^2 + 3k + 1$, but the sum of degrees is
        exactly $2m = 2k^2 + 2k + 2$. This forces $2k^2 + 2k + 2 \geq 2k^2 + 3k + 1$,
        or $1 \geq k$, a contradiction. Thus, upon selecting this vertex $v$, remove
        it from the graph, yielding a new graph $G'$ with $n - 1 = 2k$ vertices and
        at least $m - k = k^2 + 1 = \lfloor (n - 1)^2 / 4\rfloor + 1$ edges. Applying
        the induction hypothesis, $G'$ contains at least $\lfloor (n - 1) / 2\rfloor
        = k$ triangles, and hence so does $G$.
        
        \textbf{Case II} When $n = 2k$ is even, we want to show that there are
        $\lfloor n / 2\rfloor = k$ triangles if $m = \lfloor n^2 / 4\rfloor + 1 =
        k^2 + 1$. Again, we claim that there is a vertex $v$ such that $d(v) \leq k$;
        if not, the sum of degrees $2k^2 + 2 = 2m \geq n(k + 1) = 2k^2 + 2k$, forcing
        $1 \geq k$, a contradiction.

        First suppose that $d(v) < k$, hence the graph $G'$ obtained by removing $v$
        has $n - 1 = 2k - 1$ vertices and at least $m - (k - 1) = k^2 - k + 2 =
        \lfloor(n - 1)^2 / 4\rfloor + 2$ edges. Thus, it must contain at least
        $\lfloor (n - 1) / 2\rfloor = k - 1$ triangles. Now, select a triangle from
        $G'$ and remove an edge from it to obtain the graph $G''$, which thus has at
        least $\lfloor(n - 1)^2 / 4\rfloor + 1$ edges and hence contains at least $k
        - 1$ triangles. Add the missing edge back to see that $G'$ has at least $(k -
        1) + 1 = k$ triangles, and hence so does $G$.

        Otherwise, all vertices have degree at least $k$. Suppose that $G$ has less
        than $k$ triangles. Note that every edge cannot
        be part of a triangle; if so, the total number of edges would be less than
        $3k < k^2 + 1$. Thus, pick such an edge $\{x, y\}$ and remove both $x$ and
        $y$ from $G$ to obtain $G'$. Let $A$ and $B$ be the neighbouring sets of $x$
        and $y$ in $G$, excluding $x, y$. Thus, $A \cap B = \emptyset$ since $\{x,
        y\}$ does not contribute to a triangle, and $|A|, |B| \geq k - 1$ since $x,
        y$ have degrees at least $k$. In other words, $|A \cup B| \geq 2k - 2 = n -
        2$, but $|G'| = n - 2$, which means that every vertex in $G'$ is in one of
        $A$ or $B$ as well as $|A| = |B| = k - 1$. Thus, we have removed exactly $1
        + (k - 1) + (k - 1) = 2k - 1 = n - 1$ edges from $G$ to obtain $G'$, hence it
        contains $k^2 + 1 - (2k - 1) = k^2 - 2k + 2 = \lfloor (n - 2)^2 / 4\rfloor +
        1$ edges. Thus, $G'$ contains at least $\lfloor (n - 2) / 2\rfloor = k - 1$
        triangles. Since $G$ has less than $k$ triangles, this forces $G$ and $G'$ to
        have exactly $k - 1$ triangles. Thus there are no edges within $A$, nor in
        $B$; if there were, they would introduce an additional triangle into $G$ via
        $x$ or $y$. In other words, $G'$ is a bipartite graph, hence triangle free,
        which is a contradiction.
    \end{enumerate}

    \problem Let $S$ be a set of $n$ points in $\R^2$ such that the distance between
    any pair of points is at most $1$. Prove that there are at most $\lfloor n^2 /
    3\rfloor$ pairs of points in $S$ whose distance is greater than $1 / \sqrt{2}$.

    \solution Let $G$ be the graph on $n$ vertices each representing a point in $S$,
    and let two vertices be connected by an edge if and only if the distance between
    them is greater than $1 / \sqrt{2}$. We will show that this graph is $K_4$ free,
    whence Turan's Theorem guarantees that \[
        |E| \leq \left\lfloor \frac{n^2}{2} \left(1 - \frac{1}{3}\right)\right\rfloor
        = \left\lfloor \frac{n^2}{3}\right\rfloor.
    \] Suppose to the contrary that $G$ contains a $K_4$, formed by the vertices $P,
    Q, R, S$. Make sure that these labels are such that $PQRS$ is a simple
    quadrilateral. Each of the segments has length greater than $1 / \sqrt{2}$, and
    at most $1$ (this prevents any three points from being collinear, $2\cdot 1 /
    \sqrt{2} = \sqrt{2} > 1$). Let $\theta$ be the (smaller, positive) angle $\angle
    PQR$, $x = PQ$, $y = QR$, $z = PR$.  The cosine rule gives \[
        z^2 = x^2 + y^2 - 2xy\cos\theta, \qquad \cos\theta = \frac{x^2 + y^2 -
        z^2}{2xy} > 0.
    \] The latter follows because $x^2 + y^2 - z^2 > \frac{1}{2} + \frac{1}{2} - 1 =
    0$. Thus, we have $\theta < \pi / 2$. Repeating this for all possible angles, we
    see that the quadrilateral $PQRS$ is convex, saving us from accounting for reflex
    angles; to see this, note that a concave quadrilateral has at most one reflex
    angle, say $\angle PQR$. However, the interior angles $\angle PQS$ and $\angle
    SQR$ must both be strictly less than $\pi / 2$, hence their sum is less than
    $\pi$.  Furthermore, we demand that the sum of the four interior angles in the
    quadrilateral $PQRS$ must be strictly less than $4\cdot \pi / 2 = 2\pi$, which is
    absurd.

    \problem Let $G$ be a graph on $n$ vertices and without any cycle of length $4$.
    Show that $G$ has at most $n / 2 \cdot (\sqrt{n} + 1)$ edges.

    \solution Let $m$ be the number of edges in $G$. Consider an arrangement of
    vertices $w, x, y$ where $\{w, x\}$ and $\{x, y\}$ are edges -- this forms a
    V-shape. Upon fixing the extreme points $w, y$, we see that this pair can
    contribute to at most one V-shape in order for $G$ to be $C_4$ free (otherwise,
    $w, x, y, x'$ would be a 4-cycle). This means that there are at most \[
        \binom{n}{2} = \frac{1}{2}n(n - 1)
    \] such V-shapes. On the other hand, upon fixing the central point $x$, we see
    that it is part of exactly $\binom{d(x)}{2}$ V-shapes since there are $d(x)$
    neighbours of $x$ to choose from. This means that the total number of V-shapes
    is \[
        \frac{1}{2}\sum_{x \in V} [d(x)^2 - d(x)] \geq
        \frac{1}{2}\left[\frac{4m^2}{n} - 2m\right].
    \] Comparing, we have \[
        n(n - 1) \geq \frac{4m^2}{n} - 2m, \qquad 4m^2 - 2mn - n^2(n - 1) \leq 0.
    \] This is a quadratic in $m$; for this to hold, we demand \[
        m \leq \frac{1}{8}[2n + \sqrt{4n^2 + 4\cdot 4\cdot n^2(n - 1)}] =
        \frac{1}{4}[n + \sqrt{4n^3 - 3n^2}]
    \] We claim $\sqrt{4n - 3} \leq 1 + 2\sqrt{n}$, which is true since $4n - 3 \leq
    (1 + 2\sqrt{n})^2 = 1 + 4n + 4\sqrt{n}$, i.e.\ $0 \leq 4 + 4\sqrt{n}$. This
    immediately gives \[
        m \leq \frac{n}{2}\cdot \frac{1 + \sqrt{4n - 3}}{2} \leq \frac{n}{2}\cdot
        \frac{1 + 1 + 2\sqrt{n}}{2} = \frac{n}{2}(1 + \sqrt{n}).
    \] 
    

\end{document}
