\documentclass[10pt]{article}

\usepackage[T1]{fontenc}
\usepackage{geometry}
\usepackage{amsmath, amssymb, amsthm}

\geometry{a4paper, margin=1in}

\renewcommand{\labelenumi}{(\roman{enumi})}

\newcounter{prob}
\newcommand{\problem}{\stepcounter{prob}\paragraph{Exercise \arabic{prob}}}
\newcommand{\solution}{\paragraph{Solution}}

\newcommand{\C}{\mathbb{C}}
\newcommand{\R}{\mathbb{R}}
\newcommand{\Q}{\mathbb{Q}}
\newcommand{\Z}{\mathbb{Z}}
\newcommand{\N}{\mathbb{N}}

\title{MA3103: Introduction to Graph Theory and Combinatorics}
\author{Satvik Saha}
\date{}

\begin{document}
    \noindent\textbf{IISER Kolkata} \hfill \textbf{Assignment I}
    \vspace{3pt}
    \hrule
    \vspace{3pt}
    \begin{center}
    \LARGE{\textbf{MA3101 : Introduction to Graph Theory and Combinatorics}}
    \end{center}
    \vspace{3pt}
    \hrule
    \vspace{3pt}
    Satvik Saha, \texttt{19MS154} \hfill \today
    \vspace{20pt}

    \problem Show that if $n + 1$ integers are chosen from the set $\{1, 2, \dots,
    mn\}$, then there are always two which differ by less than $m$.

    \solution Consider the set $S_a = \{am + b: 1 \leq b \leq m\}$ for each $0 \leq a
    < n$. These are disjoint, and their union is precisely $\{1, 2, \dots, mn\}$.
    Since there are a total of $n$ sets, the Pigeonhole Principle guarantees that
    upon choosing $n + 1$ integers from $\{1, 2, \dots, mn\}$, there are at least two
    which belong to the same set; say $p, q \in S_a$ for some $a$.  Thus write $p =
    am + b_p$, $q = am + b_q$, let $p > q$ without loss of generality, and note that
    $p - q = b_p - b_q \leq m - 1 < m.
    

\end{document}
