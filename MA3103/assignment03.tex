\documentclass[10pt]{article}

\usepackage[T1]{fontenc}
\usepackage{geometry}
\usepackage{amsmath, amssymb, amsthm}
\usepackage[scr]{rsfso}
\usepackage{graphicx}

\geometry{a4paper, margin=1in}

\renewcommand{\labelenumi}{(\alph{enumi})}

\newcounter{prob}
\newcommand{\problem}{\stepcounter{prob}\paragraph{Exercise \arabic{prob}}}
\newcommand{\solution}{\paragraph{Solution}}

\newcommand{\C}{\mathbb{C}}
\newcommand{\R}{\mathbb{R}}
\newcommand{\Q}{\mathbb{Q}}
\newcommand{\Z}{\mathbb{Z}}
\newcommand{\N}{\mathbb{N}}

\DeclareMathOperator{\diam}{diam}
\DeclareMathOperator{\girth}{girth}

\newtheorem{theorem}{Theorem}
\newtheorem{lemma}{Lemma}

\title{MA3103: Introduction to Graph Theory and Combinatorics}
\author{Satvik Saha}
\date{}

\begin{document}
    \noindent\textbf{IISER Kolkata} \hfill \textbf{Assignment III}
    \vspace{3pt}
    \hrule
    \vspace{3pt}
    \begin{center}
    \LARGE{\textbf{MA3101 : Introduction to Graph Theory and Combinatorics}}
    \end{center}
    \vspace{3pt}
    \hrule
    \vspace{3pt}
    Satvik Saha, \texttt{19MS154} \hfill \today
    \vspace{20pt}

    \problem Prove that the number of edge disjoint Hamiltonian cycles in $K_{2n +
    1}$ for $n \geq 3$ is $n$.

    \solution It is clear that there are at most $n$ edge disjoint Hamiltonian cycles
    in $K_{2n + 1}$. To see this, pick an arbitrary vertex $x$ and note that there
    are $2n$ edges emanating from it. Each Hamiltonian cycle contains $x$, and hence
    exhausts two of these edges (one entering, one exiting). This means that there
    are at most $n$ edge disjoint Hamiltonian cycles; if there were more, that would
    necessitate more than $2n$ distinct edges coming out of $x$.

    Next, we show that there are at least $n$ edge disjoint Hamiltonian cycles. 

    Note that if $2n + 1 = p$ is prime, then the $n$ edge disjoint Hamiltonian cycles
    are precisely the sequences \[
        x_{0}, x_{d}, x_{2d}, \dots, x_{(p - 1)d}
    \] where the indices are taken modulo $p$, and $1 \leq d \leq n$. This is easily
    seen from the fact that every non-identity element of the cyclic group $\Z_p$
    generates it\footnote{Every non-identity element generates a non-trivial
    subgroup of $\Z_p$ whose order must divide $p$; this can only happen if the order
    is precisely $p$, i.e\ the element generated the entire group $\Z_p$.}; thus, the
    arithmetic progression $dk$ modulo $p$ cycles through every element of $\Z_p$.
    Any two such cycles are edge disjoint, since the (absolute) difference modulo $p$
    between the indices of any two consecutive elements in such a cycle is unique to
    that cycle\footnote{Suppose that the edge $\{x_i, x_j\}$ appears in two such
    cycles, generated by $d_1$ and $d_2$. Then, we know that $j - i \equiv \pm d_1
    \pmod{p}$ and $j - i \equiv \pm d_2 \pmod{p}$ by construction. Thus, the quantity
    $|j - i| \pmod{p}$ must be exactly one of $d_1$ or $p - d_1$, and exactly one of
    $d_2$ or $p - d_2$. If this happens to be $d_1$, then $p - d_2 > n \geq d_1$, so
    $d_1 = d_2$. Again if this happens to be $p - d_1$, then $p - d_1 > n \geq d_2$,
    so $p - d_1 = p - d_2$ or $d_1 = d_2$. This shows that this edge belongs to only
    one of these cycles.}. Note that $n + 1 \leq d \leq 2n + 1$ generate exactly the
    same cycles but in reverse, since $-d \equiv (2n + 1) - d \pmod{p}$.


    \problem Find the number of distinct Hamiltonian cycles in $K_n$ containing a
    particular edge.

    \solution Note that any permutation of the vertices of $K_n$ gives a Hamiltonian
    cycle, since there is an edge available between any two vertices. Thus, there are
    $(n - 1)! / 2$ Hamiltonian cycles in total (we divide by $2n$ to account for the
    `dihedral symmetry', i.e.\ given a permutation of vertices, we can cycle the
    vertices in $n$ ways, or reverse the direction of each of these, without getting
    a new Hamiltonian cycle\footnote{Formally, we are looking at the dihedral group
    $D_{2n}$ acting on the set of all permuted strings of vertices, and counting the
    number of orbits. Another way to derive this is to use the Orbit Stabilizer
    theorem; let $X$ be the set of all Hamiltonian cycles of $K_n$, and let $S_n$ act
    on it in the natural way, permuting the order of the vertices in a given cycle .
    To make this operation well-defined, we need to decide upon some `canonical'
    representation of a Hamiltonian cycle; label the vertices $1, \dots, n$
    beforehand, have the representation of each cycle start with $1$, and have the
    lower-labelled neighbour of $1$ follow. For instance, the cycle $3,1,4,2$ is
    canonically written as $1,3,2,4$; now, the permutation of these labels is
    well-defined. Note that any Hamiltonian cycle $x$ can be mapped to another simply
    by permuting  the order of the vertices, hence the orbit of $x$ is the entirety
    of $X$.  Cycling the order of the vertices in a Hamiltonian cycle, or reversing
    their order, gives back the same cycle; in addition, these are the only
    operations which do so. Thus, the stabilizer of a cycle $x$ is the dihedral group
    $D_{2n}$. Thus, $|S_n| = |X|\cdot |D_{2n}|$, or $|X| = n! / 2n = (n - 1)! /
    2$}).
    
    Let $n \geq 3$. Fix the edge $\{x_1, x_2\}$, and declare $x_1, x_2$ to be the
    start of our Hamiltonian cycle. There are $n - 2$ choices for the next vertex, $n
    - 3$ for the fourth, and so on. Thus, the remaining vertices can be permuted in
    $(n - 2)!$ ways giving that many cycles. Note that any two cycles obtained in
    this way are distinct -- neither sequence of vertices can be rotated nor
    reflected onto the other. Also, these sequences form a complete list of desired
    cycles -- every Hamiltonian cycle containing $\{x_1, x_2\}$ can be laid out as a
    sequence of vertices, with $x_1$ leading and $x_2$ in the second place.  Thus,
    there are precisely $(n - 2)!$ Hamiltonian cycles containing a particular edge.

    
    \problem Show that if $n$ is odd, it is not possible for a knight to visit all
    the squares of an $n \times n$ chessboard exactly once by knight's moves and
    return to its starting point.

    \solution We claim that for an odd chessboard, there exists no \emph{Knight's
    Tour}, i.e.\ a Hamiltonian cycle in the chessboard graph, whose vertices are the
    squares of the chessboard, two of them connected if and only if they are a
    knight's move away.

    It is clear that on an $n \times n$ chessboard where $n$ is odd, there are $(n^2
    - 1) / 2$ and $(n^2 + 1) / 2$ squares of each colour, say black and white without
    loss of generality (put white on the bottom right and count). This induces a
    natural colouring of the vertices of our graph. Now, a knight on a chessboard
    always alternates colours when it moves on a chessboard (something very familiar
    to chess players!); a knight on a white square can only reach black squares, and
    vice versa. This immediately shows that our chessboard graph is bipartite, with
    black vertices in one part and white vertices in the other. As a result, the
    chessboard graph contains no odd cycles\footnote{Suppose that $x_1, \dots, x_{2k
    + 1}$ is an odd cycle in a bipartite graph, with parts $A$ and $B$. Then, each
    edge joins a vertex from $A$ to one of $B$. Without loss of generality, let $x_1
    \in A$, hence $x_2 \in B$, $x_3 \in A$, \dots, $x_{2k + 1} \in A$ going forwards.
    This forces the edge $\{x_1, x_{2k + 1}\}$, contradicting the independence of the
    set of vertices $A$.}, hence no Hamiltonian cycle of length $n^2$ (which is odd).



    \problem Let $G$ be a Hamiltonian graph and let $S$ be any set of $k$ vertices in
    $G$. Prove that the graph $G - S$ has at most $k$ components.

    \solution Let \[
        x_1, x_2, \dots, x_{n - 1}, x_n\, (, x_1)
    \] be a Hamiltonian cycle in $G$ where $x_1, \dots, x_n$ are its vertices. This
    means that there are edges $\{x_i, x_{i + 1}\}$ for $1 \leq i < n$ and the edge
    $\{x_n, x_1\}$. Now, suppose that $S = \{x_{i_1}, \dots, x_{i_k}\}$, where the
    indices $i_j$ are in ascending order. Then for some index $1 \leq j < k$ note
    that the path \[
        x_{i_j + 1}, x_{i_j + 2}, \dots, x_{i_{j + 1} - 1}
    \] is contained in $G - S$ (we have simply sampled from the original Hamiltonian
    cycle; note that none of the edges in this path have been removed). In addition,
    the path \[
        x_{i_k + 1}, \dots, x_n, x_1, \dots, x_{i_1 - 1}
    \] is also in $G - S$. Thus, we have found at most $k$ paths (some of these paths
    may be empty, if two indices $i_j$ are consecutive) in $G - S$, such that every
    vertex from $G - S$ is part of one of these paths. This proves that there can be
    at most $k$ components in $G - S$.


    \problem Prove that the complement of a $d$ regular graph of order $2d + 2$ for
    $d \geq 1$ is Hamiltonian.

    \solution Suppose that $G$ is a $d$ regular graph on $n = 2d + 2$ vertices. Thus,
    each vertex is connected to $d$ other vertices. This means that in the complement
    graph $G'$, that vertex is connected to precisely the other $(n - 1) - d = d + 1$
    vertices. The sum of degrees of any two vertices in $G'$ is thus $2(d + 1) = n$.
    Thus, $G'$ is Hamiltonian by Ore's Theorem.


    \problem Show that if $G$ is a simple graph with minimum vertex degree $k$, then
    there exists a path of length $k$. Moreover, if $k \geq 2$, then there exists a
    cycle of length at least $k + 1$.

    \solution Let $G$ have $n$ vertices; clearly, $n \geq k + 1$ since each vertex
    has at least $k$ neighbours. Now, pick an arbitrary vertex and call it $x_1$.
    This has at least $k$ neighbours; pick one of them and call it $x_2$. Now, $x_2$
    has at least $k$ neighbours; at least $k - 2$ of them which are neither $x_1$ nor
    $x_2$. Pick one of them and call it $x_3$. In this way, at each stage where we
    have a path $x_1, \dots, x_j$ of length $j - 1$, where $1 < j \leq k$, the last
    vertex $x_j$ has at least $k$ neighbours, out of which at most $j - 1$ of them
    ($x_1, \dots, x_{j - 1}$) of them may have been exhausted. This leaves at least
    $k - (j - 1) \geq 1$ new vertices to choose from; pick one and call it $x_{j +
    1}$, thus growing the path. We can keep doing thus until we have obtained a path
    $x_1, \dots, x_k, x_{k + 1}$ of length $k$, as desired. \\

    Now, suppose that $k \geq 2$. Obtain the path $x_1, \dots, x_{k + 1}$ as before,
    and continue applying our algorithm. Note that our algorithm must terminate,
    since there are finitely many vertices in $G$. There are precisely two ways in
    which this can happen. \\

    \textbf{Case I}: We have run out of vertices, i.e.\ we have found a path $x_1,
    \dots, x_n$. Then, $x_1$ has at least $k$ neighbours; call (some of) them $x_{i_1}, \dots,
    x_{i_k}$ where the indices $i_j$ are in ascending order, i.e.\ $2 \leq i_1 < i_2
    < \dots < i_k$. This makes it clear that $i_k \geq k + 1$, hence we have found a
    cycle $x_1, x_2, \dots, x_{i_k}\,(, x_1)$ of length $i_k \geq k + 1$. \\

    \textbf{Case II}: We have run out of neighbours, i.e.\ we have found a path
    $x_1, \dots, x_m$, $m \geq k + 1$, where $x_m$ has no neighbours apart from
    $x_1, \dots, x_{m - 1}$. Like before, call (some of) these neighbours $x_{i_1}, \dots,
    x_{i_k}$ where the indices are in descending order, i.e.\ $m - 1 \geq i_1 > i_2 >
    \dots > i_k$. This makes it clear that $i_k \leq m - k$, hence we have found a
    cycle $x_{i_k}, x_{i_k + 1}, \dots, x_m\, (, x_{i_k})$ of length $m - i_k + 1
    \geq k + 1$.
    \\~\\
    
    \begin{lemma}
        Every tree contains at least one leaf.
    \end{lemma}
    \begin{proof}
        If the degree of every vertex in a tree on $n$ vertices is at least $2$, then
        the previous exercise guarantees the existence of a cycle of length at least
        $3$, violating the acylicity of the tree.
    \end{proof}
    

    \begin{lemma}
        A tree on $n$ vertices has exactly $n - 1$ edges.
    \end{lemma}
    \begin{proof}
        We use induction on $n$. This is clearly true for $n = 1, 2$. Suppose that
        this holds for all trees with at most $n$ vertices, and let $T$ be a tree on
        $n + 1$ vertices. Then, choose a leaf $x$ of $T$, and remove it to obtain
        $T'$. Note that $T'$ is still a tree, since we have removed only one edge.
        $T'$ is also still connected since any path between two vertices $u, v \in
        V(T)$, $u, v \neq x$ is also a valid path in $T'$; no such path could have
        included $x$ since there is only one edge coming out of $x$. Thus, $T'$ has
        $n - 2$ edges, hence $T$ must have $n - 1$ edges.
    \end{proof}

    \begin{lemma}
        Every connected graph has a spanning tree.
    \end{lemma}
    \begin{proof}
        Let $G$ be a connected graph on $n$ vertices. Then, $G$ can have finitely
        many cycles; there are finitely many choices for the $k$ vertices in a
        $k$-cycle, and there are finitely many cycle lengths (from $3$ to $n$). Now,
        suppose that $G$ contains a cycle $x_1, \dots, x_k\,(, x_1)$. Then, we can
        remove the edge $\{x_1, x_k\}$ and see that the new graph $G'$ is still
        connected. In this way, we can refine our graph, removing cycles at each
        step; this process must terminate since the number of cycles drops at each
        step. At the end, we are left with an acyclic, connected graph on all $n$
        vertices, i.e.\ a spanning tree.
    \end{proof}
    
    \problem Let $G$ be a graph with $n$ vertices, $m$ edges, and $k$ components.
    Show that \[
        n - k \leq m \leq \frac{1}{2}(n - k)(n - k + 1).
    \] 

    \solution Let the components of $G$ have vertex sets $V_1, \dots, V_k$, and let
    each component $G[V_i]$ contain $n_i$ vertices, $m_i$ edges. First, we claim that
    $n_i - 1 \leq m_i$; this is because each connected component with $n_i$ vertices
    contains a spanning tree with $n_i - 1$ edges, hence the component has at least
    $n_i - 1$ edges. Summing this over all $k$ components gives \[
        m = \sum_{i = 1}^k m_i \geq \sum_{i = 1}^k n_i - 1 = n - k.
    \] Note that the first equality is justified since there are no edges between
    components.

    Now, we claim that $m_i \leq \binom{n_i}{2} = n_i(n_i - 1) / 2$; this is because
    we can have one edge between any two vertices in a connected component, and no
    more. Thus, \[
        m = \sum_{i = 1}^k m_i \leq \frac{1}{2}\sum_{i = 1}^k n_i^2 - n_i.
    \] We examine the expression \[
        x^2 - x + (N - x)^2 - (N - x) = 2x^2 - 2Nx + N^2 - N
    \] where $N \geq 3$ is an integer. It is clear that this quadratic represents a
    parabola open upwards, with its vertex at $2N / 4 = N / 2$. Thus, this expression
    is decreasing on $[1, N/ 2]$ and increasing on $[N / 2, N - 1]$ (this can also be
    verified by differentiation), which means that it attains its maximum (on the
    interval $[1, N - 1]$) at either $x = 1, N - 1$. Indeed, both substitutions give
    \[
        x^2 - x + (N - x)^2 - (N - x) \leq (N - 1)^2 - (N - 1).
    \] 

    Returning to our original sum, apply the above repeatedly, peeling off terms one
    at a time to get \begin{align*}
        \sum_{i = 1}^k n_i^2 - n_i 
        &= n_1^2 - n_1 + \sum_{i = 2}^k n_i^2 - n_i \\
        &\leq (n_1 + n_2 - 1)^2 - (n_1 + n_2 - 1) + \sum_{i = 3}^k n_i^2 - n_i \\
        &\qquad \vdots \\
        &\leq (n_1 + n_2 + \dots + n_k - k + 1)^2 - (n_1 + n_2 + \dots + n_k - k +
        1) \\
        &= (n - k + 1)[(n - k + 1) - 1] \\
        &= (n - k + 1)(n - k).
    \end{align*}
    Thus, \[
        m \leq \frac{1}{2}\sum_{i = 1}^k n_i^2 - n_i \leq \frac{1}{2}(n - k)(n - k +
        1).
    \] 



    \problem Prove that $\girth(G) \leq 2\diam(G) + 1$. Moreover show that if
    $\girth(G) = 2\diam(G) + 1$, then the length of each cycle in that graph is
    $2\diam(G) + 1$.


    \solution If $G$ contains no cycle, we declare this result to be vacuously true.

    Otherwise, let $g = \girth(G)$, $d = \diam(G)$, and suppose that $x_1, \dots,
    x_k\, (, x_1)$ is a smallest cycle of length $g$. If $g \geq 2d + 2$, that means
    that the vertices $x_1$ and $x_{d + 2}$ have distance at least $d + 1$. To see
    this, note that the path $x_1, \dots, x_{d + 2}$ has length $d + 1$; if there is
    another, shorter path, it must be of the form $x_1, y_2, \dots, y_{s}, x_{d +
    2}$ where each $y_i$ may or may not be one of the $x_i$. By joining these paths
    (travel from $x_1$ to $x_{d + 2}$ along one, and back to $x_1$ along the other),
    we obtain at least one cycle of length strictly less than $(d + 1) + (d + 1) = 2d
    + 2$, i.e.\ less than the girth which is a contradiction. As a result, $x_1$ and
    $x_{d + 2}$ have distance at least $d + 1$, which is greater than the diameter, a
    contradiction. This forces $g \leq 2d + 1$.

    % Now, suppose that $g = 2d + 1$. Pick an arbitrary cycle $x_1, \dots, x_\ell\,(,
    % x_1)$ in $G$, and suppose that it has length $\ell \geq g + 1 = 2d + 2$. Like
    % before, note that the paths $x_1, x_2, \dots, x_{d + 2}$ and $x_1, x_\ell,
    % x_{\ell - 1}, \dots, x_{d + 2}$ have lengths at least $d + 1$. Since the diameter
    % is just $d$, there must exists another path between $x_1, y_2, \dots, y_{d'},
    % x_{d + 2}$ of length $d' \leq d$. By joining this path with $x_1, x_2, \dots,
    % x_{d + 2}$, we have found a cycle of length at most $(d + 1) + d' \leq (d + 1) +
    % d = 2d + 1 = g$. The minimality of $g$ means that $d = d'$, and there are no $y_i
    % = x_j$ (if there were, we could `pinch' the paths together at that point and get
    % a smaller cycle).


    \problem Prove that for any graph $G$, $\diam(G) \leq 2\cdot
    \operatorname{r}(G)$.

    \solution Denote the radius and diameter of $G$ by $r$ and $d$ respectively.
    This means that there exists a vertex $v$ with eccentricity $r$. Now, let $x, y$
    be two arbitrary vertices in $G$. Then, the distances $d(x, v) \leq r$ and $d(y,
    v) \leq r$. Thus, we have a sequence of vertices from $x$ to $v$ to $y$, which
    can be trimmed down (by removing redundant loops/repetitions) to a path of length
    at most $2r$. Thus, every vertex $x$ has eccentricity at most $2r$, which means
    that the diameter $d \leq 2r$.

    This argument works perfectly when $G$ has only one component: otherwise, we use
    the convention that the distance between disconnected vertices is infinite.


    \problem Let $S = \{1, 2, 3, 4, 5, 6\}$ and $G = (V, E)$ be a graph such that $V$
    is the collection of all 2 element subsets of $S$. Two vertices $u, v \in V$ are
    adjacent in $G$ if and only if $u$ and $v$ are disjoint sets.
    \begin{enumerate}
        \item Find the value of $|E|$.
        \item Prove that every pair of non-adjacent vertices in $G$ has exactly 3
        common neighbours.
        \item Find the diameter of $G$.
        \item Find the number of triangles in $G$.
    \end{enumerate}

    \solution Note that the number of vertices in $G$ is \[
        |V| = \binom{6}{2} = 15.
    \] 
    \begin{enumerate}
        \item Pick an arbitrary vertex $v = \{s_1, s_2\}$, and let the remaining elements
        of $S$ be $s_3, s_4, s_5, s_6$. Then, $v$ is adjacent to a vertex $\{s_i,
        s_j\}$ if and only if none of $s_i, s_j$ is $s_1, s_2$, i.e.\ $3 \leq i, j
        \leq 6$. This shows that $v$ is adjacent to $\binom{4}{2} = 6$ other
        vertices, hence every vertex in $G$ has degree $6$. Thus, \[
            |E| = \frac{1}{2}\sum_{v \in V} d(v) = \frac{1}{2}\cdot 15 \cdot 6 = 45.
        \] 

        \item Let $u = \{s_1, s_2\}$ and $v = \{s_2, s_3\}$ be non-adjacent vertices
        in $G$; clearly, any two such vertices must be of this form since they must
        share exactly one element. This leaves $s_4, s_5, s_6$ from $S$, which means
        that the common neighbours of $u, v$ are those comprised of these numbers.
        This gives $\binom{3}{2} = 3$ common neighbours.

        \item Let $u = \{s_1, s_2\}$ and $v = \{s_2, s_3\}$ be non-adjacent vertices.
        Then, $u \sim \{s_4, s_5\} \sim v$, hence the maximum distance between any
        two vertices is $2$. This shows that the diameter of $G$ is precisely $2$.

        \item Let $\{u, v\} \in E$, hence $u \sim v$ so we must have the form $u =
        \{s_1, s_2\}$ and $v = \{s_3, s_4\}$. If $x$ is a common neighbour of $u$ and
        $v$, then it is disjoint from both $u$ and $v$, which only leaves one choice
        $x = \{s_5, s_6\}$. Thus, every edge contributes exactly one triangle, and
        every triangle is comprised of three edges. The total number of triangles in
        $G$ is thus $|E| / 3 = 15$.
    \end{enumerate}


    \problem Prove or disprove that an $n$-cube $Q_n$ is bipartite for $n \geq 2$.

    \solution This is indeed true. Let the vertices of $Q_n$ be labelled by the
    binary strings $a_1a_2\dots a_n$ where each $a_i \in \{0, 1\}$; two vertices are
    adjacent if and only if their binary strings differ in exactly one place. Now,
    let $A$ be the set of vertices with an even number of 1's in their binary string,
    and $B$ be the set of vertices with an odd number of 1's in their binary string.
    We claim that $A$ is an independent set; this is clear because any two vertices
    in $A$ have an even number of 1's in their binary string, and hence differ in
    some even number of places (if a binary string has $k$ 1's, then changing one bit
    can only change the number of 1's by one, i.e.\ flip the parity of $k$).
    Similarly, $B$ is an independent set, since any two vertices in $B$ have an odd
    number of 1's in their binary string, and again differ in an even number of
    places. The sets $A$ and $B$ exhaust all vertices in $Q_n$, hence $Q_n$ is
    bipartite.


    \problem Prove that all 3-regular Hamiltonian graph on 10 vertices have girth
    less than 5.

    \solution Suppose note, i.e.\ let $G$ be a 3-regular Hamiltonian graph on 10
    vertices with girth at least $5$. Let $x_1, x_2, \dots, x_{10}\, (, x_1)$ be a
    Hamiltonian cycle: thus, each $x_i$ is adjacent to $x_{i - 1}$ and $x_{i + 1}$,
    along with another vertex (we are using cyclic indices, so $x_{11} \equiv x_1$,
    etc). For instance, take $x_1$: if $x_1 \sim x_i$ for any $i = 3, 4, 8, 9$, then
    we have found a cycle of length at most $4$ (namely, $x_1, x_2, x_3\, (, x_1)$,
    or $x_1, x_2, x_3, x_4\, (, x_1)$, or $x_1, x_{10}, x_9, x_8\, (, x_1)$, or $x_1,
    x_{10}, x_9\, (, x_1)$ respectively). Thus, we must have $x_1 \sim$ one of $x_5,
    x_6, x_7$. Similarly, each $x_i \sim$ one of $x_{i + 4}, x_{i + 5}, x_{i + 6}$. \\

    \textbf{Case I} $x_1 \sim x_5$: Then, $x_6 \sim$ one of $x_{10}, x_1, x_2$;
    discard $x_1$ since every vertex has degree $3$, discard $x_{10}$ since $x_1,
    x_5, x_6, x_{10}\, (, x_1)$ is a 4-cycle, discard $x_2$ since $x_1, x_5, x_6,
    x_2\, (, x_1)$ is a 4-cycle. \\
    
    \textbf{Case II} $x_1 \sim x_7$: Again, $x_6 \sim$ one of $x_{10}, x_1, x_2$;
    discard $x_1$ since every vertex has degree $3$, discard $x_{10}$ since $x_1,
    x_7, x_6, x_{10}\, (, x_1)$ is a 4-cycle, discard $x_2$ since $x_1, x_7, x_6,
    x_2\, (, x_1)$ is a 4-cycle. \\

    \textbf{Case III} $x_1 \sim x_6$: Then, $x_7 \sim$ one of $x_1, x_2, x_3$;
    discard $x_1$ since every vertex has degree $3$, discard $x_2$ since $x_1, x_6,
    x_7, x_2\, (, x_1)$ is a 4-cycle. This forces $x_7 \sim x_3$. Now, $x_2 \sim$ one
    of $x_6, x_7, x_8$; we have already used $x_6, x_7$ so $x_2 \sim x_8$. However,
    $x_2, x_8, x_7, x_3\, (, x_2)$ is a 4-cycle. \\

    In all cases, we have arrived at a contradiction. Thus, the girth of such a graph
    must be less than 5.



    \problem Find the number of spanning trees of $K_{2, n}$.

    \solution Let $A = \{a_1, a_2\}$ be one part, and let $B = \{b_1, b_2, \dots,
    b_n\}$ be the other; $A$ and $B$ are independent sets. Consider a spanning tree
    of this graph. It is clear by inspection that $K_{2, 1}$ has one spanning tree
    (the path $a_1, b_1, a_2$). Otherwise when $n \geq 2$, we know that $a_1$ and
    $a_2$ must be connected by some path, and the only possibility is $a_1 \sim b_i
    \sim a_2$ for some $i$. No other $b_j$ can have degree 2: that would imply that
    $b_j \sim a_1, a_2$, giving a cycle $a_1, b_i, a_2, b_j\, (, a_1)$. Thus, each of
    the remaining $n - 1$ vertices must be connected to exactly one of $a_1$ or $a_2$
    (we cannot leave them isolated of course). The result is indeed a spanning tree:
    every $b_i$ is connected to at least one of $a_1, a_2$, and these are connected
    to each other. Additionally, there are no cycles since only $a_1, a_2, b_i$ can
    have degree greater than $1$, and they do not form a triangle. Thus, the number
    of spanning trees of $K_{2, n}$ is precisely $n \cdot 2^{n - 1}$.


    \problem Show that if $\ell$ is the label of a vertex with degree $n$, then
    $\ell$ occurs $n - 1$ times in the Pr\"ufer code.

    \solution Suppose that $\ell$ has neighbours $\ell_1, \dots, \ell_n$ in the
    original tree. Our algorithm terminates when there is only one edge remaining.
    This immediately shows that if $n = 1$, then $\ell$ never appears in the Pr\"ufer
    code; for it to appear, it would have to be the parent of a (soon to be deleted)
    leaf, but if its only neighbour $\ell_1$ becomes a leaf, then $\{\ell, \ell_1\}$
    is the only edge remaining in the tree (there can be no other connected vertices,
    both $\ell, \ell_1$ are leaves), hence the algorithm terminates. Otherwise,
    assume $n \geq 2$. Suppose that $\ell$ survives in the algorithm; thus, the edge
    $\{\ell, \ell_1\}$ (without loss of generality) survives, so the $n - 1$
    vertices $\ell_2, \dots, \ell_n$ become deleted. Their parent $\ell$ thus appears
    in the code $n - 1$ times before termination. Otherwise, suppose that $\ell$ does
    not survive. During its deletion, $\ell$ must have been a leaf, i.e.\ $n - 1$ of
    its neighbours were removed first, again contributing $\ell$ to the Pr\"ufer code
    $n - 1$ times. The final neighbour cannot also be deleted, since that would leave
    $\ell$ isolated, hence $\ell$ does not appear any more times. In either case,
    $\ell$ appears precisely $n - 1$ times in the Pr\"ufer code. \\


    The converse is clearly true, i.e.\ if a label $\ell$ appears $n - 1$ times in a
    Pr\"ufer code, then that label $\ell$ has degree $n$ in the corresponding tree.
    Note that its appearance $n - 1$ times indicates that it had at least $n - 1$
    neighbours connected to it before being deleted during the construction of the
    code. At each of these stages, $\ell$ was still connected to the tree (otherwise,
    it couldn't be listed as the parent of the lowest labelled leaf) and hence had
    one more neighbour, raising its original degree to at least $n$. Finally, its
    degree cannot have been more than $n$; after the last ($n - 1$ th) appearance of
    the label $\ell$ in the code, there are two possible configurations of the tree
    at that stage of the algorithm. Either $\ell$ is a leaf, and hence contributes
    that single extra neighbour; or it was connected to two or more labels, say
    $\ell_1, \dots, \ell_k$. Note that the algorithm must terminate with at most one
    of the edges $\{\ell, \ell_i\}$ surviving at the end, which means that the
    remaining neighbours must be deleted at some point. Indeed, if none of the edges
    $\{\ell, \ell_i\}$ survive, that means that $k - 1 \geq 1$ of the $\ell_i$ must
    have be deleted first (i.e.\ they must become leaves of $\ell$), before the
    deletion of the final $\ell$ or $\ell_i$; this contradicts the fact that $\ell$
    never appears again. Again, if $\{\ell, \ell_i\}$ survives (thus terminating the
    algorithm), the remaining $k - 1\geq 1$ of the $\ell_i$ must be deleted, with the
    same contradiction. This proves that $\ell$ appears $n - 1$ times in the Pr\"ufer
    code if and only if it has degree $n$ in the corresponding tree. \\

    If each label $i$ appears $d_i - 1$ times in the Pr\"ufer code, the total number
    of elements in the code is \[
        \sum_{i = 1}^n d_i - 1 = 2|E| - n = 2(n - 1) - n = n - 2,
    \] as expected.


    \problem Show that the number of different labelled trees on $n$ vertices such
    that the vertex $i$ has degree $d_i$ is \[
        \frac{(n - 2)!}{\prod (d_i - 1)!}.
    \] 

    \solution Note that for a tree, \[
        \sum_{i = 1}^n d_i = 2|E| = 2n - 2, \qquad
        \sum_{i = 1}^n d_i  - 1 = 2|E| - n = n - 2.
    \] Thus, the given number is a multinomial coefficient, \[
        \binom{n - 2}{d_1 - 1, d_2 - 1, \dots, d_n - 1}.
    \] 

    Recall that a vertex of degree $d_i$ appears $d_i - 1$ times in a Pr\"fer code
    and vice versa. Also, every possible Pr\"ufer code (sequence of $n - 2$ labels)
    corresponds to exactly one tree. Thus, we need only count the number of sequences
    of length $n - 2$ in which the label $i$ appears $d_i - 1$ times. This is
    precisely the given multinomial coefficient. To see this, note that we need to
    choose $d_i - 1$ out of $n - 2$ places for each label $i$ to appear in the
    Pr\"ufer code. Another way is to note that the $n - 2$ labels can be arranged in
    $(n - 2)!$ ways; each label $i$ appears $d_i - 1$ times and hence we have
    over-counted by a factor of $(d_i - 1)!$ for each (note that the $(d_i - 1)!$
    ways in which the identical labels $i$ can be permuted are redundant). This means
    that we divide $(n - 2)!$ by each $(d_i - 1)!$, giving the desired formula.

    

\end{document}
