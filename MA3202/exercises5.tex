\documentclass[10pt]{article}

\usepackage[T1]{fontenc}
\usepackage{geometry}
\usepackage{amsmath, amssymb, amsthm}

\geometry{a4paper, margin=1in}

\renewcommand{\labelenumi}{(\roman{enumi})}

\newcounter{prob}
\newcommand{\problem}{\stepcounter{prob}\paragraph{Exercise \arabic{prob}}}
\newcommand{\solution}{\paragraph{Solution}}

\newcommand{\C}{\mathbb{C}}
\newcommand{\R}{\mathbb{R}}
\newcommand{\Q}{\mathbb{Q}}
\newcommand{\Z}{\mathbb{Z}}
\newcommand{\N}{\mathbb{N}}

\title{Exercise Sheet V}
\author{Satvik Saha}
\date{}

\begin{document}
    \noindent\textbf{IISER Kolkata} \hfill \textbf{Excercise Sheet V}
    \vspace{3pt}
    \hrule
    \vspace{3pt}
    \begin{center}
    \LARGE{\textbf{MA3202: Algebra II}}
    \end{center}
    \vspace{3pt}
    \hrule
    \vspace{3pt}
    Satvik Saha, \texttt{19MS154} \hfill \today
    \vspace{20pt}

    
    \problem Find all units in $\Z[i]$.

    \solution Let $u, v \in Z[i]$ be units with $uv = 1$. Further let $u = a + ib$,
    $v = c + id$. Then, taking square norms gives \[
        (a^2 + b^2)(c^2 + d^2) = 1,
    \] from which we must have $a^2 + b^2 = c^2 + d^2 = 1$. This forces $u = \pm 1,
    \pm i$, which are the only units in $\Z[i]$.


    \problem Prove that $\Z[i]$ is a Euclidean domain.

    \solution It can be shown that the map $d\colon \Z[i]\setminus\{0\} \to \Z_{n
    \geq 0}$, $a + ib \mapsto a^2 + b^2$ is an algorithm map. Let $x = a + ib$, $y =
    c + id$, whence $d(xy) = d(x)d(y)$ immediately gives $d(x) \leq d(xy)$. Next,
    note that \[
        \frac{x}{y} = \frac{a + ib}{c + id} = \frac{(a + ib)(c - id)}{c^2 + d^2} =
        \frac{(ac + bd) + i(bc - ad)}{c^2 + d^2}.
    \] Thus, we can set $x = (r + is)y$, where $r, s$ are rational. Furthermore, we
    can choose integers $m, n$ such that \[
        |r - m| \leq \frac{1}{2}, \qquad |s - n| \leq \frac{1}{2}.
    \] Set $q = m + in \in \Z[i]$, \[
        w = \frac{x}{y} - q = (r - m) + i(s - n), \qquad wy = x - qy \in \Z[i].
    \] Thus, we claim that $x = qy + wy$, with $d(wy) < d(y)$. Indeed, \[
        d(w) = (r - m)^2 + (s - n)^2 \leq \frac{1}{2^2} + \frac{1}{2^2} =
        \frac{1}{2},
    \] so \[
        d(wy) = d(w)d(y) \leq \frac{1}{2}d(y) < d(y).
    \] 


    \problem Show that if $F$ is a field, then $F[X]$ is a Euclidean domain.

    \solution The map which sends a polynomial to its degree is a Euclidean domain.
    Note that $\deg(pq) = \deg(p) + \deg(q)$, so $\deg(p) \leq \deg(pq)$.


    \problem Is $5$ a prime element in $\Z[\sqrt{2}]$?

    \solution Suppose that $5\mid xy$, with $x = a + b\sqrt{2}$, $y = c + d\sqrt{2}$.
    Note that $xy = (ac + 2bd) + (ad + bc)\sqrt{2}$. Define $d(x) = |a^2 - 2b^2|$,
    whence \[
        d(xy) = |(ac + 2bd)^2 - 2(ad + bc)^2| = a^2c^2 + 4b^2d^2 - 2a^2d^2 - 2b^2c^2,
    \] and $d(x)d(y) = a^2c^2 - 2a^2d^2 - 2b^2c^2 + 4b^2d^2$. In other words,
    $d(x)d(y) = d(xy)$. Thus, we must have $5^2 \mid |(a^2 - 2b^2)(c^2 - 2d^2)|$.

    Without loss of generality, we have $5\mid a^2 - 2b^2$. Now, $a, b \equiv 0, \pm
    1, \pm 2 \pmod{5}$, hence $a^2 \equiv 0, \pm 1 \pmod{5}$. As a result, $a^2 -
    2b^2 \equiv 0\pmod{5}$ only when $a, b \equiv 0 \pmod{5}$. Thus, $5\mid a +
    b\sqrt{2} = x$. This proves that $5$ is prime in $\Z[\sqrt{2}]$.


    \problem Show that $\Z[\sqrt{2}]$ is a Euclidean domain.

    \solution We have already shown that the map defined by $d(a + b\sqrt{2}) = |a^2
    - 2b^2|$ is multiplicative, hence $d(x) \leq d(xy)$. Now, let $x = a +
    b\sqrt{2}$, $y = c + d\sqrt{2}$. Then note that \[
        \frac{x}{y} = \frac{a + b\sqrt{2}}{c + d\sqrt{2}} = \frac{(a + b\sqrt{2})(c -
        d\sqrt{2})}{c^2 - 2d^2} = \frac{(ac - 2bd) + (bc - ad)\sqrt{2}}{c^2 - 2d^2}.
    \] Thus, we can set $x = (r + s\sqrt{2})y$, where $r, s$ are rational.
    Furthermore, we can choose integers $m, n$ such that \[
        |r - m| \leq \frac{1}{2}, \qquad |s - n| \leq \frac{1}{2}.
    \] Set $q = m + n\sqrt{2} \in \Z[\sqrt{2}]$, \[
        w = \frac{x}{y} - q = (r - m) + (s - n)\sqrt{2}, \qquad wy = x - qy \in
        \Z[\sqrt{2}].
    \] Thus, we claim that $x = qy + wy$, with $d(wy) < d(y)$. Indeed, \[
        d(w) = |(r - m)^2 - 2(s - n)^2| \leq |r - m|^2 + 2|s - n|^2 \leq
        \frac{1}{2^2} + 2\cdot \frac{1}{2^2} = \frac{3}{4},
    \] so \[
        d(wy) = d(w)d(y) \leq \frac{3}{4} d(y) < d(y).
    \] 


    \problem Is a subring of a Euclidean domain a Euclidean domain? What about
    principal ideal domains?

    \solution Note that $\Z[X]$ is not a principal ideal domain, since $(X, 2)$ is an
    ideal but not principal. Thus, $\Z[X]$ cannot be a Euclidean domain. However,
    $\R[X]$ is a Euclidean domain since $\R$ is a field, and $\Z[X] \subset \R[X]$ is
    a subring.

    The same shows that a subring of a principal ideal domain need not be a principal
    ideal domain.


    \problem Let $R$ be a principal ideal domain, and $P$ be a prime ideal of $R$.
    Show that $R/P$ is a principal ideal domain. Is this true for Euclidean domains?

    \solution Note that this is trivial if $P = \{0\}$. Otherwise, since $R$ is a
    principal ideal domain and $P$ is a prime ideal, $P$ is a maximal ideal. Thus,
    $R/P$ is a field, hence a simple ring with only two ideals $(0)$ and $(1)$, hence
    a principal ideal domain.

    Note that this $R/P$ is also a Euclidean domain, since it is a field with the
    trivial algorithm map $x \mapsto 1$.


    \problem Let $R$ be a Euclidean domain, and $x \in R$. Then, show that $x$ is a
    unit if and only if $d(x) = d(1)$.

    \solution First suppose that $x$ is a unit, with $xy = 1$. Then, $d(x)\leq d(xy)
    = d(1)$. On the other hand, $d(1) \leq d(1x) = d(x)$, hence $d(x) = d(1)$.

    Next, suppose that $d(x) = d(1)$. Write $1 = qx + r$; if $r = 0$, then $x$ is a
    unit. Otherwise, we demand $d(r) < d(1)$, but this is impossible since $d(1) \leq
    d(1r) = d(r)$.


    \problem Let $R$ be a factorisation domain in which any two elements have a gcd.
    Show that $R$ is a unique factorisation domain.

    \solution We need only show that every irreducible element is a prime. Let $p \in
    R$ be irreducible, and $p\mid ab$. Also suppose that $p \nmid a$; we claim that
    $p \mid b$. Note that $\gcd(ab, pb) = \gcd(a, p) b$, but $\gcd(a, p) = 1$. Thus,
    $p \mid ab, pb$ shows that $p \mid \gcd(ab, pb)$ so $p \mid b$.


    \problem Let $R$ be a principal ideal domain, $S$ be an integral domain, and
    $\varphi\colon R \to S$ be a surjective ring homomorphism which is not one-one.
    Show that $S$ is a field.

    \solution Set $P = \ker{\varphi}$. Then, we have $R/P \cong S$ which is an
    integral domain, hence $P$ is a prime ideal. Now $P = 0$ would imply that
    $\varphi$ is an isomorphism. Thus, $P \neq 0$, hence $P$ is maximal in $R$, hence
    $R/P \cong S$ is a field.


\end{document}
