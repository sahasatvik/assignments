\documentclass[11pt]{article}

\usepackage[T1]{fontenc}
\usepackage{geometry}
\usepackage{amsmath, amssymb, amsthm}
\usepackage[scr]{rsfso}
\usepackage{bm}
\usepackage[%
    hidealllines=true,%
    innerbottommargin=15,%
    nobreak=true,%
]{mdframed}
\usepackage{xcolor}
\usepackage{graphicx}
\usepackage{fancyhdr}
\usepackage{hyperref}

\geometry{a4paper, margin=1in, headheight=14pt}

\pagestyle{fancy}
\fancyhf{}
\renewcommand\headrulewidth{0.4pt}
\fancyhead[L]{\scshape MA3202: Algebra II}
\fancyhead[R]{\scshape \leftmark}
\rfoot{\footnotesize\it Updated on \today}
\cfoot{\thepage}

\newcommand{\C}{\mathbb{C}}
\newcommand{\R}{\mathbb{R}}
\newcommand{\Q}{\mathbb{Q}}
\newcommand{\Z}{\mathbb{Z}}
\newcommand{\N}{\mathbb{N}}

\newmdtheoremenv[%
    backgroundcolor=blue!10!white,%
]{theorem}{Theorem}[section]
\newmdtheoremenv[%
    backgroundcolor=violet!10!white,%
]{corollary}{Corollary}[theorem]
\newmdtheoremenv[%
    backgroundcolor=teal!10!white,%
]{lemma}[theorem]{Lemma}

\theoremstyle{definition}
\newmdtheoremenv[%
    backgroundcolor=green!10!white,%
]{definition}{Definition}[section]
\newmdtheoremenv[%
    backgroundcolor=red!10!white,%
]{exercise}{Exercise}[section]

\theoremstyle{remark}
\newtheorem*{remark}{Remark}
\newtheorem*{example}{Example}
\newtheorem*{solution}{Solution}

\surroundwithmdframed[%
    linecolor=black!20!white,%
    hidealllines=false,%
    innertopmargin=5,%
    innerbottommargin=10,%
    skipabove=0,%
    skipbelow=0,%
]{example}

\numberwithin{equation}{section}

\title{
    \Large\textsc{MA3202} \\
    \Huge \textbf{Algebra II} \\
    \vspace{5pt}
    \Large{Spring 2022}
}
\author{
    \large Satvik Saha
    \\\textsc{\small 19MS154}
}
\date{\normalsize
    \textit{Indian Institute of Science Education and Research, Kolkata, \\
    Mohanpur, West Bengal, 741246, India.} \\
}

\begin{document}
    \maketitle

    \tableofcontents

    \section{Rings}
    
    \subsection{Basic definitions}
    \begin{definition}
        A ring is a set $R$ equipped with two binary operations, namely addition and
        multiplication, such that 
        \begin{enumerate}
            \itemsep0em
            \item $(R, +)$ is an abelian group.
            \begin{enumerate}
                \itemsep0em
                \item $a + b \in R$ for all $a, b \in R$.
                \item $(a + b) + c = a + (b + c)$ for all $a, b, c \in R$.
                \item $a + b = b + a$ for all $a, b \in R$.
                \item There exists $0 \in R$ such that $a + 0 = a$ for all $a \in R$.
                \item For each $a \in R$, there exists $-a \in R$ such that $a + (-a)
                = 0$.
            \end{enumerate}
            \item $(R, \cdot)$ is a semi-group.
            \begin{enumerate}
                \itemsep0em
                \item $a\cdot b \in R$ for all $a, b \in R$.
                \item $(a\cdot b)\cdot c = a\cdot(b\cdot c)$ for all $a, b, c \in R$.
            \end{enumerate}
            \item Multiplication distributes over addition.
            \begin{enumerate}
                \itemsep0em
                \item $a\cdot (b + c) = (a\cdot b) + (a\cdot c)$ for all $a, b, c \in R$.
                \item $(b + c)\cdot a = (b\cdot a) + (c\cdot a)$ for all $a, b, c \in R$.
            \end{enumerate}
        \end{enumerate}
        \begin{remark}
            The following properties follow immediately,
            \begin{enumerate}
                \itemsep0em
                \item $0\cdot a = 0$ for all $a \in R$.
                \item $(-a)\cdot b = -(a\cdot b) = a\cdot(-b)$ for all $a, b \in R$.
                \item $(na)\cdot b = n(a\cdot b) = a\cdot (nb)$ for all $a, b \in R$.
            \end{enumerate}
        \end{remark}
    \end{definition}

    \begin{example}
        The integers $\Z$ form a ring, under the usual addition and multiplication.
    \end{example}
    \begin{example}
        All fields, for instance the rational numbers $\Q$ or the real numbers $\R$,
        are rings.
    \end{example}
    \begin{example}
        The integers modulo $n$, namely $\Z/n\Z$, form a ring.
    \end{example}
    \begin{example}
        If $R$ is a ring, then the algebra of polynomials $R[X]$ with coefficients
        from $R$ form a ring.
    \end{example}
    \begin{example}
        If $R$ is a ring, then the $n \times n$ matrices $M_n(R)$ with entries from
        $R$ form a ring.
    \end{example}

    \begin{definition}
        If $R$ is a ring and $(R, \cdot)$ is a monoid i.e.\ has an identity, then
        this identity is unique and called the unity of the ring $R$. Such a ring $R$
        is called a unit ring.
    \end{definition}
    \begin{example}
        The even integers $2\Z$ form a ring, but do not contain the identity.
    \end{example}

    \begin{definition}
        If $R$ is a ring and $(R, \cdot)$ is commutative, then $R$ is called a
        commutative ring.
    \end{definition}

    \begin{definition}
        Let $R$ be a unit ring. An element $a \in R$ is called a unit if there exists
        $b \in R$ such that $a\cdot b = 1 = b\cdot a$. This $b \in R$ is unique, and
        denoted by $a^{-1}$.
    \end{definition}
    \begin{example}
        The units in $\Z$ are $\{1, -1\}$.
    \end{example}

    \begin{definition}
        Let $R$ be a ring, and let $S \subseteq R$. We say $S$ is a subring of $R$ if
        the structure $(S, +, \cdot)$ is a ring, with addition and multiplication
        inherited from $R$.
    \end{definition}
    \begin{example}
        The rings $n\Z$ for $n \in \N$ are all subrings of $\Z$.
    \end{example}

\end{document}
