\documentclass[11pt]{article}

\usepackage[T1]{fontenc}
\usepackage{geometry}
\usepackage{amsmath, amssymb, amsthm}
\usepackage[scr]{rsfso}
\usepackage{bm}
\usepackage[%
    hidealllines=true,%
    innerbottommargin=15,%
    nobreak=true,%
]{mdframed}
\usepackage{xcolor}
\usepackage{graphicx}
\usepackage{fancyhdr}
\usepackage{hyperref}

\geometry{a4paper, margin=1in, headheight=14pt}

\pagestyle{fancy}
\fancyhf{}
\renewcommand\headrulewidth{0.4pt}
\fancyhead[L]{\scshape MA3202: Algebra II}
\fancyhead[R]{\scshape \leftmark}
\rfoot{\footnotesize\it Updated on \today}
\cfoot{\thepage}

\newcommand{\C}{\mathbb{C}}
\newcommand{\R}{\mathbb{R}}
\newcommand{\Q}{\mathbb{Q}}
\newcommand{\Z}{\mathbb{Z}}
\newcommand{\N}{\mathbb{N}}

\DeclareMathOperator{\ch}{ch}
\DeclareMathOperator{\im}{im}

\newmdtheoremenv[%
    backgroundcolor=blue!10!white,%
]{theorem}{Theorem}[section]
\newmdtheoremenv[%
    backgroundcolor=violet!10!white,%
]{corollary}{Corollary}[theorem]
\newmdtheoremenv[%
    backgroundcolor=teal!10!white,%
]{lemma}[theorem]{Lemma}

\theoremstyle{definition}
\newmdtheoremenv[%
    backgroundcolor=green!10!white,%
]{definition}{Definition}[section]
\newmdtheoremenv[%
    backgroundcolor=red!10!white,%
]{exercise}{Exercise}[section]

\theoremstyle{remark}
\newtheorem*{remark}{Remark}
\newtheorem*{example}{Example}
\newtheorem*{solution}{Solution}

\surroundwithmdframed[%
    linecolor=black!20!white,%
    hidealllines=false,%
    innertopmargin=5,%
    innerbottommargin=10,%
    skipabove=0,%
    skipbelow=0,%
]{example}

\numberwithin{equation}{section}

\title{
    \Large\textsc{MA3202} \\
    \Huge \textbf{Algebra II} \\
    \vspace{5pt}
    \Large{Spring 2022}
}
\author{
    \large Satvik Saha
    \\\textsc{\small 19MS154}
}
\date{\normalsize
    \textit{Indian Institute of Science Education and Research, Kolkata, \\
    Mohanpur, West Bengal, 741246, India.} \\
}

\begin{document}
    \maketitle

    \tableofcontents

    \section{Rings}
    
    \subsection{Basic definitions}
    \begin{definition}
        A ring is a set $R$ equipped with two binary operations, namely addition and
        multiplication, such that 
        \begin{enumerate}
            \itemsep0em
            \item $(R, +)$ is an abelian group.
            \begin{enumerate}
                \itemsep0em
                \item $a + b \in R$ for all $a, b \in R$.
                \item $(a + b) + c = a + (b + c)$ for all $a, b, c \in R$.
                \item $a + b = b + a$ for all $a, b \in R$.
                \item There exists $0 \in R$ such that $a + 0 = a$ for all $a \in R$.
                \item For each $a \in R$, there exists $-a \in R$ such that $a + (-a)
                = 0$.
            \end{enumerate}
            \item $(R, \cdot)$ is a semi-group.
            \begin{enumerate}
                \itemsep0em
                \item $a\cdot b \in R$ for all $a, b \in R$.
                \item $(a\cdot b)\cdot c = a\cdot(b\cdot c)$ for all $a, b, c \in R$.
            \end{enumerate}
            \item Multiplication distributes over addition.
            \begin{enumerate}
                \itemsep0em
                \item $a\cdot (b + c) = (a\cdot b) + (a\cdot c)$ for all $a, b, c \in R$.
                \item $(b + c)\cdot a = (b\cdot a) + (c\cdot a)$ for all $a, b, c \in R$.
            \end{enumerate}
        \end{enumerate}
        \begin{remark}
            The following properties follow immediately,
            \begin{enumerate}
                \itemsep0em
                \item $0\cdot a = 0$ for all $a \in R$.
                \item $(-a)\cdot b = -(a\cdot b) = a\cdot(-b)$ for all $a, b \in R$.
                \item $(na)\cdot b = n(a\cdot b) = a\cdot (nb)$ for all $a, b \in R$.
            \end{enumerate}
        \end{remark}
    \end{definition}

    \begin{example}
        The integers $\Z$ form a ring, under the usual addition and multiplication.
    \end{example}
    \begin{example}
        All fields, for instance the rational numbers $\Q$ or the real numbers $\R$,
        are rings.
    \end{example}
    \begin{example}
        The integers modulo $n$, namely $\Z/n\Z$, form a ring.
    \end{example}
    \begin{example}
        If $R$ is a ring, then the algebra of polynomials $R[X]$ with coefficients
        from $R$ form a ring.
    \end{example}
    \begin{example}
        If $R$ is a ring, then the $n \times n$ matrices $M_n(R)$ with entries from
        $R$ form a ring.
    \end{example}

    \begin{definition}
        If $R$ is a ring and $(R, \cdot)$ is a monoid i.e.\ has an identity, then
        this identity is unique and called the unity of the ring $R$. Such a ring $R$
        is called a unit ring. Note that we typically demand that this identity is
        distinct from the zero element.
    \end{definition}
    \begin{example}
        The even integers $2\Z$ form a ring, but do not contain the identity.
    \end{example}
    \begin{example}
        The trivial ring $\{0\}$ is typically not considered to be a unit ring, since
        must serve as the additive identity as well as the multiplicative identity.
    \end{example}

    \begin{definition}
        If $R$ is a ring and $(R, \cdot)$ is commutative, then $R$ is called a
        commutative ring.
    \end{definition}

    \begin{definition}
        Let $R$ be a unit ring. An element $a \in R$ is called a unit if there exists
        $b \in R$ such that $a\cdot b = 1 = b\cdot a$. This $b \in R$ is unique, and
        denoted by $a^{-1}$.
    \end{definition}
    \begin{example}
        The units in $\Z$ are $\{1, -1\}$.
    \end{example}


    \subsection{Subrings}

    \begin{definition}
        Let $R$ be a ring, and let $S \subseteq R$. We say $S$ is a subring of $R$ if
        the structure $(S, +, \cdot)$ is a ring, with addition and multiplication
        inherited from $R$.
    \end{definition}
    \begin{example}
        The rings $n\Z$ for $n \in \N$ are all subrings of $\Z$.
    \end{example}
    \begin{example}
        Consider the rings $2\Z \subset \Z$. Here, $\Z$ is a unit ring but $2\Z$ is
        not.
    \end{example}
    \begin{example}
        Consider the rings $4\Z/12\Z \subset 2\Z/12\Z$. Here, $2\Z/12\Z$ is not a
        unit ring but $4\Z/12\Z$ is.
    \end{example}

    \begin{lemma}
        Let $S$ be a subring of $R$. Since $(R, +)$ is an abelian group, $(S, +)$ is
        a normal subgroup of $(R, +)$. Thus, we can make sense of the quotient group
        $(R/S, +)$.
    \end{lemma}

    \begin{lemma}
        Let $S$ be a subring of $R$. Then, the quotient $(R/S, +, \cdot)$ is a ring
        with multiplication $(a + S)\cdot(b + S) = ab + S$ if and only if $ab - xy
        \in S$ for all $a, b, x, y \in R$ such that the cosets $a + S = x + S$, $b +
        S = y + S$.
    \end{lemma}
    \begin{example}
        Consider the ring $\Z$ and the subring $n\Z$. Then, the quotient $\Z/n\Z$ is
        indeed a ring.
    \end{example}
    \begin{example}
        Consider the ring $\Q$ and the subring $\Z$. It cal be shown that $\Q/\Z$ is
        not a ring under the `natural' multiplication.
    \end{example}

    
    \subsection{Ideals}

    \begin{definition}
        Let $R$ be a ring and let $I$ be a subset of $R$. We say that $I$ is an ideal
        of $R$ if $(I, +)$ is a subgroup of $(R, +)$, and $rx, xr \in I$ for all $r
        \in R$, $x \in I$.
    \end{definition}

    \begin{example}
        Consider the ring $\Z$, and the subring $n\Z$. This is an ideal of $\Z$,
        since $m(n\Z) \subseteq n\Z$. Indeed, every ideal of $\Z$ is of the form
        $n\Z$. This will follow from Euclid's Division Lemma.
    \end{example}
    \begin{example}
        The subsets $\{0\}$ and $R$ of any ring $R$ are trivial ideals.
    \end{example}

    \begin{lemma}
        Let $R$ be a ring, and $I$ be an ideal of $R$. Then, the quotient $R/I$ is a
        ring.
    \end{lemma}
    \begin{proof}
        Note that whenever $a - x \in I$, $b - y \in I$, we demand that $ab - xy \in
        I$. This can be rewritten as $(a - x)b + x(b - y) \in I$, which is clearly
        true by the properties of the ideal $I$.
    \end{proof}

    \begin{definition}
        An ideal $I \subset R$ is called finitely generated if there exist $x_1, x_2,
        \dots, x_n \in I$ such that every element of $I$ can be written as a finite
        linear combination \[
            x = r_1x_1 + \dots + r_nx_n,
        \] where $r_i \in R$. We denote $I = (x_1, x_2, \cdots, x_n)$.
    \end{definition}

    \begin{definition}
        An ideal generated by a single element is called a principal ideal.
    \end{definition}
    \begin{example}
        Every ideal of $\Z$ is a principal ideal.
    \end{example}

    \begin{lemma}
        Let $R$ be a unit ring, and $I \subseteq R$ be an ideal. Then, $I = R$ if and
        only if $I$ contains a unit.
    \end{lemma}

    \begin{definition}
        The sum of two ideals $I, J \subset R$ is defined \[
            I + J = \{x + y: x \in I, y \in J\}.
        \] Their product is defined \[
            IJ = \{\sum_{i = 1}^n x_iy_i: x_i \in I, y_i \in J\}.
        \]
    \end{definition}

    \begin{lemma}
        The sum and product of two ideals of a ring are also ideals of that ring.
    \end{lemma}

    \begin{lemma}
        Let $I, J \subset R$ be ideals in the commutative ring $R$. Then, $IJ \subset
        I \cap J$.
    \end{lemma}
    \begin{example}
        Note that for $2\Z, 2\Z \in \Z$, $(2\Z)(2\Z) = 4\Z$ but $2\Z \cap 2\Z = 2\Z$.
        A related example is $R = 2\Z$, $I = 4\Z$, $J = 6\Z$.
    \end{example}
    
    \subsection{Integral domains}
    
    \begin{definition}
        Let $R$ be a ring and $a, b \in R$, $a, b \neq 0$. If $ab = 0$, we call $a$ a
        left zero divisor and $b$ a right zero divisor.
    \end{definition}
    \begin{example}
        Consider $2, 3 \in \Z/6\Z$; then $2\cdot 3 = 6 \equiv 0$.
    \end{example}

    \begin{definition}
        A commutative ring $R$ is called an integral domain if it has no zero
        divisors.
    \end{definition}
    \begin{example}
        When $p$ is prime, the rings $\Z/p\Z$ are integral domains. Note that this
        set is a group under both $+$ and $\cdot$.
    \end{example}

    \begin{lemma}
        Every field is an integral domain.
    \end{lemma}

    \begin{theorem}
        Every finite integral domain is a field.
    \end{theorem}
    \begin{proof}
        Let $R = \{x_1, \dots, x_n\}$ be a finite integral domain. We first show that
        $R$ contains an identity $1$. Pick $x \neq 0$, and note that $x x_1, x x_2,
        \dots, x x_n$ must all be distinct: otherwise $x x_i = x x_j$ would force
        $x(x_i - x_j) = 0$. This forces $x = x x_k$ for some $x_k \neq 0$. Now, we
        claim that $x_k$ is our identity. Indeed, given any $y \neq 0$, we write $y =
        xx_l$ for some $x_l \neq 0$, hence $yx_k = x x_l x_k = x_l(x x_k) = x_lx =
        y$.

        Next, we show that every non-zero $x \in R$ has an inverse. Indeed, $1 = x_k$
        must be one of the $x x_1, \dots, x x_n$, hence $1 = x x_m$ for some non-zero
        $x_m$. This means that $x_m = x^{-1}$.
    \end{proof}

    \begin{definition}
        Let $R$ be a ring. The characteristic of $R$ is the smallest positive integer
        $n$ such that $nx = 0$ for all $x \in R$. If no such number $n$ exists, we
        say that the characteristic of $R$ is zero. We denote the characteristic of
        $R$ by $\ch(R)$.
    \end{definition}
    \begin{example}
        We have $\ch(\Z) = 0$, $\ch(\Z/n\Z) = n$.
    \end{example}

    \begin{lemma}
        Let $R$ be a unit ring. Then, $\ch(R)$ is the smallest positive integer $n$
        such that $n\cdot 1 = 0$; if no such $n$ exists, then $\ch(R)$ is zero.
    \end{lemma}

    \begin{theorem}
        Let $R$ be an integral domain. Then, $\ch(R)$ is either zero or a prime.
    \end{theorem}
    \begin{proof}
        Let $R$ be an integral domain such that $\ch(R) = n \neq 0$. If $n$ is not a
        prime, write $n = n_1n_2$ for $n_1, n_1 < n$. Then for any non-zero $x \in
        R$, write $0 = n(x^2) = (n_1x)(n_2x)$. This forces one of $n_1x, n_2x = 0$;
        say $n_1x = 0$. Now for any $y \in R$, we have $x(n_1 y) = (n_1 x)y = 0$.
        Since $x \neq 0$, we have $n_1y = 0$ for all $y \in R$, contradicting the
        minimality of $n$.
    \end{proof}


    \subsection{Simple rings}

    \begin{definition}
        A simple ring is one which has no non-trivial ideals. We typically demand
        that multiplication in $R$ is non-trivial.
    \end{definition}

    \begin{lemma}
        Every field is a simple ring.
    \end{lemma}
    \begin{proof}
        If $R$ is a field and $I\subset R$ is an ideal with non-zero $a \in I$, then
        $a^{-1} \in R$ hence $a^{-1}a = 1 \in I$. This immediately forces $I = R$.
    \end{proof}
    \begin{lemma}
        If $R$ is a commutative, simple, unit ring, then $R$ is a field.
    \end{lemma}
    \begin{proof}
        Pick non-zero $a \in R$, and set $I = (a)$. Since $R$ is simple, $I = R$,
        hence $1 \in I = (a)$. In other words, $1 = ab$ for some $b \in R$.
    \end{proof}


    \subsection{Homomorphisms and isomorphisms}
    
    \begin{definition}
        Let $R, S$ be rings, and let $\varphi\colon R \to S$. We say that $\varphi$
        is a ring homomorphism if 
        \begin{enumerate}
            \itemsep0em
            \item $\varphi(x + y) = \varphi(x) + \varphi(y)$ for all $x, y \in R$.
            \item $\varphi(xy) = \varphi(x)\varphi(y)$ for all $x, y \in R$.
            \item $\varphi(1_R) = 1_S$.
        \end{enumerate}
        We only insist on 3 if both $R$ and $S$ are unit rings.
        \begin{remark}
            The following properties follow immediately.
            \begin{enumerate}
                \itemsep0em
                \item $\varphi(0_R) = 0_S$.
                \item $\varphi(-x) = -\varphi(x)$ for all $x \in R$.
                \item $\varphi(nx) = n\varphi(x)$ for all $x \in R$, $n \in \Z$.
                \item $\varphi(x - y) = \varphi(x) - \varphi(y)$ for all $x, y \in R$.
            \end{enumerate}
        \end{remark}
    \end{definition}

    \begin{example}
        The map $\varphi\colon \Z \to \Z/n\Z$, $k \mapsto k \bmod{n}$ is a
        homomorphism.
    \end{example}

    \begin{definition}
        A bijective homomorphism between two rings is called an isomorphism. If an
        isomorphism exists between two rings, we say that they are isomorphic.
    \end{definition}

    \begin{example}
        The map $\varphi\colon \Z \to n\Z$, $k \mapsto nk$ is an isomorphism.
    \end{example}
    \begin{example}
        The map $\varphi\colon \C \to \C$, $z \mapsto \bar{z}$ is an isomorphism.
    \end{example}

    \begin{lemma}
        The only isomorphism $\Z \to \Z$ is the identity map.
    \end{lemma}
    \begin{theorem}
        The only isomorphism $\Q \to \Q$ is the identity map.
    \end{theorem}
    \begin{proof}
        Let $\varphi\colon \Q \to \Q$ be an isomorphism. We must have $\varphi(1) =
        1$, which immediately gives $\varphi(n) = n$ for all $n \in \Z$. Now for any
        rational $p / q \in \Q$, note that $1 = \varphi(q \cdot 1 / q) = q \cdot
        \varphi(1 / q)$, forcing $\varphi(1 / q) = 1 / q$. Thus, $\varphi(p / q) = p
        / q$, completing the proof.
    \end{proof}
    
    \begin{theorem}
        The only isomorphism $\R \to \R$ is the identity map.
    \end{theorem}
    \begin{proof}
        Let $\varphi\colon \R \to \R$ be an isomorphism. We must have $\varphi(q)
        = q$ for all $q \in \Q$.

        First we show that $\varphi$ is strictly increasing. Note that when $x > 0$,
        $\varphi(x) = \varphi(\sqrt{x})^2 > 0$. Thus when $x > y$, $\varphi(x - y) >
        0$, hence $\varphi(x) > \varphi(y)$.

        Now let $x \in \R$; if $\varphi(x) \neq x$, we must have one of $\varphi(x) >
        x$ or $\varphi(x) < x$. Assume the former, and find $q \in \Q$ such that
        $\varphi(x) > q > x$. Now, $q > x$ gives $q = \varphi(q) > \varphi(x)$, a
        contradiction.  An analogous argument gives a contradiction when $\varphi(x)
        < x$, completing the proof.
    \end{proof}
    
    \begin{theorem}
        The only homomorphism $\R \to \R$ is the identity map.
    \end{theorem}
    \begin{proof}
        If $\varphi\colon \R \to \R$ is a homomorphism, it is easy to check that
        $\varphi^{-1}(0)$ is an ideal. Since $\R$ is simple, this must be $\{0\}$ or
        $\R$; the latter can be ruled out since $\varphi(1) = 1$. In other words,
        $\varphi^{-1} = \{0\}$ so $\varphi$ is injective. Following the previous
        proof, $\varphi$ must be an isomorphism, hence the identity map.
    \end{proof}

    \begin{theorem}
        The only isomorphisms $\C \to \C$ which sends $\R \to \R$ are the maps $z
        \mapsto z$ and $z \mapsto \bar{z}$.
    \end{theorem}
    \begin{proof}
        The previous theorem guarantees that any such isomorphism $\varphi\colon \C
        \to \C$ is completely determined by $\varphi(i)$. Now, $-1 = \varphi(-1) =
        \varphi(i)^2$, forcing $\varphi(i) = \pm i$.
    \end{proof}

    \begin{lemma}
        The kernel of a ring homomorphism $\varphi\colon R \to S$ is an ideal of $R$.
        Its image is a subring of $S$.
    \end{lemma}
    \begin{proof}
        If $x \in \ker\varphi$, then $\varphi(x) = 0$, hence for any $r \in R$ we
        have $\varphi(rx) = \varphi(r) \varphi(x) = 0$. Thus, $rx \in
        \varphi^{-1}(0)$. Also, recall that $\varphi^{-1}(0)$ is an additive subgroup
        of $R$.
    \end{proof}

    \begin{theorem}[First isomorphism theorem]
        Let $\varphi\colon R \to S$ be a surjective ring homomorphism. Then, \[
            R / \ker{\varphi} \cong \im{\varphi}.
        \] 
    \end{theorem}
    \begin{proof}
        Denote $I = \ker{\varphi}$, so the elements of $R/I$ are the cosets $x + I$
        for $x \in R$. This gives us the natural map \[
            \phi\colon R/I \to S, \qquad x + I \mapsto \varphi(x).
        \] It can be shown that this map is well defined: if $x + I = y + I$, then $x
        - y \in I$ so $\varphi(x - y) = 0$, or $\varphi(x) = \varphi(y)$. Now,
        $\phi((x + I) + (y + I)) = \varphi(x + y) = \varphi(x) + \varphi(y) = \phi(x
        + I) + \phi(y + I)$, and $\phi((x + I)(y + I)) = \varphi(xy) =
        \varphi(x)\varphi(y) = \phi(x + I)\phi(y + I)$. Additionally, if $R$ and $S$
        are both unit rings, then $\phi(1_R + I) = \varphi(1_R) = 1_S$. Thus, $\phi$
        is a homomorphism. It is obvious that $\phi$ is surjective; also observe that
        $\phi^{-1}(0) = 0 + I$, hence $\phi$ is also injective. This proves that
        $\phi$ is an isomorphism, as desired.
    \end{proof}

    \begin{theorem}
        Let $I, J \subset R$ be ideals. Then, \[
            (I + J)/J \cong I/(I \cap J).
        \] 
    \end{theorem}
    \begin{proof}
        The map $\phi\colon I \to (I + J)/J$, $x \mapsto x + J$ can be shown to be a
        surjective homomorphism. It's kernel consists of the elements in $I$ that get
        mapped to $0 + J$, so $\ker{\phi} = I \cap J$. Applying the first isomorphism
        theorem gives the desired result.
    \end{proof}

    \begin{lemma}
        Let $I \subset R$ be an ideal, and let $\varphi\colon R \to S$ be a
        surjective ring homomorphism, then $\varphi(I)$ is an ideal in $S$.
    \end{lemma}

    \begin{theorem}[Correspondence theorem]
        Let $I \subset R$ be an ideal. Then there exists a one-to-one correspondence
        between the ideals of $R$ containing $I$ with the ideals of $R/I$.
    \end{theorem}
    \begin{proof}
        Use the surjective ring homomorphism $\phi\colon R \to R/I$, $x \mapsto x +
        I$, which maps ideals in $R$ to ideals in $R/I$. Furthermore, given ideals
        $J, J' \subset R$ such that $\varphi(J) = \varphi(J')$, note that $x \in J$
        implies $\varphi(x) \in \varphi(J) = \varphi(J')$ so $x \in J'$; this shows
        that $J = J'$, hence our map is injective. Finally, given an ideal $K$ in
        $R/I$, its pre-image under our map is the ideal $L = \{x \in R: x + I \in
        K\}.
    \end{proof}

\end{document}
