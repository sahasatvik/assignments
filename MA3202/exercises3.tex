\documentclass[10pt]{article}

\usepackage[T1]{fontenc}
\usepackage{geometry}
\usepackage{amsmath, amssymb, amsthm}

\geometry{a4paper, margin=1in}

\renewcommand{\labelenumi}{(\roman{enumi})}

\newcounter{prob}
\newcommand{\problem}{\stepcounter{prob}\paragraph{Exercise \arabic{prob}}}
\newcommand{\solution}{\paragraph{Solution}}

\newcommand{\C}{\mathbb{C}}
\newcommand{\R}{\mathbb{R}}
\newcommand{\Q}{\mathbb{Q}}
\newcommand{\Z}{\mathbb{Z}}
\newcommand{\N}{\mathbb{N}}

\title{Exercise Sheet III}
\author{Satvik Saha}
\date{}

\begin{document}
    \noindent\textbf{IISER Kolkata} \hfill \textbf{Excercise Sheet III}
    \vspace{3pt}
    \hrule
    \vspace{3pt}
    \begin{center}
    \LARGE{\textbf{MA3202: Algebra II}}
    \end{center}
    \vspace{3pt}
    \hrule
    \vspace{3pt}
    Satvik Saha, \texttt{19MS154} \hfill \today
    \vspace{20pt}

    \problem Let $R$ be a unit ring. Consider the map $f\colon \Z \to R$ defined by
    $f(n) = n\cdot 1$. \begin{enumerate}
        \itemsep0em
        \item Show that $f$ is a homomorphism.
        \item Show that $f$ is not always injective.
        \item When is $f$ an embedding (injective homomorphism)?
    \end{enumerate}

    \solution \mbox{}
    \begin{enumerate}
        \item It is clear that $f(1) = 1$; for positive $m, n$, we have $f(m) + f(n) =
        m\cdot 1 + n\cdot 1 = (m + n)\cdot 1 = f(m + n)$, and $f(m)f(n) = (m\cdot 1)
        \cdot (n \cdot 1) = (mn)\cdot 1 = f(mn)$ by distributing and counting. It is
        easy to extend this to negative $m, n$ by observing that $f(-n) = -f(n)$.

        \item Consider $f\colon \Z \to \Z/2\Z$, $n \mapsto n\cdot 1$. Then, $f(2) =
        2\cdot 1 = 1 + 1 \equiv 0 = f(0)$.

        \item It is easy to check that $f$ is an embedding precisely when $R$ has
        characteristic zero. In other words, the elements $n\cdot 1$ are all
        distinct.
    \end{enumerate}


    \problem Let $R$ be a ring and $I \subset R$ be an ideal. Show that if $R$ has
    identity, so does $R/I$. What about the converse?

    \solution If $1 \in \R$, then $1 + I$ is the multiplicative identity in $R/I$.

    The converse is not true; note that $2\Z / 6\Z = \{0, 2, 4\}$ has $4$ as the
    identity, since $4\cdot 2 = 8 \equiv 2$ and $4\cdot 4 = 16 \equiv 4$. However,
    $2\Z$ does not have an identity.


    \problem Let $R$ be a ring and $I \subset R$ be an ideal. Show that if $R$ is
    commutative, so is $R/I$. What about the converse?

    \solution If $xy = yx$ in $R$, then $(x + I)(y + I) = xy + I = (y + I)(x + I)$ in
    $R/I$.

    The converse is not true; note that $R = \Z \times M_2((\R)$ is a non-commutative
    ring, but $I = \{0\} \times M_2(\R)$ is an ideal, with $R/I \cong \Z$ being
    commutative.


    \problem Show that the characteristic of a simple ring is either $0$ or $p$,
    where $p$ is a prime number.

    \solution Suppose that $R$ has characteristic $n = ab > 0$, where $1 < a, b < n$.
    Then there exists an element $x_0 \in R$ such that $nx_0 = 0$, and $mx_0 \neq 0$ for
    all $0 < m < n$. Now, consider the elements $aR = \{ax : x \in R\}$. This is
    clearly an ideal of $\R$; since $ax_0 \neq 0$, $aR \neq \{0\}$. Since $R$ is
    simple, $aR = R$. Similarly, $bR = R$. Thus, $\{0\} = nR = (ab)R = a(bR) = aR
    \neq \{0\}$, a contradiction.


    \problem Show that $\R{X}/(X^2 + 1)$ and $\C$ are isomorphic as rings.

    \solution By Euclid's Division Lemma, every polynomial in $\R[X]$ can be uniquely
    expressed as \[
        p(x) = u(x)(x^2 + 1) + bx + a.
    \] Thus, every polynomial in $\R[X]$ belongs to exactly one equivalence class
    $[ax + b]$. Define the map \[
        \varphi\colon \R[X] \to \C, \qquad u(x)(x^2 + 1) + bx + a \mapsto a + ib.
    \] It is clear that $\varphi(1) = $, $\varphi(p + q) = \varphi(p) + \varphi(q)$.
    Now consider \[
        p(x) = u(x)(x^2 + 1) + bx + a, \qquad
        q(x) = v(x)(x^2 + 1) + dx + c.
    \] Then, \[
        p(x)q(x) = (u(x)v(x) + u(x)(dx + c) + v(x)(bx + a))(x^2 + 1) + bdx^2 + (ad +
        bc)x + ac.
    \] Tweaking this gives \[
        p(x)q(x) = (u(x)v(x) + u(x)(dx + c) + v(x)(bx + a) + bd)(x^2 + 1) + (ad +
        bc)x + ac - bd.
    \] In other words, \[
        \varphi(pq) = (ac - bd) + i(ad + bc) = \varphi(p)\varphi(q).
    \] Thus, $\varphi$ is a homomorphism, and it is easy to see that it is
    surjective. Its kernel consists of all the polynomials mapped to $0$, i.e.\ all
    polynomials of the form $u(x)(x^2 + 1)$. This is precisely $(X^2 + 1)$. Thus, the
    First Isomorphism Theorem guarantees that $R[X]/(X^2 + 1) \cong \C$.


    \problem Let $R$ be a ring. What are all ideals of $R \times R$?

    \solution All ideals of $R$ are of the form $I \times J$, where $I, J \subset R$
    are ideals of $R$.


    \problem Show that $\Z$ and $\Z\times \Z$ are not isomorphic.

    \solution Note that $\Z$ is an integral domain, but $\Z\times \Z$ is not since
    $(1, 0)$ is a zero divisor.


    \problem Are $\Z$ and $\Q$ isomorphic as rings?

    \solution No; if $\varphi\colon \Q \to \Z$ were an isomorphism, then set $a =
    \varphi(1 / 2)$. We now demand $2a = \varphi(1) = 1$, which is impossible.


    \problem Let $R = C([1, 3])$ and $I(2) = \{f \in R : f(2) = 0\}$. Prove that
    $I(2)$ is an ideal of $R$.

    \solution It is clear that $I(2)$ is a subring of $R$; it contains the zero
    function, and if $f, g \in I(2)$, then $f(2) = g(2) = 0$ hence $(f + g)(2) = 0$,
    $(-f)(2) = 0$. Furthermore if $h \in R$, then $(hf)(2) = h(2)f(2) = 0$ so $hf \in
    I(2)$.

    
    \problem Show that the ring $\Z / 14\Z$ is isomorphic to the product of $\Z/7\Z$
    and $\Z/2\Z$.

    \solution Note that $(7\Z)(2\Z) = 14\Z$. Furthermore, $7\Z + 2\Z = \Z$, since $7$
    and $2$ are co-prime. The isomorphism $\Z/(7\Z\cdot 2\Z) \cong \Z/7\Z \times
    \Z/2\Z$ now follows from the Chinese Remainder Theorem.


\end{document}
