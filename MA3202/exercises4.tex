\documentclass[10pt]{article}

\usepackage[T1]{fontenc}
\usepackage{geometry}
\usepackage{amsmath, amssymb, amsthm}

\geometry{a4paper, margin=1in}

\renewcommand{\labelenumi}{(\roman{enumi})}

\newcounter{prob}
\newcommand{\problem}{\stepcounter{prob}\paragraph{Exercise \arabic{prob}}}
\newcommand{\solution}{\paragraph{Solution}}

\newcommand{\C}{\mathbb{C}}
\newcommand{\R}{\mathbb{R}}
\newcommand{\Q}{\mathbb{Q}}
\newcommand{\Z}{\mathbb{Z}}
\newcommand{\N}{\mathbb{N}}

\title{Exercise Sheet IV}
\author{Satvik Saha}
\date{}

\begin{document}
    \noindent\textbf{IISER Kolkata} \hfill \textbf{Excercise Sheet IV}
    \vspace{3pt}
    \hrule
    \vspace{3pt}
    \begin{center}
    \LARGE{\textbf{MA3202: Algebra II}}
    \end{center}
    \vspace{3pt}
    \hrule
    \vspace{3pt}
    Satvik Saha, \texttt{19MS154} \hfill \today
    \vspace{20pt}

    
    \problem What is the field of fractions of a finite integral domain?

    \solution Every finite integral domain is a field, hence its field of fractions
    is itself.


    \problem Check whether $4\Z$ and $6\Z$ are maximal ideals of $2\Z$.

    \solution Suppose that $4\Z \subset I \subseteq 2\Z$, and $n \in I$, $n \notin
    4\Z$. Then $n = 4k + 2$ for some $k \in Z$. Now, $4k \in 4\Z \subset I$, hence
    $(4k + 2) - 4k = 2 \in I$. This immediately gives all $2k \in I$, hence $I =
    2\Z$. Thus, $4\Z$ is maximal in $2\Z$.

    Suppose that $6\Z \subset I \subseteq 2\Z$, and $n \in I$, $n \notin 6\Z$. Then,
    $n = 6k \pm 2$ for some $k \in \Z$. A similar argument as before shows that $2
    \in I$, $I = 2\Z$, thus showing that $6\Z$ is maximal in $2\Z$.


    \problem Let $R$ be a ring. What are the prime and maximal ideals of $R\times R$?

    \solution Every prime ideal of $R\times R$ is of the form $P\times R$ or $R\times
    P$, for some prime ideal $P \subset R$. Similarly, every maximal ideal of
    $R\times R$ is of the form $M\times R$ or $R\times M$, for some maximal ideal $M
    \subset R$.

    To see this, note that given an ideal $I\times J \subseteq R \times R$, the
    product of the quotient maps $R \to R/I$ and $R \to R/J$, given by the surjective
    homomorphism \[
        \varphi\colon R\times R \to R/I \times R/J, \qquad (x, y) \mapsto (x + I, y +
        J),
    \] has $I\times J$ as its kernel. Thus, $(R\times R)/(I \times J) \cong R/I
    \times R/J$. Now note that if $a \in R/I$, $b \in R/J$, then $(0, 0) = (a,
    0)\cdot (0, b)$, hence we have zero divisors in the quotient unless one of $R/I$,
    $R/J$ is $\{0\}$, i.e.\ one of $I, J = R$. Thus, the quotient is now isomorphic
    to either $R/I$ or $R/J$, which must now be an integral domain/field forcing
    either $I, J$ to be prime/maximal.


    \problem Let $\varphi\colon R \to S$ be a surjective ring homomorphism. What can
    you say about the following?
    \begin{enumerate}
        \item If $P \subset R$ is a prime ideal, then $\varphi(P)$ is a prime ideal
        of $S$.
        \item If $M \subset R$ is a maximal ideal, then $\varphi(M)$ is a maximal
        ideal of $S$.
    \end{enumerate}

    \solution \mbox{}
    \begin{enumerate}
        \item False. Note that $(X) \subset \R[X]$ is a prime ideal, and the
        evaluation map $p(x) \mapsto p(1)$ is a surjective ring homomorphism $\R[X]
        \to \R$. However, $\varphi((X)) = \R$ is not a proper ideal of $\R$, hence
        not prime.

        \item False, via the same counterexample. The ideal $(X) \subset \R[X]$ is
        indeed maximal.
    \end{enumerate}


    \problem Let $R$ be a commutative unit ring. Prove that $R[X]/(X)$ is isomorphic
    to $R$. Furthermore, the ideal $(X) = X R[X]$ is prime if and only if $R$ is an
    integral domain, and the ideal $(X) = X R[X]$ is maximal if and only if $R$ is a
    field.

    \solution The map \[
        \varphi\colon R[X] \to R, \qquad p(x) \mapsto p(0)
    \] is a surjective homomorphism, with kernel $(X)$. This immediately gives $R[X]
    / (X) \cong R$. The next two criteria follow directly from the fact that $(X)$ is
    a proper ideal in a commutative unit ring, and $R[X] / (X) \cong R$.


    \problem Find all prime and maximal ideals of $\Z / 16\Z$.

    \solution Note that the ideals of $\Z/16\Z$ are $\Z/16\Z$, $2\Z/16\Z$, $4\Z/16\Z$,
    $8\Z/16\Z$, $\{0\}$. It is clear that $2\Z/16\Z$ is maximal, hence prime.
    $4\Z/16\Z$, $8\Z/16\Z$, $\{0\}$ are not prime since $4 = 2\cdot 2$, $8 = 4\cdot
    2$, $0 = 4\cdot 4$.


    \problem Let $R$ be a commutative unit ring such that $x^2 = x$ for all $x \in
    R$. Prove that every prime ideal of $R$ is a maximal ideal.

    \solution Let $P \subset R$ be a prime ideal, and note that $R/P$ is an integral
    domain. Now, for any $x + P \in R/P$, we have \[
        (x + P)^2 = x^2 + 2xP + P^2 = x + P,
    \] hence every $y \in R/P$ satisfies $y^2 = y$, $y^2 - y = y(y - 1) = 0$. Since
    $R/P$ is an integral domain, either $y = 0$ or $y = 1$. In other words, $R/P$ is
    the field of two elements, hence $P$ is a maximal ideal in $R$.


    \problem Let $\varphi\colon R \to S$ be a ring homomorphism. What can you say
    about the following?
    \begin{enumerate}
        \item If $P\subset S$ is a prime ideal, then $\varphi^{-1}(P)$ is a prime
        ideal of $R$.
        \item If $M\subset S$ is a maximal ideal, then $\varphi^{-1}(M)$ is a maximal
        ideal of $R$.
    \end{enumerate}
    \solution \mbox{}
    \begin{enumerate}
        \item False. Note that $2\Z\times \{0\} \subset 2\Z\times 2\Z$ is a prime
        ideal, and the map $n \mapsto (n, 0)$ is a ring homomorphism $2\Z \to
        2\Z\times 2\Z$. However, $\varphi^{-1}(2\Z \times \{0\}) = 2\Z$ is not a
        proper ideal of $2\Z$, hence not prime.

        \item False. Note that $\{0\} \subset \Q$ is a maximal ideal, and the
        inclusion map $\Z \to \Q$ is a ring homomorphism. However, the pre-image of
        $\{0\}$, namely $\{0\}\subset \Z$ is not maximal.
    \end{enumerate}


    \problem Let $Z[X]$ be the ring of polynomials with integer coefficients.
    Consider the ideal \[
        J = \{f(x) \in \Z[X]: f(2) = 0\}.
    \] \begin{enumerate}
        \item Is $J$ a prime ideal?
        \item Is $J$ a maximal ideal?
    \end{enumerate}

    \solution Note that $\Z[X]$ is a commutative unit ring, and the map $f(x) \to
    f(2)$ is a surjective ring homomorphism $\Z[X] \to \Z$, whose kernel is precisely
    $J$. Thus, $\Z[X]/J \cong \Z$, which is an integral domain but not a field. Thus,
    $J$ is a prime ideal but not maximal.


    \problem Show that an integral domain and its field of fractions have the same
    characteristic.

    \solution Note that if $Z$ is an integral domain with field of fractions $F$, $Z$
    is embedded in $F$. Thus if $Z$ has characteristic $0$, so does $F$. Otherwise,
    let $n, m$ be the characteristics of $Z, F$ respectively. We immediately have $m
    \geq n$. Now, pick $[a, b] \in F$, hence $n[a, b] = [na, b] = [0, b] = 0 \in F$,
    which shows that $m \leq n$. Together, $m = n$ as desired.

    \problem Let $P, Q$ be two prime ideals in a ring $R$. Assume that $P \cap Q$ is
    a prime ideal of $R$. Prove that either $P \subset Q$ or $Q \subset P$.

    \solution Suppose to the contrary that neither $P \subset Q$, nor $Q \subset P$.
    Then, we can pick $p \in P\setminus Q \subset P$, $q \in Q\setminus P \subset Q$,
    whence $pq \in P, Q$ from the properties of ideals. Thus, $pq \in P \cap Q$;
    since the latter is a prime ideal, we must have one of $p, q \in P \cap Q$, which
    is a contradiction.


    \problem Let $R = C([1, 3])$ and $I(2) = \{f \in R: f(2) = 0\}$. \begin{enumerate}
        \item Prove that $I(2)$ is a maximal ideal.
        \item Compute $R/I(2)$.
    \end{enumerate}

    \solution Note that the evaluation map $f \mapsto f(2)$ is a surjective ring
    homomorphism $C([1, 3]) \to \R$, whose kernel is precisely $I(2)$. Thus,
    $R/I(2)\cong \R$. Since $C([1, 3])$ is a commutative unit ring and $\R$ is a
    field, $I(2)$ is a maximal ideal. It is clear that given an equivalence class in
    $R/I(2)$, all functions within that equivalence class have the same evaluation at
    $2$. In other words, \[
        R/I(2) = \{\alpha + I(2): \alpha \in \R\}.
    \] 


\end{document}
