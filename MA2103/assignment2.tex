\documentclass[10pt]{article}

\usepackage[T1]{fontenc}
\usepackage{geometry}
\usepackage{amsmath, amssymb, amsthm}
\usepackage{bm}
\usepackage{cancel}
\usepackage{xcolor}

\title{Mathematical Methods II - Assignment II}
\author{Satvik Saha}
\date{}

\geometry{a4paper, margin=1in}
\setlength\parindent{0pt}
\renewcommand{\labelenumi}{(\roman{enumi})}
\renewcommand\CancelColor{\color{red}}
% \renewcommand\qedsymbol{$\blacksquare$}
\newcommand\ve[1]{\boldsymbol{#1}}
\newcommand\ppx[1]{\frac{\partial #1}{\partial x}}
\newcommand\ppt[1]{\frac{\partial #1}{\partial t}}
\newcommand\pp[3][]{\frac{\partial^{#1}{#2}}{\partial {#3}^{#1}}}
\newcommand\ddx[1]{\frac{d #1}{d x}}
\newcommand\ddt[1]{\frac{d #1}{d t}}
\newcommand\dd[3][]{\frac{d^{#1}{#2}}{d {#3}^{#1}}}
\newcommand\norm[1]{\left\lVert#1\right\rVert}
\newcommand\grad[1]{\ve{\nabla}#1}
\newcommand\divg[1]{\ve{\nabla}\cdot#1}
\newcommand\curl[1]{\ve{\nabla}\times#1}
\newcommand\lapl[1]{\nabla^2 #1}

\begin{document}
        \par\textbf{IISER Kolkata} \hfill \textbf{Assignment II}
        \vspace{3pt}
        \hrule
        \vspace{3pt}
        \begin{center}
                \LARGE{\textbf{MA 2102 : Mathematical Methods II}}
        \end{center}
        \vspace{3pt}
        \hrule
        \vspace{3pt}
        Satvik Saha, \texttt{19MS154}\hfill\today
        \vspace{20pt}
        \subsection*{Partial Differential Equations (M.L. Boas, Chapter 13)}
        \paragraph{Section 3. Problem 4.} At $t = 0$, two flat slabs each 5 cm thick, one at $0^\circ$ and one at $20^\circ$, are stacked together,
        and then the surfaces are kept at $0^\circ$. Find the temperature as a function of $x$ and $t$ for $t > 0$. \\

        \textit{Solution.} We have our initial temperature distribution in the interior
        \[
                u(x, 0) = u_0(x) = \begin{cases} 0, &\text{if } 0 < x < 5 \\ 20, &\text{if }5 \leq x < 10 \end{cases},
        \]
        and our boundary conditions specify that $u(0, t) = u(10, t) = 0$. With this, we must solve the heat flow equation
        \[
                \pp[2]{u}{x} \;=\; \frac{1}{\alpha^2}\ppt{u}.
        \]
        As usual, we perform a partial separation of variables $u(x, t) = F(x)T(t)$, upon which we obtain
        \[
                \frac{1}{F} \dd[2]{}{x}F = \frac{1}{\alpha^2 T} \ddt{T} = \text{constant} = -k^2.
        \]
        This leads to the eqautions
        \[
        \ddt{}T + ^2\alpha^2 T = 0, \quad\quad \dd[2]{}{x}F + k^2 F = 0.
        \]
        The first equation has the solution $T(t) = \exp(-k^2\alpha^2 t)$.
        Thus, we justify the choice of $-k^2$ as our constant with the observation that $T$ remains finite as $t \to \infty$.
        The second equation admits solutions of the form $A\cos{kx} + B\sin{kx}$. The cosine part is discarded using $u(0, 0) = 0$, and
        we must have $k = n\pi /10$ from $u(10, 0) = 0$. Superimposing thses solutions, we write
        \[
                u(x, t) \;=\; \sum_{n = 1}^\infty A_n \sin\frac{n\pi x}{10} \,e^{-(n\pi\alpha /10)^2 t}.
        \]
        The coefficients $A_n$ are simply given by
        \[
                A_n = \frac{2}{10}\int_0^{10} u_0(x)\sin\frac{n\pi x}{10} \:dx = \frac{1}{5}\int_5^{10} 20\sin\frac{n\pi x}{10} \:dx
                        = \frac{40}{n\pi} (\cos{n\pi /2} - \cos{n\pi}).
        \]
        Note that the cosine part in parenthesis is $1$ for odd $n$, $0$ for multiples of $4$, and $-2$ for the rest. Thus,
        \[
                u(x, t) \;=\; \sum_{n = 1}^\infty \frac{40}{n\pi} \left(\cos\frac{n\pi}{2} - \cos{n\pi}\right) \sin\frac{n\pi x}{10} \,e^{-(n\pi\alpha /10)^2 t}.
        \]

        \paragraph{Problem 6.} The ends of a bar are initially at $20^\circ$ and $150^\circ$; at $t = 0$, the $150^\circ$ end is changed to $50^\circ$.
        Find the time-dependent temperature distribution. \\

        \textit{Solution.} We have already seen that a solution of the heat equation over a domain $[0, \ell]$ where the initial conditions
        are given by $u_0(x)$ is
        \[
                u(x, t) = \sum_{n = 1}^\infty A_n \sin\frac{n\pi x}{\ell} e^{-(n\pi \alpha /\ell)^2 t}.
        \]
        The initial value $u_0(x)$ must be a steady state solution, i.e.\ $u_0''(x) = 0$. This forces a linear form,
        \[
                u_0(x) = 20 + 130\frac{x}{\ell}.
        \]
        Now, note that at $t \to \infty$, $u \to 0$. However, we want a steady state solution where the boundaries are at $20$ and $50$ respectively.
        Such a solution obeys $u_f''(x) = 0$, which is a linear form again.
        \[
                u_f(x) = 20 + 30\frac{x}{\ell}.
        \]
        Thus, our final solution is actually the superposition
        \[
                u(x, t) = u_f(x) + \sum_{n = 1}^\infty A_n \sin\frac{n\pi x}{\ell} e^{-(n\pi \alpha /\ell)^2 t}.
        \]
        The coefficients $A_n$ are evaluted at $t = 0$ to obtain
        \[
                A_n = \frac{2}{\ell} \int_0^\ell (u_0(x) - u_f(x))\sin\frac{n\pi x}{\ell}\:dx = -\frac{200}{n\pi}\cos{n\pi}.
        \]
        Thus, 
        \[
                u(x, t) = 20 + 30\frac{x}{\ell} + \sum_{n = 1}^\infty \frac{-200}{n\pi}\cos{n\pi}\sin\frac{n\pi x}{\ell} e^{-(n\pi \alpha /\ell)^2 t}.
        \]
        Note that the coefficients $A_n$ are identical to those in the indicated equation (3.16) because $u_0(x) - u_f(x) = 100x/\ell$, which 
        is the function used in that particular problem.

        \paragraph{Problem 7.} A bar of length $\ell$ with insulated sides has its ends also insulated from time $t = 0$ on.
        Initially the temperature is $u = x$, where $x$ is the distance from one end. Determine the temperature distribution inside the bar at time $t$. \\

        \textit{Solution.} We perform separation of variables $u(x, t) = F(x)T(t)$ as usual, obtaining $T(t) = \exp(-k^2\alpha^2 t)$ and
        $F(x) = A\cos{kx} + B\sin{kx}$. Now, since the boundaries are insulated, we demand $u'(0, t) = u'(\ell, t) = 0$, which means that
        $-A\sin{kx} + B\cos{kx} = 0$ at $x = 0, \ell$. This forces $B = 0$ and $k = n\pi /\ell$, so we take superpositions
        and obtain the general solution
        \[
                u(x, t) = \frac{A_0}{2} + \sum_{n = 1}^\infty A_n \cos\frac{n\pi x}{\ell} e^{-(n\pi \alpha /\ell)^2 t}.
        \]
        We calculate the coefficients $A_n$ by setting $t = 0$, obtaining a cosine series from which we have
        \[
                A_n = \frac{2}{\ell} \int_0^\ell u_0(x)\cos\frac{n\pi x}{\ell}\:dx.
        \]
        For $n = 0$, we see that $A_0 = \ell$. Otherwise,
        \[
                A_n = \frac{2}{\ell}\int_0^\ell x\cos\frac{n\pi x}{\ell}\:dx = -\frac{2\ell}{n^2\pi^2}(1 - \cos{n\pi}).
        \]
        Putting everything together,
        \[
                u(x, t) = \frac{\ell}{2} \,-\, 
                        \sum_{n = 1}^\infty \frac{2\ell}{n^2\pi^2}(1 - \cos{n\pi})\cos\frac{n\pi x}{\ell} e^{-(n\pi \alpha /\ell)^2 t}.
        \]

        \paragraph{Problem 9.} A bar of length $2$ is initially at $0^\circ$. For $t > 0$, the $x = 0$ end is insulated and the $x = 2$ end is 
        held at $100^\circ$. Find the time dependent temperature distribution. \\

        \textit{Solution.} We perform the standard separation of variables $u(x, t) = F(x)T(t)$, obtaining $T(t) = \exp(-k^2\alpha^2 t)$
        and $F(x) = A\cos{kx} + B\sin{kx}$. Now, from $u'(0, t) = 0$, $F'(0) = B = 0$ and from $u(x\to 2, 0) = 0$, we must have $\cos{2k} = 0$,
        so $2k = (2n + 1)\pi /2$. Thus,
        \[
                u(x, t) = \sum_{n = 0}^\infty A_n \cos\frac{(2n + 1)\pi x}{4} e^{-((2n + 1)\pi \alpha /4)^2 t}.
        \]
        This is not quite right, since as $t \to \infty$, we want $u(2, t\to \infty) = 100$ while here, $u(2, t \to \infty) = 0$.
        We fix this by adding the solution $100$ to $u$, so
        \[
                u(x, t) = 100 + \sum_{n = 0}^\infty A_n \cos\frac{(2n + 1)\pi x}{4} e^{-((2n + 1)\pi \alpha /4)^2 t}.
        \]
        Note that we have absorbed the other constant from the cosine series into the sum so that we could take the proper limit as $t \to \infty$.
        We can now calculate the coefficients
        \[
                A_n = \frac{2}{2}\int_0^2 (-100)\cos\frac{(2n + 1)\pi x}{4}\:dx = -\frac{400}{(2n + 1)\pi}\cos{n\pi}.
        \]
        This process of obtaining the Fourier coefficients is justified, since our cosine series conly contains odd terms.
        Thus, there is no separate calculation for $A_0$, which would have otherwise been calculated separately if the cosine vanished.
        Putting everything together,
        \[
                u(x, t) = 100 \,-\, \sum_{n = 0}^\infty \frac{400}{(2n + 1)\pi}\cos{n\pi} \cos\frac{(2n + 1)\pi x}{4} e^{-((2n + 1)\pi \alpha /4)^2 t}.
        \]

        \paragraph{Problem 12.} Solve the ``particle'' in a box problem to find $\Psi(x, t)$ if $\Psi(x, 0) = \sin^2(\pi x)$ on $(0, 1)$.
        What is $E_n$? \\

        \textit{Solution.} 
        We start with the Schr\"odinger equation,
        \[
                -\frac{\hbar^2}{2m}\pp[2]{}{x}\Psi(x, t) + V(x)\Psi(x, t) = i\hbar \ppt{} \Psi(x, t).
        \]
        Performing a partial separation of variables $\Psi(x, t) = \psi(x)T(t)$, we have
        \[
                -\frac{\hbar^2}{2m} \frac{1}{\psi}\dd[2]{}{x}{\psi} + V = {i\hbar} \frac{1}{T}\ddt{}T = \text{constant } (E).
        \]
        The temporal part is solved by $T(t) = \exp(-iEt/\hbar)$.
        The spatial part of the equation becomes
        \[
                -\frac{\hbar^2}{2m}\dd[2]{}{x}{\psi} + V\psi = E\psi.
        \]
        For the ``particle in a box'' problem, we set $V(x) = 0$ on $(0, 1)$ and $\infty$ everywhere else. This forces $\Psi(x) = 0$
        outside $(0, 1)$. Setting $k^2 = 2mE/\hbar^2$, we see that a solution of the time independent equation is
        $\psi(x) = A\cos{kx} + B\sin{kx}$. Using the boundary conditions $\Psi(0, t) = \Psi(1, t) = 0$, we must have
        $A = 0$ and $k = n\pi$. Comparing this with $E$, we find that
        \[
                E_n = \frac{\hbar^2\pi^2n^2}{2m}.
        \]
        Thus, we obtain the general solution
        \[
                \Psi(x, t) = \sum_{n = 1}^\infty A_n \sin{n\pi x}\; e^{-iE_n t/\hbar}.
        \]
        The coefficients $A_n$ are calculated by setting $t = 0$, where $\Psi(x, 0) = \psi_0(x) = \sin^2{\pi x} = (1 - \cos{2\pi x}) /2$.
        \[
                A_n = 2\int_0^1 \psi_0(x) \sin{n\pi x}\:dx = \int_0^1 \sin{n\pi x} - \cos{2\pi x}\sin{n \pi x}\: dx
                        = \frac{4}{n\pi(4 - n^2)}(1 - \cos{n\pi}).
        \]
        Note that when $n = 2$, we have $A_2 = 0$ (indeed, all the even terms $A_{2n}$ are zero).
        Putting everything together,
        \[
                \Psi(x, t) = \sum_{n = 1}^\infty \frac{4}{n\pi(4 - n^2)}(1 - \cos{n\pi}) \sin{n\pi x}\; e^{-iE_n t/\hbar}.
        \]
        Again, note that the $n = 2$ term vanishes.



\end{document}
