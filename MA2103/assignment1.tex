\documentclass[10pt]{article}

\usepackage[T1]{fontenc}
\usepackage{geometry}
\usepackage{amsmath, amssymb, amsthm}
\usepackage{bm}
\usepackage{cancel}
\usepackage{xcolor}

\title{Mathematical Methods II - Assignment I}
\author{Satvik Saha}
\date{}

\geometry{a4paper, margin=1in}
\setlength\parindent{0pt}
\renewcommand{\labelenumi}{(\roman{enumi})}
\renewcommand\CancelColor{\color{red}}
% \renewcommand\qedsymbol{$\blacksquare$}
\newcommand\ve[1]{\boldsymbol{#1}}
\newcommand\ppx[1]{\frac{\partial #1}{\partial x}}
\newcommand\ppt[1]{\frac{\partial #1}{\partial t}}
\newcommand\pp[3][]{\frac{\partial^{#1}{#2}}{\partial {#3}^{#1}}}
\newcommand\ddx[1]{\frac{d #1}{d x}}
\newcommand\ddt[1]{\frac{d #1}{d t}}
\newcommand\dd[3][]{\frac{d^{#1}{#2}}{d {#3}^{#1}}}
\newcommand\norm[1]{\left\lVert#1\right\rVert}
\newcommand\grad[1]{\ve{\nabla}#1}
\newcommand\divg[1]{\ve{\nabla}\cdot#1}
\newcommand\curl[1]{\ve{\nabla}\times#1}
\newcommand\lapl[1]{\nabla^2 #1}

\begin{document}
        \par\textbf{IISER Kolkata} \hfill \textbf{Assignment I}
        \vspace{3pt}
        \hrule
        \vspace{3pt}
        \begin{center}
                \LARGE{\textbf{MA 2102 : Mathematical Methods II}}
        \end{center}
        \vspace{3pt}
        \hrule
        \vspace{3pt}
        Satvik Saha, \texttt{19MS154}\hfill\today
        \vspace{20pt}
        \subsection*{Partial Differential Equations (M.L. Boas, Chapter 13)}
        \paragraph{Section 1. Problem 4.} Obtain the heat flow equation
        \[
                \lapl{u} \;=\; \frac{1}{\alpha^2}\pp[2]{u}{t}
        \] as follows. The quantity of heat $Q$ flowing across a surface is proportional to the normal component of the (negative) temperature gradient,
        $(-\grad{T})\cdot\ve{n}$. Apply the discussion of the flow of water in Chapter 6 to the flow of heat.
        Thus, show that the rate of gain of heat per unit volume per unit time is proportional to $\divg{\grad{T}}$.
        But $\partial T /\partial t$ is proportional to this gain in heat, Thus show that $T$ satisfies the given heat equation. \\

        \textit{Solution.} Analogous to the case of flowing water, it is helpful to define a quantity called the heat flux density $\ve{q}$,
        which is the rate of flow of heat per unit area. We know that this is proportional to the negative gradient of the temperature,
        so $\ve{q} = -\kappa\grad{T}$. This is called Fourier's Law. For isotropic materials, the thermal conductivity $\kappa$ is a simple scalar.
        Now, given some surface $S$, the rate of flow of heat per unit area locally is proportional to the normal component of the heat flux,
        i.e. $\ve{q}\cdot\ve{n}$.
        This is indeed similar to the given equation in Chapter 6, where the role of $\ve{q}$ is played by the velocity of water $\ve{v}$. \\

        We proceed similarly by considering an elemental volume, a cuboid with sides $dx$, $dx$, $dz$. With an appropriate orientation
        of the coordinate axes, the normal vectors to the faces of the cuboid are simply the unit vectors $\ve{\hat{i}}$, $\ve{\hat{j}}$
        and $\ve{\hat{k}}$ (and their negatives). Thus, through the faces perpendicular to the $x$-axis, which have area $dy\,dz$,
        the rate of flow of heat is $(\ve{q}\cdot\ve{\hat{i}})dydz = q_x dy dz$. However, $q_x$ at the two opposing faces are different,
        say by a quantity $dq_x$ which we approximate using a Taylor series.
        \[
                q_x\big|_{\text{face 1}} = q_x\big|_{\text{face 2}} + (dx)\pp{q_x}{x}\bigg|_{\text{face 2}} +\; O((dx)^2).
        \]
        Ignoring the higher order terms in $dx$, we obtain
        \[
                dq_x = q_x\big|_{\text{face 1}} - q_x\big|_{\text{face 2}} = \left(\pp{q_x}{x}\right)dx.
        \]
        Thus, the net flow (inflow minus outflow) through these faces is given by $(\partial q_x /\partial x)dxdydz$.
        Doing the same for the remaining faces and adding up the flow rates must precisely yield the rate of loss of heat from our elemental volume,
        due to the conservation of heat. Thus,
        \[
                \pp{}{t}Q \;=\; -\left(\pp{q_x}{x} + \pp{q_y}{y} + \pp{q_z}{z}\right)dx dy dz \;=\; -\divg{\ve{q}}\, dV.
        \]
        Here, $Q$ is the internal heat in the elemental volume. On the other hand, we know that the rate of heat transfer in a material is related to the
        change in its temperature via the specific heat capacity $\sigma$ by $\delta Q = \sigma d m\, dT = \sigma\rho dV\, dT$, where $\rho$
        is the density of the material. Thus,
        \[
                \pp{}{t}(\sigma \rho T) dV \;=\; -\divg{\ve{q}}\, dV.
        \]
        Assuming that $\sigma, \rho$ remain constant and substituting our result from Fourier's Law yields
        \[
                \sigma\rho\pp{}{t}T \;=\; \kappa\divg{\grad{T}}.
        \]
        Setting $\alpha^2 = \kappa /\sigma\rho$ and recognizing $\divg{\grad{T}} = \lapl{T}$ gives us the heat equation.
        \[
                \frac{1}{\alpha^2}\ppt{}T \;=\; \lapl{T}.
        \]

        \paragraph{Section 2. Problem 5.} Show that the solutions of 
        \[
                \frac{1}{X} \dd[2]{}{x}X(x) = - \frac{1}{Y} \dd[2]{}{y}Y(y) = -k^2
        \]
        can also be written as
        \[
                X(x) = \begin{cases}e^{ikx} \\ e^{-ikx}\end{cases}, \quad\quad
                Y(y) = \begin{cases}\sinh{ky} \\ \cosh{ky}\end{cases}.
        \]
        Also show that these solutions are equivalent to 
        \[
                X(x) = \begin{cases}\sin{kx} \\ \cos{kx}\end{cases}, \quad\quad
                Y(y) = \begin{cases}e^{ky} \\ e^{-ky}\end{cases}.
        \]
        if $k$ is real and
        \[
                X(x) = \begin{cases}e^{kx} \\ e^{-kx}\end{cases}, \quad\quad
                Y(y) = \begin{cases}\sin{ky} \\ \cos{ky}\end{cases}.
        \]
        if $k$ is purely imaginary. Also show that $X = \sin{k(x - a)}$, $Y = \sinh{k(y - b)}$ are solutions. \\

        \textit{Solution.} Our equations can be simply written as 
        \[
                X'' = -k^2 X, \quad\quad Y'' = k^2 Y.
        \]
        It only remains to substitute the proposed solutions. It is indeed true that
        \begin{align*}
                \dd[2]{}{x}e^{ikx} &= -k^2 e^{ikx}, & \dd[2]{}{y}\sinh{ky} &= k^2\sinh{ky}, \\
                \dd[2]{}{x}e^{-ikx} &= -k^2 e^{-ikx}, & \dd[2]{}{y}\cosh{ky} &= k^2\cosh{ky}.
        \end{align*}
        The equivalences follow from the fact that linear combinations of solutions to our ODES are also solutions.
        When $k$ is real,
        \begin{align*}
                \sin{kx} &= \frac{1}{2i}(e^{ikx} - e^{-ikx}), & \cos{kx} = \frac{1}{2}(e^{ikx} + e^{-ikx}), \\
                e^{ky} &= \sinh{y} + \cosh{y}, & e^{-ky} = -\sinh{y} + \cosh{y}.
        \end{align*}
        When $k$ is purely imaginary, say $k = i\ell$ for real $\ell$, we have
        \begin{align*}
                e^{ikx} &= e^{-\ell x}, & e^{-ikx} &= e^{\ell x}, \\
                \sinh{ky} &= \frac{1}{2}(e^{i\ell y} - e^{-i\ell y}) = i\sin{\ell y}, & 
                        \cosh{ky} &= \frac{1}{2}(e^{i\ell y} + e^{-i\ell y}) = \cos{\ell y}.
        \end{align*}
        Thus, for purely imaginary $k$, the roles of $x$ and $y$ seem to have flipped. This is equivalent to choosing $+k^2$
        as our constant after separation instead of $-k^2$.
        Finally, we see that
        \[
                \dd[2]{}{x}\sin{k(x - a)} = -k^2\sin{k(x - a)}, \quad\quad \dd[2]{}{y}\sinh{k(y - b)} = k^2\sinh{k(y - b)}.
        \]
        All of the above follows from the fact that the derivatives follow the cycle
        \[
                \sin{x} \rightarrow \cos{x} \rightarrow -\sin{x} \rightarrow -\cos{x},
        \]
        \[
                \sinh{x} \leftrightarrow \cosh{x},
        \]
        
        \clearpage
        \paragraph{Problem 13.} Find the steady-state temperature distribution in a rectangular plate covering the area $0 < x < 10$, $0 < y < 20$,
        if the two adjacent sides along the axes are held at temperatures $T = x$ and $T = y$ and the other two sides at $0$. \\

        \textit{Solution.} We construct two solutions, $f$ and $g$ where $f$ is zero on all boundaries eccept the $y = 0$ boundary,
        and $g$ is zero on all boundaries eccept the $x = 0$ boundary. In both cases, we perform the standard separation of variables
        to obtain
        \[
                \frac{1}{X}\dd[2]{}{x}X = -\frac{1}{Y}\dd[2]{}{y}Y = \text{constant}.
        \]
        Now, in the case of $f$, we demand a periodic solution for $X$, so we choose a negative constant. Thus,
        \[
                X_f(x) = A\sin{k_f x} + B\cos{k_f x}, \quad\quad Y_f(y) = Ce^{k_f y} + De^{-k_f y}.
        \]
        Using $X_f(0) = X_f(10) = 0$, we have $B = 0$ and $k_f = n \pi / 10$. From $Y_f(20) = 0$, we choose $Y_f(y) = \sinh{k_f(20 - y)}$.
        Thus, our solution is of the form
        \[
                f(x, y) \;=\; \sum_{n = 1}^\infty A_n \sin{\frac{n\pi x}{10}}\sinh{\frac{n \pi (20 - y)}{10}}.
        \]
        At $y = 0$, we demand $f(x, 0) = x$. We end up with a sine series, whose coefficients $A_n$ satisfy
        \[
                A_n\sinh\frac{20 n \pi}{10} = \frac{2}{10}\int_{0}^{10} x \sin\frac{n\pi x}{10} \:dx =
                        - \frac{20\cos{n\pi}}{n\pi}.
        \]
        We have used the identity
        \[
                \int x\sin{ax}\:dx = \frac{1}{a^2}(\sin{ax} - ax\cos{ax}).
        \]
        We follow an analogous process for $g$, essentially switching the roles of $x$ and $y$ to obtain
        \[
                g(x, y) \;=\; \sum_{n = 1}^\infty B_n \sin{\frac{n\pi y}{20}}\sinh{\frac{n \pi (10 - x)}{20}}.
        \]
        At $x = 0$, we demand $g(0, y) = y$. Thus, 
        \[
                B_n\sinh\frac{10 n \pi}{20} = \frac{2}{20}\int_{0}^{20} y \sin\frac{n\pi y}{20} \:dy =
                        - \frac{40\cos{n\pi}}{n\pi}.
        \]
        Hence, our solution is simply the sum $f + g$, so 
        \begin{align*}
                T(x, y) \;=\; &\sum_{n = 1}^\infty \frac{-20\cos{n\pi}}{n\pi}\left\{\sinh{2n\pi}\right\}^{-1}
                                        \sin{\frac{n\pi x}{10}}\sinh{\frac{n \pi (20 - y)}{10}} \\
                        &+\, \sum_{n = 1}^\infty \frac{-40\cos{n\pi}}{n\pi}\left\{\sinh\frac{n\pi}{2}\right\}^{-1}
                                        \sin{\frac{n\pi y}{20}}\sinh{\frac{n \pi (10 - x)}{20}}.
        \end{align*}

        \paragraph{Problem 14.} Find the steady-state temperature distribution in a semi-infinite plate of width 10 cm if the two long
        sides are insulated, the far end (at $\infty$) is at $0$, and the bottom edge is at $T = f(x) = x - 5$. Repeat for
        $f(x) = x$ at $y = 0$ and find the value of $T$ for large $y$. \\

        \textit{Solution.} After performing separation of variables, we choose
        \[
                X(x) = A\sin{k x} + B\cos{k x}, \quad\quad Y(y) = Ce^{k y} + De^{-k y}.
        \]
        Now, since $T \to 0$ as $y \to \infty$, we must have $C = 0$. The fact that the left and right boundaries are insulated means that
        $\partial T /\partial x = 0$, which requires $X'(x) = Ak\cos{kx} - Bk\sin{kx} = 0$ at $x = 0, 10$. Thus, $A = 0$ and $k = n \pi /10$.
        Putting this together, we have
        \[
                T(x, y) = \frac{A_0}{2} \,+\, \sum_{n = 1}^\infty A_n \cos\frac{n \pi x}{10}e^{-n\pi y /10},
        \]
        where the coefficients $A_n$ are given by
        \[
                A_n = \frac{2}{10} \int_0^{10} (x - 5) \cos\frac{n\pi x}{10}\: dx.
        \]
        Note that $A_0$ vanishes. Otherwise, we have
        \[
                A_{n \geq 1} = -\frac{20}{\pi^2 n^2}(1 - \cos{n\pi}).
        \]
        We have used the identity
        \[
                \int x\cos{ax}\: dx = \frac{1}{a^2}(\cos{ax} + ax \sin{ax}).
        \]
        Thus,
        \[
                T(x, y) = \sum_{n = 1}^\infty \frac{-20}{\pi^2n^2}(1 - \cos{n\pi}) \cos\frac{n\pi x}{10} e^{-n\pi y /10}.
        \]
        Note that as $y \to \infty$, $T \to 0$ as desired. \\

        In the second case, where $f(x) = x$, we use the same reasoning as before to obtain $X(x)$. For $y \to \infty$, we assume that
        the temperature is finite, and this is enough to justify the same choice $Y(y) = e^{-ky}$ as before. Thus, the only difference
        in our new solution lies in the coefficients $A_n$.
        \[
                A_n = \frac{2}{10} \int_0^{10} x \cos\frac{n\pi x}{10}\: dx.
        \]
        This time, $A_0$ does \textit{not} vanish, and is instead equal to $10$. The remaining coefficients $A_{n \geq 1}$ remain identical,
        so the final solution is given by
        \[
                \tilde{T}(x, y) = 5 \,+\, \sum_{n = 1}^\infty \frac{-20}{\pi^2n^2}(1 - \cos{n\pi}) \cos\frac{n\pi x}{10} e^{-n\pi y /10}.
        \]
        Here, as $y \to \infty$, we see that $\tilde{T} \to 5$. This is easily explained by noting that a constant function is a solution of
        Laplace's equation. Thus, by adding a constant temperature of $5$ degrees to the entire plate in the first problem, we obtain the required
        boundary conditions in the second problem and do not change the insulation requirement. We may also interpret this as performing a
        change in units of temperature, effectively shifting all values up by $5$.


        \paragraph{Problem 16.} Show that there is only one function $u$ which takes given values on the (closed) boundary of a region and
        satisfies Laplace’s equation $\lapl{u} = 0$ in the interior of the region. \\

        \textit{Solution.} Let the interior of the region be labeled $D$, and let the boundary conditions be given such that $u = f(\ve{x})$,
        for $\ve{x}$ on the boundary $\partial D$. We select two solutions $u_1$ and $u_2$, and construct $U = u_1 - u_2$. Note that
        by the linearity of the Laplacian,
        \[ \lapl{U} = \lapl{(u_1 - u_2)} = \lapl{u_1} - \lapl{u_2} = 0\]
        in the interior $D$, and $U = u_1 - u_2 = f - f = 0$ on the boundary $\partial D$.

        We now invoke Green's first identity,
        \[
                \int_D \phi\lapl\psi - \grad\phi\cdot\grad\psi\: dV \;=\; \oint_{\partial D} (\phi\grad\psi) \cdot\ve{n}\: dS.
        \]
        Setting $U = \phi = \psi$, we have
        \[
                \int_D U\lapl{U} - \norm{\grad{U}}^2 \: dV \;=\; \oint_{\partial D} (U\grad{U})\cdot\ve{n}\: dS = 0,
        \]
        since $U = 0$ on the boundary. Also, $\lapl{U} = 0$ in the interior, so
        \[
                \int_D \norm{\grad{U}}^2\: dV \;=\; 0.
        \]
        This is possible only if $\grad{U} = 0$ everywhere in the interior. Now, choose a point $\ve{y} \in \partial D$, and let $\ve{x} \in D$
        be arbitrary. We choose a path $\gamma$ from $\ve{y}$ to $\ve{x}$.
        By the Fundamental Theorem of Calculus for line integrals, we have
        \[
                U(\ve{x}) - U(\ve{y}) \;=\; \int_\gamma (\grad U)\cdot\:d\ve{\ell} = 0.
        \]
        Thus, we have $U(\ve{x}) = U(\ve{y}) = 0$ for all $\ve{x} \in D$. Hence. $u_1 = u_2$ everywhere, so the solution to Laplace's equation
        is unique.
\end{document}
