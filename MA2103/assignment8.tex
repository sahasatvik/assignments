\documentclass[10pt]{article}

\usepackage[T1]{fontenc}
\usepackage{geometry}
\usepackage{amsmath, amssymb, amsthm}
\usepackage{bm}
\usepackage{cancel}
\usepackage{xcolor}
\usepackage{graphicx}
\usepackage{caption}
\usepackage{subcaption}
\usepackage{hyperref}
\usepackage{float}

\title{Mathematical Methods II - Assignment VIII}
\author{Satvik Saha}
\date{}

\geometry{a4paper, margin=1in}
\setlength\parindent{0pt}
\renewcommand{\labelenumi}{(\alph{enumi})}
\renewcommand\CancelColor{\color{red}}
% \renewcommand\qedsymbol{$\blacksquare$}
\newcommand\ve[1]{\boldsymbol{#1}}
\newcommand\ppx[1]{\frac{\partial #1}{\partial x}}
\newcommand\ppt[1]{\frac{\partial #1}{\partial t}}
\newcommand\pp[3][]{\frac{\partial^{#1}{#2}}{\partial {#3}^{#1}}}
\newcommand\ddx[1]{\frac{d #1}{d x}}
\newcommand\ddt[1]{\frac{d #1}{d t}}
\newcommand\dd[3][]{\frac{d^{#1}{#2}}{d {#3}^{#1}}}
\newcommand\norm[1]{\left\lVert#1\right\rVert}
\newcommand\grad[1]{\ve{\nabla}#1}
\newcommand\divg[1]{\ve{\nabla}\cdot#1}
\newcommand\curl[1]{\ve{\nabla}\times#1}
\newcommand\lapl[1]{\nabla^2 #1}

\begin{document}
        \par\textbf{IISER Kolkata} \hfill \textbf{Assignment VIII}
        \vspace{3pt}
        \hrule
        \vspace{3pt}
        \begin{center}
                \LARGE{\textbf{MA 2103 : Mathematical Methods II}}
        \end{center}
        \vspace{3pt}
        \hrule
        \vspace{3pt}
        Satvik Saha, \texttt{19MS154}\hfill\today
        \vspace{20pt}
        \subsection*{Probability and Statistics (M.L. Boas, Chapter 15)}

        \paragraph{Section 2. Problem 15.} Let two dice be thrown. Calculate the probabilities of the following.
        \begin{enumerate}
                \item The sum is greater than or equal to 4.
                \item The sum is even.
                \item The sum is divisible by 3.
                \item The sum is equal to 7, if the sum is odd.
                \item The product of the two numbers is 12.
        \end{enumerate}

        \textit{Solution.} We denote the sum of the dice as the random variable $X$. The sample space of rolls is simply all ordered pairs
        of integers $1$ through $6$, i.e.\ $S = \{(a, b)\colon a, b \in \{1, 2, 3, 4, 5, 6\}\}$. Each such roll is equally likely, of which
        there are $6^2 = 36$, hence each is associated with a probability of $1 /36$. With this, we can group the rolls by their sum
        and obtain the sample space in terms of the sums. This gives us the distribution of $X$ as follows.
        \begin{center}
                \begin{tabular}{r|ccccccccccc}
                        Sum ($x$)& 2&3&4&5&6&7&8&9&10&11&12 \\
                        $P(X = x)$& $\frac{1}{36}$& $\frac{2}{36}$ & $\frac{3}{36}$ & $\frac{4}{36}$ & $\frac{5}{36}$ & $\frac{6}{36}$ & $\frac{5}{36}$
                        & $\frac{4}{36}$ & $\frac{3}{36}$ & $\frac{2}{36}$ & $\frac{1}{36}$
                \end{tabular}
        \end{center}
        Note that these events are mutually exclusive and exhaustive, which is why they qualify as a sample space.
        Thus, we can calculate the related probabilities.
        \begin{enumerate}
                \item For the sum to be greater than of equal to $4$, we can have $X = 4, 5, 6, 7, 8, 9, 10, 11, 12$.
                To get the probability, we simply add the individual probabilities up.
                \[
                        P(X \geq 4) = \frac{3}{36} + \frac{4}{36} + \frac{5}{36} + \frac{6}{36} + \frac{5}{36} + \frac{4}{36} + \frac{3}{36}
                        + \frac{2}{36} + \frac{1}{36} = \frac{33}{36} = \frac{11}{12}.
                \]
                An alternate way to solve this would be to note that the complement event is $X < 4$, and
                \[
                        P(X < 4) = \frac{1}{36} + \frac{2}{36} = \frac{3}{36} = \frac{1}{12}.
                \]
                This gives $P(X \geq 4) = 1 - P(X < 4) = 11 /12$.

                \item The sum is even when $X = 2, 4, 6, 8, 10, 12$. Thus,
                \[
                        P(X \text{ is even}) = \frac{1}{36} + \frac{3}{36} + \frac{5}{36} + \frac{5}{36} + \frac{3}{36} + \frac{1}{36} 
                                = \frac{18}{36} = \frac{1}{2}.
                \]

                \item The sum is divisible by 3 when $X = 3, 6, 9, 12$. Thus,
                \[
                        P(3\text{ divides }X) = \frac{2}{36} + \frac{5}{36} + \frac{4}{36} + \frac{1}{36} = \frac{12}{36} = \frac{1}{3}.
                \]
                \item Firstly, the sum is odd precisely when it is not even, so $P(X\text{ is odd}) = 1 - 1 /2 = 1 /2$.
                Now, the probability that the sum is equal to 7 can be read off the table as $P(X = 7) = 6 /36 = 1 /6$.
                The probability that $X = 7$ and is odd is the same as the probability that it is 7, since the sum is guaranteed to be
                odd once it's a 7.
                Thus, we calculate the conditional probability
                \[
                        P(X = 7 | X\text{ is odd}) = \frac{P(X = 7 \text{ and } X \text{ is odd})}{P(X\text{ is odd})} = \frac{1 /6}{1 /2} = \frac{1}{3}.
                \]

                \item The product of the two numbers is 12 in precisely the following rolls.
                \[
                        (2, 6) \quad (3, 4)\quad (4, 3) \quad (6, 2).
                \]
                Thus, the probability that the product of the two numbers is 12 is given by $4 /36  = 1 /9$.
        \end{enumerate}

        \paragraph{Problem 17.} Two dice are thrown. The number on the first die is even, and the number on the second less than 4.
        \begin{enumerate}
                \item What are the possible sums and their probabilities?
                \item What is the most probable sum?
                \item What is the probability that the sum is even?
        \end{enumerate}

        \textit{Solution.} To calculate the conditional probabilities, note that $P(\text{first number is even}) = 1 /2$ and
        $P(\text{second number} < 4) = 1 /2$. This is because the first number can be 2, 4, or 6 and the second can be 1, 2, or 3.
        Thus, any roll satisfying these two constraints comes with probability $(1 /3)\times(1 /3) = 1 /9$.
        \begin{enumerate}
                \item We list the sums associated with each roll below.
                \begin{center}
                \begin{tabular}{c|ccc}
                          & 2 & 4 & 6 \\\hline
                        1 & 3 & 5 & 7 \\
                        2 & 4 & 6 & 8 \\
                        3 & 5 & 7 & 9
                \end{tabular}
                \end{center}
                Thus, the probabilities associated with each sum are obtained by a simple count.
                If $X$ is the random variable denoting the sum,
                \begin{center}
                \begin{tabular}{c|ccccccc}
                        Sum ($x$)  & 3 & 4 & 5 & 6 & 7 & 8 & 9 \\
                        $P(X = x)$ & $\frac{1}{9}$ & $\frac{1}{9}$ & $\frac{2}{9}$ & $\frac{1}{9}$ & $\frac{2}{9}$ & $\frac{1}{9}$ & $\frac{1}{9}$
                \end{tabular}
                \end{center}

                \item It is clear that the sums with the highest probabilities are 5 and 7.

                \item Even sums occur when $X = 4, 6, 8$. Adding up the associated probabilities,
                \[
                        P(X \text{ is even}) = \frac{1}{9} + \frac{1}{9} + \frac{1}{9} = \frac{3}{9} = \frac{1}{3}.
                \]
        \end{enumerate}

        \paragraph{Problem 18.} Are the following correct non-uniform sample spaces for a throw of two dice?
        If so, find the probabilities of the given sample points. If not show what is wrong.
        \begin{enumerate}
                \item First die shows an even number. \\ First die shows an odd number.
                \item Sum of two numbers on dice is even. \\ First die is even and second odd. \\ First die is odd and second even.
                \item First die shows a number less than or equal to 3. \\ At least one die shows a number greater than 3.
        \end{enumerate}

        \textit{Solution.} A list of events comprises a sample space of an experiment if they are mutually exclusive and exhaustive.
        \begin{enumerate}
                \item \textbf{Yes.} The events are exclusive, since if the first die shows an even number, it cannot show an odd number,
                and vice versa. They are also exhaustive, since the first number must either be even or odd.

                Note that the first number is even half of the time, i.e.\ when it is 2, 4, or 6. Thus, the associated probabilities 
                with the sample space are $1 /2$ and $1 /2$ respectively.

                \item \textbf{Yes.} Each of the numbers is precisely one of `even' (E) or `odd' (O). Thus, we have the combinations
                EE, EO, OE, and OO, which are all mutually exclusive and exhaust all possibilities. The middle two (EO and OE) are listed
                as events, and the first and last (EE and OO) are precisely the same as the sum being even. Note that the sum can be even
                only if both numbers have the same parity. Thus, the three events are mutually exclusive and exhaustive.

                Note that the events $E$ and $O$ individually have probability $1 /2$, thus each of their four combinations has probability
                $1 /4$. The first event combines two of these cases, hence has twice the probability. Thus, the associated probabilities 
                with the sample space are $1 /2, 1 /4$ and $1 /4$.

                \item \textbf{No.} Consider the roll (1, 6). This is falls under both cases, since the first die shows a number
                less than three, while the second shows a number greater than three. Thus, the events are not mutually exclusive. 
        \end{enumerate}

        \paragraph{Section 3. Problem 13.} 
        \begin{enumerate}
                \item A candy vending machine is out of order. The probability that you get a candy bar (with or without return of your money) is
                $1 /2$, the probability that you get your money back (with or without candy) is $1 /3$, and the probability that you
                get both the candy and your money back is $1 /12$. What is the probability that you get nothing at all?

                \item Suppose you try again to get a candy bar as in part (a).
                Set up the 16-point sample space corresponding to the possible results of your two attempts to buy a candy bar, and find
                the probability that you get two candy bars (and no money back); that you get no candy and lose your money both times;
                that you just get your money back both times.
        \end{enumerate}

        \textit{Solution.} Let $A$ denote the event where we get a candy bar, and let $B$ denote the event where we get our money back.
        We have been given
        \[
                P(A) = \frac{1}{2}, \qquad P(B) = \frac{1}{2}, \qquad P(A \cap B) = \frac{1}{12}.
        \]
        \begin{enumerate}
                \item The event where we get \textit{something} back is denoted by $A \cup B$. We use the relation
                \[
                        P(A \cup B) = P(A) + P(B) - P(A \cap B) = \frac{1}{2} + \frac{1}{3} - \frac{1}{12} = \frac{9}{12} = \frac{3}{4}.
                \]
                Thus, the complement event where we get \textit{nothing} has probability $1 - P(A \cup B) = 1 /4$.

                \item The two attempts are independent, so their probabilities multiply.
                Note that
                \begin{align*}
                        P(A \cap \overline{B}) &= P(A) - P(A \cap B) = \frac{1}{2} - \frac{1}{12} = \frac{5}{12}, \\
                        P(\overline{A} \cap B) &= P(B) - P(A \cap B) = \frac{1}{3} - \frac{1}{12} = \frac{3}{12}.
                \end{align*}
                Thus, we have the following probabilities.
                \begin{center}
                \begin{tabular}{cccc}
                        $\overline{AB}$ & $A\overline{B}$ & $\overline{A}B$ & $AB$ \\
                        $\frac{3}{12}$  & $\frac{5}{12}$  & $\frac{3}{12}$  & $\frac{1}{12}$
                \end{tabular}
                \end{center}
                With this, we tabulate the 16 point sample space for two attempts. The horizontal axis denotes the first attempt, the vertical
                one the second (although the order is immaterial). This is calculated simply by cross multiplying the distribution above.
                \begin{center}
                \begin{tabular}{c|cccc}
                                        & $\overline{AB}$  & $A\overline{B}$  & $\overline{A}B$  & $AB$ \\\hline
                        $\overline{AB}$ & $\frac{9}{144}$  & $\frac{15}{144}$ & $\frac{9}{144}$  & $\frac{3}{144}$ \\
                        $A\overline{B}$ & $\frac{15}{144}$ & $\frac{25}{144}$ & $\frac{15}{144}$ & $\frac{5}{144}$ \\
                        $\overline{A}B$ & $\frac{9}{144}$  & $\frac{15}{144}$ & $\frac{9}{144}$  & $\frac{3}{144}$ \\
                        $AB$            & $\frac{3}{144}$  & $\frac{5}{144}$  & $\frac{3}{144}$  & $\frac{1}{144}$
                \end{tabular}
                \end{center}
                With this, we note the probabilities as follows.
                \begin{itemize}
                        \item Two candy bars and no money back, $P(AA\overline{BB}) = 25 /144$.
                        \item No candy and no money, $P(\overline{AABB}) = 9 /144 = 1 /16$.
                        \item No candy, money both times, $P(\overline{AA}BB) = 9 /144 = 1 /16$.
                \end{itemize}
        \end{enumerate}

        \paragraph{Problem 17.}
        \begin{enumerate}
                \item There are 3 red and 5 black balls in one box and 6 red and 4 white balls in another.
                If you pick a box at random, and then pick a ball from it at random, what is the probability that it is red? Black? White?
                That it is either red or white?

                \item Suppose the first ball selected is red and is not replaced before a second ball is drawn.
                What is the probability that the second ball is red also?

                \item If both balls are red, what is the probability that they both came from the same box?
        \end{enumerate}

        \textit{Solution.}
        \begin{enumerate}
                \item There are 8 balls in the first box (call it $A$), and 10 balls in the second (call it $B$).
                Thus, the probability of drawing red from the boxes are $3 /8$ and $6 /10$ respectively.
                Assuming that the likelihood of picking either box is equal, we have the probability of picking a red ball as
                \[
                        P(R) = P(R \,|\, A)P(A) + P(R \,|\, B)P(B) = 
                                \frac{3}{8}\cdot \frac{1}{2} + \frac{6}{10}\cdot \frac{1}{2} = \frac{78}{160} = \frac{39}{80}.
                \]
                Similarly, the probability of picking a black ball is given by
                \[
                        P(K) = P(K \,|\, A)P(A) + P(K \,|\, B)P(B) = 
                                \frac{5}{8}\cdot \frac{1}{2} + \frac{0}{10}\cdot \frac{1}{2} = \frac{5}{16}.
                \]
                The probability of picking a white ball is
                \[
                        P(W) = P(W \,|\, A)P(A) + P(W \,|\, B)P(B) = 
                                \frac{0}{8}\cdot \frac{1}{2} + \frac{4}{10}\cdot \frac{1}{2} = \frac{4}{20} = \frac{1}{5}.
                \]
                Picking neither either red or white is equivalent to picking black. Thus, the probability of picking either red or white is
                \[
                        P(R \cup W) = 1 - P(K) = \frac{11}{16}.
                \]

                \item We calculate the probabilities of each combination of red from each box.
                \begin{align*}
                        P(R_AR_A) &= \frac{1}{2}\cdot \frac{3}{8}\cdot \frac{1}{2}\cdot \frac{2}{7} = \frac{6}{224}, \\
                        P(R_AR_B) &= \frac{1}{2}\cdot \frac{3}{8}\cdot \frac{1}{2}\cdot \frac{6}{10} = \frac{18}{320}, \\
                        P(R_BR_A) &= \frac{1}{2}\cdot \frac{6}{10}\cdot \frac{1}{2}\cdot \frac{3}{8} = \frac{18}{320}, \\
                        P(R_BR_B) &= \frac{1}{2}\cdot \frac{6}{10}\cdot \frac{1}{2}\cdot \frac{5}{9} = \frac{30}{360}. 
                \end{align*}
                Adding these up, we have $P(RR) = 187 / 840$. Conditioning this on the event that the first ball was red, we have
                \[
                        P(RR \,|\, R) = \frac{P(R \,|\, RR) \, P(RR)}{P(R)} = \frac{187 /840}{39 /80} = \frac{374}{819}.
                \]

                \item Both balls come from the same box with probability
                \[
                        P(R_AR_A) + P(R_BR_B) = \frac{6}{224} + \frac{30}{360} = \frac{37}{336}.
                \]
                Conditioning this on the event that both balls were red, we have
                \[
                        P(RR \text{ same box} \,|\, RR) = \frac{P(RR \,|\, RR \text{ same box})\, P(RR \text{ same box})}{P(RR)} = 
                                \frac{37 /336}{187 /840} = \frac{185}{374}.
                \]
        \end{enumerate}

        \paragraph{Problem 21.}
        Two people are taking turns tossing a pair of coins; the first person to toss two alike wins.
        What are the probabilities of winning for the first player and for the second player? \\

        \textit{Solution.} Let the probability of the first player winning be $p$. On any turn, the sample space of tossing
        two coins is $\{H H, H T, T H, T T\}$, with all four events being equally likely. Two coins are alike in two of these cases,
        hence the probability of winning a turn is $1 /2$. This means that the first player wins the first turn with probability $1 /2$.
        Otherwise, the second player gets her turn with probability $1 /2$. Now, the situation for the second player is identical to that
        of the first by symmetry, barring the previous turn, so she now wins with probability $p$. Thus, the probability of the second
        player winning from the very beginning is equal to the probability of her advancing to her turn, times that of her winning thereon,
        i.e. $p /2$. Since exactly one of the two players much win, these are mutually exclusive and exhaustive events, so their
        probabilities must sum to $1$. Thus,
        \[
                1 = p + \frac{p}{2}.
        \]
        Solving, we obtain $p = 2 /3$, which is the probability of the first player winning. The probability of the second player winning
        is $1 - p = p /2 = 1 /3$.

\end{document}
