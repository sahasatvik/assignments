\documentclass[11pt]{article}

\usepackage[T1]{fontenc}
\usepackage{geometry}
\usepackage{amsmath, amssymb, amsthm}
\usepackage[%
    hidealllines=true,%
    innerbottommargin=15,%
]{mdframed}
\usepackage{xcolor}
\usepackage{graphicx}
\usepackage{fancyhdr}
\usepackage{lipsum}
\usepackage{bm}
\usepackage{siunitx}

\geometry{a4paper, margin=1in, headheight=14pt}

\pagestyle{fancy}
\fancyhf{}
\renewcommand\headrulewidth{0.4pt}
\fancyhead[L]{\scshape PH2202: Thermal physics}
\fancyhead[R]{\scshape \leftmark}
\rfoot{\footnotesize\it Updated on \today}
\cfoot{\thepage}

\def\C{\mathbb{C}}
\def\R{\mathbb{R}}
\def\Q{\mathbb{Q}}
\def\Z{\mathbb{Z}}
\def\N{\mathbb{N}}
\newcommand\ve[1]{\boldsymbol{#1}}
\newcommand\ddx[1]{\frac{d #1}{d x}}
\newcommand\ddt[1]{\frac{d #1}{d t}}
\newcommand\dd[3][]{\frac{d^{#1}{#2}}{d {#3}^{#1}}}
\newcommand\ppx[1]{\frac{\partial #1}{\partial x}}
\newcommand\ppt[1]{\frac{\partial #1}{\partial t}}
\newcommand\pp[3][]{\frac{\partial^{#1}{#2}}{\partial {#3}^{#1}}}
\newcommand\norm[1]{\left\lVert#1\right\rVert}
\newcommand\grad[1]{\ve{\nabla}#1}
\newcommand\divg[1]{\ve{\nabla}\cdot#1}
\newcommand\curl[1]{\ve{\nabla}\times#1}
\newcommand\lapl[1]{\nabla^2 #1}

\newmdtheoremenv[%
    backgroundcolor=blue!10!white,%
]{theorem}{Proposition}[section]
% \newmdtheoremenv[%
%     backgroundcolor=violet!10!white,%
% ]{corollary}{Corollary}[theorem]
% \newmdtheoremenv[%
%     backgroundcolor=teal!10!white,%
% ]{lemma}[theorem]{Lemma}

\theoremstyle{definition}
\newmdtheoremenv[%
    backgroundcolor=green!10!white,%
]{definition}{Definition}[section]
% \newmdtheoremenv[%
%     backgroundcolor=red!10!white,%
% ]{exercise}{Exercise}[section]
\newmdenv[%
    backgroundcolor=cyan!5!white,%
    innertopmargin=10,%
]{cyanbox}
\newenvironment{boxedeq}%
    {\begin{cyanbox}\begin{equation}}%
    {\end{equation}\end{cyanbox}}
\newenvironment{boxedeq*}%
    {\begin{cyanbox}\begin{equation*}}%
    {\end{equation*}\end{cyanbox}}

\theoremstyle{remark}
\newtheorem*{remark}{Remark}
\newtheorem*{example}{Example}

\numberwithin{equation}{section}

\title{
    \Large\textsc{PH2202} \\
    % \vspace{10pt}
    \Huge \textbf{Thermal physics} \\
    \vspace{5pt}
    \Large{Spring 2021}
}
\author{
    \large Satvik Saha%
    % \thanks{Email: \tt ss19ms154@iiserkol.ac.in}
    \\\textsc{\small 19MS154}
}
\date{\normalsize
    \textit{Indian Institute of Science Education and Research, Kolkata, \\
    Mohanpur, West Bengal, 741246, India.} \\
    % \vspace{10pt}
    % \today
}

\begin{document}
    \maketitle

    Thermal physics deals with the topic of \textit{temperature}.
    Temperature is a statistical property -- thus, it makes no sense to talk of the
    temperature of one, two, or even a handful of particles.

    \tableofcontents

    \section{Kinetic Theory of Gases}
    \subsection{The molecular picture of matter}
    Imagine looking into a container filled with steam, and magnifying by a factor
    of \num{e10}. A cubic metre might contain around \num{20} molecules, all of
    which are in constant motion, colliding with the walls and each other.
    Suppose that one of the walls is a piston. The molecules which collide with the
    piston and impart a force on it; in order to fix the piston in place, a
    counter force must be applied.
    \begin{definition}[Pressure]
        The force per unit area applied by a gas on the walls of its container is
        called the pressure of the gas.
    \end{definition}
    Now provide the system with heat. We know that the temperature of the gas must
    increase -- what this means is that the speeds of the molecules increase, on
    average.
    \begin{definition}[Temperature]
        The temperature of a gas is a measure of the average kinetic energy of the
        constituent particles.
    \end{definition}
    Instead, consider an adiabatic container, which stops all flow of heat into and
    out of the gas. By compressing the gas with the piston, we observe that the
    temperature of the gas also rises.

    Now, take away heat from the system. The temperature drops and the molecules
    tend to be close to each other. This is because of the dipolar attractive forces
    between the molecules (which varies as the inverse cube of the distance of the
    dipoles, and is hence comparatively short range).
    On the other hand, they cannot get too close, since once the electron clouds of
    the molecules start to overlap, a repulsive force is introduced. At a certain
    point, we reach a condensed form of matter: liquid water.

    Liquid water is very much incompressible, yet the molecules freely move and
    slide around, without any periodic arrangement. The molecules at the surface are
    attracted by like molecules inside; this cohesive force keeps the liquid 
    condensed. This tendency of a liquid to minimize its surface area is related to
    the phenomenon of surface tension. Some molecules on the surface are energetic
    enough to escape this cohesive attraction and leave the liquid -- this is called
    evaporation. Heating a liquid simply increases the average kinetic energy of the
    molecules, thus increasing the rate of evaporation.
    When these energetic molecules leave the liquid, the average kinetic energy of
    the liquid drops, hence it cools down. This is the phenomenon of latent heat.

    When this happens in a closed container, the process of evaporation cannot go on
    indefinitely, since the air has a limited capacity for holding moisture.
    Condensation is the process where these airborne molecules return to the liquid.
    At a certain point, the rates of evaporation and condensation become equal, and
    we obtain a saturated vapour.

    Return to the liquid, and take away even more heat. Now, the motion of the
    molecules decrease to a point where they occupy fixed positions. They are still
    in motion, but their movement is restricted around their mean position.
    This is the crystal state. The lower the temperature, the smaller the
    oscillations and vibrations.

    \subsection{Basic assumptions}
    \begin{enumerate}
        \itemsep0em 
        \item Gases are made up of a large number of molecules, and all molecules of
        one gas are identical.
        \item Molecules of a gas are always moving. The number of molecules per unit
        volume remains constant, i.e.\ the density remains constant.
        \item Molecules behave as elastic spheres during collisions. Kinetic energy
        and momenta are conserved, and the collision time is negligible compared to
        the mean path time.
        \item No force acts on any molecule, except during collisions.
        Intermolecular forces are only short ranged. Between collisions, the
        molecules continue moving with uniform velocity in a straight line.
        \item The entire gas is isotropic; for all molecules, all directions are the
        same.
    \end{enumerate}
    
    \subsection{Ideal gases}
    We start by considering very simple collections of molecules. We assume that
    they are identical, spherical, with negligible size and with no intermolecular
    interactions. They only undergo elastic collisions.

    % \begin{definition}[Mean free path]
    %     The mean free path is the average distance covered by a gas molecule before
    %     colliding, either with another gas particle or the walls of its container.
    % \end{definition}

    \begin{theorem}[Ideal gas law]
        The ideal gas law gives a relation between the pressure $p$, the volume $V$,
        the temperature $T$, and the number of moles $n$ of an ideal gas.
        \[
            pV \,=\, nRT.
        \]
        Here, the constant of proportionality $R$ is called the ideal gas constant,
        with value \[
            R \,\approx\, \SI{8.314}{\joule\per\mole\per\kelvin}.
        \]
    \end{theorem}

\end{document}
% vim: set tabstop=4 shiftwidth=4 softtabstop=4:
