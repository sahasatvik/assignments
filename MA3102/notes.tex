\documentclass[11pt]{article}

\usepackage[T1]{fontenc}
\usepackage{geometry}
\usepackage{amsmath, amssymb, amsthm}
\usepackage[scr]{rsfso}
\usepackage[%
    hidealllines=true,%
    innerbottommargin=15,%
    nobreak=true,%
]{mdframed}
\usepackage{xcolor}
\usepackage{graphicx}
\usepackage{fancyhdr}
\usepackage{hyperref}

\geometry{a4paper, margin=1in, headheight=14pt}

\pagestyle{fancy}
\fancyhf{}
\renewcommand\headrulewidth{0.4pt}
\fancyhead[L]{\scshape MA3102: Algebra I}
\fancyhead[R]{\scshape \leftmark}
\rfoot{\footnotesize\it Updated on \today}
\cfoot{\thepage}

\newcommand{\C}{\mathbb{C}}
\newcommand{\R}{\mathbb{R}}
\newcommand{\Q}{\mathbb{Q}}
\newcommand{\Z}{\mathbb{Z}}
\newcommand{\N}{\mathbb{N}}

\newmdtheoremenv[%
    backgroundcolor=blue!10!white,%
]{theorem}{Theorem}[section]
\newmdtheoremenv[%
    backgroundcolor=violet!10!white,%
]{corollary}{Corollary}[theorem]
\newmdtheoremenv[%
    backgroundcolor=teal!10!white,%
]{lemma}[theorem]{Lemma}

\theoremstyle{definition}
\newmdtheoremenv[%
    backgroundcolor=green!10!white,%
]{definition}{Definition}[section]
\newmdtheoremenv[%
    backgroundcolor=red!10!white,%
]{exercise}{Exercise}[section]

\theoremstyle{remark}
\newtheorem*{remark}{Remark}
\newtheorem*{example}{Example}
\newtheorem*{solution}{Solution}

\surroundwithmdframed[%
    linecolor=black!20!white,%
    hidealllines=false,%
    innertopmargin=5,%
    innerbottommargin=10,%
    skipabove=0,%
    skipbelow=0,%
]{example}

\numberwithin{equation}{section}

\title{
    \Large\textsc{MA3102} \\
    \Huge \textbf{Algebra I} \\
    \vspace{5pt}
    \Large{Autumn 2021}
}
\author{
    \large Satvik Saha
    \\\textsc{\small 19MS154}
}
\date{\normalsize
    \textit{Indian Institute of Science Education and Research, Kolkata, \\
    Mohanpur, West Bengal, 741246, India.} \\
}

\begin{document}
    \maketitle

    \tableofcontents

    \section{Groups and Symmetries}
    \subsection{Symmetries of plane figures}
    A symmetry of a plane figure can be thought of as a rigid motion which
    \emph{preserves its structure}, i.e.\ sends it to itself.

    For example, consider an equilateral triangle; there is the identity symmetry
    (which does nothing), two rotations by $2\pi / 3$ and $2\pi / 3$, and three
    reflections. This gives us a total of $6$ symmetries. Coincidentally, the plane
    symmetries of an equilateral triangle are precisely the set of $3! = 6$
    permutations of its vertices.

    The same cannot be said of a square; there are $4! = 24$ of its vertices, but
    only $8$ of them are rigid motions. Here, we see $4$ rotations and $4$
    reflections.

    In general, a regular $n$-gon has $2n$ plane symmetries, of which $n$ are
    rotations and $n$ are reflections. This can be seen by noting that a symmetry of
    an $n$-gon is completely determined by its action on an edge; once the final
    positions of the first two vertices is determined, the rest are forced. There are
    $n$ positions for the first vertex, which leaves only $2$ positions for the
    second vertex. One of these choices results in a rotation (since it preserves the
    cyclicity of the vertices) and the other a reflection (since it reverses the
    cyclicity of the vertices).

    Note that these symmetries can be \emph{composed}, i.e.\ applied in succession.
    For example, a rotation by $2\pi / n$ can be applied repeatedly to obtain every
    possible rotational symmetry. Similarly, we can perform rotations and reflections
    in succession, and we always end up with another symmetry. This composition is
    associative, there is an identity symmetry, and each symmetry has an inverse.
    The collection of such symmetries forms a \emph{group}.

    The group of plane symmetries of a regular $n$-gon is called the \emph{dihedral
    group}, denoted as $D_{2n}$.

    \subsection{Basic definitions}
    \begin{definition}
        A group is a set $G$ with a binary operation of composition, satisfying the
        following properties.
        \begin{enumerate}
        \itemsep0em
            \item \emph{Associativity}: For all $a, b, c \in G$, $a(bc) = (ab)c$.
            \item \emph{Existence of an identity element}: There exists $e \in G$
            such that for all $a \in G$, $ae = e = ea$.
            \item \emph{Existence of inverse elements}: For all $a \in G$, there
            exists $b \in G$ such that $ab = e = ba$. We denote $b = a^{-1}$.
        \end{enumerate}
    \end{definition}
    \begin{example}
        The integers $\Z$ form a group under addition.
    \end{example}
    \begin{example}
        The set $\{-1, +1\}$ forms a group under multiplication.
    \end{example}
    \begin{example}
        The symmetries of a tetrahedron form a group under composition of symmetries.
    \end{example}
    
\end{document}
