\documentclass[10pt]{article}

\usepackage[T1]{fontenc}
\usepackage{geometry}
\usepackage{amsmath, amssymb, amsthm}

\title{Geodynamics - Assignment I}
\author{Satvik Saha}
\date{}

\geometry{a4paper, margin=1in}
% \setlength\parindent{0pt}
\renewcommand{\labelenumi}{(\alph{enumi})}

\begin{document}
        \noindent\textbf{IISER Kolkata} \hfill \textbf{Assignment I}
        \vspace{3pt}
        \hrule
        \vspace{3pt}
        \begin{center}
                \LARGE{\textbf{ES 2102 : Hydrology and Geodynamics}}
        \end{center}
        \vspace{3pt}
        \hrule
        \vspace{3pt}
        Satvik Saha, \texttt{19MS154}\hfill\today
        \vspace{20pt}

        \paragraph{Solution 1} The tipping point at which the plate becomes subductable, i.e.\ negatively buoyant, is when the
        lithospheric mantle buoyancy (which is negative, pulling the crust down) becomes equal to and exceeds the crustal buoyancy. Thus, we want
        \[
                (\rho_{am} - \rho_c)h_c g \;\leq\; \frac{1}{2}\alpha\rho_{am}(h_p - h_c)(T_{am} - T_{bc})g.
        \]
        Rearranging, we have
        \[
                h_p \;\geq\; h_c + \frac{2(\rho_{am} - \rho_c)h_c}{\alpha\rho_{am}(T_{am} - T_{bc})}
        \]
        \begin{enumerate}
                \item Using the given values, with $h_c = 7$ km, we obtain the height of the oceanic plate,
                \[
                        h_p \;\geq\; h_c + \frac{2\times(3250 - 3000)}{3\times 10^{-5}\times 3250 \times (1250 - 554)}\,h_c = (1 + 7.37) h_c
                                \;\approx\; 58.6 \text{ km}.
                \]
                The height of the mantle lithosphere is simply $h_{lm} = h_p - h_c \geq 51.6$ km. We use the relation $h_{lm} = 10\sqrt{A}$
                where $A$ is the mantle lithosphere age in million years. Thus,
                \[
                        A \;=\; \left(\frac{h_{lm}}{10}\right)^2 \,\geq\, 5.16^2 \,=\, 26.6\text{ million years}.
                \]
                \item From the given relation, we see that $h_{lm} = h_p - h_c \propto h_c$. Thus, the thickness of lithospheric mantle, hence
                the age $A \propto h_{lm}^2$ must increase with an increase in the crust thickness $h_c$. Specifically,
                \[
                        A \propto h_{lm}^2 \propto h_c^2.
                \]
                We have already calculated the coefficient of proportionality as $h_{lm} = 7.37\, h_c$, so $A = 0.737^2 h_c^2 = 0.54 h_c^2$.
                For $h_c$ between $6$ and $8$ km, we see that $A$ ranges between 19.6 and 34.8 million years.
        \end{enumerate}

        \paragraph{Solution 2} For the subduction of oceanic crust of thickness between $10$ and $40$ km, the plate must have a thickness
        of at least $h_p = 8.37 h_c$, which is between 83.7 and 334.8 km.
        Of course, in order for this crust to subduct underneath another plate, it must be even more negatively buoyant than the other plate.

        Such an ocean lithosphere with a plateau embedded in it would not undergo a typical subduction process, since the plateau would
        offer resistance as soon as it reaches a trench. In such an ocean-ocean plate subduction zone, the plateau would force a subduction flip.
        This is because the plateau would act as an obstruction, locking the plate from moving further downwards, thus forcing the other plate
        to subduct.

        Furthermore, note that the required thickness of such a subductable oceanic plate is very high. If such a plate is forced
        underneath another plate without attaining this thickness, it will undergo flat slab subduction until enough lithospheric mantle
        develops to make it negatively buoyant, hence subductable.
        
\end{document}
