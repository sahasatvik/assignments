\documentclass[10pt]{article}

\usepackage[T1]{fontenc}
\usepackage{geometry}
\usepackage{amsmath, amssymb, amsthm}
\usepackage{chemformula}
\usepackage{array} 
\usepackage{tikz, tikz-3dplot}

\geometry{a4paper, margin=1in}
\setlength\parindent{0pt}
% \renewcommand\qedsymbol{$\blacksquare$}
\newcolumntype{L}{l@{\quad\quad}}

\begin{document}
        \par\textbf{IISER Kolkata} \hfill \textbf{Problem Sheet II}
        \vspace{3pt}
        \hrule
        \vspace{3pt}
        \begin{center}
                \LARGE{\textbf{CH1101 : Elements of Chemistry}}
        \end{center}
        \vspace{3pt}
        \hrule
        \vspace{3pt}
        Satvik Saha, \texttt{19MS154}\hfill\today
        \vspace{20pt}

        \begin{enumerate}
                \item An \textit{eigenfunction} of a given operator $D$ is any (non-zero) function $f$ which, when operated upon by $D$,
                        gets multiplied by some scalar $\lambda$ called its \textit{eigenvalue}, i.e.,
                        \[
                                Df \;=\; \lambda f.
                        \]
                        For example, consider the differentiation operator $D_x = \frac{\mathrm{d} }{\mathrm{d}x}$. Note that
                        \[
                                D_x \exp(kx) \;=\; k \cdot\exp(kx).
                        \]
                        Thus, $\exp(kx)$ is an eigenfunction of the operator $D_x$, with an eigenvalue of $k$.
                
                \item Below is the point $P(x, y, z)$ in a spherical polar coordinate system.
                        \begin{center}
                        \tdplotsetmaincoords{60}{120}
                        \begin{tikzpicture}[scale=4, tdplot_main_coords]
                        \coordinate (O) at (0, 0, 0) node[left]{$O$};
                        \draw[thick, -latex] (0, 0, 0) -- (1, 0, 0) node[anchor=north east]{$x$};
                        \draw[thick, -latex] (0, 0, 0) -- (0, 1, 0) node[anchor=north west]{$y$};
                        \draw[thick, -latex] (0, 0, 0) -- (0, 0, 1) node[anchor=south]{$z$};
                        
                        \tdplotsetcoord{P}{1}{30}{60}
                        \draw[-stealth, color=red] (O) -- (P) node[above right]{$P$};
                        \draw[dashed, color=red] (O) -- (Pxy) node[below right]{$P'$};
                        \draw[dashed, color=red] (P) -- (Pxy);
                        \tdplotdrawarc[-stealth]{(O)}{0.2}{0}{60}{anchor=north}{$\phi$}
                        \tdplotsetthetaplanecoords{60}
                        \tdplotdrawarc[tdplot_rotated_coords, -stealth]{(0,0,0)}{0.5}{0}{30}{anchor=south west}{$\theta$}
                        \end{tikzpicture}
                        \end{center}
                        \begin{align*}
                                r \;&=\; \sqrt{x^2 + y^2 + z^2} \\
                                \theta \;&=\; \arccos\left( \frac{z}{\sqrt{x^2 + y^2 + z^2}}\right) \\
                                \phi \;&=\; \arctan\left(\frac{y}{x}\right)
                        \end{align*}
                        We must have $r \geq 0$, $0\leq\theta\leq\pi$ and $0\leq\phi\leq 2\pi$.
                \item The \textit{wavefunction} $\psi$ of a particle is a mathematical entity (a complex valued function) which contains
                all of the dynamical information about the system. In this way, it can be considered as the central carrier of
                information in quantum mechanics.

                \item
                \begin{enumerate}
                        \item \textbf{Wave:} Particles were not known to be able to pass through seemingly solid metal foil.
                        \item \textbf{Particle:} All known particles were known to travel at speeds less than that of light --
                                the speed of an electron is consistent with this.
                        \item \textbf{Wave:} Shadows are commonly cast by waves such as light.
                        \item \textbf{Particle:} Charged particles in motion were known to interact with electrical fields,
                                while waves do not do so.
                \end{enumerate} 
                \item The first transition involves the absorption of a $95 \text{ nm}$ photon, which corresponds to 
                a frequency of $\approx 3.2 \times 10^{15} \text{ Hz}$, i.e., near ultraviolet light.

                The second transition involves the emission of a photon of $1282 \text{ nm}$, which corresponds to
                a frequency of $\approx 2.3 \times 10^{14} \text{ Hz}$, i.e., near infrared light.
                
                \item We will use $\lambda = h/p$, $p = mv$.
                An \ch{O_2} molecule weighs $32 \text{ amu} \approx 5.3 \times 10^{-26} \text{ kg}$. Thus, its momentum is $2.5 \times 10^{-23}
                \text{ kg m/s}$, and its de Broglie wavelength is $2.6 \times 10^{-11} \text{m} = 26 \text{ pm}$. Clearly, this is
                a small fraction ($10.7 \text{ \%}$) of the molecular length of \ch{O_2}.

                \item 
                \begin{enumerate}
                        \item We have $l \in \{0, 1, 2, \dots, (n-1)\}$, i.e., $n = 7$ possible values for $l$.
                        \item A 6d subshell corresponds to $n = 6$, $l = 2$. Thus, we have $m \in \{0, \pm 1, \dots, \pm l\}$,
                        i.e., $2l + 1 = 5$ possible values for $m$.
                        \item A 3p subshell corresponds to $n = 3$, $l = 1$. Thus we have $2l + 1 = 3$ possible values for $m$.
                        \item For $n = 4$, we have exactly $4$ subshells, i.e, 4s, 4p, 4d and 4f.
                \end{enumerate}

                \item
                \begin{enumerate}
                        \item 6p consists of a linear combination of the $\psi_{61-1}$, $\psi_{610}$ and $\psi_{611}$ wavefunctions, such
                        that it has no imaginary component.
                        \item 3d consists of a linear combination of the $\psi_{32-2}$, $\psi_{32-1}$, $\psi_{320}$, $\psi_{321}$ and $\psi_{322}$
                        wavefunctions.
                \end{enumerate}
        \end{enumerate}
\end{document}
