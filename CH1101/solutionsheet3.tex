\documentclass[10pt]{article}

\usepackage[T1]{fontenc}
\usepackage{geometry}
\usepackage{amsmath, amssymb, amsthm}
\usepackage{chemformula}
\usepackage{array} 
\usepackage{tikz, tikz-3dplot}

\geometry{a4paper, margin=1in}
\setlength\parindent{0pt}
\renewcommand{\labelenumii}{(\roman{enumii})}
% \renewcommand\qedsymbol{$\blacksquare$}
\newcolumntype{L}{l@{\quad\quad}}

\begin{document}
        \par\textbf{IISER Kolkata} \hfill \textbf{Problem Sheet III}
        \vspace{3pt}
        \hrule
        \vspace{3pt}
        \begin{center}
                \LARGE{\textbf{CH1101 : Elements of Chemistry}}
        \end{center}
        \vspace{3pt}
        \hrule
        \vspace{3pt}
        Satvik Saha, \texttt{19MS154}\hfill\today
        \vspace{20pt}

        \begin{enumerate}
                \item We have
                \begin{align*}
                        P_{1s}(r) \;&=\; 4\pi r^2 \left( \psi(r) \right)^2 \\
                        \psi_{1s}(r) \;&=\; K \exp(-r/a_0)
                \end{align*}
                Thus, for an electron in the 1s orbital, $P(r)$ increases, reaches a maxima, then decreases asymptotically
                to zero.
                
                Although $\psi(r)$ is maximum at the nucleus, note that $r^2 \to 0$. Thus,
                for very small $r$,
                \begin{align*}
                        P(r) \;\propto\; \lim_{\;\;r \to 0^+} r^2 \cdot \exp(-r) \;=\; 0
                \end{align*}
                Conversely, for very large $r$, we take the limit
                \begin{align*}
                        P(r) \;\propto\; \lim_{\;\;r \to \infty} r^2 \cdot \exp(-r) \;=\; 0
                \end{align*}
                Here, we have used L'H\^ opital's Rule.

                The extrema of $P(r)$ occur at $r$ such that 
                \begin{align*}
                        \frac{\mathrm{d} }{\mathrm{d}r}\; P(r) \;&=\; 0 \\
                        \frac{\mathrm{d} }{\mathrm{d}r}\; r^2 \cdot \exp(-2r/a_0) \;&=\; 0 \\
                        r \;&=\; a_0
                \end{align*}
                Note that $\frac{\mathrm{d}^2 }{\mathrm{d}r^2} P(r)$ at $a_0$ is negative, which means that $P(r)$ shows a maximum there.

                \item 

                \item
                \begin{enumerate}
                        \item The highest probability of finding an electron in the 3d orbital is closest to the nucleus,
                        with $r/a_0 \approx 9$.
                        \item At $r \approx 0.1a_0$, the probability of finding an electron is maximum for the 3s orbital.
                \end{enumerate}

                \item 

                \item Let $x = r/a_0$. We have
                \[ R_{31} \;=\; N_{31} (6x  - x^2) \exp(-x/3). \]
                Cleary, the exponential part of this function never reaches zero. The quadratic part has its zeroes at $0$ and $6$.
                We thus know that the corresponding radial distribution function for the 3p orbital has a radial node at $r = 6a_0$.

                \textit{(We ignore $r = 0$ since the probability of finding an electron at the nucleus is meaningless.)}

                \item \mbox{} 
                \begin{center}
                \begin{tabular}{r|ccc}
                        &       \text{Radial nodes}     &       \text{Angular nodes}    &       \text{Total nodes} \\\hline        
                        1s&     0&      0&      0\\
                        2s&     1&      0&      1\\
                        2p&     0&      1&      1\\
                        3s&     2&      0&      2\\
                        3p&     1&      1&      2
                \end{tabular}
                \end{center}
                        
                \item
                \begin{enumerate}
                        \item This is a 3p$_z$ orbital, with one radial node and one angular node.
                        \item $a$ represents an angular node. $b$ represents a radial node.
                        \item Each line in the contour plot is drawn such that the probability of finding an electron on any point
                        on that line is exactly the same. Thus, each line is an iso-probable curve.
                        Red lines are drawn where the wavefunction of the electron is postive, and blue lines are drawn where the
                        wavefunction is negative.
                        \item $c < d < e$, in order of the probability of finding an electron on that curve.
                        \item $d > f$, by horizontal symmetry of the wavefunction.
                \end{enumerate}
                
                \item 
                
        \end{enumerate}
\end{document}
