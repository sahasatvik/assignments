\documentclass[11pt]{article}

\usepackage[T1]{fontenc}
\usepackage{geometry}
\usepackage{amsmath, amssymb, amsthm}
\usepackage[%
    hidealllines=true,%
    innerbottommargin=15,%
]{mdframed}
\usepackage{xcolor}
\usepackage{graphicx}
\usepackage{fancyhdr}
\usepackage{lipsum}

\geometry{a4paper, margin=1in, headheight=14pt}

\pagestyle{fancy}
\fancyhf{}
\renewcommand\headrulewidth{0.4pt}
\fancyhead[L]{\scshape MA2201: Analysis II}
\fancyhead[R]{\leftmark}
\rfoot{\footnotesize\it Updated on \today}
\cfoot{\thepage}

\def\C{\mathbb{C}}
\def\R{\mathbb{R}}
\def\Q{\mathbb{Q}}
\def\Z{\mathbb{Z}}
\def\N{\mathbb{N}}
\newcommand\ddx[1]{\frac{d #1}{d x}}
\newcommand\ddt[1]{\frac{d #1}{d t}}
\newcommand\dd[3][]{\frac{d^{#1}{#2}}{d {#3}^{#1}}}
\newcommand\ppx[1]{\frac{\partial #1}{\partial x}}
\newcommand\ppt[1]{\frac{\partial #1}{\partial t}}
\newcommand\pp[3][]{\frac{\partial^{#1}{#2}}{\partial {#3}^{#1}}}
\newcommand\norm[1]{\left\lVert#1\right\rVert}

\newmdtheoremenv[%
    backgroundcolor=blue!10!white,%
]{theorem}{Theorem}[section]
\newmdtheoremenv[%
    backgroundcolor=violet!10!white,%
]{corollary}{Corollary}[theorem]
\newmdtheoremenv[%
    backgroundcolor=teal!10!white,%
]{lemma}[theorem]{Lemma}

\theoremstyle{definition}
\newmdtheoremenv[%
    backgroundcolor=green!10!white,%
]{definition}{Definition}[section]
\newmdtheoremenv[%
    backgroundcolor=red!10!white,%
]{exercise}{Exercise}[section]

\theoremstyle{remark}
\newtheorem*{remark}{Remark}
\newtheorem*{example}{Example}

\numberwithin{equation}{section}

\title{
    \Large\textsc{MA2201} \\
    % \vspace{10pt}
    \Huge \textbf{Analysis II} \\
    \vspace{5pt}
    \Large{Spring 2021}
}
\author{
    \large Satvik Saha%
    % \thanks{Email: \tt ss19ms154@iiserkol.ac.in}
    \\\textsc{\small 19MS154}
}
\date{\normalsize
    \textit{Indian Institute of Science Education and Research, Kolkata, \\
    Mohanpur, West Bengal, 741246, India.} \\
    % \vspace{10pt}
    % \today
}

\begin{document}
    \maketitle

    \section{Sequences and series of functions}
    \begin{definition}
        Let the functions $f_n\colon X \to Y$ be defined for all $n \in \mathbb{N}$
        and let the sequences $\{f_n(x)\}$ converge for all $x \in X$. We define the
        function $f\colon X \to Y$ as \[
            f(x) = \lim_{n \to \infty} f_n(x) 
        \] for all $x \in X$, and call $f$ the limit of $\{f_n\}$.
        We also say that $\{f_n\}$ converges to $f$ pointwise on $X$.
    \end{definition}
    \begin{example}
        Consider the functions $f_n\colon [0, 1] \to \R$, $x \mapsto x^n$.
        It can be shown that $x^n \to 0$ when $x \in [0, 1)$ and $x^n \to 1$ and
        $x^n \to 1$ when $x = 1$. Thus, $f = \lim_{n \to \infty} f_n$ is well
        defined. \[
            f(x) = \begin{cases}
                0, &\text{ if } 0 \leq x < 1 \\
                1, &\text{ if } x = 1
            \end{cases}.
        \] 
    \end{example}
    \begin{remark}
        Note that while each $f_n$ is continuous in this example, the limit $f$
        is not.
    \end{remark}
    \begin{example}
        Consider the functions $f_n\colon \R \to \R$, $x \mapsto x/n$.
        We see that $f_n \to 0$. Note that $0$ here denotes the zero function.
    \end{example}
    
    \begin{definition}
        Let the functions $f_n\colon X \to Y$ be defined for all $n \in \mathbb{N}$
        and let the sums $\{\sum f_n(x)\}$ converge for all $x \in X$.
        We define the function $f\colon X \to Y$ as \[
            f(x) = \sum_{n = 1}^\infty f_n(x)
        \] for all $x \in X$, and call $f$ the sum of the series $\sum f_n$.
    \end{definition}
    \begin{example}
        Consider the functions $f_n\colon (0, 1) \to \R$, $x \mapsto x^n$.
        Note that the sum \[
            \sum_{n = 1}^{\infty} x^n = x + x^2 + x^3 + \dots 
                = \frac{x}{1 - x} 
        \] does indeed converge for all $x \in (0, 1)$. Thus, the sum $f = \sum f_n$
        is well defined. \[
            f(x) = \frac{x}{1 - x}.
        \] 
    \end{example}

\end{document}
% vim: set tabstop=4 shiftwidth=4 softtabstop=4:
