\documentclass[11pt]{article}

\usepackage[T1]{fontenc}
\usepackage{geometry}
\usepackage{amsmath, amssymb, amsthm}
\usepackage[scr]{rsfso}
\usepackage[%
    hidealllines=true,%
    innerbottommargin=15,%
    nobreak=true,%
]{mdframed}
\usepackage{xcolor}
\usepackage{graphicx}
\usepackage{fancyhdr}
\usepackage{lipsum}
\usepackage{hyperref}

\geometry{a4paper, margin=1in, headheight=14pt}

\pagestyle{fancy}
\fancyhf{}
\renewcommand\headrulewidth{0.4pt}
\fancyhead[L]{\scshape MA2201: Analysis II}
\fancyhead[R]{\scshape Differentiation}
\rfoot{\footnotesize\it Updated on \today}
\cfoot{\thepage}

\def\C{\mathbb{C}}
\def\R{\mathbb{R}}
\def\Q{\mathbb{Q}}
\def\Z{\mathbb{Z}}
\def\N{\mathbb{N}}
\newcommand\ddx[1]{\frac{d #1}{d x}}
\newcommand\ddt[1]{\frac{d #1}{d t}}
\newcommand\dd[3][]{\frac{d^{#1}{#2}}{d {#3}^{#1}}}
\newcommand\ppx[1]{\frac{\partial #1}{\partial x}}
\newcommand\ppt[1]{\frac{\partial #1}{\partial t}}
\newcommand\pp[3][]{\frac{\partial^{#1}{#2}}{\partial {#3}^{#1}}}
\newcommand\norm[1]{\left\lVert#1\right\rVert}

\newcounter{module}
\setcounter{module}{3}

\newmdtheoremenv[%
    backgroundcolor=blue!10!white,%
]{theorem}{Theorem}[module]
\newmdtheoremenv[%
    backgroundcolor=violet!10!white,%
]{corollary}{Corollary}[theorem]
\newmdtheoremenv[%
    backgroundcolor=teal!10!white,%
]{lemma}[theorem]{Lemma}

\theoremstyle{definition}
\newmdtheoremenv[%
    backgroundcolor=green!10!white,%
]{definition}{Definition}[module]
\newmdtheoremenv[%
    backgroundcolor=red!10!white,%
]{exercise}{Exercise}[module]

\theoremstyle{remark}
\newtheorem*{remark}{Remark}
\newtheorem*{example}{Example}
\newtheorem*{solution}{Solution}

\surroundwithmdframed[%
    linecolor=black!20!white,%
    hidealllines=false,%
    innertopmargin=5,%
    innerbottommargin=10,%
    skipabove=0,%
    skipbelow=0,%
]{example}

\numberwithin{equation}{module}

\title{
    \Large\textsc{MA2201: Analysis II} \\
    % \vspace{10pt}
    \Huge \textbf{Integration} \\
    \vspace{5pt}
    \Large{Spring 2021}
}
\author{
    \large Satvik Saha%
    % \thanks{Email: \tt ss19ms154@iiserkol.ac.in}
    \\\textsc{\small 19MS154}
}
\date{\normalsize
    \textit{Indian Institute of Science Education and Research, Kolkata, \\
    Mohanpur, West Bengal, 741246, India.} \\
    % \vspace{10pt}
    % \today
}

\begin{document}
    \maketitle

    \begin{definition}[Partition]
        A partition $Q$ of an interval $[a. b]$ is a finite sequence of numbers \[
            a = x_0 < x_1 < \dots < x_{n - 1} < x_n = b.
        \] The norm of a partition is defined as \[
            \norm{Q} = \max |x_{j + 1} - x_j|.
        \] A tagged partition $\dot{Q}$ is a partition $Q$ together with a set of
        numbers $t_j$ such that $t_j \in [x_j, x_{j + 1}]$.
    \end{definition}

    \begin{definition}[Riemann sum]
        The Riemann sum of a function $f$ on an interval $[a, b]$ with respect to a
        tagged partition $\dot{Q}$ is defined as \[
            S(f, \dot{Q}) = \sum_{j = 0}^{n - 1}  f(t_j)(x_{j + 1} - x_j). 
        \] 
    \end{definition}

    \begin{definition}[Riemann integral]
        A function $f$ is called Riemann integrable on an interval $[a, b]$ if 
        there is some $\ell \in \R$ where for every $\epsilon > 0$, there exists
        $\delta > 0$ such that all tagged partitions $\dot{Q}$ of $[a, b]$ with
        $\Vert\dot{Q}\Vert < \delta$ satisfy \[
            |S(f, \dot{Q}) - \ell| < \epsilon.
        \] The number $\ell$ is the value of the Riemann integral, \[
            \int_a^b f = \ell.
        \] 
    \end{definition}

    \begin{theorem}
        If a function is Riemann integrable on an interval, then the value of the
        integral is unique.
    \end{theorem}
    \begin{proof}
        Let $f$ be Riemann integrable on $[a, b]$, with integral values $\ell$ and
        $\ell'$. Then, for every $\epsilon > 0$, we find $\delta > 0$ such that for
        all tagged partitions $\dot{Q}$ with $\Vert\dot{Q}\Vert < \delta$, \[
            |S(f, \dot{Q}) - \ell| < \epsilon, \qquad
            |S(f, \dot{Q}) - \ell'| < \epsilon.
        \] Note that such a partition $\dot{Q}$ always exists. Thus, \[
            |\ell - \ell'| \leq |S(f, \dot{Q}) - \ell| + |S(f, \dot{Q}) - \ell'| <
            2\epsilon
        \] for all $\epsilon > 0$, which forces $\ell = \ell'$.
    \end{proof}

    \begin{definition}[Darboux sums]
        Given a partition $Q$ of $[a, b]$ and a function $f$, define \[
            m_j = \inf_{t \in [x_{j}, x_{j + 1}]} f(t), \qquad
            M_j = \sup_{t \in [x_{j}, x_{j + 1}]} f(t).
        \] The lower and upper Darboux sums are defined as \[
            L(f, Q) = \sum_{j = 0}^{n - 1} m_j(x_{j + 1} - x_j), \qquad
            U(f, Q) = \sum_{j = 0}^{n - 1} M_j(x_{j + 1} - x_j).
        \] 
    \end{definition}

    \begin{definition}[Darboux integrals]
        The lower and upper Darboux integrals of a function $f$ are defined as \[
            L_f = \inf_Q L(f, Q), \qquad U_f = \sup_Q U(f, Q).
        \] Here, the infimum and supremum is taken over all possible partitions $Q$
        of $[a, b]$.

        If $L_f = U_f$, then the common integral is simply called the Darboux
        integral, \[
            \int_a^b f = L_f = U_f.
        \] Such a function $f$ is called Darboux integrable.
    \end{definition}

    \begin{theorem}
        Riemann and Darboux integrability are equivalent and assign the same value
        to the integrals.
    \end{theorem}

\end{document}
% vim: set tabstop=4 shiftwidth=4 softtabstop=4:
