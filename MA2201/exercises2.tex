\documentclass[10pt]{article}

\usepackage[T1]{fontenc}
\usepackage{geometry}
\usepackage{amsmath, amssymb, amsthm}

\title{Analysis II - Assignment II}
\author{Satvik Saha}
\date{}

\geometry{a4paper, margin=1in}
\setlength\parindent{0pt}
\renewcommand{\labelenumi}{(\roman{enumi})}
% \renewcommand\qedsymbol{$\blacksquare$}
\def\C{\mathbb{C}}
\def\R{\mathbb{R}}
\def\Q{\mathbb{Q}}
\def\Z{\mathbb{Z}}
\def\N{\mathbb{N}}

\newtheorem{theorem}{Theorem}
\newtheorem{lemma}[theorem]{Lemma}
\newtheorem{corollary}{Corollary}[theorem]

\begin{document}
    \par\textbf{IISER Kolkata} \hfill \textbf{Assignment II}
    \vspace{3pt}
    \hrule
    \vspace{3pt}
    \begin{center}
            \LARGE{\textbf{MA 2201 : Analysis II}}
    \end{center}
    \vspace{3pt}
    \hrule
    \vspace{3pt}
    Satvik Saha, \texttt{19MS154}\hfill\today
    \vspace{20pt}

    \begin{lemma}
        The series $\sum_{n = 1}^\infty 1 /n^2$ converges to a real number between
        $1$ and $2$.
    \end{lemma}
    \begin{proof}
        Note that the partial sums of the given series are monotonically increasing,
        since all terms are positive. Also,
        \begin{align*}
            S_n = \sum_{k = 1}^n \frac{1}{k^2} &= 1 + \sum_{k = 2}^n
            \frac{1}{k^2} \\
            &< 1 + \sum_{k = 2}^n \frac{1}{k(k - 1)} \\
            &= 1 + \sum_{k = 2}^n \left[ \frac{1}{k - 1} - \frac{1}{k} \right] \\
            &= 1 + 1 - \frac{1}{n} \\
            & < 2.
        \end{align*}
        Thus, the sequence of partial sums of the given series is bounded above,
        which means that it must converge by the monotone convergence theorem.
    \end{proof}

    \begin{lemma}
        If a series $\sum_{n = 1}^\infty \alpha_n$ converges absolutely, then it 
        also converges ordinarily.
    \end{lemma}
    \begin{proof}
        Let $\epsilon > 0$. From the Cauchy criterion and the absolute convergence
        of $\sum_{n = 1}^\infty \alpha_n$, we can choose $N \in \N$
        such that for all $m, n \geq N$, \[
            \left|\sum_{k = 1}^m |\alpha_k| - \sum_{k = 1}^n |\alpha_k|
            \right| = \sum_{k = n + 1}^m |\alpha_k| < \epsilon.
        \] Using the triangle inequality, we have \[
            \left|\sum_{k = n + 1}^m \alpha_k \right| \leq \sum_{k = n + 1}^m
            |\alpha_k| < \epsilon,
        \] which proves that $\sum_{n = 1}^\infty \alpha_n$ converges by the Cauchy
        criterion.
    \end{proof}


    \paragraph{Solution 1.} We have been given the functions \[
        f_n \colon \R \setminus \{-1 /n^2\} \to \R, f_n(x) = \frac{1}{1 +
        n^2x},
    \] and we wish to examine the convergence of the series of functions $\sum_{n =
    1}^\infty f_n$. \\

    \textbf{Pointwise and absolute convergence : }
    Note that when $x = 0$, the pointwise sum is $\sum_{n = 1}^\infty 1$, which
    clearly diverges. When $x > 0$, note that $f_n(x) > 0$ for all $n \in \N$ so the
    sequence of partial sums is monotonically increasing. Also, the series is
    bounded above since \[
        \sum_{k = 1}^n f_k(x) = \sum_{k = 1}^n \frac{1}{1 + k^2x} < \sum_{k = 1}^n
        \frac{1}{k^2x} \leq \frac{2}{x},
    \] so $\sum_{n = 1}^\infty f_n(x)$ converges by the monotone convergence
    theorem. Note that this convergence is absolute, since all terms are positive
    in any case. \\

    Similarly, when $x < -1$, $1 + n^2x < 0$ which means that all $f_n(x)$ are
    negative, so the sequence of partial sums is monotonically decreasing. Now, the
    series is bounded below since \[
        \sum_{k = 1}^n f_k(x) = \sum_{k = 1}^n \frac{1}{1 + k^2x} \geq \sum_{k = 1}^n
        \frac{1}{k^2 + k^2x} \geq \frac{2}{1 + x},
    \] so $\sum_{n = 1}^\infty f_n(x)$ converges by the monotone convergence
    theorem. Note that this convergence is absolute, since all terms are negative in
    any case, hence the absolute sum is simply the negative of the ordinary sum. \\

    Note that the sum $\sum_{n = 1}^\infty f_n(x)$ is not defined when $x \in S =
    \{-1, -1 /2^2, -1 /3^2, \dots\}$. Suppose that $-1 < x < 0$ and $x \notin S$.
    Using the Archimedean property, choose $N \in \N$ such that $N^2 |x| > 2$, which
    means that $1 + n^2x < n^2 x /2 < 0$ for all $n \geq N$. Now for all $n \geq N$,
    we have the tail of the series \[
        \sum_{k = N}^n f_k(x) = \sum_{k = N}^n \frac{1}{1 + k^2x} \geq \sum_{k =
        N}^n \frac{1}{k^2x /2} \geq \frac{4}{x},
    \] so the tail $\sum_{n = N}^\infty f_n(x)$ converges by the monotone
    convergence theorem. Thus, the complete series $\sum_{n = 1}^\infty f_n(x)$ must
    also converge, since the sum of the first $N - 1$ terms is just a finite number.
    Note that we have shown that the tail of the series from the
    $N$\textsuperscript{th} term onwards converges absolutely, since all those terms
    are negative, hence the absolute sum is the negative of the ordinary sum. The
    first $N - 1$ terms are positive, so the complete absolute series including them
    is also convergent. \\

    Thus, setting $S_0 = S \cup \{0\} = \{0, -1, -1 /2^2, -1 /3^2, \dots\}$, we have
    shown that $\sum_{n = 1}^\infty f_n$ is convergent pointwise and absolutely on
    $\R \setminus S_0$. \\

    \textbf{Uniform convergence : } 
    We show that $\sum_{n = 1}^\infty$ converges uniformly on any subset of $\R
    \setminus S_0$ which does not have $0$ as a limit point, i.e.\ the series
    converges uniformly on $(-\infty, a] \cup [b, \infty) \setminus S_0$, where $a <
    -1$, $b > 0$. \\

    First, let $x \in [b, \infty)$ with $b > 0$. Then, \[
        |f_n(x)| = \frac{1}{1 + n^2x} < \frac{1}{n^2x} \leq \frac{1}{n^2b},
    \] hence $\sum_{n = 1}^\infty f_n$ converges uniformly on $[b, \infty)$ by the
    Weierstrass M-test. \\

    Again, if $x \in (\infty, a]$ with $a < 0$, for all $n \geq N$ such that $N^2|a|
    > 2$, we have $1 + n^2x < n^2 x/2 \leq n^2a / 2$, so \[
        |f_n(x)| = \frac{1}{|1 + n^2x|} \leq \frac{2}{n^2|a|}
    \] hence the series converges uniformly by the Weierstrass M-test again. \\
    
    \textbf{Disproof of uniform convergence : } Suppose that the series
    converges uniformly on $x \in (0, \infty)$. This means that for $\epsilon = 1
    /3$, we can choose $N \in \N$ such that for all $n \geq N$, we have \[
        \left|\sum_{n = 1}^{N + 1} f_{n}(x) - \sum_{n = 1}^{N}f_n(x) \right| = |f_{N
        + 1}(x)| < \epsilon = \frac{1}{3}
    \] for all $x \in (0, \infty)$ by the Cauchy criterion. 
    On the other hand, \[
        |f_{N + 1}(1 / (N + 1)^2)| = \frac{1}{2} > \frac{1}{3},
    \] which is a contradiction. \\

    More broadly, suppose that $\sum f_n$ converges uniformly on any set $J_0$ which
    has $0$ as a limit point. Note that $J_0$ cannot contain $0$. This means that we
    can choose $x_0 \in J_0$ such that $|x_0| < 1 /(N + 1)^2$. Now, \[
        |f_{N + 1}(x_0)| = \frac{1}{|1 + (N + 1)^2x_0|} \geq \frac{1}{1 + (N +
        1)^2|x_0|} >
        \frac{1}{2} > \frac{1}{3},
    \] which is again a contradiction. 


    \paragraph{Solution 2.} We have been given functions $f_n$ such that for some $M
    > 0$, \[
        |f_n(x)| \leq M|x|^n
    \] for all $n \in \N$. We wish to examine the convergence of the series $\sum_{n
    = 1}^\infty f_n$. \\

    The series need not converge at all when $|x| \geq 1$. We supply the example
    $f_n(x) = x^n$, $M = 1$. Note that when $|x| \geq 1$, the series $\sum_{n =
    1}^\infty x^n$ does not converge because it fails the Cauchy criterion
    ($n$\textsuperscript{th} term test). We have $|x|^n \to \infty$ when $|x| > 1$
    and $|x|^n \to 1$ when $|x| = 1$, whereas we demand $f_n(x) \to 0$ for the
    series to converge. Thus, we need only look at the interval $(-1, +1)$. \\

    Note that this same counterexample shows that $\sum_{n = 1}^\infty f_n$ need not
    converge uniformly on $(-1, +1)$. If $\sum_{n = 1}^\infty x^n$
    converged uniformly on $(-1, +1)$, then for $\epsilon = 1 /2$, we could choose
    $N \in \N$ such that for $n = N$, $m = N + 1$, \[
        \left|\sum_{n = 1}^{N + 1} f_n(x) - \sum_{n = 1}^N f_n(x)\right| = 
        |f_{N + 1}(x)| = |x|^{N + 1} < \epsilon = \frac{1}{2}
    \] by the Cauchy criterion. On the other hand, note that $x_0 = 
    (1 / 2)^{1 / (N + 1)} \in (-1, +1)$, so we demand \[
        |f_{N + 1}(x_0)| = \frac{1}{2} < \frac{1}{2},
    \] which is absurd. \\

    The series does converge absolutely, hence pointwise on $(-1, +1)$. For fixed $x
    \in (-1, +1)$, the geometric series \[
        \sum_{n = 1}^\infty M|x|^n = \frac{M|x|}{1 - |x|}
    \] converges. Since $0 \leq |f_n(x)| \leq M|x|^n$, the series $\sum_{n =
    1}^\infty |f_n(x)|$ must converge by the comparison test. This means that
    $\sum_{n = 1}^\infty f_n$ converges absolutely on $(-1, +1)$, which in turn
    means that it converges ordinarily (pointwise) on the same. \\

    The series converges pointwise, absolutely, and uniformly on the compact
    interval $[-a, +a]$ where $0 < a < 1$. This is because for $|x| < a < 1$, \[
        |f_n(x)| \leq M|x|^n \leq Ma^n,
    \] so setting $M_n = Ma^n$, the series \[   
        \sum_{n = 1}^\infty  M_n = \sum_{n = 1}^\infty Ma^n = \frac{Ma}{1 - a}
    \] is convergent, which gives the uniform convergence of both $\sum_{n =
    1}^\infty f_n$ and $\sum_{n = 1}^\infty |f_n|$ by the Weierstrass M-test. This
    of course implies the pointwise convergence of the same. \\

\end{document}
