\documentclass[11pt]{article}

\usepackage[T1]{fontenc}
\usepackage{geometry}
\usepackage{amsmath, amssymb, amsthm}
\usepackage[scr]{rsfso}
\usepackage[%
    hidealllines=true,%
    innerbottommargin=15,%
    nobreak=true,%
]{mdframed}
\usepackage{xcolor}
\usepackage{graphicx}
\usepackage{fancyhdr}
\usepackage{lipsum}
\usepackage{hyperref}

\geometry{a4paper, margin=1in, headheight=14pt}

\pagestyle{fancy}
\fancyhf{}
\renewcommand\headrulewidth{0.4pt}
\fancyhead[L]{\scshape MA2201: Analysis II}
\fancyhead[R]{\scshape Differentiation}
\rfoot{\footnotesize\it Updated on \today}
\cfoot{\thepage}

\def\C{\mathbb{C}}
\def\R{\mathbb{R}}
\def\Q{\mathbb{Q}}
\def\Z{\mathbb{Z}}
\def\N{\mathbb{N}}
\newcommand\ddx[1]{\frac{d #1}{d x}}
\newcommand\ddt[1]{\frac{d #1}{d t}}
\newcommand\dd[3][]{\frac{d^{#1}{#2}}{d {#3}^{#1}}}
\newcommand\ppx[1]{\frac{\partial #1}{\partial x}}
\newcommand\ppt[1]{\frac{\partial #1}{\partial t}}
\newcommand\pp[3][]{\frac{\partial^{#1}{#2}}{\partial {#3}^{#1}}}
\newcommand\norm[1]{\left\lVert#1\right\rVert}

\newcounter{module}
\setcounter{module}{2}

\newmdtheoremenv[%
    backgroundcolor=blue!10!white,%
]{theorem}{Theorem}[module]
\newmdtheoremenv[%
    backgroundcolor=violet!10!white,%
]{corollary}{Corollary}[theorem]
\newmdtheoremenv[%
    backgroundcolor=teal!10!white,%
]{lemma}[theorem]{Lemma}

\theoremstyle{definition}
\newmdtheoremenv[%
    backgroundcolor=green!10!white,%
]{definition}{Definition}[module]
\newmdtheoremenv[%
    backgroundcolor=red!10!white,%
]{exercise}{Exercise}[module]

\theoremstyle{remark}
\newtheorem*{remark}{Remark}
\newtheorem*{example}{Example}
\newtheorem*{solution}{Solution}

\surroundwithmdframed[%
    linecolor=black!20!white,%
    hidealllines=false,%
    innertopmargin=5,%
    innerbottommargin=10,%
    skipabove=0,%
    skipbelow=0,%
]{example}

\numberwithin{equation}{module}

\title{
    \Large\textsc{MA2201: Analysis II} \\
    % \vspace{10pt}
    \Huge \textbf{Differentiation} \\
    \vspace{5pt}
    \Large{Spring 2021}
}
\author{
    \large Satvik Saha%
    % \thanks{Email: \tt ss19ms154@iiserkol.ac.in}
    \\\textsc{\small 19MS154}
}
\date{\normalsize
    \textit{Indian Institute of Science Education and Research, Kolkata, \\
    Mohanpur, West Bengal, 741246, India.} \\
    % \vspace{10pt}
    % \today
}

\begin{document}
    \maketitle

    The origins of differential calculus lie in our attempts to approximate various
    functions using linear ones. Suppose that we have been given a curve described
    by the function $f$, and we want to \textit{locally} approximate the function
    around a point $x$ using a straight line. In other words, for a small shift $h$,
    we want to write \[
        f(x + h) \approx f(x) + kh.
    \] Here, $k$ is the slope of the straight line. In order to obtain $k$, we can
    rearrange the above to get \[
        k \approx \frac{f(x + h) - f(x)}{h}.
    \] As we pick smaller and smaller neighbourhoods of $x$, we want our
    approximation to get better and better. Thus, if such an approximation is
    possible, then the value of $k$ must stabilize. This means that the limit \[
        k = \lim_{h \to 0} \frac{f(x + h) - f(x)}{h}
    \] must exist. Note that this immediately forces the continuity of $f$, since \[
        \lim_{h \to 0} f(x + h) - f(x) = \lim_{h \to 0} h \cdot \lim_{h \to 0}
        \frac{f(x + h) - f(x)}{h} = 0k = 0,
    \] whereby $\lim_{x \to a} f(x) = f(a)$. Splitting the limit is justified
    because the individual limits exist.
    If such a limit $k$ exists, we call it the derivative of $f$ at $x$, denoted 
    $f'(x)$.
    % If the derivative of $f$ exists for all $x$ in some interval $(a, b)$, then we
    % say that $f$ is differentiable on $(a, b)$, and $f'\colon (a, b) \to \R$ is
    % called the derivative of $f$ over $(a, b)$.
    We are now able to write \[
        f(x + h) \approx f(x) + f'(x)h.
    \] 

    \begin{definition}[Derivative]
        The derivative of a function $f\colon [a, b] \to \R$ at a point $x \in [a,
        b]$ is defined as \[
            f'(x) = \lim_{h \to 0} \frac{f(x + h) - f(x)}{h},
        \] if such a limit exists. Note that we only demand a one-sided limit if $x$
        is an endpoint.
        If the derivative of $f$ exists at every point in $[a, b]$, we say that $f$
        is differentiable on $[a, b]$.
    \end{definition}
    
    Note that the process we described can be generalised to multivariable
    functions.
    
\end{document}
% vim: set tabstop=4 shiftwidth=4 softtabstop=4:
