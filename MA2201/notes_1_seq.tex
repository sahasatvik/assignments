\documentclass[11pt]{article}

\usepackage[T1]{fontenc}
\usepackage{geometry}
\usepackage{amsmath, amssymb, amsthm}
\usepackage[%
    hidealllines=true,%
    innerbottommargin=15,%
    nobreak=true,%
]{mdframed}
\usepackage{xcolor}
\usepackage{graphicx}
\usepackage{fancyhdr}
\usepackage{lipsum}

\geometry{a4paper, margin=1in, headheight=14pt}

\pagestyle{fancy}
\fancyhf{}
\renewcommand\headrulewidth{0.4pt}
\fancyhead[L]{\scshape MA2201: Analysis II}
\fancyhead[R]{\scshape Sequences of functions}
\rfoot{\footnotesize\it Updated on \today}
\cfoot{\thepage}

\def\C{\mathbb{C}}
\def\R{\mathbb{R}}
\def\Q{\mathbb{Q}}
\def\Z{\mathbb{Z}}
\def\N{\mathbb{N}}
\newcommand\ddx[1]{\frac{d #1}{d x}}
\newcommand\ddt[1]{\frac{d #1}{d t}}
\newcommand\dd[3][]{\frac{d^{#1}{#2}}{d {#3}^{#1}}}
\newcommand\ppx[1]{\frac{\partial #1}{\partial x}}
\newcommand\ppt[1]{\frac{\partial #1}{\partial t}}
\newcommand\pp[3][]{\frac{\partial^{#1}{#2}}{\partial {#3}^{#1}}}
\newcommand\norm[1]{\left\lVert#1\right\rVert}

\newcounter{module}
\setcounter{module}{1}

\newmdtheoremenv[%
    backgroundcolor=blue!10!white,%
]{theorem}{Theorem}[module]
\newmdtheoremenv[%
    backgroundcolor=violet!10!white,%
]{corollary}{Corollary}[theorem]
\newmdtheoremenv[%
    backgroundcolor=teal!10!white,%
]{lemma}[theorem]{Lemma}

\theoremstyle{definition}
\newmdtheoremenv[%
    backgroundcolor=green!10!white,%
]{definition}{Definition}[module]
\newmdtheoremenv[%
    backgroundcolor=red!10!white,%
]{exercise}{Exercise}[module]

\theoremstyle{remark}
\newtheorem*{remark}{Remark}
\newtheorem*{example}{Example}

\surroundwithmdframed[%
    linecolor=black!20!white,%
    hidealllines=false,%
    innertopmargin=5,%
    innerbottommargin=10,%
    skipabove=0,%
    skipbelow=0,%
]{example}

\numberwithin{equation}{module}

\title{
    \Large\textsc{MA2201: Analysis II} \\
    % \vspace{10pt}
    \Huge \textbf{Sequences of functions} \\
    \vspace{5pt}
    \Large{Spring 2021}
}
\author{
    \large Satvik Saha%
    % \thanks{Email: \tt ss19ms154@iiserkol.ac.in}
    \\\textsc{\small 19MS154}
}
\date{\normalsize
    \textit{Indian Institute of Science Education and Research, Kolkata, \\
    Mohanpur, West Bengal, 741246, India.} \\
    % \vspace{10pt}
    % \today
}

\begin{document}
    \maketitle

    \begin{definition}[Sequences of functions]
        Let the functions $f_n\colon X \to Y$ be defined for all $n \in \mathbb{N}$
        and let the sequences $\{f_n(x)\}$ converge for all $x \in X$. Define the
        function $f\colon X \to Y$ as \[
            f(x) = \lim_{n \to \infty} f_n(x) 
        \] for all $x \in X$. We call $f$ the limit of $\{f_n\}$, or say that
        $\{f_n\}$ converges to $f$ pointwise on $X$.
    \end{definition}
    \begin{example}
        Consider the functions $f_n\colon [0, 1] \to \R$, $x \mapsto x^n$.
        It can be shown that $x^n \to 0$ when $x \in [0, 1)$ and 
        $x^n \to 1$ when $x = 1$. Thus, $f = \lim_{n \to \infty} f_n$ is well
        defined. \[
            f(x) = \begin{cases}
                0, &\text{ if } 0 \leq x < 1 \\
                1, &\text{ if } x = 1
            \end{cases}.
        \] 
        Note that while each $f_n$ is continuous in this example, the limit $f$
        is not.
    \end{example}
    \begin{example}
        Consider the functions $f_n\colon \R \to \R$, $x \mapsto x/n$.
        We see that $f_n \to 0$. Note that $0$ here denotes the zero function.
    \end{example}
    \begin{example}
        Consider the functions $f_n\colon \R \to \R$, \[
            f_n(x) = \begin{cases}
                x^2, &\text{ if }|x| \leq n\\
                +n, &\text{ if } x > +n \\
                -n, &\text{ if } x < -n
            \end{cases}.
        \] 
        This converges pointwise to $f\colon \R \to \R$, $x \mapsto x^2$.
        Note that for any $x \in \R$, we can find sufficiently large $N \in \N$ such
        that $|x| \leq N$.  This means that the tail of the sequence $\{f_n(x)\}$
        becomes a constant sequence $\{x^2\}$ from the $N$\textsuperscript{th} term
        onwards, so $f_n(x) \to x^2$ for all $x \in \R$.
    \end{example}
    \begin{example}
        Consider the functions $f_n\colon \R \to \R$, \[
            f_n(x) = \lim_{m \to \infty} (\cos{n!\pi x})^{2m}.
        \] We observe that $f_n(x) = 1$ only when $n!x$ is an integer.
        Now, if $x$ is rational, $n!x$ will become an integer for sufficiently large
        $n$, whereas if $x$ is irrational, $n!x$ can never be an integer. Thus,
        we see that $f_n \to f$, where $f\colon \R \to \R$ is the Dirichlet
        function defined as \[
            f(x) = \begin{cases}
                1, &\text{ if } x \in \Q \\
                0, &\text{ if } x \notin \Q
            \end{cases}.
        \] 
        Note that $f$ is discontinuous everywhere!
    \end{example}
    
    \begin{definition}[Series of functions]
        Let the functions $f_n\colon X \to Y$ be defined for all $n \in \mathbb{N}$
        and let the series $\sum f_n(x)$ converge for all $x \in X$.
        Define the function $f\colon X \to Y$ as \[
            f(x) = \sum_{n = 1}^\infty f_n(x)
        \] for all $x \in X$. We call $f$ the sum of the series $\sum f_n$.
    \end{definition}
    \begin{example}
        Consider the functions $f_n\colon (0, 1) \to \R$, $x \mapsto x^n$.
        Note that the sum \[
            \sum_{n = 1}^{\infty} x^n = x + x^2 + x^3 + \dots 
                = \frac{x}{1 - x} 
        \] does indeed converge for all $x \in (0, 1)$. Thus, the sum $f = \sum f_n$
        is well defined. \[
            f(x) = \frac{x}{1 - x}.
        \] 
    \end{example}

    
    \section{Uniform convergence}
    
    \begin{definition}[Uniform convergence]
        Let the functions $f_n\colon X \to Y$ be defined for all $n \in \N$.
        We say that the sequence $\{f_n\}$ converges uniformly on $X$ to $f$ if
        for every $\epsilon > 0$, there exists $N \in \N$ such that for all $n \geq
        N$ and $x \in X$, we have \[
            |f_n(x) - f(x)| < \epsilon.
        \] 
        \begin{remark}
            Note that for convergence $f_n \to f$, we need only find $N$
            depending on $\epsilon$ and $x$. Uniform convergence requires $N$
            depending on $\epsilon$ which ensures the inequality for \emph{all} 
            $x \in X$.
        \end{remark}
    \end{definition}
    \begin{example}
        Consider $f_n\colon \R \to \R$, $x \mapsto x + 1 /n$. We see that $f_n \to
        f$ uniformly on $\R$, where $f\colon \R \to \R$, $x \mapsto x$.
        Note that given $\epsilon > 0$, we find $N \in \N$ such that $N\epsilon >
        1$ using the Archimedean property. Thus, for all $n \geq N$ and $x \in \R$
        we have \[
            |f_n(x) - f(x)| = \frac{1}{n} \leq \frac{1}{N} < \epsilon.
        \] 
    \end{example}

    \begin{lemma}
        The sequence of functions $\{f_n\}$ does not converge uniformly on $X$ to
        its pointwise limit $f$ if there exists some $\epsilon_0 > 0$, some
        subsequence $\{f_{n_k}\}$ and some sequence $\{x_k\}$ in $X$ such that for
        all $k \in \N$, \[
            |f_{n_k}(x_k) - f(x_k)| \geq \epsilon_0.
        \] 
    \end{lemma}
    \begin{example}
        The sequence of functions $\{f_n\}$ where $f_n\colon [0, 1] \to \R$, $x
        \mapsto x^n$ does not converge uniformly on $[0, 1]$.
        We have already described $f = \lim_{n \to \infty} f_n$.
        Set $\epsilon_0 = 1 /2$, $x_k = (1 /2)^{1 /k}$ and $n_k = k$. Thus, \[
            |f_{n_k}(x_k) - f(x_k)| = \frac{1}{2} \geq \epsilon_0.
        \]
        Note that $x_k \to 1$, which is the point of discontinuity of $f$.
    \end{example}
    \begin{example}
        The sequence of functions $\{f_n\}$ where $f_n\colon \R \to \R$, $x \mapsto
        x /n$ does not converge uniformly on $\R$.
        Recall that $f_n \to 0$, but when $\epsilon_0 = 1$, $n_k = x_k = k$, we have
        \[
            |f_{n_k}(x_k) - f(x_k)| = 1 \geq \epsilon_0.
        \] 
    \end{example}

    \begin{theorem}[Cauchy criterion for uniform convergence]
        The sequence of real-valued functions $\{f_n\}$ converges uniformly on $X$
        if and only if for every $\epsilon > 0$, there exists $N \in \N$ such that
        for all $m, n \geq N$ and $x \in X$, we have \[
            |f_n(x) - f_m(x)| < \epsilon.
        \] 
    \end{theorem}
    \begin{proof}
        First suppose that $\{f_n\}$ converges uniformly on $X$, and $f_n \to f$.
        This means that given $\epsilon > 0$, we can choose $N \in \N$ such that \[
            |f_n(x) - f(x)| < \frac{\epsilon}{2}
        \] for all $n \geq N$, $x \in X$.
        Thus, for all $m, n \geq N$ and $x \in X$, we have \[
            |f_n(x) - f_m(x)| \leq |f_n(x) - f(x)| + |f_m(x) - f(x)| <
            \frac{\epsilon}{2} + \frac{\epsilon}{2} = \epsilon,
        \] as desired.

        Now suppose that the Cauchy criterion holds. Given $\epsilon > 0$,
        there exists $N \in \N$ such that for all $m, n \geq N$ and $x \in X$, \[
            |f_n(x) - f_m(x)| < \epsilon.
        \]
        Recall that the Cauchy criterion for real sequences guarantees that the
        sequence $\{f_n(x)\}$ converges, thus the function
        $f = \lim_{n \to \infty} f_n$ is well defined.
        To show that the convergence of $f_n \to f$ is uniform, fix $n$ and let $m
        \to \infty$, so $f_m(x) \to f(x)$. Thus for all $n \geq N$ and $x \in X$, \[
            |f_n(x) - f(x)| < \epsilon,
        \] as desired.
    \end{proof}

    \begin{theorem}
        Let $f_n\colon X \to \R$ and let $f_n \to f$. Set \[
            M_n = \sup |f_n(x) - f(x)|.
        \]
        Then, $\{f_n\}$ converges uniformly on $X$ to $f$ if and only if $M_n \to
        0$.
    \end{theorem}
    \begin{proof}
        Suppose that $f_n \to f$ uniformly on $X$.
        Let $\epsilon > 0$ be arbitrary, and let $N \in \N$ be such that
        for all $n \geq N$ and $x \in X$, \[
            |f_n(x) - f(x)| < \frac{\epsilon}{2}.
        \]
        This means that for all $n \geq N$, \[
            M_n = \sup |f_n(x_n) - f(x_n)| \leq \frac{\epsilon}{2} < \epsilon.
        \]
        Also note that all $M_n \geq 0$, since they are the supremums of non-negative
        quantities.
        This means that $M_n \to 0$, as desired.

        Now suppose that $M_n \to 0$. This means that for arbitrary $\epsilon > 0$,
        we can find $N \in \N$ such that for all $n \geq N$, we have \[
            |M_n| = \sup |f_n(x) - f(x)| < \epsilon.
        \] 
        Now, from the properties of the supremum, we see that for all $n \geq N$ and
        $x \in X$, \[
            |f_n(x) - f(x)| \leq \sup |f_n(x) - f(x)| < \epsilon.
        \] 
        This proves that $f_n \to f$ uniformly.
    \end{proof} 
    \begin{example}
        Consider $f_n\colon [0, 1 /2] \to \R$, $x \mapsto x^n$. We see that $f_n \to
        0$, and that \[
            M_n = \sup|f_n(x) - f(x)| = \frac{1}{2^n} \to 0.
        \] 
        Thus, $\{f_n\}$ converges uniformly on $[0, 1 /2]$ to $0$.
    \end{example}

    \begin{theorem}[Weierstrass M-test]
        Let $f_n\colon X \to \R$ and suppose that for all $n \in \N$ and $x \in X$,
        \[
            |f_n(x)| \leq M_n.
        \]
        Then the series $\sum f_n$ converges uniformly on $X$ if $\sum M_n$
        converges.
    \end{theorem}
    \begin{proof}
        Let $\epsilon > 0$. Since $\sum M_n$ converges, we can use the Cauchy
        criterion for the convergence of real series to choose $N \in \N$ such that
        for all $m \geq n \geq N$, \[
            \left|\sum_{k = n}^m f_k(x)\right| \leq \sum_{k = n}^m M_k \leq \epsilon
        \] for all $x \in X$.
        Note that the left hand side is simply $|s_m(x) - s_{n - 1}(x)|$ where
        $s_k(x)$ is the $k$\textsuperscript{th} partial sum of the series $\sum
        f_n(x)$. Thus, the Cauchy criterion gives the uniform convergence of 
        $\{s_n\}$, hence the uniform convergence of the series $\sum f_n$.
    \end{proof}
    \begin{remark}
        The converse is not true. Simply setting $f_n = 0$, we observe that the
        series $\sum f_n$ converges uniformly on $\R$ to $0$. On the other hand,
        $|f_n(x)| \leq 1$ for all $x \in \R$, and the series $\sum 1$ diverges to
        $\infty$.
    \end{remark}

    \section{Continuity}
    \begin{theorem}
        Let the functions $f_n\colon X \to Y$ be continuous, and suppose that 
        $f_n \to f$ uniformly on $X$ in a metric space. Then, $f$ is continuous on
        $X$.
    \end{theorem}
    \begin{proof}
        Let $\epsilon > 0$. We wish to show that $f$ is continuous at 
        arbitrary $x_0 \in X$.
        
        Since $f_n \to f$ uniformly on $X$, we find $N \in \N$ such that for all $x
        \in X$ and $n \geq N$, we have \[
            |f_n(x) - f(x)| < \frac{\epsilon}{3}.
        \] In particular, this holds for $n = N$, and $x = x_0$.
        
        The continuity of each $f_n$ on $X$ means $f_N$ is
        continuous on $X$ in particular, so we can find $\delta > 0$ such that 
        whenever $|x - x_0| < \delta$, we have \[
            |f_N(x) - f_N(x_0)| < \frac{\epsilon}{3}.
        \]

        Putting these together, for every $x \in X$ such that $|x - x_0| < \delta$,
        we have
        \begin{align*}
            |f(x) - f(x_0)| 
                \,&\leq\, |f(x) - f_N(x)| + |f_N(x) - f_N(x_0)| + 
                    |f_N(x_0) - f(x_0)| \\
                \,&<\, \frac{\epsilon}{3} + \frac{\epsilon}{3} +
                    \frac{\epsilon}{3} \\
                \,&=\, \epsilon.
        \end{align*}
        This means that $f$ is continuous at $x_0$ for arbitrary $x_0 \in X$, i.e.\
        $f$ is continuous on $X$.
    \end{proof}
    \begin{corollary}
        Let the functions $f_n\colon X \to Y$ be continuous, and let $f_n \to f$
        pointwise on $X$. If $f$ is not continuous on $X$, then the
        sequence of functions $\{f_n\}$ does not converge uniformly on $X$.
    \end{corollary}
    \begin{proof}
        This is simply the contrapositive of the given theorem.
    \end{proof}
    \begin{example}
        The functions $f_n\colon [0, 1] \to \R$, $x \mapsto x^n$ do not converge
        uniformly on $[0, 1]$ because each $f_n$ is continuous, but their limit 
        $\lim_{n \to \infty} f_n$ is discontinuous at $x = 1$.
    \end{example}
    
\end{document}
% vim: set tabstop=4 shiftwidth=4 softtabstop=4:
