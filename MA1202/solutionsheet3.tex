\documentclass[10pt]{article}

\usepackage[T1]{fontenc}
\usepackage{geometry}
\usepackage{amsmath, amssymb, amsthm}
% \usepackage{physics}
% \usepackage{esint}
\DeclareMathOperator{\Real}{Re}

\title{Mathematical Methods I - Assignment III}
\author{Satvik Saha}
\date{}

\geometry{a4paper, margin=1in}
\setlength\parindent{0pt}
\renewcommand{\labelenumi}{(\roman{enumi})}
\renewcommand{\thefootnote}{\fnsymbol{footnote}}
% \renewcommand\qedsymbol{$\blacksquare$}

\newcounter{prob}
\def\problem{\stepcounter{prob}\paragraph{Problem \arabic{prob}}}
\def\solution{\paragraph{Solution}}

\begin{document}
        \par\textbf{IISER Kolkata} \hfill \textbf{Assignment III}
        \vspace{3pt}
        \hrule
        \vspace{3pt}
        \begin{center}
                \LARGE{\textbf{MA 1202 : Mathematical Methods I}}
        \end{center}
        \vspace{3pt}
        \hrule
        \vspace{3pt}
        Satvik Saha, \texttt{19MS154}\hfill\today
        \vspace{20pt}

        \problem Check whether the function $f$, defined by 
        \[
                f(x, y) \;=\; \frac{x - 1 - iy}{(x - 1)^2 + y^2},
        \]
        is analytic by the following methods.
        \begin{enumerate}
                \item Using the Cauchy-Riemann equations.
                \item Expressing $f$ in the form $f(x, y) \equiv g(z, \bar{z})$.
        \end{enumerate}
        \textit{Here, $x = \Re(z)$ and $y = \Im(z)$ for $z \in \mathbb{C}$.} 
        \solution Note that using the identity $a^2 + b^2 = (a + ib)(a - ib)$, we have
        \[
        f(x, y) \;=\; \frac{1}{x - 1 + iy},
        \]
        which is not defined at $z = 1$, i.e.\ $(x, y) = (1, 0)$.
        \begin{enumerate}
                \item We write
                \[
                f(x, y) \;=\; u(x, y) \,+\, iv(x, y),
                \]
                where
                \[
                u(x, y) \;=\; \frac{x - 1}{(x - 1)^2 + y^2}, \quad\text{ and }\quad v(x, y) \;=\; \frac{-y}{(x - 1)^2 + y^2} .
                \]
                Note that both $u$ and $v$ are continuous except at $z = 1$, where they are both undefined. Otherwise, on $z \in \mathbb{C}\setminus\{1\}$,
                we demand
                \[
                \frac{\partial u}{\partial x} \;=\; \frac{\partial v}{\partial y}, \quad\text{ and }\quad
                \frac{\partial u}{\partial y} \;=\; -\frac{\partial v}{\partial x}.
                \]
                Indeed, notating $h_X \equiv {\partial h}/{\partial X}$, we have
                \begin{align*}
                u_x \;&=\; \frac{[(x - 1)^2 + y^2] - [2(x - 1)(x - 1)]}{[(x - 1)^2 + y^2]^2} \;=\; \frac{-(x-1)^2 + y^2}{[(x - 1)^2 + y^2]^2} \\
                v_y \;&=\; \frac{-[(x-1)^2 + y^2] - [-y(2y)]}{[(x-1)^2 + y^2]^2} \;=\; \frac{-(x-1)^2 + y^2}{[(x-1)^2 + y^2]^2} \\\\ 
                u_y \;&=\; \frac{0 - [(x - 1)(2y)]}{[(x - 1)^2 + y^2]^2} \;=\; \frac{-2(x - 1)y}{[(x - 1)^2 + y^2]^2} \\
                v_x \;&=\; \frac{0 - [-2y(x - 1)]}{[(x - 1)^2 + y^2]^2}  \;=\; \frac{2(x - 1)y}{[(x - 1)^2 + y^2]^2} 
                \end{align*}
                Thus, $u_x = v_y$ and $u_y = -v_x$ for all $z \in \mathbb{C}\setminus\{1\}$. Hence, $f$ is analytic on $\mathbb{C}\setminus\{1\}$.

                \item Writing $z = x + iy$, we have
                \[
                f(z) \;=\; \frac{\bar{z} - 1}{(z - 1)(\bar{z} - 1)} \;=\; \frac{1}{z - 1}.
                \]
                We see that $f \equiv g(z, \bar{z})$ is free of the second complex variable $\bar{z}$, so $f_{\bar{z}} = 0$. Hence,
                $f$ is analytic on $\mathbb{C}\setminus\{1\}$.
        \end{enumerate}

        \problem Compute the contour integral
        \[
                \oint_C \frac{z \exp{z}}{z^2 + 1} \:\mathrm{d}z,
        \]
        where $C$ is a circle of radius $2$, centered at $0$, and oriented counterclockwise.
        \solution We carry out the partial fraction decomposition
        \[
        \frac{z}{z^2 + 1} \;=\; \frac{1}{2}\,\frac{(z + i) + (z - i)}{(z + i)(z - i)} \;=\; \frac{1}{2}\left[\frac{1}{z - i} \,+\, \frac{1}{z + i}\right].
        \]

        Thus, we split our integral
        \[
        I \;=\; \oint_C \frac{z \exp{z}}{z^2 + 1} \:\mathrm{d}z \;=\; \frac{1}{2}\oint_C \frac{\exp{z}}{z - i} \:\mathrm{d}z 
                \,+\, \frac{1}{2}\oint_C \frac{\exp{z}}{z + i} \:\mathrm{d}z \;=\; \frac{1}{2}I_1 \,+\, \frac{1}{2}I_2.
        \]

        Note that $\exp(z)$ is analytic, and $C$ contains both $i$ and $-i$. Thus, Cauchy's Integral Formula yields
        \[
        I_1 \;=\; 2\pi i \exp(i), \quad\quad I_2 \;=\; 2\pi i \exp(-i).
        \]

        Using $e^{i\varphi} + e^{-i\varphi} = 2\cos{\varphi}$ yields 
        \[
        I \;=\; 2 \pi i \cos{1}.
        \]
        
\end{document}
