\documentclass[10pt]{article}

\usepackage[T1]{fontenc}
\usepackage{geometry}
\usepackage{amsmath, amssymb, amsthm}
% \usepackage{physics}
\DeclareMathOperator{\Real}{Re}

\title{Mathematical Methods I - Assignment II}
\author{Satvik Saha}
\date{}

\geometry{a4paper, margin=1in}
\setlength\parindent{0pt}
\renewcommand{\labelenumi}{(\roman{enumi})}
\renewcommand{\thefootnote}{\fnsymbol{footnote}}
% \renewcommand\qedsymbol{$\blacksquare$}

\newcounter{prob}
\def\problem{\stepcounter{prob}\paragraph{Problem \arabic{prob}}}
\def\solution{\paragraph{Solution}}

\begin{document}
        \par\textbf{IISER Kolkata} \hfill \textbf{Assignment II}
        \vspace{3pt}
        \hrule
        \vspace{3pt}
        \begin{center}
                \LARGE{\textbf{MA 1202 : Mathematical Methods I}}
        \end{center}
        \vspace{3pt}
        \hrule
        \vspace{3pt}
        Satvik Saha, \texttt{19MS154}\hfill\today
        \vspace{20pt}
        
        \problem Determine the first three terms of each fundamental solution of the ODE
        \[
                y'' \,-\, xy \;=\; 0,
        \]
        by assuming a power series expansion around the point $x_0 = 2$.

        \solution
        Note that $x_0 = 2$ is a regular point.

        We propose a solution of the form
        \[
        y(x) \;=\; \sum_{n = 0}^\infty a_n (x - 2)^n.
        \]
        Hence,
        \[
        y'(x) \;=\; \sum_{n = 1}^\infty n a_n (x - 2)^{n - 1}, \quad\quad y''(x) \;=\; \sum_{n = 2}^\infty n(n - 1)a_n (x - 2)^{n - 2}.
        \]
        Plugging this into the ODE,
        \[
        \sum_{n = 2}^\infty n(n - 1)a_n (x - 2)^{n - 2} \;-\; x\sum_{n = 0}^\infty a_n (x-2)^n \;=\; 0.
        \]
        After some manipulation,
        \[
        \sum_{n = 2}^\infty n(n - 1)a_n (x - 2)^{n - 2} \;-\; (x - 2 + 2)\sum_{n = 0}^\infty a_n (x-2)^n \;=\; 0,
        \]
        \[
        \sum_{n = 2}^\infty n(n - 1)a_n (x - 2)^{n - 2} \;-\; \sum_{n = 0}^\infty a_n (x-2)^{n + 1}  \,-\, 2\sum_{n = 0}^\infty a_n(x-2)^n\;=\; 0.
        \]
        Shifting indices,
        \[
        \sum_{n = 0}^\infty (n + 2)(n + 1) a_{n + 2}(x-2)^n \;-\; \sum_{n = 1}^\infty a_{n - 1}(x-2)^n \,-\, 2\sum_{n = 0}^\infty a_n(x-2)^n \;=\; 0.
        \]
        \[
        2a_2 \,-\, 2a_0 \,+\, \sum_{n = 1}^\infty \left[(n + 2)(n + 1) a_{n + 2} \;-\; a_{n - 1} \,-\, 2a_n \right](x-2)^n \;=\; 0.
        \]

        Setting coefficients to zero, we have
        \begin{align*}
                2a_2 \,-\, 2a_0 \;&=\; 0       & n &= 0 \\
                (n+2)(n+1)a_{n + 2} - a_{n - 1} - 2a_n \;&=\; 0         & n &= 1, 2, 3, \dots
        \end{align*}
        Thus,
        \begin{align*}
                a_0 \;&=\; a_2          &n&=0\\
                6a_3 \;&=\; a_0 + 2a_1      &n& = 1\\
                12a_4 \;&=\; a_1 + 2a_2     &n& = 2
        \end{align*}
        Thus,
        \begin{align*}
                y(x) \;&=\; a_0 \,+\, a_1(x-2) \,+\, a_2(x-2)^2 \,+\, a_3(x-2)^3 \,+\, a_4(x-2)^4 \,+\, \dots\\
                        \;&=\; a_0 + a_1(x-2) \,+\, a_0(x-2)^2 \,+\, \left( \frac{a_0}{6} + \frac{a_1}{3}\right)(x-2)^3 \,+\, 
                                \left( \frac{a_1}{12} + \frac{a_0}{6} \right)(x-2)^4 \,+\, \dots \\
                        \;&=\; a_0 \left[1 + (x - 2)^2 + \frac{1}{6}(x-2)^3 + \frac{1}{6}(x-2)^4 + \dots\right] + 
                                a_1\left[(x-2) + \frac{1}{3}(x-2)^3 + \frac{1}{12}(x-2)^4 + \dots\right].
        \end{align*}

        Hence, our fundamental solutions are
        \begin{align*}
                y_1(x) \;&=\; 1 \,+\, (x-2)^2 \,+\, \frac{1}{6}(x-2)^3 \,+\, \dots\\
                y_2(x) \;&=\; (x-2) \,+\, \frac{1}{3}(x-2)^3 \,+\, \frac{1}{12}(x-2)^4 \,+\, \dots
        \end{align*}

        
        \problem Solve the indicial equation, assuming a power series expansion near the regular singulat point $x_0=0$, obtained from the ODE
        \[
                x^2 y'' \,+\, \left(x^2 + \frac{5}{36}\right)y \;=\; 0.
        \]

        Find the first three terms for the power series expansion of the fundamental solution corresponding to the largest root of the 
        indicial equation.

        \solution
        We note that the limit
        \[
        \lim_{x \to 0} x^2 \cdot\frac{x^2 + \frac{5}{36}}{x^2} \;=\; \frac{5}{36}
        \]
        is finite.

        We propose a solution of the form
        \[
        y(x) \;=\; x^r\sum_{n = 0}^\infty a_n x^n.
        \]
        Hence,
        \[
        y'(x) \;=\; \sum_{n = 0}^\infty (n + r) a_n x^{n  + r - 1}, \quad\quad y''(x) \;=\; \sum_{n = 0}^\infty (n + r)(n + r - 1)a_n x^{n + r - 2}.
        \]
        Plugging this into our equation, we have
        \[
        \sum_{n = 0}^{\infty}(n + r)(n + r - 1)a_n x^{n + r} \,+\, \sum_{n = 0}^\infty a_n x^{n + r + 2} \,+\,
        \frac{5}{36}\sum_{n = 0}^\infty a_n x^{n + r} \;=\; 0.
        \]
        Shifting indices and rearranging,
        \[
        \sum_{n = 0}^\infty (n + r)(n + r - 1)a_n x^{n + r} \,+\, \sum_{n = 2}^\infty a_{n - 2} x^{n + r} \,+\, \frac{5}{36}\sum_{n = 0}^\infty a_n x^{n + r} \;=\;0.
        \]
        \[
        \left[r(r-1)a_0 + (r+1)r a_1x + \frac{5}{36}a_0 + \frac{5}{36}a_1x\right]x^r + \sum_{n = 2}^\infty \left[ (n + r)(n + r - 1)a_n  + a_{n - 2} + \frac{5}{36}a_n \right] x^{n + r} \;=\;0.
        \]
        Our indicial equation thus is
        \[
        r(r-1) \,+\, \frac{5}{36} \;=\; 0.
        \]
        This has roots
        \[
        r_+ \;=\; \frac{5}{6}\, \quad\quad r_- \;=\; \frac{1}{6}.
        \]
        We choose $r = r_+ = 5 / 6$. Setting coefficients to zero, we have the relations
        \begin{align*}
                55a_1 \,+\, 5a_1 \;&=\; 0,       &&\\
                (6n + 5)(6n - 1)a_n \,+\, 36 a_{n-2} \,+\, 5a_n \;&=\; 0, &&n \geq 2.
        \end{align*}
        Hence,
        \begin{align*}
                55a_1 + 5a_1 \;=\; 0 \implies a_1 \;&=\; 0 \\
                187a_2 + 36a_0 + 5a_2 \;=\; 0 \implies a_2 \;&=\; \frac{3}{16}a_0\\
                391a_3 + 36a_1 + 5a_3 \;=\; 0 \implies a_3 \;&=\; 0\\
                667a_4 + 36a_2 + 5a_4 \;=\; 0 \implies a_4 \;&=\; \frac{3}{56}a_2 \;=\; \frac{9}{896}a_0
        \end{align*}
        More generally, $a_{n} = 3a_{n-2}/(3n^2 + 2n)$ for all $n \geq 2$, so all odd terms vanish.

        Hence, the corresponding solution is 
        \begin{align*}
                y(x) \;&=\; x^{r_+}(a_0 + a_1x + a_2x^2 + a_3 x^3 + a_4x^2 + \dots) \\
                        \;&=\; a_0 x^{5 /6}\left(1 - \frac{3}{16}x^2 + \frac{9}{896} x^4 + \dots\right).
        \end{align*}
        The fundamental solution is thus
        \[
                y_+(x) \;=\; x^{5 /6}\left(1 - \frac{3}{16}x^2 + \frac{9}{896} x^4 + \dots\right).
        \]
\end{document}
