\documentclass[11pt]{article}

\usepackage[T1]{fontenc}
\usepackage{geometry}
\usepackage{amsmath, amssymb, amsthm}
\usepackage[%
    hidealllines=true,%
    innerbottommargin=15,%
]{mdframed}
\usepackage{xcolor}
\usepackage{graphicx}
\usepackage{fancyhdr}
\usepackage{lipsum}
\usepackage{bm}
\usepackage{braket}
\usepackage{siunitx}

\geometry{a4paper, margin=1in, headheight=14pt}

\pagestyle{fancy}
\fancyhf{}
\renewcommand\headrulewidth{0.4pt}
\fancyhead[L]{\scshape PH2201: Basic Quantum Mechanics}
\fancyhead[R]{\scshape \leftmark}
\rfoot{\footnotesize\it Updated on \today}
\cfoot{\thepage}

\def\C{\mathbb{C}}
\def\R{\mathbb{R}}
\def\Q{\mathbb{Q}}
\def\Z{\mathbb{Z}}
\def\N{\mathbb{N}}
\newcommand\ve[1]{\boldsymbol{#1}}
\newcommand\ddx[1]{\frac{d #1}{d x}}
\newcommand\ddt[1]{\frac{d #1}{d t}}
\newcommand\dd[3][]{\frac{d^{#1}{#2}}{d {#3}^{#1}}}
\newcommand\ppx[1]{\frac{\partial #1}{\partial x}}
\newcommand\ppt[1]{\frac{\partial #1}{\partial t}}
\newcommand\pp[3][]{\frac{\partial^{#1}{#2}}{\partial {#3}^{#1}}}
\newcommand\norm[1]{\left\lVert#1\right\rVert}
\newcommand\grad[1]{\ve{\nabla}#1}
\newcommand\divg[1]{\ve{\nabla}\cdot#1}
\newcommand\curl[1]{\ve{\nabla}\times#1}
\newcommand\lapl[1]{\nabla^2 #1}

\newmdtheoremenv[%
    backgroundcolor=blue!10!white,%
]{theorem}{Proposition}[section]
% \newmdtheoremenv[%
%     backgroundcolor=violet!10!white,%
% ]{corollary}{Corollary}[theorem]
% \newmdtheoremenv[%
%     backgroundcolor=teal!10!white,%
% ]{lemma}[theorem]{Lemma}

\theoremstyle{definition}
\newmdtheoremenv[%
    backgroundcolor=green!10!white,%
]{definition}{Definition}[section]
\newmdtheoremenv[%
    backgroundcolor=red!10!white,%
]{exercise}{Exercise}[section]
\newmdenv[%
    backgroundcolor=cyan!5!white,%
    innertopmargin=10,%
]{cyanbox}
\newenvironment{boxedeq}%
    {\begin{cyanbox}\begin{equation}}%
    {\end{equation}\end{cyanbox}}
\newenvironment{boxedeq*}%
    {\begin{cyanbox}\begin{equation*}}%
    {\end{equation*}\end{cyanbox}}

\theoremstyle{remark}
\newtheorem*{remark}{Remark}
\newtheorem*{example}{Example}

\numberwithin{equation}{section}

\title{
    \Large\textsc{PH2201} \\
    % \vspace{10pt}
    \Huge \textbf{Basic Quantum Mechanics} \\
    \vspace{5pt}
    \Large{Spring 2021}
}
\author{
    \large Satvik Saha%
    % \thanks{Email: \tt ss19ms154@iiserkol.ac.in}
    \\\textsc{\small 19MS154}
}
\date{\normalsize
    \textit{Indian Institute of Science Education and Research, Kolkata, \\
    Mohanpur, West Bengal, 741246, India.} \\
    % \vspace{10pt}
    % \today
}

\begin{document}
    \maketitle

    \section{Introduction}
    The quantum world differs from the classical world in many aspects, most of
    which we seldom encounter in our daily lives and are hence unintuitive.
    \begin{itemize}
        \item The physical world is not deterministic; uncertainty is
        intrinsic to the quantum world.
        This is sometimes illustrated by the Schr\"odinger's cat thought
        experiment.
        \item Both light and matter exhibit characteristics of waves as well
        as those of particles.
        However, a single object cannot exhibit both of these properties 
        simultaneously.
        \item Physical quantities may be quantized -- they may be
        constrained to have discrete values rather than vary continuously.
    \end{itemize}

    \subsection{Blackbody radiation}
    A blackbody is an object which absorbs all radiation incident on it, and
    reflects none. It also emits radiation of all frequencies.

    Kirchoff's Law says that the rates of emission and absorption of radiation of a
    body in thermal equilibrium will be equal. By thermal equilibrium, we mean that
    the temperatures of the body and its surroundings are equal.

    The power emitted by a blackbody is given by the Stefan-Boltzmann Law.
    \begin{boxedeq}
        P \,=\, \sigma A T^4.
    \end{boxedeq}
    Here, $\sigma \approx \SI{5.67e-8}{\joule\second^{-1}\meter^{-2}\kelvin^{-4}}$
    is called the Stefan-Boltzmann constant.

    We may break down the total energy density $\rho \propto T^4$ in terms of the
    contributions from each frequency, so \[
        \rho = \int_0^\infty \rho(\nu) \:d\nu.
    \] 
    It turns out that $\rho(\nu)$ is non-monotonic. This cannot be explained by
    classical mechanics (Rayleigh-Jean's Law), which predicts that $\rho(\nu)$
    is unbounded with increasing frequency -- the famous ultraviolet catastrophe.

    Wien's Law describes the positions of the peaks in $\rho(\nu)$.
    \begin{boxedeq}
        \lambda_{peak} \,=\, \frac{w}{T}.
    \end{boxedeq}
    Here, $w \approx \SI{2.9e-3}{\milli\kelvin}$.
    Note that at $T \approx \SI{300}{\kelvin}$, the peak wavelength $\lambda_{peak}$
    is in the infrared range: this is why night vision googles are useful.
    
    Consider a collection of electromagnetic waves in a blackbody cavity, with
    temperature $T$. This can be seen as the superposition of normal modes.
    The classical approach to the blackbody problem is to suppose that the energy
    density at a particular frequency is proportional to the frequency. \[
        \rho(\nu) = \overline{E} n(\nu),
    \] where $n(\nu)$ is the number density of wave modes with frequency $\nu$, and
    $E$ is the average energy of the radiation.

    The classical law of equipartition of energy gives \[
        \overline{E} = k_B T,
    \] where $k_B$ is the Boltzmann constant.

    The wavenumber of for modes within the cavity is given by \[
        \ve{k} = \frac{2\pi}{L}\ve{n},
    \] where $\ve{n} = (n_x, n_y, n_z)$ with integral components. Now, \[
        \nu = \frac{c}{\lambda} = \frac{c}{L}n.
    \] Treating $n$ as a continuous variable and using $dV = 4\pi n^:dn$, we write
    \[
        n(\nu)\:d\nu = \frac{8\pi}{c^3}\nu^2\:d\nu.
    \] This leads to the Rayleigh-Jean Law,
    \begin{boxedeq}
        \rho(\nu)\:d\nu = \overline{E} n(\nu) \:d\nu 
            = \frac{8\pi k_B T}{c^3}\nu^2\:d\nu.
    \end{boxedeq}

    Planck's considered the probability distribution for the energy, \[
        P(E) = \frac{1}{k_B T}e^{-E /k_B T}.
    \] This is the Boltzmann distribution. It can be shown that \[
        \overline{E} = \frac{\int E P(E) \:dE}{\int P(E) \: dE} = k_B T,
    \] which recovers the Rayleigh-Jean Law.

    Planck's idea was to restrict $E$ to discrete values; integral multiples of the
    frequency $\nu$. This leads to \[
        \overline{E} = \frac{\sum E P(E)}{\sum P(E)} 
            = \frac{h\vu}{e^{nh\nu /k_B T} - 1}.
    \] 
    This gives us the Planck distribution,
    \begin{boxedeq}
        \rho(\nu)\:d\nu 
            = \frac{8\pi h}{c^3} \frac{\nu^3}{e^{nh\nu /k_B T} - 1} \:d\nu
    \end{boxedeq}

    When $h\nu \ll 1$, we recover the Rayleigh-Jean limit. When $h\nu \gg 1$, we get
    the Wien limit.

    Now we calculate, \[
        \rho = \int_0^\infty \rho(\nu)\:d\nu = \frac{8\pi^5k_B^4}{15c^3h^3}T^4,
    \] which recovers the Stefan-Boltzmann Law with \[
        \sigma \,=\, \frac{2\pi^4k_B^4}{15c^2h^3}.
    \] 
    Also, the maxima of the Planck distribution recovers Wien's Law, with \[
        \nu_{max} \,\approx\, 2.8 \frac{k_B T}{h}.
    \] 
    We have introduced Planck's constant, which is given by $h \approx
    \SI{6.626e-34}{\joule\second}$.


\end{document}
% vim: set tabstop=4 shiftwidth=4 softtabstop=4:
