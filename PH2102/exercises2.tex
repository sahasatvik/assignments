\documentclass[10pt]{article}

\usepackage[T1]{fontenc}
\usepackage{geometry}
\usepackage{amsmath, amssymb, amsthm}
\usepackage{bm}
\usepackage{cancel}
\usepackage{xcolor}

\title{Mechanics II - Exercises}
\author{Satvik Saha}
\date{}

\geometry{a4paper, margin=1in}
\setlength\parindent{0pt}
\renewcommand{\labelenumi}{(\roman{enumi})}
\renewcommand\CancelColor{\color{red}}
% \renewcommand\qedsymbol{$\blacksquare$}
\let\vec\boldsymbol
\newcommand{\uveci}{{\bm{\hat{\textnormal{\bfseries\i}}}}}
\newcommand{\uvecj}{{\bm{\hat{\textnormal{\bfseries\j}}}}}
\newcommand{\uveck}{{\bm{\hat{\textnormal{\bfseries k}}}}}
\newcommand\norm[1]{\left\lVert#1\right\rVert}
\newcommand\grad[1]{\vec{\nabla}#1}
\newcommand\divg[1]{\vec{\nabla}\cdot#1}
\newcommand\curl[1]{\vec{\nabla}\times#1}
\newcommand\pp[2]{\frac{\partial #1}{\partial #2}}

\newcounter{prob}
\def\problem{\stepcounter{prob}\paragraph{Exercise \arabic{prob}}}
\def\solution{\paragraph{Solution}}
\setcounter{prob}{2}

\begin{document}
        \par\textbf{IISER Kolkata} \hfill \textbf{Exercises}
        \vspace{3pt}
        \hrule
        \vspace{3pt}
        \begin{center}
                \LARGE{\textbf{PH 2102 : Mechanics II}}
        \end{center}
        \vspace{3pt}
        \hrule
        \vspace{3pt}
        Satvik Saha, \texttt{19MS154}, Group E\hfill\today
        \vspace{20pt}

        \problem A rod of length $L_0$ moves with a speed $v$ along the $x$-axis, making an angle $\theta_0$ to the axis. Find (a) the
        length of the rod with respect to a stationary observer and (b) the angle relative to a stationary $x$-axis.

        \solution We let $S(x, y)$ be the stationary frame and let $S'(x', y')$ be the frame of the moving rod.
        Note that the frame $S'$ moves with a velocity $\vec{v} = v\uveci$ with respect to frame $S$.
        Suppose that in $S'$, one end of the rod is fixed at the origin and the other is at coordinates $(x_0', y_0')$.
        We infer that $L_0 = \sqrt{x_0'^2 + y_0'^2}$. Now, we can apply the Lorentz transformations for the spatial coordinates
        \[
                x' = \gamma(x - vt), \qquad y' = y, \qquad z' = z.
        \]
        Note that these follow since the relative motion between $S$ and $S'$ is purely directed along the $x$-axis.
        Thus, at some given time $t$, the coordinates of $(x_0', y_0')$ are $(x_0'/\gamma + vt, y_0)$ in $S$, and the origin $(0', 0')$ of $S'$
        is at $(vt, 0)$ in $S$. Thus, the length of the rod in the stationary frame $S$ is simply $L = \sqrt{(x_0'/\gamma)^2 + y_0'^2}$.
        By introducing a third point in $S'$, the point of projection of the far end of the rod onto the $x$-axis $(x_0', 0)$, we
        see that this transforms to $(x_0 /\gamma + vt, 0)$ in $S$. Thus, it is immediately clear that only lengths along the $x$-axis
        are contracted, which justifies the approach of taking components along axes and treating them separately.
        This allows us to use the length contraction formulae to directly write $\Delta x = \Delta x' /\gamma$ and $\Delta y = \Delta y'$.
        Recognizing $x_0' = L_0\cos\theta_0$ and $y_0' = L_0\sin\theta_0$, we have
        \[
                L = \sqrt{L_0^2(1 - \beta^2)\cos^2\theta_0 + L_0^2\sin^2\theta_0} = L_0\sqrt{1 - \beta^2\cos^2\theta_0}, \tag{a}
        \]
        where $\beta = v /c$, $\gamma = 1 /\sqrt{1 - \beta^2}$. Note that the presence of the $\beta\cos\theta_0$ term indicates that
        we could have solved our problem in another way, taking components of the velocity along axes along and perpendicular to the rod. \\

        It is also important to note that the rod does indeed remain linear in both frames. To see this, take an intermediate point on the rod in $S'$,
        say $(\lambda,\; y_0'\lambda /x_0')$.
        This transforms to $(\lambda /\gamma + vt,\; y_0'\lambda/x_0') \equiv (x, y)$ in $S$, which clearly describes
        a straight line passing through $(vt, 0)$ and $(x_0'/\gamma + vt, y_0')$.
        \[
                \gamma(x - vt) = \lambda = \frac{x_0'}{y_0'}y \implies y = \gamma\frac{y_0'}{x_0'}(x - vt).
        \]
        Note that the slope of the rod has changed -- it is now $\gamma y_0' /x_0'$.
        Thus, the inclination $\theta$ of the rod in the stationary frame $S$ is steeper.
        \[
                \theta = \arctan\left(\gamma\frac{y_0'}{x_0'}\right) = \arctan\left(\gamma \tan\theta_0\right). \tag{b}
        \]

        \problem Show that infinite velocity in $S$ implies infinite velocity in $S'$, where $S$ and $S'$ are two frames
        moving with a constant velocity with respect to each other.

        \solution Let the velocities in question be $\vec{u}$ and $\vec{u}'$ in the frames $S$ and $S'$.
        Suppose $S'$ is moving with velocity $\vec{v}$ with respect to $S$.
        We can always choose our coordinate system such that $\vec{v}$ is directed along the $x$-axis in both frames.
        Thus, using our velocity addition formulae, we have
        \[
                u_x = \frac{u_x' + v}{1 + vu_x'/c^2}, \qquad u_y = \frac{u_y'}{\gamma(1 + vu_x'/c^2)}, \qquad u_z = \frac{u_z'}{\gamma(1 + vu_x'/c^2)}.
        \]
        Suppose $\vec{u}'$ is finite. Thus, all three components $u_x'$, $u_y'$ and $u_z'$ must be finite, since
        \[
                \norm{\vec{u}'} = \sqrt{u_x'^2 + u_y'^2 + u_z'^2} > |u_x'|,
        \]
        and so on for each component. Thus, all the numerators of $u_x$, $u_y$, $u_z$ are finite. Also note that $1 + u_x' v/c^2 > 0$, and 
        $\gamma = 1 /\sqrt{1 - v^2 /c^2}$ is real and non-zero.
        This is true because the framework of Special Relativity demands $|u_x'| \leq c$ and $|v| < c$, otherwise our
        transformation equations would fail.
        Thus, each of the components $u_x$, $u_y$, $u_z$ is finite, from which we see that $\vec{u}$ is finite.
        The contrapositive of this statement is that if $\vec{u}$ were infinite, then $\vec{u}'$ would also be infinite.

        \problem Consider a light beam passing through a horizontal column of water moving with velocity $v$ in the positive $x$ direction.
        (a) Determine the speed $u$ of light measured in the lab frame when the beam travels in the same direction as the flow of water.
        (b) Obtain this velocity in the first order approximation for $v \ll c$.

        \solution Let the frame of the lab be $S$. The frame of the water $S'$ moves with velocity $v$ with respect to $S$.
        In this frame $S'$, we know that the speed of light in the water is given by $u' = c / n$, where $n$ is the refractive index of water.
        Transforming to frame $S$ while noting that all quantities are directed along the $x$ axis, we see that
        \[
                u \;=\; \frac{u' + v}{1 + uv/c^2} \;=\; \frac{c/n + v}{1 + v/nc}. \tag{a}
        \]
        When $v \ll c$, we can approximate $(1 + v/nc)^{-1} = 1 - v/nc + v^2 /n^2c^2 - \dots \approx 1 - v/nc$.
        Thus, we obtain the first order approximation
        \[
                u \;\approx\; \left(\frac{c}{n} + v\right)\left(1 - \frac{v}{nc}\right) \;=\; \frac{c}{n} + v - \frac{v}{n^2} - \cancel{\frac{v^2}{nc}}
                        \;\approx\; \frac{c}{n} + v\left(1 - \frac{1}{n^2}\right). \tag{b}
        \]

        \problem Suppose we discover a particle with $v > c$ with a relativistic mass $m_v$. What can we say about its properties?

        \solution We note that the usual expression relating rest mass and relativistic mass,
        \[
                m_v = \gamma m_0 = \frac{m_0}{\sqrt{1 - v^2 /c^2}},
        \]
        makes little sense here since $\sqrt{1 - v^2 /c^2}$ is now purely imaginary.
        Indeed, the momentum $p = m_v v$ also becomes purely imaginary, so $E = \sqrt{(pc)^2 + (m_v c^2)^2}$ is also purely imaginary.
        Thus, these physical quantities loes their ordinary meanings with regards to this particle. \\

        Pick an inertial frame moving with velocity $u$ in the same direction as our particle. In that frame, the particle will move with speed
        \[
                v' = \frac{v - u}{1 - uv/c^2}.
        \]
        Since we only have access to inertial frames such that $|u| < c$, yet $v > c$, the quantity $v - u$ can never vanish, i.e.\ 
        there is no frame in which the particle will appear to be at rest relative to us.
        Hence, our particle has no proper `rest mass', which is in accordance with our previous observation that $m_0$ must be imaginary.
        Furthermore, it ought to be possible for us to choose $u = c^2 / v < c$. In this case, $v' \to \pm\infty$ (positive when approaching 
        from the left, negative when approaching from the right). In other words, the speed of the particle can be made as large
        desired, simply by choosing $u$ appropriately. Additionally, the velocity of the particle changes direction (sign) on either
        side of $u = c^2 /v$.

        \problem For a particle in motion, $E = 110$ MeV. Suppose its rest mass is $100$ MeV$/c^2$. What is its velocity in units of $c$?

        \solution Using $E = \gamma m_0 c^2$, where $\gamma = 1 /\sqrt{1 - v^2 /c^2}$, we square and rearrange to obtain
        \[
                v = c\;\sqrt{1 - \frac{m_0^2 c^4}{E^2}} = c\;\sqrt{1 - \frac{100^2}{110^2}} \approx 0.42\,c.
        \]

        \problem If $h = 6.626 \times 10^{-34}$ Js, and the wavelength of a sodium light is $5800$ angstrom, what is the relativistic
        mass of a photon corresponding to such light?

        \solution The energy of the photon is given by $E = hc /\lambda$. We also know that the relativistic mass $m$ of the photon is related to 
        its energy as $E = mc^2$. Thus, we obtain
        \[
                m = \frac{h}{c\lambda} = \frac{6.626 \times 10^{-34}}{5800 \times 10^{-10} \times 299792458} \approx 3.81 \times 10^{-36}\text{ kg}.
        \]
        Alternatively, we may calculate $E = hc/\lambda \approx 3.42 \times 10^{-19}$ J $\approx 2.14$ eV, so $m \approx 2.14$ eV$/c^2$.

\end{document}
