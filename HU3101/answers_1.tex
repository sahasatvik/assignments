\documentclass[11pt]{article}

\usepackage[T1]{fontenc}
\usepackage{geometry}
\usepackage{amsmath, amssymb, amsthm}
\usepackage{hyperref}

\geometry{a4paper, margin=1in}

\theoremstyle{remark}
\newtheorem*{answer}{Answer}

\title{HU3101: History and Philosophy of Science}
\author{Satvik Saha}
\date{}

\begin{document}
    \noindent\textbf{IISER Kolkata} \hfill \textbf{Class Test I}
    \vspace{3pt}
    \hrule
    \vspace{3pt}
    \begin{center}
    \LARGE{\textbf{HU3101: History and Philosophy of Science}}
    \end{center}
    \vspace{3pt}
    \hrule
    \vspace{3pt}
    Satvik Saha, \texttt{19MS154} \hfill \today
    \vspace{20pt}

    \paragraph{Question 1.} Describe how religious orthodoxy had at various stages
    stood in the way of development of science in India.
    \begin{answer}
        Throughout history, it has been observed that new and revolutionary ideas
        have always been met with resistance -- in the context of the development of
        science, this has often been a result of religious orthodoxy, the idea that
        the old texts are unerring and complete. Here, we focus on science in India.

        Charaka was a pioneer in medical science during the 2\textsuperscript{nd}
        century AD. In his texts (the \emph{Charaka Samhita}), we find knowledge of
        physiology and anatomy, how to diagnose and make predictions, and how to
        administer various treatments, procedures, and medicines. In one instance,
        Charaka prescribes beef, and claims that the flesh of a cow is beneficial
        for treating certain conditions\footnote{Quoting from a translation of the
        text, ``disorders caused by an excess of vayu, rhinitis, irregular fever, dry
        cough, fatigue, and also in cases of excessive appetite resulting from hard
        manual labour''.}. On the other hand, there are passages which venerate cows
        and brahmins, the cow being a sacred animal in the Hindu practice. Such
        self-contradictory views have been regarded as uncharacteristic of a person
        of Charaka's stature. It is now thought that the latter passages had been
        added later by brahmins, perhaps to confuse those reading the text.

        Sushruta was a pioneer in surgery between the 4\textsuperscript{th} and
        5\textsuperscript{th} century AD. His works reveal numerous surgical
        procedures, aided by sophisticated surgical instruments. One which stands out
        is his description of early plastic surgery, involving the grafting of skin;
        such mastery clearly indicates great and precise knowledge of anatomy.
        Sushruta advocated the dissection of corpses as a way for students to learn,
        thus gaining direct knowledge via experiments and observation. However, the
        \emph{Manusmriti} which codified the caste system of four Varnas prohibited
        people of different castes from mixing/touching in any way. Thus, the
        practice of dissection was heavily discouraged and from the
        7\textsuperscript{th} century AD onwards, anatomy became long forgotten
        knowledge. Both of these examples illustrate how religious orthodoxy became
        detrimental to the advancement of medical science in India.

        Aryabhata I, who lived during the 5\textsuperscript{th} century AD, is the
        oldest Indian mathematician whose works are available today. Among his many
        achievements, he recognized that the earth is spherical, and that its
        rotation about an axis causes the apparent westward motion of the stars in
        the night sky. He was able to predict solar and lunar eclipses, and also
        realized that the luminosity of the moon and the planets is due to the sun,
        the only truly luminous object in our solar system. However, his work was not
        accepted or continued by his successors, mainly due to criticism. In
        particular, Varahamihira was a great astronomer of the following century in
        his own right. In his works, we see that he is aware of the true nature of
        eclipses, much like Aryabhata I. In spite of this, he continues to talk about
        the astrology of Rahu and Ketu, which supposedly cause eclipses according the
        the wisdom of the brahmins. Brahmagupta, who was perhaps the greatest
        scientist of his time, held exactly the same position. He too knew and
        acknowledged that eclipses are unrelated with Rahu, but warns that this
        knowledge would render the brahmanical rituals meaningless and the Vedas
        false. Thus, he advises astronomers like Arybhata and Varahamira not to
        oppose these general beliefs. This is yet another instance of
        self-contradiction, perhaps due to pressure to somehow justify the rituals
        and practices of the brahmins.

        These instances, in which empirical observations have been sidelined in order
        to conform to and accommodate religious beliefs, show how religious orthodoxy
        stood in the way of scientific progress in India.
    \end{answer}

    \paragraph{Question 2.} Science is intrinsically international in nature.
    Discuss, citing examples.

    \paragraph{Question 3.} What do you understand by ``Thales discovered nature''?
    Explain, citing examples.
    \begin{answer}
        Thales of Miletus lived during the Ionian phase of Greek Science, from
        approximately 600 BC onwards. The Ancient Greeks recognised an entire
        pantheon of gods, each responsible for some facet of nature. For example,
        Zeus was the god of the sky, Poseidon the god of the sea, and Hades the god
        of the dead. Thus, all natural phenomena were ascribed to their whims.
        Lightning storms represented the anger of Zeus, earthquakes the anger of
        Poseidon, and so on. As a result, there was no real attempt to understand
        these phenomena on a deeper level.

        Thales was not satisfied with this state of affairs, and instead proposed
        that these natural phenomena could be explained in a different way: the
        interaction of matter, in accordance with natural laws. This was a bold step
        since he claimed that natural phenomena are not acts of god, but simply a
        result of \emph{nature} taking its course. This is what we mean by Thales
        \emph{discovering  nature} -- he brought nature into the forefront of
        thought, giving rise to what we would now call \emph{science}. Even during
        the time of Newton, science was practised under the banner of \emph{natural
        philosophy}.

        To illustrate this point, we give a few examples of Milesian theories of
        nature proposed by Thales and his successors. Drawing from experience,
        Thales thought that the earth floated on a vast ocean, much like a boat;
        earthquakes are a result of disturbances in this ocean, causing the earth to
        shake. Anaximander though that lightning arose from clouds being split up by
        the wind. We may note that despite missing several crucial details (chief
        being the nature of electricity), this is surprisingly insightful.
        Presumably, he noted a correlation between the occurrence of lightning and
        violent storms accompanied by huge clouds and high speed winds.

        In any case, both these hypotheses offer much to think about without directly
        invoking god, and are certainly improvements over the prevailing beliefs of
        the time. They are based in observation and common experience; Anaximander's
        hypothesis even has a certain predictive power and can be loosely verified.
        The Milesians entertained open debate regarding their ideas, introducing an
        element of competition: who can produce the best theory to explain this
        event? This is why we say that the Milesians were the first to do real
        science.
    \end{answer}

    \paragraph{Question 4.} In the light of Thomas Kuhn's paradigm theory, show how
    the Newtonian paradigm changed into the Einsteinian paradigm.
    \begin{answer}
        Thomas Kuhn's definition of a paradigm in science differs from common usage
        of the word. By paradigm, he refers to s set of universally recognized
        scientific achievements which (for a time) provide model problems and
        solutions for a community of practitioners (scientists within the relevant
        field of study). Thus, the paradigm of the present guides the direction of
        research done in that field, and provides a lens with which to view and
        interpret new discoveries. It does so by setting up questions about the
        unknown and making suitable predictions; it is then up to scientists to
        investigate these predictions. If these are verified, this strengthens or at
        least increases confidence in the paradigm. If contradicted, such a novel
        discovery may well spawn an entirely new paradigm.

        In order to understand the paradigm shift from the gravitational theory of
        Newton to that of Einstein, we must first recognize why Newton's ideas
        dominated scientific thought for over two centuries. Newton's most celebrated
        contributions lie in mechanics: the three Laws of Motion and the Law of
        Universal Gravitation. The idea that the earth must attract bodies in order
        for them to stay on their surface, or fall, is not necessarily
        new\footnote{We highlight Bhaskacharya's comments on the attractive power of
        the earth as early as 12\textsuperscript{th} century AD, discussed in Answer
        5.}. What Newton realized is this: the reason why an apple falls to the
        ground \emph{is exactly the same} as why the moon orbits the earth. Indeed,
        Newton's description of this new \emph{gravitational force} was not merely
        qualitative, but quantitative, described by the formula \[
            \mathbf{F}_{12} = G \frac{m_1m_2}{r^2}\, \hat{\mathbf{r}}_{12}.
        \] With this and his Laws of Motion\footnote{We must mention that in order to
        even begin formalising his ideas mathematically, Newton had to invent an
        entirely new branch of mathematics, namely \emph{calculus}.}, Newton could
        convincingly and accurately explain (and predict) the motions of \emph{every
        celestial body} observed, recovering Kepler's Laws\footnote{Thanks to Kepler,
        the fact that heavenly bodies traced elliptical orbits was already known
        during Newton's time; what was missing was a theory explaining this
        phenomenon. There is a wonderful anecdote detailed in Bill Bryson's \emph{A
        Short History of Nearly Everything}, in which Edmund Halley, Robert Hooke,
        and Christopher Wren made a wager regarding who could solve this problem
        first. Halley approached Newton himself, who when given the problem remarked
        immediately that he had already performed the necessary calculations. Upon
        searching his papers, he could not find them! Pressed by Halley, he redid his
        calculations, which he further expanded upon and published under the title
        \emph{Philosophi\ae Naturalis Principia Mathematica} --- at Halley's
        expense.}. With the same set of equations, he could also explain every
        phenomenon pertaining to mechanics (motion) on earth. This is what we mean by
        the Newtonian paradigm. Over the next couple of centuries, most of Newton's
        predictions were verified, inspiring great confidence. When analysing any
        sort of motion, Newton's Laws were indispensable.

        With the advancement of technology in the 19\textsuperscript{th} century,
        astronomers began making measurements with astounding precision. Suddenly, an
        anomalous phenomenon was discovered. The orbit of Mercury, which is
        elliptical, isn't a static shape in space but rather rotates with time; we
        say that the perihelion of Mercury \emph{precesses}. This too was predicted
        by Newton --- the problem was that his calculations were off by a rate of 43
        arcseconds per century! Yet another inexplicable phenomenon was the bending
        of light by the sun. These are clear indications that Newton's theories were
        incomplete at best, wrong at worst.

        In the 20\textsuperscript{th} century, Einstein published both the Special
        and General Theory of Relativity, resolving all of these problems in one
        stroke. The concept of gravity was refashioned from a \emph{force} -- action
        at a distance -- to a consequence of the \emph{geometry} of
        space-time\footnote{Again, formalising these ideas took a great deal of
        mathematics: \emph{Riemannian geometry}. This time it was already available
        in a rudimentary form, but only as a mathematical curiosity not applied to
        physics. A considerable amount of time was spent by Einstein (racing against
        Hilbert) in expressing his ideas using this language.}. A revolutionary new
        idea was that the passage of time is not steady and absolute, but rather
        subject to both relative motion as well as gravity. It turns out that in the
        absence of extremes of these effects, Einstein's equations reduce to those of
        Newton's, which explains why Newtonian mechanics stood the test of time for
        so long. The full predictions of Einstein's theories are far reaching, and we
        are exploring and verifying them even today.

        Once again we emphasize how the Newtonian paradigm set up questions and
        predictions in physics regarding the motions of heavenly bodies. It was the
        failure of just a few of these predictions that lead to its demise; it was
        the success of the Einsteinian paradigm in resolving these same questions
        that cemented it in modern physics.
    \end{answer}

    \paragraph{Question 5.} Explain how Bhaskacharya defended his position on
    gravitational attraction vis-\`a-vis the believers in the Puranic myths.
    \begin{answer}
        Bhaskacharya, or Bhaskara II, was a great Indian mathematician of the
        12\textsuperscript{th} century. Among his many achievements are solutions to
        the Pell equation (using his own \emph{Chakravala} method) and a rudimentary
        form of differential calculus. Within his works (the
        \emph{Siddhanta-shiromani}), one finds an allusion to the concept of gravity.
        Here, he muses upon how falling bodies must attracted by the earth. After
        all, if this were not the case, then there ought to be no bias towards any
        particular direction for the body to fall; there is something special about
        the direction towards the earth, directly beneath a person. Using his
        knowledge about the spherical shape of the earth, he further notes that a
        person on the other side of the earth can comfortably stand and walk just
        like him; while this other person appears to be upside down from our
        perspective, the reality is that the other person has a different direction
        for `beneath'. At any point on the earth, `beneath' or the direction objects
        fall is always towards the earth's center.

        Bhaskacharya's work implies that the earth is a body suspended in space,
        unsupported by anything else. This contradicted the accepted wisdom of the
        Puranas, which stated that the earth was supported by one of the heads of the
        serpent Vasuki. Thus, believers in the Puranas naturally questioned, ``How
        can a body as large as the earth remain suspended without support?''

        Bhaskacharya countered with the question, ``If the earth is indeed supported
        by some (presumably enormous) body, how can such a body remain suspended
        without support?'' In other words, if one accepts that every body must be
        supported by another, one must ask, ``How is the body at the base of the chain
        supported?'' There are two possibilities: either there is a final supporting
        body which is unsupported, thereby contradicting our initial assumption
        (every body must be supported), or there is an infinite chain of bodies each
        supporting the other, which is absurd. The latter fallacy is called
        \emph{infinite regression}.

        Thus, Bhaskacharya argued that if one is prepared to accept the fact that a
        final supporting body (such as Vasuki) can remain suspended on its own
        strength, then one must also accept that the earth itself ought to be able to
        remain suspended in space\footnote{Note that this does not explicitly rule
        out a finite number of supporting bodies. After this, one can argue that the
        earth remaining suspended on its own is a much simpler explanation than the
        earth supported by Vasuki which in turn is suspended on its own. Of course,
        the best way to decide is by direct observation; today, we have photographic
        evidence. Even in those days, astronomy ought to have been sufficiently
        advanced to make the required observations.}.
    \end{answer}
    

\end{document}
