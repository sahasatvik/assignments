\documentclass[11pt]{article}

\usepackage[T1]{fontenc}
\usepackage{geometry}
\usepackage{amsmath, amssymb, amsthm}
\usepackage{hyperref}

\geometry{a4paper, margin=1in}

\theoremstyle{remark}
\newtheorem*{answer}{Answer}

\title{HU3101: History and Philosophy of Science}
\author{Satvik Saha}
\date{}

\begin{document}
    \noindent\textbf{IISER Kolkata} \hfill \textbf{Class Test II}
    \vspace{3pt}
    \hrule
    \vspace{3pt}
    \begin{center}
    \LARGE{\textbf{HU3101: History and Philosophy of Science}}
    \end{center}
    \vspace{3pt}
    \hrule
    \vspace{3pt}
    Satvik Saha, \texttt{19MS154} \hfill \today
    \vspace{20pt}

    \paragraph{Question 1.} What are the Four Great Chinese Inventions? Do you think
    the importance of these inventions has been over-emphasized by Western scholars
    at the cost of many other achievements of the Chinese?
    \begin{answer}
        The Four Great Chinese Inventions, named so by Joseph Needham, are
        papermaking, printing, gunpowder, and the magnetic compass.
        \begin{enumerate}
            \item \emph{Papermaking.} Paper is a relatively cheap, easily
            manufactured and distributable medium for writing (it was initially used
            for wrapping). This means that records of Chinese history are very well
            preserved. Another consequence is the introduction of paper currency.

            \item \emph{Printing.} Block printing fully leverages the power of paper
            as a means of sharing information. Now, a pattern engraved in a block of
            wood can be copied accurately onto a surface as many times as required;
            this technique was used for textile printing before it was applied to
            writing. The Chinese innovated further with movable type, in which
            separate characters are engraved onto individual pieces of ceramic or
            wood, which can be arranged and fixed to create the desired text --
            later, they can be removed and rearranged for a different text. The use
            of metal movable type is the finishing touch on this invention.

            \item \emph{Gunpowder.} The explosive power of gunpowder had been
            perfected by the Chinese over centuries. This is a huge technological
            advancement in the field of war.

            \item \emph{Magnetic compass.} While practically all ancient
            civilisations had developed means of navigation by studying the night
            sky, this method has obvious drawbacks -- it can only be employed on
            clear nights. The magnetic compass provides a simple, reliable method of
            navigation. Early versions involved magnetized material floating on a
            bowl of water (which allows free rotation); later, the needle was
            suspended by a thread, giving a `dry compass'.
        \end{enumerate}

        News of these inventions began to make their way to Europe only during the
        16\textsuperscript{th} century by sailors and explorers. These four
        inventions became instrumental in pulling Europe out of the Dark Ages, and
        enabling them to colonize the world\footnote{To quote Karl Marx, ``Gunpowder,
        the compass, and the printing press were the three great inventions which
        ushered in bourgeois society. Gunpowder blew up the knightly class, the
        compass discovered the world market and found the colonies, and the printing
        press was the instrument of Protestantism and the regeneration of science in
        general; the most powerful lever for creating the intellectual
        prerequisites.''}. In this context, it makes sense why these Four Great
        Inventions have been highlighted by Western scholars, seeing that they
        stimulated and accelerated the development of capitalism in Europe.

        On the other hand, this is by no means an exhaustive list of China's
        achievements. Some more inventions include rice planting, silk,
        porcelain, tea, and fire arrows. These undoubtedly had a great impact on the
        lives of the Chinese people, and deserve a place in history. It has been
        argued however that the Four Great Inventions had been underutilized by the
        Chinese; they were a heavily bureaucratic society, without much focus on
        exploration. The West took these inventions and created weapons with which
        they conquered foreign nations. Another observation is that after the
        founding of the Qing dynasty in the mid 17\textsuperscript{th} century, there
        aren't any Chinese inventions rivalling the importance of these ones.

        Thus, the Four Great Inventions represent a small fraction of the
        achievements of the Chinese civilization. It is however natural for Western
        scholars to have attached more importance to these four, given their
        universal nature and their impact (perhaps indirect, via the West) on the
        globe.
    \end{answer}

    \paragraph{Question 2.} ``Under the Abbasid dynasty Baghdad became a great centre
    for translating scientific and philosophical texts.'' Discuss.
    \begin{answer}
        The Abbasid dynasty (750 - 1258 AD) coincides with what is now called the
        Islamic Golden Age. Thanks to their geographical location in Central Asia and
        Portugal, they were able to learn from and assimilate many aspects of the
        Greeks, the Indians, the Persians, and the Assyrians. Many of the caliphs
        during this period were enthusiastic patrons of scholars and scientists. The
        result was the Gr\ae co-Arabic Translation movement. A large number of
        (mostly Greek) texts were translated into Arabic from languages such as Greek
        and Sanskrit (note that the original texts were not always available, but
        neighbouring civilizations such as the Indians had acquired this knowledge
        and written translations of their own earlier). The center of this movement
        was the capital city of Baghdad.

        The second Abbasid caliph, al-Mansur sponsored many such translations. He
        invited scientists and learned people to his court, especially astronomers.
        After Indian astronomers presented the \emph{Zij al-Sindhind} (an
        astronomical handbook) to him, al-Mansur had this translated from Sanskrit
        into Arabic. He sponsored the translation of Ptolemy's \emph{Almagest},
        Euclid's \emph{Elements} (more than once!), as well as medical texts of Galen
        and Hippocrates. Al-Mansur also had the texts of Brahmagupta, such as
        \emph{Brahmasphutasiddhanta} and \emph{Khandakandadhyaka}, translated.  

        The seventh caliph al-Mamun is also notable in this respect; he was also a
        patron of astronomy, and built the first astronomical observatories in
        Baghdad. He even sent scholars to gather and bring home texts from foreign
        lands, continuing the rich culture of translation. The great Thabit ibn Qurra
        further translated the works of Ptolemy, Euclid, Archimedes, and Appollonius.
        It is said that al-Mamun's pursuit of knowledge was inspired by a dream he
        once had involving a discussion with Aristotle.

        We must mention the House of Wisdom which acted as a key academic and
        scholastic center during this period, well-funded and supported by the
        caliphs (especially those discussed). This was also a center of learning, at
        a time when universities did not exist as we know them.

        The fruits of this translation movement are evident in the works and
        achievements of Abbasid scholars. By bringing together so many classic texts
        from neighbouring nations, they were placed at a considerable advantage. They
        made use of this knowledge in the fields of mathematics, astronomy,
        geography, medicine, optics, and chemistry. For instance, al-Khwarizmi
        studied the works of Indian mathematicians, learning Sanskrit for this
        purpose, and developed algebra. The numbers we use today are known as
        Hindu-Arabic numerals; these most likely originated in India and reached the
        rest of the world via the work of Islamic scholars. Thus, the Islamic
        scholars inherited and furthered the scientific knowledge of many
        civilizations, preserving it and helping it spread.
    \end{answer}

    \paragraph{Question 3.} Describe how the paper industry gave a boost to the
    Baghdad economy.

    \paragraph{Question 4.} Thales or Uddalaka Aruni -- whom would you consider the
    first scientist of the world?
    \begin{answer}
        Uddalaka Aruni was a Vedic sage from the 8\textsuperscript{th} century BC,
        preceding Thales by roughly two hundred years. The \emph{Chandogya Upanishad}
        speaks of him, and his questions about nature (regarding truth, reality, the
        possibility of eternal and unchanging things) reveal that he was a
        philosopher. This makes him one of the first recorded in history. However, we
        are concerned with whether or not he was a \emph{scientist} too; if so, this
        would place him ahead of Thales as the first scientist in (recorded) history.

        We swiftly recapitulate why Thales is considered a scientist: he `discovered
        nature', in the sense that he sought to explain natural phenomena as the
        result of the interaction of matter in accordance with natural laws. In doing
        so, he did not involve mythology or the gods. In an attempt to explain the
        origin/nature of matter, he proposed that the `primary principle' is water.
        Furthermore, the earth floats on water, and its motion on this ocean of water
        causes earthquakes. His name is perhaps best known to mathematicians; Thales
        Theorem\footnote{Any triangle inscribed in a circle such that one of its
        sides is a diameter must be right angled.} is one of the most well known
        theorems in geometry, and is perhaps the oldest theorem to be named after a
        person.

        Uddalaka Aruni's philosophy exhibits a clear materialistic bent. He too was
        interested in understanding the origin of matter in terms of first
        principles. Note that he rejects the idealist Vendantic ideas of everything
        arising from `non-being'; he questions how can such a thing be possible,
        being arising from non-being? From his observations, everything around him
        seemed to arise from some basic substances, and understanding those basic
        substances would reveal the true nature of things. For instance, every clay
        utensil and object is at its heart just clay; every copper ornament is just
        copper in a different form; every iron scissors is just iron. He means this
        in the sense that these fundamental things underlie the `infinite variety' of
        things of the world. He identifies three basic principles: fire (tejas),
        water (ap), and food (anna), and posits that everything in the universe
        (living and non-living, material and mind) evolved from these. He theorizes
        that water arises from fire (heat causes people to sweat) and that food
        arises from water (crops flourish during rain).

        Perhaps the most striking achievement of Uddalaka Aruni is that he is the
        first person in history to show a clear connection between material food and
        the conscious mind. In order to demonstrate this to his son Svetaketu (who
        had returned after years of studying the Vedas), he instructed him to fast
        for fifteen days. After this, he quizzed Svetaketu on what he'd learnt, and
        saw that his son couldn't answer! Only when Svetaketu had eaten and recovered
        could he recall his hymns and melodies. Uddalaka explains how the body can be
        divided into sixteen parts, the last representing `breath' (life) and only
        needing water to survive. By fasting, the fifteen parts of his body had been
        depleted, making him unable to retain his memory. Upon eating food again, his
        faculties returned, just as how a fire can be rejuvenated by replenishing its
        fuel. Uddalaka further explains his theory of how different parts of food
        contribute to different parts of the body: some entering the flesh, some the
        bones, and some constituting the mind. While the exact details are not
        relevant, we must note that he gives the human mind an explicitly material
        origin, arising from the subtlest parts of the food one consumes.

        Looking back at Uddalaka's work, we see the seeds of a scientific approach of
        making observations, creating hypotheses, and testing them via experiment.
        Uddalaka much like Thales invokes a materialistic origin of everything,
        without speaking of any religious mysticism or idealism. Thus, it is
        certainly fair to call Uddalaka Aruni the first scientist in the world.
    \end{answer}

    \paragraph{Question 5.} What were Karl Popper's chief arguments against Logical
    Positivists?
    \begin{answer}
        Karl Popper's chief arguments can be condensed into the following.
        \begin{enumerate}
            \item \emph{The Problem of Induction.} Many philosophers, notably
            Popper and David Hume, have considered the complete verification of
            certain (perhaps scientific) statements to be impossible. To illustrate,
            consider the statement ``All swans are white''. In order to `verify' such
            a statement, one would have to gather many swans and check their colours;
            however, there always seems to be a logical gap between \[
                \text{All swans that we have checked are white}
                \stackrel{??}{\implies} \text{All swans are white}.
            \] Thus, the problem arises when we want to generalize. In doing so, we
            are performing a sort of inductive reasoning, which may or may not be
            justified. In this scenario, an arbitrarily large amount of supporting
            evidence (white swans) cannot satisfactorily establish our statement.

            \item \emph{Confirmation bias.} The idea that a statement is meaningful
            only if it can be verified encourages one to seek this confirmation
            wherever one can find it, ignoring any contradictory evidence in the
            process. Thus, a principle of verification is not enough; one must not
            only accept, but actively seek out contradictory evidence in order to
            properly test one's hypothesis.

            \item \emph{Meaningful vs Scientific.} Logical positivism seems to
            maintain that all meaningful (verifiable) knowledge equates to scientific
            knowledge, or that anything which fails our `scientific test' is
            automatically meaningless. This eliminates many parts of metaphysics,
            some of which Popper considers essential for driving science forward, for
            instance questions about how things originate, consciousness, free-will,
            etc. Thus, not all knowledge is scientific.
        \end{enumerate}

        Popper introduced the notion of \emph{falsifiability} as a better test than
        \emph{verifiability}. While our statement about white swans cannot be
        adequately verified even by a mountain of evidence, a single counterexample
        is enough to prove it false. \[
            \text{This particular swan is black} \implies \text{Not all swans are
            white}.
        \] The logical step is perfectly clear in such cases.
        Note that our statement comes equipped with its potential falsification; it
        admits a scenario (the existence of a swan that is not white) which would
        render the statement false. Such statements are \emph{falsifiable}. Popper's
        idea is to label these statements \emph{scientific}. Some statements (such as
        those in metaphysics) may be unscientific, but that is no reason to call them
        \emph{meaningless}. After all, certain statements may not be falsifiable in
        one era, but with the progression of science and technology become
        falsifiable in the next. Thus, the means by which science ought to progress
        is by \emph{conjecture and refutation} instead of induction.

        Popper can thus create a clearer distinction between the \emph{scientific}
        and the \emph{pseudoscientific}. An example of the latter is astrology; by
        the yardstick of verification, one might cite multiple instances on which
        astrological predictions held true and present this as supporting/confirming
        evidence. However, Popper's standards demand that we look at counterexamples
        instead of ignoring them. Similarly, those statements which continuously
        `move goalposts' in light of new evidence are also to be considered
        pseudoscientific, since they resist falsification by changing the conditions
        at every turn.
    \end{answer}

    \paragraph{Question 6.} What were the major differences between the Chinese and
    the Indian philosophers and how did these differences influence the development
    of science and technology in the two civilizations?
    \begin{answer}
        While there are many similarities between the Greek and Indian approaches to
        science and philosophy (with the latter directly borrowing many elements from
        the former), the development of science in China has been regarded to be
        fairly unique. Following are some features which characterize this.
        \begin{enumerate}
            \item \emph{Bureaucracy.} This factor is not directly a philosophical
            take, but it sets the stage and context for these developments. The
            Chinese society revolved heavily around the State. While civilizations
            such as India produced thinkers like Aryabhata, Brahmagupta,
            Bhaskacharya, etc.\ who pursued knowledge independently, the Chinese
            State was far more involved in the process of gathering and organizing
            facts. This resulted in a organized, consistent development of
            technologies and ideas which spread quickly throughout the country.
            However, the same State involvement often hampered the progress of
            science and the acquisition of knowledge for its own sake; the motive
            behind these pursuits was often simply to increase revenue or gear for
            war. A famous example is that of general Cheng Ho in the
            15\textsuperscript{th} century, who conquered Ceylon and sailed as far as
            the African coast until he was ordered to return by the Mandarinate.
            While the Chinese were clearly capable of such expeditions, rivalling
            those of Columbus by a century, they were uninterested. The Emperor was
            more focused on problems closer to home, namely invading Tartar tribes.
            Furthermore, China was primarily an agricultural society, thus its
            economy had no real dependence on foreign trade. In conclusion, the
            development of science received little to no funding from the State (which
            was resistant to change and new ideas). Civil service was the most
            important objective for the people.

            \item \emph{Laws of Nature.} The Chinese civilization had little interest
            in abstract codified laws. This seems to be a result of the tyranny and
            authoritarian nature of politicians belonging to the Legalist school of
            thought. As a result, they did not develop what we would call `Laws of
            Nature'. The Confucian school of thought was concerned with order and
            structure, both within society and in the universe; the patterns of
            nature arise as a by-product of this harmonious co-ordination. Thus, they
            lacked the abstract thinking which made the Indian and Greek scientists
            and philosophers so well known even today. This is perhaps illustrated
            best in the field of astronomy; in order to speak of modelling the solar
            system like Aryabhata, or to begin contemplating the rudimentary ideas of
            gravity and how the earth lies in space like Bhaskacharya, a groundwork
            of abstract thinking and a concept of extrapolating from daily
            observations (formulating universal laws) seems essential. In retrospect,
            the Chinese did not use the modern systems of observation, experiment,
            and inference.

            \item \emph{Creator beings.} The most popular Chinese schools of thought
            did not speak of a Creator being, since their philosophies and
            explanations for how the world works did not require one. As we have
            already said, the Confucian school dealt with organization of society and
            nature. While things like `good behaviour' (ethics, morality) may have
            been considered \emph{holy}, there was no need for them to be
            \emph{divine} in the sense of mirroring an all-powerful being. This is of
            course unlike the Indians who revered the Vedas and maintained a pantheon
            of gods and goddesses. They considered these beings to be the creators of
            everything, including the Laws of Nature which they strived to decipher
            and understand; thus, the concept of a Creator gave thinkers a clear
            motive/direction. In contrast, the Chinese were not as motivated to
            uncover such abstract laws; even if a supreme being did establish
            these Laws of Nature, the Chinese argued that lesser beings such as
            humans could not hope to understand them. We note this focus on
            materialism in the Confucian philosophy, compared to the Indians who
            believed in the concept of reincarnation and thus valued the mind/soul as
            some eternal entity.

            The Taoist school of though was against Confucianism and sought to become
            closer to Nature by living outside society and making observations.
            However, they too did not make the necessary jump of hypothesis and
            experiment, and hence made little progress. They were also distrustful of
            reason and logic; a bold step.
        \end{enumerate}

        The Chinese approach can be summarized succinctly as `practice over theory'.
        They made huge technological leaps, were diligent observers and classifiers
        of knowledge, and maintained an efficient and cohesive society. However, they
        did not make the transition into the modern approach towards science. In
        contrast, the achievements of the Indians are much more pronounced in the
        areas of abstract forms of thinking.
    \end{answer}
    

\end{document}
