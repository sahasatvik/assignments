\documentclass[11pt]{article}

\usepackage[T1]{fontenc} 
\usepackage{geometry}
\usepackage{amsmath, amssymb, amsthm}
\usepackage{hyperref}

\geometry{a4paper, margin=1in}

\theoremstyle{remark}
\newtheorem*{answer}{Answer}

\title{HU3101: History and Philosophy of Science}
\author{Satvik Saha}
\date{}

\begin{document}
    \noindent\textbf{IISER Kolkata} \hfill \textbf{Class Test V}
    \vspace{3pt}
    \hrule
    \vspace{3pt}
    \begin{center}
    \LARGE{\textbf{HU3101: History and Philosophy of Science}}
    \end{center}
    \vspace{3pt}
    \hrule
    \vspace{3pt}
    Satvik Saha, \texttt{19MS154} \hfill \today
    \vspace{20pt}


    \paragraph{Question 1.} In the light of the various theoretical models of
    'Colonial Science', discuss the scientific career of Radhanath Sikdar.

    \begin{answer}
        India under British rule progressed very slowly in terms of science; the
        administration did very little to promote science, or at least science which
        does not directly benefit their economy, politics, or military. Basalla's
        diffusionist model of colonial science involves the movement of modern
        scientific material and tradition originating from Europe, practised in the
        colonies by Europeans, then transplanted into India by the efforts of Indian
        scientists. Deepak Kumar describes colonial science as one in which
        ``result-oriented research in applied science heavily supersedes the
        curiosity-oriented research in pure science''. As Sumit Sarkar notes, this
        means that scientific endeavours were focused on areas such as cartography,
        botany, geology, medicine, at the detriment of fields such as mathematics,
        physics, chemistry.

        Radhanath Sikdar's name stands out as a great Indian scientist of the 19th
        century. Born in 1813 into a modest family in Calcutta, we was sent to Hindu
        college at the age of 11 years; his father planned for him to become a clerk.
        It quickly became apparent that Radhanath was a brilliant student, and thus
        he continued his studies for seven years. During this time, he became
        proficient in English, Greek, Latin, Sanskrit, as well as science and
        mathematics.  Strangely enough, he also authored a prize-winning article,
        titled \emph{The cultivation of science is not more favourable to individual
        happiness, nor more usesful and honourable to a nation than that of polite
        literature}.

        Radhanath was a follower of Derozio, who promoted free-thinking among his
        students. In the context of science and philosophy, Radhanath was exposed to
        the ideas of Bacon and Hume. Under the guidance of John Tytler, he studied
        Newton's \emph{Principia}, Euclid's \emph{Elements}, Jephson's
        \emph{Fluxions}, Windhouse's \emph{Analytical Geometry} and \emph{Astronomy}.
        Naturally, his greatest talents were in mathematics, and he soon joined the
        Great Trigonometrical Survey of India project as a computer. In 1832, he was
        sent to Dehradun.

        The Great Trigonometrical Survey of India (GTS) was tasked with accurately
        mapping important points of India; this was both a scientific endeavour, as
        well as a means for the Government of India to acquire better maps and
        geographic information to rule (and exploit) India. George Everest was a
        prominent figure in this project, first as Chief Assistant-Surveyor, then
        Superintendent, Surveyor General of India, and a Fellow of the Royal Society.
        In order execute his plan of surveying, via 35 stations between Dehradun and
        Sironj, he was in search of talented natives well versed in trigonometry.
        Radhanath was recommended to him by Tytler, and his proficiency was not lost
        on Everest, who commented
        \begin{quote}
            \dots there are few in India whether European or native that can at all
            compete with him. Even in Europe these mathematical attainments would
            rank very high.
        \end{quote}
        He further describes Radhanath as a
        \begin{quote}
            \dots hardy, energetic young man, ready to undergo any fatigue, and
            acquire a practical knowledge of all parts of his profession. ... There
            are a few of my instruments that he cannot manage; and none of my
            computations of which he is not thoroughly master. He can not only apply
            formulae but investigate them.
        \end{quote}
        On one instance, he even prevented Radhanath from transferring to a different
        government department.

        Radhanath was a great asset to the GTS; one of his jobs was to carry geodetic
        surveys—the study of the earth’s geometric shape orientation in space and
        gravitational field. He was capable of deriving his own, new working formulae
        from first principles and applying them to the task at hand.  He published
        \emph{A set of tables for facilitating the computation of trigonometrical
        survey and the projection of maps for India}, which proved to be immensely
        useful for years to come.

        After Everest retired in 1843, he was succeeded by Andrew Waugh. On his
        orders, Radhanath was promoted to Chief Computer and transferred back to
        Calcutta. Here, he was set to work on measuring the heights of mountains,
        based on data collected by multiple teams of field observers. The process of
        gathering this data was complication by the fact that the snow peaks under
        consideration were practically on the India-Nepal border, and were thus
        observed from nearly a hundred miles away. On great advancement made by
        Radhanath in this regard is his understanding of the phenomenon of
        refraction, and the precise effect this has on the data. In 1852, Radhanath
        used a combination of six sets of observations to calculate the height of
        `Peak XV' as 29,000 feet, superseding Kanchenjunga as the highest mountain in
        the world.

        Radhanath also made significant contributions to meteorology; in 1852, he
        became the superintendent of an observatory in Calcutta, and soon began the
        process of correcting barometer readings to a standard temperature. This has
        to be done since temperature affects barometers in two ways: the brass scale
        as well as the mercury expand and contact in different ways. Without access
        to work done in Europe, he constructed his own reduction formulae and tables
        using his knowledge of physics from first principles. The practice of
        regular, hourly observations began under his supervision, and the result was
        the first proper climatological data set of a city in India. Again, his
        published work on this subject including surveying manuals and computational
        tables remained invaluable even in the 19th century.

        In 1854, Radhanath and Peary Chand Mitra founded the \emph{Masik Patrika} for
        women, where he wrote articles in simple, vernacular language. In this
        respect, he rebelled against the sanskritized form of Bengali favoured by
        Vidyasagar and Akshay Kumar Datta; ``Of what worth is a piece of Bengali
        writing, if it cannot be  grasped readily by every housewife?''. On the other
        hand, he wrote nothing about science in Bengali, perhaps finding the language
        a bit too imprecise. Thus, he played practically no role in popularising
        science, or influencing scientific thought in general society; something
        which is essential for carrying out reformation.

        Radhanath retired in 1862, and taught mathematics in the General Assembly's
        Insitution. In 1864, he was honoured as a Corresponding Member of the German
        Philosophical Society. He died in 1870.

        It can be argued that Radhanath was the first modern Indian scientist: he was
        trained in mathematics and physics, participated and made a living from the
        GTS which was a great scientific endeavour, and he even made several original
        contributions in the field. We have observed that the European influence was
        critical to his training, and it seems fair to say that without Derozio or
        Tytler's influence, he would not have gone very far in this field. He learnt
        from European scientific literature, as this was the most advanced of the
        time. Once he established himself in the GTS however, he quickly began
        cultivating a wealth of knowledge by himself, making several key
        contributions. As remarked earlier, these were in areas of practical or
        applied science, and ultimately served to benefit the British Government's
        grip on the nation. Thus, Radhanath's career conforms well with both
        mentioned models of colonial science. \\

        Radhanath's relationship with the British administration was rocky at best;
        it is evident that despite heaping him with praise, Radhanath was not treated
        with the respect that a scientist of his stature deserved. For instance, his
        salary was well below that of his colleagues, despite the quality of his
        work.  There is also a well documented incident in which he protested the
        unlawful exploitation of his own departmental workers by Magistrate
        Vansittart; after many sketchy proceedings and a court trial, he was fined a
        sum of 200 rupees; Everest seems to have been complicit in this matter.
        Without going in to too much detail, perhaps the greatest injustice was the
        very naming of `Peak XV' -- Mount Everest -- by Waugh, who did so in honour
        of his predecessor and mentor. This was inspire of Everest's own policy of
        deferring to the native name (which would have been \emph{Deodhunga}), again
        on sketchy grounds of lack of information. In a meeting of the Geographical
        Society, Waugh was awarded the Victoria Gold Medal for his contributions,
        accepted by Everest on his behalf. Sikdar's role seems to have been either
        quietly understated or his name omitted entirely, despite being the first
        person to even suspect that Peak XV was the tallest mountain in the world.
        Another instance is that of the book \emph{Manual of Surveying for India},
        whose preface acknowledges Radhanath's name as essential, with many of its
        chapters being entirely his own. After Radhanath's death, the third edition
        was published without this acknowledgements, causing outrage even among a
        group of Englishmen: Colonel John Macdonald spoke out against this ``robbery
        of the dead'', and was subsequently suspended and demoted.

    \end{answer}

    \paragraph{Question 2.} Francis Bacon's and Rene Descartes' approach to science
    was quite dissimilar --- yet both of them are hailed as the precursor to modern
    science. Discuss.

    \begin{answer}
        Bacon and Descartes are two figures who contributed to the scientific method,
        in an attempt to answer questions such as: what does it mean to \emph{do}
        modern science? How does one proceed with a given problem or idea? When
        should one be satisfied with an answer?

        We start with Francis Bacon, who was a fairly distinguished English subject.
        He declared ``all knowledge as his province'', and wrote several volumes on
        history, along with a science-fiction novel (\emph{The New Atlantis}). We
        emphasize his key insight -- \emph{knowledge is power}. His formulation of
        science and learning is in service of gaining practical knowledge for the
        enrichment of mankind; while these ideas seem commonplace today, these
        were important steps in the seventeenth century.

        At the heart of the Baconian method is induction from hard evidence. In order
        to verify a hypothesis or understand a phenomenon, a scientist must collect
        data (vast amounts of data), look for patterns, and then seek to generalize,
        perhaps arriving at some principle of nature. Again, there is a great deal of
        emphasis on the collection of materials/data; naturally, Bacon was a great
        encyclopedist. His stance was to reject Aristotle's \emph{anticipation of
        nature}, and instead focus on \emph{interpretation of nature}. Bacon felt
        that the former was too rigid and restrictive for science: after all, if one
        makes observations merely to confirm what one already `knows', has anything
        new been gained? In this manner, one cannot discover new things simply by
        building upon old knowledge with pure logic and syllogisms: one must
        introduce new data in order to make progress. Bacon thus advocates
        interpretation of nature, wherein the constant acquisition of data via
        experiment leads to new discoveries, ideas, and a broader perspective.
        Progress can be achieved because with more knowledge, one gains the ability
        to do even more experiments and repeat the process; along the way, the
        practical innovations lead to the betterment of society. Bacon thus calls for
        large scale collaboration -- \emph{organized science} -- since the task of
        collecting data is long and arduous. We see this influence in the formation
        of the Royal Society, and other similar scientific bodies.

        Bacon also identifies certain obstacles to scientific progress -- false
        conceptions -- and calls these \emph{Idols of the human mind}.
        \begin{enumerate}
            \item \textit{Idols of the Tribe:} These originate from human nature; the
            human mind and perceptions are fundamentally `crooked', and it's possible
            that we are simply incapable of grasping the truths of nature on some
            basic level.

            \item \textit{Idols of the Cave:} These are present on the scale of the
            individual: preconceptions or ideas based on personal experience without
            the proper factual backing/evidence. This `personal cave of the mind'
            prevents one from looking out into the real world.

            \item \textit{Idols of the Marketplace:} These originate from public
            communication, likened to rumours and gossip. We often hear words and
            phrases in passing and have these ideas enter our minds without actually
            understanding them (social media?!).

            \item \textit{Idols of the Theatre:} These originate from dogmatic, old,
            outdated theories and philosophies; such constructions describe a
            fictional world, which may well work internally and be reasoned about,
            but there is no reason to believe that they have any bearing on reality.
        \end{enumerate}
    \end{answer}


    Rene Descartes was a Frenchman, often called the `Father of Modern Philosophy'.
    He is famed to have said `Cogito, ergo sum': I think, therefore I am. His most
    famous work is perhaps \emph{Discourse on the Method}: in one chapter, he
    describes four rules used to arrive at knowledge.
    \begin{enumerate}
        \itemsep0em
        \item Skepticism, i.e.\ never accept anything which one does not clearly know
        to be true.
        \item Divide and conquer, i.e.\ divide a problem into as many parts as one can.
        \item Prioritization, i.e.\ start with the simplest problems and work ones
        way upwards, building on what one knows.
        \item Thoroughness, i.e.\ be as general and comprehensive as possible, in
        order to avoid omissions.
    \end{enumerate}
    Descartes emphasizes deductive thought: one can produce ideas and theories from
    intuition, and draw conclusions from pure reason. Performing an experiment to
    verify these conclusions plays a smaller role, almost an auxiliary one. 

    One interesting aspect of Cartesianism is dualism, in this case the separation of
    religion and science. In other words, these fields should not impose oneself on
    the other. This proves to be an important step towards a modern science.

    Of course, Descartes made several contributions to science and mathematics
    itself; Cartesian geometry, the $x$, $y$, $z$ notation for unknowns and $a$, $b$,
    $c$ for knowns, the superscripts indicating powers are a direct result. One
    interesting perspective was that all beings, humans and animals, were
    fundamentally machines; only humans possessed a certain spark -- a soul -- that
    enabled self-awareness and consciousness. While this leads to quite cruel
    interpretations (for instance, the idea that animals do not actually feel pain),
    this model of biological machines reacting to external stimuli is quite useful,
    say when talking about reflexes. \\

    Through the works of Bacon and Descartes, we see the beginnings of modern
    science. Bacon's method of induction rightfully puts the spotlight on experiment
    (not just as a means of verification, but as a means of discovery) as the only
    real way of understanding nature. On the other hand, there is a creative aspect
    to science -- the formulation of a hypothesis or a great idea -- that cannot be
    denied, and we see that intuition and deduction certainly plays a role. Without
    this human touch, it becomes impractical to generate principles purely by
    processing immense amounts of data. There are also awkward questions which can be
    raised regarding induction, for instance: how much data should one be satisfied
    with? Establishing a principle using this scheme seems to be impossible. We have
    already seen that the study of this very problem of induction, among other
    things, lead Popper to his falsification principle.

\end{document}
