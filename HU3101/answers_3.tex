\documentclass[11pt]{article}

\usepackage[T1]{fontenc}
\usepackage{geometry}
\usepackage{amsmath, amssymb, amsthm}
\usepackage{hyperref}

\geometry{a4paper, margin=1in}

\theoremstyle{remark}
\newtheorem*{answer}{Answer}

\title{HU3101: History and Philosophy of Science}
\author{Satvik Saha}
\date{}

\begin{document}
    \noindent\textbf{IISER Kolkata} \hfill \textbf{Class Test III}
    \vspace{3pt}
    \hrule
    \vspace{3pt}
    \begin{center}
    \LARGE{\textbf{HU3101: History and Philosophy of Science}}
    \end{center}
    \vspace{3pt}
    \hrule
    \vspace{3pt}
    Satvik Saha, \texttt{19MS154} \hfill \today
    \vspace{20pt}

    \paragraph{Question 6.} Copernicus ushered in the heliocentric model of the solar
    system. And yet, there were many objections to his theory. Why?

    \begin{answer}
        Copernicus advocated a heliocentric model of the solar system, contrary to
        the traditional Ptolemaic picture of a geocentric `universe'. He proposed
        that the planets (including the Earth) orbit the Sun in fixed circular paths.

        Following are some objections to this theory.
        \begin{enumerate}
            \item \emph{Lack of observational support}: Copernicus offers very little
            observational evidence to support his hypothesis. As a result, the idea
            that the planets obey \emph{uniform circular motion} does not hold up
            (Mercury in particular has a fairly high orbital eccentricity of
            $\approx 0.2$).  This is a fairly serious deficiency in his model.

            Later, the observations of Tycho Brahe became invaluable for future
            astronomers. Kepler used these to establish a more refined heliocentric
            model.

            \item \emph{`Mystical arguments'}: Copernicus based many of his arguments
            on the works of Greek and Islamic scholars; for instance Aristarchus of
            Samos and Philolaus postulated that the Earth is in motion. He also tried
            to justify his model as superior on the grounds that it is simpler and
            more elegant. This is not sufficient to choose one model over the other.

            \item \emph{The motion of the Earth}: Aristotelian physics demands that
            the fast moving planets be made of a different, lighter material than
            that of the Earth. This is because Aristotle believed that in order for
            a body to stay in motion, a persistent `force' of sorts must be applied;
            the planets must thus be made of some `Aether' whose natural state is
            that of motion. For instance, Tycho Brahe praised the Copernican model
            for its mathematical brilliance (it dispenses with epicycles, and
            provides a simple explanation of the retrograde motion of the planets),
            but critiques it upon this very point.
            \begin{quote}
                This innovation expertly and completely circumvents all that is
                superfluous or discordant in the system of Ptolemy. On no point does
                it offend the principle of mathematics. Yet it ascribes to the Earth,
                that hulking, lazy body, unfit for motion, a motion as quick as that
                of the aethereal torches, and a triple motion at that.
            \end{quote}
            Tycho Brahe devised his own model of the solar system, identical to the
            Copernican one in every way except that the Sun orbits the Earth, while
            the other planets orbit the Sun.

            Thus, the known physical/mechanical principles of the time were
            insufficient to support the Copernican model.

            \item \emph{The vastness of space}: Tycho Brahe used measurements of the
            parallax of stars to note that if the Copernican model was correct, the
            stars would have to be extremely far away and huge (larger than the Sun),
            which seemed to be an absurdity (contradicting the Ptolemaic model) at
            the time. Copernicus defended this position on religious grounds, saying
            that the Bible does not forbid such a hypothesis; after all, such a
            majestic universe would be even more fitting for an all-powerful Creator.

            Today, we may recognize this as a success of the Copernican model.

            \item \emph{Missed insights}: In the absence of a universal
            theory of motion, there was little need for a `better' model than the
            existing geocentric ones, since both were able to explain the (then
            known) observations with comparable accuracy. While the Copernican model
            is mathematically simpler, the real value in this model is the shift in
            perspective if offers by placing all heavenly bodies at par with each
            other. This hints at a deeper driving force behind this configuration,
            later discovered and explained by Newton.

            \item \emph{Religious/theological objections}: The Copernican model
            offends many religious principles of the time, most of which regard the
            Earth as the center of the universe (due to its apparent `importance').
            For instance, the Bible seems to treat the Earth as immobile and fixed in
            space, a direct contradiction to the heliocentric model. This is a hurdle
            which Copernicus' successors, most notably Galileo, had to face.
        \end{enumerate}

        In short, the Ptolemaic model, Aristotelian principles, and the religious
        views of the time all settled upon a geocentric model; challenging this
        system would require an entirely new kind of physics to be discovered.
    \end{answer}
    
    \paragraph{Question 1.} Kepler and Galileo consolidated the Copernican
    Revolution, aided by Tycho Brahe. Discuss.

    \begin{answer}
        The Copernican Revolution is chiefly concerned with the deprecation of the
        Ptolemaic geocentric model of the solar system, and the acceptance of the
        Copernican heliocentric model. In the process, the notion of the centrality
        of the Earth was discarded, and the grand scale of the cosmos gained some
        more appreciation.

        In the context of physics, there were two challenges to be solved. First, the
        orbits of the planets had to be precisely (and simply) described. Second,
        their motion, most importantly that of the Earth (even though we don't feel
        it), had to be explained. The former was accomplished by Brahe (through
        observation), Kepler (through his mathematical interpretation), and Galileo
        (through his telescopic observations and his experiments with mechanics).

        \begin{enumerate}
            \item \emph{Tycho Brahe}: He collected an incredible amount of
            astronomical data before the advent of telescopes, upon which future
            astronomers like Kepler worked. He was also a great instrument maker. By
            discovering a new star in the Cassiopeia constellation, he refuted the
            Aristotelian idea that the sphere of stars is fixed. By discovering a
            comet whose trajectory passed through the `solid spheres of the planets',
            he showed that these Aristotelian objects do not exist; the orbits of the
            planets must be explained in some other way. However, we have seen that he
            objected to the Copernican model on the grounds that the Earth cannot
            move.

            \item \emph{Johannes Kepler}: He worked on Brahe's data, and through his
            calculations of the orbit of Mars, he realized that the planet traced an
            ellipse (with the Sun at one of the foci). This culminated in Kepler's
            Three Laws of Planetary Motion, which accurately and quantitatively
            described the orbits of all planets.

            \item \emph{Galileo Galilei}: He advanced the technology of telescopes
            substantially, and through his observations of Jupiter, he discovered
            four orbiting moons. This again contradicts the Aristotelian view that
            all heavenly bodies orbit the Earth. Furthermore, this Jupiter-moon
            system formed a small Copernican system of its own, supporting the idea
            that smaller bodies orbit larger ones, the Earth being no more special
            than the other planets in this regard. Further observations of the phases
            of Venus brought the Ptolemaic model into greater suspicion.

            Galileo is perhaps remembered mostly for his conflict with the Church: he
            championed heliocentricism which contradicted the Holy Scripture. As a
            result, he was forced to recant and live under house arrest for the rest
            of his life. His book \emph{Two New Sciences} is regarded as one of the
            greatest works of physics before Newton.
        \end{enumerate}
    \end{answer}

    \paragraph{Question 2.} What are the three basic principles of Dialectical
    Materialism? Cite at least one modern scientist's interpretation of Dialectical
    Materialism.

    \begin{answer}
        Dialectical Materialism is part of the Marxist philosophy, dialectics being
        discourse between two or more parties via reasoned argumentation. Friedrich
        Engels postulated the following three basic principles in \emph{Dialectics
        of Nature}. These principles have been used as a heuristic by Richard
        Lewontin and Stephen Jay Gould in their work on evolutionary biology.
        \begin{enumerate}
            \item \emph{The Law of Unity and Conflict of Opposites}: According to
            Gould, this represents the idea that the components of a biological
            system are often inextricably interdependent.

            \item \emph{The Law of Transformation of Quantity to Quality}: According
            to Gould, this is the idea that by applying many small inputs to a
            system, one can change its state.

            \item \emph{The Law of the Negation of the Negation}: According to Gould,
            this is the idea that complex systems cannot revert to previous states,
            this giving a direction of time (the negation of the negation need not
            always be as simple as the original).
        \end{enumerate}

        Thus, Gould uses these principles to talk about complete systems and the
        interactions of their components (which are inputs/outputs to the system) in
        a holistic way. \\

        Lewontin interprets the principles as the ideas that \begin{enumerate}
            \itemsep0em
            \item Creation and destruction are dual parts of Nature.
            \item A process influences the conditions around it, such as those
            leading to its creation/destruction.
            \item History often leaves an important trace.
        \end{enumerate}
        He uses dialectical materialism not as a method, but rather as a thumb-rule
        for avoiding dogmatism in science.
    \end{answer}

    \paragraph{Question 3.} Karl Popper was a staunch anti-Marxist. Yet, much of
    Popper's ideas were anticipated by Engels. Discuss.

    \begin{answer}
        The idea of dialectics -- conflict of opposites -- is integral to the Marxist
        philosophy detailed by Engels. In \emph{Dialectics of Nature}, Engels
        observes that everything in Nature seems to arise out of struggle between
        opposites; what is left is often a more refined version of the `inputs'. In
        science, different hypotheses play the same role. The discoveries of new
        facts introduces a pressure by which the number of satisfactory hypotheses is
        whittled down (a hypothesis which incorrectly describes known facts must be
        discarded, or at least corrected). In this manner, given sufficiently many
        data points, once can reach a distilled law or theory. Often, existing
        hypotheses are not enough to explain the facts; thus, we begin anew with a
        new set of hypotheses.

        We note that similar to Popper's falsification principle, attention is
        shifted away from facts which \emph{support} a hypothesis to facts which
        \emph{contradict} it. A hypothesis should be open to criticism and attempted
        falsification; thus, dialectics is compatible with Popper's ideas of avoiding
        dogmatism in science. Furthermore, the dialectic `evolution' of theories is
        practically identical to Popper's formula for the advancement of scientific
        knowledge, \[
            \text{PS}_1 \to \text{TT}_1 \to \text{EE}_1 \to \text{PS}_2,
        \] i.e.\ a Problem Statement gives rise to a set of Tentative Theories, which
        pass through Error Elimination. At the end, we have new and potentially more
        interesting Problem Statements, and the cycle repeats. At each cycle, science
        really begins with conjecture, not with observation.

        We recall that Popper's method of conjecture and refutation was a response to
        the Problem of Induction.  However, in \emph{Dialectics of Nature}, Engels
        too confronts this `Baconian superciliousness' of induction (forming
        hypotheses based on supporting observations). While inductive reasoning
        worked out very well for Newton, Engels criticises how the same principles of
        induction often lead to contradictory results in science in more recent
        times\footnote{``By induction it was discovered 100 years ago that crayfish
        and spiders were insects and all lower animals were worms. By induction it
        has now been found that this is nonsense ... Wherein then lies the advantage
        of the so-called inductive conclusion ... ?''}. He further illustrates how
        induction was not the only line of reasoning in use, by putting forward
        Carnot's work on thermodynamics -- this involves mentally constructing an
        ideal engine (not a physical, observable one), and drawing conclusions from
        this thought experiment. He notes how this system resembles the old `Greek
        intuition'.

        Engels does not however claim that inductive reasoning is completely
        unsuitable for science, rather that induction and deduction belong together
        and `supplement each other'. Popper also acknowledges a quasi-inductive trend
        in science; while the individual steps in scientific discovery may be
        deductive in nature, the global picture often reveals that scientific
        theories progress from specific to generic. \\

        There is another aspect of Popper's philosophy anticipated by Engels -- his
        stance on indeterminacy. Regarding the problem of free will, Popper says that
        ``freedom is not just chance but, rather, the result of a subtle interplay
        between something almost random or haphazard, and something like a
        restrictive or selective control.'' On may find in Engel's writings that he
        too denies determinism, which `disposes of chance'. To show that this is not
        how Nature works, he cites the work of Darwin and his theory of evolution in
        which species are not unchangeable. Rather, individuals accrue `infinite,
        accidental differences' over time. Thus, while the basic physical processes
        in an organism may be obey strict natural laws, chance plays an important
        role on a larger scale.
    \end{answer}

    \paragraph{Question 4.} What do you make of the debate between Newton and
    Leibniz regarding the priority of inventing calculus?

    \begin{answer}
        We state at the outset that it is very likely that Newton and Leibniz
        developed their ideas on calculus independently -- the greatest source of
        complications seem to be Newton's reluctance to actually publish his ideas at
        the appropriate time.

        On one hand, we have Newton who employed the notion of \emph{fluxions}
        (essentially time derivatives, using infinitesimals) as early as 1665; we
        only know this by perusing his manuscripts, which came to light only after
        his death. Over time, Newton revised the use of infinitesimals to something
        resembling the modern notion of a limit, thus defending against criticisms of
        the awkwardness of `division by zero'. This same method of fluxions can be
        found in a geometric form in Newton's \emph{Principia} of 1687, used in
        service of calculating tangents, extrema, and centers of gravity of curves.

        Interestingly, Newton corresponded with Leibniz in 1676 (in reply to a query
        on his work on infinite series, something that Leibniz was also working on)
        and briefly mentions his method of fluxions: albeit in cipher. Even up to
        this point, Leibniz was oblivious to Newton's actual work on calculus.

        Leibniz was working on his own \emph{differential calculus} by 1675. He
        visited London in 1676 and John Collins, who saw that Leibniz was interested
        in series, prepared a compendium of work including (what little that had been
        published) of Newton's fluxional calculus. This is perhaps the weakest link
        in the case for Leibniz; he says that these notes contained nothing new to
        him. Continuing his correspondence with Newton, he explained his method of
        differential calculus in a letter in 1677 (this is the source of allegations
        that Newton plagiarized Leibniz). His full work on calculus was published in
        a treatise in 1684.

        We note that Leibniz visited Paris in 1673 and 1675 which was when he was
        exposed to the works of British mathematicians including Gregory (on series)
        and Newton. However, these may be put aside on the grounds that he was far
        too young at the time to properly appreciate or absorb those ideas.

        The dispute seems to have only begun proper after Newton's publication of his
        \emph{Principia}, where he explicitly claims to have developed fluxions
        decades ago. Various mathematicians began taking sides and accusing the other
        party of plagiarism; however, up to 1700, neither Newton nor Leibniz were
        directly involved.

        It is likely that Leibniz only became fully aware of the power of Newton's
        fluxional calculus when the latter published two treatises in 1704; this made
        him realize that Newton's methods were practically identical to his own. On
        the other side, supporters of Newton saw this as further evidence that
        Leibniz's work was an imitation of Newton. This culminates in 1712, by which
        time Newton was very much involved in the accusations against Leibniz. The Royal Society
        (of which Newton was the president) began set up a committee to look into
        these accusations and settle this debate. Their report in favour of Newton
        was published as the \emph{Commercium Epistolicum} in 1713.

        Today, with the advantage of being able to go through their early
        manuscripts, we give both Newton and Leibniz credit for independent
        discovery. Perhaps the most banal place where they left their mark is in
        notation, both of which are in common use even today. \[
            \dot{f} \;\cong\; \frac{df}{dt}, \qquad y' \;\cong\; \frac{dy}{dx}.
        \]
    \end{answer}

    \paragraph{Question 5.} Show how the centre of science gradually shifted from
    Eastern Europe to Western Europe.

    

\end{document}
