\documentclass[11pt]{article}

\usepackage[T1]{fontenc} 
\usepackage{geometry}
\usepackage{amsmath, amssymb, amsthm}
\usepackage{hyperref}

\geometry{a4paper, margin=1in}

\theoremstyle{remark}
\newtheorem*{answer}{Answer}

\title{HU3101: History and Philosophy of Science}
\author{Satvik Saha}
\date{}

\begin{document}
    \noindent\textbf{IISER Kolkata} \hfill \textbf{Class Test VII}
    \vspace{3pt}
    \hrule
    \vspace{3pt}
    \begin{center}
    \LARGE{\textbf{HU3101: History and Philosophy of Science}}
    \end{center}
    \vspace{3pt}
    \hrule
    \vspace{3pt}
    Satvik Saha, \texttt{19MS154} \hfill \today
    \vspace{20pt}

    \paragraph{Question 1.} Jagadish Chandra Bose left microwave research and opted
    for plant physiology. Do you think nationalistic sentiments coupled with
    revivalist Hindu philosophy had a role to play in this change over?
    
    \begin{answer} The following is an extract from Jagadish Chandra Bose's discourse
    at the Royal Institution, May 1901. We highlight its significance as Bose's first
    real foray into the world of physiology. After demonstrating each of his results
    by a series of experiments, he concludes ---
        \begin{quote}
            I have shown you this evening autographic records of the history of
            stress and strain in the living and non-living. How similar are the
            writings! So similar indeed that you cannot tell one apart from the
            other \dots

            Amongst such phenomena, how can we draw a line of demarcation, and say,
            here the physical ends, and the physiological begins? Such absolute
            barriers do not exist.

            Do not these records tell us of some property of matter common and
            persistent? Do they not show us that the responsive processes, seen in
            life, have been foreshadowed in non-life? -- that the physiological is
            related to the physicochemical? -- that there is no abrupt break, but a
            uniform and continuous march of law?

            \dots

            It was when I came upon the mute witness of these self-made records, and
            perceived in them one phase of a pervading unity that bears within it all
            things — the mote that quivers in ripples of light, the teeming life upon
            our earth, and the radiant suns that shine above us — it was then that I
            understood for the first time a little of that message proclaimed by my
            ancestors on the banks of the Ganges thirty centuries ago --- 

            `They who see but one, in all the changing manifoldness of the universe,
            unto them belongs Eternal Truth, unto none else, unto none else!'
        \end{quote}
    \end{answer}

    In order to explain Bose's seemingly sudden change in scientific interest, we
    must first look at earlier events. Bose was perhaps the first true experimental
    scientist of India, given full recognition by his peers. This was extremely
    impressive, given the prejudices of the `Western world' regarding the capacity of
    an Indian to carry out exact science. 
    \begin{quote}
        Intellectual acuteness in Metaphysics and Languages had always been frankly
        acknowledged, but it was assumed that India had no aptitude for the exact
        methods of science. For science, therefore, India must look to the West for
        her teachers.
    \end{quote}
    Bose had indeed completed his studies abroad, and even worked under the
    illustrious Lord Rayleigh; soon after returning to the Presidency College,
    University of Calcutta however, he began pursuing his original work. Seemingly
    unhindered by a lack of proper laboratory equipment, he designed and built his
    own. His experimental work pioneered the field of `electrical waves',
    revolutionising radio and microwave optics\footnote{Bose showed that these
    electrical waves obeyed all those properties of a beam of ordinary light:
    reflection, refraction, total reflection, polarisation, etc. Another notable
    aspect of his research was that he was the first to use semiconductor junctions
    to detect these waves.}. His work brought him international acclaim in the
    scientific circles; there is an anecdote in which after Bose delivered a
    discourse on his paper, Lord Kelvin himself ``not only broke into the warmest
    praise, but limped upstairs into the ladies' gallery and shook Mrs.\ Bose by both
    hands, with glowing congratulations on her husband's brilliant work.''


    During his work on electrical wave receivers, Bose observed a curious anomaly;
    the receiver (essentially metallic springs) gradually lost its sensitivity over
    time. This was analogous to `fatigue'. Indeed, allowing the receiver to rest for
    some time restored its sensitivity. However, there was an even more perplexing
    discovery: letting the receiver rest for too long (several days) would render it
    insensitive again! The only was to restore its sensitivity was by applying an
    electric shock. This required entirely an entirely new theory to explain. Bose
    began testing the electrical sensitivities of metals, non-metals, and metalloids
    in various combinations to build up some sort of classification. This `electric
    touch' was found to be periodic (with respect to atomic weight). Some metals like
    potassium stood out, acting precisely in the opposite way of others like iron.
    Bose deduced that this phenomenon was governed by the chemical nature of the
    substance, and arose from molecular changes\footnote{Today we might recognize
    these phenomena as belonging to semiconductor electronics.}. This culminates in
    the paper \emph{On the Strain Theory of Photographic Action}, tying up many
    apparently unrelated phenomena. The action of electrical waves on his receivers
    was exactly that of light on a photographic plate. In his investigations, Bose
    also developed his `artificial retina', showing that the range of human vision is
    based on precisely this electrical sensitivity of the retina, like in a
    photographic plate.

    In the above experiments, we note a certain parallelism between the metallic and
    the living (the concept of fatigue for instance) in the context of electrical
    response. Soon, Bose began investigating living tissue directly, and the first
    step was the \emph{Response of Inorganic and Living Matter}. Here, he compares
    the reaction curve of muscle tissue in response to an electrical stimulus, with
    that of ferric oxide. Both exhibit the same phenomena of fatigue, rest, recovery,
    and reactivation. He also drew an analogy with poisoning of muscles with the
    introduction of impurities in metal like potassium. 
    \begin{quote}
        In all the phenomena above described continuity is not broken. It is
        difficult to draw a line and say `here the physical phenomenon ends and the
        physiological begins' or `that is a phenomenon of dead matter, and this is a
        vital phenomenon peculiar to the living.' These lines of demarcation would be
        quite arbitrary.
    \end{quote}
    Bose's work seemed to draw appreciation from physicists, but not from physiologists.
    Undeterred he continued his research and the idea came to him that in order to
    establish this `continuity' between the inorganic and the organic, he ought to be
    looking at plant matter. Bose gathered leaves from his garden, vegetables from
    the greengrocers, and repeated his experiments on them. Now, he observed similar
    electrical responses in metal, plant, and animal. The responses in the living
    tissues would disappear on applying poisons, and it could be revived up to a
    threshold (beyond which the tissue dies). Bose now did something strange: he
    introduced these poisons to various metals, and tested their responses. Lo and
    behold, their behaviour was extremely similar to that of living tissue!
    \begin{quote}
        So striking was this correspondence, that one day when Bose was beginning to
        show his records to Sir Michael Foster, the veteran physiologist of
        Cambridge, the latter picked one up and said,

        ``Come now, Bose, what is the novelty in this curve? We have known it for at
        least the last half-century.''

        ``What do you think it is?'' said Bose.

        ``Why, a curve of muscle response, of course.''

        ``Pardon me; it is the response of metallic tin.''

        ``What!'' said Foster, jumping up -- ``Tin! Did you say tin?''

        On explanation, his wonder knew no bounds; and he hurried Bose to make a
        communication to the Royal Society, which he (then Secretary) offered to
        communicate.
    \end{quote}
    Bose presented these very findings in his Friday Evening Discourse at the Royal
    Institution, May 1901; we have quoted an extract from this in the very beginning.
    \\

    The above events have been presented to demonstrate that Bose's interest and
    research into plant physiology seem to arise naturally from his work on electric
    waves in physics. Besides, there doesn't appear to be any need to justify one's
    research subjects since scientific curiosity ought to suffice; what is most
    important (but out of the scope of this topic) is that the results should stand
    for themselves. Still, this picture would be incomplete without accounting for
    Bose's personal philosophy, the driving force behind his work. \\

    The reference to the Vedas in Bose's speech represents his affinity towards
    ancient Indian spiritual tradition. Indeed, many thinkers of the time, say Swami
    Vivekananda and Rabindranath Tagore to name the most influential, encouraged this
    revivalism of Hindu philosophy. This was meant to support nationalistic sentiment
    during this period of India's freedom struggle. Bose himself came to subscribe to
    the Advaita Vedanta philosophy, which described an all-pervading consciousness,
    the Brahman. This explains his (increasingly frequent with time) allusions to the
    `unity of life and the non-living' or `unity in apparent diversity' in the
    various phenomena he observed. In the words of V.A. Shepard, 
    \begin{quote}
        Bose's insistence on the unity of the living and non-living arose from a
        deeply held philosophical position, Vedanta in inspiration, a monism that
        regarded the world as a single unified entity, where mind and matter were
        aspects of the same thing.
    \end{quote}
    Thus as Bose's scientific claims began to dip into metaphysics over the years,
    this question of philosophical bias or loss in objectivity arose. The idea of
    consciousness is perhaps the heaviest topic in philosophy, but here too Bose made
    bold claims.
    \begin{quote}
        \dots even a speck of protoplasm has a faculty of choice.
    \end{quote}
    \begin{quote}
        \dots all matter was one, how unified all life was \dots there was no such
        thing as brute matter, but that spirit suffused matter in which it was
        enshrined.
    \end{quote}
    \begin{quote}
        Consciousness and sensation are thus regarded as inseparably associated with
        the nervous system and nervous reaction. If this be so, then my recent
        scientific results prove beyond a shadow of doubt that many plants possess
        not merely a rudimentary, but a highly elaborated nervous system.
    \end{quote}

    There is another sense in which Bose searched for unity in his work --- the unity
    of scientific disciplines. He emphasized that science is in search of knowledge
    which is ultimately whole and entire, requiring an interdisciplinary approach to
    fully comprehend. Bose said 
    \begin{quote}
        \dots in the West, the prevailing tendency at the moment is, after a period of
        synthesis, to return upon the excessive sub-division of learning. The result
        of this specialisation is rather to accentuate the distinctiveness of the
        various sciences, so that for a while the great unity of all tends perhaps to
        be obscured.
    \end{quote}
    Thus, the general sentiment was that the `Eastern' approach of scientists like
    Bose was better suited to the development of science. In this pursuit, Bose spoke 
    \begin{quote}
         There will soon rise a Temple of Learning where the teacher cut off from
         worldly distractions would go on with his ceaseless pursuit after truth, and
         dying, hand on his work to his disciples. Nothing would seem laborious in
         his inquiry; never is he to lose sight of his quest, never is he to let it
         go obscured by any terrestrial temptation. For he is the Sanyasin spirit,
         and India is the only country where so far from there being a conflict
         between science and religion, knowledge is regarded as religion itself.
    \end{quote}
    These efforts were realized with the establishment of the Bose Institute,
    Kolkata in 1917. \\


    Bose's work and ideas were greatly admired by the aforementioned scholars,
    Vivekananda and Tagore, who sent addresses of praise soon after Bose revealed his
    first results (on the unity of metal and animal tissue). Sister Nivedita, a
    disciple of Vivekananda, worked closely with Bose editing his book
    \emph{Responses in the Living and the Non-Living}. Vivekananda must surely have
    had some influence on Bose, with his declaration that 
    \begin{quote}
        It seems to us, and to all who care to know, that the conclusions of modern
        science are the very conclusions the Vedanta reached ages ago; only in modern
        science they are written in the language of matter.
    \end{quote}


    In conclusion, we can identify three factors behind Bose's investigations into
    plant physiology. First, his research in physics combined with natural curiosity
    lead him along this line of enquiry. Second, his struggles with Western prejudice
    in the field of science motivated him to push back and uphold Eastern ideas such
    as unity among the science (hence his complete lack of hesitation towards his
    interdisciplinary research). Third, support from revivalists of Vedanta
    philosophy inspired him to continue his work in plant physiology with renewed
    confidence.


\end{document}
