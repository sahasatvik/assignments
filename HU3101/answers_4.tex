\documentclass[11pt]{article}

\usepackage[T1]{fontenc}
\usepackage{geometry}
\usepackage{amsmath, amssymb, amsthm}
\usepackage{hyperref}

\geometry{a4paper, margin=1in}

\theoremstyle{remark}
\newtheorem*{answer}{Answer}

\title{HU3101: History and Philosophy of Science}
\author{Satvik Saha}
\date{}

\begin{document}
    \noindent\textbf{IISER Kolkata} \hfill \textbf{Class Test IV}
    \vspace{3pt}
    \hrule
    \vspace{3pt}
    \begin{center}
    \LARGE{\textbf{HU3101: History and Philosophy of Science}}
    \end{center}
    \vspace{3pt}
    \hrule
    \vspace{3pt}
    Satvik Saha, \texttt{19MS154} \hfill \today
    \vspace{20pt}


    \paragraph{Question 1.} What are the most enduring contributions of Aristotle in
    science?

    \begin{answer}
        Aristotle was one of the great thinkers of the Ancient Greek civilization.
        In the context of science, he is mostly known for ideas which were later
        disproved by future scientists (say Galileo, who showed that objects in
        free-fall do so at the same rate, regardless of mass). However, his
        methodology and ways about thinking about problems in science were important
        stepping stones.

        Following are some areas of science where his contributions stand out.
        \begin{enumerate}
            \item \textit{Natural science:} At the heart of Aristotle's natural
            philosophy were the four classical elements: air, earth, fire, water. To
            this, he added a fifth: aether, which constitutes the heavenly bodies.
            Each of these elements is associated with a quality.
            \begin{enumerate}
                \item \textit{Air} is hot and wet.
                \item \textit{Earth} is cold and dry. 
                \item \textit{Fire} is hot and dry.
                \item \textit{Water} is cold and wet.
                \item \textit{Aether} is weightless and unchangeable.
            \end{enumerate}
            The first four form the terrestrial, `corruptible' sphere in the
            universe: the Earth at its center. Here, the elements can change from one
            form to another, since they share certain properties. For instance, water
            can change into air (vapour) by heating, or into earth by drying.

            Surrounding it are the concentric, celestial, crystal spheres, on which
            the Moon, Sun, the planets, and finally the firmament of the stars are
            embedded. These are in constant circular motion at a constant rate, as it
            is the aether's tendency to remain in motion.

            \item \textit{Physics:} Aristotle uses the formulation of the elements to
            explain the motion of objects, which he divides into two types: natural
            and unnatural. Natural motion is that which happens of its own accord,
            without being forced in any (apparent) way. Recall how he explained the
            natural motion of the heavenly bodies by invoking the tendency of aether
            to be in circular motion. Similarly, he said that earth and water tend to
            move downwards, while air and fire move upwards. In this way, the natural
            motion of objects is related to the ratios of the four elements
            comprising it. Thus, objects of earth and water move down towards the
            center of the Earth (indeed, the universe), explaining `gravity'. In
            addition to natural motion, these elements have related natural places:
            earth and water at the center of the universe, air above water (bubbles
            rise in water), and fire higher still (but not as high as the Moon's
            celestial sphere). The natural motion of these elements is thus their
            attempts at reaching their natural place; they have a certain potential
            to do so, which when actualised is seen as motion.

            Aristotle said that if a heavier object and an otherwise identical
            lighter object are dropped, the heavier one would wall faster. In other
            words, the speed of free-fall is directly proportional to the object's
            weight, inversely to the surrounding medium. When coming to natural
            motion unnatural motion, Aristotle said that a heavier object takes more
            force to move; the greater the force, the faster the motion.
            Furthermore, an object's natural state is one of rest, so upon removing
            the applied force, the motion will cease. Put together, Aristotle says
            that an object in a vacuum would fall arbitrarily/infinitely fast, and
            thus argues that such a vacuum is impossible (things such as air would
            move into this vacuum and fill it up).

            Today, we know that these formulations are incorrect; objects in
            free-fall move together regardless of weight, and the apparent `natural
            state of rest' is the work of friction (which Aristotle was unaware of).
            It seems simple enough to disprove these via experimentation, yet it took
            well over a millennium to actually do so (Galileo and Newton stand out in
            the context of mechanics). This is partly because of Aristotle's
            methodology of arriving at these principles via common experience and
            reason alone, without emphasis on rigorously verifying these results via
            experiments in the modern sense.

            \item \textit{Teleology:} Related to the idea of motion, Aristotle talks
            about his \textit{four causes}: why do things happen/change? The answers
            to such questions have these four aspects.
            \begin{enumerate}
                \item \textit{Material cause:} The physical matter comprising an
                object.
                \item \textit{Formal cause:} The shape/arrangement/appearance of an
                object.
                \item \textit{Efficient cause:} An (external) agent which interacts
                with an object.
                \item \textit{Final cause:} The purpose an object exists to serve,
                often called \textit{telos}.
            \end{enumerate}
            To illustrate this, consider a table: its existence is determined by wood
            (material), the actual shape of the table (form), the carpenter who built
            it with his tools (agent), and its role as something to dine on top of
            (purpose).

            \item \textit{Biology:} Aristotle's chief contribution is his systematic
            approach to classification of life-forms, which earned him the title of
            ``father of zoology'' -- this is perhaps his most enduring contribution
            to science. He made observations for two years in the island of Lesbos,
            and accumulated massive amounts of data in \textit{History of Animals},
            \textit{Generation of Animals}, \textit{Movement of Animals}, and
            \textit{Parts of Animals}. He also took into account observations made by
            others, such as fishermen and sailors. He also carried out dissections in
            order to carefully study the structure of animals. Some of his anatomical
            observations (for instance the hectococtyl arm of cephalopods) were not
            fully believed/appreciated until very recently (the 19th century in this
            instance). Armed with this data, he broadly classified animals into those
            with blood (vertebrates) and those without (invertebrates). These were
            further divided into categories such as Man (an indivisble form), birds
            (of which he distinguished around 500 species), Cetaceans, Fish
            (egg-laying and non-egg-laying), Crustaceans, Insects, etc. These fifteen
            or so categories were each given qualities (hot/cold/wet/dry), a type of
            `soul', and arranged in a hierarchy with Man at the top and Plants and
            Minerals at the bottom.

            Aristotle identified five major biological processes, essential to the
            functioning of an organism. 
            \begin{enumerate}
                \item \textit{Metabolism:} The intake of food (matter), which is used
                to by the organism to live/grow/reproduce.
                \item \textit{Temperature regulation:} How the blood is heated and
                cooled by metabolism and the outside air, with the process driven by
                the heart and lungs.
                \item \textit{Information processing:} The ability of an organism to
                sense its surroundings, process it, and act accordingly.
                \item \textit{Inheritance:} The transmission of the parents'
                characteristics to their offspring.
                \item \textit{Embryonic development:} The formation and development
                of an embryo.
            \end{enumerate}
            Together, these form a system called a `soul'
            (this is purely biological in nature). A vegetative soul such as in
            plants is associated with reproduction/growth, a sensitive sould such as
            in animals with sensation/movement. Humans also have a rational soul,
            associated with thought.

            Aristotle also observed many simple patterns in his data. For instance,
            smaller animals like mice have significantly shorter lifespans, shorter
            gestation periods, and larger brood sizes compared to larger ones like
            elephants (we know that these quantities vary logarithmically). We
            highlight again that while Aristotle did not go out of his way to carry
            out motivated experiments, his system of gathering lots of data and
            finding correlations was quite fruitful, and may even be called
            scientific in some sense (this is often the first stage in scientific
            enquiry, i.e.\ looking at patterns).
        \end{enumerate}
    \end{answer}

    \paragraph{Question 3.} Despite the glorious achievements of science in the 19th
    century, the century ended with a theoretical pessimism. Why?

    \begin{answer}
        The 19th century was indeed a glorious time for science and technology, with
        world-changing advancements in all fields: physics, chemistry, biology, and
        engineering to name a few. First, we explore some of these below.
        \begin{enumerate}
            \item \textit{Classical mechanics:} Newton's Laws proved to be an
            incredible achievement in physics, and these principles were tested and
            used extensively; as Newton's predictions continued to be proved right
            time and again, this inspired great confidence in these theories. Two
            important reformulations of Newtonian mechanics were Lagrangian mechanics
            (earlier in 1788) and Hamiltonian mechanics (in 1833).

            Here, we wish to emphasize that with the success of classical mechanics,
            the general opinion in this era was that Newton was right beyond a shadow
            of a doubt. This air of confidence (at least in the field of physics) is
            what defined this century.

            \item \textit{Electromagnetism:} The connection between electrical and
            magnetic phenomena was first observed by Oersted: a moving electrical
            current (say in a wire) can deflect the needle of a magnetic compass.
            This was followed by Faraday's great and numerous experiments (often
            public demonstrations which were incredibly popular), by which he
            discovered the reverse principle of electromagnetic induction: a moving
            magnet can induce a flow of current. His assistant James Clerk Maxwell
            went on to formalize these ideas mathematically, and in doing so united
            electricity, magnetism -- as well as optics! His theory indicated the
            possibility of self-propagating electromagnetic waves (oscillations of
            electric and magnetic fields), and he calculated that these waves move at
            a speed which was very close to the (then known) speed of light. Thus, he
            proposed that light itself is nothing more than an electromagnetic wave.
            Maxwell's equations are known as one of the most beautiful in physics.
            \begin{align*}
                \epsilon_0 \nabla\cdot \mathbf{E} &= \rho & 
                \nabla\times \mathbf{E} &= -\frac{\partial \mathbf{B}}{\partial t} \\
                \nabla \cdot \mathbf{B} &= 0 &
                \epsilon_0c^2\nabla\times \mathbf{B} &= \mathbf{j} + \epsilon_0\frac{\partial
                \mathbf{E}}{\partial t}
            \end{align*}

            \item \textit{Thermodynamics:} Sadie Carnot, known today as the `father of
            thermodynamics', published \textit{Reflections on the Motive Power of
            Fire} in 1824, where he talks about heat, energy, and engine efficiency.
            Instead of looking at a steam engine with all its complication, he
            studies an abstract `heat engine'. With this idealized picture, he showed
            that the most efficient engine possible is the Carnot cycle: its
            efficiency is a function only of the temperatures of the source and sink
            of heat, and is always strictly less than one. This work was later
            extended by Clapeyron, and entered the body of work later contributed by
            other greats in the field.

            The fact that heat is not a `fluid' as advocated by the caloric theory,
            but is instead produced by motion was demonstrated by James Joule. He
            calculated a relation between a unit of heat, and a unit of mechanical
            energy: this \emph{mechanical equivalent of heat} says that 1 cal
            $\approx$ 4.15 joule (in modern units). Later, Helmholtz wrote on the
            conservation of energy in his 1847 paper, examining various phenomena in
            mechanics, heat, light, electricity, magnetism in the context of this
            energy.

            Lord Kelvin (William Thomson) was also a key contributor in this field.
            For instance, he developed the absolute temperature scale which bears his
            name, he studied and improved the work of Carnot and Clausius, and
            collaborated with Joule (the Joule-Thomson effect is well a known
            phenomenon). He also contemplated the second law of thermodynamics, and
            spoke of the heat death of the universe. Later, Ludwig Boltzmann, Willard
            Gibbs, and Maxwell developed the field of statistical mechanics, which
            offered another interpretation of entropy.

            \item \textit{Periodic Table of Elements:} With new element after element
            being discovered in the field of chemistry, there were many attempts at
            classifying them. Notable is Newlands' law of octaves, where he used the
            fact that when elements are arranged in order of increasing atomic mass,
            several properties repeat in groups of eight. Dmitri Mendeleev devised
            the first truly useful periodic table of elements, which in a slightly
            modified form is used even today. One of its merits was its predictive
            power: Mendeleev saw that in order to abide by the rules of his table
            (periodic properties down a group, increasing atomic weight across a
            period), he would have to leave gaps. He declared that these gaps
            represented new elements which had not been discovered yet, and made
            stunningly accurate predictions of their properties, which were confirmed
            within the end of the century. This revolutionized the study of chemistry.

            \item \textit{Theory of Evolution:} The introduction of the Theory of
            Evolution by Natural Selection by Charles Darwin and Alfred Wallace was
            revolutionary in the field of biology: Darwin's \emph{On the Origin of
            Species} of 1859 was an incredibly popular book. The theory speaks of how
            individuals in a species can accumulate small, inheritable changes over
            long periods of time; those changes that are favourable to the survival
            and reproduction of the individual survive while unfavourable ones do
            not. Thus, the characteristics of a species gradually change over time.
            Furthermore, different groups of individuals from the same species may
            undergo a different route of changes (perhaps they end up developing in
            different geographic locations) until they are sufficiently distinct to
            be regarded as different species. This is how multiple species can
            originate from common ones. A corollary of Darwin's theory was that
            mankind and today's primates descended from a common ancestor. These
            ideas were greatly opposed by Christian fundamentalists, as their
            religion seems to imply that all modern species were present at the time
            of Creation, unchanged over time. Today, we know that the theory of
            evolution is indispensable to biology, appearing in some form or another
            as a guiding principle in every corner.

            \item \textit{Engineering:} The 19th century say many technological and
            engineering innovations; we mention just a few, closely related to our
            discussion on physics. Siemens used the steam engine coupled to a dynamo
            for power generation. Reversing this principle and improving it, Ferraris
            and Tesla create their own induction motors. Parsons built the first
            practical large steam turbines for generating electricity.
        \end{enumerate}

        By the end of the century, there seemed to be a general feeling that all the
        basic principles of nature had been discovered: first Newton's classical
        mechanics, then thermodynamics and electromagnetism in physics, the periodic
        table in chemistry, and the theory of evolution in biology. In the words of
        one writer, the work of science was at an end, with \begin{quote}
            \dots only a few turrets and pinnacles to be added, a few roof bosses to
            be carved.\footnote{See Bill Bryson's \emph{A Short History of Nearly
            Everything}.}
        \end{quote}
        Quite a few scientists believed that all that remained for science to do was
        to make better and better measurements.

        In physics at least, there were three major unresolved problems. \begin{enumerate}
            \itemsep0em
            \item The nature of the all pervading luminiferous ether, the medium for
            the propagation of light.
            \item The ultraviolet catastrophe, a failure of the Rayleigh-Jean law.
            \item The anomalous precession of Mercury, a discrepancy of 42 arcseconds
            per century.
        \end{enumerate}
        The first was shown by Michelson and Morley (the former strongly shared the
        sentiment that all natural phenomena trace back to this ether) to be
        non-existent. The second demanded the power of quantum mechanics; the third
        demanded the power of general relativity. Both of these are defining theories
        of modern physics.

        Even in biology, the fundamental mechanism behind inheritance -- the role of
        the DNA molecule -- was yet to be discovered and understood (DNA itself had
        been isolated by Miescher in 1869, but its purpose had not been determined).
        Another problem was that evolution of species necessitated immense amounts of
        time; so do various geological processes such as mountain building or the
        passage of glaciers. There was an abundance of palaeontological evidence
        supporting the idea that the Earth was much older than previously thought;
        yet eminent scientists such as Lord Kelvin estimated its age at no more than
        80 million years, with his final prediction at 20 million years. This is
        because there was no known process/source of energy that could keep the Sun
        burning for more longer that (the source of the Sun's energy was thought to
        be gravitational collapse).  This changed with the discovery of radiation by
        Becquerel and Rutherford, then the subsequent discovery of nuclear fusion. It
        is interesting to note that the size of the Earth had been measured by
        Eratosthenes around 200 BC; the weight of the Earth had been properly
        measured by Cavendish around 1800 AD; the age of the Earth was only
        accurately determined by Clair Patterson to be around 4.5 billion years
        (using radiometric dating of meteorites) in 1956.
    \end{answer}
    

\end{document}
