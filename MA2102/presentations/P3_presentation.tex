\documentclass[handout]{beamer}
\usetheme{metropolis}
\setbeamercovered{transparent}

\usepackage{amsmath, amssymb, amsthm}
\usepackage{bm}
\def\u{\bm{u}}
\def\v{\bm{v}}
\def\w{\bm{w}}
\def\L{\mathcal{L}}
\def\dim{\operatorname{dim}}
\def\spn{\operatorname{span}}

\title{
        Term presentation \\
        Problem 3
}
\author{Satvik Saha, 19MS154}
\institute{
        MA2102: Linear Algebra I \\
        Indian Institute of Science Education and Research, Kolkata
}
\date{\today}

\begin{document}
        \maketitle

        \begin{frame}{Problem statement}
                Let $V$ and $W$ be vector spaces over the same field $F$.
                Show that the set $\L(V, W)$, consisting of linear maps from $V$ to $W$, is a vector space.
                If $V$ and $W$ are finite dimensional, then find the dimension of $\L(V, W)$.
        \end{frame}

        \begin{frame}{Preliminaries}
                A vector space $V$ over a field $F$ is a set equipped with a binary operation $+\colon V\times V \to F$ called addition, 
                and an operation $\cdot\colon F\times V \to V$ called scalar multiplication, such that
                \begin{enumerate}
                        \item $\u + \v \in V$, for all $\u, \v \in V$.
                        \item $\lambda\u \in V$, for all $\u \in V$, $\lambda \in F$.
                        \item $\u + \v = \v + \u$, for all $\u, \v \in V$.
                        \item $(\u + \v) + \w = \u + (\v + \w)$, for all $\u, \v, \w \in V$.
                        \item There exists $\mathbf{0} \in V$ such that $\mathbf{0} + \v = \v$ for all $\v \in V$.
                        \item For all $\v \in V$, there exists $\u \in V$ such that $\v + \u = \mathbf{0}$. We denote $\u = -\v$.
                        \item $\lambda(\u + \v) = \lambda \u + \lambda \v$, for all $\u, \v \in V$, $\lambda \in F$.
                        \item $(\lambda \mu)\v = \lambda(\mu\v)$ for all $\v \in V$, $\lambda, \mu \in F$.
                        \item $(\lambda + \mu)\v = \lambda\v + \mu\v$, for all $\v \in V$, $\lambda, \mu \in F$.
                        \item There exists $1 \in F$ such that $1\v = \v$ for all $\v \in V$.
                \end{enumerate}
        \end{frame}

        \begin{frame}{Preliminaries}
                A basis of a vector space $V$ over a field $F$ is a set of linearly independent vectors in $V$ such that any element of $V$
                can be written as a finite linear combination of them. \\~\\

                The dimension of a vector space $V$ is equal to number of elements in a basis of $V$. This is well defined by the 
                Replacement Theorem, which guarantees that any two bases will have the same size.
        \end{frame}

        \begin{frame}{Preliminaries}
                A linear map between the vector spaces $V$ and $W$ is a map $T\colon V \to W$ such that for all $\u, \v \in V$ and $\lambda \in F$,
                \begin{align*}
                        T(\u + \v) \,&=\, T(\u) + T(\v), \\
                        T(\lambda \v) \,&=\, \lambda T(\v).
                \end{align*}
        \end{frame}

        \begin{frame}{$\L(V, W)$ as a vector space}
                Let $T, T_1, T_2\colon V \to W$ be linear maps and let $\lambda \in F$. We define addition and scalar multiplication 
                on $\L(V, W)$ as follows.
                \begin{align*}
                        (T_1 + T_2)(\v) \,&=\, T_1(\v) + T_2(\v) &\text{ for all } \v \in V, \\
                        (\lambda T)(\v) \,&=\, \lambda T(\v) &\text{ for all } \v \in V.
                \end{align*}
        \end{frame}

        \begin{frame}{$\L(V, W)$ as a vector space: Closure}
                $T_1 + T_2$ and $\lambda T$ are both linear maps in $\L(V, W)$.
                \begin{align*}
                        (T_1 + T_2)(\u + \mu\v) &= T_1(\u + \mu\v) + T_2(\u + \mu\v) \\
                                &= T_1(\u) + \mu T_1(\v) + T_2(\u) + \mu T_2(\v) \\
                                &= (T_1 + T_2)(\u) + \mu (T_1 + T_2)(\v). \\\\
                        (\lambda T)(\u + \mu \v) &= \lambda T(\u + \mu \v) \\
                                &= \lambda T(\u) + \lambda\mu T(\v)) \\
                                &= (\lambda T)(\u) + \mu (\lambda T)(\v).
                \end{align*}
        \end{frame}

        \begin{frame}{$\L(V, W)$ as a vector space: Commutativity and Associativity of addition}
                For all $\v\in V$, note that the commutativity of addition in $W$ gives
                \[
                        (T_1 + T_2)(\v) = T_1(\v) + T_2(\v) = T_2(\v) + T_1(\v) = (T_2 + T_1)(\v).
                \]
                The associativity of addition in $W$ gives
                \begin{align*}
                        ((T_1 + T_2) + T_3)(\v) &= T_1(\v) + (T_2 + T_3)(\v) = T_1(\v) + T_2(\v) + T_3(\v), \\
                        (T_1 + (T_2 + T_3))(\v) &= (T_1 + T_2)(\v) + T_3(\v) = T_1(\v) + T_2(\v) + T_3(\v).
                \end{align*}
                Thus, $T_1 + T_2 = T_2 + T_1$ and $(T_1 + T_2) + T_3 = T_1 + (T_2 + T_3)$.
        \end{frame}

        \begin{frame}{$\L(V, W)$ as a vector space: Existence of an additive identity and inverses}
                Define the linear map $\mathbf{0}_\L\colon V \to W$, $\v \mapsto \mathbf{0}_W$. For any $T \in \L(V, W)$, for all $\v \in V$.
                \[
                        (\mathbf{0}_\L + T)(\v) = \mathbf{0}_\L(\v) + T(\v) = \mathbf{0}_W + T(\v) = T(\v).
                \]
                Define $T'\colon V \to W$, $\v \mapsto -T(\v)$. Then,
                \[
                        (T + T')(\v) = T(\v) + T'(\v) = T(\v) - T(\v) = \mathbf{0}_W = \mathbf{0}_\L(\v).
                \]
                Thus, $\mathbf{0}_\L + T = T$ and $T + T' = \mathbf{0}_\L$.
        \end{frame}
        
        \begin{frame}{$\L(V, W)$ as a vector space: Distributivity of scaling}
                For $\lambda, \mu \in F$, for all $\v \in V$,
                \begin{align*}
                        (\lambda (T_1 + T_2))(\v) &= \lambda (T_1 + T_2)(\v) \\
                                &= \lambda(T_1(\v) + T_2(\v)) \\
                                &= \lambda T_1(\v) + \lambda T_2(\v) \\
                                &= (\lambda T_1)(\v) + (\lambda T_2)(\v) \\
                                &= (\lambda T_1 + \lambda T_2)(\v). \\\\
                        ((\lambda + \mu)T)(\v) &= (\lambda + \mu)T(\v) \\
                                &= \lambda T(\v) + \mu T(\v) \\
                                &= (\lambda T)(\v) + (\mu T)(\v) \\
                                &= (\lambda T + \mu T)(\v).
                \end{align*}
                Thus, $\lambda (T_1 + T_2) = \lambda T_1 + \lambda T_2$ and $(\lambda + \mu)T = \lambda T + \mu T$.
        \end{frame}
        
        \begin{frame}{$\L(V, W)$ as a vector space: Scaling}
                For $\lambda, \mu \in F$, for all $\v \in V$,
                \begin{align*}
                        ((\lambda \mu)T)(\v) &= (\lambda \mu) T(\v) \\
                                &= \lambda (\mu T(\v)) \\
                                &= \lambda (\mu T)(\v) \\
                                &= (\lambda (\mu T))(\v).
                \end{align*}
                Thus, $(\lambda \mu) T = \lambda (\mu T)$. \\~\\
                
                Pick the scalar $1 \in F$ which satisfies $1\w = \w$ for all $\w \in W$. Then
                \[
                        (1T)(\v) = 1(T(\v)) = T(\v),
                \]
                so $1T = T$.
        \end{frame}

        \begin{frame}
                Thus, we have verified that $\L(V, W)$ is a vector space, with the given structure of addition and scaling.
        \end{frame}

        \begin{frame}{Dimension of $\L(V, W)$ when $V$ and $W$ are finite dimensional}
                Let $\beta = \{\v_1, \dots, \v_n\}$ be a basis of $V$ and let $\gamma = \{\w_1, \dots, \w_m\}$ be a basis of $W$. \\~\\

                Define the linear maps
                \[
                        T_{ij}\colon V \to W, \qquad \v_k \mapsto \delta_{ik}\w_j,
                \]
                for all $i = 1, \dots, n$ and $j = 1, \dots, m$. We claim that the set of all such $T_{ij}$ comprises a basis of $\L(V, W)$. \\~\\
                \pause
                Note that 
                \[
                        T_{ij}(\lambda_1\v_1 + \dots + \lambda_n\v_n) = \lambda_i\w_j.
                \]
        \end{frame}

        \begin{frame}{$\spn\{T_{ij}\} = \L(V, W)$}
                Suppose $T\colon V \to W$ is a linear map in $\L(V, W)$. For each of the basis vectors $\v_i \in \beta$, there exist unique
                scalars $a_{ij}$ such that
                \[
                        T(\v_i) \,=\, a_{i1}\w_1 + a_{i2}\w_2 + \dots + a_{im}\w_m.
                \]
                \pause
                We see that
                \[
                        T \,=\, \sum_{i = 1}^n \sum_{j = 1}^m a_{ij} T_{ij}.
                \]
                \pause
                To prove this, pick any $\v \in V$ and write $\v = \lambda_1\v_1 + \dots + \lambda_n\v_n$. Then,
                \[
                        T(\v) = \sum_{i = 1}^n \lambda_i T(\v_i) = \sum_{i = 1}^n \sum_{j = 1}^m \lambda_i a_{ij}\w_j
                                = \sum_{i = 1}^n \sum_{j = 1}^m a_{ij} T_{ij}(\v).
                \]
        \end{frame}

        \begin{frame}{$\{T_{ij}\}$ is linearly independent}
                Consider the linear combination 
                \[
                        \sum_{i = 1}^n \sum_{j = 1}^m c_{ij} T_{ij} = \mathbf{0}.
                \]
                \pause
                By successively evaluating this map on $\v_k$ for $k = 1, \dots, n$, we see that
                \[
                        \sum_{i = 1}^n \sum_{j = 1}^m c_{ij} T_{ij}(\v_k) = 
                        \sum_{i = 1}^n \sum_{j = 1}^m c_{ij} \delta_{ik}\w_j = \sum_{j = 1}^m c_{kj}\w_j = \mathbf{0}.
                \]
                The linear independence of $\gamma = \{\w_1, \dots, \w_m\}$ forces $c_{kj} = 0$.
        \end{frame}

        \begin{frame}
                Thus, the set of all $T_{ij}$ is a linearly independent set which spans $\L(V, W)$.
                Hence, this comprises a basis of $\L(V, W)$. \\~\\

                This basis contains $mn$ elements. Thus,
                \[
                        \dim\L(V, W) \,=\, mn,
                \]
                where $n = \dim{V}$ and $m = \dim{W}$ are finite.
        \end{frame}



\end{document}
