\documentclass[handout]{beamer}
\usetheme{metropolis}
\setbeamercovered{transparent}

\usepackage{amsmath, amssymb, amsthm}
\usepackage{bm}
\def\x{\bm{x}}
\def\y{\bm{y}}
\def\v{\bm{v}}
\def\xcol{\begin{bmatrix} x_1 \\ x_2 \\ \vdots \\ x_m \end{bmatrix} }
\def\ycol{\begin{bmatrix} y_1 \\ y_2 \\ \vdots \\ y_n \end{bmatrix} }
\def\yrow{\begin{bmatrix} y_1 & y_2 & \cdots & y_n \end{bmatrix} }
\def\rank{\operatorname{rank}}
\def\dim{\operatorname{dim}}

\title{
        Term presentation \\
        Problem 4
}
\author{Satvik Saha, 19MS154}
\institute{
        MA2102: Linear Algebra I \\
        Indian Institute of Science Education and Research, Kolkata
}
\date{\today}

\begin{document}
        \maketitle

        \begin{frame}{Problem statement}
                Show that a matrix $A$ is of rank 1 if and only if $A = \x\y^\top$ for some non-zero column vectors $\x$ and $\y$. \\~\\
                \pause
                \[
                        A = \begin{bmatrix}
                                a_{11} & a_{12} & \cdots & a_{1n} \\
                                a_{21} & a_{22} & \cdots & a_{2n} \\
                                \vdots & \vdots & \ddots & \vdots \\
                                a_{m1} & a_{m2} & \cdots & a_{mn}
                        \end{bmatrix}, \qquad
                        \x = \xcol, \qquad
                        \y = \ycol.
                \]
        \end{frame}

        \begin{frame}{Preliminaries}
                The \emph<1>{column rank} of a matrix $A$ is equal to the dimension of the column space of $A$. \\~\\
                The \emph<1>{row rank} of a matrix $A$ is equal to the dimension of the row space of $A$. \\~\\
                \pause
                The column rank and row rank of any matrix $A \in M_{m\times n}(F)$ are equal. \\~\\
                The \emph<2>{rank} of $A$ is equal to the column rank of $A$. 
        \end{frame}

        \begin{frame}{If $A = \x\y^\top$, then $\rank{A} = 1$}
                \begin{align*}
                        A = \x\y^\top &= \xcol\yrow \\ \\ \pause
                                &= \begin{bmatrix}
                                        x_1y_1 & x_1y_2 & \cdots & x_1y_n \\
                                        x_2y_1 & x_2y_2 & \cdots & x_2y_n \\
                                        \vdots & \vdots & \ddots & \vdots \\
                                        x_my_1 & x_my_2 & \cdots & x_my_n \\
                                \end{bmatrix} \\ \\ \pause
                                &= \begin{bmatrix}
                                        y_1\x & y_2\x & \cdots & y_n\x
                                \end{bmatrix}.
                \end{align*}        
        \end{frame}
        
        \begin{frame}{If $A = \x\y^\top$, then $\rank{A} = 1$}
        The column space of $A$ consists of all finite linear combinations of the columns of $A$. \\~\\
        \pause
        For any element $\v$ in the column space of $A$, we can write
        \[
                \v = \lambda_1y_1\x + \lambda_2y_2\x + \cdots + \lambda_ny_n\x = \lambda \x.
        \]
        \pause
        The column vector $\x$ spans the column space of $A$. 
        Furthermore, there is some $y_i\x\neq\mathbf{0}$ in the column space of $A$. 
        \[
                \rank{A} \leq 1 \leq \rank{A} \;\Rightarrow\; \rank{A} = 1.
        \]
        \end{frame}
        
        \begin{frame}{If $\rank{A} = 1$, then $A = \x\y^\top$}
                The column space of $A$ admits a singleton basis. Set $\x$ equal to this element. \\
                \[
                        A = \begin{bmatrix}
                                \lambda_1\x & \lambda_2\x & \cdots & \lambda_n\x
                        \end{bmatrix}.
                \] \\~\\
                \pause
                Set
                \[
                        \y = \begin{bmatrix}
                                \lambda_1 \\ \lambda_2 \\ \vdots \\ \lambda_n
                        \end{bmatrix}.
                \]\\~\\
                With this choice of column vectors $\x$ and $\y$, we have A = $\x\y^\top$.
        \end{frame}
\end{document}
