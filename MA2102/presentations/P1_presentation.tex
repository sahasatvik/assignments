\documentclass[handout]{beamer}
\usetheme{metropolis}
\setbeamercovered{transparent}

\usepackage{amsmath, amssymb, amsthm}
\usepackage{bm}
\def\v{\bm{v}}
\def\w{\bm{w}}
\def\dim{\operatorname{dim}}
\def\deg{\operatorname{deg}}
\def\det{\operatorname{det}}

\title{
        Term presentation \\
        Problem 1
}
\author{Satvik Saha, 19MS154}
\institute{
        MA2102: Linear Algebra I \\
        Indian Institute of Science Education and Research, Kolkata
}
\date{\today}

\begin{document}
        \maketitle

        \begin{frame}{Problem statement}
                Find the eigenvalues and corresponding eigenvectors of the matrix
                \[
                        A = \begin{bmatrix}
                                1 & 2 & 2 \\ 2 & 1 & 2 \\ 2 & 2 & 1
                        \end{bmatrix}.
                \]
                Find an invertible matrix $P$ such that $P^{-1}AP$ is diagonal.
        \end{frame}

        \begin{frame}{Preliminaries}
                An eigenvector of a matrix $A$ is a vector $\v$ such that
                \[
                        A\v = \lambda\v,
                \]
                for some scalar $\lambda$, which is called the eigenvalue of the eigenvector $\v$. \\~\\
                \pause
                The eigenvalues of a matrix $A$ are precisely the roots of the characteristic polynomial
                \[
                        \det(A - \lambda I_n) = 0.
                \]
        \end{frame}
        
        \begin{frame}{Preliminaries}
                Let $P \in M_n(F)$ be a matrix whose columns are eigenvectors $\v_1, \v_2, \dots, \v_n$ of $A$.
                Then, $AP = PD$, where $D$ is the diagonal matrix of the corresponding eigenvalues.
                \begin{align*}
                        AP &= \begin{bmatrix} A\v_1 & A\v_2 & \cdots & A\v_n \end{bmatrix}\\
                        &= \begin{bmatrix} \lambda_1\v_1 & \lambda_2\v_2 & \cdots & \lambda_n\v_n \end{bmatrix} \\
                        &= \begin{bmatrix}
                                \v_1 & \v_2 & \cdots & \v_n
                        \end{bmatrix}
                        \begin{bmatrix}
                                \lambda_1 & 0 & \cdots & 0 \\
                                0 & \lambda_2 & \cdots & 0 \\
                                \vdots & \vdots & \ddots & \vdots \\
                                0 & 0 & \cdots & \lambda_n
                        \end{bmatrix}.
                \end{align*}
                \pause
                If the eigenvectors $\v_1, \dots, \v_n$ are linearly independent, the matrix $P$ is invertible.
                Then, we can write  $P^{-1}AP = D$.
        \end{frame}

        \begin{frame}{Computing eigenvalues}
                We first write the characteristic polynomial $p(\lambda) = \det(A - \lambda I_n)$.
                \begin{align*}
                        \det \begin{bmatrix}
                                1 - \lambda & 2 & 2 \\ 2 & 1 - \lambda & 2 \\ 2 & 2 & 1 - \lambda
                        \end{bmatrix}
                                \,&=\, (1 - \lambda)^3  + 2\cdot 8 - 3\cdot 4(1 - \lambda) \\
                                \,&=\, 1 - 3\lambda + 3\lambda^2 - \lambda^3 + 16 - 12 + 12\lambda \\
                                \,&=\, 5 + 9\lambda + 3\lambda^2 - \lambda^3.
                \end{align*}
                \pause
                By inspection, $p(5) = 0$, so $5$ is a root of $p$. Synthetic division gives
                \[
                        p(\lambda) = (5 - \lambda)(1 + 2\lambda + \lambda^2) = (5 - \lambda)(1 + \lambda)^2.
                \]
                Thus, the eigenvalues of $A$ are $-1$ and $5$.
        \end{frame}

        \begin{frame}{Computing eigenvectors}
                We seek $\v$ such that $A\v = \lambda\v$, i.e.\ $(A - \lambda I_n)\v = \mathbf{0}$. \\~\\

                When $\lambda = 5$,
                \begin{align*}
                        \begin{bmatrix}
                                1 - 5 & 2 & 2 \\ 2 & 1 - 5 & 2 \\ 2 & 2 & 1 - 5
                        \end{bmatrix}
                        \begin{bmatrix}
                                v_{1} \\ v_{2} \\ v_{3}
                        \end{bmatrix}
                        &= \begin{bmatrix}
                                -4v_1 + 2v_2 + 2v_3 \\
                                2v_1 - 4v_2 + 2v_3 \\
                                2v_1 + 2v_2 - 4v_3
                        \end{bmatrix}
                        &= \begin{bmatrix}
                                0 \\ 0 \\ 0
                        \end{bmatrix}.
                \end{align*}
                This forces $v_1 = v_2 = v_3$. We choose $v_1 = v_2 = v_3 = 1$.
        \end{frame}
        
        \begin{frame}{Computing eigenvectors}
                When $\lambda = -1$,
                \begin{align*}
                        \begin{bmatrix}
                                1 + 1 & 2 & 2 \\ 2 & 1 + 1 & 2 \\ 2 & 2 & 1 + 1
                        \end{bmatrix}
                        \begin{bmatrix}
                                v_{1} \\ v_{2} \\ v_{3}
                        \end{bmatrix}
                        &= \begin{bmatrix}
                                2v_1 + 2v_2 + 2v_3 \\
                                2v_1 + 2v_2 + 2v_3 \\
                                2v_1 + 2v_2 + 2v_3
                        \end{bmatrix}
                        &= \begin{bmatrix}
                                0 \\ 0 \\ 0
                        \end{bmatrix}.
                \end{align*}
                This only imposes $v_1 + v_2 + v_3 = 0$. The set of solutions $[-v_2 - v_3\quad v_2\quad v_3]^\top$ form a two dimensional
                subspace of $\mathbb{R}^3$. We choose two linearly independent vectors from this subspace by setting $v_2 = 0, v_3 = 1$ in the first case
                and $v_2 = 1, v_3 = 0$ in the second.
        \end{frame}

        \begin{frame}
                Thus, the eigenvalues and corresponding eigenvectors of $A$ are as follows.
                \begin{align*}
                        \lambda = 5,  & \qquad \v_1 = \begin{bmatrix}1 \\ 1 \\ 1\end{bmatrix}, \\\\
                        \lambda = -1, & \qquad \v_2 = \begin{bmatrix}-1 \\ 0 \\ 1 \end{bmatrix}, \qquad
                                \v_3 = \begin{bmatrix}-1 \\ 1 \\ 0 \end{bmatrix}.
                \end{align*}    
        \end{frame}

        \begin{frame}{Diagonalization}
                We perform Gauss Jordan elimination on the matrix $P = [\v_1\;\v_2\;\v_3]$, whose columns are the eigenvectors of $A$. \\
                \begin{align*}
                        \left[\begin{array}{@{}ccc|ccc}
                                1 & -1 & -1     & 1 & 0 & 0 \\
                                1 &  0 &  1     & 0 & 1 & 0 \\
                                1 &  1 &  0     & 0 & 0 & 1
                        \end{array}\right] &\longrightarrow
                        \left[\begin{array}{@{}ccc|ccc}
                                3 &  0 &  0     & 1 & 1 & 1 \\
                                1 &  0 &  1     & 0 & 1 & 0 \\
                                1 &  1 &  0     & 0 & 0 & 1
                        \end{array}\right] \longrightarrow \\\\
                        \left[\begin{array}{@{}ccc|ccc}
                                3 &  0 &  0     & 1 & 1 & 1 \\
                                0 &  0 &  1     & -\frac{1}{3} & \frac{2}{3} & -\frac{1}{3} \\
                                0 &  1 &  0     & -\frac{1}{3} & -\frac{1}{3} & \frac{2}{3}
                        \end{array}\right] &\longrightarrow
                        \left[\begin{array}{@{}ccc|ccc}
                                1 &  0 &  0     & \frac{1}{3} & \frac{1}{3} & \frac{1}{3} \\
                                0 &  1 &  0     & -\frac{1}{3} & -\frac{1}{3} & \frac{2}{3} \\
                                0 &  0 &  1     & -\frac{1}{3} & \frac{2}{3} & -\frac{1}{3}
                        \end{array}\right].
                \end{align*}
        \end{frame}

        \begin{frame}{Diagonalization}
                \begin{align*}
                        P^{-1}AP &= \frac{1}{3} \begin{bmatrix}
                                1 & 1 & 1 \\ -1 & -1 & 2 \\ -1 & 2 & -1
                        \end{bmatrix}
                        \begin{bmatrix}
                                1 & 2 & 2 \\ 2 & 1 & 2 \\ 2 & 2 & 1
                        \end{bmatrix}
                        \begin{bmatrix}
                                1 & -1 & -1 \\ 1 & 0 & 1 \\ 1 & 1 & 0
                        \end{bmatrix} \\
                        &= \frac{1}{3} \begin{bmatrix}
                                1 & 1 & 1 \\ -1 & -1 & 2 \\ -1 & 2 & -1
                        \end{bmatrix}
                        \begin{bmatrix}
                                5 & 1 & 1 \\ 5 & 0 & -1 \\ 5 & -1 & 0
                        \end{bmatrix} \\
                        &= \begin{bmatrix}
                                5 & 0 & 0 \\ 0 & -1 & 0 \\ 0 & 0 & -1
                        \end{bmatrix}
                \end{align*}
        \end{frame}


\end{document}
