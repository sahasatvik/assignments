\documentclass[10pt]{article}

\usepackage[T1]{fontenc}
\usepackage{geometry}
\usepackage{amsmath, amssymb, amsthm}
\usepackage{bm}
\def\dim{\operatorname{dim}}
\def\deg{\operatorname{deg}}
\def\spn{\operatorname{span}}

\newtheorem{lemma}{Lemma}

\title{MA2102 - Solution V}
\author{Satvik Saha}
\date{}

\geometry{a4paper, margin=1in}
\setlength\parindent{0pt}

\begin{document}
        \par\textbf{IISER Kolkata} \hfill \textbf{Problem V}
        \vspace{3pt}
        \hrule
        \vspace{3pt}
        \begin{center}
                \LARGE{\textbf{MA2102 : Linear Algebra I}}
        \end{center}
        \vspace{3pt}
        \hrule
        \vspace{3pt}
        Satvik Saha, \texttt{19MS154}\hfill\today
        \vspace{20pt}

        Let $a \in \mathbb{R}$. Consider the set
        \[
                S_a^n = \left\{ 1,\, (x - a),\, (x - a)^2,\, \dots,\, (x - a)^n \right\}.
        \]
        Show that $S_a^n$ is a basis for $P_n(\mathbb{R})$, the space of polynomials of degree at most $n$. \\

        We notate the superscript $S_a^n$ to denote the highest degree term $(x - a)^n$.

        \paragraph{Solution}
        We first calculate the dimension of $P_n(\mathbb{R})$ and exhibit the standard basis.
        \begin{lemma}
                The set $S_0^n = \{1, x, x^2, \dots, x^n\}$ is a basis of $P_n(\mathbb{R})$.
                \begin{proof}
                        Note that any polynomial in $P_n(\mathbb{R})$ is of the form
                        \[
                                p(x) = a_0 + a_1x + \dots + a_nx^n.
                        \]
                        This means that $P_n(\mathbb{R}) \subseteq \spn S_0^n$. We now show linear independence by considering the linear
                        combination
                        \[
                                c_0 + c_1x + c_2x^2 + \dots + c_nx^n \,=\, \mathbf{0}.
                        \]
                        Because the polynomial on the left is equated with the zero polynomial on the right, it must identically evaluate to $0$
                        no matter the choice of $x \in \mathbb{R}$.
                        We choose $n + 1$ distinct reals $x$, which we exhibit as $n + 1$ roots of the polynomial on the left.
                        However, the degree of this polynomial is at most $n$.
                        We conclude that the polynomial on the left is the zero polynomial, so $c_0 = c_1 = \dots = c_n$.
                        
                        This proves that the set $S_0^n$ is a basis of $P_n(\mathbb{R})$.
                \end{proof}
        \end{lemma}
        We use the fact that $\dim P_n(\mathbb{R}) = n + 1$, from Lemma 1. Thus, we need only show that $S_a^n$ is linearly independent for $a \neq 0$.
        This is sufficient to prove that $S_a^n$ is a basis of $P_n(\mathbb{R})$, because the Replacement Theorem guarantees that any
        linearly independent set of size $\dim P_n(\mathbb{R})$ will be a basis.
        \begin{lemma}
                The polynomial $(x - a)^n - x^n$ has degree at most $n - 1$, for $n \in \mathbb{N}$.
                \begin{proof}
                        We expand $(x - a)^n$ using the Binomial Theorem to obtain
                        \[
                                (x - a)^n = x^n - nax^{n - 1} + \binom{n}{2}a^2x^{n - 2} + \dots + (-1)^n a^n.
                        \]
                        Subtracting $x^n$ from both sides leaves terms of degree at most $n - 1$ on the right. Thus,
                        \[
                                (x - a)^n - x^n \,\in\, P_{n - 1}(\mathbb{R}) \,\subset\, P_n(\mathbb{R}).
                        \]
                        The coefficients of this $n - 1$ degree polynomial are the binomial coefficients with alternating sign, as seen above.
                \end{proof}
        \end{lemma}

        With this, we prove that $S_a^n$ for $a \neq 0$ is a basis of $P_n(\mathbb{R})$ by induction. 
        For $n = 0$, the claim is trivial, since we have $S_a^0 = S_0^0 = \{1\}$, which is a linearly independent set in $P_0(\mathbb{R})$,
        hence a basis of $P_0(\mathbb{R})$. \\

        For $n = 1$, consider the linear combination of elements from $S_a^1 = \{1,\, (x - a)\}$
        \[
                c_0 + c_1(x - a) \,=\, \mathbf{0},
        \]
        for arbitrary $c_0, c_1 \in \mathbb{R}$.
        Successively set $x = a$ and $x = 0$.
        Thus, $c_0 = 0$ and $c_0 - c_1a = 0$, whence $c_0 = c_1 = 0$.
        This shows that $S_a^1$ is linearly independent in $P_1(\mathbb{R})$, hence a basis of $P_1(\mathbb{R})$. \\

        Suppose that for $n = k$, the set $S_a^k = \left\{1,\, (x - a),\, \dots, (x - a)^k\right\}$ is a basis of $P_k(\mathbb{R})$.
        Consider the linear combination of elements from $S_a^{k + 1}$,
        \[
                c_0 + c_1(x - a) + \dots + c_k(x - a)^k + c_{k + 1}(x - a)^{k + 1} \,=\, \mathbf{0}.
        \]
        Subtract and add $c_{k + 1}x^{k + 1}$.
        \begin{align*}
                \Big[c_0 + c_1(x - a) + \dots 
                + c_k(x - a)^k + c_{k + 1}\left((x - a)^{k + 1} - x^{k + 1}\right)\Big] + c_{k + 1}x^{k + 1} \,=\, \mathbf{0}.
        \end{align*}
        Note that the portion in square brackets is a polynomial of degree at most $k$.
        This is because $(x - a)^{k + 1} - x^{k + 1}$ has degree at most $k$ by Lemma 2, and the remaining terms also have degree at most $k$.
        Thus, we expand this bracketed polynomial in the basis $S_a^k$.
        \[
                p_k(x) \,=\, a_0 + a_1x + \dots + a_kx^k \in P_k(\mathbb{R}).
        \]
        Replacing this in the previous equation,
        \[
                a_0 + a_1x + \dots + a_kx^k + c_{k + 1}x^{k + 1} \,=\, \mathbf{0}.
        \]
        The linear independence of $S_0^{k + 1} = \{1, x, \dots, x^{k + 1}\}$ from Lemma 1 gives $a_0 = a_1 = \dots = a_k = c_{k + 1} = 0$.
        Substituting this back into the original linear combination with the $c_j$ coefficients, we have
        \[
                c_0 + c_1(x - a) + \dots + c_k(x - a)^k \,=\, \mathbf{0}.
        \]
        The induction hypothesis, whereby $S_a^k$ is linearly independent, gives $c_0 = c_1 = \dots = c_k = 0 = x_{k + 1}$.
        This shows that $S_a^{k + 1}$ is linearly independent in $P_{k + 1}(\mathbb{R})$, which means it is a basis of $P_{k + 1}(\mathbb{R})$. \\

        Thus, by the principle of mathematical induction, the set $S_a^n$ is a basis of $P_n(\mathbb{R})$ for all integers $n \geq 0$.
        This completes the proof.

\end{document}
