\documentclass[11pt]{article}

\usepackage[T1]{fontenc}
\usepackage{geometry}
\usepackage{amsmath, amssymb, amsthm}
\usepackage[scr]{rsfso}
\usepackage{bm}
\usepackage[%
    hidealllines=true,%
    innerbottommargin=15,%
    nobreak=true,%
]{mdframed}
\usepackage{xcolor}
\usepackage{graphicx}
\usepackage{fancyhdr}
\usepackage{hyperref}

\geometry{a4paper, margin=1in, headheight=14pt}

\pagestyle{fancy}
\fancyhf{}
\renewcommand\headrulewidth{0.4pt}
\fancyhead[L]{\scshape MA3201: Topology}
\fancyhead[R]{\scshape \leftmark}
\rfoot{\footnotesize\it Updated on \today}
\cfoot{\thepage}

\newcommand{\C}{\mathbb{C}}
\newcommand{\R}{\mathbb{R}}
\newcommand{\Q}{\mathbb{Q}}
\newcommand{\Z}{\mathbb{Z}}
\newcommand{\N}{\mathbb{N}}

\newmdtheoremenv[%
    backgroundcolor=blue!10!white,%
]{theorem}{Theorem}[section]
\newmdtheoremenv[%
    backgroundcolor=violet!10!white,%
]{corollary}{Corollary}[theorem]
\newmdtheoremenv[%
    backgroundcolor=teal!10!white,%
]{lemma}[theorem]{Lemma}

\theoremstyle{definition}
\newmdtheoremenv[%
    backgroundcolor=green!10!white,%
]{definition}{Definition}[section]
\newmdtheoremenv[%
    backgroundcolor=red!10!white,%
]{exercise}{Exercise}[section]

\theoremstyle{remark}
\newtheorem*{remark}{Remark}
\newtheorem*{example}{Example}
\newtheorem*{solution}{Solution}

\surroundwithmdframed[%
    linecolor=black!20!white,%
    hidealllines=false,%
    innertopmargin=5,%
    innerbottommargin=10,%
    skipabove=0,%
    skipbelow=0,%
]{example}

\numberwithin{equation}{section}

\title{
    \Large\textsc{MA3201} \\
    \Huge \textbf{Topology} \\
    \vspace{5pt}
    \Large{Spring 2022}
}
\author{
    \large Satvik Saha
    \\\textsc{\small 19MS154}
}
\date{\normalsize
    \textit{Indian Institute of Science Education and Research, Kolkata, \\
    Mohanpur, West Bengal, 741246, India.} \\
}

\begin{document}
    \maketitle

    \tableofcontents

    \section{Introduction}
    
    \subsection{Topological spaces}

    \begin{definition}
        A topology on some set $X$ is a family $\tau$ of subsets of $X$,
        satisfying the following.
        \begin{enumerate}
            \itemsep0em
            \item $\emptyset, X \in \tau$.
            \item All unions of elements from $\tau$ are in $\tau$.
            \item All finite intersections of elements from $\tau$ are in $\tau$.
        \end{enumerate}
        The sets from $\tau$ are declared to be open sets in the topological space
        $(X, \tau)$.
    \end{definition}
    \begin{example}
        Any set $X$ admits the indiscrete topology $\tau_{id} = \{\emptyset, X\}$, as
        well as the discrete topology $\tau_{d} = \mathcal{P}(X)$. Both of these are
        trivial examples.
    \end{example}
    \begin{example}
        Let $X$ be a set. The cofinite topology on $X$ is the collection of
        complements of finite sets, along with the empty set. Note that when $X$ is
        finite, this is simply the discrete topology.
    \end{example}

    \begin{definition}
        Let $\tau, \tau'$ be two topologies on the set $X$. We say that $\tau$ is
        finer than $\tau'$ if $\tau$ has more open sets than $\tau'$. In such a case,
        we also say that $\tau'$ is coarser than $\tau$.
    \end{definition}

    \begin{definition}
        Let $(X, \tau)$ be a topological space. We say that $\beta \subseteq
        \tau$ is a base of the topology $\tau$ such that every open set $U \in \tau$
        is expressible as a union of elements from $\beta$. 
    \end{definition}

    \begin{definition}
        Let $X$ be a set, and let $\beta$ be a collection of subsets of $X$
        satisfying the following.
        \begin{enumerate}
            \itemsep0em
            \item For every $x \in X$, there exists $x \in B \in \beta$.
            \item For every $x \in X$ such that $x \in B_1 \cap B_2$, $B_1, B_2 \in
            \beta$, there exists $x \in B \subseteq B_1 \cap B_2$ such that $B \in
            \beta$.
        \end{enumerate}
        Then, $\beta$ generates a topology on $X$, namely the collection of all
        unions of elements of $\beta$.
    \end{definition}


    \subsection{Continuous maps}

    \begin{definition}
        Let $f\colon X \to Y$ be a function between the topological spaces $(X,
        \tau_X)$ and $(Y, \tau_Y)$. We say that $f$ is continuous if for every $U
        \in \tau_Y$, we have $f^{-1}(U) \in \tau_X$. In other words, the pre-image of
        every open set in $Y$ must be open in $X$.
    \end{definition}

    \begin{definition}
        Let $f\colon X \to Y$ be a function between the topological spaces $(X,
        \tau_X)$ and $(Y, \tau_Y)$. We say that $f$ is a homeomorphism if $f$ is
        continuous, $f$ is invertible, and $f^{-1}$ is continuous. We also say that
        $X$ and $Y$ are homeomorphic when such a homeomorphism between them exists.
    \end{definition}

\end{document}
