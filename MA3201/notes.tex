\documentclass[11pt]{article}

\usepackage[T1]{fontenc}
\usepackage{geometry}
\usepackage{amsmath, amssymb, amsthm}
\usepackage[scr]{rsfso}
\usepackage{bm}
\usepackage[%
    hidealllines=true,%
    innerbottommargin=15,%
    nobreak=true,%
]{mdframed}
\usepackage{xcolor}
\usepackage{graphicx}
\usepackage{fancyhdr}
\usepackage{hyperref}

\geometry{a4paper, margin=1in, headheight=14pt}

\pagestyle{fancy}
\fancyhf{}
\renewcommand\headrulewidth{0.4pt}
\fancyhead[L]{\scshape MA3201: Topology}
\fancyhead[R]{\scshape \leftmark}
\rfoot{\footnotesize\it Updated on \today}
\cfoot{\thepage}

\newcommand{\C}{\mathbb{C}}
\newcommand{\R}{\mathbb{R}}
\newcommand{\Q}{\mathbb{Q}}
\newcommand{\Z}{\mathbb{Z}}
\newcommand{\N}{\mathbb{N}}

\newmdtheoremenv[%
    backgroundcolor=blue!10!white,%
]{theorem}{Theorem}[section]
\newmdtheoremenv[%
    backgroundcolor=violet!10!white,%
]{corollary}{Corollary}[theorem]
\newmdtheoremenv[%
    backgroundcolor=teal!10!white,%
]{lemma}[theorem]{Lemma}

\theoremstyle{definition}
\newmdtheoremenv[%
    backgroundcolor=green!10!white,%
]{definition}{Definition}[section]
\newmdtheoremenv[%
    backgroundcolor=red!10!white,%
]{exercise}{Exercise}[section]

\theoremstyle{remark}
\newtheorem*{remark}{Remark}
\newtheorem*{example}{Example}
\newtheorem*{solution}{Solution}

\surroundwithmdframed[%
    linecolor=black!20!white,%
    hidealllines=false,%
    innertopmargin=5,%
    innerbottommargin=10,%
    skipabove=0,%
    skipbelow=0,%
]{example}

\numberwithin{equation}{section}

\title{
    \Large\textsc{MA3201} \\
    \Huge \textbf{Topology} \\
    \vspace{5pt}
    \Large{Spring 2022}
}
\author{
    \large Satvik Saha
    \\\textsc{\small 19MS154}
}
\date{\normalsize
    \textit{Indian Institute of Science Education and Research, Kolkata, \\
    Mohanpur, West Bengal, 741246, India.} \\
}

\begin{document}
    \maketitle

    \tableofcontents

    \section{Introduction}
    
    \subsection{Topological spaces}

    \begin{definition}
        A topology on some set $X$ is a family $\tau$ of subsets of $X$,
        satisfying the following.
        \begin{enumerate}
            \itemsep0em
            \item $\emptyset, X \in \tau$.
            \item All unions of elements from $\tau$ are in $\tau$.
            \item All finite intersections of elements from $\tau$ are in $\tau$.
        \end{enumerate}
        The sets from $\tau$ are declared to be open sets in the topological space
        $(X, \tau)$.
    \end{definition}
    \begin{example}
        Any set $X$ admits the indiscrete topology $\tau_{id} = \{\emptyset, X\}$, as
        well as the discrete topology $\tau_{d} = \mathcal{P}(X)$. Both of these are
        trivial examples.
    \end{example}
    \begin{example}
        Let $X$ be a set. The cofinite topology on $X$ is the collection of
        complements of finite sets, along with the empty set. Note that when $X$ is
        finite, this is simply the discrete topology.
    \end{example}

    \begin{definition}
        Let $\tau, \tau'$ be two topologies on the set $X$. We say that $\tau$ is
        finer than $\tau'$ if $\tau$ has more open sets than $\tau'$. In such a case,
        we also say that $\tau'$ is coarser than $\tau$.
    \end{definition}


    \subsection{Topological bases}

    \begin{definition}
        Let $(X, \tau)$ be a topological space. We say that $\beta \subseteq
        \tau$ is a base of the topology $\tau$ such that every open set $U \in \tau$
        is expressible as a union of elements from $\beta$. 
    \end{definition}

    \begin{definition}
        Let $X$ be a set, and let $\beta$ be a collection of subsets of $X$
        satisfying the following.
        \begin{enumerate}
            \itemsep0em
            \item For every $x \in X$, there exists $x \in B \in \beta$.
            \item For every $x \in X$ such that $x \in B_1 \cap B_2$, $B_1, B_2 \in
            \beta$, there exists $B \in \beta$ such that $x \in B \subseteq B_1 \cap
            B_2$.
        \end{enumerate}
        Then, $\beta$ generates a topology on $X$, namely the collection of all
        unions of elements of $\beta$.
    \end{definition}

    \begin{lemma}
        Let $\tau$ be a topology on $X$, and let $\beta \subseteq \tau$ be a
        collection of open sets. Then, $\beta$ is a basis of $\tau$, or generates
        $\tau$, if for every $x \in U \in \tau$, there exists $B \in \beta$ such that
        $x \in B \subseteq U$.
    \end{lemma}
    \begin{example}
        The collection of all open balls in $\R^n$ form a basis of the usual topology.
    \end{example}

    \begin{lemma}
        Let $X$ be equipped with the topologies $\tau$ and $\tau'$, and let $\beta$
        and $\beta'$ be the respective bases of these topologies. Then, $\tau$ is
        finer than $\tau'$ if and only if given $x \in B' \in \beta'$, there exists
        $x \in B \in \beta$ such that $B \subseteq B'$.
    \end{lemma}
    \begin{example}
        The collections of open balls in $\R^n$ generate the same topology as the
        collection of all open rectangles in $\R^n$.
    \end{example}
    \begin{example}
        Consider the topologies on $\R$ generated by the following bases.
        \begin{enumerate}
            \itemsep0em
            \item $\beta_1 = \{(a, b): a, b \in \R, a < b\}$.
            \item $\beta_2 = \{[a, b): a, b \in \R, a < b\}$.
            \item $\beta_3 = \{(a, b): a, b \in \R, a < b\} \cup \{(a, b) \setminus
            K\}$ where $K = \{1 / n: n \in \Z\}$.
        \end{enumerate}
        We call the topology generated by $\beta_2$ the lower limit topology, denoted
        $\R_\ell$. The topology generated by $\beta_3$ is denoted $\R_K$. Both of
        these are strictly finer than the standard topology.
    \end{example}

    \begin{definition}
        A sub-basis for some topology on $X$ is a collection $\rho$ of subsets of $X$
        whose union is the whole of $X$. The topology generated by $\rho$ is defined
        to be the topology generated by the collection of all finite intersections of
        elements of $\rho$.
    \end{definition}


    \subsection{Product topology}
    
    \begin{definition}
        Let $(X_1, \tau_1)$, $(X_2, \tau_2)$ be topological spaces. Then $\tau_1
        \times \tau_2$ generates the product topology on $X_1 \times X_2$.
    \end{definition}
    \begin{example}
        The product topology on $\R \times \R$, where $\R$ is equipped with the
        standard topology, coincides with the standard topology on $\R^2$.
    \end{example}

    \begin{lemma}
        If $\beta_1, \beta_2$ are bases of the topologies $\tau_1, \tau_2$, then
        $\beta_1 \times \beta_2$ and $\tau_1 \times \tau_2$ generate the same product
        topology.
    \end{lemma}
    \begin{proof}
        Given $(x_1, x_2) \in U$ where $U \subseteq X_1\times X_2$ is open in the
        product topology, recall that $U$ can be written as a union of the basic open
        sets $U_{1i} \times U_{2i}$, where $U_{1i} \in \tau_1$ and $U_{2i} \in
        \tau_2$. Suppose that $(x_1, x_2) \in U_1 \times U_2$. Thus, we can choose
        $B_1 \in \beta_1$, $B_2 \in \beta_2$ such that $x_1 \in B_1 \subseteq U_1$
        and $x_2 \in B_2 \subseteq U_2$. Thus, $(x_1, x_2) \in B_1 \times B_2
        \subseteq U_1 \times U_2 \subseteq U$.
    \end{proof}

    \begin{definition}
        The projection maps are defined as $\pi_i\colon X_1 \times \cdots X_k \to
        X_i$, $(x_1, \dots, x_k) \mapsto x_i$.
    \end{definition}

    \begin{lemma}
        The collection of elements of the form $\pi_1^{-1}(U_1)$ or
        $\pi_2^{-1}(U_2)$, where $U_1 \in \tau_1$ and $U_2 \in \tau_2$, forms a
        sub-basis of the product topology on $X_1 \times X_2$.
    \end{lemma}
    \begin{proof}
        Note that $\pi_1^{-1}(X_1) = X_1 \times X_2$. Now it is easy to see that
        finite intersections of elements of the form $U_1 \times X_2$ or $X_1 \times
        U_2$ where $U_1, U_2$ are open, are all of the form $U_1 \times U_2$ which is
        precisely a basis of the product topology.
    \end{proof}
    \begin{corollary}
        We can restrict ourselves to the sub-basis of elements of the form
        $\pi_1^{-1}(B_1)$ or $\pi_2^{-1}(B_2)$, where $B_1 \in \beta_1$, $B_2 \in
        \beta_2$ for some bases $\beta_1$, $\beta_2$ of $\tau_1, \tau_2$.
    \end{corollary}


    \subsection{Subspace topology}
    
    \begin{definition}
        Let $(X, \tau)$ be a topological space, and let $Y \subset X$. Then the
        collection $U \cap Y$ for all $U \in \tau$ comprises the subspace topology
        $\tau_Y$ on $Y$ induced by the topology $\tau$ on $X$.
    \end{definition}

    \begin{lemma}
        If $\beta$ is a basis for the topology on $X$, and $Y \subset X$, then the
        collection $B \cap Y$ for all $B \in \beta$ generates the subspace topology
        on $Y$.
    \end{lemma}

    \begin{lemma}
        An open set of $Y$ is open in $X$ if $Y$ is open in $X$.
    \end{lemma}
    \begin{proof}
        Let $U \subset Y$ be open in $Y$, then $U = V \cap Y$ for some open set $V$
        in $X$. If additionally $Y$ is open in $X$, this immediately shows that $U$
        is open in $X$.
    \end{proof}

    \begin{theorem}
        Let $(X, \tau_X)$, $(Y, \tau_Y)$ be topological spaces, and let $A \subseteq
        X$, $B \subseteq Y$. Then, there are two ways of assigning a natural topology
        on $A \times B$.
        \begin{enumerate}
            \itemsep0em
            \item Take the product topology on $X \times Y$, and consider the
            subspace topology induced by it on $A \times B$.
            \item Take the subspace topologies on $A$ induced by $\tau_X$, $B$
            induced by $\tau_Y$, and consider the product topology generated by them
            on $A \times B$.
        \end{enumerate}
        These two methods generate the same topology on $A \times B$.
    \end{theorem}
    \begin{proof}
        Open sets in 1 look like $(U \times V) \cap (A \times B)$, where $U \in
        \tau_X$, $V \in \tau_Y)$. Open sets in 2 look like $(U' \cap A) \times (V' \cap
        B)$, where $U' \in \tau_X$, $V' \in \tau_Y$, which can be rewritten as $(U'
        \times V') \cap (A \times B)$. It is easy to see that these describe
        precisely the same sets.
    \end{proof}


    \subsection{Order topology}

    \begin{definition}
        Let $X$ be a set with a simple order $<$. Then the collection of sets of the
        form $(a, b)$, $[a_0, b)$, $(a, b_0]$ where $a_0$ is the minimal element of
        $X$, $b_0$ is the maximal element of $X$, generate the order topology on
        $X$.
    \end{definition}
    \begin{example}
        The order topology on $\N$ is precisely the discrete topology.
    \end{example}

    \begin{definition}
        Let $X_1, X_2$ be simply ordered sets. The dictionary order on $X_1 \times
        X_2$ is defined as follows: $(x_1, x_2) < (y_1, y_2)$ if $x_1 < y_1$, or if
        $x_1 = y_1$ and $x_2 < y_2$.
    \end{definition}
    
    \begin{example}
        Consider $X = \{1, 2\} \times \N$, where both $\{1, 2\}$ and $\N$ are endowed
        with the discrete topology. Note that the product topology on $X$ is the
        discrete topology.

        Now consider the dictionary order on $X$. Here, $(1, 1)$ is the smallest
        element, so we can list the elements of $X$ in ascending order. Note that
        every $(1, m) < (2, n)$, for all $m, n \in \N$. Now, note that all singletons
        $\{(1, m)\}$ are open in the order topology on $X$. The same is true for the
        singletons $\{(1, n)\}$ for all $n > 1$. However, the singleton $\{(2, 1)\}$
        is \emph{not} open in the order topology.
    \end{example}

    \begin{example}
        Consider $\R$ with the usual topology, and $X = [0, 1) \cup \{2\}$. Then,
        $\{2\}$ is open in the subspace topology on $X$, but it is not open in the
        order topology on $X$.
    \end{example}

    \begin{lemma}
        The open rays of the form $(a, +\infty)$ and $(-\infty, a)$ in $X$ form a
        sub-basis of the order topology on $X$.
    \end{lemma}
    \begin{proof}
        Note that $(a, b) = (-\infty, b) \cap (a, +\infty)$, $[a_0, b) = (-\infty,
        b)$, and $(a, b_0] = (a, +\infty)$.
    \end{proof}

    \begin{definition}
        Let $X$ be a simply ordered set, and $Y \subseteq X$. Then, we say that $Y$
        is convex in $X$ if given $a, b \in Y$ such that $a < b$, the interval $(a,
        b) = \{x \in X: a < x < b\} \subseteq Y$.
    \end{definition}

    \begin{theorem}
        Let $Y$ be convex in $X$. Then, the subspace topology and the order topology
        on $Y$ induced from the order topology on $X$ coincide.
    \end{theorem}


    \subsection{Closed sets}
    
    \begin{definition}
        Let $(X, \tau)$ be a topological space. A set $F \subseteq X$ is said to be
        closed in $X$ if $F^c = X\setminus F \in \tau$.
    \end{definition}
    \begin{example}
        The sets $\emptyset, X$ are closed in every topological space $(X, \tau)$.
    \end{example}
    \begin{example}
        In a set equipped with the discrete topology, every set is both open and
        closed.
    \end{example}

    \begin{lemma}
        Arbitrary intersections, and finite unions of closed sets are closed.
    \end{lemma}

    \begin{theorem}
        Let $(X, \tau)$ be a topological space, and let $Y \subset X$ be equipped
        with the subspace topology. Then, a set $F \subseteq Y$ is closed in $Y$ if
        and only if $F = Y \cap G$, where $G$ is closed in $X$.
    \end{theorem}
    \begin{proof}
        Let $F \subset Y$. Now, $F$ is closed in $Y$, $Y\setminus F = Y \cap F^c$ is
        open in $Y$, $Y \cap F^c = Y \cap U$ where $U$ is open in $X$, $F =
        Y \cap (Y \cap F^c)^c = Y \cap (Y \cap U)^c = Y \cap U^c$ where $U^c$ is
        closed. The steps are reversible.
    \end{proof}

    \begin{lemma}
        A closed set of $Y$ is closed in $X$ if $Y$ is closed in $X$.
    \end{lemma}

    \subsection{Interiors and closures}
    
    \begin{definition}
        Let $A \subseteq X$ where $(X, \tau)$ is a topological space. \begin{enumerate}
            \itemsep0em
            \item The interior of $A$ is defined as the union of all open sets
            contained in $A$. This is denoted by $A^\circ$.
            \item The closure of $A$ is defined as the intersection of all closed
            sets containing $A$. This is denoted by $\overline{A}$.
        \end{enumerate}
        \begin{remark}
            The interior of a set is open, and the closure of a set is closed.
        \end{remark}
    \end{definition}
    
    \begin{lemma}
        Let $Y \subset X$ be topological spaces, and let $A \subseteq Y$. Also let
        $\overline{A}_X$, $\overline{A}_Y$ denote the closures of $A$ in $X$, $Y$
        respectively. Then, $\overline{A}_Y = \overline{A}_X \cap Y$.
    \end{lemma}

    \begin{theorem}
        Let $A \subset X$. Then, \begin{enumerate}
            \itemsep0em
            \item $x \in \overline{A}$ if and only if every open set containing $x$
            has non-empty intersection with $A$.
            \item $x \in \overline{A}$ if and only if every basic open set containing
            $x$ has non-empty intersection with $A$, given that the topology on $X$
            is generated by those basic open sets.
        \end{enumerate}
    \end{theorem}


    \begin{definition}
        Let $A \subseteq X$ where $(X, \tau)$ is a topological space. We say that $x
        \in X$ is a limit point of $X$ if for every open set $U$ containing $x$, the
        deleted neighbourhood $U \setminus\{x\}$ has non-empty intersection with $A$.
        The set of limit points of $A$ is denoted by $A'$.
    \end{definition}

    \begin{example}
        Let $X$ be a set endowed with the discrete topology. Then, given any set $A
        \subseteq X$, we have $A' = \emptyset$.
    \end{example}

    \begin{lemma}
        A closed set contains all its limit points.
    \end{lemma}
    \begin{proof}
        Let $F \subseteq X$ be closed in $X$, and let $x \in F'$. Then given any open
        set containing $x$, we have $U \cap F \supseteq (U\setminus\{x\}) \cap F \neq
        \emptyset$, hence $x \in \overline{F} = F$.
    \end{proof}

    \begin{lemma}
        Let $A \subseteq X$ where $(X, \tau)$ is a topological space. Then,
        $\overline{A} = A \cup A'$.
    \end{lemma}
    \begin{proof}
        It is clear that $\overline{A} \supseteq A \cup A'$. Now pick $x \in
        \overline{A}$. If $x \notin A$, then we know that given any open
        neighbourhood $U$ of $x$, we have non-empty $U \cap A$. Furthermore, this
        intersection can never contain $x$, hence $x \in A'$. This proves that
        $\overline{A} \subseteq A \cup A'$.
    \end{proof}


    \subsection{Convergence of sequences}

    \begin{definition}
        Let $\{x_n\}_{n = 1}^\infty$ be a sequence of points from $(X, \tau)$, and
        let $x \in X$. We say that this sequence converges to $x$, denoted $x_n \to
        x$, if every open neighbourhood of $x$ contains the tail of this sequence. In
        other words, given $U \in \tau$ such that $x \in U$, there must exist $N \in
        \N$ such that $x_n \in U$ for all $n \geq N$.
    \end{definition}

    \begin{example}
        Let $X = \{a, b, c\}$, and $\tau = \{\emptyset, \{b\}, \{a, b\}, \{b, c\},
        X\}$. Then, the constant sequence of $b$'s converges to all three points $a,
        b, c$.
    \end{example}
    \begin{example}
        Let $X = \R$, and $\tau$ be the collection of all intervals $(-a, a)$
        together with $\emptyset, \R$. Then, the constant sequence of $0$'s converges
        to every point in $\R$.
    \end{example}

    \begin{definition}
        Let $(X, \tau)$ be a topological space. We say that this topological space is
        Hausdorff if given any two distinct points $x, y \in X$, there exist open
        sets $U, V \in \tau$ such that $x \in U$, $y \in V$, and $U \cap V =
        \emptyset$.
    \end{definition}

    \begin{example}
        The real numbers under the standard topology is Hausdorff.
    \end{example}

    \begin{theorem}
        Let $(X, \tau)$ be a Hausdorff topological space, and let $\{x_n\}_{n =
        1}^\infty$ be a sequence of points in $X$. Then, this sequence can converge
        to at most one point in $X$.
    \end{theorem}
    \begin{proof}
        Suppose that $\{x_n\}_{n = 1}^\infty$ converges to distinct points $x, y \in
        X$. Then there exist disjoint open neighbourhoods $U, V$ such that $x \in U$,
        $y \in V$. Convergence means that both $U$ and $V$ contain a tail of the
        sequence, which is a contradiction.
    \end{proof}

    \begin{lemma}
        The singleton sets in a Hausdorff space are closed.
    \end{lemma}
    \begin{proof}
        Let $x \in X$ where $(X, \tau)$ is Hausdorff. Pick $y \neq x$, whence there
        exist $U_y, V_y \in \tau$, such that $x \in U_y$, $y \in V_y$, and $U_y \cap
        V_y = \emptyset$. In particular, $\{x\} \cap V_y = \emptyset$. We now have \[
            X\setminus \{x\} = \bigcup_{y \neq x} V_y,
        \] which is open.
    \end{proof}

    \begin{theorem}
        The topology induced by a metric is Hausdorff.
    \end{theorem}
    \begin{proof}
        Given a metric space $X$ and distinct points $x, y \in X$, we set $r = |x -
        y|$, $U = B(x, r/3)$, $V = B(y, r/3)$.
    \end{proof}
    

    \section{Continuous maps}

    \begin{definition}
        Let $f\colon X \to Y$ be a function between the topological spaces $(X,
        \tau_X)$ and $(Y, \tau_Y)$. We say that $f$ is continuous if for every $U
        \in \tau_Y$, we have $f^{-1}(U) \in \tau_X$. In other words, the pre-image of
        every open set in $Y$ must be open in $X$.
    \end{definition}

    \begin{lemma}
        A function $f\colon X \to Y$ is continuous if and only if given a base
        $\beta$ of $Y$, we have $f^{-1}(U) \in \tau_X$ for every $U \in \beta$.
    \end{lemma}
    \begin{example}
        The identity function $\operatorname{id}\colon \R_\ell \to \R$ is continuous, while the
        identity function $\operatorname{id}\colon \R \to \R_\ell$ is not. This is
        because the topology on $\R_\ell$ is strictly finer than that on $\R$.
    \end{example}

    \begin{lemma}
        A function $f\colon X \to Y$ is continuous if and only if for every closed
        set $F \subseteq Y$, we have $f^{-1}(F)$ closed in $X$.
    \end{lemma}

    \begin{lemma}
        A function $f\colon X \to Y$ is continuous if and only if given any $x \in X$
        and an open set $V \subseteq Y$ such that $f(x) \in V$, there exists an open
        set $U \subseteq X$ such that $x \in U$, $f(U) \subseteq V$.
    \end{lemma}

    \begin{theorem}
        The composition of continuous functions is continuous.
    \end{theorem}

    
    \subsection{Restricting and enlarging the domain}

    \begin{lemma}
        Let $f\colon X \to Y$ be continuous, and let $A \subset X$. Then the
        restriction of $f$ to $A$ is continuous.
    \end{lemma}

    \begin{theorem}
        Let $f\colon X \to Y$, and let $X$ be the union of the collection of open
        sets $\{A_\lambda\}_{\lambda \in \Lambda}$. If the restrictions of $f$ to
        each $A_\lambda$ are continuous, then $f$ is continuous.
    \end{theorem}
    \begin{proof}
        Pick $x \in X$, hence $x \in A_\lambda$ for some $\lambda \in \Lambda$. Now
        if $f(x) \in V \subset Y$, where $V$ is open in $Y$, then the continuity of
        the restriction of $f$ to $A_\lambda$ gives us an open set $U \subseteq
        A_\lambda$ such that $f(U) \subseteq V$. Finally since $A_\lambda$ is open in
        $X$, so is $U$.
    \end{proof}
    
    \begin{definition}
        Let $X$ be the union of the collection of open sets $\{A_\lambda\}_{\lambda
        \in \Lambda}$. We say that this collection is a locally finite cover of $X$
        if given $x \in X$, there exists a neighbourhood $U$ of $x$ such that $U \cap
        A_\lambda$ is non-empty for only finitely many $\lambda \in \Lambda$.
    \end{definition}
    
    \begin{theorem}
        Let $f\colon X \to Y$, and let $\{F_\lambda\}_{\lambda \in \Lambda}$ be a
        locally finite collection of closed sets covering $X$. If the restrictions of
        $f$ to each $F_\lambda$ are continuous, then $f$ is continuous.
    \end{theorem}

    \begin{corollary}[Pasting lemma]
        Let $X = A\cup B$, with $A, B$ closed in $X$. Let $f\colon A \to Y$, $g\colon
        B \to Y$ be continuous, with $f(x) = g(x)$ on $A \cap B$. Then the function
        $h\colon X \to Y$, defined by $x\mapsto f(x)$ on $A$ and $x\mapsto g(x)$ on
        $B$, is continuous.
    \end{corollary}

    \begin{definition}
        A path is a continuous function $\gamma\colon [0, 1] \to X$.
    \end{definition}

    \begin{lemma}
        Two paths $\gamma_1, \gamma_2$ can be concatenated when $\gamma_1(1) =
        \gamma_2(0)$.
    \end{lemma}


    \subsection{Projection maps}

    \begin{theorem}
        The projection maps $\pi_i\colon X_1\times \dots \times X_k \to X_i$ are
        continuous, when the domain is equipped with the product topology.
        Furthermore, the product topology is the coarsest topology on the domain for
        which the projection maps are continuous.
    \end{theorem}

    \begin{lemma}
        Let $f\colon A \to X_1 \times \dots \times X_k$, where the co-domain is
        equipped with the product topology. Then, $f$ is continuous if and only if
        the component functions $f_i = \pi_i\circ f$ are continuous.
    \end{lemma}
    \begin{proof}
        Note that if $f$ is continuous, the compositions $\pi_i\circ f$ are
        immediately continuous. Conversely suppose that each $f_i$ is continuous, and
        write \[
            f(t) = (f_1(t), \dots, f_k(t)).
        \] The sets $U_1 \times \dots \times U_k$, where each $U_i \subseteq X_i$ is
        open, form a basis of the co-domain. Furthermore, their pre-images under $f$
        are $f_1^{-1}(U_1) \cap \dots \cap f_k^{-1}(U_k)$, which are open in $A$.
        This shows that $f$ is continuous.
    \end{proof}


    \begin{definition}
        Let $J$ be an arbitrary index set. A $J$-tuple of elements in a set $X$ is a
        function $x\colon J \to X$, formally denoted $(x_\alpha)_{\alpha \in J}$.
        If $\{X_\alpha\}_{\alpha \in J}$ is a family of sets, their Cartesian product
        is defined as \[
            \prod_{\alpha \in J} X_\alpha = \{x\colon J \to \bigcup_{\alpha \in J}
            X_\alpha\colon x_\alpha \in X_\alpha\}.
        \] 
        \begin{remark}
            The fact that we can choose an element from each set in an uncountable
            collection relies on the Axiom of Choice.
        \end{remark}
    \end{definition}

    \begin{definition}
        Let $\{X_\alpha\}_{\alpha \in J}$ be a collection of topological spaces. The
        topology generated by $\prod_{\alpha \in J} U_\alpha$, where each $U_\alpha
        \subseteq X_\alpha$ is open, is called the box topology on $\prod_{\alpha \in
        J} X_\alpha$.
    \end{definition}
    

    \subsection{Homeomorphisms}

    \begin{definition}
        Let $f\colon X \to Y$ be a function between the topological spaces $(X,
        \tau_X)$ and $(Y, \tau_Y)$. We say that $f$ is a homeomorphism if $f$ is
        continuous, $f$ is bijective, and $f^{-1}$ is continuous. We also say that
        $X$ and $Y$ are homeomorphic when such a homeomorphism between them exists.
    \end{definition}
    \begin{example}
        The interval $(-1, 1)$ is homeomorphic to $\R$; for instance, the map $x
        \mapsto \tan(\pi x/ 2)$ on $(-1, 1)$ is a homeomorphism. A simpler
        construction is the map $x \mapsto x / (1 - x^2)$.
    \end{example}

\end{document}
