\documentclass[11pt]{article}

\usepackage[T1]{fontenc}
\usepackage{geometry}
\usepackage{amsmath, amssymb, amsthm}
\usepackage[scr]{rsfso}

\geometry{a4paper}

\newtheorem*{theorem}{Theorem}
\newtheorem*{corollary}{Corollary}
\newtheorem*{lemma}{Lemma}

\theoremstyle{definition}
\newtheorem{definition}{Definition}[section]

\theoremstyle{remark}
\newtheorem*{remark}{Remark}
\newtheorem*{example}{Example}

\title{
    \Large\textsc{MA4103: Analysis V} \\
    \Huge \textbf{Selected notes} \\
    \vspace{5pt}
    \Large{1 November, 2022}
}
\author{
    \large Satvik Saha
    \\\textsc{\small 19MS154}
}
\date{\normalsize
    \textit{Indian Institute of Science Education and Research, Kolkata, \\
    Mohanpur, West Bengal, 741246, India.} \\
}

\begin{document}

    \maketitle

    \section*{The Fourier Transform}

    For $f \in L^1(\mathbb{T})$, we define its Fourier transform as the map \[
        \hat{f}\colon \mathbb{Z} \to \mathbb{C}, \qquad
        n \mapsto \frac{1}{2\pi} \int_{-\pi}^{\pi} f(t) e^{-int} \:dt.
    \] Its Fourier series at some $\theta \in [-\pi, \pi]$ is given by \[
        \sum_{n = -\infty}^{\infty} \hat{f}(n) e^{in\theta} \equiv
        \lim_{N \to \infty} \underbrace{\sum_{n = -N}^{N} \hat{f}(n)
        e^{in\theta}}_{S_N f(\theta)}.
    \] Thus, we have an association \[
        L^1(\mathbb{T}) \to \mathscr{F}(\mathbb{Z}), \qquad
        f \mapsto \hat{f}.
    \]

    \begin{lemma}
        The map $f \mapsto \hat{f}$ is a homomorphism of algebras
        $(L^1(\mathbb{T}), +, *)$ and $(\mathscr{F}(\mathbb{Z}), +, \cdot)$. In other
        words, \[
            \widehat{(f + g)}(n) = \hat{f}(n) + \hat{g}(n), \qquad
            \widehat{(\alpha f)}(n) = \alpha \hat{f}(n), \qquad
            \widehat{(f * g)}(n) = \hat{f}(n)\, \hat{g}(n)
        \] for all $f, g \in L^1(\mathbb{T}), \alpha \in \mathbb{C}, n \in
        \mathbb{Z}$.
    \end{lemma}

    \section*{Trigonometric polynomials}

    We say that $g$ is a trigonometric polynomial if it is of the form \[
        g(\theta) = \sum_{n = -N}^N a_n e^{in\theta}.
    \] For example, the family of polynomials \[
        Q_n(\theta) = c_n\left(\frac{1 + \cos\theta}{2}\right)^n
    \] can be shown to be trigonometric polynomials simply by substituting
    $2\cos\theta = e^{i\theta} + e^{-i\theta}$ and expanding.

    \begin{theorem}
        The collection of all trigonometric polynomials is dense in
        $L^1(\mathbb{T})$.
    \end{theorem}
    \begin{proof}[Sketch of proof]
        Since $C(\mathbb{T})$ is dense in $L^1(\mathbb{T})$, it is enough to show
        that any $f \in C(\mathbb{T})$ can be written as the limit of trigonometric
        polynomials. Recall that we already have $f * Q_n \to f$ in $L^\infty$, hence
        in $L^1$; this is because the family $\{Q_n\}$ forms a summability kernel,
        hence an approximate identity, and the $L^1$ norm on the compact space of
        finite measure $\mathbb{T}$ is dominated by the $L^\infty$ norm. Again, this
        means that it is enough to show that all $f * Q_n$ are trigonometric
        polynomials.

        Denote $\varphi_n(t) = e^{int}$. Then, \[
            \hat{\varphi}_n(k) = \int_{\mathbb{T}} e^{int} e^{-ikt} \:dt =
            \delta_{nk}.
        \] Thus, for any trigonometric polynomial \[
            g = \sum_{n = -N}^N a_n\varphi_n,
        \] we have \[
            \hat{g}(k) = \sum_{n = -N}^N a_n\delta_{nk} = \begin{cases}
                a_k, & |k| \leq N, \\
                0, & \text{otherwise}.
            \end{cases}
        \] Now compute, \[
            (f * \varphi_n)(t) = \frac{1}{2\pi} \int f(t - x)\, e^{inx}\:dx =
            \frac{1}{2\pi} \int f(t - x)\, e^{-in(t - x)} e^{int}\:dx =
            \hat{f}(n)\varphi_n(t).
        \] Thus, write \[
            (f * g)(t) = \sum_{n = -N}^N a_n (f * \varphi_n)(t) =
            \sum_{n = -N}^N a_n \hat{f}(n) \varphi_n(t),
        \] which immediately shows that all convolutions of $f$ with trigonometric
        polynomials are trigonometric polynomials too.
    \end{proof}

    \begin{corollary}
        The set $\{e^{int}\}_{n \in \mathbb{Z}}$ is an orthonormal basis of
        $L^2(\mathbb{T})$.
    \end{corollary}
    \begin{proof}
        The given set is orthonormal, and the collection of all its finite linear
        combinations (i.e.\ trigonometric polynomials) is dense in $L^2(\mathbb{T})$;
        the latter follows from the previous proof.
    \end{proof}

    \begin{lemma}[Riemann-Lebesgue]
        Let $f \in L^1(\mathbb{T})$. Then, \[
            |\hat{f}(n)| \leq \Vert f\Vert_1, \qquad
            \lim_{|n| \to \infty} |\hat{f}(n)| = 0.
        \] As a result, the map $L^1(\mathbb{T}) \to \mathscr{F}(\mathbb{Z})$
        described earlier is a bounded linear operator mapping into the subspace
        $c_0 \subseteq \mathscr{F}(\mathbb{Z})$.
        \begin{remark}
            This map is injective, but not surjective.
        \end{remark}
    \end{lemma}
    \begin{proof}[Sketch of proof]
        Use the density of trigonometric polynomials to conclude that it is enough to
        show that the above formulae hold for all $g$ of the form \[
            g(\theta) = \sum_{n = -N}^N a_n e^{in\theta}.
        \] However, we already know that \[
            \hat{g}(k) = \sum_{n = -N}^N a_n\delta_{nk} = \begin{cases}
                a_k, & |k| \leq N, \\
                0, & \text{otherwise},
            \end{cases}
        \] from which it is immediate that $\lim_{|k| \to \infty} |\hat{g}(k)| = 0$.

        To show that the map $L^1(\mathbb{T}) \to c_0$ is injective, pick $f \in
        L^1(\mathbb{T})$ such that $\hat{f}(n) = 0$ for all $n \in \mathbb{Z}$. Then,
        each $f * Q_n = 0$; but $f * Q_n \to f$ in $L^1$ forces $f = 0$ almost
        everywhere.
    \end{proof}


    \section*{Convergence of Fourier series}

    \begin{theorem}
        Let $f \in L^1(\mathbb{T})$, and suppose that \[
            \sum_{n = -\infty}^\infty |\hat{f}(n)| < \infty.
        \] Then, \[
            f(\theta) = \sum_{n = -\infty}^\infty \hat{f}(\theta) e^{in\theta} \qquad
            \text{a.e.}
        \] If we further suppose that $f \in C(\mathbb{T})$, then equality holds
        everywhere.
    \end{theorem}
    \begin{proof}[Sketch of proof]
        Setting \[
            g(\theta) = \sum_{n = -\infty}^\infty \hat{f}(\theta) e^{in\theta},
            \qquad
            g_N(\theta) = \sum_{n = -N}^N \hat{f}(\theta) e^{in\theta},
        \] we claim that $g$ is continuous. Indeed, as $N \to \infty$, we have \[
            \Vert g - g_N\Vert_\infty \leq \sum_{|n| > N} |\hat{f}(n)| \to 0
        \] since the tail of the convergent series $\sum_{n \in \mathbb{Z}}
        |\hat{f}(n)|$ vanishes. Thus, $g$ being the uniform limit of continuous
        functions must also be continuous. This in turn gives all $\hat{g}(n) =
        \hat{f}(n)$, hence $g = f$ almost everywhere (we have $\widehat{(g - f)} = 0$
        and know that the Fourier transform is injective).
    \end{proof}


    \begin{lemma}
        Let $f \in C^1(\mathbb{T})$. Then, for all $n \in \mathbb{Z}$, we have \[
            \widehat{(f')}(n) = in\, \hat{f}(n).
        \] If we further suppose that $f \in C^2(\mathbb{T})$, then there exists $C >
        0$ for which the following holds for all $n \in \mathbb{Z}\setminus\{0\}$. \[
            |\hat{f}(n)| \leq \frac{C}{n^2}.
        \]
        \begin{remark}
            The second condition immediately gives $\sum_{n \in \mathbb{Z}}
            |\hat{f}(n)| < \infty$, guaranteeing the convergence and equality of the
            Fourier series of $f$ everywhere.
        \end{remark}
    \end{lemma}

\end{document}
