\documentclass[11pt]{article}

\usepackage[T1]{fontenc}
\usepackage{geometry}
\usepackage{amsmath, amssymb, amsthm}
\usepackage[scr]{rsfso}
\usepackage{bm}
\usepackage[%
    hidealllines=true,%
    innerbottommargin=15,%
    nobreak=true,%
]{mdframed}
\usepackage{xcolor}
\usepackage{graphicx}
\usepackage{fancyhdr}
\usepackage{hyperref}

\geometry{a4paper, margin=1in, headheight=14pt}

\pagestyle{fancy}
\fancyhf{}
\renewcommand\headrulewidth{0.4pt}
\fancyhead[L]{\scshape MA4103: Analysis V}
\fancyhead[R]{\scshape \leftmark}
\rfoot{\footnotesize\it Updated on \today}
\cfoot{\thepage}

\newcommand{\C}{\mathbb{C}}
\newcommand{\R}{\mathbb{R}}
\newcommand{\Q}{\mathbb{Q}}
\newcommand{\Z}{\mathbb{Z}}
\newcommand{\N}{\mathbb{N}}

\newcommand{\M}{\mathcal{M}}

\newmdtheoremenv[%
    backgroundcolor=blue!10!white,%
]{theorem}{Theorem}[section]
\newmdtheoremenv[%
    backgroundcolor=violet!10!white,%
]{corollary}{Corollary}[theorem]
\newmdtheoremenv[%
    backgroundcolor=teal!10!white,%
]{lemma}[theorem]{Lemma}

\theoremstyle{definition}
\newmdtheoremenv[%
    backgroundcolor=green!10!white,%
]{definition}{Definition}[section]
\newmdtheoremenv[%
    backgroundcolor=red!10!white,%
]{exercise}{Exercise}[section]

\theoremstyle{remark}
\newtheorem*{remark}{Remark}
\newtheorem*{example}{Example}
\newtheorem*{solution}{Solution}

\surroundwithmdframed[%
    linecolor=black!20!white,%
    hidealllines=false,%
    innertopmargin=5,%
    innerbottommargin=10,%
    skipabove=0,%
    skipbelow=0,%
]{example}

\title{
    \Large\textsc{MA4103} \\
    \Huge \textbf{Analysis V} \\
    \vspace{5pt}
    \Large{Autumn 2022}
}
\author{
    \large Satvik Saha
    \\\textsc{\small 19MS154}
}
\date{\normalsize
    \textit{Indian Institute of Science Education and Research, Kolkata, \\
    Mohanpur, West Bengal, 741246, India.} \\
}

\begin{document}
    \maketitle

    \tableofcontents

    \section{Signed measures}

    As a motivating example, consider the space of linear functionals on $\C^n$; each
    element of this space is a linear function $\phi\colon \C^n \to \C$. If we denote
    the standard basis of $\C^n$ by $\{e_i\}$ and its dual basis by $\{\phi_i\}$,
    any linear functional $\phi$ can be written as \[
        \phi = \sum_{i = 1}^n \phi(e_i)\, \phi_i.
    \] Now we switch to a different perspective; note that each element of $\C^n$ is
    an $n$-tuple of complex numbers of the form $(z_1, z_n, \dots, z_n)$. In other
    words, it is an assignment of complex numbers to the indices $X = \{1, 2, \dots,
    n\}$. Thus, each element of $\C^n$ can be associated with a function $f\colon X
    \to \C$, and vice versa. This gives us an identification of $\C^n$ with the space
    of functions $C(X)$ (all such functions are continuous when $X$ is equipped with
    the discrete topology). For instance, the basis vectors $\{e_i\}$ of $C(X)$ can
    now be represented by the functions \[
        e_i\colon X \to \C, \qquad j \mapsto \delta_{ij}.
    \] Now, in order to extract the `coordinates' $z_i \equiv f(i)$ from some $f \in
    C(X)$, we may define the following Dirac measures $\mu_i$ on $X$, concentrated at
    the index $i$. \[
        \mu_i\colon \mathscr{P}(X) \to C, \qquad
        E \mapsto \begin{cases}
            1, &\text{ if } i \in A, \\
            0, &\text{ otherwise}.
        \end{cases}
    \] This means that for any $f \in C(X)$, \[
        f(i) = \int f\:d\mu_i.
    \] The dual basis $\{\phi_i\}$ behaves as $\phi_i(e_j) = \delta_{ij}$, so \[
        \phi_i(f) = \phi_i\sum_{j = 1}^n f(j)\,e_j = f(i) = \int f\:d\mu_i.
    \] Thus, for any linear functional $\phi\colon C(X) \to \C$, we have \[
        \phi(f) = \sum_{i = 1}^n \phi(e_i)\, \phi_i(f) = \sum_{i = 1}^n \phi(e_i)\,
        \int f\:d\mu_i = \int f \sum_{i = 1}^n \phi(e_i)\:d\mu_i.
    \] If we could make sense of the measure $\mu$ described by \[
        \mu = \sum_{i = 1}^n \phi(e_i)\: \mu_i,
    \] then we could write \[
        \phi(f) = \int f\:d\mu.
    \] Note that the coefficients $\phi(e_i)$ are complex numbers, so $\mu$ is not
    necessarily a measure in the conventional sense!


    \subsection{Basic definitions}

    \begin{definition}
        Let $\M$ be a $\sigma$-algebra over $X$. A function $\nu\colon \M \to
        [-\infty, +\infty]$ is called a signed measure if \begin{enumerate}
            \itemsep0em
            \item $\nu(\emptyset) = 0$.
            \item For all countable collections of disjoint measurable sets
            $\{E_i\}$, we have \[
                \nu\left(\bigcup_{i = 1}^\infty E_i\right) = \sum_{i = 1}^\infty
                \nu(E_i).
            \]
        \end{enumerate}
        \begin{remark}
            For disjoint sets $E, F \in \M$, we want \[
                \nu(E \cup F) = \nu(E) + \nu(F).
            \] Thus, we cannot allow the situation where $\nu(E) = +\infty$ and
            $\nu(F) = -\infty$, or vice versa.
        \end{remark}
        \begin{remark}
            Note that in condition 2, the union on the left hand side is independent
            of order, while the infinite sum on the right hand side is not. Thus, we
            demand that either the sum on the right converges absolutely, or the
            following: neither one of the sums \[
                \sum_{i: \nu(E_i) \geq 0} \nu(E_i), \qquad
                \sum_{j: \nu(E_j) < 0} \nu(E_j)
            \] should diverge to $\infty$.
        \end{remark}
    \end{definition}

    \begin{lemma}
        Let $E, F \in \M$, with $E \subseteq F$. Then, \begin{enumerate}
            \item If $|\nu(E)| < \infty$, then \[
                \nu(F\setminus E) = \nu(F) - \nu(E).
            \]
            \item If $\nu(E) = \pm\infty$, then $\nu(F) = \pm\infty$.
        \end{enumerate}
    \end{lemma}
    \begin{proof}
        Using $\nu(F) = \nu(F\setminus E) + \nu(E)$, we obtain \emph{1} when
        $|\nu(E)| < \infty$. Otherwise, for $\nu(E) = \infty$, we cannot have
        $\nu(F\setminus E) = \mp\infty$, hence we must have $\nu(F) = \pm \infty$.
    \end{proof}

    \begin{corollary}
        If any measurable set $E\subseteq X$ has $\nu(E) = \pm\infty$, then $\nu(X) =
        \pm\infty$. This immediately shows that we cannot have two measurable sets
        $E, F \in \M$ with $\nu(E) = +\infty$, $\nu(F) = -\infty$. In other words, a
        signed measure has either of the following forms. \[
            \nu\colon \M \to [-\infty, +\infty), \quad\text{or}\quad
            \nu\colon \M \to (-\infty, +\infty].
        \]
    \end{corollary}

    \begin{example}
        Consider a non-negative measure $\mu$, and a function $f\in L^1(\mu)$. Then,
        the measure defined by \[
            \nu(E) = \int_E f\:d\mu
        \] is a (finite) signed measure.
    \end{example}
    \begin{example}
        Consider any two non-negative measures $\mu_1, \mu_2 \geq 0$. Then, the
        measure $\nu$ defined by \[
            \nu = \mu_1 - \mu_2
        \] where either one of $\mu_1, \mu_2$ is finite is a signed measure. Indeed,
        any signed measure is of this form.
    \end{example}

    \begin{lemma}[Continuity from below]
        Let $\{E_i\}$ be a collection of measurable sets such that each $E_i
        \subseteq E_{i + 1}$. Then, \[
            \lim_{n \to \infty} \nu(E_n) = \nu\left(\bigcup_{n = 1}^\infty
            E_n\right).
        \]
    \end{lemma}

    \begin{lemma}[Continuity from above]
        Let $\{E_i\}$ be a collection of measurable sets such that each $E_i
        \supseteq E_{i + 1}$, and $|\nu(E_1)| < \infty$. Then, \[
            \lim_{n \to \infty} \nu(E_n) = \nu\left(\bigcap_{n = 1}^\infty
            E_n\right).
        \]
    \end{lemma}

    \subsection{The Hahn decomposition}

    \begin{definition}
        A set $P \in \M$ is called a positive set for $\nu$ if for every measurable
        subset $E \subseteq P$, we have $\nu(E) \geq 0$.
    \end{definition}
    \begin{definition}
        A set $N \in \M$ is called a negative set for $\nu$ if for every measurable
        subset $E \subseteq N$, we have $\nu(E) \leq 0$.
    \end{definition}
    \begin{definition}
        A set $F \in \M$ is called a null set for $\nu$ if for every measurable
        subset $E \subseteq F$, we have $\nu(E) = 0$.
    \end{definition}

    \begin{example}
        Let $\mu$ be a non-negative measure and let $f$ be a measurable function.
        Define the signed measure \[
            \nu(E) = \int_E f\:d\mu.
        \] It is clear that any measurable subset of $f^{-1}[0, \infty]$ is a
        positive set for $\nu$, any measurable subset of $f^{-1}[-\infty, 0]$ is a
        negative set for $\nu$, and any measurable subset of $f^{-1}(0)$ is a null
        set for $\nu$.
    \end{example}

    \begin{lemma}
        Every measurable subset of a positive(/negative/null) set is
        positive(/negative/null).
    \end{lemma}
    \begin{proof}
        Let $P$ be a positive set for $\nu$, and $E \subseteq P$ be a measurable
        subset. We claim that for all measurable subsets $F \subseteq E$, $\nu(F)
        \geq 0$. This is immediate from the fact that $F \subseteq E \subseteq P$ is
        a measurable subset of the positive set $P$. The cases for negative and null
        sets are analogous.
    \end{proof}

    \begin{lemma}
        Countable unions of positive(/negative/null) sets are
        positive(/negative/null).
    \end{lemma}
    \begin{proof}
        Let $\{P_i\}$ be a countable collection of positive measurable sets, and let
        $P = \bigcup_{i = 1}^\infty P_i$. Define the sets \[
            Q_i = P_i \setminus \bigcup_{j = 1}^{i - 1} P_j,
        \] and note that $P = \bigcup_{i = 1}^\infty Q_i$, with the collection
        $\{Q_i\}$ being disjoint. Furthermore, each $Q_i \subseteq P_i$ is a positive
        set. Now for any measurable $E \subseteq P$, each $E \cap Q_i \subseteq Q_i$
        is a positive set, hence \[
            E = \bigcup_{i = 1}^\infty E \cap Q_i = \sum_{i = 1}^\infty \nu(E \cap
            Q_i) \geq 0.
        \] The cases for negative and null sets are analogous.
    \end{proof}

    \begin{lemma}
        Let $E \in \M$ such that $\nu(E) > 0$. Then, there exists a measurable subset
        $\tilde{E} \subseteq E$ such that $\tilde{E}$ is a positive set for $\nu$
        and $\nu(\tilde{E}) > 0$.
    \end{lemma}

    \begin{theorem}[Hahn Decomposition]
        Let $\nu$ be a signed measure on $(X, \M)$. Then, there exists a positive set
        $P$ and a negative set $N$ for $\nu$ such that $P \cup N = X$, $P \cap N =
        \emptyset$. Furthermore, if $P, N$ and $P', N'$ are two decompositions of
        $X$, then the symmetric difference $P\Delta P' = N\Delta N'$ is a null set
        for $\nu$.
    \end{theorem}
    \begin{proof}
        Without loss of generality, suppose that $\nu$ is of the form $\nu\colon \M
        \to [-\infty, +\infty)$. Let $\mathscr{P} \subseteq$ be the collection of all
        positive measurable sets for $\nu$. Note that this collection is non-empty,
        since $\emptyset \in \mathscr{P}$. Set \[
            m = \sup_{E \in \mathscr{P}} \nu(E).
        \] Note that from the properties of the supremum, there exists a sequence of
        sets $\{P_i\}_{i \in \N}$ from $\mathscr{P}$ such that $\nu(P_i) \to m$. Set
        $P = \bigcup_{i = 1}^\infty P_i$, and note that $P$ is a positive set for
        $\nu$, i.e.\ $P \in \mathscr{P}$. We claim that $\nu(P) = m$. To see this,
        define \[
            Q_i = \bigcup_{j = 1}^i P_i,
        \] and note that $\{Q_i\}$ is an increasing sequence of positive measurable
        sets for $\nu$. Also, each $Q_i \supseteq P_i$, so $0 \leq \nu(P_i) \leq
        \nu(Q_i) \leq m$. Taking limits, we have $\nu(Q_i) \to m$. Thus, continuity
        from below gives \[
            \nu(P) = \nu\left(\bigcup_{i = 1}^\infty Q_i\right) = \lim_{n \to \infty}
            \nu(Q_i) = m.
        \]

        Now, set $N = X\setminus P$. We claim that $N$ is a negative set for $\nu$.
        Indeed, if not, then we could find some measurable $E \subseteq N$ with
        $\nu(E) > 0$. By the previous lemma, this yields a positive set $\tilde{E}
        \subseteq E$ for $\nu$ with $\nu(\tilde{E}) > 0$. Now, $\tilde{E} \cup P$ is
        a positive set for $\nu$, i.e.\ $\tilde{E} \cup P \in \mathscr{P}$. Also,
        $\tilde{E} \cap P = \emptyset$, so \[
            \nu(\tilde{E} \cup P) = \nu(\tilde{E}) + \nu(P) > m.
        \] This contradicts the maximality of $m$.

        Thus, we have obtained a Hahn decomposition $P, N$ of $X$. If $P', N'$ is
        another Hahn decomposition of $X$, then \[
            P\Delta P = (P\setminus P') \cup (P'\setminus P) = (N^c\setminus N'^c)
            \cup (N'^c\setminus N^c) = (N'\setminus N) \cup (N\setminus N') = N\Delta
            N'.
        \] Furthermore, $P\setminus P' \subseteq P$ is a positive set for $\nu$;
        $P\setminus P' \subseteq P'^c = N'$ is also a negative set for $\nu$. This
        means that $P\setminus P'$ must be a null set for $\nu$. The same reasoning
        shows that $P'\setminus P$ is a null set for $\nu$, hence so is $P\Delta P'$.
    \end{proof}

    \begin{corollary}
        Given a signed measure $\nu$ on $(X, \M)$, consider the Hahn decomposition
        $P, N$ of $X$ and define the measures $\nu^+$, $\nu^-$ as \[
            \nu^+(E) = \nu(E \cap P) \geq 0, \qquad
            \nu^-(E) = -\nu(E \cap N) \geq 0.
        \] Then, we have the decomposition of $\nu$ as the difference of the non-negative
        measures $\nu^+$, $\nu^-$ as \[
            \nu = \nu^+ - \nu^-.
        \]
    \end{corollary}

\end{document}
