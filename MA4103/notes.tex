\documentclass[11pt]{article}

\usepackage[T1]{fontenc}
\usepackage{geometry}
\usepackage{amsmath, amssymb, amsthm}
\usepackage[scr]{rsfso}
\usepackage{bm}
\usepackage[%
    hidealllines=true,%
    innertopmargin=15,%
    innerbottommargin=15,%
    nobreak=true,%
]{mdframed}
\usepackage{xcolor}
\usepackage{graphicx}
\usepackage{fancyhdr}
\usepackage{hyperref}
\usepackage{enumitem}

\geometry{a4paper, margin=1in, headheight=14pt}

\setlength{\parindent}{0em}

\setlist[enumerate]{topsep=0em, itemsep=0em}

\pagestyle{fancy}
\fancyhf{}
\renewcommand\headrulewidth{0.4pt}
\fancyhead[L]{\scshape MA4103: Analysis V}
\fancyhead[R]{\scshape \leftmark}
\rfoot{\footnotesize\it Updated on \today}
\cfoot{\thepage}

\newcommand{\C}{\mathbb{C}}
\newcommand{\R}{\mathbb{R}}
\newcommand{\Q}{\mathbb{Q}}
\newcommand{\Z}{\mathbb{Z}}
\newcommand{\N}{\mathbb{N}}

\newcommand{\M}{\mathcal{M}}
\newcommand{\Lloc}{L^1_\text{loc}}

\newmdtheoremenv[%
    backgroundcolor=blue!10!white,%
]{theorem}{Theorem}[section]
\newmdtheoremenv[%
    backgroundcolor=violet!10!white,%
]{corollary}{Corollary}[theorem]
\newmdtheoremenv[%
    backgroundcolor=teal!10!white,%
]{lemma}[theorem]{Lemma}

\theoremstyle{definition}
\newmdtheoremenv[%
    backgroundcolor=green!10!white,%
]{definition}{Definition}[section]
\newmdtheoremenv[%
    backgroundcolor=red!10!white,%
]{exercise}{Exercise}[section]

\theoremstyle{remark}
\newtheorem*{remark}{Remark}
\newtheorem*{example}{Example}
\newtheorem*{solution}{Solution}

\surroundwithmdframed[%
    linecolor=black!20!white,%
    hidealllines=false,%
    innerbottommargin=10,%
    skipabove=0,%
    skipbelow=0,%
]{example}

\title{
    \Large\textsc{MA4103} \\
    \Huge \textbf{Analysis V} \\
    \vspace{5pt}
    \Large{Autumn 2022}
}
\author{
    \large Satvik Saha
    \\\textsc{\small 19MS154}
}
\date{\normalsize
    \textit{Indian Institute of Science Education and Research, Kolkata, \\
    Mohanpur, West Bengal, 741246, India.} \\
}

\begin{document}
    \maketitle

    \tableofcontents
    \setlength{\parskip}{1em}

    \section{Signed measures}

    As a motivating example, consider the space of linear functionals on $\C^n$; each
    element of this space is a linear function $\phi\colon \C^n \to \C$. If we denote
    the standard basis of $\C^n$ by $\{e_i\}$ and its dual basis by $\{\phi_i\}$,
    any linear functional $\phi$ can be written as \[
        \phi = \sum_{i = 1}^n \phi(e_i)\, \phi_i.
    \] Now we switch to a different perspective; note that each element of $\C^n$ is
    an $n$-tuple of complex numbers of the form $(z_1, z_n, \dots, z_n)$. In other
    words, it is an assignment of complex numbers to the indices $X = \{1, 2, \dots,
    n\}$. Thus, each element of $\C^n$ can be associated with a function $f\colon X
    \to \C$, and vice versa. This gives us an identification of $\C^n$ with the space
    of functions $C(X)$ (all such functions are continuous when $X$ is equipped with
    the discrete topology). For instance, the basis vectors $\{e_i\}$ of $C(X)$ can
    now be represented by the functions \[
        e_i\colon X \to \C, \qquad j \mapsto \delta_{ij}.
    \] Now, in order to extract the `coordinates' $z_i \equiv f(i)$ from some $f \in
    C(X)$, we may define the following Dirac measures $\mu_i$ on $X$, concentrated at
    the index $i$. \[
        \mu_i\colon \mathcal{P}(X) \to C, \qquad
        E \mapsto \begin{cases}
            1, &\text{ if } i \in A, \\
            0, &\text{ otherwise}.
        \end{cases}
    \] This means that for any $f \in C(X)$, \[
        f(i) = \int f\:d\mu_i.
    \] The dual basis $\{\phi_i\}$ behaves as $\phi_i(e_j) = \delta_{ij}$, so \[
        \phi_i(f) = \phi_i\sum_{j = 1}^n f(j)\,e_j = f(i) = \int f\:d\mu_i.
    \] Thus, for any linear functional $\phi\colon C(X) \to \C$, we have \[
        \phi(f) = \sum_{i = 1}^n \phi(e_i)\, \phi_i(f) = \sum_{i = 1}^n \phi(e_i)\,
        \int f\:d\mu_i = \int f \sum_{i = 1}^n \phi(e_i)\:d\mu_i.
    \] If we could make sense of the measure $\mu$ described by \[
        \mu = \sum_{i = 1}^n \phi(e_i)\: \mu_i,
    \] then we could write \[
        \phi(f) = \int f\:d\mu.
    \] Note that the coefficients $\phi(e_i)$ are complex numbers, so $\mu$ is not
    necessarily a measure in the conventional sense!


    \subsection{Basic definitions}

    \begin{definition}
        Let $\M$ be a $\sigma$-algebra over $X$. A function $\nu\colon \M \to
        [-\infty, +\infty]$ is called a signed measure if \begin{enumerate}
            \item $\nu(\emptyset) = 0$.
            \item For all countable collections of disjoint measurable sets
            $\{E_i\}$, we have \[
                \nu\left(\bigcup_{i = 1}^\infty E_i\right) = \sum_{i = 1}^\infty
                \nu(E_i).
            \]
        \end{enumerate}
        \begin{remark}
            For disjoint sets $E, F \in \M$, we want \[
                \nu(E \cup F) = \nu(E) + \nu(F).
            \] Thus, we cannot allow the situation where $\nu(E) = +\infty$ and
            $\nu(F) = -\infty$, or vice versa.
        \end{remark}
        \begin{remark}
            Note that in condition 2, the union on the left hand side is independent
            of order, while the infinite sum on the right hand side is not. Thus, we
            demand that either the sum on the right converges absolutely, or the
            following: neither one of the sums \[
                \sum_{i: \nu(E_i) \geq 0} \nu(E_i), \qquad
                \sum_{j: \nu(E_j) < 0} \nu(E_j)
            \] should diverge to $\infty$.
        \end{remark}
    \end{definition}

    \begin{lemma}
        Let $E, F \in \M$, with $E \subseteq F$. Then, \begin{enumerate}
            \item If $|\nu(E)| < \infty$, then \[
                \nu(F\setminus E) = \nu(F) - \nu(E).
            \]
            \item If $\nu(E) = \pm\infty$, then $\nu(F) = \pm\infty$.
        \end{enumerate}
    \end{lemma}
    \begin{proof}
        Using $\nu(F) = \nu(F\setminus E) + \nu(E)$, we obtain \emph{1} when
        $|\nu(E)| < \infty$. Otherwise, for $\nu(E) = \infty$, we cannot have
        $\nu(F\setminus E) = \mp\infty$, hence we must have $\nu(F) = \pm \infty$.
    \end{proof}

    \begin{corollary}
        If any measurable set $E\subseteq X$ has $\nu(E) = \pm\infty$, then $\nu(X) =
        \pm\infty$. This immediately shows that we cannot have two measurable sets
        $E, F \in \M$ with $\nu(E) = +\infty$, $\nu(F) = -\infty$. In other words, a
        signed measure has either of the following forms. \[
            \nu\colon \M \to [-\infty, +\infty), \quad\text{or}\quad
            \nu\colon \M \to (-\infty, +\infty].
        \]
    \end{corollary}

    \begin{example}
        Consider a non-negative measure $\mu$, and a function $f\in L^1(\mu)$. Then,
        the measure defined by \[
            \nu(E) = \int_E f\:d\mu
        \] is a (finite) signed measure.
    \end{example}
    \begin{example}
        Consider any two non-negative measures $\mu_1, \mu_2 \geq 0$. Then, the
        measure $\nu$ defined by \[
            \nu = \mu_1 - \mu_2
        \] where either one of $\mu_1, \mu_2$ is finite is a signed measure. Indeed,
        any signed measure is of this form.
    \end{example}

    \begin{lemma}[Continuity from below]
        Let $\{E_i\}$ be a collection of measurable sets such that each $E_i
        \subseteq E_{i + 1}$. Then, \[
            \lim_{n \to \infty} \nu(E_n) = \nu\left(\bigcup_{n = 1}^\infty
            E_n\right).
        \]
    \end{lemma}

    \begin{lemma}[Continuity from above]
        Let $\{E_i\}$ be a collection of measurable sets such that each $E_i
        \supseteq E_{i + 1}$, and $|\nu(E_1)| < \infty$. Then, \[
            \lim_{n \to \infty} \nu(E_n) = \nu\left(\bigcap_{n = 1}^\infty
            E_n\right).
        \]
    \end{lemma}

    \subsection{The Hahn-Jordan decomposition theorems}

    \begin{definition}
        A set $P \in \M$ is called a positive set for $\nu$ if for every measurable
        subset $E \subseteq P$, we have $\nu(E) \geq 0$.
    \end{definition}
    \begin{definition}
        A set $N \in \M$ is called a negative set for $\nu$ if for every measurable
        subset $E \subseteq N$, we have $\nu(E) \leq 0$.
    \end{definition}
    \begin{definition}
        A set $F \in \M$ is called a null set for $\nu$ if for every measurable
        subset $E \subseteq F$, we have $\nu(E) = 0$.
    \end{definition}

    \begin{example}
        Let $\mu$ be a non-negative measure and let $f$ be a measurable function.
        Define the signed measure \[
            \nu(E) = \int_E f\:d\mu.
        \] It is clear that any measurable subset of $f^{-1}[0, \infty]$ is a
        positive set for $\nu$, any measurable subset of $f^{-1}[-\infty, 0]$ is a
        negative set for $\nu$, and any measurable subset of $f^{-1}(0)$ is a null
        set for $\nu$.
    \end{example}

    \begin{lemma}
        Every measurable subset of a positive(/negative/null) set is
        positive(/negative/null).
    \end{lemma}
    \begin{proof}
        Let $P$ be a positive set for $\nu$, and $E \subseteq P$ be a measurable
        subset. We claim that for all measurable subsets $F \subseteq E$, $\nu(F)
        \geq 0$. This is immediate from the fact that $F \subseteq E \subseteq P$ is
        a measurable subset of the positive set $P$. The cases for negative and null
        sets are analogous.
    \end{proof}

    \begin{lemma}
        Countable unions of positive(/negative/null) sets are
        positive(/negative/null).
    \end{lemma}
    \begin{proof}
        Let $\{P_i\}$ be a countable collection of positive measurable sets, and let
        $P = \bigcup_{i = 1}^\infty P_i$. Define the sets \[
            Q_i = P_i \setminus \bigcup_{j = 1}^{i - 1} P_j,
        \] and note that $P = \bigcup_{i = 1}^\infty Q_i$, with the collection
        $\{Q_i\}$ being disjoint. Furthermore, each $Q_i \subseteq P_i$ is a positive
        set. Now for any measurable $E \subseteq P$, each $E \cap Q_i \subseteq Q_i$
        is a positive set, hence \[
            E = \bigcup_{i = 1}^\infty E \cap Q_i = \sum_{i = 1}^\infty \nu(E \cap
            Q_i) \geq 0.
        \] The cases for negative and null sets are analogous.
    \end{proof}

    \begin{lemma}
        Let $\nu\colon \M \to [-\infty, +\infty)$ be a signed measure, and let $E \in
        \M$ such that $\nu(E) > 0$. Then, there exists a measurable subset $\tilde{E}
        \subseteq E$ such that $\tilde{E}$ is a positive set for $\nu$ and
        $\nu(\tilde{E}) > 0$.
    \end{lemma}
    \begin{proof}
        If $E$ is a positive set, we are done. Otherwise, there exists some $F
        \subseteq E$ such that $\nu(F) < 0$. Note that \[
            \nu(E\setminus F) = \nu(E) - \nu(F) > 0.
        \] For any non-positive set $A$, define the set \[
            S_A = \{n \in \N : \nu(B) < -1 / n \text{ for some } B \subseteq A, B \in
            \M\}.
        \] When $A$ is not a positive set, $S_A$ is clearly non-empty and hence has a
        minimum element by the Well Ordering Principle.

        We have a non-positive set $E$, hence $S_E$ has a minimum element $n_1$.
        Choose $F_1 \subseteq E$ such that $\nu(F_1) < -1 / n_1$, set $E_1 =
        E\setminus F_1$, and note that \[
            \nu(E_1) = \nu(E) - \nu(F_1) > \nu(E) + \frac{1}{n_1} > 0.
        \] If $E_1$ is a positive set we are done. Otherwise, $S_{E_1}$ has a minimum
        element $n_2$, hence we can pick $F_2 \subseteq F_1$ such that $\nu(F_2) < -1
        / n_2$. Setting $E_2 = E_1\setminus F_2 = E\setminus(F_1 \cup F_2)$, we have
        \[
            \nu(E_2) = \nu(E_1) - \nu(F_2) > \nu(E) + \frac{1}{n_1} + \frac{1}{n_2} >
            0.
        \] Again, if $E_2$ is a positive set, we are done; otherwise, we refine it to
        obtain $E_3$ as before, and so on. In this manner, if at any stage the set
        $E_k = E\setminus \bigcup_{i = 1}^k F_i$ is not a positive set, we let $n_{k
        + 1}$ be the minimum element of $S_{E_k}$, pick $F_{k + 1} \subseteq E_k$
        such that $\nu(F_{k + 1}) < -1 / n_{k + 1}$, and note that \[
            \nu(E_{k + 1}) = \nu(E_k) - \nu(F_{k + 1}) > \nu(E) + \sum_{i = 1}^{k +
            1} \frac{1}{n_i} > 0.
        \] We stop this process if at any stage $E_k$ is a positive set; otherwise,
        we obtain infinite sequences of sets $\{E_i\}$ and $\{F_i\}$. Set \[
            A = E\setminus\bigcup_{i = 1}^\infty F_i = \bigcap_{i = 1}^\infty E_i.
        \] We claim that $A$ is a positive set for $\nu$. Note that the sets
        $\{F_i\}$ are all disjoint subsets of $E$, which has finite measure, hence \[
            \nu\left(\bigcup_{i = 1}^\infty F_i\right) = \sum_{i = 1}^\infty \nu(F_i)
            > -\infty.
        \] This shows that the series \[
            \sum_{i = 1}^\infty \frac{1}{n_i} < -\sum_{i = 1}^\infty \nu(F_i) < \infty.
        \] For this convergence to hold, the terms $1 / n_i \to 0$, $n_i \to \infty$.
        Also note that \[
            \nu(A) = \nu(E) - \sum_{i = 1}^\infty \nu(F_i) > \nu(E) + \sum_{i =
            1}^\infty \frac{1}{n_i} > 0.
        \] Suppose that $A$ is not positive for $\nu$. Then, $S_A$ has a minimum
        element $m$, hence we can pick $B \subseteq A$ such that $\nu(B) < -1 / m$.
        Since $n_i \to \infty$, we can fix $k$ such that $n_k > m$. But, \[
            B \subseteq A = \bigcap_{i = 1}^\infty E_i \subseteq E_k, \qquad
            \nu(B) < -\frac{1}{m} < -\frac{1}{n_k}.
        \] This contradicts the minimality of $n_k$ with respect to the set
        $S_{E_k}$. This shows that $A \subseteq E$ is indeed a positive set of
        positive measure, as desired.
    \end{proof}

    \begin{theorem}[Hahn Decomposition]\label{theorem:hahn_decomposition}
        Let $\nu$ be a signed measure on $(X, \M)$. Then, there exists a positive set
        $P$ and a negative set $N$ for $\nu$ such that $P \cup N = X$, $P \cap N =
        \emptyset$. Furthermore, if $P, N$ and $P', N'$ are two decompositions of
        $X$, then the symmetric difference $P\Delta P' = N\Delta N'$ is a null set
        for $\nu$.
    \end{theorem}
    \begin{proof}
        Without loss of generality, suppose that $\nu$ is of the form $\nu\colon \M
        \to [-\infty, +\infty)$. Let $\mathscr{P} \subseteq \M$ be the collection of
        all positive measurable sets for $\nu$. Note that this collection is
        non-empty, since $\emptyset \in \mathscr{P}$. Set \[
            m = \sup_{E \in \mathscr{P}} \nu(E).
        \] Note that from the properties of the supremum, there exists a sequence of
        sets $\{P_i\}_{i \in \N}$ from $\mathscr{P}$ such that $\nu(P_i) \to m$. Set
        $P = \bigcup_{i = 1}^\infty P_i$, and note that $P$ is a positive set for
        $\nu$, i.e.\ $P \in \mathscr{P}$. We claim that $\nu(P) = m$. To see this,
        define \[
            Q_i = \bigcup_{j = 1}^i P_i,
        \] and note that $\{Q_i\}$ is an increasing sequence of positive measurable
        sets for $\nu$. Also, each $Q_i \supseteq P_i$, so $0 \leq \nu(P_i) \leq
        \nu(Q_i) \leq m$. Taking limits, we have $\nu(Q_i) \to m$. Thus, continuity
        from below gives \[
            \nu(P) = \nu\left(\bigcup_{i = 1}^\infty Q_i\right) = \lim_{n \to \infty}
            \nu(Q_i) = m.
        \]

        Now, set $N = X\setminus P$. We claim that $N$ is a negative set for $\nu$.
        Indeed, if not, then we could find some measurable $E \subseteq N$ with
        $\nu(E) > 0$. By the previous lemma, this yields a positive set $\tilde{E}
        \subseteq E$ for $\nu$ with $\nu(\tilde{E}) > 0$. Now, $\tilde{E} \cup P$ is
        a positive set for $\nu$, i.e.\ $\tilde{E} \cup P \in \mathscr{P}$. Also,
        $\tilde{E} \cap P = \emptyset$, so \[
            \nu(\tilde{E} \cup P) = \nu(\tilde{E}) + \nu(P) > m.
        \] This contradicts the maximality of $m$.

        Thus, we have obtained a Hahn decomposition $P, N$ of $X$. If $P', N'$ is
        another Hahn decomposition of $X$, then \[
            P\Delta P = (P\setminus P') \cup (P'\setminus P) = (N^c\setminus N'^c)
            \cup (N'^c\setminus N^c) = (N'\setminus N) \cup (N\setminus N') = N\Delta
            N'.
        \] Furthermore, $P\setminus P' \subseteq P$ is a positive set for $\nu$;
        $P\setminus P' \subseteq P'^c = N'$ is also a negative set for $\nu$. This
        means that $P\setminus P'$ must be a null set for $\nu$. The same reasoning
        shows that $P'\setminus P$ is a null set for $\nu$, hence so is $P\Delta P'$.
    \end{proof}

    \begin{corollary}
        Given a signed measure $\nu$ on $(X, \M)$, consider the Hahn decomposition
        $P, N$ of $X$ and define the measures $\nu^+$, $\nu^-$ as \[
            \nu^+(E) = \nu(E \cap P) \geq 0, \qquad
            \nu^-(E) = -\nu(E \cap N) \geq 0.
        \] Then, we have the decomposition of $\nu$ as the difference of the non-negative
        measures $\nu^+$, $\nu^-$ as \[
            \nu = \nu^+ - \nu^-.
        \]
    \end{corollary}

    \begin{definition}
        Let $\mu, \nu$ be two measures on $(X, \M)$. We say that $\mu$ and $\nu$ are
        mutually singular, denoted $\mu \perp \nu$, if there exists $E, F \in \M$
        such that $E \cup F = X$, $E \cap F = \emptyset$, $E$ is a null set for
        $\mu$, and $F$ is a null set for $\nu$.
    \end{definition}
    \begin{example}
        For any signed measure $\nu$, the corresponding Hahn decomposition $P, N$ of
        $X$ also gives the decomposition $\nu = \nu^+ - \nu^-$ where $\nu^+, \nu^-$
        are positive measures. Then, $\nu^+ \perp \nu^-$, with $N$ being a null set
        for $\nu^+$, $P$ for $\nu^-$.
    \end{example}

    \begin{theorem}[Jordan Decomposition]\label{theorem:jordan_decomposition}
        Let $\nu$ be a signed measure on $(X, \M)$. Then, there exist unique positive
        measures $\nu^+, \nu^-$ on $\M$ such that $\nu = \nu^+ - \nu^-$ and $\nu^+
        \perp \nu^-$.
    \end{theorem}
    \begin{proof}
        We have already shown that every signed measure $\nu$ admits such $\nu^+,
        \nu^-$ via the Hahn decomposition $P, N$ of $X$. Thus, it is enough to show
        that if $\nu = \mu^+ - \mu^-$ for positive measures $\mu^+, \mu^-$, and
        $\mu^+\perp\mu^-$ with $E, F$ being null sets for $\mu^+, \mu^-$, then $\mu^+
        = \nu^+$ and $\mu^- = \nu^-$.

        Let $A \in \M$. Then, \[
            0 \leq \mu^+(A) = \mu^+(A \cap F) + \mu^+(A \cap E) = \mu^+(A \cap F)
        \] since $A \cap E \subseteq E$ which is a null set for $\mu^+$.
        Additionally, $A \cap F \subseteq F$ which is a null set for $\mu^-$, so \[
            0 \leq \mu^+(A \cap F) = \mu^+(A \cap F) - \mu^+(A \cap F) = \nu(A \cap
            F).
        \] This shows that every subset of $F$ has positive $\nu$-measure, i.e.\ $F$
        is a positive set for $\nu$. Similarly, \[
            0 \geq -\mu^-(A) = -\mu^-(A \cap E) = -\nu(A \cap E),
        \] which shows that $E$ is a negative set for $\nu$. Thus, $F, E$ is a Hahn
        decomposition of $X$. Theorem~\ref{theorem:hahn_decomposition} immediately
        tells us that $P\Delta F = N\Delta E$ is a null set for $\nu$. Now, \[
            \mu^+(A) = \nu(A \cap F) = \nu(A \cap F \cap P) + \nu(A \cap F \cap N),
        \] but $A \cap F \cap N \subseteq (P\cap E) \cup (F \cap N) = P\Delta F$
        which is a null set for $\nu$. Thus, \[
            \mu^+(A) = \nu(A \cap F \cap P).
        \] Using the same arguments, \[
            \nu^+(A) = \nu(A \cap P) = \nu(A \cap P \cap F) + \nu(A \cap P \cap E),
        \] but $A \cap P \cap E \subseteq (P \cap E) \cup (F \cap N) = P\Delta F$,
        hence \[
            \nu^+(A) = \nu(A \cap F \cap P) = \mu^+(A).
        \] Thus, $\nu^+ = \mu^+$. An analogous argument shows that $\nu^- = \mu^-$.
    \end{proof}
    \begin{remark}
        The Hahn and Jordan decomposition theorems together give the existence of the
        decomposition of any signed measure $\nu = \nu^+ - \nu^-$ along with its
        uniqueness in a certain sense.
    \end{remark}


    \subsection{The total variation measure}

    \begin{definition}
        Let $\nu$ be a signed measure on $(X, \M)$, and let $\nu = \nu^+ - \nu^-$
        for positive measures $\nu^+, \nu^-$ with $\nu^+ \perp \nu^-$. Then, the
        total variation measure $|\nu|$ of $\nu$ is defined as \[
            |\nu| = \nu^+ + \nu^- \geq 0.
        \]
    \end{definition}

    \begin{lemma}
        Let $E \in \M$. The following are equivalent. \begin{enumerate}
            \item $E$ is a null set for $\nu$.
            \item $\nu^+(E) = \nu^-(E) = 0$.
            \item $|\nu|(E) = 0$.
        \end{enumerate}
    \end{lemma}
    \begin{proof}
        Let $P, N$ be the Hahn decomposition of $X$ associated with $\nu$.

        (\emph{1}\Rightarrow \emph{2}) Note that $E \cap N, E \cap P \subseteq E$
        are null sets for $\nu$. Thus, \begin{align*}
            \nu^+(E) &= \nu^+(E \cap P) + \nu^+(E \cap N) \\
            &= \nu^+(E \cap P) \\
            &= \nu^+(E \cap P) - \nu^-(E \cap P) \\
            &= \nu(E \cap P) \\
            &= 0.
        \end{align*}
         Similarly, \begin{align*}
            \nu^-(E) &= \nu^-(E \cap P) + \nu^-(E \cap N) \\
            &= \nu^-(E \cap N) \\
            &= -\nu^+(E \cap N) + \nu^-(E \cap N) \\
            &= -\nu(E \cap N) \\
            &= 0.
        \end{align*}

        (\emph{2}\Rightarrow \emph{3}) follows trivially.

        (\emph{3}\Rightarrow \emph{1}) By the positivity of $\nu^+, \nu^-$, we
        immediately have $\nu^+(E) = \nu^-(E) = 0$. Thus, for any measurable $F
        \subseteq E$, we have $\nu^+(F) = \nu^-(F) = 0$, hence $\nu(F) = 0$.
        Therefore, $E$ is a null set for $\nu$.
    \end{proof}

    \begin{lemma}
        Let $\nu, \mu$ be signed measures. The following are equivalent. \begin{enumerate}
            \item $\nu \perp \mu$.
            \item $\nu^+ \perp \mu$ and $\nu^- \perp \mu$
            \item $|\nu| \perp \mu$.
        \end{enumerate}
    \end{lemma}
    \begin{proof}
        Let $P, N$ be the Hahn decomposition associated with $\nu$.

        (\emph{1}\Rightarrow \emph{2}) Let $\nu \perp \mu$ via the decomposition $E,
        F$ of $X$. Then $F$ is a null set for $\nu$, $E$ for $\mu$. We claim that
        $\nu^+\perp \mu$ via the decomposition $N \cup F, P \cap E$. It is clear that
        $P \cap E \subseteq E$ is a null set for $\mu$; we now show that $N \cup F$
        is a null set for $\nu^+$, i.e.\ $\nu^+(N \cup F) = 0$. Indeed, \[
            0 \leq \nu^+(N \cup F) \leq \nu^+(N) + \nu^+(F) = \nu^+(F) = 0.
        \] We have used the fact that $N$ is a null set for $\nu^+$, and $F$ is a
        null set for $\nu$ together with the previous lemma.

        The proof that $\nu^-\perp \mu$ via the decomposition $N \cup F, P \cap E$
        is analogous.

        (\emph{2}\Rightarrow \emph{3}) Let $\nu^+\perp\mu$ via $E_1, F_1$, and
        $\nu^-\perp\mu$ via $E_2, F_2$. We claim that $|\nu| \perp \mu$ via the
        decomposition $E_1 \cup E_2, F_1 \cap F_2$. Is is clear that $E_1 \cup E_2$
        is a null set for $\mu$ since it is the union of the null sets $E_1, E_2$ for
        $\mu$. We must show that $F_1 \cap F_2$ is a null set for $|\nu|$, i.e.\
        $|\nu|(F_1 \cap F_2) = 0$; but this is clear from the fact that $F_1, F_2$
        are null sets for $\nu^+, \nu^-$, hence \[
            0 \leq |\nu|(F_1 \cap F_2) = \nu^+(F_1 \cap F_2) + \nu^-(F_1 \cap F_2)
            \leq \nu^+(F_1) + \nu^-(F_2) = 0.
        \]

        (\emph{3}\Rightarrow \emph{1}) Let $|\nu| \perp \mu$ via $E, F$. We claim
        that $\nu\perp\mu$ via the same decomposition $E, F$; indeed $|\nu|(F) = 0$
        immediately shows that $F$ is a null set for $\nu$ by the previous lemma.
    \end{proof}

    \begin{lemma}
        Let $\nu$ be a signed measure.
        \begin{enumerate}
            \item $\nu$ is a finite measure if and only if $|\nu|$ is a finite measure.
            \item $\nu$ is a $\sigma$-finite measure if and only if $|\nu|$ is a
            $\sigma$-finite measure.
        \end{enumerate}
    \end{lemma}
    \begin{proof}
        The finiteness of $\nu$ implies that $\nu^+$ and $\nu^-$ are finite hence
        their sum $|\nu|$ is finite, and vice versa. Next, if $X$ is a countable
        union of $\{X_i\}$ each of which has finite $\nu$-measure, each $\nu^+(X_i) <
        \infty$, $\nu^-(X_i) < \infty$ i.e.\ each $X_i$ has finite $|\nu|$-measure,
        and vice versa.
    \end{proof}

    \begin{definition}
        The space of functions $L^1(\nu)$, where $\nu$ is a signed measure, is
        defined by \[
            L^1(\nu) = L^1(\nu^+) \cap L^1(\nu^-).
        \]
    \end{definition}

    \begin{lemma}
        The spaces $L^1(\nu) = L^1(|\nu|)$.
    \end{lemma}
    \begin{proof}
        Note that \[
            \int_P |f|\:d\nu = \int_X |f|\:d\nu^+ = \int_P |f|\:d|\nu|, \qquad
            \int_N |f|\:d\nu = -\int_X |f|\:d\nu^- = -\int_N |f|\:d|\nu|. \qedhere
        \]
    \end{proof}

    \begin{definition}
        Let $\nu$ be a signed measure and $\mu$ be a positive measure. We say that
        $\nu$ is absolutely continuous with respect to $\mu$, denoted $\nu\ll\mu$, if
        $\mu(E) = 0 \implies \nu(E) = 0$ where $E \in \M$.
        \begin{remark}
            If $\nu \ll \mu$, then all null sets for $\mu$ are null sets for $\nu$.
        \end{remark}
        \begin{remark}
            If $\nu \ll \mu$ and $\nu \perp \mu$, then $\nu = 0$. Indeed, if $X = E
            \cup F$, $E \cap F = 0$ with $F$ null for $\nu$, $E$ \null for $\mu$,
            then for any $A \in \M$, we have $\nu(A) = \nu(A \cap E)$. But $A \cap E
            \subseteq E$ is a null set for $\mu$, hence $\mu(A \cap E) = 0 \implies
            \nu(A) = \nu(A \cap E) = 0$.
        \end{remark}
    \end{definition}

    \begin{lemma}
        Let $\nu$ be a signed measure and let $\mu$ be a positive measure. The
        following are equivalent. \begin{enumerate}
            \item $\nu \ll \mu$.
            \item $\nu^+ \ll \mu$ and $\nu^- \ll \mu$.
            \item $|\nu| \ll \mu$.
        \end{enumerate}
    \end{lemma}
    \begin{proof}
        (\emph{1}\Rightarrow \emph{2}) If $\mu(E) = 0$, then $\mu(F) = 0$ for every
        measurable $F \subseteq E$, hence $\nu(F) = 0$. In other words, $E$ is a null
        set for $\nu$, hence $\nu^+(E) = \nu^-(E) = 0$ by a previous lemma.


        (\emph{2}\Rightarrow \emph{3}) If $\mu(E) = 0$, then $\nu^+(E) = \nu^-(E) =
        0$, hence $|\nu|(E) = \nu^+(E) - \nu^-(E) = 0$.

        (\emph{3}\Rightarrow \emph{1}) If $\mu(E) = 0$, then $|\nu|(E) = 0$ hence $E$
        is a null set for $\nu$ by a previous lemma, giving $\nu(E) = 0$.
    \end{proof}


    \begin{lemma}
        Let $\nu, \mu$ be finite measures. Then, $\nu \ll \mu$ if and only if the
        following holds: for every $\epsilon > 0$, there exists $\delta > 0$ such
        that whenever $\mu(E) < \delta$, we have $|\nu(E)| < \epsilon$.
    \end{lemma}
    \begin{proof}
        (\Leftarrow) If $\mu(E) = 0$, then for any $\epsilon > 0$, we must have
        $\delta > 0$ such that $\mu(E) < \delta \implies |\nu(E)| < \epsilon$. This
        forces $\nu(E) = 0$, hence $\nu \ll \mu$.

        (\Rightarrow) Let $\nu \ll \mu$. Suppose that there exists $\epsilon > 0$
        such that for all $\delta > 0$, we have sets $E_\delta$ where $\mu(E_\delta)
        < \delta$ but $\nu(E_\delta) \geq \epsilon$. Set $\delta = 2^{-n}$, and
        obtain corresponding $E_n$ where $\mu(E_n) < 2^{-n}$, $\nu(E_n) \geq
        \epsilon$. Now, set \[
            F_n = \bigcup_{i = n}^\infty E_n, \qquad F = \bigcap_{i = 1}^\infty F_i.
        \] Then, each $F_n \supseteq F_{n + 1}$, and \[
            \mu(F_n) \leq \sum_{i = n}^\infty \frac{1}{2^i} = \frac{2}{2^n} \to 0.
        \] Thus, continuity from above with the finiteness of $\mu$ gives \[
            \mu(F) = \lim_{n \to \infty} \mu(F_n) = 0.
        \] But each $\nu(F_n) \geq \nu(E_n) \geq \epsilon$, hence $\nu(F) \geq
        \epsilon$ by the finiteness of $\nu$, a contradiction.
    \end{proof}

    \begin{corollary}
        Let $f\in L^1(\mu)$. Then for all $\epsilon > 0$, there exists $\delta > 0$
        such that whenever $\mu(E) < \delta$, we have \[
            |\int_E f \:d\mu| < \epsilon.
        \]
    \end{corollary}



    \subsection{The Radon-Nikodym theorem}

    \begin{theorem}[Radon-Nikodym]
        Let $\nu$ be a $\sigma$-finite measure, and let $\mu$ be a $\sigma$-finite
        positive measure. Then, we have the following. \begin{enumerate}
            \item There exists a unique decomposition $\nu = \nu_1 + \nu_2$ such that
            $\nu_1 \ll \mu$ and $\nu_2 \perp \mu$. Both $\nu_1$ and $\nu_2$ are
            $\sigma$-finite.

            \item There exists a measurable, $\mu$-integrable (in the extended sense)
            function $f$ such that \[
                \nu_1(E) = \int_E f \:d\mu.
            \] Furthermore, if two such functions satisfy the above, they must be
            equal $\mu$-almost everywhere.
        \end{enumerate}
    \end{theorem}
    \begin{example}
        Let $m$ be the Lebesgue measure and let $c$ be the counting measure on $\R$.
        Clearly, $m \ll c$. If there existed a function $f$ such that \[
            m(E) = \int_E f \:dc,
        \] then we would have \[
            0 = m(\{x\}) = \int_{\{x\}} f\:dc = f(x)
        \] for each $x \in \R$, forcing $f = 0$ hence $m = 0$ a contradiction.
    \end{example}

    \begin{definition}
        Let $\nu$ be a $\sigma$-finite measure, and let $\mu$ be a positive
        $\sigma$-finite measure, such that $\nu \ll \mu$. Using the Radon-Nikodym
        theorem, we can pick a measurable function $f$ such that \[
            \nu(E) = \int_E f \:d\mu
        \] for all $E \in \M$. Then, the function \[
            f \equiv \frac{d\nu}{d\mu}
        \] is called the Radon-Nikodym derivative of $\nu$ with respect to $\mu$.

        \begin{remark}
            If $d\nu = f\:d\mu$, then $d|\nu| = |f|\:d\mu$.
        \end{remark}
    \end{definition}

    \begin{lemma}
        Let $\nu \ll \mu$ and $\mu \ll \lambda$. Then, \[
            \frac{d\nu}{d\lambda} = \frac{d\nu}{d\mu} \frac{d\mu}{d\lambda}
        \] $\lambda$-almost everywhere.
    \end{lemma}



    \section{Complex measures}

    \subsection{Basic definitions}

    \begin{definition}
        Let $\M$ be a $\sigma$-algebra over $X$. A function $\nu\colon \M \to
        \C$ is called a complex measure if \begin{enumerate}
            \item $\nu(\emptyset) = 0$.
            \item For all countable collections of disjoint measurable sets
            $\{E_i\}$, we have \[
                \nu\left(\bigcup_{i = 1}^\infty E_i\right) = \sum_{i = 1}^\infty
                \nu(E_i).
            \]
        \end{enumerate}
        \begin{remark}
            A complex measure only assumes finite values.
        \end{remark}
        \begin{remark}
            A complex measure $\nu$ can be decomposed as \[
                \nu = \nu_r + i \nu_i
            \] where $\nu_r$, $\nu_i$ are signed measures.
        \end{remark}
    \end{definition}

    \begin{definition}
        A function $f \in L^1(\nu)$ if and only if $f \in L^1(\nu_r) \cap
        L^1(\nu_i)$.
    \end{definition}

    \begin{lemma}
        Let $\nu, \mu$ be complex measures. The following are equivalent.
        \begin{enumerate}
            \item $\nu \perp \mu$.
            \item $\nu_r \perp \mu_r$ and $\nu_i \perp \mu_i$.
        \end{enumerate}
    \end{lemma}

    \begin{lemma}
        Let $\nu$ be a complex measure, and let $\mu$ be a positive measure. The
        following are equivalent.
        \begin{enumerate}
            \item $\nu \ll \mu$.
            \item $\nu_r \ll \mu_r$ and $\nu_i \ll \mu_i$.
        \end{enumerate}
    \end{lemma}

    \subsection{The Radon-Nikodym theorem and the total variation measure}

    \begin{theorem}[Radon-Nikodym]
        Let $\nu$ be a complex measure, and let $\mu$ be a $\sigma$-finite positive
        measure. Then, there exists a unique decomposition \[
            \nu = \nu_1 + \nu_2
        \] such that $\nu_1 \ll \mu$, $\nu_2 \perp \mu$. There also exists a function
        $f \in L^1(\mu)$ such that \[
            \nu_1(E) = \int_E f \:d\mu.
        \] Furthermore, $f$ us unique $\mu$-almost everywhere.
    \end{theorem}

    \begin{lemma}
        Let $\nu$ be a complex measure and let $\mu_1, \mu_2$ be $\sigma$-finite
        positive measures, such that for $f_1 \in L^1(\mu_1)$, $f_2 \in L^1(\mu_2)$, \[
            d\nu = f_1\:d\mu_1 = f_2\:d\mu_2.
        \] Then, \[
            |f_1|\:d\mu_1 = |f_2|\:d\mu_2.
        \]
    \end{lemma}

    \begin{definition}
        Let $\nu$ be a complex measure, and let $\mu$ be a positive measure such that
        $d\nu = f\:d\mu$. Then, $d|\nu| = |f|\:d\mu$ defines a measure $|\nu|$ called
        the total variation measure.

        \begin{remark}
            Such a measure exists since $\nu \ll \nu_r^+ + \nu_r^- + \nu_i^+ +
            \nu_i^-$ supplies us with a Radon-Nikodym derivative $f$.
        \end{remark}
        \begin{remark}
            This definition is independent of our choice of $\mu$, hence $f$ by the
            previous lemma.
        \end{remark}
    \end{definition}

    \begin{lemma}
        \[
            |\nu(E)| \leq |\nu|(E).
        \]
    \end{lemma}

    \begin{lemma}
        \[
            |\nu| = \inf\{\lambda : |\nu(E)| \leq \lambda(E) \text{ for all } E \in
            \M\}.
        \]
    \end{lemma}

    \begin{lemma}
        \[
            \nu \ll |\nu|, \qquad
            \left|\frac{d\nu}{d|\nu|}\right| = 1\quad|\nu|\text{-almost everywhere}.
        \]
    \end{lemma}

    \begin{lemma}
        \[
            L^1(\nu) = L^1(|\nu|).
        \]
    \end{lemma}

    \begin{lemma}
        \[
            |\int f\:d\nu| \leq \int |f|\:d|\nu|.
        \]
    \end{lemma}

    \begin{lemma}
        \[
            |\nu_1 + \nu_2| \leq |\nu_1| + |\nu_2|.
        \]
    \end{lemma}


    \begin{theorem}
        Let $\mathscr{M}(X)$ be the collection of all complex measures on $(X, \M)$.
        Then, given $\nu_1, \nu_2 \in \mathscr{M}(X)$, we have $\nu_1 + \nu_2 \in
        \mathscr{M}(X)$ and $\alpha\nu_1 \in \mathscr{M}(X)$ for all $\alpha \in \C$.
        Thus, $\mathscr{X}(X)$ is a vector space. Furthermore, \[
            \Vert \nu\Vert = |\nu|(X)
        \] defines a norm on $\mathscr{M}(X)$. With this, $\mathscr{M}(X)$ is a
        Banach space.
    \end{theorem}



    \section{Differentiation on Euclidean Spaces}


    \subsection{Locally integrable functions}

    \begin{definition}
        Let $f\colon \R^n \to \C$ be measurable. We say that $f$ is locally
        integrable if for every bounded measurable $E \subseteq \R^n$, \[
            \int_E |f|\:dm
        \] is finite. The class of all such functions is denoted
        $\Lloc(\R^n)$.
    \end{definition}

    \begin{definition}
        For $f \in \Lloc(\R^n)$, define the average value of $f$ on $B_r(x)$ as \[
            A_r(f)(x) = \frac{1}{m(B_r(x))} \int_{B_r(x)} f\:dm.
        \]
    \end{definition}

    \begin{lemma}
        $A_r(f)(x)$ is jointly continuous in $r$ and $x$.
    \end{lemma}

    \begin{definition}
        For $f \in \Lloc(\R^n)$, define its Hardy-Littlewood maximal function as \[
            Hf(x) = \sup_{r > 0} A_r(|f|)(x) = \sup_{r > 0} \frac{1}{m(B_r(x))}
            \int_{B_r(x)} |f|\:dm.
        \]
    \end{definition}


    \subsection{The Maximal Function Theorem}

    \begin{lemma}
        Let $\mathscr{C}$ be a collection of open balls in $\R^n$, and let \[
            U = \bigcup_{B \in \mathscr{C}} B.
        \] If $c < m(U)$, then there exist disjoint $B_1, \dots, B_k \in \mathscr{C}$
        such that \[
            c < 3^n \sum_{i = 1}^k m(B_i).
        \]
    \end{lemma}

    \begin{theorem}[Maximal Function]
        There exists a constant $C > 0$ such that for all $f \in L^1$ and all $\alpha
        > 0$, \[
            m(\{x : Hf(x) > \alpha\}) \leq \frac{C}{\alpha} \int |f| \:dm.
        \]
    \end{theorem}

    \begin{theorem}
        If $f \in \Lloc(\R^n)$, then \[
            \lim_{r \to 0} A_r(f)(x) = f(x).
        \] almost everywhere on $\R^n$.
    \end{theorem}


    \subsection{The Lebesgue Differentiation Theorem}

    \begin{definition}
        For $f \in \Lloc(\R^n)$, define the Lebesgue set $L_f$ of $f$ as the
        collection of all $x \in \R^n$ such that \[
            \lim_{r \to 0} \frac{1}{m(B_r(x))} \int_{B_r(x)} |f(x) - f(x')| \:dx' = 0.
        \]
    \end{definition}

    \begin{lemma}
        For $f \in \Lloc(\R^n)$, we have $m(L_f^c) = 0$.
    \end{lemma}

    \begin{definition}
        We say that the family of measurable sets $\{E_r\}$ shrink nicely to $x
        \in \R^n$ if each $E_r \subseteq B_r(x)$, and there exists $\alpha > 0$ such
        that each $m(E_r) > \alpha m(B_r(x))$.

        \begin{remark}
            It is possible that none of the sets $E_r$ contains $x$!
        \end{remark}
    \end{definition}

    \begin{theorem}[Lebesgue Differentiation]
        Let $f \in \Lloc(\R^n)$. Then for all $x \in L_f$, we have \[
            \lim_{r \to 0} \frac{1}{m(E_r)} \int_{E_r} |f(x) - f(x')|\:dx', \qquad
            \lim_{r \to 0} \frac{1}{m(E_r)} \int_{E_r} f \:dm = f(x)
        \] for all families $\{E_r\}$ which shrink nicely to $x$.
    \end{theorem}


    \begin{definition}
        A positive Borel measure $\nu$ on $\R^n$ is called regular if the following
        hold. \begin{enumerate}
            \item $\nu(K)$ is finite for every compact $K \subseteq \R^n$.
            \item $\nu(E) = \inf\{\nu(U): E \subseteq U, U \subseteq \R^n\text{ is
            open}\}$ for every Borel measurable $E$.
        \end{enumerate}

        \begin{remark}
            The second condition follows from the first.
        \end{remark}
    \end{definition}


    \begin{theorem}
        Let $\nu$ be a regular signed or complex Borel measure on $\R^n$, whose
        Radon-Nikodym representation is given by \[
            d\nu = f\:dm + d\lambda.
        \] Then, for every family $\{E_r\}$ that shrinks nicely to $x \in \R^n$, we
        have \[
            \lim_{r \to 0} \frac{\nu(E_r)}{m(E_r)} - f(x)
        \] almost everywhere with respect to $m$.
    \end{theorem}



    \section{Total and Bounded Variation}

    \subsection{Total variation}

    \begin{theorem}
        Let $F\colon \R \to \R$ be increasing, and let \[
            G(x) = F(x^+) = \lim_{t \to x^+} F(x).
        \] Then, the following hold. \begin{enumerate}
            \item $G$ is increasing and right continuous.
            \item The set of discontinuities of $F$ is countable.
            \item Both $F, G$ are differentiable almost everywhere, with $F' = G'$
            almost everywhere.
        \end{enumerate}
    \end{theorem}

    \begin{definition}
        Let $F\colon \R \to \R$. The total variation function $T_F$ of $F$ is defined
        as \[
            T_F(x) = \sup\left\{\sum_{i = 1}^n |F(x_i) - F(x_{i - 1})| : -\infty <
            x_0 < \dots < x_n = x\right\}.
        \]
    \end{definition}

    \begin{lemma}
        \[
            T_F(b) - T_F(a) = \sup\left\{\sum_{i = 1}^n |F(x_i) - F(x_{i - 1})| : a =
            x_0 < \dots < x_n = b\right\}.
        \]
    \end{lemma}

    \begin{lemma}
        The total variation function $T_F$ of $F$ is increasing, taking values in
        $[0, \infty]$.
    \end{lemma}


    \subsection{Bounded variation}

    \begin{definition}
        We say that $F$ is of bounded variation if \[
            T(\infty) = \lim_{x \to \infty} T(x)
        \] is finite. The class of such functions is denoted $BV(\R)$.

        Furthermore, the class of functions $F$ such that $T_F(b) - T_F(a)$ is finite
        is denoted $BV[a, b]$.
    \end{definition}

    \begin{example}
        If $F$ is bounded and increasing, then $F \in BV(\R)$, with \[
            T_F(x) = F(x) - F(-\infty).
        \]
    \end{example}
    \begin{example}
        If $F, G \in BV(\R)$, then $\alpha F + \beta G \in BV(\R)$ for all $\alpha,
        \beta \in \C$.
    \end{example}
    \begin{example}
        If $F$ is differentiable and $F'$ is bounded, then $F \in BV[a, b]$.
    \end{example}

    \begin{lemma}
        If $F \in BV(\R)$, then $T_F \pm F$ are bounded and increasing functions.
    \end{lemma}

    \begin{theorem}
        The following results hold.
        \begin{enumerate}
            \item $F \in BV(\C)$ if and only if $\Re(F), \Im(F) \in BV(\R)$.
            \item $F \in BV(\R)$ if and only if $F$ is the difference of two bounded
            and increasing functions. Note that \[
                F = \frac{1}{2}(T_F + F) - \frac{1}{2}(T_F - F).
            \]
            \item If $F \in BV(\R)$, all of the following limits exist. \[
                F(x^+) = \lim_{t \to x^+} F(x), \qquad
                F(x^-) = \lim_{t \to x^-} F(x), \qquad
                F(\pm\infty) = \lim_{x \to \pm\infty} F(x).
            \]
            \item If $F \in BV(\R)$, then the set of points on which $F$ is
            discontinuous is countable.
            \item If $F \in BV(\R)$, and $G(x) = F(x^+)$, then $F' = G'$ almost
            everywhere.
        \end{enumerate}
    \end{theorem}

    \begin{definition}
        For $F \in BV(\R)$, the following is called the Jordan representation of $F$.
        \[
            F = \frac{1}{2}(T_F + F) - \frac{1}{2}(T_F - F).
        \] The two terms are called the positive and negative variations of $F$
        respectively.
    \end{definition}

    \begin{lemma}
        Denote the positive and negative parts of $x$ as $x^\pm = \frac{1}{2}(|x| \pm
        x)$. Then, \[
            \frac{1}{2}(T_F \pm F)(x) = \sup\left\{\sum_{i = 1}^n \left[F(x_i) -
            F(x_{i - 1})\right]^\pm : -\infty < x_0 < \dots < x_n = x\right\}.
        \]
    \end{lemma}


    \subsection{Normalized Bounded Variation}

    \begin{definition}
        Denote \[
            T_F(-\infty) = \lim_{x \to -\infty} T_F(x).
        \] We say that $F \in BV(\R)$ is of normalized bounded variation if $F$ is
        right continuous and $T_F(-\infty)$ is finite. The class of such functions is
        denoted $NBV(\R)$.

        \begin{remark}
            If $F \in BV(\R)$, then $G(x) = F(x^+) - F(-\infty)$ is in $NBV(\R)$.
        \end{remark}
    \end{definition}

    \begin{lemma}
        If $F \in NBV(\C)$, then $T_F(-\infty) = 0$. Furthermore, if $F$ is right
        continuous, so is $T_F$.
    \end{lemma}

    \begin{theorem}
        Let $\mu$ be a complex Borel measure on $\R^n$. Let \[
            F(x) = \mu(-\infty, x].
        \] Then, $F \in NBV(\C)$.

        Conversely, if $F \in NBV(\C)$, there exists a complex Borel measure $\mu_F$
        such that \[
            \mu_F(-\infty, x] = F(x).
        \] Furthermore, \[
            |\mu_F| = \mu_{T_F}.
        \]
    \end{theorem}


\end{document}
