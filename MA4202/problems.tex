\documentclass[10pt]{article}

\usepackage[T1]{fontenc}
\usepackage{geometry}
\usepackage{amsmath, amssymb, amsthm}
\usepackage{bm}

\geometry{a4paper, margin=1in}

\renewcommand{\labelenumi}{(\roman{enumi})}

\newcounter{prob}
\newcommand{\problem}{\stepcounter{prob}\paragraph{Exercise \arabic{prob}}}
\newcommand{\solution}{\paragraph{Solution}}

\newcommand{\C}{\mathbb{C}}
\newcommand{\R}{\mathbb{R}}
\newcommand{\Q}{\mathbb{Q}}
\newcommand{\Z}{\mathbb{Z}}
\newcommand{\N}{\mathbb{N}}

\title{MA4202: Ordinary Differential Equations}
\author{Satvik Saha}
\date{}

\begin{document}
    \noindent\textbf{IISER Kolkata} \hfill \textbf{Exercises}
    \vspace{3pt}
    \hrule
    \vspace{3pt}
    \begin{center}
    \LARGE{\textbf{MA4202: Ordinary Differential Equations}}
    \end{center}
    \vspace{3pt}
    \hrule
    \vspace{3pt}
    Satvik Saha, \texttt{19MS154} \hfill \today
    \vspace{20pt}



    \problem Let $x: [-1, 1] \to \R$ be a continuous function satisfying \[
        x(t) = x(0) + \int_0^t x(s)\:ds.
    \] Show that \[
        x^2(t) = x^2(0) + 2\int_0^t x^2(s)\:ds.
    \]

    \solution Observe that $x'(t) = x(t)$, hence integrating by parts yields \[
        \int_0^t x^2(s)\:ds = \int_0^t x(s) x'(s)\:ds = x^2(s)\Big|_0^t - \int_0^t
        x(s)x'(s)\:ds,
    \] whence \[
        2\int_0^t x^2(s)\:ds = x^2(t) - x^2(0).
    \]


    \problem Consider the IVP \[
        \dot{x} = x^2 + t^2, \qquad
        x(0) = 1.
    \] Prove that for some $b > 0$, there is a solution defined on $[0, b]$. Also
    find $c > 0$ such that there is no solution on $[0, c]$.

    \solution Fix $d = 1$, $r = 1$. The map $(t, x) \mapsto x^2 + t^2$ is bounded by
    $M = 5$ on $[t_0 - d, t_0 + d] \times \overline{B_r(x_0)} = [-1, 1]\times[0, 2]$.
    Thus, Peano's Theorem guarantees a solution on the interval $[0, b]$ with $b =
    \min(c, r/M) = 1/5$.

    Note that for any solution $x$, we must have \[
        x'(t) \geq x^2(t), \qquad -\frac{d}{dt}\left(\frac{1}{x}\right) \geq 1,
    \] whence \[
        1 - \frac{1}{x(t)} \geq t, \qquad x(t) \geq \frac{1}{1 - t}.
    \] Thus, there is no solution on $[0, 1]$.


    \problem Determine the maximal interval of existence for the following IVP. \[
        \dot{x} = y\cos^2{x} + \sin{t}\cos{y} + 1, \qquad
        \dot{y} = \sin{y} + x, \qquad
        x(0) = 0, \quad
        y(0) = 1.
    \]

    \solution Framing the system of equations as $\dot{\bm{x}} = f(t, \bm{x})$ note
    that \[
        |f(t, \bm{x})| \leq |y\cos^2{y} + \sin{t}\cos{y} + 1| + |\sin{y} + x|
        \leq |y| + |x| + 3 \leq 2|\bm{x}| + 3.
    \] Furthermore, $f$ is $C^1$; thus the maximal interval of existence for any
    solution of the given IVP is $\R$.


    \problem Maximize the interval length in the Picard-Lindel\"of Theorem for the
    solution of the IVP \[
        \dot{x} = 5 + x^2, \qquad
        x(0) = 1.
    \]

    \solution For $r > 0$, the maximum value of the map $(t, x) \mapsto 5 + x^2$ on
    $\R\times \overline{B_r(x_0)} = \R\times [1 - r, 1 + r]$ is $M = 5 + (1 + r)^2$.
    Also, \[
        |f(t, x) - f(t, y)| = |x^2 - y^2| = |x + y|\dot |x - y| \leq (2 + 2r) |x -
        y|,
    \] hence $L = (2 + 2r)$ is the Lipschitz constant for $f$. Thus, we must choose
    $h < \min(r/M, 1/L) = \min(r/(6 + 2r + r^2), 1/(2 + 2r))$. This is maximised at
    $r = \sqrt{6}$.


    \problem Show that the sequence of Picard iterates of the IVP \[
        \dot{x} = x^{1 / 3}, \qquad
        x(0) = 0
    \] converges, but the IVP does not have a unique solution.

    \solution It is clear that all Picard iterates of this IVP are identically zero,
    but we have a family of solutions $\{x_\alpha\}_{\alpha \geq 0}$ described by
    \[
            x_\alpha(t) = \begin{cases}
                0, &\text{ if } x \in [0, \alpha], \\
                k(t - \alpha)^{3 / 2}, &\text{ if } x \in [\alpha, \infty).
            \end{cases}
    \]


    \problem Let $f\colon \R \to \R$ be a continuous function, and let $x\colon I \to
    \R$ be a solution of $x' = f(x)$ for an interval $I$. Show that $x$ is a monotone
    function.

    \solution Suppose to the contrary that $x'(a) > 0$ and $x'(b) < 0$ for some $a, b
    \in I$. Without loss of generality, let $a < b$, $x(a) \leq x(b)$. Pick $\tau \in
    (a, b)$ such that $x(\tau)$ is maximum, and let $\sigma$ be the largest number in
    $[a, \tau]$ such that $x(\sigma) = x(b)$. Then, we must have all $x(t) \geq
    x(\sigma)$ for $t \in [\sigma, \tau]$, hence $x'(\sigma) \geq 0$. But, \[
        0 \leq x'(\sigma) = f(x(\sigma)) = f(x(b)) = x'(b) < 0,
    \] a contradiction.


    \problem Let $T$ be a linear operator on $\R^n$ that leaves a subspace $E
    \subseteq \R^n$ invariant. Show that $e^T$ also leaves $E$ invariant.

    \solution Note that for any $x \in \R^n$, we have \[
        e^Tx = \lim_{n \to \infty} \sum_{k = 1}^n \frac{T^kx}{k!}.
    \] Each $T^nx \in E$, so each term in the limit is in $E$ as well. Since
    linear subspaces of $\R^n$ are closed, the limit $e^Tx \in E$.


    \problem Can the Arzela-Ascoli Theorem be applied to the sequence of functions
    $t \mapsto \sin(nt)$ on $[0, \pi]$?

    \solution No; the given family is not equicontinuous. Suppose to the contrary
    that there exists $\delta > 0$ such that $|\sin(nt) - \sin(ns)| < 1/2$ for all $n
    \in \N$ whenever $|s - t| < \delta$. Then we can pick $N \in \N$ such that
    $\pi/2N < \delta$. Thus, $|\pi/2N - 0| < \delta$, but $|\sin(N\cdot\pi/2N) -
    \sin(0)| = 1 > 1/2$, a contradiction.

\end{document}
